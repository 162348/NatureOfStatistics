\documentclass[uplatex,dvipdfmx]{jsreport}
\title{Analysis Now}
\author{}
\pagestyle{headings} \setcounter{secnumdepth}{4}
\usepackage{mathtools}
%\mathtoolsset{showonlyrefs=true} %labelを附した数式にのみ附番される設定.
%\usepackage{amsmath} %mathtoolsの内部で呼ばれるので要らない.
\usepackage{amsfonts} %mathfrak, mathcal, mathbbなど.
\usepackage{amsthm} %定理環境.
\usepackage{amssymb} %AMSFontsを使うためのパッケージ.
\usepackage{ascmac} %screen, itembox, shadebox環境.全てLATEX2εの標準機能の範囲で作られたもの.
\usepackage{comment} %comment環境を用いて,複数行をcomment outできるようにするpackage
\usepackage{wrapfig} %図の周りに文字をwrapさせることができる.詳細な制御ができる.
\usepackage[usenames, dvipsnames]{xcolor} %xcolorはcolorの拡張.optionの意味はdvipsnamesはLoad a set of predefined colors. forestgreenなどの色が追加されている.usenamesはobsoleteとだけ書いてあった.
\setcounter{tocdepth}{2} %目次に表示される深さ.2はsubsectionまで
\usepackage{multicol} %\begin{multicols}{2}環境で途中からmulticolumnに出来る.

\usepackage{url}
\usepackage[dvipdfmx,colorlinks,linkcolor=blue,urlcolor=blue]{hyperref} %生成されるPDFファイルにおいて、\tableofcontentsによって書き出された目次をクリックすると該当する見出しへジャンプしたり、さらには、\label{ラベル名}を番号で参照する\ref{ラベル名}やthebibliography環境において\bibitem{ラベル名}を文献番号で参照する\cite{ラベル名}においても番号をクリックすると該当箇所にジャンプする.囲み枠はダサいので,colorlinksで囲み廃止し,リンク自体に色を付けることにした.
\usepackage{pxjahyper} %pxrubrica同様,八登崇之さん.hyperrefは日本語pLaTeXに最適化されていないから,hyperrefとセットで,(u)pLaTeX+hyperref+dvipdfmxの組み合わせで日本語を含む「しおり」をもつPDF文書を作成する場合に必要となる機能を提供する
\definecolor{花緑青}{cmyk}{0.52,0.03,0,0.27}
\definecolor{サーモンピンク}{cmyk}{0,0.65,0.65,0.05}
\definecolor{暗中模索}{rgb}{0.2,0.2,0.2}

\usepackage{tikz}
\usetikzlibrary{positioning,automata} %automaton描画のため
\usepackage{tikz-cd}
\usepackage[all]{xy}
\def\objectstyle{\displaystyle} %デフォルトではxymatrix中の数式が文中数式モードになるので,それを直す.\labelstyleも同様にxy packageの中で定義されており,文中数式モードになっている.

\usepackage[version=4]{mhchem} %化学式をTikZで簡単に書くためのパッケージ.
\usepackage{chemfig} %化学構造式をTikZで描くためのパッケージ.
\usepackage{siunitx} %IS単位を書くためのパッケージ

\usepackage{ulem} %取り消し線を引くためのパッケージ
\usepackage{pxrubrica} %日本語にルビをふる.八登崇之(やとうたかゆき)氏による.

\usepackage{graphicx} %rotatebox, scalebox, reflectbox, resizeboxなどのコマンドや,図表の読み込み\includegraphicsを司る.graphics というパッケージもありますが,graphicx はこれを高機能にしたものと考えて結構です(ただし graphicx は内部で graphics を読み込みます)

\usepackage[breakable]{tcolorbox} %加藤晃史さんがフル活用していたtcolorboxを,途中改ページ可能で.
\tcbuselibrary{theorems} %https://qiita.com/t_kemmochi/items/483b8fcdb5db8d1f5d5e
\usepackage{enumerate} %enumerate環境を凝らせる.
\usepackage[top=15truemm,bottom=15truemm,left=10truemm,right=10truemm]{geometry} %足助さんからもらったオプション

%%%%%%%%%%%%%%% 環境マクロ %%%%%%%%%%%%%%%

\usepackage{listings} %ソースコードを表示できる環境.多分もっといい方法ある.
\usepackage{jvlisting} %日本語のコメントアウトをする場合jlistingが必要
\lstset{ %ここからソースコードの表示に関する設定.lstlisting環境では,[caption=hoge,label=fuga]などのoptionを付けられる.
%[escapechar=!]とすると,LaTeXコマンドを使える.
  basicstyle={\ttfamily},
  identifierstyle={\small},
  commentstyle={\smallitshape},
  keywordstyle={\small\bfseries},
  ndkeywordstyle={\small},
  stringstyle={\small\ttfamily},
  frame={tb},
  breaklines=true,
  columns=[l]{fullflexible},
  numbers=left,
  xrightmargin=0zw,
  xleftmargin=3zw,
  numberstyle={\scriptsize},
  stepnumber=1,
  numbersep=1zw,
  lineskip=-0.5ex
}
%\makeatletter %caption番号を「[chapter番号].[section番号].[subsection番号]-[そのsubsection内においてn番目]」に変更
%    \AtBeginDocument{
%    \renewcommand*{\thelstlisting}{\arabic{chapter}.\arabic{section}.\arabic{lstlisting}}
%    \@addtoreset{lstlisting}{section}
%    }
%\makeatother
\renewcommand{\lstlistingname}{算譜} %caption名を"program"に変更

\newtcolorbox{tbox}[3][]{%
colframe=#2,colback=#2!10,coltitle=#2!20!black,title={#3},#1}

%%%%%%%%%%%%%%% フォント %%%%%%%%%%%%%%%

\usepackage{textcomp, mathcomp} %Text Companionとは,T1 encodingに入らなかった文字群.これを使うためのパッケージ.\textsectionでブルバキに!
\usepackage[T1]{fontenc} %8bitエンコーディングにする.comp系拡張数学文字の動作が安定する.

%%%%%%%%%%%%%%% 数学記号のマクロ %%%%%%%%%%%%%%%

\newcommand{\abs}[1]{\lvert#1\rvert} %mathtoolsはこうやって使うのか!
\newcommand{\Abs}[1]{\left|#1\right|}
\newcommand{\norm}[1]{\|#1\|}
\newcommand{\Norm}[1]{\left\|#1\right\|}
%\newcommand{\brace}[1]{\{#1\}}
\newcommand{\Brace}[1]{\left\{#1\right\}}
\newcommand{\paren}[1]{\left(#1\right)}
\newcommand{\bracket}[1]{\langle#1\rangle}
\newcommand{\brac}[1]{\langle#1\rangle}
\newcommand{\Bracket}[1]{\left\langle#1\right\rangle}
\newcommand{\Brac}[1]{\left\langle#1\right\rangle}
\newcommand{\Square}[1]{\left[#1\right]}
\renewcommand{\o}[1]{\overline{#1}}
\renewcommand{\u}[1]{\underline{#1}}
\renewcommand{\iff}{\;\mathrm{iff}\;} %nLabリスペクト
\newcommand{\pp}[2]{\frac{\partial #1}{\partial #2}}
\newcommand{\ppp}[3]{\frac{\partial #1}{\partial #2\partial #3}}
\newcommand{\dd}[2]{\frac{d #1}{d #2}}
\newcommand{\floor}[1]{\lfloor#1\rfloor}
\newcommand{\Floor}[1]{\left\lfloor#1\right\rfloor}
\newcommand{\ceil}[1]{\lceil#1\rceil}

\newcommand{\iso}{\xrightarrow{\,\smash{\raisebox{-0.45ex}{\ensuremath{\scriptstyle\sim}}}\,}}
\newcommand{\wt}[1]{\widetilde{#1}}
\newcommand{\wh}[1]{\widehat{#1}}

\newcommand{\Lrarrow}{\;\;\Leftrightarrow\;\;}

%ノルム位相についての閉包 https://newbedev.com/how-to-make-double-overline-with-less-vertical-displacement
\makeatletter
\newcommand{\dbloverline}[1]{\overline{\dbl@overline{#1}}}
\newcommand{\dbl@overline}[1]{\mathpalette\dbl@@overline{#1}}
\newcommand{\dbl@@overline}[2]{%
  \begingroup
  \sbox\z@{$\m@th#1\overline{#2}$}%
  \ht\z@=\dimexpr\ht\z@-2\dbl@adjust{#1}\relax
  \box\z@
  \ifx#1\scriptstyle\kern-\scriptspace\else
  \ifx#1\scriptscriptstyle\kern-\scriptspace\fi\fi
  \endgroup
}
\newcommand{\dbl@adjust}[1]{%
  \fontdimen8
  \ifx#1\displaystyle\textfont\else
  \ifx#1\textstyle\textfont\else
  \ifx#1\scriptstyle\scriptfont\else
  \scriptscriptfont\fi\fi\fi 3
}
\makeatother
\newcommand{\oo}[1]{\dbloverline{#1}}

\DeclareMathOperator{\grad}{\mathrm{grad}}
\DeclareMathOperator{\rot}{\mathrm{rot}}
\DeclareMathOperator{\divergence}{\mathrm{div}}
\newcommand{\False}{\mathrm{False}}
\newcommand{\True}{\mathrm{True}}
\DeclareMathOperator{\tr}{\mathrm{tr}}
\newcommand{\M}{\mathcal{M}}
\newcommand{\cF}{\mathcal{F}}
\newcommand{\cD}{\mathcal{D}}
\newcommand{\fX}{\mathfrak{X}}
\newcommand{\fY}{\mathfrak{Y}}
\newcommand{\fZ}{\mathfrak{Z}}
\renewcommand{\H}{\mathcal{H}}
\newcommand{\fH}{\mathfrak{H}}
\newcommand{\bH}{\mathbb{H}}
\newcommand{\id}{\mathrm{id}}
\newcommand{\A}{\mathcal{A}}
% \renewcommand\coprod{\rotatebox[origin=c]{180}{$\prod$}} すでにどこかにある.
\newcommand{\pr}{\mathrm{pr}}
\newcommand{\U}{\mathfrak{U}}
\newcommand{\Map}{\mathrm{Map}}
\newcommand{\dom}{\mathrm{Dom}\;}
\newcommand{\cod}{\mathrm{Cod}\;}
\newcommand{\supp}{\mathrm{supp}\;}
\newcommand{\otherwise}{\mathrm{otherwise}}
\newcommand{\st}{\;\mathrm{s.t.}\;}
\newcommand{\lmd}{\lambda}
\newcommand{\Lmd}{\Lambda}
%%% 線型代数学
\newcommand{\Ker}{\mathrm{Ker}\;}
\newcommand{\Coker}{\mathrm{Coker}\;}
\newcommand{\Coim}{\mathrm{Coim}\;}
\newcommand{\rank}{\mathrm{rank}}
\newcommand{\lcm}{\mathrm{lcm}}
\newcommand{\sgn}{\mathrm{sgn}}
\newcommand{\GL}{\mathrm{GL}}
\newcommand{\SL}{\mathrm{SL}}
\newcommand{\alt}{\mathrm{alt}}
%%% 複素解析学
\renewcommand{\Re}{\mathrm{Re}\;}
\renewcommand{\Im}{\mathrm{Im}\;}
\newcommand{\Gal}{\mathrm{Gal}}
\newcommand{\PGL}{\mathrm{PGL}}
\newcommand{\PSL}{\mathrm{PSL}}
\newcommand{\Log}{\mathrm{Log}\,}
\newcommand{\Res}{\mathrm{Res}\,}
\newcommand{\on}{\mathrm{on}\;}
\newcommand{\hatC}{\hat{\C}}
\newcommand{\hatR}{\hat{\R}}
\newcommand{\PV}{\mathrm{P.V.}}
\newcommand{\diam}{\mathrm{diam}}
\newcommand{\Area}{\mathrm{Area}}
\newcommand{\Lap}{\Laplace}
\newcommand{\f}{\mathbf{f}}
\newcommand{\cR}{\mathcal{R}}
\newcommand{\const}{\mathrm{const.}}
\newcommand{\Om}{\Omega}
\newcommand{\Cinf}{C^\infty}
\newcommand{\ep}{\epsilon}
\newcommand{\dist}{\mathrm{dist}}
\newcommand{\opart}{\o{\partial}}
%%% 解析力学
\newcommand{\x}{\mathbf{x}}
%%% 集合と位相
\renewcommand{\O}{\mathcal{O}}
\renewcommand{\S}{\mathcal{S}}
\renewcommand{\U}{\mathcal{U}}
\newcommand{\V}{\mathcal{V}}
\renewcommand{\P}{\mathcal{P}}
\newcommand{\R}{\mathbb{R}}
\newcommand{\N}{\mathbb{N}}
\newcommand{\C}{\mathbb{C}}
\newcommand{\Z}{\mathbb{Z}}
\newcommand{\Q}{\mathbb{Q}}
\newcommand{\TV}{\mathrm{TV}}
\newcommand{\ORD}{\mathrm{ORD}}
\newcommand{\Tr}{\mathrm{Tr}\;}
\newcommand{\Card}{\mathrm{Card}\;}
\newcommand{\Top}{\mathrm{Top}}
\newcommand{\Disc}{\mathrm{Disc}}
\newcommand{\Codisc}{\mathrm{Codisc}}
\newcommand{\CoDisc}{\mathrm{CoDisc}}
\newcommand{\Ult}{\mathrm{Ult}}
\newcommand{\ord}{\mathrm{ord}}
\newcommand{\maj}{\mathrm{maj}}
%%% 形式言語理論
\newcommand{\REGEX}{\mathrm{REGEX}}
\newcommand{\RE}{\mathbf{RE}}

%%% Fourier解析
\newcommand*{\Laplace}{\mathop{}\!\mathbin\bigtriangleup}
\newcommand*{\DAlambert}{\mathop{}\!\mathbin\Box}
%%% Graph Theory
\newcommand{\SimpGph}{\mathrm{SimpGph}}
\newcommand{\Gph}{\mathrm{Gph}}
\newcommand{\mult}{\mathrm{mult}}
\newcommand{\inv}{\mathrm{inv}}
%%% 多様体
\newcommand{\Der}{\mathrm{Der}}
\newcommand{\osub}{\overset{\mathrm{open}}{\subset}}
\newcommand{\osup}{\overset{\mathrm{open}}{\supset}}
\newcommand{\al}{\alpha}
\newcommand{\K}{\mathbb{K}}
\newcommand{\Sp}{\mathrm{Sp}}
\newcommand{\g}{\mathfrak{g}}
\newcommand{\h}{\mathfrak{h}}
\newcommand{\Exp}{\mathrm{Exp}\;}
\newcommand{\Imm}{\mathrm{Imm}}
\newcommand{\Imb}{\mathrm{Imb}}
\newcommand{\codim}{\mathrm{codim}\;}
\newcommand{\Gr}{\mathrm{Gr}}
%%% 代数
\newcommand{\Ad}{\mathrm{Ad}}
\newcommand{\finsupp}{\mathrm{fin\;supp}}
\newcommand{\SO}{\mathrm{SO}}
\newcommand{\SU}{\mathrm{SU}}
\newcommand{\acts}{\curvearrowright}
\newcommand{\mono}{\hookrightarrow}
\newcommand{\epi}{\twoheadrightarrow}
\newcommand{\Stab}{\mathrm{Stab}}
\newcommand{\nor}{\mathrm{nor}}
\newcommand{\T}{\mathbb{T}}
\newcommand{\Aff}{\mathrm{Aff}}
\newcommand{\rsub}{\triangleleft}
\newcommand{\rsup}{\triangleright}
\newcommand{\subgrp}{\overset{\mathrm{subgrp}}{\subset}}
\newcommand{\Ext}{\mathrm{Ext}}
\newcommand{\sbs}{\subset}\newcommand{\sps}{\supset}
\newcommand{\In}{\mathrm{In}}
\newcommand{\Tor}{\mathrm{Tor}}
\newcommand{\p}{\mathfrak{p}}
\newcommand{\q}{\mathfrak{q}}
\newcommand{\m}{\mathfrak{m}}
\newcommand{\cS}{\mathcal{S}}
\newcommand{\Frac}{\mathrm{Frac}\,}
\newcommand{\Spec}{\mathrm{Spec}\,}
\newcommand{\bA}{\mathbb{A}}
\newcommand{\Sym}{\mathrm{Sym}}
\newcommand{\Ann}{\mathrm{Ann}}
%%% 代数的位相幾何学
\newcommand{\Ho}{\mathrm{Ho}}
\newcommand{\CW}{\mathrm{CW}}
\newcommand{\lc}{\mathrm{lc}}
\newcommand{\cg}{\mathrm{cg}}
\newcommand{\Fib}{\mathrm{Fib}}
\newcommand{\Cyl}{\mathrm{Cyl}}
\newcommand{\Ch}{\mathrm{Ch}}
%%% 数値解析
\newcommand{\round}{\mathrm{round}}
\newcommand{\cond}{\mathrm{cond}}
\newcommand{\diag}{\mathrm{diag}}
%%% 確率論
\newcommand{\calF}{\mathcal{F}}
\newcommand{\X}{\mathcal{X}}
\newcommand{\Meas}{\mathrm{Meas}}
\newcommand{\as}{\;\mathrm{a.s.}} %almost surely
\newcommand{\io}{\;\mathrm{i.o.}} %infinitely often
\newcommand{\fe}{\;\mathrm{f.e.}} %with a finite number of exceptions
\newcommand{\F}{\mathcal{F}}
\newcommand{\bF}{\mathbb{F}}
\newcommand{\W}{\mathcal{W}}
\newcommand{\Pois}{\mathrm{Pois}}
\newcommand{\iid}{\mathrm{i.i.d.}}
\newcommand{\wconv}{\rightsquigarrow}
\newcommand{\Var}{\mathrm{Var}}
\newcommand{\xrightarrown}{\xrightarrow{n\to\infty}}
\newcommand{\au}{\mathrm{au}}
\newcommand{\cT}{\mathcal{T}}
%%% 情報理論
\newcommand{\bit}{\mathrm{bit}}
%%% 積分論
\newcommand{\calA}{\mathcal{A}}
\newcommand{\calB}{\mathcal{B}}
\newcommand{\D}{\mathcal{D}}
\newcommand{\Y}{\mathcal{Y}}
\newcommand{\calC}{\mathcal{C}}
\renewcommand{\ae}{\mathrm{a.e.}\;}
\newcommand{\cZ}{\mathcal{Z}}
\newcommand{\fF}{\mathfrak{F}}
\newcommand{\fI}{\mathfrak{I}}
\newcommand{\E}{\mathcal{E}}
\newcommand{\sMap}{\sigma\textrm{-}\mathrm{Map}}
\DeclareMathOperator*{\argmax}{arg\,max}
\DeclareMathOperator*{\argmin}{arg\,min}
\newcommand{\cC}{\mathcal{C}}
\newcommand{\comp}{\complement}
\newcommand{\J}{\mathcal{J}}
\newcommand{\sumN}[1]{\sum_{#1\in\N}}
\newcommand{\cupN}[1]{\cup_{#1\in\N}}
\newcommand{\capN}[1]{\cap_{#1\in\N}}
\newcommand{\Sum}[1]{\sum_{#1=1}^\infty}
\newcommand{\sumn}{\sum_{n=1}^\infty}
\newcommand{\summ}{\sum_{m=1}^\infty}
\newcommand{\sumk}{\sum_{k=1}^\infty}
\newcommand{\sumi}{\sum_{i=1}^\infty}
\newcommand{\sumj}{\sum_{j=1}^\infty}
\newcommand{\cupn}{\cup_{n=1}^\infty}
\newcommand{\capn}{\cap_{n=1}^\infty}
\newcommand{\cupk}{\cup_{k=1}^\infty}
\newcommand{\cupi}{\cup_{i=1}^\infty}
\newcommand{\cupj}{\cup_{j=1}^\infty}
\newcommand{\limn}{\lim_{n\to\infty}}
\renewcommand{\l}{\mathcal{l}}
\renewcommand{\L}{\mathcal{L}}
\newcommand{\Cl}{\mathrm{Cl}}
\newcommand{\cN}{\mathcal{N}}
\newcommand{\Ae}{\textrm{-a.e.}\;}
\newcommand{\csub}{\overset{\textrm{closed}}{\subset}}
\newcommand{\csup}{\overset{\textrm{closed}}{\supset}}
\newcommand{\wB}{\wt{B}}
\newcommand{\cG}{\mathcal{G}}
\newcommand{\Lip}{\mathrm{Lip}}
\newcommand{\Dom}{\mathrm{Dom}}
%%% 数理ファイナンス
\newcommand{\pre}{\mathrm{pre}}
\newcommand{\om}{\omega}

%%% 統計的因果推論
\newcommand{\Do}{\mathrm{Do}}
%%% 数理統計
\newcommand{\bP}{\mathbb{P}}
\newcommand{\compsub}{\overset{\textrm{cpt}}{\subset}}
\newcommand{\lip}{\textrm{lip}}
\newcommand{\BL}{\mathrm{BL}}
\newcommand{\G}{\mathbb{G}}
\newcommand{\NB}{\mathrm{NB}}
\newcommand{\oR}{\o{\R}}
\newcommand{\liminfn}{\liminf_{n\to\infty}}
\newcommand{\limsupn}{\limsup_{n\to\infty}}
%\newcommand{\limn}{\lim_{n\to\infty}}
\newcommand{\esssup}{\mathrm{ess.sup}}
\newcommand{\asto}{\xrightarrow{\as}}
\newcommand{\Cov}{\mathrm{Cov}}
\newcommand{\cQ}{\mathcal{Q}}
\newcommand{\VC}{\mathrm{VC}}
\newcommand{\mb}{\mathrm{mb}}
\newcommand{\Avar}{\mathrm{Avar}}
\newcommand{\bB}{\mathbb{B}}
\newcommand{\bW}{\mathbb{W}}
\newcommand{\sd}{\mathrm{sd}}
\newcommand{\w}[1]{\widehat{#1}}
\newcommand{\bZ}{\mathbb{Z}}
\newcommand{\Bernoulli}{\mathrm{Bernoulli}}
\newcommand{\Mult}{\mathrm{Mult}}
\newcommand{\BPois}{\mathrm{BPois}}
\newcommand{\fraks}{\mathfrak{s}}
\newcommand{\frakk}{\mathfrak{k}}
\newcommand{\IF}{\mathrm{IF}}
\newcommand{\bX}{\mathbf{X}}
\newcommand{\bx}{\mathbf{x}}
\newcommand{\indep}{\raisebox{0.05em}{\rotatebox[origin=c]{90}{$\models$}}}
\newcommand{\IG}{\mathrm{IG}}
\newcommand{\Levy}{\mathrm{Levy}}
\newcommand{\MP}{\mathrm{MP}}
\newcommand{\Hermite}{\mathrm{Hermite}}
\newcommand{\Skellam}{\mathrm{Skellam}}
\newcommand{\Dirichlet}{\mathrm{Dirichlet}}
\newcommand{\Beta}{\mathrm{Beta}}
\newcommand{\bE}{\mathbb{E}}
\newcommand{\bG}{\mathbb{G}}
\newcommand{\MISE}{\mathrm{MISE}}
\newcommand{\logit}{\mathtt{logit}}
\newcommand{\expit}{\mathtt{expit}}
\newcommand{\cK}{\mathcal{K}}
\newcommand{\dl}{\dot{l}}
\newcommand{\dotp}{\dot{p}}
\newcommand{\wl}{\wt{l}}
%%% 函数解析
\renewcommand{\c}{\mathbf{c}}
\newcommand{\loc}{\mathrm{loc}}
\newcommand{\Lh}{\mathrm{L.h.}}
\newcommand{\Epi}{\mathrm{Epi}\;}
\newcommand{\slim}{\mathrm{slim}}
\newcommand{\Ban}{\mathrm{Ban}}
\newcommand{\Hilb}{\mathrm{Hilb}}
\newcommand{\Ex}{\mathrm{Ex}}
\newcommand{\Co}{\mathrm{Co}}
\newcommand{\sa}{\mathrm{sa}}
\newcommand{\nnorm}[1]{{\left\vert\kern-0.25ex\left\vert\kern-0.25ex\left\vert #1 \right\vert\kern-0.25ex\right\vert\kern-0.25ex\right\vert}}
\newcommand{\dvol}{\mathrm{dvol}}
\newcommand{\Sconv}{\mathrm{Sconv}}
\newcommand{\I}{\mathcal{I}}
\newcommand{\nonunital}{\mathrm{nu}}
\newcommand{\cpt}{\mathrm{cpt}}
\newcommand{\lcpt}{\mathrm{lcpt}}
\newcommand{\com}{\mathrm{com}}
\newcommand{\Haus}{\mathrm{Haus}}
\newcommand{\proper}{\mathrm{proper}}
\newcommand{\infinity}{\mathrm{inf}}
\newcommand{\TVS}{\mathrm{TVS}}
\newcommand{\ess}{\mathrm{ess}}
\newcommand{\ext}{\mathrm{ext}}
\newcommand{\Index}{\mathrm{Index}}
\newcommand{\SSR}{\mathrm{SSR}}
\newcommand{\vs}{\mathrm{vs.}}
\newcommand{\fM}{\mathfrak{M}}
\newcommand{\EDM}{\mathrm{EDM}}
\newcommand{\Tw}{\mathrm{Tw}}
\newcommand{\fC}{\mathfrak{C}}
\newcommand{\bn}{\mathbf{n}}
\newcommand{\br}{\mathbf{r}}
\newcommand{\Lam}{\Lambda}
\newcommand{\lam}{\lambda}
\newcommand{\one}{\mathbf{1}}
\newcommand{\dae}{\text{-a.e.}}
\newcommand{\td}{\text{-}}
\newcommand{\RM}{\mathrm{RM}}
%%% 最適化
\newcommand{\Minimize}{\text{Minimize}}
\newcommand{\subjectto}{\text{subject to}}
\newcommand{\Ri}{\mathrm{Ri}}
%\newcommand{\Cl}{\mathrm{Cl}}
\newcommand{\Cone}{\mathrm{Cone}}
\newcommand{\Int}{\mathrm{Int}}
%%% 圏
\newcommand{\varlim}{\varprojlim}
\newcommand{\Hom}{\mathrm{Hom}}
\newcommand{\Iso}{\mathrm{Iso}}
\newcommand{\Mor}{\mathrm{Mor}}
\newcommand{\Isom}{\mathrm{Isom}}
\newcommand{\Aut}{\mathrm{Aut}}
\newcommand{\End}{\mathrm{End}}
\newcommand{\op}{\mathrm{op}}
\newcommand{\ev}{\mathrm{ev}}
\newcommand{\Ob}{\mathrm{Ob}}
\newcommand{\Ar}{\mathrm{Ar}}
\newcommand{\Arr}{\mathrm{Arr}}
\newcommand{\Set}{\mathrm{Set}}
\newcommand{\Grp}{\mathrm{Grp}}
\newcommand{\Cat}{\mathrm{Cat}}
\newcommand{\Mon}{\mathrm{Mon}}
\newcommand{\CMon}{\mathrm{CMon}} %Comutative Monoid 可換単系とモノイドの射
\newcommand{\Ring}{\mathrm{Ring}}
\newcommand{\CRing}{\mathrm{CRing}}
\newcommand{\Ab}{\mathrm{Ab}}
\newcommand{\Pos}{\mathrm{Pos}}
\newcommand{\Vect}{\mathrm{Vect}}
\newcommand{\FinVect}{\mathrm{FinVect}}
\newcommand{\FinSet}{\mathrm{FinSet}}
\newcommand{\OmegaAlg}{\Omega$-$\mathrm{Alg}}
\newcommand{\OmegaEAlg}{(\Omega,E)$-$\mathrm{Alg}}
\newcommand{\Alg}{\mathrm{Alg}} %代数の圏
\newcommand{\CAlg}{\mathrm{CAlg}} %可換代数の圏
\newcommand{\CPO}{\mathrm{CPO}} %Complete Partial Order & continuous mappings
\newcommand{\Fun}{\mathrm{Fun}}
\newcommand{\Func}{\mathrm{Func}}
\newcommand{\Met}{\mathrm{Met}} %Metric space & Contraction maps
\newcommand{\Pfn}{\mathrm{Pfn}} %Sets & Partial function
\newcommand{\Rel}{\mathrm{Rel}} %Sets & relation
\newcommand{\Bool}{\mathrm{Bool}}
\newcommand{\CABool}{\mathrm{CABool}}
\newcommand{\CompBoolAlg}{\mathrm{CompBoolAlg}}
\newcommand{\BoolAlg}{\mathrm{BoolAlg}}
\newcommand{\BoolRng}{\mathrm{BoolRng}}
\newcommand{\HeytAlg}{\mathrm{HeytAlg}}
\newcommand{\CompHeytAlg}{\mathrm{CompHeytAlg}}
\newcommand{\Lat}{\mathrm{Lat}}
\newcommand{\CompLat}{\mathrm{CompLat}}
\newcommand{\SemiLat}{\mathrm{SemiLat}}
\newcommand{\Stone}{\mathrm{Stone}}
\newcommand{\Sob}{\mathrm{Sob}} %Sober space & continuous map
\newcommand{\Op}{\mathrm{Op}} %Category of open subsets
\newcommand{\Sh}{\mathrm{Sh}} %Category of sheave
\newcommand{\PSh}{\mathrm{PSh}} %Category of presheave, PSh(C)=[C^op,set]のこと
\newcommand{\Conv}{\mathrm{Conv}} %Convergence spaceの圏
\newcommand{\Unif}{\mathrm{Unif}} %一様空間と一様連続写像の圏
\newcommand{\Frm}{\mathrm{Frm}} %フレームとフレームの射
\newcommand{\Locale}{\mathrm{Locale}} %その反対圏
\newcommand{\Diff}{\mathrm{Diff}} %滑らかな多様体の圏
\newcommand{\Mfd}{\mathrm{Mfd}}
\newcommand{\LieAlg}{\mathrm{LieAlg}}
\newcommand{\Quiv}{\mathrm{Quiv}} %Quiverの圏
\newcommand{\B}{\mathcal{B}}
\newcommand{\Span}{\mathrm{Span}}
\newcommand{\Corr}{\mathrm{Corr}}
\newcommand{\Decat}{\mathrm{Decat}}
\newcommand{\Rep}{\mathrm{Rep}}
\newcommand{\Grpd}{\mathrm{Grpd}}
\newcommand{\sSet}{\mathrm{sSet}}
\newcommand{\Mod}{\mathrm{Mod}}
\newcommand{\SmoothMnf}{\mathrm{SmoothMnf}}
\newcommand{\coker}{\mathrm{coker}}

\newcommand{\Ord}{\mathrm{Ord}}
\newcommand{\eq}{\mathrm{eq}}
\newcommand{\coeq}{\mathrm{coeq}}
\newcommand{\act}{\mathrm{act}}

%%%%%%%%%%%%%%% 定理環境(足助先生ありがとうございます) %%%%%%%%%%%%%%%

\everymath{\displaystyle}
\renewcommand{\proofname}{\bf [証明]}
\renewcommand{\thefootnote}{\dag\arabic{footnote}} %足助さんからもらった.どうなるんだ?
\renewcommand{\qedsymbol}{$\blacksquare$}

\renewcommand{\labelenumi}{(\arabic{enumi})} %(1),(2),...がデフォルトであって欲しい
\renewcommand{\labelenumii}{(\alph{enumii})}
\renewcommand{\labelenumiii}{(\roman{enumiii})}

\newtheoremstyle{StatementsWithStar}% ?name?
{3pt}% ?Space above? 1
{3pt}% ?Space below? 1
{}% ?Body font?
{}% ?Indent amount? 2
{\bfseries}% ?Theorem head font?
{\textbf{.}}% ?Punctuation after theorem head?
{.5em}% ?Space after theorem head? 3
{\textbf{\textup{#1~\thetheorem{}}}{}\,$^{\ast}$\thmnote{(#3)}}% ?Theorem head spec (can be left empty, meaning ‘normal’)?
%
\newtheoremstyle{StatementsWithStar2}% ?name?
{3pt}% ?Space above? 1
{3pt}% ?Space below? 1
{}% ?Body font?
{}% ?Indent amount? 2
{\bfseries}% ?Theorem head font?
{\textbf{.}}% ?Punctuation after theorem head?
{.5em}% ?Space after theorem head? 3
{\textbf{\textup{#1~\thetheorem{}}}{}\,$^{\ast\ast}$\thmnote{(#3)}}% ?Theorem head spec (can be left empty, meaning ‘normal’)?
%
\newtheoremstyle{StatementsWithStar3}% ?name?
{3pt}% ?Space above? 1
{3pt}% ?Space below? 1
{}% ?Body font?
{}% ?Indent amount? 2
{\bfseries}% ?Theorem head font?
{\textbf{.}}% ?Punctuation after theorem head?
{.5em}% ?Space after theorem head? 3
{\textbf{\textup{#1~\thetheorem{}}}{}\,$^{\ast\ast\ast}$\thmnote{(#3)}}% ?Theorem head spec (can be left empty, meaning ‘normal’)?
%
\newtheoremstyle{StatementsWithCCirc}% ?name?
{6pt}% ?Space above? 1
{6pt}% ?Space below? 1
{}% ?Body font?
{}% ?Indent amount? 2
{\bfseries}% ?Theorem head font?
{\textbf{.}}% ?Punctuation after theorem head?
{.5em}% ?Space after theorem head? 3
{\textbf{\textup{#1~\thetheorem{}}}{}\,$^{\circledcirc}$\thmnote{(#3)}}% ?Theorem head spec (can be left empty, meaning ‘normal’)?
%
\theoremstyle{definition}
 \newtheorem{theorem}{定理}[section]
 \newtheorem{axiom}[theorem]{公理}
 \newtheorem{corollary}[theorem]{系}
 \newtheorem{proposition}[theorem]{命題}
 \newtheorem*{proposition*}{命題}
 \newtheorem{lemma}[theorem]{補題}
 \newtheorem*{lemma*}{補題}
 \newtheorem*{theorem*}{定理}
 \newtheorem{definition}[theorem]{定義}
 \newtheorem{example}[theorem]{例}
 \newtheorem{notation}[theorem]{記法}
 \newtheorem*{notation*}{記法}
 \newtheorem{assumption}[theorem]{仮定}
 \newtheorem{question}[theorem]{問}
 \newtheorem{counterexample}[theorem]{反例}
 \newtheorem{reidai}[theorem]{例題}
 \newtheorem{ruidai}[theorem]{類題}
 \newtheorem{problem}[theorem]{問題}
 \newtheorem{algorithm}[theorem]{算譜}
 \newtheorem*{solution*}{\bf{[解]}}
 \newtheorem{discussion}[theorem]{議論}
 \newtheorem{remark}[theorem]{注}
 \newtheorem{remarks}[theorem]{要諦}
 \newtheorem{image}[theorem]{描像}
 \newtheorem{observation}[theorem]{観察}
 \newtheorem{universality}[theorem]{普遍性} %非自明な例外がない.
 \newtheorem{universal tendency}[theorem]{普遍傾向} %例外が有意に少ない.
 \newtheorem{hypothesis}[theorem]{仮説} %実験で説明されていない理論.
 \newtheorem{theory}[theorem]{理論} %実験事実とその(さしあたり)整合的な説明.
 \newtheorem{fact}[theorem]{実験事実}
 \newtheorem{model}[theorem]{模型}
 \newtheorem{explanation}[theorem]{説明} %理論による実験事実の説明
 \newtheorem{anomaly}[theorem]{理論の限界}
 \newtheorem{application}[theorem]{応用例}
 \newtheorem{method}[theorem]{手法} %実験手法など,技術的問題.
 \newtheorem{history}[theorem]{歴史}
 \newtheorem{usage}[theorem]{用語法}
 \newtheorem{research}[theorem]{研究}
 \newtheorem{shishin}[theorem]{指針}
 \newtheorem{yodan}[theorem]{余談}
 \newtheorem{construction}[theorem]{構成}
% \newtheorem*{remarknonum}{注}
 \newtheorem*{definition*}{定義}
 \newtheorem*{remark*}{注}
 \newtheorem*{question*}{問}
 \newtheorem*{problem*}{問題}
 \newtheorem*{axiom*}{公理}
 \newtheorem*{example*}{例}
 \newtheorem*{corollary*}{系}
 \newtheorem*{shishin*}{指針}
 \newtheorem*{yodan*}{余談}
 \newtheorem*{kadai*}{課題}
%
\theoremstyle{StatementsWithStar}
 \newtheorem{definition_*}[theorem]{定義}
 \newtheorem{question_*}[theorem]{問}
 \newtheorem{example_*}[theorem]{例}
 \newtheorem{theorem_*}[theorem]{定理}
 \newtheorem{remark_*}[theorem]{注}
%
\theoremstyle{StatementsWithStar2}
 \newtheorem{definition_**}[theorem]{定義}
 \newtheorem{theorem_**}[theorem]{定理}
 \newtheorem{question_**}[theorem]{問}
 \newtheorem{remark_**}[theorem]{注}
%
\theoremstyle{StatementsWithStar3}
 \newtheorem{remark_***}[theorem]{注}
 \newtheorem{question_***}[theorem]{問}
%
\theoremstyle{StatementsWithCCirc}
 \newtheorem{definition_O}[theorem]{定義}
 \newtheorem{question_O}[theorem]{問}
 \newtheorem{example_O}[theorem]{例}
 \newtheorem{remark_O}[theorem]{注}
%
\theoremstyle{definition}
%
\raggedbottom
\allowdisplaybreaks
%\usepackage{mathtools}
%\mathtoolsset{showonlyrefs=true} %labelを附した数式にのみ附番される設定.
%\usepackage{amsmath} %mathtoolsの内部で呼ばれるので要らない.
\usepackage{amsfonts} %mathfrak, mathcal, mathbbなど.
\usepackage{amsthm} %定理環境.
\usepackage{amssymb} %AMSFontsを使うためのパッケージ.
\usepackage{ascmac} %screen, itembox, shadebox環境.全てLATEX2εの標準機能の範囲で作られたもの.
\usepackage{comment} %comment環境を用いて,複数行をcomment outできるようにするpackage
\usepackage{wrapfig} %図の周りに文字をwrapさせることができる.詳細な制御ができる.
\usepackage[usenames, dvipsnames]{xcolor} %xcolorはcolorの拡張.optionの意味はdvipsnamesはLoad a set of predefined colors. forestgreenなどの色が追加されている.usenamesはobsoleteとだけ書いてあった.
\setcounter{tocdepth}{2} %目次に表示される深さ.2はsubsectionまで
\usepackage{multicol} %\begin{multicols}{2}環境で途中からmulticolumnに出来る.

\usepackage{url}
\usepackage[dvipdfmx,colorlinks,linkcolor=blue,urlcolor=blue]{hyperref} %生成されるPDFファイルにおいて、\tableofcontentsによって書き出された目次をクリックすると該当する見出しへジャンプしたり、さらには、\label{ラベル名}を番号で参照する\ref{ラベル名}やthebibliography環境において\bibitem{ラベル名}を文献番号で参照する\cite{ラベル名}においても番号をクリックすると該当箇所にジャンプする.囲み枠はダサいので,colorlinksで囲み廃止し,リンク自体に色を付けることにした.
\usepackage{pxjahyper} %pxrubrica同様,八登崇之さん.hyperrefは日本語pLaTeXに最適化されていないから,hyperrefとセットで,(u)pLaTeX+hyperref+dvipdfmxの組み合わせで日本語を含む「しおり」をもつPDF文書を作成する場合に必要となる機能を提供する
\definecolor{花緑青}{cmyk}{0.52,0.03,0,0.27}
\definecolor{サーモンピンク}{cmyk}{0,0.65,0.65,0.05}
\definecolor{暗中模索}{rgb}{0.2,0.2,0.2}

\usepackage{tikz}
\usetikzlibrary{positioning,automata} %automaton描画のため
\usepackage{tikz-cd}
\usepackage[all]{xy}
\def\objectstyle{\displaystyle} %デフォルトではxymatrix中の数式が文中数式モードになるので,それを直す.\labelstyleも同様にxy packageの中で定義されており,文中数式モードになっている.

\usepackage[version=4]{mhchem} %化学式をTikZで簡単に書くためのパッケージ.
\usepackage{chemfig} %化学構造式をTikZで描くためのパッケージ.
\usepackage{siunitx} %IS単位を書くためのパッケージ

\usepackage{ulem} %取り消し線を引くためのパッケージ
\usepackage{pxrubrica} %日本語にルビをふる.八登崇之(やとうたかゆき)氏による.

\usepackage{graphicx} %rotatebox, scalebox, reflectbox, resizeboxなどのコマンドや,図表の読み込み\includegraphicsを司る.graphics というパッケージもありますが,graphicx はこれを高機能にしたものと考えて結構です(ただし graphicx は内部で graphics を読み込みます)

\usepackage[breakable]{tcolorbox} %加藤晃史さんがフル活用していたtcolorboxを,途中改ページ可能で.
\tcbuselibrary{theorems} %https://qiita.com/t_kemmochi/items/483b8fcdb5db8d1f5d5e
\usepackage{enumerate} %enumerate環境を凝らせる.
\usepackage[top=15truemm,bottom=15truemm,left=10truemm,right=10truemm]{geometry} %足助さんからもらったオプション

%%%%%%%%%%%%%%% 環境マクロ %%%%%%%%%%%%%%%

\usepackage{listings} %ソースコードを表示できる環境.多分もっといい方法ある.
\usepackage{jvlisting} %日本語のコメントアウトをする場合jlistingが必要
\lstset{ %ここからソースコードの表示に関する設定.lstlisting環境では,[caption=hoge,label=fuga]などのoptionを付けられる.
%[escapechar=!]とすると,LaTeXコマンドを使える.
  basicstyle={\ttfamily},
  identifierstyle={\small},
  commentstyle={\smallitshape},
  keywordstyle={\small\bfseries},
  ndkeywordstyle={\small},
  stringstyle={\small\ttfamily},
  frame={tb},
  breaklines=true,
  columns=[l]{fullflexible},
  numbers=left,
  xrightmargin=0zw,
  xleftmargin=3zw,
  numberstyle={\scriptsize},
  stepnumber=1,
  numbersep=1zw,
  lineskip=-0.5ex
}
%\makeatletter %caption番号を「[chapter番号].[section番号].[subsection番号]-[そのsubsection内においてn番目]」に変更
%    \AtBeginDocument{
%    \renewcommand*{\thelstlisting}{\arabic{chapter}.\arabic{section}.\arabic{lstlisting}}
%    \@addtoreset{lstlisting}{section}
%    }
%\makeatother
\renewcommand{\lstlistingname}{算譜} %caption名を"program"に変更

\newtcolorbox{tbox}[3][]{%
colframe=#2,colback=#2!10,coltitle=#2!20!black,title={#3},#1}

%%%%%%%%%%%%%%% フォント %%%%%%%%%%%%%%%

\usepackage{textcomp, mathcomp} %Text Companionとは,T1 encodingに入らなかった文字群.これを使うためのパッケージ.\textsectionでブルバキに!
\usepackage[T1]{fontenc} %8bitエンコーディングにする.comp系拡張数学文字の動作が安定する.

%%%%%%%%%%%%%%% 数学記号のマクロ %%%%%%%%%%%%%%%

\newcommand{\abs}[1]{\lvert#1\rvert} %mathtoolsはこうやって使うのか!
\newcommand{\Abs}[1]{\left|#1\right|}
\newcommand{\norm}[1]{\|#1\|}
\newcommand{\Norm}[1]{\left\|#1\right\|}
%\newcommand{\brace}[1]{\{#1\}}
\newcommand{\Brace}[1]{\left\{#1\right\}}
\newcommand{\paren}[1]{\left(#1\right)}
\newcommand{\bracket}[1]{\langle#1\rangle}
\newcommand{\brac}[1]{\langle#1\rangle}
\newcommand{\Bracket}[1]{\left\langle#1\right\rangle}
\newcommand{\Brac}[1]{\left\langle#1\right\rangle}
\newcommand{\Square}[1]{\left[#1\right]}
\renewcommand{\o}[1]{\overline{#1}}
\renewcommand{\u}[1]{\underline{#1}}
\renewcommand{\iff}{\;\mathrm{iff}\;} %nLabリスペクト
\newcommand{\pp}[2]{\frac{\partial #1}{\partial #2}}
\newcommand{\ppp}[3]{\frac{\partial #1}{\partial #2\partial #3}}
\newcommand{\dd}[2]{\frac{d #1}{d #2}}
\newcommand{\floor}[1]{\lfloor#1\rfloor}
\newcommand{\Floor}[1]{\left\lfloor#1\right\rfloor}
\newcommand{\ceil}[1]{\lceil#1\rceil}

\newcommand{\iso}{\xrightarrow{\,\smash{\raisebox{-0.45ex}{\ensuremath{\scriptstyle\sim}}}\,}}
\newcommand{\wt}[1]{\widetilde{#1}}
\newcommand{\wh}[1]{\widehat{#1}}

\newcommand{\Lrarrow}{\;\;\Leftrightarrow\;\;}

%ノルム位相についての閉包 https://newbedev.com/how-to-make-double-overline-with-less-vertical-displacement
\makeatletter
\newcommand{\dbloverline}[1]{\overline{\dbl@overline{#1}}}
\newcommand{\dbl@overline}[1]{\mathpalette\dbl@@overline{#1}}
\newcommand{\dbl@@overline}[2]{%
  \begingroup
  \sbox\z@{$\m@th#1\overline{#2}$}%
  \ht\z@=\dimexpr\ht\z@-2\dbl@adjust{#1}\relax
  \box\z@
  \ifx#1\scriptstyle\kern-\scriptspace\else
  \ifx#1\scriptscriptstyle\kern-\scriptspace\fi\fi
  \endgroup
}
\newcommand{\dbl@adjust}[1]{%
  \fontdimen8
  \ifx#1\displaystyle\textfont\else
  \ifx#1\textstyle\textfont\else
  \ifx#1\scriptstyle\scriptfont\else
  \scriptscriptfont\fi\fi\fi 3
}
\makeatother
\newcommand{\oo}[1]{\dbloverline{#1}}

\DeclareMathOperator{\grad}{\mathrm{grad}}
\DeclareMathOperator{\rot}{\mathrm{rot}}
\DeclareMathOperator{\divergence}{\mathrm{div}}
\newcommand{\False}{\mathrm{False}}
\newcommand{\True}{\mathrm{True}}
\DeclareMathOperator{\tr}{\mathrm{tr}}
\newcommand{\M}{\mathcal{M}}
\newcommand{\cF}{\mathcal{F}}
\newcommand{\cD}{\mathcal{D}}
\newcommand{\fX}{\mathfrak{X}}
\newcommand{\fY}{\mathfrak{Y}}
\newcommand{\fZ}{\mathfrak{Z}}
\renewcommand{\H}{\mathcal{H}}
\newcommand{\fH}{\mathfrak{H}}
\newcommand{\bH}{\mathbb{H}}
\newcommand{\id}{\mathrm{id}}
\newcommand{\A}{\mathcal{A}}
% \renewcommand\coprod{\rotatebox[origin=c]{180}{$\prod$}} すでにどこかにある.
\newcommand{\pr}{\mathrm{pr}}
\newcommand{\U}{\mathfrak{U}}
\newcommand{\Map}{\mathrm{Map}}
\newcommand{\dom}{\mathrm{Dom}\;}
\newcommand{\cod}{\mathrm{Cod}\;}
\newcommand{\supp}{\mathrm{supp}\;}
\newcommand{\otherwise}{\mathrm{otherwise}}
\newcommand{\st}{\;\mathrm{s.t.}\;}
\newcommand{\lmd}{\lambda}
\newcommand{\Lmd}{\Lambda}
%%% 線型代数学
\newcommand{\Ker}{\mathrm{Ker}\;}
\newcommand{\Coker}{\mathrm{Coker}\;}
\newcommand{\Coim}{\mathrm{Coim}\;}
\newcommand{\rank}{\mathrm{rank}}
\newcommand{\lcm}{\mathrm{lcm}}
\newcommand{\sgn}{\mathrm{sgn}}
\newcommand{\GL}{\mathrm{GL}}
\newcommand{\SL}{\mathrm{SL}}
\newcommand{\alt}{\mathrm{alt}}
%%% 複素解析学
\renewcommand{\Re}{\mathrm{Re}\;}
\renewcommand{\Im}{\mathrm{Im}\;}
\newcommand{\Gal}{\mathrm{Gal}}
\newcommand{\PGL}{\mathrm{PGL}}
\newcommand{\PSL}{\mathrm{PSL}}
\newcommand{\Log}{\mathrm{Log}\,}
\newcommand{\Res}{\mathrm{Res}\,}
\newcommand{\on}{\mathrm{on}\;}
\newcommand{\hatC}{\hat{\C}}
\newcommand{\hatR}{\hat{\R}}
\newcommand{\PV}{\mathrm{P.V.}}
\newcommand{\diam}{\mathrm{diam}}
\newcommand{\Area}{\mathrm{Area}}
\newcommand{\Lap}{\Laplace}
\newcommand{\f}{\mathbf{f}}
\newcommand{\cR}{\mathcal{R}}
\newcommand{\const}{\mathrm{const.}}
\newcommand{\Om}{\Omega}
\newcommand{\Cinf}{C^\infty}
\newcommand{\ep}{\epsilon}
\newcommand{\dist}{\mathrm{dist}}
\newcommand{\opart}{\o{\partial}}
%%% 解析力学
\newcommand{\x}{\mathbf{x}}
%%% 集合と位相
\renewcommand{\O}{\mathcal{O}}
\renewcommand{\S}{\mathcal{S}}
\renewcommand{\U}{\mathcal{U}}
\newcommand{\V}{\mathcal{V}}
\renewcommand{\P}{\mathcal{P}}
\newcommand{\R}{\mathbb{R}}
\newcommand{\N}{\mathbb{N}}
\newcommand{\C}{\mathbb{C}}
\newcommand{\Z}{\mathbb{Z}}
\newcommand{\Q}{\mathbb{Q}}
\newcommand{\TV}{\mathrm{TV}}
\newcommand{\ORD}{\mathrm{ORD}}
\newcommand{\Tr}{\mathrm{Tr}\;}
\newcommand{\Card}{\mathrm{Card}\;}
\newcommand{\Top}{\mathrm{Top}}
\newcommand{\Disc}{\mathrm{Disc}}
\newcommand{\Codisc}{\mathrm{Codisc}}
\newcommand{\CoDisc}{\mathrm{CoDisc}}
\newcommand{\Ult}{\mathrm{Ult}}
\newcommand{\ord}{\mathrm{ord}}
\newcommand{\maj}{\mathrm{maj}}
%%% 形式言語理論
\newcommand{\REGEX}{\mathrm{REGEX}}
\newcommand{\RE}{\mathbf{RE}}

%%% Fourier解析
\newcommand*{\Laplace}{\mathop{}\!\mathbin\bigtriangleup}
\newcommand*{\DAlambert}{\mathop{}\!\mathbin\Box}
%%% Graph Theory
\newcommand{\SimpGph}{\mathrm{SimpGph}}
\newcommand{\Gph}{\mathrm{Gph}}
\newcommand{\mult}{\mathrm{mult}}
\newcommand{\inv}{\mathrm{inv}}
%%% 多様体
\newcommand{\Der}{\mathrm{Der}}
\newcommand{\osub}{\overset{\mathrm{open}}{\subset}}
\newcommand{\osup}{\overset{\mathrm{open}}{\supset}}
\newcommand{\al}{\alpha}
\newcommand{\K}{\mathbb{K}}
\newcommand{\Sp}{\mathrm{Sp}}
\newcommand{\g}{\mathfrak{g}}
\newcommand{\h}{\mathfrak{h}}
\newcommand{\Exp}{\mathrm{Exp}\;}
\newcommand{\Imm}{\mathrm{Imm}}
\newcommand{\Imb}{\mathrm{Imb}}
\newcommand{\codim}{\mathrm{codim}\;}
\newcommand{\Gr}{\mathrm{Gr}}
%%% 代数
\newcommand{\Ad}{\mathrm{Ad}}
\newcommand{\finsupp}{\mathrm{fin\;supp}}
\newcommand{\SO}{\mathrm{SO}}
\newcommand{\SU}{\mathrm{SU}}
\newcommand{\acts}{\curvearrowright}
\newcommand{\mono}{\hookrightarrow}
\newcommand{\epi}{\twoheadrightarrow}
\newcommand{\Stab}{\mathrm{Stab}}
\newcommand{\nor}{\mathrm{nor}}
\newcommand{\T}{\mathbb{T}}
\newcommand{\Aff}{\mathrm{Aff}}
\newcommand{\rsub}{\triangleleft}
\newcommand{\rsup}{\triangleright}
\newcommand{\subgrp}{\overset{\mathrm{subgrp}}{\subset}}
\newcommand{\Ext}{\mathrm{Ext}}
\newcommand{\sbs}{\subset}\newcommand{\sps}{\supset}
\newcommand{\In}{\mathrm{In}}
\newcommand{\Tor}{\mathrm{Tor}}
\newcommand{\p}{\mathfrak{p}}
\newcommand{\q}{\mathfrak{q}}
\newcommand{\m}{\mathfrak{m}}
\newcommand{\cS}{\mathcal{S}}
\newcommand{\Frac}{\mathrm{Frac}\,}
\newcommand{\Spec}{\mathrm{Spec}\,}
\newcommand{\bA}{\mathbb{A}}
\newcommand{\Sym}{\mathrm{Sym}}
\newcommand{\Ann}{\mathrm{Ann}}
%%% 代数的位相幾何学
\newcommand{\Ho}{\mathrm{Ho}}
\newcommand{\CW}{\mathrm{CW}}
\newcommand{\lc}{\mathrm{lc}}
\newcommand{\cg}{\mathrm{cg}}
\newcommand{\Fib}{\mathrm{Fib}}
\newcommand{\Cyl}{\mathrm{Cyl}}
\newcommand{\Ch}{\mathrm{Ch}}
%%% 数値解析
\newcommand{\round}{\mathrm{round}}
\newcommand{\cond}{\mathrm{cond}}
\newcommand{\diag}{\mathrm{diag}}
%%% 確率論
\newcommand{\calF}{\mathcal{F}}
\newcommand{\X}{\mathcal{X}}
\newcommand{\Meas}{\mathrm{Meas}}
\newcommand{\as}{\;\mathrm{a.s.}} %almost surely
\newcommand{\io}{\;\mathrm{i.o.}} %infinitely often
\newcommand{\fe}{\;\mathrm{f.e.}} %with a finite number of exceptions
\newcommand{\F}{\mathcal{F}}
\newcommand{\bF}{\mathbb{F}}
\newcommand{\W}{\mathcal{W}}
\newcommand{\Pois}{\mathrm{Pois}}
\newcommand{\iid}{\mathrm{i.i.d.}}
\newcommand{\wconv}{\rightsquigarrow}
\newcommand{\Var}{\mathrm{Var}}
\newcommand{\xrightarrown}{\xrightarrow{n\to\infty}}
\newcommand{\au}{\mathrm{au}}
\newcommand{\cT}{\mathcal{T}}
%%% 情報理論
\newcommand{\bit}{\mathrm{bit}}
%%% 積分論
\newcommand{\calA}{\mathcal{A}}
\newcommand{\calB}{\mathcal{B}}
\newcommand{\D}{\mathcal{D}}
\newcommand{\Y}{\mathcal{Y}}
\newcommand{\calC}{\mathcal{C}}
\renewcommand{\ae}{\mathrm{a.e.}\;}
\newcommand{\cZ}{\mathcal{Z}}
\newcommand{\fF}{\mathfrak{F}}
\newcommand{\fI}{\mathfrak{I}}
\newcommand{\E}{\mathcal{E}}
\newcommand{\sMap}{\sigma\textrm{-}\mathrm{Map}}
\DeclareMathOperator*{\argmax}{arg\,max}
\DeclareMathOperator*{\argmin}{arg\,min}
\newcommand{\cC}{\mathcal{C}}
\newcommand{\comp}{\complement}
\newcommand{\J}{\mathcal{J}}
\newcommand{\sumN}[1]{\sum_{#1\in\N}}
\newcommand{\cupN}[1]{\cup_{#1\in\N}}
\newcommand{\capN}[1]{\cap_{#1\in\N}}
\newcommand{\Sum}[1]{\sum_{#1=1}^\infty}
\newcommand{\sumn}{\sum_{n=1}^\infty}
\newcommand{\summ}{\sum_{m=1}^\infty}
\newcommand{\sumk}{\sum_{k=1}^\infty}
\newcommand{\sumi}{\sum_{i=1}^\infty}
\newcommand{\sumj}{\sum_{j=1}^\infty}
\newcommand{\cupn}{\cup_{n=1}^\infty}
\newcommand{\capn}{\cap_{n=1}^\infty}
\newcommand{\cupk}{\cup_{k=1}^\infty}
\newcommand{\cupi}{\cup_{i=1}^\infty}
\newcommand{\cupj}{\cup_{j=1}^\infty}
\newcommand{\limn}{\lim_{n\to\infty}}
\renewcommand{\l}{\mathcal{l}}
\renewcommand{\L}{\mathcal{L}}
\newcommand{\Cl}{\mathrm{Cl}}
\newcommand{\cN}{\mathcal{N}}
\newcommand{\Ae}{\textrm{-a.e.}\;}
\newcommand{\csub}{\overset{\textrm{closed}}{\subset}}
\newcommand{\csup}{\overset{\textrm{closed}}{\supset}}
\newcommand{\wB}{\wt{B}}
\newcommand{\cG}{\mathcal{G}}
\newcommand{\Lip}{\mathrm{Lip}}
\newcommand{\Dom}{\mathrm{Dom}}
%%% 数理ファイナンス
\newcommand{\pre}{\mathrm{pre}}
\newcommand{\om}{\omega}

%%% 統計的因果推論
\newcommand{\Do}{\mathrm{Do}}
%%% 数理統計
\newcommand{\bP}{\mathbb{P}}
\newcommand{\compsub}{\overset{\textrm{cpt}}{\subset}}
\newcommand{\lip}{\textrm{lip}}
\newcommand{\BL}{\mathrm{BL}}
\newcommand{\G}{\mathbb{G}}
\newcommand{\NB}{\mathrm{NB}}
\newcommand{\oR}{\o{\R}}
\newcommand{\liminfn}{\liminf_{n\to\infty}}
\newcommand{\limsupn}{\limsup_{n\to\infty}}
%\newcommand{\limn}{\lim_{n\to\infty}}
\newcommand{\esssup}{\mathrm{ess.sup}}
\newcommand{\asto}{\xrightarrow{\as}}
\newcommand{\Cov}{\mathrm{Cov}}
\newcommand{\cQ}{\mathcal{Q}}
\newcommand{\VC}{\mathrm{VC}}
\newcommand{\mb}{\mathrm{mb}}
\newcommand{\Avar}{\mathrm{Avar}}
\newcommand{\bB}{\mathbb{B}}
\newcommand{\bW}{\mathbb{W}}
\newcommand{\sd}{\mathrm{sd}}
\newcommand{\w}[1]{\widehat{#1}}
\newcommand{\bZ}{\mathbb{Z}}
\newcommand{\Bernoulli}{\mathrm{Bernoulli}}
\newcommand{\Mult}{\mathrm{Mult}}
\newcommand{\BPois}{\mathrm{BPois}}
\newcommand{\fraks}{\mathfrak{s}}
\newcommand{\frakk}{\mathfrak{k}}
\newcommand{\IF}{\mathrm{IF}}
\newcommand{\bX}{\mathbf{X}}
\newcommand{\bx}{\mathbf{x}}
\newcommand{\indep}{\raisebox{0.05em}{\rotatebox[origin=c]{90}{$\models$}}}
\newcommand{\IG}{\mathrm{IG}}
\newcommand{\Levy}{\mathrm{Levy}}
\newcommand{\MP}{\mathrm{MP}}
\newcommand{\Hermite}{\mathrm{Hermite}}
\newcommand{\Skellam}{\mathrm{Skellam}}
\newcommand{\Dirichlet}{\mathrm{Dirichlet}}
\newcommand{\Beta}{\mathrm{Beta}}
\newcommand{\bE}{\mathbb{E}}
\newcommand{\bG}{\mathbb{G}}
\newcommand{\MISE}{\mathrm{MISE}}
\newcommand{\logit}{\mathtt{logit}}
\newcommand{\expit}{\mathtt{expit}}
\newcommand{\cK}{\mathcal{K}}
\newcommand{\dl}{\dot{l}}
\newcommand{\dotp}{\dot{p}}
\newcommand{\wl}{\wt{l}}
%%% 函数解析
\renewcommand{\c}{\mathbf{c}}
\newcommand{\loc}{\mathrm{loc}}
\newcommand{\Lh}{\mathrm{L.h.}}
\newcommand{\Epi}{\mathrm{Epi}\;}
\newcommand{\slim}{\mathrm{slim}}
\newcommand{\Ban}{\mathrm{Ban}}
\newcommand{\Hilb}{\mathrm{Hilb}}
\newcommand{\Ex}{\mathrm{Ex}}
\newcommand{\Co}{\mathrm{Co}}
\newcommand{\sa}{\mathrm{sa}}
\newcommand{\nnorm}[1]{{\left\vert\kern-0.25ex\left\vert\kern-0.25ex\left\vert #1 \right\vert\kern-0.25ex\right\vert\kern-0.25ex\right\vert}}
\newcommand{\dvol}{\mathrm{dvol}}
\newcommand{\Sconv}{\mathrm{Sconv}}
\newcommand{\I}{\mathcal{I}}
\newcommand{\nonunital}{\mathrm{nu}}
\newcommand{\cpt}{\mathrm{cpt}}
\newcommand{\lcpt}{\mathrm{lcpt}}
\newcommand{\com}{\mathrm{com}}
\newcommand{\Haus}{\mathrm{Haus}}
\newcommand{\proper}{\mathrm{proper}}
\newcommand{\infinity}{\mathrm{inf}}
\newcommand{\TVS}{\mathrm{TVS}}
\newcommand{\ess}{\mathrm{ess}}
\newcommand{\ext}{\mathrm{ext}}
\newcommand{\Index}{\mathrm{Index}}
\newcommand{\SSR}{\mathrm{SSR}}
\newcommand{\vs}{\mathrm{vs.}}
\newcommand{\fM}{\mathfrak{M}}
\newcommand{\EDM}{\mathrm{EDM}}
\newcommand{\Tw}{\mathrm{Tw}}
\newcommand{\fC}{\mathfrak{C}}
\newcommand{\bn}{\mathbf{n}}
\newcommand{\br}{\mathbf{r}}
\newcommand{\Lam}{\Lambda}
\newcommand{\lam}{\lambda}
\newcommand{\one}{\mathbf{1}}
\newcommand{\dae}{\text{-a.e.}}
\newcommand{\td}{\text{-}}
\newcommand{\RM}{\mathrm{RM}}
%%% 最適化
\newcommand{\Minimize}{\text{Minimize}}
\newcommand{\subjectto}{\text{subject to}}
\newcommand{\Ri}{\mathrm{Ri}}
%\newcommand{\Cl}{\mathrm{Cl}}
\newcommand{\Cone}{\mathrm{Cone}}
\newcommand{\Int}{\mathrm{Int}}
%%% 圏
\newcommand{\varlim}{\varprojlim}
\newcommand{\Hom}{\mathrm{Hom}}
\newcommand{\Iso}{\mathrm{Iso}}
\newcommand{\Mor}{\mathrm{Mor}}
\newcommand{\Isom}{\mathrm{Isom}}
\newcommand{\Aut}{\mathrm{Aut}}
\newcommand{\End}{\mathrm{End}}
\newcommand{\op}{\mathrm{op}}
\newcommand{\ev}{\mathrm{ev}}
\newcommand{\Ob}{\mathrm{Ob}}
\newcommand{\Ar}{\mathrm{Ar}}
\newcommand{\Arr}{\mathrm{Arr}}
\newcommand{\Set}{\mathrm{Set}}
\newcommand{\Grp}{\mathrm{Grp}}
\newcommand{\Cat}{\mathrm{Cat}}
\newcommand{\Mon}{\mathrm{Mon}}
\newcommand{\CMon}{\mathrm{CMon}} %Comutative Monoid 可換単系とモノイドの射
\newcommand{\Ring}{\mathrm{Ring}}
\newcommand{\CRing}{\mathrm{CRing}}
\newcommand{\Ab}{\mathrm{Ab}}
\newcommand{\Pos}{\mathrm{Pos}}
\newcommand{\Vect}{\mathrm{Vect}}
\newcommand{\FinVect}{\mathrm{FinVect}}
\newcommand{\FinSet}{\mathrm{FinSet}}
\newcommand{\OmegaAlg}{\Omega$-$\mathrm{Alg}}
\newcommand{\OmegaEAlg}{(\Omega,E)$-$\mathrm{Alg}}
\newcommand{\Alg}{\mathrm{Alg}} %代数の圏
\newcommand{\CAlg}{\mathrm{CAlg}} %可換代数の圏
\newcommand{\CPO}{\mathrm{CPO}} %Complete Partial Order & continuous mappings
\newcommand{\Fun}{\mathrm{Fun}}
\newcommand{\Func}{\mathrm{Func}}
\newcommand{\Met}{\mathrm{Met}} %Metric space & Contraction maps
\newcommand{\Pfn}{\mathrm{Pfn}} %Sets & Partial function
\newcommand{\Rel}{\mathrm{Rel}} %Sets & relation
\newcommand{\Bool}{\mathrm{Bool}}
\newcommand{\CABool}{\mathrm{CABool}}
\newcommand{\CompBoolAlg}{\mathrm{CompBoolAlg}}
\newcommand{\BoolAlg}{\mathrm{BoolAlg}}
\newcommand{\BoolRng}{\mathrm{BoolRng}}
\newcommand{\HeytAlg}{\mathrm{HeytAlg}}
\newcommand{\CompHeytAlg}{\mathrm{CompHeytAlg}}
\newcommand{\Lat}{\mathrm{Lat}}
\newcommand{\CompLat}{\mathrm{CompLat}}
\newcommand{\SemiLat}{\mathrm{SemiLat}}
\newcommand{\Stone}{\mathrm{Stone}}
\newcommand{\Sob}{\mathrm{Sob}} %Sober space & continuous map
\newcommand{\Op}{\mathrm{Op}} %Category of open subsets
\newcommand{\Sh}{\mathrm{Sh}} %Category of sheave
\newcommand{\PSh}{\mathrm{PSh}} %Category of presheave, PSh(C)=[C^op,set]のこと
\newcommand{\Conv}{\mathrm{Conv}} %Convergence spaceの圏
\newcommand{\Unif}{\mathrm{Unif}} %一様空間と一様連続写像の圏
\newcommand{\Frm}{\mathrm{Frm}} %フレームとフレームの射
\newcommand{\Locale}{\mathrm{Locale}} %その反対圏
\newcommand{\Diff}{\mathrm{Diff}} %滑らかな多様体の圏
\newcommand{\Mfd}{\mathrm{Mfd}}
\newcommand{\LieAlg}{\mathrm{LieAlg}}
\newcommand{\Quiv}{\mathrm{Quiv}} %Quiverの圏
\newcommand{\B}{\mathcal{B}}
\newcommand{\Span}{\mathrm{Span}}
\newcommand{\Corr}{\mathrm{Corr}}
\newcommand{\Decat}{\mathrm{Decat}}
\newcommand{\Rep}{\mathrm{Rep}}
\newcommand{\Grpd}{\mathrm{Grpd}}
\newcommand{\sSet}{\mathrm{sSet}}
\newcommand{\Mod}{\mathrm{Mod}}
\newcommand{\SmoothMnf}{\mathrm{SmoothMnf}}
\newcommand{\coker}{\mathrm{coker}}

\newcommand{\Ord}{\mathrm{Ord}}
\newcommand{\eq}{\mathrm{eq}}
\newcommand{\coeq}{\mathrm{coeq}}
\newcommand{\act}{\mathrm{act}}

%%%%%%%%%%%%%%% 定理環境(足助先生ありがとうございます) %%%%%%%%%%%%%%%

\everymath{\displaystyle}
\renewcommand{\proofname}{\bf [証明]}
\renewcommand{\thefootnote}{\dag\arabic{footnote}} %足助さんからもらった.どうなるんだ?
\renewcommand{\qedsymbol}{$\blacksquare$}

\renewcommand{\labelenumi}{(\arabic{enumi})} %(1),(2),...がデフォルトであって欲しい
\renewcommand{\labelenumii}{(\alph{enumii})}
\renewcommand{\labelenumiii}{(\roman{enumiii})}

\newtheoremstyle{StatementsWithStar}% ?name?
{3pt}% ?Space above? 1
{3pt}% ?Space below? 1
{}% ?Body font?
{}% ?Indent amount? 2
{\bfseries}% ?Theorem head font?
{\textbf{.}}% ?Punctuation after theorem head?
{.5em}% ?Space after theorem head? 3
{\textbf{\textup{#1~\thetheorem{}}}{}\,$^{\ast}$\thmnote{(#3)}}% ?Theorem head spec (can be left empty, meaning ‘normal’)?
%
\newtheoremstyle{StatementsWithStar2}% ?name?
{3pt}% ?Space above? 1
{3pt}% ?Space below? 1
{}% ?Body font?
{}% ?Indent amount? 2
{\bfseries}% ?Theorem head font?
{\textbf{.}}% ?Punctuation after theorem head?
{.5em}% ?Space after theorem head? 3
{\textbf{\textup{#1~\thetheorem{}}}{}\,$^{\ast\ast}$\thmnote{(#3)}}% ?Theorem head spec (can be left empty, meaning ‘normal’)?
%
\newtheoremstyle{StatementsWithStar3}% ?name?
{3pt}% ?Space above? 1
{3pt}% ?Space below? 1
{}% ?Body font?
{}% ?Indent amount? 2
{\bfseries}% ?Theorem head font?
{\textbf{.}}% ?Punctuation after theorem head?
{.5em}% ?Space after theorem head? 3
{\textbf{\textup{#1~\thetheorem{}}}{}\,$^{\ast\ast\ast}$\thmnote{(#3)}}% ?Theorem head spec (can be left empty, meaning ‘normal’)?
%
\newtheoremstyle{StatementsWithCCirc}% ?name?
{6pt}% ?Space above? 1
{6pt}% ?Space below? 1
{}% ?Body font?
{}% ?Indent amount? 2
{\bfseries}% ?Theorem head font?
{\textbf{.}}% ?Punctuation after theorem head?
{.5em}% ?Space after theorem head? 3
{\textbf{\textup{#1~\thetheorem{}}}{}\,$^{\circledcirc}$\thmnote{(#3)}}% ?Theorem head spec (can be left empty, meaning ‘normal’)?
%
\theoremstyle{definition}
 \newtheorem{theorem}{定理}[section]
 \newtheorem{axiom}[theorem]{公理}
 \newtheorem{corollary}[theorem]{系}
 \newtheorem{proposition}[theorem]{命題}
 \newtheorem*{proposition*}{命題}
 \newtheorem{lemma}[theorem]{補題}
 \newtheorem*{lemma*}{補題}
 \newtheorem*{theorem*}{定理}
 \newtheorem{definition}[theorem]{定義}
 \newtheorem{example}[theorem]{例}
 \newtheorem{notation}[theorem]{記法}
 \newtheorem*{notation*}{記法}
 \newtheorem{assumption}[theorem]{仮定}
 \newtheorem{question}[theorem]{問}
 \newtheorem{counterexample}[theorem]{反例}
 \newtheorem{reidai}[theorem]{例題}
 \newtheorem{ruidai}[theorem]{類題}
 \newtheorem{problem}[theorem]{問題}
 \newtheorem{algorithm}[theorem]{算譜}
 \newtheorem*{solution*}{\bf{[解]}}
 \newtheorem{discussion}[theorem]{議論}
 \newtheorem{remark}[theorem]{注}
 \newtheorem{remarks}[theorem]{要諦}
 \newtheorem{image}[theorem]{描像}
 \newtheorem{observation}[theorem]{観察}
 \newtheorem{universality}[theorem]{普遍性} %非自明な例外がない.
 \newtheorem{universal tendency}[theorem]{普遍傾向} %例外が有意に少ない.
 \newtheorem{hypothesis}[theorem]{仮説} %実験で説明されていない理論.
 \newtheorem{theory}[theorem]{理論} %実験事実とその(さしあたり)整合的な説明.
 \newtheorem{fact}[theorem]{実験事実}
 \newtheorem{model}[theorem]{模型}
 \newtheorem{explanation}[theorem]{説明} %理論による実験事実の説明
 \newtheorem{anomaly}[theorem]{理論の限界}
 \newtheorem{application}[theorem]{応用例}
 \newtheorem{method}[theorem]{手法} %実験手法など,技術的問題.
 \newtheorem{history}[theorem]{歴史}
 \newtheorem{usage}[theorem]{用語法}
 \newtheorem{research}[theorem]{研究}
 \newtheorem{shishin}[theorem]{指針}
 \newtheorem{yodan}[theorem]{余談}
 \newtheorem{construction}[theorem]{構成}
% \newtheorem*{remarknonum}{注}
 \newtheorem*{definition*}{定義}
 \newtheorem*{remark*}{注}
 \newtheorem*{question*}{問}
 \newtheorem*{problem*}{問題}
 \newtheorem*{axiom*}{公理}
 \newtheorem*{example*}{例}
 \newtheorem*{corollary*}{系}
 \newtheorem*{shishin*}{指針}
 \newtheorem*{yodan*}{余談}
 \newtheorem*{kadai*}{課題}
%
\theoremstyle{StatementsWithStar}
 \newtheorem{definition_*}[theorem]{定義}
 \newtheorem{question_*}[theorem]{問}
 \newtheorem{example_*}[theorem]{例}
 \newtheorem{theorem_*}[theorem]{定理}
 \newtheorem{remark_*}[theorem]{注}
%
\theoremstyle{StatementsWithStar2}
 \newtheorem{definition_**}[theorem]{定義}
 \newtheorem{theorem_**}[theorem]{定理}
 \newtheorem{question_**}[theorem]{問}
 \newtheorem{remark_**}[theorem]{注}
%
\theoremstyle{StatementsWithStar3}
 \newtheorem{remark_***}[theorem]{注}
 \newtheorem{question_***}[theorem]{問}
%
\theoremstyle{StatementsWithCCirc}
 \newtheorem{definition_O}[theorem]{定義}
 \newtheorem{question_O}[theorem]{問}
 \newtheorem{example_O}[theorem]{例}
 \newtheorem{remark_O}[theorem]{注}
%
\theoremstyle{definition}
%
\raggedbottom
\allowdisplaybreaks
\usepackage[math]{anttor}
\begin{document}
\tableofcontents

\chapter{General Topology}

\section{順序集合}

\section{位相}

\section{収束}

\begin{tcolorbox}[colframe=ForestGreen, colback=ForestGreen!10!white,breakable,colbacktitle=ForestGreen!40!white,coltitle=black,fonttitle=\bfseries\sffamily,
title=]
    収束性を議論するにあたって,
    第1可算な空間は数列の概念で十分だが,
    弱位相を備えたHilbert空間などは足りない.
\end{tcolorbox}

\subsection{ネットの収束}

\begin{tcolorbox}[colframe=ForestGreen, colback=ForestGreen!10!white,breakable,colbacktitle=ForestGreen!40!white,coltitle=black,fonttitle=\bfseries\sffamily,
title=]
    Kellyは"A topologist is a person who cannot tell the difference between a doughnut and a coffee cup".と言った人でもある.
\end{tcolorbox}

\begin{definition}[direction, directed set / filtered set, net / generalized sequences, eventuality filter]\mbox{}
    \begin{enumerate}
        \item 空でない集合$D$上の,任意の2元について上界を持つ(upward-filtering / upward-directed)ような前順序$\le$を\textbf{方向}といい,組$(D,\le)$を\textbf{有向集合}または\textbf{フィルター付き集合}という.
        \item 有向集合$D$からの写像$i:D\to X$を\textbf{$X$上のネット}または\textbf{有向系}という.ネットも$(x_n)_{n\in D}$と表し,$x_n=i(n)$とする.
    \end{enumerate}
\end{definition}

\begin{definition}[Kelley 1955]\label{def-subnet-Kelley}
    $(y_\beta)_{\beta\in B}$が$(x_\al)_{\al\in A}$の部分ネットであるとは,次の2条件を満たす写像$f:B\to A$が存在することをいう:
    \begin{enumerate}
        \item $\forall_{\beta\in B}\;x_{f(\beta)}=y_\beta$.
        \item strongly cofinal:$\forall_{\alpha\in A}\;\exists_{\beta\in B}\;\forall_{\beta_1\ge\beta}\;f(\beta_1)\ge\alpha$.
    \end{enumerate}
\end{definition}

\begin{remark}
    WillardもKelley \ref{def-subnet-Kelley}も,(ネット$n:D\to X$同様)$f$に単射性を要求していない点に注意.したがって,$A=\N$としても,通常の部分列の定義よりは一般的である.
\end{remark}

\begin{definition}[eventuality filter / cofinitely often]\mbox{}
    \begin{enumerate}
        \item $\nu:D\to X$を集合$X$上のネットとする.\textbf{ネット$\nu$が定めるフィルター}$F_\nu$とは,
        \[F_\nu:=\Brace{A\in P(X)\mid \exists_{i\in D}\;\forall_{j\ge i}\;\nu_j\in A}\]
        のことである.\footnote{この条件を「$\nu$は結局(eventually) $A$に収まる」と表現する.$D=\N$のとき,$\fe$の条件と同値.}
    \end{enumerate}
\end{definition}

\begin{definition}[convergence of net, limit point, cluster / accumulation point, Cauchy, universal]
    $X$を位相空間,$F$を$S$上のフィルターとする.
    \begin{enumerate}
        \item ネット$n:D\to X$が$x\in X$に\textbf{収束}するとは,任意の$x$の開近傍(従って近傍)$A\in\O(x)$に,$n$が終局する$A\in F_n$ことをいう.このとき,$x$を極限点という.$X$がHausdorffのとき,一意に定まる.
        \item ネット$n:D\to X$が$x\in X$に\textbf{集積}するとは,任意の$x$の開近傍(従って近傍)$A\in\O(x)$に,$n$が無限回入ることをいう:$\forall_{i\in D}\;\exists_{j\ge i}\;n_j\in A$.このとき,$x$を集積点という.実は,任意の集積点は必ず部分ネットの極限点になる.
        \footnote{これは英語でclusterと\url{https://ncatlab.org/nlab/show/filter}に乗っているが,filterについてはclusterとaccumulateは同じ定義になるのだろうか?}
        \item $X$を距離空間とする.ネット$n:D\to X$が\textbf{コーシー}であるとは,終局フィルター$F_n$について,$\forall_{r>0}\;\exists_{A\in F_n}\;\diam(A)<r$を満たすことをいう.
        \item ネット$n:D\to X$が普遍的であるとは,任意の集積点が極限点であることをいう.同値だが,任意の$Y\in P(X)$について,$Y\in F_\nu\lor X\setminus Y\in F_\nu$が成り立つ.
    \end{enumerate}
\end{definition}

\begin{example}
    自明なフィルター(the improper filter)$P(X)$は全ての一点集合を含むので,全ての点に収束し,どの点にも密集しない.
\end{example}

\begin{lemma}[集積点の特徴付け]
    $B$をupward-filtering\footnote{$\cup$について上に閉じていること}な$X$の部分集合系とする.ネット$(x_\lambda)_{\lambda\in\Lambda}$について,次の2条件は同値.
    \begin{enumerate}
        \item $B$の任意の元に無限回入る.
        \item 部分ネット$(x_{h(\mu)})_{\mu\in M}$が存在して,$B$に終局する.
    \end{enumerate}
\end{lemma}

\begin{corollary}
    位相空間$X$上のネットの任意の集積点$x$について,ある部分ネットが存在して$x$に終局する.
\end{corollary}

\begin{theorem}[existence of universal nets (AC)]
    任意のネットは,部分ネットとして普遍ネットを持つ.
\end{theorem}

\subsection{ネットによる完備性の定義}

局所凸空間$(X,\F)$において,Cauchyネットの定義は$\forall_{r>0}\;\forall_{m\in\F}\;\forall_{j,k\in D}\;\exists_{j_0\in D}\;j,k\ge j_0\Rightarrow m(x_j-x_k)<r$に同値.
これが必ず極限を持つことは通常の意味でのノルム空間の完備性と同値.
こうして完備性の概念は一般の位相空間に拡張できる.

また,ネットの考え方により,位相線型空間で無限和が考えられる.
局所凸空間$X$の無限族$(x_i)_{i\in I}$の和は,添字集合$I$の全ての有限部分集合のなす有向集合$D$について,
$x_J:=\sum_{j\in J}x_j\;(J\in D)$が(ただの有限和なので)それぞれ定まり,ネット$(x_J)_{J\in D}$を得る.
これが収束する時,これを$(x_i)_{i\in I}$の無限和とすれば良い.
$X$が完備である時,$\sum_{i\in I}\norm{x_i}<\infty$である場合を除いて,無限和が存在する.

\subsection{ネットによる位相空間論}

\begin{tcolorbox}[colframe=ForestGreen, colback=ForestGreen!10!white,breakable,colbacktitle=ForestGreen!40!white,coltitle=black,fonttitle=\bfseries\sffamily,
title=]
    $P(X)$のうち,どのフィルターが終局フィルターであるか,すなわち,どのネットが収束するかを指定することと,空間に位相を指定することは同値になる.
\end{tcolorbox}

\begin{proposition}[(AC)]
    部分空間$Y\subset X$と点$x\in X$について,次の2条件は同値.
    \begin{enumerate}
        \item $x\in\o{Y}$.
        \item $x$に収束する$Y$のネットが存在する.
    \end{enumerate}
\end{proposition}

\section{連続性}

\subsection{ネットによる議論}

\begin{proposition}[連続性の特徴付け]\label{prop-characterization-of-continuousness}
    $f:X\to Y$を位相空間$(X,\sigma),(Y,\tau)$の射とする.$x\in X$について,次の3条件は同値.
    \begin{enumerate}
        \item $f$は$x$で連続:$\forall_{A\in\O(f(x))}\;f^{-1}(A)\in\O(x)$.
        \item $\forall_{A\in\O(f(x))}\;\exists_{B\in\O(x)}\;f(B)\subset A$.
        \item $x$に収束する任意のネット$(x_\lambda)_{\lambda\in\Lambda}$について,$f(x_\lambda)\to f(x)$.すなわち,$\O(f(x))\subset F_{f(x_\lambda)}$.
    \end{enumerate}
\end{proposition}

\begin{proposition}[位相同型の特徴付け]
    全単射な連続写像$f:X\to Y$について,次の2条件は同値.
    \begin{enumerate}
        \item 逆写像$f^{-1}$も連続である.
        \item $f$は開写像である.
    \end{enumerate}
\end{proposition}

\subsection{誘導される位相}

\begin{tcolorbox}[colframe=ForestGreen, colback=ForestGreen!10!white,breakable,colbacktitle=ForestGreen!40!white,coltitle=black,fonttitle=\bfseries\sffamily,
title=]
    無限積に入る積位相は,代数的な有限項制限が入るから,直感よりはよっぽど弱い位相になる.
\end{tcolorbox}

\begin{definition}[initial topology]
    集合$X$と写像の族$(f:X\to Y_f)_{f\in\F}$を考える.$Y_f$の位相$\tau_f$に対して,$X$に$\F$を連続にする最弱の位相が定まる.
    すなわち,$\Brace{f^{-1}(A)\in P(X)\mid A\in\tau_f,f\in\F}$が生成する位相である.
\end{definition}

\begin{proposition}\label{prop-net-in-initial-topology}
    集合$X$に写像の族$\F$が定める始位相を入れる.
    このとき,次の2条件は同値.
    \begin{enumerate}
        \item ネット$(x_\lambda)_{\lambda\in\Lambda}$は$x$に収束する.
        \item 任意の$f\in\F$について,ネット$(f(x_\lambda))_{\lambda\in\Lambda}$は$f(x)$に収束する.
    \end{enumerate}
\end{proposition}

\begin{corollary}\label{cor-continuous-function-to-initial-topology}
    集合$X$に写像の族$\F$が定める始位相を入れる.このとき,位相空間$Z$からの写像$g:Z\to X$について,次の2条件は同値.
    \begin{enumerate}
        \item $g$は連続.
        \item 任意の$f\in\F$について,$f\circ g:Z\to Y_f$は連続.
    \end{enumerate}
\end{corollary}

\begin{definition}[final topology]
    集合$Y$への写像の族$(f:X_f\to Y)_{f\in\F}$を考える.$Y$には$\F$を連続にする最強の位相が考えられる.
    すなわち,$\Brace{A\in P(Y)\mid\forall_{f\in\F}\;f^{-1}(A)\in\tau_f}$が生成する位相である.
\end{definition}

\begin{proposition}
    集合$Y$に写像の族$\F$が定める終位相を入れる.このとき,位相空間$Z$への写像$g:Y\to Z$について,次の2条件は同値.
    \begin{enumerate}
        \item $g$は連続.
        \item 任意の$f\in\F$について,$g\circ f:X_f\to Z$は連続.
    \end{enumerate}
\end{proposition}

\begin{corollary}\label{cor-for-quotient-space}
    $Q:X\to X/\sim$が定める商位相について,
    \begin{enumerate}
        \item 任意の同値類が$X$-閉であることと,対応する$X/\sim$の点が閉であることは同値.
        \item $Q$が開写像であることと,任意の開集合$A\osub X$の飽和$f^{-1}(f(A))$が開であること,すなわち,$\wt{A}:=\Brace{x\in X\mid x\sim y,y\in A}$が開であることに同値.
    \end{enumerate}
\end{corollary}

\section{分離}

\begin{definition}[separating family]
    $X$上の写像の族$\F$が$X$の点を分離するとは,$\forall_{x,y\in X}\;x\ne y\Rightarrow[\exists_{f\in\F}\;f(x)\ne f(y)]$を満たすことをいう.
\end{definition}

\subsection{Hausdorff性}

\begin{proposition}
    位相空間$(X,\tau)$について,
    \begin{enumerate}
        \item $X$はHausdorffである.
        \item $X$の任意のネットの極限点は高々1つである.
    \end{enumerate}
\end{proposition}

\begin{proposition}\label{prop-for-initial-topology-begin-Hausdorff}
    $X$上に,$X$の点を分離する写像の族$\F$が定める始位相を考える.このとき,$Y_f$が全てHausdorffであるなら,$X$もHausdorffである.
\end{proposition}

\begin{corollary}\label{cor-product-of-Hausdorff-spaces}
    Hausdorff空間の積はHausdorffである.
\end{corollary}

\subsection{正規空間とUrysohnの補題}

\begin{tcolorbox}[colframe=ForestGreen, colback=ForestGreen!10!white,breakable,colbacktitle=ForestGreen!40!white,coltitle=black,fonttitle=\bfseries\sffamily,
title=]
    To ascertaion the existence of an ample supply of continuous real functions on the space, we need a more severe separation condition.
    $T_4$くらい位相が強ければ,隆起関数のような,真に変化のある連続関数の存在が保証される.
\end{tcolorbox}

\begin{definition}[normal]
    Hausdorff空間$X$が\textbf{正規}であるとは,任意の互いに素な閉集合$E,F$が,互いに素な開集合$A,B$で分離できることをいう:$E\subset A,F\subset B$.
    Hausdorff性を課さないこともある.
\end{definition}

\begin{proposition}
    次の2条件は同値.
    \begin{enumerate}
        \item $X$は正規である.
        \item 任意の$F\subset B$を満たす閉集合$F$と開集合$B$に対して,開集合$A$であって$F\subset A\subset\o{A}\subset B$を満たすものが存在する.
    \end{enumerate}
\end{proposition}
\begin{proof}
    $B=X\setminus E$として閉集合$E$と$F$の分離を考えれば良い.
\end{proof}

\begin{theorem}[Urysohn's lemma]
    $(X,\tau)$を正規空間とする.互いに素な閉集合の組$E,F$に対して,連続関数$f:X\to [0,1]$であって,$f(E)=\{0\},f(F)=\{1\}$を満たすものが存在する.
\end{theorem}
\begin{remarks}
    距離空間は正規であるが,この場合は$f(x):=\frac{d(E,x)}{d(E,x)+d(F,x)}$を考えれば良い.
\end{remarks}

\begin{proposition}[Tietze's theorem]
    $(X,\tau)$を正規空間とする.任意の閉集合$F$上の任意の有界連続関数$f:F\to\R$は,$X$上に有界連続に延長する.
\end{proposition}
\begin{remarks}
    $\R$は絶対引き戻し(absolute retract / extensor)であることを主張している.
    $K$が絶対引き戻しであるとは,$X$が正規で$A\overset{\mathrm{closed}}{\subset}X$を閉部分集合,$f:A\to K$をその上の連続関数としたとき,必ず連続延長$\o{f}:X\to K$が存在することをいう.
\end{remarks}

\subsection{完備距離空間の内部hom}

\begin{proposition}
    $Y$を完備距離空間とする.位相空間$X$からの連続関数の空間$C(X,Y)$は,一様距離
    \[d_\infty(f,g):=\sup\Brace{d(f(x),g(x))\land 1\in\R\mid x\in X}\]
    について完備距離空間となる.
\end{proposition}

\subsection{下半連続性}

\begin{tcolorbox}[colframe=ForestGreen, colback=ForestGreen!10!white,breakable,colbacktitle=ForestGreen!40!white,coltitle=black,fonttitle=\bfseries\sffamily,
title=]
    下半連続な関数全体の集合$C^{1/2}(X)$は,位相$\tau$の情報を湛えている.
\end{tcolorbox}

\begin{definition}
    写像$f:X\to\R$について,
    \begin{enumerate}
        \item $f$が下半連続であるとは,$\forall_{t\in\R}\;f^{-1}((t,\infty))\in\tau$を満たすことをいう.
        \item $f$が上半連続であるとは,$\forall_{t\in\R}\;f^{-1}((-\infty,t))\in\tau$を満たすことをいう.
        \item $X$上の下半連続な関数全体の集合を$C^{1/2}(X)$で表す.
    \end{enumerate}
\end{definition}

\begin{lemma}[位相の特徴付け]
    位相空間$(X,\tau)$について,次の2条件は同値.
    \begin{enumerate}
        \item $\chi_A\in C^{1/2}(X)$.
        \item $A\in\tau$.
    \end{enumerate}
\end{lemma}

\begin{proposition}[下半連続性の特徴付け]\label{prop-characterization-of-lsc}
    関数$f:X\to\R$について,次の2条件は同値.
    \begin{enumerate}
        \item $f$は下半連続.
        \item 極限点を持つネット$(x_\lambda)_{\lambda\in\Lambda}$について,$f(\lim x_\lambda)\le\liminf f(x_\lambda)$が成り立つ.
    \end{enumerate}
\end{proposition}

\begin{proposition}[下半連続関数の空間は凸錐である]\mbox{}\label{prop-subalgebra-of-lsc-function}
    \begin{enumerate}
        \item $C^{1/2}(X)$の任意個の上限は再び$C^{1/2}(X)$の元.
        \item $C^{1/2}(X)$の有限個の下限は再び$C^{1/2}(X)$の元.
        \item $C^{1/2}(X)$は和と積について閉じている.
        \item $C^{1/2}(X)$は一様収束について閉じている.
    \end{enumerate}
\end{proposition}

\begin{proposition}[下半連続関数の描像]\label{prop-lower-semicontinuous-functions}
    位相空間$X$は,連続関数の族$C(X,[0,1])$が点と閉集合を分離するとする($X$が正規であるときこれを満たす).
    任意の下半連続関数$f:X\to\R_+$はある連続関数の族の上限である.
\end{proposition}

\section{コンパクト性}

\subsection{ネットによる議論}

\begin{theorem}[コンパクト性の特徴付け]\label{thm-characterization-of-compactness}
    位相空間$(X,\tau)$について,次の5条件は同値.
    \begin{enumerate}
        \item 任意の開被覆は有限な部分被覆を持つ.
        \item 閉集合系$\Delta$はどの有限交叉も空で無いとすると,$\Delta$の共通部分は非空である.
        \item $X$上のネットは集積点を持つ.
        \item $X$上の普遍ネットは収束する.
        \item $X$上のネットは収束する部分ネットを持つ.
    \end{enumerate}
\end{theorem}
\begin{remark}
    Jacobson位相を備えた環のprimitive ideal spaceはコンパクトだがHausdorffでない.
\end{remark}

\begin{lemma}[コンパクト集合の分離]
    $C$をHausdorff空間$X$のコンパクト集合とする.任意の$x\in X\setminus C$について,互いに素な部分集合$A,B$が存在して$C\subset A,x\in B$を満たす.
\end{lemma}

\begin{proposition}
    Hausdorff空間のコンパクト集合は閉集合である.
\end{proposition}

\subsection{コンパクトハウスドルフ空間}

\begin{tcolorbox}[colframe=ForestGreen, colback=ForestGreen!10!white,breakable,colbacktitle=ForestGreen!40!white,coltitle=black,fonttitle=\bfseries\sffamily,
title=]
    コンパクトハウスドルフ空間はrigidである:これより弱い位相はHausdorffではなくなる.
    実際,コンパクトハウスドルフ空間$(X,\tau)$からそれより弱い位相$\sigma$を入れた$(X,\sigma)$への恒等写像$\id_X:X\to X$は,$\sigma$がHausdorffなら同型を定める.
\end{tcolorbox}

\begin{proposition}
    任意のコンパクトHausdorff空間は正規である.
\end{proposition}

\subsection{古典的結果}

\begin{tcolorbox}[colframe=ForestGreen, colback=ForestGreen!10!white,breakable,colbacktitle=ForestGreen!40!white,coltitle=black,fonttitle=\bfseries\sffamily,
title=]
    KelleyはTychonoffの定理と選択公理が同値であることを示した.
\end{tcolorbox}

\begin{theorem}[Tychonoff]
    コンパクト空間の族$(X_i)$に対して,積空間$\prod_{i\in I}X_i$はコンパクトである.
\end{theorem}
\begin{proof}
    $X:=\prod_{i\in I}X_i$の任意の普遍ネット$(x_\lambda)$を取る.
    射影$\pr_i:X\epi X_i$は連続だから,ネット$(\pr_i(x_\lambda))$も普遍的である(任意の$Y\subset X_i$について$Y$または$X_i\setminus Y$に収束するという条件は保たれる\ref{prop-characterization-of-continuousness}).
    $X_i$はコンパクトだから,これは極限点$x_i\in X_i$を持つ.全射性より,$x\in\pr_i^{-1}(x_i)$が取れる.
    このとき,$\forall_{i\in I}\;\pr_i(x)=x_i$を満たすように取れる.
    始位相の性質\ref{prop-net-in-initial-topology}より,$\forall_{i\in I}\;\pi_i(x_\lambda)\to\pi_i(x_i)$は$x_\lambda\to x$に同値.よって,$X$もコンパクト(コンパクト性の特徴付け\ref{thm-characterization-of-compactness}).
\end{proof}

\begin{theorem}[covering theorem of E. Borel]
    $\R^n$の有界閉集合はコンパクトである.
\end{theorem}

\begin{definition}[Tychonoff cube]
    積空間$T:=[0,1]^\N$をTychonoff cubeという.Tychonoffの定理と\ref{cor-product-of-Hausdorff-spaces}より,これはコンパクトハウスドルフ空間である.
\end{definition}

\begin{lemma}
    $d(x,y):=\sum\frac{\abs{x_n-y_n}}{2^n}$は$T$に距離を定める.
\end{lemma}

\begin{theorem}[Urysohnの距離付け定理]\label{thm-Urysohn-metrization}
    第2可算な正規空間は,Tychonoff cubeに埋め込める.特に,距離化可能である.
\end{theorem}

\section{局所コンパクト性}

\begin{tcolorbox}[colframe=ForestGreen, colback=ForestGreen!10!white,breakable,colbacktitle=ForestGreen!40!white,coltitle=black,fonttitle=\bfseries\sffamily,
title=]
    Euclid空間,離散空間,そのほか多くの解析的・幾何学的対象は局所コンパクトである.
    コンパクト化の技法が重要になる.
\end{tcolorbox}

\begin{definition}
    位相空間$X$が\textbf{局所コンパクト}であるとは,任意の点がコンパクトな近傍を持つことをいう.これは,$X$が相対コンパクトな集合からなる開基を持つことに同値.
\end{definition}

\subsection{一点コンパクト化}

\begin{definition}[compactification]
    位相空間$(X,\tau)$について,
    \begin{enumerate}
        \item コンパクト化とは,コンパクト空間$(\wt{X},\wt{\tau})$とその稠密な部分集合への同型$i:X\mono\wt{X}$との組$(\wt{X},\wt{\tau},i)$をいう.
        \item 特に$\wt{X}\setminus X$が一点集合となる場合を一点コンパクト化という.
    \end{enumerate}
\end{definition}
\begin{remarks}
    Hausdorff空間のコンパクト化はHausdorffとは限らない.
    が,completely regular空間はHausdorffなコンパクト化を許す.
    なお,正規空間はcompletely regularである.
\end{remarks}

\begin{proposition}
    任意のコンパクトでない位相空間$(X,\tau)$について,
    \begin{enumerate}
        \item 一点コンパクト化$(\wt{X},\wt{\tau})$が存在する.
        \item $\wt{X}$がHausdorffであることと,$X$が局所コンパクトなHausdorff空間であることは同値.
    \end{enumerate}
\end{proposition}

\subsection{局所コンパクトハウスドルフ空間}

\begin{tcolorbox}[colframe=ForestGreen, colback=ForestGreen!10!white,breakable,colbacktitle=ForestGreen!40!white,coltitle=black,fonttitle=\bfseries\sffamily,
title=]
    局所コンパクトハウスドルフ空間はほぼ多様体であることに注意.隆起関数はこのクラスに対して定義される.
    関数族$C_c(X)$は$X$のコンパクト集合と閉集合を分離するには十分大きいクラスである.
    $\o{C_c(X)}=C_0(X)$である.
\end{tcolorbox}

\begin{proposition}
    局所コンパクトハウスドルフ空間の任意の開集合と閉集合は,相対位相に関して再び局所コンパクトハウスドルフである.
\end{proposition}

\begin{proposition}[隆起関数の存在]
    局所コンパクトハウスドルフ空間$(X,\tau)$の任意のコンパクト集合$C$とこれを含む開集合$C\subset A\osub X$について,$A$に含まれるコンパクト台を持つ連続関数$f:X\to[0,1]$が存在して,$f(C)=\{1\}$を満たす.
\end{proposition}

\begin{definition}[functions vanish at infinity]\label{def-functions-vanish-at-infinity}
    連続関数$f:X\to\R$が,任意の$\ep>0$に対して$\Brace{x\in X\mid\abs{f(x)}\ge 0}$がコンパクトであることは,$X$の一点コンパクト化への連続延長が$\wt{f}(\infty)=0$を満たすことに同値.
    これを,無限遠点で消える関数という.
    そのような関数全体からなる集合を$C_0(X)$で表す.
\end{definition}


\subsection{パラコンパクト性}

\begin{tcolorbox}[colframe=ForestGreen, colback=ForestGreen!10!white,breakable,colbacktitle=ForestGreen!40!white,coltitle=black,fonttitle=\bfseries\sffamily,
title=]
    局所コンパクトハウスドルフ空間は正規とは限らない.正規になるためには,$\sigma$-コンパクト性,または局所コンパクト性の下では同値だが,パラコンパクト性が必要である.
    多様体論も,1の分割を使うために,正規性を達成する対象のみを多様体とした.
\end{tcolorbox}


\begin{lemma}\label{lemma-characterization-of-paracompactness}
    局所コンパクトで連結な空間$X$について,次の3条件は同値.
    \begin{enumerate}
        \item パラコンパクトである.
        \item $\sigma$-コンパクトである.
        \item $\cup E_n=X$を満たす相対コンパクトな開集合の増大列$(E_n)$で,$\o{E_n}\subset E_{n+1}$を満たすものが取れる.
    \end{enumerate}
\end{lemma}

\begin{proposition}[正規性の十分条件]\mbox{}
    \begin{enumerate}
        \item パラコンパクトなHausdorff空間は正規である.
        \item $\sigma$-コンパクトは局所コンパクトHausdorff空間は正規である.
    \end{enumerate}
\end{proposition}

\begin{corollary}
    任意の第2可算な局所コンパクトHausdorff空間は距離化可能である.
\end{corollary}
\begin{proof}
    Urysohnの距離付け定理\ref{thm-Urysohn-metrization}より.
\end{proof}

\begin{proposition}[1の分割]
    正規空間$X$の局所有限な開被覆$(A_n)$について,
    これに従属する1の分割が存在する.
\end{proposition}
\begin{remark}
    Zornの補題に依れば,任意の局所有限な開被覆$(A_n)$についても取れる.
\end{remark}

\begin{proposition}\mbox{}
    \begin{enumerate}
        \item 位相多様体について,第2可算であることと,パラコンパクトかつ連結部分が可算個であることは同値.
        \item 第2可算ならば,Lindelöfである.すなわち,可算被覆性を持つ.
        \item 局所コンパクトで$\sigma$-コンパクトはHausdorff空間は可算被覆性を持つ.
    \end{enumerate}
\end{proposition}

\chapter{Banach Space}

\begin{quotation}
    Banach空間とは,CompMetの中の線型空間である.
    Banach空間の射は,有界な線型写像と,short linear mapとの2つの取り方がある.
    nLabでは前者をBant,後者をBanと表すようだ.

    位相線型空間とは,連続群のように,Topの中の線型空間である.
\end{quotation}

\begin{notation}\mbox{}
    \begin{enumerate}
        \item $B(0,r)$を閉球とする.$B$を閉単位球とする.$rB:=\Brace{rx\mid x\in B}$と表す.
        \item 具体的に$\bF=\R,\C$とするが,ほとんどの結果は離散的でない付値体一般,または局所体(local field),すなわち,局所コンパクトHausdorffな体一般に拡張できる.
        \item ノルム位相についての閉包を$\dbloverline{Y}$で表す.
    \end{enumerate}
\end{notation}

\section{ノルム空間}

\begin{tcolorbox}[colframe=ForestGreen, colback=ForestGreen!10!white,breakable,colbacktitle=ForestGreen!40!white,coltitle=black,fonttitle=\bfseries\sffamily,
title=]
    連続と線型が出会うと有界と変化すること,内部homを持つことなど,とにかく美しい描像がある.
\end{tcolorbox}

\subsection{ノルム空間と射}

\begin{tcolorbox}[colframe=ForestGreen, colback=ForestGreen!10!white,breakable,colbacktitle=ForestGreen!40!white,coltitle=black,fonttitle=\bfseries\sffamily,
title=]
    ノルム空間が定める位相($(\lambda B+x)_{\lambda>0,x\in X}$が生成する位相)は,ノルムを連続にする最弱位相になる.
    この位相について完備なノルム空間をBanach空間という.
    するとノルム空間の間の線型写像が連続であることと有界であることは同値で,これが射となる.
    なお,作用素論では,作用素の語は部分関数,線型写像の語は写像として,定義を使い分ける.
    作用素$x:X\to Y$の定義域を$D(x)\subset X$と表す,という具合である.
\end{tcolorbox}

\begin{definition}[valuation, valued field, discrete valuation]
    体$K$と全順序Abel群$G$とについて,\textbf{$G$-値付値}とは,関数$v:K\to G\cup\{\infty\}$であって,次を満たすものをいう:
    \begin{enumerate}
        \item 制限$v|_{K^\times}$の値域は$G$に含まれ,群準同型$K^\times\to G$を定める.
        \item $v(0)=\infty$.
        \item $v(x+y)\ge\min(v(x),v(y))$.
    \end{enumerate}
    組$(K,v)$を\textbf{付値体}という(暗黙に$v$が定める距離が定める位相を入れる).$G=\Z$である時,$v$を\textbf{離散付値}という.
\end{definition}

\begin{definition}[norm, seminorm, equivalence, Banach space]
    $X$を$k$-線型空間とし,体$k$には絶対値$\abs{\;}:X\to\R_{\ge 0}$が備わっているとする.
    \begin{enumerate}
        \item 実数値関数$\norm{\;}:X\to\R$が\textbf{ノルム}であるとは,次の3条件を満たすことをいう.
        \begin{enumerate}[(a)]
            \item (positivity / faithfulness) $\forall_{u\in X}\;\norm{u}=0\Rightarrow u=0$.\footnote{これは通常$\forall_{u\in X}\;\norm{u}\ge 0$と等号成立条件が$u=0$と分けて書かれる.この主張だけで十分である理由は,(2)より$\norm{-u}=\norm{u}$であり,(3),(1)より$\norm{0}\le 2\norm{u}$が従うので,非負値であることが3条件から従う.}
            \item (linearity / homogeneity) $\forall_{\alpha\in k}\;\forall_{u\in X}\;\norm{\alpha u}=\abs{\alpha}\norm{u}$.
            \item (triangle inequality / subadditivity) $\forall_{u,v\in X}\;\norm{u+v}\le\norm{u}+\norm{v}$.
        \end{enumerate}
        \item 条件(1)が成り立たない場合,\textbf{セミノルム}または\textbf{半ノルム}という.
        \item ノルムが同値であるとは,$\exists_{C_1,C_2\in\R_{>0}}\;\forall_{u\in X}\;C_1\norm{u}_1\le\norm{u}_2\le C_2\norm{u}_1$.これはノルムが定める距離が同値であることに同値.\footnote{これは,$\id$についてのLipschitz連続性の条件と見れば良い.}
        よって,ノルムが生成する位相が同相であることに同値.\footnote{距離空間の定める位相については,開球が近傍のフィルターの基底となるから,位相については簡単である.}
        \item ノルム空間が,ノルムが定める距離について完備であるとき,\textbf{Banach空間}という.\footnote{局所凸空間$(X,\F)$において,Cauchyネットの定義は$\forall_{r>0}\;\forall_{m\in\F}\;\forall_{j,k\in D}\;\exists_{j_0\in D}\;j,k\ge j_0\Rightarrow m(x_j-x_k)<r$に同値.これが必ず極限を持つことは通常の意味でのノルム空間の完備性と同値.}
        \item ノルム空間の間の線型写像を(線型)\textbf{作用素}という.
    \end{enumerate}
\end{definition}
\begin{remark}
    $\norm{x}-\norm{y}\le\norm{x-y}$より,
    $\abs{\norm{x}-\norm{y}}\le\norm{x-y}$であるため,ノルムは(Lipschitz)連続である.
\end{remark}

\begin{proposition}[有界性:連続性の特徴付け]
    $T:X\to Y$を線型写像とする.次の3条件は同値.
    \begin{enumerate}
        \item $T$は連続である.
        \item ある$x\in X$において$T$は連続である.
        \item $T$は\textbf{有界}である:$\exists_{\al\ge 0}\;\forall_{x\in X}\;\norm{Tx}\le\al\norm{x}$.これは作用素ノルムが有限であることを意味する.
    \end{enumerate}
\end{proposition}
\begin{proof}\mbox{}
    \begin{description}
        \item[(1)$\Rightarrow$(2)] 自明.
        \item[(2)$\Rightarrow$(3)] $x\in X$で連続であるから,$\exists_{\delta>0}\;\forall_{y\in X}\;\norm{x-y}<\delta\Rightarrow\norm{Tx-Ty}\le 1$.
        いま,$\forall_{z\in X\setminus 0}\;\Norm{\paren{\delta\frac{z}{\norm{z}}+x}-x}=\delta\le\delta$より,
        \[\Norm{T\delta\frac{z}{\norm{z}}+Tx-Tx}\le 1\Lrarrow\norm{Tz}\le\delta^{-1}\norm{z}.\]
        \item[(3)$\Rightarrow$(1)]
        $\forall_{x,y\in X}\;\norm{Ty-Tx}=\norm{T(y-x)}\le\al\norm{y-x}$より.
    \end{description}
\end{proof}
\begin{remark}
    なおさらに,
    \begin{enumerate}
        \item $f$は一様連続.
        \item $f$はLipschitz連続.
        \item $\norm{f}$が有限.
    \end{enumerate}
    も同値になる.\footnote{\url{https://ncatlab.org/nlab/show/Banach+space}}
\end{remark}

\subsection{作用素の空間}

\begin{tcolorbox}[colframe=ForestGreen, colback=ForestGreen!10!white,breakable,colbacktitle=ForestGreen!40!white,coltitle=black,fonttitle=\bfseries\sffamily,
title=Banには内部homを持つ閉圏としての構造がある]
    Banach空間の射の集合とは有界線型作用素の空間としたから,「閉単位球をどれくらい飛ばすか」という作用素ノルムが定まり,これについてBanach空間となる.
    これは射集合であるからもちろん代数の構造をもち(乗法は合成に一致),Banach代数と呼ばれる.
    Banは$\R$を単位として閉圏をなす.\footnote{\url{https://ncatlab.org/nlab/show/Banach+space}}
\end{tcolorbox}

\begin{definition}[operator norm]
    ノルム空間$X,Y$の間の有界作用素の集合を$B(X,Y)$で表す.
    これは作用素ノルム\[\norm{T}:=\sup\Brace{\norm{Tx}\in\R_{\ge 0}\mid x\in X,\norm{x}\le 1}\]によって再びノルム空間になる.
    作用素の合成を乗法として,\footnote{合成$S\circ T$を乗法とするのは行列を念頭におくと違和感がない.}
    劣乗法性$\norm{ST}\le\norm{S}\norm{T}$が成り立つ\footnote{$\norm{T}$が1以上か1以下かで場合分けすれば良い}.
    この劣乗法性により,乗法はノルムの定める位相について(両側)連続になるので,$B(X)$はノルム空間で,かつ,結合的多元環の構造も持つ.
    これをノルム代数という.
\end{definition}

\begin{definition}[normed algebra, Banach algebra]\mbox{}
    \begin{enumerate}
        \item \textbf{ノルム代数}または\textbf{ノルム環}とは,連続な双線型写像$\cdot :A\times A\to A$によって結合代数の構造も持つノルム空間$A$のことをいう.\footnote{積が連続とは,$\exists_{C\in\R_{>0}}\;\norm{ST}\le C\norm{S}\norm{T}$を含意する.}
        \item ノルム空間としての$A$が完備である場合,これを\textbf{Banach代数}という.
    \end{enumerate}
\end{definition}

\begin{proposition}\label{prop-internal-hom}
    $X,Y$をノルム空間,$Y$を完備とする.このとき,$B(X,Y)$はBanach空間である.
    特に$B(X)$はBanach代数である.
\end{proposition}
\begin{proof}
    $B(X,Y)$の任意のCauchy列$(T_n)$が収束することを示せば良い.
    任意の$x\in X$について,$(T_nx)$も$Y$のCauchy列であり,$Y$は完備であるから,極限$\lim_{n\to\infty}T_nx$が定まる.これを$Tx$として,対応$T:X\to Y$を定め,これが$T\in B(X,Y)$で$\lim_{n\to\infty}T_n=T$であることを示す.
    \begin{description}
        \item[線形性] $T(x+y)=\lim_{n\to\infty}T_n(x+y)=\lim_{n\to\infty}(T_nx+T_ny)=Tx+Ty$.
        \item[有界性と$T_n\to\ T$] 
        任意の$n\in\N$と$x\in X$について,
        \begin{align*}
            \norm{Tx-T_nx}&=\lim_{m\to\infty}\norm{T_mx-T_nx}=\lim_{m\to\infty}\Norm{(T_m-T_n)\frac{x}{\norm{x}}\norm{x}}&ノルムの連続性\\
            &\le\limsup_{m\to\infty}\norm{T_m-T_n}\cdot\norm{x}
        \end{align*}
        が成り立つ.
        $(T_n)$はCauchy列としたから,$n\to\infty$のとき,$T_nx\xrightarrow{n\to\infty}Tx$.また,$n$を十分大きくとれば,$T$が有界であることもわかる.
    \end{description}
\end{proof}

\subsection{完備性の特徴づけ}

\begin{tcolorbox}[colframe=ForestGreen, colback=ForestGreen!10!white,breakable,colbacktitle=ForestGreen!40!white,coltitle=black,fonttitle=\bfseries\sffamily,
title=]
    これまではノルム空間論であった.線形写像が連続であることは有界性と同値だが,線形空間が完備であることは何を引き起こすか?
    実は,絶対収束級数というものの見方は,Cauchy列と表裏一体である.

    自己射の不動点は,作用している群に関する不変量の概念の卵である.
\end{tcolorbox}

\begin{theorem}[絶対収束級数なるクラスの定義]
    ノルム空間$X$と任意の列$(x_n)$について,次の2条件は同値.
    \begin{enumerate}
        \item $(x_n)$が定める級数が有界$\sum^\infty_{n=1}\norm{x_n}<\infty$ならば収束する.
        \item $X$は完備である.
    \end{enumerate}
\end{theorem}
\begin{proof}\mbox{}
    \begin{description}
        \item[(2)$\Rightarrow$(1)] 絶対収束級数はCauchy列を定めるため.
        \item[(1)$\Rightarrow$(2)] Cauchy列$(x_n)$を任意に取る.
        $\norm{x'_{n+1}-x'_n}\le\frac{1}{2^n}$を満たす部分列$(x'_n)$が取れる.
        これについて,
        \[\sum_{n=1}^\infty\norm{x'_n}\le\norm{x'_1}+\sum_{n=1}^\infty\norm{x'_{n+1}-x'_n}<\infty\]
        より,部分列$(x'_n)$は収束する.よって,元のCauchy列$(x_n)$も収束する.
    \end{description}
\end{proof}

\begin{theorem}
    $X$をBanach空間,$Y$をその閉部分空間とする.自己写像$f:Y\to Y$が縮小写像ならば($1$より小さいLipschitz定数を持つならば),$Y$内に不動点が一意的に存在する.
\end{theorem}

\subsection{Banach空間の圏とテンソル積}

\begin{tcolorbox}[colframe=ForestGreen, colback=ForestGreen!10!white,breakable,colbacktitle=ForestGreen!40!white,coltitle=black,fonttitle=\bfseries\sffamily,
title=]
    距離空間の圏と同様,Banach空間の圏の射には選択の余地がある.
    Banach空間の同型は通常(位相線型空間の延長と見て)可逆な有界線型写像とするから,
    射を有界線型写像として得る圏をBantと書き,"isomorphic category"という.
    一方で,Banach空間を特に距離的に見て,同型を線型な等長同型とするとき,射はshort mapであり,得る圏をBanと書く.
    これを"isometric category"という.

    Banach空間の台となる線型空間のテンソル積にノルムを入れる方法はいくつかある.
    projectiveとinjectiveの2つが代表的である.
    一般に代数は,一般のモノイダル圏について定義される.\footnote{\url{http://nlab-pages.s3.us-east-2.amazonaws.com/nlab/show/associative+unital+algebra}}
    このことが示唆するように,我々の数学的対称についても,テンソル構造が肝要となる.
\end{tcolorbox}

\subsubsection{圏Ban}

\begin{definition}[projective tensor product of Banach space]
    2つのBanach空間$X,Y$について,$X\times Y$上で自由生成された線型空間$F(X\times Y)$上のノルム
    \[\Norm{\sum_{1\le i\le n}a_i(x_i\otimes y_i)}=\sum_{1\le i\le n}\abs{a_i}\norm{x_i}\cdot\norm{y_i}\]
    に関する完備化$\o{F}(X\times Y)$を,双線型関係が生成する部分空間の閉包で割った商空間を$X\otimes_\Ban Y$で表す.
\end{definition}
\begin{remarks}
    これは,線型空間としてのテンソル積$V\otimes W$上にノルム
    \[\Norm{\sum_i\al_iv_iw_i}_\pi:=\inf\Brace{\sum_i\abs{\al_i}\norm{v_i}_V\norm{w_i}_W\in\R_{\ge 0}\;\middle|\; x=\sum_i\al_iv_iw_i}.\]
    を入れて完備化して得たもの$V\hat{\otimes}_\pi W$とみなせる.これをprojective cross normという.
\end{remarks}

\begin{lemma}
    圏Banは射影テンソル積について対称なモノイダル閉圏となる.すなわち,テンソル積を備え,通常のinternal homについて閉じている.極めてCCCに近い振る舞いの良い圏である!
    すると,Banach代数とは,Banにおけるモノイド対象である(半群対象として定義して考察する手法も大事になってくる).
\end{lemma}
\begin{remark}
    したがって圏論者はこちらの圏を重視し,Bantは包含関手$\Ban\mono\TVS$の像が生成する充満部分圏であると見る.
    実際,BantはBanの構造のうち距離の構造を忘れている.
\end{remark}

\begin{lemma}
    圏BanでのBanach空間の「台集合」は,単位閉球となる:
    $\Hom_\Ban(\R,X)\simeq_\Set B=\Brace{x\in X\mid\norm{x}\le 1}$.
\end{lemma}

\subsubsection{その他のテンソル積}

\begin{definition}[injective tensor product]
    $\lambda,\mu$をそれぞれ$V,W$上の線型汎関数とする.
    テンソル積$V\otimes W$をノルム
    \[\norm{x}_\ep:=\sup\Brace{\abs{(\lambda\otimes\mu)(x)}\in\R_{\ge 0}\;\middle|\;\norm{\lambda}_{V^*},\norm{\mu}_{W^*}\le 1}\]
    について完備化したもの$V\hat{\otimes}_\ep W$を\textbf{入射的テンソル積}という.
\end{definition}

\begin{definition}[tensor product of Hilbert space]
    $V,W$をHilbert空間とする.
    $\brac{v_1w_1,v_2w_2}:=\brac{v_1,v_2}\brac{w_1,w_2}$によって定まる内積が定めるノルム$\norm{x}_\sigma$に関する内積空間$V\otimes W$の完備化を,テンソル積$V\hat{\otimes}_\sigma W$という.
\end{definition}

\subsubsection{クロスノルム}

\begin{tcolorbox}[colframe=ForestGreen, colback=ForestGreen!10!white,breakable,colbacktitle=ForestGreen!40!white,coltitle=black,fonttitle=\bfseries\sffamily,
title=]
    見てきたように,テンソル積の空間には積によってノルムを入れるのがreasonableに思える.
\end{tcolorbox}

\begin{definition}
    $V\otimes W$上のクロスノルム$\chi$とは,次の2条件を満たすものを言う:
    \begin{enumerate}
        \item $\forall_{v\in V,w\in W}\;\norm{v\otimes w}_\chi=\norm{v}_V\norm{w}_W$.
        \item $\forall_{\lambda\in V^*,\mu\in W^*}\;\norm{\lambda\otimes\mu}_{\chi^*}=\norm{\lambda}_{V^*}\norm{\mu}_{W^*}$.
    \end{enumerate}
\end{definition}

\subsubsection{Metについて}

\begin{tcolorbox}[colframe=ForestGreen, colback=ForestGreen!10!white,breakable,colbacktitle=ForestGreen!40!white,coltitle=black,fonttitle=\bfseries\sffamily,
title=]
    射は計量写像で,これは特にLipshitz連続,特に一様連続である.
    モノ射は単射な計量写像(short map),エピ射は像が稠密な計量写像である.
    等長写像は非縮小な計量写像で必然的にモノ射である.
    等長写像はCauchy列をCauchy列に写すから,完備性を保つ.
    全写な等長写像が同型である.
    Metはbalancedではない,$\Q\mono\R$が反例である.\footnote{\url{https://ja.wikipedia.org/wiki/距離空間の圏}}
\end{tcolorbox}

\subsection{ノルム空間の商}

\begin{tcolorbox}[colframe=ForestGreen, colback=ForestGreen!10!white,breakable,colbacktitle=ForestGreen!40!white,coltitle=black,fonttitle=\bfseries\sffamily,
title=]
    商空間では,新しい原点$Y$への最短距離をノルムとする.
    完備性の遺伝については,環論で見た完全列を通じた議論だ,双対性を感じる.
\end{tcolorbox}

\begin{proposition}[商空間]\label{prop-quotient-Banach-space}
    $Y\subset X$をノルム空間の部分空間とし,$Q:X\epi X/Y$を商空間への線形商写像とする.
    \begin{enumerate}
        \item $\norm{Qx}:=\inf\Brace{\norm{x-y}\in\R\mid y\in Y}$は代表元$x+Y$の取り方に依らず,$X/Y$上にセミノルムを定める.
        \item $Y$が$X$のノルム閉集合であることと,これがノルムとなることは同値.\footnote{定める位相がHausdorffであるかの議論と全くパラレルだ.}
        \item $X$がBanach空間で$Y$がその閉部分空間であるとき,$X/Y$もBanach空間である.
    \end{enumerate}
\end{proposition}
\begin{proof}\mbox{}
    \begin{enumerate}
        \item まず,次が成り立つ:
        \begin{align*}
            \forall_{x_1,x_2\in X}\;\forall_{\ep>0}\;\exists_{y_1,y_2\in Y}\quad\norm{Qx_1}+\norm{Qx_2}+\ep&\ge\norm{x_1-y_1}+\norm{x_2-y_2}\\
            &\ge\norm{(x_1+x_2)-(y_1+y_2)}\ge\norm{Q(x_1+x_2)}.
        \end{align*}
        よって,$\norm{Qx_1+Qx_2}\le\norm{Qx_1}+\norm{Qx_2}$からノルムの劣加法性,斉次性は商写像$Q$の線形性から従う.
        \item $\norm{Qx}=0\Rrightarrow x+Y=0+Y$の必要十分条件を導けば良い.
        $\norm{Qx}=0$のとき,商ノルムの定義より,列$(y_n)$が存在して,$\norm{y_n-x}\xrightarrow{n\to\infty}0$.
        すなわち,$\lim_{n\to\infty}y_n=x$.
        これはつまり,$\norm{Qx}=0\Rrightarrow x+Y=0+Y$は$x\in\oo{Y}$に同値.
        よって,商ノルムが実際にノルムであることと,$\oo{Y}=Y$は同値.
        \item $X/Y$のCauchy列$(z_n)$を取る.すると,$\norm{z'_{n+1}-z'_n}<2^{-n}$を満たす部分列が取れる.
        これに対して,$\norm{x_{n+1}-x_n}<2^{-n}$を満たす$x_n\in Q^{-1}(z'_n)$が存在する.実際,ある$x'_n\in X$について$Qx'_n=z'_n$であるが,ある$y\in Y$が存在して$\norm{x'_n-x_{n-1}-y}<2^{-n}$であるから,$x_n:=x'_n-y$と定めれば良いことと,帰納法により従う.
        よって$(x_n)$はCauchy列だから収束し,$Q$は連続だから$(z'_n)$も収束する.$(z_n)$はCauchy列だから,これも$Qx$に収束する.
    \end{enumerate}
\end{proof}
\begin{remark}[商ノルムのwell-definedness]
    (3)の証明内の議論の通り,
    $\norm{Qx}\le\norm{x}$であるから,商写像は連続である.
    よって,この商空間のノルムが定める位相は商位相より強くないことは分かるが,実は\textbf{関係$x_1-x_2\in Y$の定める商位相に一致}する(系\ref{cor-for-quotient-space}).
    $Y$が閉で$X$がBanachであるとき,商写像は開写像であることは,系\ref{cor-for-quotient-space}からもわかれば,開写像定理からもわかる.
\end{remark}
\begin{remark}[連続延長の失敗]
    商写像$Q$は$X$の単位開球を$X/Y$の単位開球に全射に写すが,単位閉球は一般にはそうとは限らない.
    実際,$z\in X/Y$が$\norm{z}=r$とは,$x\in Q^{-1}(z)$について$r=\inf\Brace{\norm{x-y}\ge 0\mid y\in Y}$ということだから,$r<1$ならば,ある$y'\in Y$について$\norm{x-y'}<1$と逆像を見つけることができるが,
    $r=1$の場合は逆像が単位閉球内にみつかるとは限らない.
    実際,一般の作用素について\ref{lemma-image-of-ball}のような描像がある.境界上で特異的な振る舞いをする.
\end{remark}

\begin{proposition}[商空間の普遍性]
    $T\in B(X,Y)$について,閉部分空間$Z\subset X$について$Z\subset\Ker T$が成り立つならば,下図
    を可換にする$\o{T}$であって$\norm{\o{T}}=\norm{T}$を満たすものがただ一つ存在する.
    \[\xymatrix{
        X\ar[r]^-T\ar[d]_-Q&Y\\
        X/Z\ar@{.>}[ur]_-{\o{T}}
    }\]
\end{proposition}

\begin{proposition}[Banach空間の標準分解]
    ノルム空間$X$とその部分空間$Y$について,$Y$と$X/Y$がBanach空間ならば,$X$もBanach空間である.
\end{proposition}

\subsection{Banach空間の内部構造}

\begin{tcolorbox}[colframe=ForestGreen, colback=ForestGreen!10!white,breakable,colbacktitle=ForestGreen!40!white,coltitle=black,fonttitle=\bfseries\sffamily,
title=連続線形延長の算譜]
    Banach空間論も,有限の範囲では,$k^n$だと思って扱える.
    この消息を正しく捉えるには「稠密」がキーワードになる.
\end{tcolorbox}

\begin{proposition}[自明なBanach部分空間]\label{prop-finite-subspaces}
    ノルム空間$X$の任意の有限次元部分空間$Y$はBanach空間であり,特に閉である.
    また,$\dim Y=n$ならば,任意の線型同型$k^n\iso Y$は位相同型でもある.
\end{proposition}

\begin{proposition}[Bounded Linear Transformation theorem]\label{prop-extension-of-operator-on-dense-subset}
    $X,Y$をBanach空間,$X_0$を$X$の稠密な部分空間とする.このとき,任意の作用素$T_0\in B(X_0,Y)$は一意的な延長$T\in B(X,Y)$をもち,$\norm{T}=\norm{T_0}$を満たす.
\end{proposition}

\begin{proposition}
    任意のノルム空間$X$について,Banaxh空間$\o{X}$であって,$X$を稠密な部分空間として含むものが同型を除いて一意的に存在する.
\end{proposition}

\subsection{ノルム空間の例}

\begin{tcolorbox}[colframe=ForestGreen, colback=ForestGreen!10!white,breakable,colbacktitle=ForestGreen!40!white,coltitle=black,fonttitle=\bfseries\sffamily,
title=]
    一般化すると,位相空間上の体値関数に,適切なノルムを入れることで構成する.
    ほとんどの例はその退化と見れる.
\end{tcolorbox}

\begin{example}[体の積]\mbox{}
    \begin{enumerate}
        \item $\bF^n$には$p\in[1,\infty]$について,$\norm{x}_p:=\paren{\sum\abs{x_k}^p}^{1/p}\;(p<\infty)$,$\norm{x}_\infty=\max\abs{x_k}$などのノルムが入り,いずれも同値である.
        これは特殊な例で,普通Banach空間には自然なノルムが一意に定まる.距離空間に同値な距離が大量にあるのと対照的である.
        \item 体$\bF$の添字集合$J$に関する直積($l^\infty$-直和)を$l^\infty(J)$で表す.
        \item 体$\bF$の添字集合$J$に関する直和($l^1$-直和)を$l^1(J)$で表す.
        \item $J=\N$のとき,これを省略して書く.$l^\infty$は有界列の空間,$l^1$は絶対収束列の空間である.$\sup_{k\in\N}\abs{a_k}<\infty$を満たす数列の空間$c_b$について,$c_b=l^\infty$である.$l^1\subsetneq(l^\infty)^*$である.
        \item $0$に収束する数列全体の空間を$c_0$で表すと,収束列は有界列であるから,上限ノルムについて$l^\infty$の閉部分空間となる.したがって$c_0$もBanachであり,その双対空間を考えると,$\N$上の有限符号付き測度であるから,$(c_0)^*=l^1$となる\ref{exp-Riesz-Markov-theorem-to-sequence-spaces}.また,$c_0$はいかなるBanach空間のpredualにもならない\ref{cor-predual-of-c0}.
        \item こうしてノルムが大事になってくる,$L^p(J)$の入り口である.$l^p$空間は$p\ge 1$についてBanach空間,$0<p<1$について$F$-空間である.
        \item $a_k=0\;\fe$を満たす数列全体の空間を$c_c$または$c_{00}$で表す.これはコンパクト収束位相\footnote{任意の有限集合上で収束する}について局所凸空間である.
    \end{enumerate}
\end{example}

\begin{example}[位相空間上の体値関数]\label{exp-Banach-spaces}
    $X$を局所コンパクトハウスドルフ空間とする.
    \begin{enumerate}
        \item $C_c(X):=\Brace{f\in C(X)\mid\supp f\text{ is compact}}$は,
        一様ノルム$\norm{f}_\infty:=\sup_{x\in X}\abs{f(x)}$について,
        完備でないノルム空間となる.
        \begin{enumerate}[(i)]
            \item $X$がコンパクトのとき,
            $C_b(X,\C)=C_c(X,\C)$で,これは$C^*$-環である.
            これを\textbf{連続関数環}という.
            この閉部分環であって,定数を含み,$X$の点を分離するものを\textbf{一様環}という.
            これは再び単位的なBanachとなる.
            \item $X\subset\R^n$が開または閉集合の場合,これは局所コンパクトハウスドルフで,
            さらにRiemann積分を通じて種々の$p\in[1,\infty]$-ノルムも入る.
            この完備化は,\textbf{Lebesgue空間}$L^p(X)$となり,Banach空間である.
        \end{enumerate}
        \item 一般の位相空間$X$について,$C_b(X):=\Brace{f\in C(X)\mid\Im f\text{ is bounded}}$は,
        一様ノルムについてBanach空間となり,各点ごとの乗法について\underline{Banach代数}になる.
        \begin{enumerate}[(i)]
            \item 離散空間$X$は局所コンパクトである.このとき,$X$上の任意の関数は連続になる:$C(X)=\Map(X,\bF)$.この場合,$C_b(I)$を$l^\infty(I)$と表す.
        \end{enumerate}
        \item 
        $C_0(X):=\Brace{f\in C(X)\mid\forall_{\ep>0}\;\Brace{x\in X\mid\abs{f(x)}\ge\ep}\text{ is compact}}$
        は\underline{可換な$C^*$-環}で\ref{def-functions-vanish-at-infinity},$\o{C_c(X)}=C_0(X)\subset C_b(X)$が成り立つ.
            $X$がコンパクトであることとBanach環$C_0(X)$が単位的であることは同値.
    \end{enumerate}
    $X$がコンパクトのとき,$(C(X))^*$は符号付Radon測度の空間で,$(C_0(X))^*$は正則なBorel測度全体のなす空間となる.
\end{example}

\begin{example}[測度空間上の体値関数の同値類]\label{exp-Banach-space-of-Radon-integrable-functions}
    局所コンパクトハウスドルフ空間$X$上のRadon積分$\int:C_c(X)\to\R$について,
    可積分関数$f$のなす空間
    \begin{align*}
        \L^p(X)=\left\{f\in\L(X)\;\middle|\;\int\abs{f}^p<\infty\quad(p\in[1,\infty))\right\},\quad\L^\infty(X)=\Brace{f\in\L(X)\;\middle|\;\esssup\abs{f}<\infty}
    \end{align*}
    は,セミノルム
    \[\norm{f}_p:=\paren{\int\abs{f}^p}^{1/p}\;(1\le p<\infty),\quad\norm{f}_\infty:=\esssup\abs{f}=\inf\left\{s\in\R\;\middle|\;\int(\abs{f}-\abs{f}\land s)=0\right\}\]と定めると,
    を持つ.
    $\cN(X)$を零関数$\int\abs{f}=0$のなす部分空間とすると,$L^p(X):=\L^p(X)/\cN(X)$はノルム空間となり,さらにBanach空間となる(Riesz-Fischerの定理).

    測度空間$X$と言ったが,測度もRadon積分から定義できるので明示的には出てこないことに注意.
\end{example}
\begin{remark}[本質的上限のwell-definedness]
    固定された$f$について,ノルム$\norm{f}_p$は$p>1$について連続.
    このとき,$\norm{f}_p\xrightarrow{p\to\infty}\norm{f}_\infty$と,本質的上限に収束する.
\end{remark}
    
\begin{theorem}[$L^p$空間の描像]\label{thm-Lp-of-dual-spaces}
    $(X,\Om,\mu)$を測度空間とし,$1\le p<\infty,1/p+1/q=1$を共役指数とする.
    $g\in L^q(\mu)$に対して,$F_g:L^p(\mu)\to\bF$を$F_g(f):=\int fgd\mu$で定める.
    \begin{enumerate}
        \item $1<p<\infty$のとき,$F:L^q(\mu)\iso L^p(\mu)^*$は等長同型を定める.
        \item $p=1$で$(X,\Om,\mu)$が$\sigma$-有限のとき,$F:L^\infty(\mu)\iso L^1(\mu)^*$は等長同型を定める.
    \end{enumerate}
\end{theorem}

\begin{example}[導関数ノルムを備えたBanach空間]\label{exp-Banach-space-with-derivative-norm}
    $\Om\osub\R^n$上の$C^m$-級関数であって,$m$階以下の導関数がすべて$\Om$上有界である関数の全体を$B^m(\Om)$または$C_b^m(\Om)$で表す.
    これはある種の1-ノルム$\norm{u}=\sum_{\abs{\al}\le m}\sup_{x\in\Om}\abs{D^\al u(x)}$について,Banach空間となる.
    $C_b^0(\Om)$は$C_b(\Om)$に一致する.
    $C_c^m(\Om)\subset C_b^m(\Om)$は変わらない.
\end{example}

\begin{example}[Lipschitz連続な関数のBanach空間]
    \[\Lip[a,b]:=\Brace{u\in C([a,b])\mid\exists_{L\in\R}\;\forall_{x,x'}\;\abs{u(x')-u(x)}\le L\abs{x'-x}}\]
    は,ノルム
    \[\norm{u}:=\sup_{x\in[a,b]}\abs{u(x)}+\sup_{x\ne x'\in[a,b]}\frac{\abs{u(x')-u(x)}}{x'-x}\]
    についてBanach空間となる.
\end{example}

\begin{example}[群上のBanach algebra]
    特別なクラスである.
    \begin{enumerate}
        \item 局所コンパクト群$G$上のRadon測度$(C(G))^*$はBanach環をなす.積は測度の畳み込みとする.
        \item Lebesgue空間$L^1(\R)$は畳み込みを積として非単位的なBanach代数をなす.単位元はDirac関数に相当する.
        $\R$は一般の局所コンパクトハウスドルフな位相群$G$,Lebesgue測度はHaar測度に一般化出来る.
        $L^1(G)$が単位的であることは,$G$が離散群であることに同値.
    \end{enumerate}
\end{example}

\subsection{可分Banach空間の消息}

\begin{tcolorbox}[colframe=ForestGreen, colback=ForestGreen!10!white,breakable,colbacktitle=ForestGreen!40!white,coltitle=black,fonttitle=\bfseries\sffamily,
title=]
    収束列の空間$l^1$には,可算なSchauder基底が存在する.$c_0\mono c_\infty\mono c_b=l^\infty$で,$l^\infty$は可分でない.
    この包含関係は稠密からは程遠い.なお,$c_\infty$は収束列の空間とした.
    また,$c_c\mono l^p\mono k^q\mono c_0\;(0<p<q<\infty)$が成り立つ,包含関係は後者の位相について稠密である.
    なお,上述の包含関係はすべて真に成り立つ.
\end{tcolorbox}

\begin{example}[Tsirelson space (74)]
    どの部分空間も$l^p\;(1\le p<\infty)$とも$c_0$とも同型でないようなBanach空間が存在する.
\end{example}

\begin{theorem}[Bessaga and Pelczynski]\label{thm-Bessaga-and-Pelczynski}
    Banach空間$X$の列$(x_n)$について,これが生成する閉部分空間$\Span\Brace{x_n}$が$c_0$に同型な部分空間を含むための十分条件は,
    \begin{enumerate}
        \item $\inf_{n\in\N}\norm{x_n}>0$.
        \item $\exists_{C\ge0}\;\forall_{k\ge1}\;\forall_{\ep_j\in\{\pm 1\}}\;\Norm{\sum^k_{j=1}\ep_jx_j}\le C$.
    \end{enumerate}
\end{theorem}

\begin{theorem}
    任意の可分Banach空間は,ある$l^1$の商空間と等長同型である.
\end{theorem}

\subsection{ノルム空間の構成}

\begin{tcolorbox}[colframe=ForestGreen, colback=ForestGreen!10!white,breakable,colbacktitle=ForestGreen!40!white,coltitle=black,fonttitle=\bfseries\sffamily,
title=]
    Banach空間の族に対して,その直積であって,各点毎の一様ノルムを入れたものを直積といい,再びBanach空間となる.
    一方で,代数的直和を考えると,様々なノルムを使用できるが,Banach性が保たれるとは限らず,適宜完備化を考える必要がある.
    前者を$l^\infty$-直和,後者を$l^1$-直和と呼ぶ.圏Banでは前者は直積,後者は余直積である.
    直和の完備化は,直積の言葉で捉えられる.

    ただし,これらの概念は族が有限のとき,ノルムの取り方の違いのみに退化する.
\end{tcolorbox}

\subsubsection{直積}

\begin{tcolorbox}[colframe=ForestGreen, colback=ForestGreen!10!white,breakable,colbacktitle=ForestGreen!40!white,coltitle=black,fonttitle=\bfseries\sffamily,
title=]
    ノルム空間の直積は,線型空間の直積のうち,一様有界な元からなる部分空間として定義される.
    すると,圏Banはこれについて完備である.
    ノルムは$\infty$ノルムとする.
    体$\F$の添字集合$A$に関するこの意味での直積を$l^\infty(A):=\prod_{a\in A}\F$で表す.
\end{tcolorbox}

\begin{definition}[direct product of normed spaces]
    ノルム空間の族$(X_j)_{j\in J}$に対して,積空間$\prod_{j\in J}X_j$の元であって,次のように定めるノルムが有限である元$\norm{x}_\infty:=\sup\norm{\pr_j(x)}<\infty$\footnote{$\abs{J}<\infty$のとき,これは常に満たされる.}(すなわち,$\norm{\pr_j(x)}$が有界である元)のなす空間を,ノルム空間の\textbf{直積}という.
\end{definition}

\begin{proposition}[Banach空間の直積に対する閉性]
    Banach空間の族$(X_j)$について,直積$\prod_{j\in J}X_j$もBanach空間となる.
\end{proposition}

\subsubsection{直和}

\begin{tcolorbox}[colframe=ForestGreen, colback=ForestGreen!10!white,breakable,colbacktitle=ForestGreen!40!white,coltitle=black,fonttitle=\bfseries\sffamily,
title=]
    ノルム空間の直和とは,線型空間としての直和の完備化をいう.
    この余直積についても,圏Banは完備である(ノルムは1ノルム$\Norm{\bigoplus_{s\in S}x_s}=\sum_{s\in S}\norm{x_s}$とする).
    加群の直和を「代数的直積」とも呼ぶ.
    ノルム空間の直和は直積の言葉で捉えられる.
    体$\bF$の添字集合$A$に関する直和を,$l^1(A)$で表す.
\end{tcolorbox}

\begin{definition}[algebraic direct product / direct sum]
    $\sum_{j\in J}X_j:=\Brace{x\in\prod_{j\in J}\;\middle|\;\pr_j(x)=0\fe}$上に,$p$-ノルム($p\in[1,\infty]$)を考えたものを,\textbf{代数的直積}という.
    $p=\infty$の場合を\textbf{直和}とも呼ぶ.
\end{definition}

\begin{proposition}[代数的直積の完備化]\label{prop-completion-of-algebraic-direct-product}
    Banach空間の族$(X_j)$について,$p$-ノルムについて代数的直積を取ったノルム空間$\sum_{j\in J}X_j$を考える.
    \begin{enumerate}
        \item $p\in[1,\infty)$のとき,$\sum_{j\in J}X_j$の完備化は$\Brace{x\in\prod_{j\in J}X_j\;\middle|\;\sum_{j\in J}\norm{\pr_j(x)}^p<\infty}$に一致する.
        \item $p=\infty$のとき,$\sum_{j\in J}X_j$の完備化は$\Brace{x\in\prod_{j\in J}X_j\;\middle|\;\norm{\pr_-(x)}:J\to\R \in C_0(J),ただしJは離散空間とする}$に一致する.
    \end{enumerate}
\end{proposition}
\begin{remarks}
    結局,圏Banでの直和とは,$l^1$-直和
    \[\bigoplus_{i}^1W_i:=\Brace{(w_i)_i\in\prod_{i}W_i\;\middle|\;\sum_i\norm{w_i}<\infty}\]
    である.
\end{remarks}

\begin{example}\label{exp-direct-sum-of-norm-spaces}
    $J$を離散空間とする.
    \begin{enumerate}
        \item $X_j=F$としたとき,$\prod_{j\in J}F=:l^\infty(J)$と表し,有界関数$J\to F$のなす空間となる.
        \item $p<\infty$ノルムに関して空間$l^p(J)$は,$L^p(J)$と同一視できる.
        \item 直和は$\sum_{j\in J}F=:c_0(J)$と表し,無限遠で消える関数$f:J\to F$,すなわち,$\forall_{\ep>0}\;\abs{\Brace{j\in J\mid\abs{f(j)}\ge\ep}}<\infty$を満たす関数$f$の空間となる.
        \item $J=\N$のとき,$(J)$は省略して$l^p,c_0$などと表す.$c_0$は単に,$0$に収束する数列のなす空間である.
    \end{enumerate}
\end{example}

\subsubsection{Banach空間の直和}

\begin{tcolorbox}[colframe=ForestGreen, colback=ForestGreen!10!white,breakable,colbacktitle=ForestGreen!40!white,coltitle=black,fonttitle=\bfseries\sffamily,
title=]
    Hilbert空間の$l^2$-直和は再びHilbert空間となり,これをHilbert空間の直和と呼んでしまう.
    これが圏Hilbの余直積となる.
\end{tcolorbox}

\begin{definition}
    $W$をBanach空間の族とする.
    \begin{enumerate}
        \item $l^p$-直和とは,$\bigoplus_{i}^pW_i:=\Brace{(w_i)_i\in\prod_{i}W_i\;\middle|\;\sqrt[q]{\sum_i\norm{w_i}^p}<\infty}$.
        \item $l^\infty$-直和とは,$\bigoplus_{i}^pW_i:=\Brace{(w_i)_i\in\prod_{i}W_i\;\middle|\;\sup_{i}\norm{w_i}<\infty}$.
        \item $l^1$-直和とは,$\bigoplus_{i}^pW_i:=\Brace{(w_i)_i\in\prod_{i}W_i\;\middle|\;\sum_i\norm{w_i}<\infty}$.
    \end{enumerate}
\end{definition}

\subsection{ノルム位相の性質}

\begin{proposition}[有限次元空間の特徴付け]\label{prop-unit-ball-in-normed-space}
    ノルム空間$X$の閉単位球$B$について,次の2条件は同値.
    \begin{enumerate}
        \item $B$はノルム位相についてコンパクトである.
        \item $X$は有限次元である.
    \end{enumerate}
\end{proposition}
\begin{remark}
    一方で,弱位相についてはコンパクトになる.
\end{remark}

\section{カテゴリ}

\begin{tcolorbox}[colframe=ForestGreen, colback=ForestGreen!10!white,breakable,colbacktitle=ForestGreen!40!white,coltitle=black,fonttitle=\bfseries\sffamily,
title=]
    ここではノルム空間$X,Y$はBanachであるとして,そのノルム位相の性質を調べる.
    閉グラフ定理を除いては,複素解析学の世界の一般化にも見える.
\end{tcolorbox}

\subsection{Baire category theorem}

\begin{tcolorbox}[colframe=ForestGreen, colback=ForestGreen!10!white,breakable,colbacktitle=ForestGreen!40!white,coltitle=black,fonttitle=\bfseries\sffamily,
title=]
    3つのBanach空間上の作用素の基本結果は,全てBaireの範疇定理の上に拠って立つ.
    これは,完備距離空間の稠密開集合の可算交叉は再び稠密である(くらいに「濃い」)ことを主張している.
\end{tcolorbox}

\begin{definition}[Baire space, nowhere dense]\mbox{}
    \begin{enumerate}
        \item 閉包が内点を持たない集合を\textbf{疎集合}という.集合が疎であることと,その補集合が稠密であることは同値.
        \item 可算個の疎集合の合併として表せる集合を\textbf{第一類}という.
        \item そうでない集合,すなわち,任意の稠密開集合の可算共通部分は稠密であるような位相空間を\textbf{第二類}または\textbf{Baire空間}という.
    \end{enumerate}
\end{definition}
\begin{proof}
    任意の半径$r>0$の閉球$B_0$を取り,これと$\cap_{n=1}^\infty A_n$との共通部分が空でないことを示せば,$\cap_{n=1}^\infty A_n$の稠密性が示せる.

    いま,$A_1\cap B_0^\circ$は空でない開集合だから,ある半径$2^{-1}r$より小さい閉球$B_1$が取れる.
    これを繰り返すことで,$B_n\subset A_n\cap B^\circ_{n-1},r(B_n)<2^{-n}r$を満たす閉球の列$(B_n)$が取れる.
    $X$は完備だから,$\exists_{x\in X}\;\{x\}=\cap_{n=1}^\infty B_n\subset B_0\cap\paren{\cap_{n=1}^\infty A_n}$が成り立ち,共通部分が空でないことがわかった.
\end{proof}

\begin{proposition}[Baire category theorem 1]
    任意の完備距離空間$X$はBaire空間である.
    
    すなわち,$(A_n)$を$X$の稠密開集合の列とすると,この共通部分$\cap_{n\in\N}A_n$は$X$で稠密である.
    また双対命題は,閉集合列を用いて$X=\cup_{n=1}^\infty F_n$と表せたとき,少なくとも一つの$F_n$は疎でない(内点を持つ).
\end{proposition}
\begin{remark}
    実はZFの下で従属選択公理と呼ばれる弱い選択公理と同値になる.
\end{remark}

\begin{proposition}[BCT2]
    任意の局所コンパクトハウスドルフ空間はBaire空間である.
\end{proposition}
\begin{remarks}
    こちらは函数解析学では使わないが,任意の有限次元多様体がBaire空間であることがわかる.多様体がパラコンパクトでない場合でも成り立つ.
    なお,局所コンパクトでない完備距離空間も,距離化可能でない局所コンパクトハウスドルフ空間も存在することに注意.
\end{remarks}

\subsection{開写像定理}

\begin{tcolorbox}[colframe=ForestGreen, colback=ForestGreen!10!white,breakable,colbacktitle=ForestGreen!40!white,coltitle=black,fonttitle=\bfseries\sffamily,
title=]
    開写像定理と逆写像定理という,複素解析と並行な議論.
\end{tcolorbox}

\begin{lemma}[単位閉球の像の描像]\label{lemma-image-of-ball}
    $X,Y$をBanach空間とする.
    $T\in B(X,Y)$による単位閉球$B(0,1)$の像が,$Y$のある球$B(0,r)\;(r>0)$の中で稠密であるとする.この時,$\forall_{\ep\in(0,1)}\; B(0,(1-\ep)r)\subset T(B(0,1))$.
\end{lemma}
\begin{proof}\mbox{}
    \begin{description}
        \item[方針] $A:=T(B(0,1))$と表すと,これは$B(0,r)$上稠密である.任意の$y\in B(0,r)$と$\ep\in(0,1)$を取り,$y\in(1-\ep)^{-1}A$が従うことを示せば良い.
        \item[構成] まず,$y$に収束する$A$の列$(y_n)$を構成する.$A$は$B(0,r)$上稠密だから,$\exists_{y_1\in A}\;\norm{y-y_1}<\ep r$を満たす.
        次に,$y-y_1\in B(0,\ep r)$であることに注目すると,$\ep A$はこの上で稠密だから,$\exists_{y_2\in\ep A}\;\norm{y-y_1-y_2}<\ep^2r$を満たす.
        これを繰り返すことで,$y_n\in\ep^{n-1}A,\Norm{y-\sum^n_{k=1}y_k}<\ep^nr$を満たす$A$の列$(y_n)$が取れ,対応する$X$の列$(x_n)$が$\norm{x_n}\le\ep^{n-1},Tx_n=y_n$を満たすように取れる.
        \item[証明] するとこの列$(x_n)$は絶対収束級数$x:=\sum_{n\in\N}x_n$を定めるが,$Tx=y$であり,また$\norm{x}\le\sum_{n\in\N}\ep^{n-1}=(1-\ep)^{-1}$を満たすから,$y\in(1-\ep)^{-1}A$.
    \end{description}
\end{proof}

\begin{theorem}[開写像定理]\label{thm-open-mapping-theorem}
    $X,Y$をBanach空間とし,有界線型作用素$T\in B(X,Y)$を全射とする.このとき,$T$は開写像である.
\end{theorem}
\begin{proof}\mbox{}
    \begin{description}
        \item[方針] $T$の線形性と,$X$の位相は開球を基として生成されることより,$T(B(0,1))$が$0$を内点に持つことを示せれば十分である.
        実際このとき,任意の開集合の基底$U(x,r)$について,$B(x,\delta)\subset T(B(x,r))\subset\oo{T(U(x,r))}$より,$U(x,\delta)\subset T(U(x,r))$が取れることがわかる.
        他の点についても,$X,Y$の各点の等質性より従う.\footnote{原点に引き戻して考えるのは,位相群と同じ.}
        \item[証明] 全射性より,$Y=T(X)=\cup_{n\in\N}\oo{T(B(0,n))}$.
        Baireの定理より,$\exists_{n\in\N}\;B(y,\ep)\subset\oo{T(B(0,n))}$.
        よって,$T(B(0,n))$は$B(y,\ep)$上稠密,
        $T(B(0,1))$は$B(y/n,\ep/n)$上稠密である.
        $2B(0,\ep/n)\subset B(y/n,\ep/n)-B(y/n,\ep/n)$と,像$T(B(0,1))$が対称凸であることより,$B(0,\ep/n)$上稠密でもある.
        よって補題より,$\forall_{\delta\in(0,\ep/n)}\;B(0,\delta)\subset T(B(0,1))$.
    \end{description}
\end{proof}

\begin{corollary}[逆写像定理]\label{cor-inverse-mapping-theorem}
    Banach空間の間の任意の全単射な有界線型作用素は,有界な逆射を持つ.
\end{corollary}

\begin{corollary}[ノルムが同値であることの十分条件]
    線型空間$X$が,2つのノルム$\norm{-}_1,\norm{-}_2$についてBanach空間をなし,$\exists_{\al>0}\;\norm{-}_1\le\al\norm{-}_2$が成り立つとする.
    この時,$\beta>0$が存在して,$\norm{-}_2\le\beta\norm{-}_1$も満たす.
\end{corollary}

\subsection{閉グラフ定理}

\begin{tcolorbox}[colframe=ForestGreen, colback=ForestGreen!10!white,breakable,colbacktitle=ForestGreen!40!white,coltitle=black,fonttitle=\bfseries\sffamily,
title=全空間で定義された閉作用素は有界である]
    関数が連続であることとグラフが閉であることは同値である.同様のことが作用素でも起こる.
    
    グラフが閉である作用素は\textbf{閉作用素}という.
    グラフが閉で,全空間$X$上で定義された線型作用素は有界であるが,部分集合$D(T)\subsetneq X$上で定義された線型作用素については一般には有界とは限らない.
\end{tcolorbox}

\begin{theorem}[closed graph theorem]\label{thm-closed-graph-theorem}
    作用素$T:X\to Y$のグラフ$G(T):=\Brace{(x,y)\in X\times Y\mid Tx=y}$が直積空間$X\times Y$の閉集合ならば,$T$は有界である.
\end{theorem}
\begin{proof}
    仮定より,$G(T)$は$X\times Y$内の閉部分空間をなす,特にBanachである.
    このとき,$\pr_1,\pr_2$はいずれもノルム減少的であるから,特に有界である.また$\pr_1$は全単射を定めるから,逆写像定理より,有界な逆$\pr_1^{-1}\in B(X,G(T))$を持つ.
    $T=\pr_2\circ\pr_1^{-1}$より,$T$も有界.
\end{proof}

\subsection{一様有界性の原理}

\begin{tcolorbox}[colframe=ForestGreen, colback=ForestGreen!10!white,breakable,colbacktitle=ForestGreen!40!white,coltitle=black,fonttitle=\bfseries\sffamily,
title=]
    内部homの構造について,有界性が綺麗に対応する.
    すなわち,有界作用素の族が各点有界ならば,一様有界である.
    結局,有界線型作用素列が一様有界であることを示すには,任意の$x\in X$について$\ev_x$の像が収束列を定めることを示せば良い.
\end{tcolorbox}

\begin{theorem}[作用素族は各点有界ならば一様有界 (Banach-Steinhaus)]
    $B(X,Y)$の族$(T_\lambda)$について,
    各点有界(全ての$x\in X$について列$(T_\lambda x)$が$Y$で有界)ならば,
    作用素ノルム$\{\norm{T_\lambda}\in\R\mid\lambda\in\Lambda\}$も有界である.
\end{theorem}
\begin{proof}
    $Y_\Lambda:=\prod_{\lambda\in\Lambda}Y$を直積空間,$T:=\prod_{\lambda\in\Lambda}T_\lambda:X\to Y_\Lambda$を積作用素とすると,
    仮定より$\forall_{x\in X}\;\norm{T_\lambda x}<\infty$だから,$T$はたしかにwell-definedである.
    この$T$が有界であることを示せば,積作用素の普遍性$\forall_{\lambda\in\Lambda}\;T_\lambda=\pr_\lambda\circ T$より,$\forall_{\lambda\in\Lambda}\;\norm{T_\lambda}\le\norm{T}<\infty$が従う.

    $(x,y)\in X\times Y_\Lambda$に収束する列$(x_n,Tx_n)$を任意にとり,$Tx=y$を示せば良い.
    各$\lambda\in\Lambda$について$T_\lambda$は連続であるから,$T_\lambda x=\pr_\lambda(y)$.これは$y=Tx$を意味する.
\end{proof}
\begin{remark}[Banach-Steinhaus]
    $Y$は一般のノルム空間とできる.
    基本的には,積作用素のノルムが上界として見つかるのが原理である.
    次のように,対偶命題について
    一般化されて主張されることもある(Rudin)\footnote{ある点$x\in X$が存在して,$\{\norm{T_\lambda}\}$は有界でない,というだけでなく,そのような$x$は\textbf{稠密な}開集合の可算共通部分として取れる.}:
    $X$をBanach空間,$Y$をノルム空間とし,$\{\Lambda_\al\}_{\al\in A}\subset B(X,Y)$を集合とする.
    このとき,次のうちどちらか一方のみが,必ず成り立つ.
    \begin{enumerate}
        \item 作用素ノルムが有界である:$\exists_{M<\infty}\;\forall_{\al\in A}\;\norm{\Lambda_\al}\le M$.
        \item $X$のある稠密な$G_\delta$-集合\footnote{開集合の可算共通部分によって表現される集合}が存在して,$\forall_{x\in G_\delta}\;\sup_{\al\in A}\norm{\Lambda_\al x}=\infty$.
    \end{enumerate}
\end{remark}

\begin{corollary}[作用素ネットが各点有界であることの十分条件]
    $(T_\lambda)_{\lambda\in\Lambda}$を,$B(X,Y)$のネットであって,任意の$x\in X$について$Y$のネット$(T_\lambda x)_{\lambda\in\Lambda}$は有界で収束するとする.
    このとき,$T\in B(X,Y)$が存在して,これに各点収束する$\forall_{x\in X}\;T_\lambda x\to Tx$.
\end{corollary}
\begin{proof}
    $Tx:=\lim_{\lambda\in\Lambda}T_\lambda x$によって定まる作用素$T:X\to Y$が有界であることを示せば良い.
    この$\{T_\lambda\}$は各点有界族だったから一様有界でもある:$\exists_{\alpha\in\R}\;\forall_{\lambda\in\Lambda}\;\norm{T_\lambda}\le\al$.
    したがって,$\forall_{x\in X}\;\norm{Tx}\le\al\norm{x}$.すなわち,$T$は有界である.
\end{proof}
\begin{remark}
    命題\ref{prop-extension-of-operator-on-dense-subset}より,$X$の稠密な部分集合上で任意のネットが有界で収束することを示せば十分.

\end{remark}

\section{双対空間}

\begin{tcolorbox}[colframe=ForestGreen, colback=ForestGreen!10!white,breakable,colbacktitle=ForestGreen!40!white,coltitle=black,fonttitle=\bfseries\sffamily,
title=]
According to Helmut H. Schaefer, "the study of a locally convex space in terms of its dual is the central part of the modern theory of topological vector spaces, for it provides the deepest and most beautiful results of the subject."
\end{tcolorbox}

\subsection{Hahn-Banachの有界線型汎関数拡張定理}

\begin{tcolorbox}[colframe=ForestGreen, colback=ForestGreen!10!white,breakable,colbacktitle=ForestGreen!40!white,coltitle=black,fonttitle=\bfseries\sffamily,
title=双対空間の元の存在]
    超平面の分離定理は,これの特別な場合である.
    このことも含めて,$\abs{X^*}\ne\emptyset$を主張する位相線型空間論でのHahn-Banachの定理は,位相空間論におけるUrysohnの補題と同じ立ち位置である.
    Urysohnの補題は正規空間において閉集合を分離するという主張であるが,連続関数を構成する際にも用いられる.
    It ensures that such a space will have enough continuous linear functionals such that the topological dual space is interesting.\footnote{\url{https://ncatlab.org/nlab/show/Hahn-Banach+theorem}}
    さらに言えば,選択公理と同じ役割をするともみれる.
\end{tcolorbox}

\begin{definition}[dual space, Minkowski functional]
    $X$を$F$-ノルム空間とする.
    \begin{enumerate}
        \item 汎関数の空間$X^*:=B(X,F)$を\textbf{双対空間}という.$F$は完備だからこれはBanach空間である\ref{prop-internal-hom}.
        \item 関数$m:X\to\R$が\textbf{Minkowski汎関数}または\textbf{劣線形汎関数}であるとは,次の2条件を満たすことをいう:
        \begin{enumerate}[(a)]
            \item (劣加法性) $m(x+y)\le m(x)+m(y)$.
            \item (正斉次性) $\forall_{t\in\R_{\ge 0}}\;m(tx)=tm(x)$.
        \end{enumerate}
    \end{enumerate}
\end{definition}
\begin{remarks}
    Minkowski汎関数は線型とは限らない.
    これは,セミノルム(よりも一般的に距離的な概念)を表現する装置である.
    実際セミノルムはMinkowski汎関数である.
    だから値は実数となっている.
\end{remarks}

\begin{lemma}[fundamental lemma]
    $m:X\to\R$を実線型空間上のMinkowski汎関数とし,$\varphi:Y\to\R$を部分空間上の汎関数で$m$で抑えられるものとする:$\forall_{y\in Y}\;\varphi(y)\le m(y)$.
    このとき,$X$上の汎関数$\wt{\varphi}:X\to\R$であって$m$によって抑えられる延長$\wt{\varphi}|_Y=\varphi$が存在する.
\end{lemma}
\begin{proof}\mbox{}
    \begin{description}
        \item[$m$以下の延長の存在] いま$Y$を任意の部分空間,$\varphi$をその上の線型汎関数とする.
        このとき,任意の$x\in X\setminus Y$に対して,延長$\wt{\varphi}:Y+\R x\to\R$であって$m$によって抑えられるものが存在することを示す.

        いま,$\al:=\wt{\varphi}(x)$の定め方であって,$\forall_{s\in\R,y\in Y}\;\wt{\varphi}(y+sx)=\varphi(y)+s\al$かつ$\varphi(y)+s\al\le m(y+sx)$を$s=\pm 1$の場合について満たすことが必要十分.
        これは,$\forall_{y,z\in Y}\;\varphi(y)-m(y-x)\le\al\le-\varphi(z)+m(z+x)$が必要十分.
        仮定より,
        \begin{align*}
            -\varphi(z)+m(z+x)-\varphi(y)+m(y-x)&=m(y-x)+m(z+x)-\varphi(y+z)\\
            &\ge m(y+z)-\varphi(y+z)\ge 0
        \end{align*}
        だから,$\Square{\sup_{y\in Y}\Brace{\varphi(y)-m(y-x)},\sup_{z\in Y}\Brace{-\varphi(z)+m(z+x)}}\ne\emptyset$より,たしかに条件を満たす延長は存在する.
        \item[Zornの補題により極大な延長を探す]
        こうして,$Y\subset Z\subset X$を満たす部分空間$Z$と,その上の$\varphi$の$m$以下の延長$\psi$の組$(Z,\psi)$全体からなる集合$\Lambda$を考えると,これは空でない.
        また,$(Z_1,\psi_1)\le(Z_2,\psi_2):\Leftrightarrow Z_1\subset Z_2\land\psi_2|_{Z_1}=\psi_1$と定めると,$\Lambda$は順序集合をなす.
        さらにこれは帰納的であることを示す.
        $N=\Brace{(Z_\mu,\psi_\mu)}_{\mu\in M}\subset\Lambda$を全順序部分集合としたとき,$Z:=\cup_{\mu\in M}Z_\mu,\psi(z):=\psi_\mu(z)\;(z\in Z_\mu)$と定めると,$(Z,\psi)\in\Lambda$で,$N$の上界である.
        よってZornの補題より,$\Lambda$の極大元$(Z,\wt{\varphi})$が存在するが,仮に$Z\ne X$としたら,$Z+\R x\;(x\in X\setminus Z)$上の延長を考えることで$Z$の極大性に矛盾する.よって,$Z=X$.
    \end{description}
\end{proof}

\begin{theorem}[Hahn-Banach extension theorem]
    $m:X\to\R$を線型空間上のセミノルム,$\varphi$を部分空間$Y\subset X$上の汎関数であり$\abs{\varphi}\le m$を満たすとする.
    このとき,汎関数$\wt{\varphi}$であって,$\abs{\wt{\varphi}}\le m$を満たし,$\wt{\varphi}|_Y=\varphi$を満たす延長が存在する.
\end{theorem}
\begin{proof}
    $\bF=\R$の場合,セミノルムはMinkowski汎関数で,$\abs{\varphi}\le m\Rightarrow\varphi\le m$だから,これは補題の特別な場合である.
    よって,$\bF=\C$の場合を考える.
\end{proof}
\begin{remarks}
    もちろん,$X$がノルム空間で,$f:Y\to\C$がその上の有界な線型汎関数であるときにも適用可能で,この場合,$\norm{F}=\norm{f}$を満たす延長が可能.
\end{remarks}

\subsection{有界線型汎関数延長定理の系}

\begin{tcolorbox}[colframe=ForestGreen, colback=ForestGreen!10!white,breakable,colbacktitle=ForestGreen!40!white,coltitle=black,fonttitle=\bfseries\sffamily,
title=]
    ノルム空間$X$の双対空間$X^*$の正規化された$B^*$の元について,
    \begin{enumerate}
        \item 任意に選んだ$x_0\in X\setminus\{0\}$に対して$\varphi(x_0)=\norm{x_0}$を満たすものが取れる.
        \item 任意に選んだ閉部分空間$Y$と,その外の点$x\in X\setminus Y$に対して,$Y$上で消えていて,$\varphi(x)=\inf_{y\in\R}\norm{x-y}$を満たすように取れる.
    \end{enumerate}
    これにより,$X$が自明でないならば,$X^*$も自明でなく,さらに$X$の点を分離することが保証される.$X$と$X^*$とのinterplayが産む豊かな理論の始まりである.
\end{tcolorbox}

\begin{proposition}[複素線型空間は実線形空間]
    $V$を複素線型空間とする.
    \begin{enumerate}
        \item $f\in V^*$の実部を$u\in V^*$とする.このとき,$\forall_{x\in V}\;f(x)=u(x)-iu(ix)$.
        \item $u\in B(V,\R)$に対して,$f(x):=u(x)-iu(ix)$と定めると,$f\in V^*$である.
        \item $V$はノルム空間であり,$f,u\in V^*$は$\forall_{x\in V}\;f(x)=u(x)-iu(ix)$を満たすとする.このとき,$\norm{f}=\norm{u}$.
    \end{enumerate}
\end{proposition}

\begin{corollary}[汎関数の構成]\label{cor-Hahn-Banach}
    ノルム空間$X$において,
    任意の元$x\in X\setminus\{0\}$に対して,
    汎関数$\varphi\in X^*$が存在して,$\norm{\varphi}=1$かつ$\varphi(x)=\norm{x}$を満たす.
\end{corollary}
\begin{proof}
    $\varphi:\bF x\to\bF$を,$\varphi(\al x)=\al\norm{x}$で定めると,$\norm{\varphi}=1$である.
    これはノルム$\norm{-}$より大きくないままの$X$上への延長が存在するが,$\norm{\al x}=1$を満たす$\al x\in\bF x$について$\varphi(\al x)=1$であるから,$\norm{\varphi}=1$のままである.
\end{proof}
\begin{remarks}[state]
    汎関数による点の分離能力は十分に高い.定理\ref{thm-Spectrum-is-compact}などに使う.
    また,ノルム空間$X$が単位的$C^*$-代数であるとき,この条件を満たす汎関数を\textbf{状態}という.\footnote{\url{https://ncatlab.org/nlab/show/state on a star-algebra}}
    局所コンパクトハウスドルフ空間上の連続関数環$C(X)$上の状態とは,確率測度が定める積分である.
    ヒルベルト空間$H$上に表現された$C^*$-環の状態とは,任意の単位ベクトル$\xi\in B(H)$が定める汎関数$T\mapsto(T\xi|\xi)$であり,この操作は物理量の測定の期待値を与える.
\end{remarks}

\begin{corollary}
    ノルム空間$X$において,任意の閉部分空間$Y\subset X$と$x\in X\setminus Y$に対して,汎関数$\varphi\in X^*$が存在して,$\norm{\varphi}=1$かつ$\varphi|_Y=0$かつ$\varphi(x)=\inf\Brace{\norm{x-y}\in\R\mid y\in Y}$を満たす.
\end{corollary}
\begin{remarks}
    \ref{prop-標準写像の随伴}に関連する.
    この時点ですでに,分離定理にステートメントが似てきた.
    この系は,部分空間$Y\subset X$について,次の2条件が同値である,というようにも定式化出来る.
    \begin{enumerate}
        \item $x_0\in \oo{Y}$.
        \item $Y$上で零だが,$f(x_0)\ne0$を満たす有界線型汎関数$f\in X^*$は存在しない.
    \end{enumerate}
\end{remarks}

\begin{corollary}
    $x\in X$のノルムは,$X^*$を用いて次のようにも表せる:
    \[\norm{x}=\sup\Brace{\abs{f(x)}\in\bF\mid f\in B^*}.\]
    特に,$x\in X$に関する評価写像は$\ev_x\in X^*$で,$\norm{\ev_x}=\norm{x}$.
\end{corollary}
\begin{proof}
    系\ref{cor-Hahn-Banach}による.
\end{proof}

\subsection{零化空間と再双対空間}

\begin{tcolorbox}[colframe=ForestGreen, colback=ForestGreen!10!white,breakable,colbacktitle=ForestGreen!40!white,coltitle=black,fonttitle=\bfseries\sffamily,
title=]
    Hahn-Banachの拡張定理より,標準的な単射$\kappa_X:X\mono (X^*)^*$は等長写像であることが従う.
    これがBanのisoでもあるとき,$X$を回帰的という(Hahn-Banachの定理より等長写像であることが従うから,Banの同型でもある).

    annihilatorの概念は一般の加群について定義され,可換環に対しては随伴と関係が深く,内積に関する場合を直交補空間という.
\end{tcolorbox}

\begin{definition}[annihilator]
    部分空間$Y\subset X$と$Z\subset X^*$について,
    \begin{enumerate}
        \item $Y$の零化空間とは,$Y^\perp=\Brace{\varphi\in X^*\mid\forall_{y\in Y}\;\varphi(y)=0}$を指す.
        \item $Z$の零化空間とは,$Z^\perp=\Brace{x\in X\mid\forall_{\varphi\in Z}\;\varphi(x)=0}$を指す.
    \end{enumerate}
\end{definition}

\begin{lemma}\mbox{}
    \begin{enumerate}
        \item 一般に$Y\subset(Y^\perp)^\perp$である.
        \item $Y$が閉部分空間である時,$Y=(Y^\perp)^\perp$である.
        \item 一方で,$Z$が閉部分空間であってもお,$Z=(Z^\perp)^\perp$とは限らない.
    \end{enumerate}
    もちろん,$X$が有限次元の場合はいずれの等号も常に成り立つ.
\end{lemma}

\begin{proposition}\label{prop-dense-subspace-of-Banach-space}
    局所凸空間$X$の部分集合$A$が生成する部分空間を$Y$とする.
    \begin{enumerate}
        \item $A^\perp=Y^\circ$である.
        \item $A$が$X$上線形稠密である($\o{Y}=X$)ことと,$A^{\perp\perp}=X$は同値.
    \end{enumerate}
\end{proposition}
\begin{proof}\mbox{}
    \begin{enumerate}
        \item a
        \item 系\ref{cor-dense-subspace}と同様に証明できるはず.
    \end{enumerate}
\end{proof}

\begin{definition}[bidual space, double-dual embedding, reflexive]
    $X$をノルム空間とすると,$X^*,X^{**}$はBanach空間である.
    \begin{enumerate}
        \item $i:X\to X^{**}$を$i(x)(\varphi)=\varphi(x)$と定めると,これは単射線型作用素である.
        \item すると$i$はノルム減少作用素で特に有界であることはすぐにわかるが,実は等長写像を定める(系\ref{cor-Hahn-Banach}).
        よって,$i$を埋め込みとして,$X$を$X^{**}$の部分空間と同一視し,$X$がBanach空間でない場合は,$i(X)\subset X^{**}$の閉包を取ることで完備化できる.
        \item $i$が全射でもある(したがって$i$はBanの同型である)とき,Banach空間$X$を\textbf{回帰的}という.
    \end{enumerate}
\end{definition}
\begin{proof}
    $i$が単射であるのは明らか.任意の$x\in X$について,$\norm{i(x)}=\sup_{\varphi\in B^*}\varphi(x)\le\norm{x}$であるが,系\ref{cor-Hahn-Banach}より,$\varphi(x)=x$を満たす$\varphi\in B^*$が取れるから,実際$\norm{i(x)}=\norm{x}$とわかる.
\end{proof}
\begin{example}\mbox{}
    \begin{enumerate}
        \item 有限次元線型空間は回帰的である.
        \item $L^p$空間は$1<p<\infty$のとき,回帰的である.
        \item Rieszの表現定理より,任意のHilbert空間は回帰的である.
    \end{enumerate}
\end{example}

\subsection{回帰的Banach空間}

\begin{theorem}\label{thm-characterization-of-reflexive-Banach-spaces}
    Banach空間$X$について,次の2条件は同値.
    \begin{enumerate}
        \item $X$は回帰的である.
        \item ノルム閉単位球$B$は,$\sigma(X,X^*)$-位相についてコンパクト.
    \end{enumerate}
\end{theorem}

\begin{corollary}\mbox{}
    \begin{enumerate}
        \item $X$が回帰的であるとき,任意の有界線型作用素$E\to X,X\to F$について,合成$E\to F$は弱コンパクトである.
        \item (Davis-Figiel-Johnson-Pelczynski 74) 任意の弱コンパクトな有界線型作用素$E\to F$について,ある回帰的なBanach空間$X$が存在してこれに沿って分解する.
    \end{enumerate}
\end{corollary}

\subsection{双対ペア}

\begin{tcolorbox}[colframe=ForestGreen, colback=ForestGreen!10!white,breakable,colbacktitle=ForestGreen!40!white,coltitle=black,fonttitle=\bfseries\sffamily,
title=]
    2つの線型空間上のペアリングと呼ばれる双線型汎函数を与え,これを用いて互いに他へ位相を流入させることを考える.
    そのとき現れる有限次元的性格を持つ位相のうち,よく使われるのが弱$*$位相とMackey位相である.
    ペアリングはHilbert空間における射影を,内積がない空間でも模倣したものになる.
    その内積がある空間では直交補空間$X^\perp$に当たる概念が零化空間で,これを支える道具が極集合である.零化空間は$A^\perp=\brac{A}^\circ$と特徴づけられる.\ref{prop-dense-subspace-of-Banach-space}.
\end{tcolorbox}

\begin{definition}[algebraic duality / dual pair, duality]\mbox{}
    \begin{enumerate}
        \item 2つの線型空間$X,Y$が\textbf{(代数的)双対}または\textbf{双対ペア}であるとは,双線型写像$\brac{-,-}:X\times Y\to\bF$であって,部分空間$\brac{-,Y}:=\Brace{b(-,y)\in X^*\mid y\in Y}\subset X^*$は$X$上の点を分離し,$\brac{X,-}\in Y^*$は$Y$上の点を分離するようなものが存在することをいう.\footnote{これは$\forall_{x\in X\setminus\{0\}}\;\exists_{y\in Y}\;\brac{x,y}\ne 0$に同値で,双線型形式が非退化であるともいう.pairing bilinear formという.}
        \item $X,Y$がノルム空間でもあり,さらに$\brac{-,Y}\subset X^*,\brac{X,-}\subset Y^*$を満たす(すなわち有界になる)とき,単に\textbf{双対}または\textbf{双対ペア}という.
        \item $X$において,任意の$y\in Y$を用いて$p_y(-):=\abs{\brac{-,y}}$とおけば,これは$X$上の半ノルムを定める.族$(p_y)_{y\in Y}$が$X$上に定める始位相を$\sigma(X,Y)$位相という.同様に,$Y$上にも$\sigma(Y,X)$位相が考えられる.
        すなわち,$\sigma(X,Y)$位相とは,$Y$の元を(双対を通じて)$X$上の線型汎関数とみなしたとき,これらを連続にする最弱の位相である.
    \end{enumerate}
\end{definition}

\begin{example}[自然なペアリング]\mbox{}\label{exp-dual-pair}
    \begin{enumerate}
        \item 組$(X,X^*)$は双対ペアである.双線型形式は$\brac{x,\varphi}=\varphi(x)$で与えられる.
        \item 組$(X^{**},X^*)$も双対ペアである.双線型形式は$\brac{z,\varphi}=z(\varphi)$で与えられる.
    \end{enumerate}
    $E$における$\sigma(E,E^*)$位相も,$E^*$における$\sigma(E^*,E^{**})$位相も\textbf{弱位相}と呼ぶ.
    一方で,$\sigma(E^*,E)$による$E^*$上の位相を\textbf{弱$*$-位相}\ref{def-weak-star-topology}という.
    埋め込み$i:X\mono X^{**}$の存在より,弱$*$位相は弱位相より弱い.
    が,いずれの場合も局所凸である.
\end{example}

\subsection{極集合}

\begin{tcolorbox}[colframe=ForestGreen, colback=ForestGreen!10!white,breakable,colbacktitle=ForestGreen!40!white,coltitle=black,fonttitle=\bfseries\sffamily,
title=]
    一般の位相線形空間でも,双対ペアがあれば,Hilbert空間のように扱うことで幾何学的な考察が可能になる.
    これは普遍的な威力を持つ強力な手法である.
    直交性の概念よりも,凸錐が重要な位置を占める.
\end{tcolorbox}

\begin{definition}[real polar, bipolar]\label{def-polar}
    双対ペア$(X,Y)$と部分集合$A\subset X$に対して,
    \begin{enumerate}
        \item 対応する$Y$の部分集合$\Brace{y\in Y\mid \forall_{x\in A}\;\Re(x,y)\ge-1}$を$A$の\textbf{極集合}といい,$A^\circ$または$A^r$または$P(A)$で表す.
        \item $A^\circ$の極集合を$A^{\circ\circ}$と表し,\textbf{双極集合}という.
    \end{enumerate}
\end{definition}
\begin{remark}
    極集合の定義に単位円板$\sup_{x\in X}\abs{\brac{x,y}}\le1$を用いることもあり,この時をabsolute polarという.$\partial BA=A$を満たす集合$A$(これは$A$が均衡集合であることに同値\footnote{\url{https://en.wikipedia.org/wiki/Polar_set}})については両定義は一致するが,そうでない場合は違うが,その後の議論では同種の命題が成り立つことが知られている.
\end{remark}

\begin{definition}[convex cone]
    順序体$\R$上の局所凸線型空間内の集合$A$について,
    \begin{enumerate}
        \item $A$が錐であるとは,$\R_+A\subset A$を満たすことをいう.
        \item $A$が\textbf{凸錐}であるとは,正係数の線型結合に閉じていることをいう:$A+A\subset A,\R_+A\subset A$.
        \item 凸錐$A$が$A\cap -A=0$を満たすとき,$0$に頂点を持つという.
    \end{enumerate}
\end{definition}

\begin{lemma}\mbox{}\label{lemma-polar}
    \begin{enumerate}
        \item $A^\circ$は凸で,$\sigma(Y,X)$-閉である.
        \item $A$が絶対凸であるときは$A^\circ$も絶対凸であり,$A^\circ=\Brace{y\in Y\mid\forall_{x\in A}\;\abs{\brac{x,y}}\le 1}$と表せる.
        \item $A$が部分空間であるときは$A^\circ$も$Y$の部分空間であり,$A^\circ=\Brace{y\in Y\mid\forall_{x\in A}\;\brac{x,y}=0}$と表せる.
        \item $A$が$0$に頂点を持つ凸錐であるときは$A^\circ$も$0$に頂点を持つ凸錐であり,$A^\circ=\Brace{y\in Y\mid\forall_{x\in A}\;\brac{x,y}\ge0}$と表せる.
    \end{enumerate}
\end{lemma}

\begin{theorem}[双極定理]\label{thm-polar-theorem}
    $(X,Y)$を双対ペアとする.部分集合$A\subset X$について,双極集合$A^{\circ\circ}$は$A\cup\{0\}$の$\sigma(X,Y)$-閉凸包である.
\end{theorem}

\begin{corollary}
    局所凸空間$X$は,再双対空間$X^{**}$上で,$\sigma(X^{**},X^*)$-稠密である.
\end{corollary}

\subsection{随伴作用素}

\begin{tcolorbox}[colframe=ForestGreen, colback=ForestGreen!10!white,breakable,colbacktitle=ForestGreen!40!white,coltitle=black,fonttitle=\bfseries\sffamily,
title=]
    随伴とは,標準的な双対双線型形式$\ev_(-)$を介して,互いに相等する関係にある元をいう.
\end{tcolorbox}

\begin{definition}[adjoint operator]
    ノルム空間の射$T\in B(X,Y)$に対して,\textbf{随伴作用素}$T^*:Y^*\to X^*$を$\brac{x,T^*\varphi}=\brac{Tx,\varphi}$で定まる.
\end{definition}

\begin{lemma}[$*$-作用素の関手性]
    $S\in B(X,Y),R\in B(Y,Z),\al\in\bF$について,
    \begin{enumerate}
        \item $(\al T+S)^*=\al T^*+S^*$.
        \item $(RT)^*=T^*R^*$.
    \end{enumerate}
\end{lemma}

\begin{proposition}[$*$-作用素の等長性]
    $T\in B(X,Y)$をノルム空間の射とすると,その随伴作用素も$T^*\in B(Y^*,X^*)$であり,$\norm{T^*}=\norm{T}$.
\end{proposition}

\begin{proposition}[随伴であるための十分条件]
    $X,Y$をBanach空間,$T:X\to Y,S:Y^*\to X^*$を任意の作用素であり,\\
    $\forall_{x\in X,\varphi\in Y^*}\;\brac{Tx,\varphi}=\brac{x,S\varphi}$を満たすとする.
    このとき,$S,T$はいずれも有界で,$S=T^*$である.
\end{proposition}
\begin{remarks}
    これは,非有界な作用素を,その随伴を通じて調べる試みの失敗も意味する.
    非有界な自己共役作用素の理論は,必ず部分写像(not everywhere defined)の考え方がついてくることがわかる.
\end{remarks}

\subsection{核}

\begin{proposition}
    線型汎函数$f:X\to\bF$について,
    \begin{enumerate}
        \item $f\ne 0$ならば,$X\simeq\Ker f\oplus\bF$.
        \item $X$が局所凸とする.$f$が連続であることと,$\Ker f$が閉部分空間であることとは同値.
    \end{enumerate}
\end{proposition}
\begin{proof}\mbox{}
    \begin{enumerate}
        \item $f\ne 0$のとき,$f$は全射だから,$x\in f^{-1}(1)$が取れる.すると任意の$y\in X$について,$y=(y-f(y)x)+f(y)x\in\Ker f+\C x$である.実際,$f(y-f(y)x)=f(y)-f(y)f(x)=0$.
    \end{enumerate}
\end{proof}

\subsection{転置}

\begin{definition}[transpose]
    線型作用素$x:X\to Y$に対して,precomposition ${}^t\!x:=x^*:Y^*\to X^*;f\mapsto f\circ x$を\textbf{転置写像}という.
\end{definition}

\begin{lemma}
    $X,Y$をノルム空間とする.$\norm{{}^t\!x}=\norm{x}$.
\end{lemma}

\section{弱位相}

\begin{tcolorbox}[colframe=ForestGreen, colback=ForestGreen!10!white,breakable,colbacktitle=ForestGreen!40!white,coltitle=black,fonttitle=\bfseries\sffamily,
title=]
    作用素環の意味での弱位相/強位相とは,混用されるが全くの別物である.
    区別の意味で,こちらはBanach空間としての弱位相などと呼ばれる.

    $X^*$(の部分集合)が$X$に定める始位相を弱位相といい,$X^{**}$(の部分集合)が$X^*$に定める始位相を$*$-弱位相という.
\end{tcolorbox}

\subsection{局所凸性のセミノルム空間としての特徴付け}

\begin{tcolorbox}[colframe=ForestGreen, colback=ForestGreen!10!white,breakable,colbacktitle=ForestGreen!40!white,coltitle=black,fonttitle=\bfseries\sffamily,
title=]
    解析学における位相線型空間は,ほとんど局所凸である.
    そして局所凸空間は,セミノルム空間であるから,ノルム空間論の延長と捉えられる.
    こうして,位相空間論的な近傍系の議論ではなく,セミノルムから議論することが出来る.
    セミノルム空間の位相は,ノルム空間の位相はノルムを連続にする最弱位相である点を一般化する.
\end{tcolorbox}

\begin{definition}[topological vector space, locally convex]\mbox{}
    \begin{enumerate}
        \item $X$が\textbf{位相線型空間}であるとは,線型空間$X$に,演算$X\times Y\to X;(x,y)\mapsto x+y,\bF\times X\to X;(\al,x)\mapsto\al x$が連続になるような
        Hausdorff位相が定義されている空間をいう.\footnote{Hausdorffでない位相が定義されている場合は,$0$の近傍系について$N:=\cap_{U\in\O(x)}U$とおくと閉部分空間で,$E/N$を考えるとHausdorff性を満たす.}
        \item 位相線型空間$X$が\textbf{局所凸}であるとは,$0$の近傍フィルター$\O(0)$が凸集合からなる開基を持つことをいう.等質性$\O(x)=x+\O(0)$に注意.\footnote{平行移動が連続になるという設定は,連続群の理論の特殊化そのものである.}
        \item 位相線型空間$X$上の連続な線型汎関数全体からなる集合$X^*$は再び線型空間となり,これを双対空間という.$X$が局所凸である限り,Hahn-Banachの定理より,$X^*$は十分大きい.
    \end{enumerate}
\end{definition}

\begin{definition}[seminorm topology]\label{def-seminorm-topology}
    線型空間$X$とこれを分離するセミノルムの族$\F$を考える.すなわち,関数族$\F\times X\to\Map(X,\R);(m,y)\mapsto m(\cdot-y)$は$X$の点を分離する.
    これが$X$に定める始位相を,\textbf{$\F$が定めるセミノルム位相}とする.\footnote{一般に,擬距離の族によって定められた位相を持つ位相空間を,\textbf{ゲージ空間}という.}
    換言すれば,次のフィルター準基が各$\O(x)$に生成する位相である:
    \[\Brace{y\in X\mid\abs{m(y-z)-m(x-z)}<\ep,\;m\in\F,z\in Z,\ep>0}\]
    すると,$\R$はHausdorffであるから,セミノルム位相はHausdorffとなる\ref{prop-for-initial-topology-begin-Hausdorff}.
\end{definition}
\begin{remarks}
    セミノルムはそのままではHausdorffな位相を定めないから,このように族$\F$を用意して定義とする.
    $X$を分離する族$\F$が単元集合$\{m\}$であるためには,$m$がノルムであることが必要十分.
    任意の$\F$の元$m\in\F$が,$X$の汎関数$\varphi\in Y\subset X^*$の族$Y$を用いて$m=\abs{\varphi}$と表せるとき,このセミノルム位相を\textbf{$Y$が定める弱位相}$\sigma(X,Y)$という.
\end{remarks}

\begin{lemma}[セミノルム位相の性質]
    $\F$が定めるセミノルム空間$X$において,
    \begin{enumerate}
        \item ある有限集合$\{m_1,\cdots,m_n\}\subset\F$について,
        次のように表される集合全体が,近傍フィルター$\O(x)$の基底をなす:
        \[\bigcap_{k\in[n]}\Brace{y\in X\mid m_k(y-x)<\ep,\ep>0}\]
        \item ネット$(x_\lambda)$が収束することは,$\forall_{m\in\F}\;m(x_\lambda-x)\to0$に同値.
        \item $f:Y\to X$が連続であることは,任意の$Y$の収束ネット$(y_\lambda)$に対して,$\forall_{m\in\F}\;m(f(y_\lambda)-f(y))\to 0$が成り立つことに同値.
    \end{enumerate}
\end{lemma}
\begin{proof}\mbox{}
    \begin{enumerate}
        \item 特に$z=x$の場合を考えると,$\{y\in X\mid m(y-x)<\ep\}$は準基である.しかし,$m$の劣加法性(三角不等式)より,$m(y-x)<\ep$ならば$\abs{m(y-z)-m(x-z)}<\ep$が従う.
        よって準基としてはこの形のもののみを考えれば良いから,これらの有限交叉の全体は基底をなす.
        \item \ref{prop-net-in-initial-topology}より.
        \item \ref{cor-continuous-function-to-initial-topology}より.
    \end{enumerate}
\end{proof}

\begin{proposition}[局所凸性の特徴付け]\label{prop-characterization-of-locally-convex-spaces}
    位相線型空間$X$について,次の2条件は同値.
    \begin{enumerate}
        \item $X$は局所凸である.
        \item $X$の位相はセミノルム位相である.
    \end{enumerate}
\end{proposition}
\begin{proof}\mbox{}
    \begin{description}
        \item[(2)$\Rightarrow$(1)] 任意の$X$の収束ネット$x_\lambda\to x,y_\lambda\to y$について,$x_\lambda+y_\lambda\to x+y$を示せば良い.
    \end{description}
\end{proof}
\begin{remarks}
    この命題より,ノルム空間の理論の延長として局所凸空間を考えることができる.
    以降,線型空間$X$とその上の半ノルムの族$\F$との組$(X,\F)$を\textbf{局所凸空間}といい,混用する.
    実は開写像定理\ref{thm-open-mapping-theorem}の証明ですでに線形性と凸性との関係を使っている.
\end{remarks}

\subsection{弱位相を定める汎関数}

\begin{tcolorbox}[colframe=ForestGreen, colback=ForestGreen!10!white,breakable,colbacktitle=ForestGreen!40!white,coltitle=black,fonttitle=\bfseries\sffamily,
title=]
    命題\ref{prop-separating-space-of-functionals}は,$\sigma(X,Y)$-位相に関する一般理論を主張する.
    線型汎関数の空間$Y$を連続にする最弱な位相は,正確に$Y$のみの線型汎関数を連続にする.
    $Y$として評価写像を取れば,これは収束に関する言葉で特徴づけられることになる(弱位相の収束の特徴付け\ref{prop-characterization-of-convergence-in-wtopology},$w^*$-位相の収束の特徴付け\ref{lemma-characterization-of-convergence-in-w*topology}).
\end{tcolorbox}

\begin{lemma}
    $\varphi,\varphi_1,\cdots,\varphi_n\in X^*$を線型空間$X$上の汎関数とする.次の3条件は同値.
    \begin{enumerate}
        \item $\exists_{\Brace{\al_1,\cdots,\al_n}\subset\bF}\;\varphi=\sum_{k\in[n]}\al_k\varphi_k$.
        \item $\exists_{\al>0}\;\forall_{x\in X}\;\abs{\varphi(x)}\le\al\max_{k\in[n]}\abs{\varphi_k(x)}$.
        \item $\cap_{k\in[n]}\Ker\varphi_k\subset\Ker\varphi$.
    \end{enumerate}
\end{lemma}

\begin{proposition}[$\sigma(X,Y)$-位相の一般理論]\label{prop-separating-space-of-functionals}
    線型空間$X$と,$X$の点を分離する線型汎関数のなす線型空間$Y$を考える.
    $X$に$Y$定める弱位相$\sigma(X,Y)$を入れて考える.
    このとき,線型汎関数$\varphi$がこの弱位相について連続ならば,$\varphi\in Y$である.
\end{proposition}
\begin{corollary}[symmetric version]
    $(X,Y)$は,双線型写像$\brac{-,-}$によって代数的双対組であるとする.すると,$\brac{-,Y},\brac{X,-}$はそれぞれ$X,Y$に弱位相を定める.
    これについて,$X,Y$は局所凸な位相空間になり,$X^*=Y,Y^*=X$を満たす.
\end{corollary}

\subsection{弱位相の性質}

\begin{tcolorbox}[colframe=ForestGreen, colback=ForestGreen!10!white,breakable,colbacktitle=ForestGreen!40!white,coltitle=black,fonttitle=\bfseries\sffamily,
    title=]
    これは,命題\ref{prop-separating-space-of-functionals}に始まる$\sigma(X,Y)$-位相の一般理論の一部であるが,弱位相の得意な点は,ノルム位相とは有限次元の場合を除いて一般的には異なるが,双対空間は等しく$X^*$になることである.
    凸集合について,閉性は同値になる.
\end{tcolorbox}

\begin{proposition}[弱収束は各点収束]\label{prop-characterization-of-convergence-in-wtopology}
    $X$上のネット$(x_\lambda)_{\lambda\in\Lambda}$について,次の2条件は同値.
    \begin{enumerate}
        \item $x$に弱位相$\sigma(X,X^*)$について収束する.
        \item $\forall_{\varphi\in X^*}\;\varphi(x_\lambda)\to\varphi(x)$.
    \end{enumerate}
\end{proposition}
\begin{remarks}
    これは$w^*$-収束の特徴付け\ref{lemma-characterization-of-convergence-in-w*topology}の特別な場合である.
\end{remarks}

\begin{proposition}\label{prop-closedness-of-convex-sets}
    凸集合$C\subset X$について,次の2条件は同値.
    \begin{enumerate}
        \item ノルム閉である.
        \item 弱閉である.
    \end{enumerate}
\end{proposition}
\begin{proof}
    (1)$\Rightarrow$(2)は明らかだから,(2)$\Rightarrow$(1)を示す.
    $x\in X\setminus C$を任意に取る.
    すると,$C$と互いに素な$x$中心の開球$B$が取れる.
    Hahn-Banachの分離定理\ref{thm-Hahn-Banach-separation-theorem}より,
    $U:=\Brace{y\in X\mid\Re\varphi(y)\ge t}$なる$C$の弱位相における閉近傍が見つかる.
    この補集合は,$x\in C$の開近傍である.よって,$X\setminus C$は弱位相について開いている.
\end{proof}
\begin{remarks}
    弱位相はノルム位相よりも開集合の数は同じか少ない.
    閉集合の数も同様であるが,凸集合については保たれる.
    この議論を一般化すると,\textbf{凸集合が閉であるという条件は,同じ双対を持つどの局所凸位相についても同値である}.
\end{remarks}

\subsection{絶対凸集合}

\begin{tcolorbox}[colframe=ForestGreen, colback=ForestGreen!10!white,breakable,colbacktitle=ForestGreen!40!white,coltitle=black,fonttitle=\bfseries\sffamily,
title=]
    集合$C$が定める計測関数
    \[m_C(x):=\inf\Brace{s>0\mid s^{-1}x\in C}\]
    が半ノルムになるために$C$に必要な条件が絶対凸性である.
\end{tcolorbox}

\begin{definition}[balanced, absolutely convex]
    $\bF$-線型空間$X$の部分集合$C$について,
    \begin{enumerate}
        \item $\forall_{\al\in\bF}\;\abs{\al}\le 1\Rightarrow\al C\subset C$を満たすとき,$C$を\textbf{均衡集合}または\textbf{円板}という.
        \item 均衡な凸集合を\textbf{絶対凸}という.
        \item $C$を含む絶対凸集合全体の共通部分を,絶対凸包という.
    \end{enumerate}
\end{definition}

\begin{lemma}[絶対凸性の特徴付け]
    $\bF$-線型空間$X$の部分集合$C$について,次の2条件は同値.
    \begin{enumerate}
        \item $C$は絶対凸である.
        \item $\forall_{x,y\in C}\;\forall_{\lambda_1,\lambda_2\in\bF}\;\abs{\lambda_1}+\abs{\lambda_2}\le 1\Rightarrow \lambda_1x+\lambda_2y\in C$.
    \end{enumerate}
\end{lemma}

\begin{lemma}[均衡性]
    $C$を均衡集合とする.
    \begin{enumerate}
        \item $C=-C$.
        \item $\forall_{r,\lambda\in\R\setminus\{0\}}\;\lambda x\in rC\Leftrightarrow x\in\frac{r}{\abs{\lambda}}C$.
        \item $C$が定める計量関数$m_C$について,$\forall_{\lambda\in\R}\;m_C(\lambda x)=\abs{\lambda}m_C(x)$.
    \end{enumerate}
\end{lemma}
\begin{proof}\mbox{}
    \begin{enumerate}
        \item $-C\subset C$.$a\ne0$のとき$ax\in C\Leftrightarrow x\in a^{-1}C$だから,$x\in C\Rightarrow -x\in C\Rightarrow x\in -C$より,$C\subset-C$も得る.
        \item (1)より,$\lambda x\in rC\Leftrightarrow\abs{\lambda}x\in rC\Leftrightarrow\frac{r}{\abs{\lambda}}C$.
    \end{enumerate}
\end{proof}

\subsection{計測関数とHahn-Banachの分離定理}\label{sub-Hahn-Banach-separation}

\begin{tcolorbox}[colframe=ForestGreen, colback=ForestGreen!10!white,breakable,colbacktitle=ForestGreen!40!white,coltitle=black,fonttitle=\bfseries\sffamily,
title=]
    Topにおける「正規性」にあたるものは,TVSでは
    「局所凸」である.\footnote{Among non-locally convex spaces, however, there are examples such that the only continuous linear functional is the constant map onto $0 \in k$.\url{https://ncatlab.org/nlab/show/linear+functional}}
    そして正規性では「閉集合を分離する関数」を考えたが,局所凸線型空間では,$0$の凸開近傍を考える.樽\ref{def-barrel}などの概念を参照.
\end{tcolorbox}

\begin{lemma}[gauge]
    $X$を位相線型空間,$C\osub X$を$0$における凸開近傍とする.
    関数$m:X\to\R_+$を
    \[m(x):=\inf\Brace{s>0\mid s^{-1}x\in C}\]
    と定めると,これはMinkowski汎関数で,$C=\Brace{x\in X\mid m(x)<1}$と表せる.
\end{lemma}
\begin{proof}\mbox{}
    \begin{description}
        \item[well-definedness] $C$は$0$の開近傍としたから,ある$N\in\N$が存在して,$\forall_{x\in X}\;\forall_{n\ge N}\;n^{-1}x\in C$.よって,$\forall_{x\in X}\;m(x)<\infty$であり,たしかに関数が定まる.なお,このことから,任意の$0$の近傍は併呑集合であることがわかる.
        \item[正斉次性] $\forall_{t\ge 0}\;m(tx)=\inf_{tx\in sC,s>0}s=\inf_{x\in t^{-1}sC,s<0}s=tm(x)$.
        \item[劣加法性] 任意の$x,y\in X$を取る.すると,ある正実数$s,t>0$が存在して,$s^{-1}x\in C$かつ$t^{-1}y\in C$.
        $C$の凸性より,
        \[\frac{s}{s+t}s^{-1}x+\frac{t}{s+t}t^{-1}y=(s+t)^{-1}(x+y)\in C\]
        であるから,$m$の定義より,$m(x+y)\le s+t$.$s,t$の取り方に依らないから,$m(x+y)\le m(x)+m(y)$が従う.
        \item[$C$の表現] $C$は開集合であるから,$x\in C$ならば,ある$\ep>0$が存在して,$(1+\ep)x\in C$.よって,$m(x)\le(1+\ep)^{-1}<1$.
        逆に,$x\in X$は$m(x)<1$を満たすならば,ある$s<1$について$s^{-1}x\in C$ということだから,$C$が$0$の近傍であることと凸性より,
        \[x=(1-s)0+s(s^{-1}x)\in C.\]
    \end{description}
\end{proof}
\begin{remarks}[absolutely convex]\label{remarks-seminorm-and-absolutely-convex-sets}
    このように定義された関数を,しばしば$C$の計測関数(gauge)という.
    $C$の凸性は$m$の劣加法性に対応している.
    \footnote{実は,$\Brace{s>0\mid s^{-1}x\in C}$が空でないという仮定の下で,$C$が凸であるということは$C$が吸収的であるということと同値\ref{lemma-characterizing-absorbant}.なお,任意の位相線型空間において,$0$の近傍は併呑である.}
    このときのMinkowski汎関数$m$が半ノルムにもなるため(正斉次性だけでなく,さらに強い同次性を満たすため)の$C$の条件を\textbf{絶対凸性}という.
    $C$が単位球であるとき,$m_C$はノルムを定める.

    実は,$X$上の半ノルムとこれを計測関数として定める絶対凸集合との間には全単射の対応が存在する!
\end{remarks}

\begin{lemma}
    $C$を絶対凸とする.
    $\Ker m_C=\bigcap_{t>0}tC$は,$C$に含まれる最大の線型部分空間である.
\end{lemma}

\begin{theorem}[Hahn-Banach separation theorem]\label{thm-Hahn-Banach-separation-theorem}
    $A,B$を非空で互いに素な,位相線型空間$X$における凸集合とする.
    $A$が開ならば,$\varphi\in X^*$と$t\in\R$が存在して,$\forall_{x\in A,y\in B}\;\Re\varphi(x)<t\le\Re\varphi(y)$を満たす.
\end{theorem}

\subsection{$w^*$-位相}

\begin{tcolorbox}[colframe=ForestGreen, colback=ForestGreen!10!white,breakable,colbacktitle=ForestGreen!40!white,coltitle=black,fonttitle=\bfseries\sffamily,
title=]
    汎関数の空間$X^*$には標準的な分離族$X\mono X^{**}$が考えられ,これについての始位相を$*$-弱位相という.
    これは,$x\in X$での評価写像$\ev_x$を連続にする最弱の位相である.
    $X^*$の作用素ノルムが定める位相より弱く,弱位相よりも弱いが,
    $w^*$-位相は各点収束位相として特徴付けられるため,自然な対象である.
    Alaogluの定理より,$\dim(X)<\infty$の場合のみノルム位相と$w^*$-位相は一致する.
    回帰的な場合のみ弱位相と$w^*$-位相は一致する.
\end{tcolorbox}

\begin{definition}[weak-star topology]\label{def-weak-star-topology}
    ノルム空間$X$は,\ref{exp-dual-pair}(2)の埋め込み$i:X\mono X^{**}$により,線型汎函数の空間$X^*$上の分離族とみれる.
    こうして$X^*$に引き起こされる$\sigma(X^*,X)$位相を,\textbf{$*$-弱位相}という.
    \footnote{すなわち,二重共役空間の部分空間$X\mono X^{**}$に属する写像を全て連続にする最弱の位相.}
    こうして,$X^*$は局所凸な位相線型空間となり,$X$を双対空間と同一視できる(命題\ref{prop-separating-space-of-functionals}).
\end{definition}
\begin{remarks}
    $*$-弱位相の始位相としての定義より,$X^*$にノルムが定める位相よりも,弱位相$\sigma(X^*,X^{**})$よりも弱い.
    このことから名前がついた.
\end{remarks}

\begin{lemma}[$*$-弱収束は各点収束]\label{lemma-characterization-of-convergence-in-w*topology}
    $*$-弱位相における収束は各点収束である.すなわち,$X^*$上のネット$(\varphi_\lambda)_{\lambda\in\Lambda}$について,次の2条件は同値.
    \begin{enumerate}
        \item $w^*$-位相について$\varphi$に収束する.
        \item $\forall_{x\in X}\;\varphi_\lambda(x)\to\varphi(x)$.
    \end{enumerate}
\end{lemma}

\begin{proposition}[$w^*$-閉な部分空間の表現]
    $X$をノルム空間とし,$Z$を双対空間$X^*$の$*$-閉な部分空間とする.
    このとき,任意の$\varphi\in X^*\setminus Z$について,$x\in Z^\perp$が存在して,$\brac{x,\varphi}\ne 0$を満たす.
\end{proposition}

\begin{corollary}
    $X^*$の任意の$w^*$-閉な部分空間について,あるノルム閉な部分空間$Y$が存在して,$Y^\perp$と表せる.
\end{corollary}

\subsection{Hahn-Banachの応用}

\begin{corollary}
    [\cite{John Conway}IV.3.14]
    $X$を局所凸空間,$Y$をその部分空間とする.次の2条件は同値.
    \begin{enumerate}
        \item $Y$は$X$上稠密である.
        \item $X$上の連続な線型汎函数であって$Y$上で消えるものは零関数に限る.
    \end{enumerate}
\end{corollary}
\begin{remarks}
    一致の定理はこの特別な場合?
\end{remarks}

\begin{corollary}[双対空間の稠密部分空間の特徴付け]\label{cor-separating-subspace-is-dense}
    部分空間$Y\subset X^*$をについて,次の2条件は同値.
    \begin{enumerate}
        \item Banach空間$X$の分離族である.
        \item $Y$は$X^*$上$w^*$-稠密である.
    \end{enumerate}
\end{corollary}
\begin{proof}
    $X^*$上$w^*$-稠密でないと仮定すると,$Y$上で$0$だが,恒等的に$0$ではないような$w^*$-連続な線型汎関数が存在することになるが,
    $w^*$-位相を備えた空間$X^*$上で連続な汎関数は$(\ev_x)_{x\in X^*}$に限る.
\end{proof}

\subsection{随伴と$w^*$-位相}

\begin{tcolorbox}[colframe=ForestGreen, colback=ForestGreen!10!white,breakable,colbacktitle=ForestGreen!40!white,coltitle=black,fonttitle=\bfseries\sffamily,
title=]
    随伴とは$\sigma(X,X^*)$-位相に関する対応であるから,$\sigma(X^*,X)$の言葉で特徴づけられるのは自然なことである.
\end{tcolorbox}

\begin{proposition}[随伴の特徴付け]\mbox{}
    \begin{enumerate}
        \item Banach空間$X,Y$間の作用素$T\in B(X,Y)$について,その随伴$T^*:Y^*\to X^*$は$w^*$-連続である.
        \item 任意の$w^*$-連続な作用素$S:Y^*\to X^*$\footnote{有界性,すなわち$S\in B(Y^*,X^*)$は仮定しない}について,ある$T\in B(X,Y)$が存在して,$S=T^*$と表せる.特に,$S\in B(Y^*,X^*)$.
    \end{enumerate}
\end{proposition}

\begin{proposition}\label{prop-標準写像の随伴}
    ノルム空間$X$の閉部分空間$Y$と,包含写像$I:Y\to X$と商写像$Q:X\to X/Y$について,
    \begin{enumerate}
        \item 随伴$Q^*$と包含写像$J:Y^\perp\to X^*$とを同一視できる.
        \item 随伴$I^*$と商写像$R:X^*\to X^*/Y^\perp$とを同一視できる.
    \end{enumerate}
\end{proposition}

\section{$w^*$-コンパクト}

\begin{tcolorbox}[colframe=ForestGreen, colback=ForestGreen!10!white,breakable,colbacktitle=ForestGreen!40!white,coltitle=black,fonttitle=\bfseries\sffamily,
title=この位相が,無限をコンパクトに翻訳し,取扱可能にしてくれる.]
    Banach空間上の汎関数のなす空間=双対空間(例えば測度の空間など)の解析にあたって,最も大事な位相である$w^*$-位相を考える.
    コンパクト集合がたくさんあることが良い.
    特にコンパクトな凸集合には強力な一般論があり,基本的な対象となる.
    「これにより,コンパクト性を通じて無限の世界が私たちの手元に届けられるようになった.」\cite{作用素環}
\end{tcolorbox}

\subsection{Alaogluの定理}

\begin{tcolorbox}[colframe=ForestGreen, colback=ForestGreen!10!white,breakable,colbacktitle=ForestGreen!40!white,coltitle=black,fonttitle=\bfseries\sffamily,
title=]
    実は,単位閉球が$\sigma(X^*,X)$-コンパクトであることは,$w^*$-位相による$X^*$の双対空間が$X$になることに起因するから,\footnote{凸集合についてはノルム閉と弱閉が同値である議論と並行だろう.}
    ノルム空間の単位閉球がコンパクトになることは,そのノルム空間が回帰的であることに同値.
\end{tcolorbox}

\begin{notation}
    ノルム空間の双対空間$X^*$の閉単位球を$B^*$と表す.
\end{notation}

\begin{lemma}
    ノルム空間$X$について,双対空間$X^*$の単位閉球$B^*$は$w^*$-閉集合でもあるが,一般に$0$の$w^*$-近傍であるとは限らない.
\end{lemma}

\begin{theorem}[Alaoglu's theorem]\label{thm-Alaoglu}
    ノルム空間$X$について,双対空間$X^*$の単位閉球$B^*$は$w^*$-コンパクトである.
\end{theorem}
\begin{remarks}[Bourbaki-Alaoglu]
    実は一般に,$X$における$0$の絶対凸近傍$U$の極集合$U^\circ$が$\sigma(X^*,X)$-コンパクトになる.
    これはBourbaki-Alaogluの定理という.
\end{remarks}

\subsection{Krein-Milmanの定理}

\begin{tcolorbox}[colframe=ForestGreen, colback=ForestGreen!10!white,breakable,colbacktitle=ForestGreen!40!white,coltitle=black,fonttitle=\bfseries\sffamily,
title=コンパクト凸集合の代数的特徴付けを与える]
    凸集合には,凸結合(convex conbination)を線型結合のようなものだと思うと「基底」なるべき概念があり,それらの情報だけで全体の形を復元できる.
    極点に当たる関数は大抵特殊な振る舞いをするので,取っ掛かりになる.
    その後,凸結合に対する安定性(これは線形性が十分条件となることが注意)と連続性を確認すれば,目標の凸集合上での成立を示せる.
    実際,デルタ分布だけでなく,Cauchyの積分表示が対象とする関数クラス,ユニタリ行列,直交行列はすべてある凸集合の極点として特徴付けられる.
\end{tcolorbox}

\begin{definition}[face, extreme point, extremal boundary]
    線型空間$X$の凸集合$C$について,
    \begin{enumerate}
        \item $C$の\textbf{面}とは,凸な部分集合$F\subset C$であって,$\forall_{\lambda\in(0,1),x,y\in C}\;\lambda x+(1-\lambda)y\in F\Rightarrow x\in F\land y\in F$を満たすものをいう.
        \item 特に,一点集合からなる面を,\textbf{極点}または\textbf{端点}という.すなわち,$C$の異なる2点を結ぶ線分上の点として表せない点である.
        \item $C$の極点全体からなる集合を,\textbf{極境界}といい,$\partial C$や$\Ex(C)$で表す.
    \end{enumerate}
\end{definition}

\begin{proposition}[極点の特徴付け]
    線型空間$X$の凸集合$K$と点$p\in K$について,次の5条件は同値.
    \begin{enumerate}
        \item $p\in K$は極点である.
        \item $K\setminus\{p\}$は凸集合である.
        \item $p$は$K$内の非退化な開線分上の点として表せない.
        \item $\forall_{x\in X}\;p+x\in K\land p-x\in K\Rightarrow x=0$.
        \item $\{p\}$は$K$の面である.
    \end{enumerate}
\end{proposition}

\begin{theorem}[Krein-Milman]
    線型空間$X$に,双対空間$X^*$が定める弱位相を備えて考える.
    コンパクトな凸集合$C\subset X$について,極境界$\partial C$の凸包は$C$上弱稠密である:$C=\o{\Conv}(\Ex(C))$.
\end{theorem}

\begin{example}[凸集合の極点]\mbox{}
    \begin{enumerate}
        \item コンパクトハウスドルフ空間上の連続関数の空間$C(X)$の単位球は,
        $X$が無限集合のとき,
        $\infty$-ノルムについてコンパクトでない\ref{prop-unit-ball-in-normed-space}.
        しかし一般に
        $\forall_{x\in X}\;\abs{f(x)}=1$を満たす関数$f\in C(X)$が極点となるが,
        $\bF=\R$で$X$が連結ならば,これはただ2つで,$F=\C$ならば,これはユニタリ関数を意味し,ノルム閉単位球の中で一様に稠密である.
        ($C(X)$は回帰的でないから,ノルム閉単位球は弱コンパクトではないのに!?)
        \item $L^1(X)\;(X\subset\R^n)$について,単位球はコンパクトでなく,極点を持たない.
        \item $L^p(X)\;(p\in(1,\infty))$は回帰的だから,閉単位球は$w^*$-コンパクトである.極境界は位相境界に一致する.これは$p$-ノルムが「一様に丸」くて尖った点のない球を与えるという幾何的消息を示唆している.
        \item $L^\infty(X)$の単位球の極境界は$\Brace{f\in \L^\infty(X)\mid\abs{f(x)}=1\;\ae}$.
        \item 単調増加関数の集合は凸で,各点収束位相についてコンパクトである.極点は$\Im f\subset\{0,1\}$をみたす関数.
        \item 開集合$\Om\subset\C$上の正則関数で$\norm{f}_\infty\le1$を満たすものがなす凸集合$\O(\Om)$の極点は$f(z)=\frac{\al}{z-z_0}\;(z_0\notin\Om,\abs{\al}=d(z_0,\Om))$と表せる関数.これはCauchyの積分公式で対称となる関数のクラスにほかならない!
        \item $M_n(\bF)=B(\bF^n)$に$\bF^n$の$2$-ノルムから定まる作用素ノルムを考え,これについてノルム単位球は$\bF^n$の等長同型.すなわち,$\bF=\R$のときは直交行列で,$\bF=\C$のときはユニタリ行列.
        \item $B(H)\;(\dim H\ge\aleph_0)$は回帰的で($B^1(H)$の双対空間と考えられる),単位球は$w^*$-コンパクトである($\sigma$-弱位相という).その極点は$T^*T=I$を満たす等長同型か,$TT^*=I$を満たす余等長同型である.
        また,この極点の凸包はそのまま閉単位球となる.実際,ユニタリ作用素の凸包ですでに開単位球を含む.
        \item $B(H)_\sa$は実Banach空間で,$\sigma$-弱位相の双対空間である.この極点は対称変換$S=S^*,S^2=I$である.
        \item $B(H)_+$は閉錐で,$B(H)$を生成する.この単位球の極点は直交射影$P=P^*,P^2=P$である.このクラスの作用素に対する結果がスペクトル定理である.
    \end{enumerate}
\end{example}

\begin{corollary}\label{cor-predual-of-c0}
    $X^*\simeq c_0$となるようなBanach空間$X$は存在しない.
\end{corollary}
\begin{remark}
    他にも,$L^1([0,1])$もpredualを持たない.
\end{remark}

\subsection{確率測度のなす部分空間}

\begin{tcolorbox}[colframe=ForestGreen, colback=ForestGreen!10!white,breakable,colbacktitle=ForestGreen!40!white,coltitle=black,fonttitle=\bfseries\sffamily,
title=]
    例として,確率測度がBanach代数の双対空間$(C(X))^*$の中でなす部分空間としての性質を見る.
    測度を,関数の双対空間の元と考える.つまり,セミパラ理論でも見た,関数への作用素としての測度の理解である.
    これは,積分を有界線形作用素とみる世界観の始まりでもある.「加法的集合関数」なる概念は時代遅れである.
\end{tcolorbox}

\begin{proposition}
    $\infty$-ノルムを備えたコンパクトハウスドルフ空間上のBanach代数$C(X)$を考える.
    $C(X)$の双対空間を$M(X)$,$P(X):=\Brace{\mu\in M(X)\mid\norm{\mu}\le 1,\mu(1)=1}$を確率測度のなす部分空間とする.
    \begin{enumerate}
        \item $P(X)$は$M(X)$の凸集合である.
        \item $P(X)$は$w^*$-コンパクトである.
        \item $P(X)$の極点はDirac測度$\delta_x\;(x\in X),\forall_{f\in C(X)}\;\delta_x(f)=f(x)$である.
    \end{enumerate}
\end{proposition}
\begin{proof}\mbox{}
    \begin{enumerate}
        \item $\mu_1,\mu_2\in P(X)$と$\lambda\in(0,1)$を任意に取ると,$\lambda\mu_1+(1-\lambda)\mu_2\in P(X)$がわかる.
        \item 線型汎函数$\ev_1:M(X)\to\bF;\mu\mapsto\mu(1)$は$w^*$-位相について連続である
        $M(X)=(C(X))^*$の閉単位球$B^*$は$w^*$-コンパクト\ref{thm-Alaoglu}である.
    \end{enumerate}
\end{proof}
\begin{remarks}
    コンパクトハウスドルフ空間$X$上の確率測度は,
    $(C(X))^*$上において,
    有限な台を持つ測度(=Dirac測度の凸結合)によって
    各点近似(各点収束の位相で近似)が出来る.
    これは経験過程に対する大数の法則である.
\end{remarks}

\subsection{Krein-Smulianの定理}

\begin{tcolorbox}[colframe=ForestGreen, colback=ForestGreen!10!white,breakable,colbacktitle=ForestGreen!40!white,coltitle=black,fonttitle=\bfseries\sffamily,
title=$w^*$-閉性の特徴付けは,単位閉球の言葉によってなされる.]
    ある凸集合が弱$*$-閉であることを示したいとき,弱位相での特徴付け\ref{prop-closedness-of-convex-sets}は弱$*$-位相が弱位相より真に弱い時には失敗するから(ノルム閉だが$w^*$-閉でない集合が存在する:$\Ker x^{**}\;(x^{**}\in X^{**}\setminus X)$など),
    別の特徴付けを探したい.
    Krein-Smulianの定理は,弱$*$-位相一般について成り立つ.
\end{tcolorbox}

\begin{theorem}[Krein-Smulian theorem]
    $X$をBanach空間,$B^*$を双対空間$X^*$の閉単位球とする.
    凸集合$C\subset X^*$について,次の2条件は同値.
    \begin{enumerate}
        \item $\forall_{r>0}\;rB^*\cap C$は$w^*$-閉である,すなわち,$w^*$-コンパクトである.
        \item $C$は$w^*$-閉である.
    \end{enumerate}
\end{theorem}
\begin{proof}\mbox{}
    \begin{description}
        \item[方針] (2)$\Rightarrow$(1)は明らかだから,(1)$\Rightarrow$(2)を示す.
        $\forall_{r>0}\;rB^*\cap C$が$w^*$-閉であるとき,$C$はノルム閉でもある.\footnote{実際,$C$の任意のノルム収束列$(x_n)$を取ると,これは有界列だから,$\exists_{r>0}\;\{x_n\}\subset rB^*\cap C$.$rB^*\cap C$は$w^*$-閉という仮定よりノルム閉でもあるから,$(x_n)$は$C$上でノルム収束する.}
        よって,$\varphi\in X^*\setminus C$のとき,$0\notin C-\varphi$だから,ある$r>0$が存在して$rB^*\cap(C-\varphi)=\emptyset$が成り立つ.
        そこで,$r=1$とし,$C$を$C-\varphi$として取り直すことで,$B^*\cap C=\emptyset$なる状況について,必ず$0\notin\o{C}$であることを示せば良い.
        \item[主張] $X^*$の閉球は,極集合として$(rB)^\circ=r^{-1}B^*$と表せる.
        
        $r^{-1}B^*\subset(rB)^\circ$は明らか.実際,任意の$\varphi\in r^{-1}B^*$について,$\norm{\varphi}\le r^{-1}$より,任意の$x\in rB$に対して,$\abs{\varphi(x)}\le\norm{x}\norm{\varphi}\le1$だから,特に$\Re\varphi(x)\ge-1$.よって,$\varphi\in(rB)^\circ$.

        $r^{-1}B^*\supset(rB)^\circ$について,任意に$\varphi\notin r^{-1}B^*$を取る.
        $w^*$-位相を備えた空間$X^*$の双対空間は$X$であることに注意すると,$\varphi$の$r^{-1}B^*$と交わらない任意の開球と$r^{-1}B^*$とについて,
        Hahn-Banachの分離定理\ref{thm-Hahn-Banach-separation-theorem}より,$x\in X$と$t\in\R$が存在して,
        \[\Re\brac{x,\varphi}<t\le\Re\brac{x,r^{-1}B^*}\]
        を満たす.ここで,$\Re\brac{x,r^{-1}B^*}=[-r^{-1}\norm{x},r^{-1}\norm{x}]\subset\R$だから(作用素ノルムの定義と系\ref{cor-Hahn-Banach}より),
        $x\in rB$を$\norm{x}=r$を満たすように上式を正規化することで,$\Re\brac{x,\varphi}<t\le-1$,特に$\varphi\notin(rB)^\circ$を得る.
        \item[有界線型作用素の構成]
        $X$の有限部分集合の列$(F_n)$を次のように定める:
        \begin{enumerate}
            \item $F_1=\{0\}$.
            \item $D:=(n+1)B^*\cap C\cap P(F_1)\cap\cdots\cap P(F_n)=\emptyset$とおくと,$w^*$-コンパクトな凸集合の有限共通部分だから$D$も$w^*$-コンパクトな凸集合で\ref{lemma-polar},帰納的に$D\cap nB^*=\emptyset$を満たす($n=1$のときは$C\cap B^*=\emptyset$は仮定した).
            すなわち,$\emptyset=D\cap P(n^{-1}B)=\cap_{x\in n^{-1}B}D\cap P(\{x\})$であるから,$D$のコンパクト性より,$D\cap P(F_{n+1})=\emptyset$を満たす有限部分集合$F_{n+1}\subset n^{-1}B$が取れる.
        \end{enumerate}
        こうして構成した$(F_n)$は
        \[F_n\subset(n-1)^{-1}B\quad nB^*\cap C\cap P(F_1)\cap\cdots\cap P(F_n)=\emptyset\]
        を満たす.
        したがって,$\cup_{n\in\N}F_n=\{x_n\}$は$0$に収束する点列と思える.
        こうして,次の写像\ref{exp-direct-sum-of-norm-spaces}
        \[\xymatrix@R-2pc{
            T:X^*\ar[r]&c_0(\N)\\
            \rotatebox[origin=c]{90}{$\in$}&\rotatebox[origin=c]{90}{$\in$}\\
            \varphi\ar@{|->}[r]&(\brac{x_n,\varphi})_{n\in\N}
        }\]
        は有界線型作用素となる.
        \item[数列の空間への対応を用いて証明]
        $\varphi\in C$について,十分大きな$m\in\N$についても$mB^*\cap\{\varphi\}\cap P(\{x_n\mid n\in\N\})=\emptyset$より,$\inf_{n\in\N}\Re\brac{x_n,\varphi}\le-1$,よって特に$\norm{T\varphi}_\infty\ge1$.
        すなわち,像$T(C)$も凸であるが,$c_0(\N)$における単位開球$B_0$と互いに素である.
        よってHahn-Banachの分離定理\ref{thm-Hahn-Banach-separation-theorem}より,ある$\lambda=(\lambda_n)\in (c_0(\N))^*=l^1$と$t\in\R$が存在して,
        \[\Re\brac{B_0,\lambda}<t\le\Re\brac{T(C),\lambda}\]
        を満たす.上式を$\norm{\lambda}_1=1$を満たすように正規化すると,$1\le t$をみたすように$t$を取れる.
        よって,$x:=\sum_{n\in\N}\lambda_nx_n$とおくと$x\in X$で,任意の$\varphi\in C$について
        \[\Re\brac{x,\varphi}=\sum_{n\in\N}\Re\brac{\lambda_nx_n,\varphi}=\Re\brac{T\varphi,\lambda}\ge t\ge1\footnote{$X$と$X^*$上の双線型形式から,$l^1$上の内積に写している.}\]
        が成り立つ.特に,$0\notin C$で,$C$の$w^*$-閉包にも含まれないことがわかる.
    \end{description}
\end{proof}
\begin{remarks}
    Banach空間$X$のノルム位相と弱位相について,凸集合$A\subset X$が弱閉であることと$\forall_{r>0}\;A\cap rB$が弱閉であることは同値.なぜならば,$\Rightarrow$:凸集合$rB$はノルム閉であるから弱閉でもある\ref{prop-closedness-of-convex-sets}.$\Leftarrow$:$A$がノルム閉であることを示せば十分だが,$A$のノルム収束列は有界で,あるノルム閉集合$\exists_{r>0}\;A\cap rB$に含まれるから,結局$A$はノルム閉.
    $X$が回帰的である場合は,弱$*$-位相は弱位相と同じ強さだから,全く同様の事実が成り立つ.しかし,弱$*$-位相が弱い場合は?
\end{remarks}

\begin{corollary}
    部分空間$Z\subset X^*$(任意の凸集合$Z$)について,次の2条件は同値.
    \begin{enumerate}
        \item $Z$は$w^*$-閉である.
        \item $Z\cap B^*$は$w^*$-閉である.
    \end{enumerate}
    特に,$B^*$は$w^*$-コンパクト\ref{thm-Alaoglu}だから,$Z\cap B^*$も$w^*$-閉ならば$w^*$-コンパクトである.\footnote{コンパクト集合と閉集合との共通部分はコンパクト.コンパクト性の特徴付けから示す.}
\end{corollary}
\begin{proof}
    (1)$\Rightarrow$(2)は明らか.(2)$\Rightarrow$(1)を考える.
    部分空間$Z$は凸集合である.
    $Z\cap B^*$が$w^*$-閉ならば,$rB^*\;(r>0)$も$w^*$-閉だから,$Z\cap rB^*$も$w^*$-閉.よってKrein-Smulianの定理より$Z$は$w^*$-閉.
\end{proof}

\begin{corollary}\label{cor-characterization-of-weak-star-continuousness}
    $X^*$上の汎関数$x\in X^{**}$について,次の2条件は同値.
    \begin{enumerate}
        \item $w^*$-連続である.すなわち,$x\in X$.\footnote{$w^*$位相の定義\ref{def-weak-star-topology}と命題\ref{prop-separating-space-of-functionals}より.}
        \item 閉単位球への制限$x|_{B^*}$が$w^*$-連続である.
    \end{enumerate}
\end{corollary}
\begin{proof}
    (1)$\Rightarrow$(2)は明らかだから(2)$\Rightarrow$(1)を示す.
    $x:X^*\to\bF$が零であるとき,(2)$\Rightarrow$(1)は成り立つ.
    任意の凸な閉集合$E\subset\bF$に対して,$(x|_{B^*})^{-1}(E)=x^{-1}(E)\cap B^*$は$w^*$-閉集合だから,$x^{-1}(E)$も$w^*$-閉集合である.
\end{proof}

\subsection{可分性と距離化可能性}

\begin{tcolorbox}[colframe=ForestGreen, colback=ForestGreen!10!white,breakable,colbacktitle=ForestGreen!40!white,coltitle=black,fonttitle=\bfseries\sffamily,
title=]
    弱位相と$w^*$-位相は決して距離化可能でないが,有界な集合に限ると距離化可能たり得る\cite{John Conway}V.5.1.
\end{tcolorbox}

\begin{theorem}[単位閉球の距離化可能性]\label{thm-metrizability-of-ball}
    $X$をBanach空間とする.次の2条件は同値.
    \begin{enumerate}
        \item $X$は可分である.
        \item $X^*$の単位閉球$B^*$は$w^*$位相に関して距離化可能である.
    \end{enumerate}
\end{theorem}

\begin{proposition}[Schur's property 1921]
    $l^1$の列は,弱収束するならばノルム収束する.
\end{proposition}
\begin{remarks}
    弱位相に関する同様の結論は,定理について$X\mono X^{**}$を考えることで,$X^*$が可分ならば単位閉球$B\subset X$は距離化可能であることがわかる.
    しかし,双対空間が可分である例は(回帰的な例を除くと)少なく,代表的なのは$X=c_0,X^*=l^1$の場合のみである.
    なお,$X$が可分だからといって$B$は弱位相に関して距離化可能ではないことが,この命題からわかる.
    $l^1$の閉単位球が弱位相について距離化可能だとすると,$l^1$上の弱位相とノルム位相が一致してしまい,矛盾.
    これが,弱位相を調べる際には点列のみを考えては粒度が足りない好例である.
\end{remarks}

\subsection{Banach空間値関数の可測性}

\begin{tcolorbox}[colframe=ForestGreen, colback=ForestGreen!10!white,breakable,colbacktitle=ForestGreen!40!white,coltitle=black,fonttitle=\bfseries\sffamily,
title=]
    有限次元空間では退化していた3つの可測性の概念が考え得る.
    \begin{description}
        \item[可測性] $X^*$が$X$に誘導する$\sigma$-代数はBorel $\sigma$-代数$\B(X)$より真に小さくなり得る.
        \item[強可測性] Lebesgue積分の定義で肝要であるのは,可測関数のクラスは単関数の各点収束閉包である事実である.これを抽出する.
    \end{description}
    強可測性の特徴付けをPettis measurability theoremが与える.
    このとき,弱可測性との関係は,タイト性に似た,可分性のバージョン「可分値」が鍵を握る.
    これより,弱可測$\Rightarrow$可測$\Rightarrow$強可測で,強可測ならば可測かつ可分値.
\end{tcolorbox}

\begin{notation}
    $(S,\A)$を可測空間,$X$をBanach空間とする.
    部分集合$Y\subset X^*$が$X$に誘導する$\sigma$-代数を$\sigma(Y)$で表すと,
    \[\sigma(Y)=\Brace{\cC(Y)\in P(Y)\mid\exists_{x_1^*,\cdots,x_n^*\in Y}\;\forall_{B\in\B(\K^n)}\;\cC(Y)=\Brace{x\in X\mid (\brac{x,x_1^*},\cdots,\brac{x,x_n^*})\in B}}.\]
    また,$\K$-値関数の線型空間$F\subset\Map(S,\K)$について,
    \[F\otimes X:=\Brace{\sum^N_{n=1}f_n\otimes x_n\in\Map(S,X)\mid\forall_{n\in[N]}\;f_n\in F,x_n\in X,N\in\N}\]
\end{notation}

\subsubsection{可測性}

\begin{definition}[measurable]
    関数$f:(S,\A)\to (X,\B(X))$について,
    \begin{enumerate}
        \item $f$が可測であるとは,$\forall_{B\in\B(X)}\;f^{-1}(B)\in\A$.
        \item 
    \end{enumerate}
\end{definition}

\begin{proposition}[可分ならば縮退する]
    $X$を可分,$Y\subset X^*$は$w^*$-稠密な部分空間とする.このとき,$\sigma(Y)=\sigma(X^*)=\B(X)$.
\end{proposition}

\begin{corollary}[可測性の特徴付け]
    $X$は可分であるとき,関数$f:S\to X$について次の2条件は同値.
    \begin{enumerate}
        \item $f$は可測.
        \item 任意の$x^*\in X^*$について,$\brac{f(-),x^*}:S\to\K$は可測.
    \end{enumerate}
\end{corollary}

\subsubsection{強可測性}

\begin{definition}[simple]
    関数$f:S\to X$が\textbf{単純}であるとは,$\exists_{N\in\N}\;\forall_{n\in[N]}\;\exists_{A_n\in\A,x_n\in X}\;f=\sum_{n=1}^N1_{A_n}\otimes x_n$.
\end{definition}

\begin{definition}[strongly measurable]
    関数$f:S\to X$が\textbf{強可測}または\textbf{Bochner可測}であるとは,単関数列$\{f_n\}\subset\Map(S,X)$が存在して,その各点極限である:$\lim_{n\to\infty}f_n=f$.
\end{definition}

\begin{lemma}
    $X$が可分であるとき,関数$f:S\to X$について次の2条件は同値.
    \begin{enumerate}
        \item $f$は強可測.
        \item $f$は可測.
    \end{enumerate}
\end{lemma}

\subsubsection{弱可測性と特徴付け}

\begin{definition}[separably valued, weakly measurable]
    関数$f:S\to X$について,
    \begin{enumerate}
        \item $f$が\textbf{可分値}であるとは,$\Im f$が可分であることをいう.
        \item $f$が\textbf{弱可測}であるとは,任意の$x^*\in X^*$について,$S$上の$\K$-値関数$s\mapsto\brac{f,x^*}(s):=\brac{f(s),x^*}$は可測であることをいう.すなわち,$\forall_{g\in X^*}\;g\circ f:S\to\K$が可測であることをいう.
    \end{enumerate}
\end{definition}

\begin{theorem}[Pettis measurability theorem]
    $Y\subset X^*$を$w^*$-稠密な部分空間とする.関数$f:S\to X$について,次の3条件は同値.
    \begin{enumerate}
        \item $f$は強可測.
        \item $f$は可分値で,弱可測.
        \item $f$は可分値で,$\forall_{g\in Y^*}\;g\circ f:S\to\K$は可測.
    \end{enumerate}
    また,ある閉部分空間$X_0\subset X$について$\Im f\subset X_0$が成り立つとき,$f$は$X_0$-値単関数の列の各点収束極限である.
\end{theorem}

\begin{corollary}[単関数列は単調増加に取れる]
    $f:S\to X$が強可測ならば,単関数列$(f_n)$が存在して,$\norm{f_n(x)}\le\norm{f(x)}$かつ$\forall_{x\in X}\;f_n(x)\to f(x)$.
\end{corollary}

\begin{corollary}
    $f:S\to X$について,ある閉部分空間$X_0\subset X$について$\Im f\subset X_0$が成り立つとする.
    このとき,次の2条件は同値.
    \begin{enumerate}
        \item $f:S\to X$は強可測.
        \item $f:S\to X_0$は強可測.
    \end{enumerate}
\end{corollary}

\begin{corollary}
    強可測関数の列の各点収束極限は強可測である.
\end{corollary}

\begin{corollary}
    関数$f:S\to X$について,次の2条件は同値.
    \begin{enumerate}
        \item $f$は強可測.
        \item $f$は可分値で,可測.
    \end{enumerate}
\end{corollary}

\subsubsection{可測性の保存}

\begin{lemma}
    $E$を可分距離空間,$F$を距離空間とする.
    可測関数$f:E\to F$の像は可分である.
\end{lemma}

\begin{corollary}
    関数$f:S\to X$は強可測で,$\phi:X\to Y$は可測とする.このとき,$\phi\circ f:S\to Y$も強可測.
\end{corollary}

\subsection{測度空間上の関数}

\begin{tcolorbox}[colframe=ForestGreen, colback=ForestGreen!10!white,breakable,colbacktitle=ForestGreen!40!white,coltitle=black,fonttitle=\bfseries\sffamily,
title=]
    Banach空間値関数については,可測性の概念が3つに分かれるだけでなく,積分の定義に際しても新たな注意が必要になる.
    それは,面積確定でない可測集合$A_n$上に値を持つ単関数を省く必要がある.
    Lebesgue積分では高々$\infty,-\infty$しか出てこないため,非負単関数を別に考えれば良かったが,今回は$\infty x+\infty y\;(x,y\in X)$は全く定義不可能な演算になる.
    こうして,$\sigma$-有限性がさらに肝要になる.

    定義域に測度$\mu$が定まっているとき,これを用いて定義を強めることが出来る.
    これが可積分性の概念に繋がる.ここでは,$\mu$-可測性なる概念として定義する.
    この方がむしろ見通しが良い.
\end{tcolorbox}

\begin{notation}
    測度空間$(S,\A,\mu)$を考える.
\end{notation}

\begin{definition}[strongly $\mu$-measurable]\mbox{}
    \begin{enumerate}
        \item $X$値の$\mu$-単関数とは,$x_n\in X$と$\mu(A_n)<\infty$を満たす$A_n\in\A$を用いて$f=\sum_{n=1}^N1_{A_n}\otimes x_n$と表せるものをいう.
        \item $f:S\to X$が\textbf{$\mu$-強可測}であるとは,$f$に$\mu$-概収束する$\mu$-単関数列が存在することをいう.$\mu$-強可測関数全体の集合の同値類がなす線型空間を$L^0(S;X)$で表す.
        \item $f:S\to X$が\textbf{$\mu$-本質的に可分値}であるとは,可分な閉部分空間$X_0\subset X$が存在して,$f(s)\in X_0\;\mu\dae$が成り立つことをいう.
        \item $f:S\to X$が\textbf{$\mu$-弱可測}であるとは,任意の$g\in X^*$について,$g\circ f:S\to\R$が$\mu$-可測であることをいう.
    \end{enumerate}
\end{definition}

\begin{example}
    定数関数$1$は常に強可測であるが,これが$\mu$-強可測でもあることは$\mu$が$\sigma$-有限であることに同値.
\end{example}

\begin{proposition}[強可測関数は$\sigma$-有限な空間に本質的な台を持つ]
    $f:S\to X$は$\mu$-強可測であるとする.このとき,可測な分割$S=S_0\sqcup S_1\;(S_0,S_1\in\A)$が存在して,次の2条件を満たす:
    \begin{enumerate}
        \item $f=0\;\mu\dae\;\on S_0$.
        \item $\mu$は$S_1$上$\sigma$-有限.
    \end{enumerate}
\end{proposition}

\begin{proposition}[$\mu$-強可測性と一般の強可測性]
    関数$f:S\to X$について,
    \begin{enumerate}
        \item $f$が$\mu$-強可測ならば,$f$はある強可測関数に殆ど至る所等しい.
        \item $\mu$が$\sigma$-有限で,$f$が殆ど至る所ある強可測関数に等しいならば,$f$は$\mu$-強可測である.
    \end{enumerate}
\end{proposition}

\subsection{Bochner積分}

\begin{tcolorbox}[colframe=ForestGreen, colback=ForestGreen!10!white,breakable,colbacktitle=ForestGreen!40!white,coltitle=black,fonttitle=\bfseries\sffamily,
    title=]
    Banach空間に測度を導入して積分を考えたい.
    Lebesgue積分に相当する概念がBochner積分である.
    $X$-値関数のLebesgue関数$L^p(S;X)$をBochner空間という.
    一方で,汎関数$\brac{f,x^*}$の通常の意味でのLebesgue積分と可換になるべき連続な線型汎関数はただ一つに定まる.これがPettis積分である.

    Bochner積分は代数的場の量子論で頻繁に用いられる.
    優収束定理は引き続き成り立つが,Radon-Nikodymnoの定理は成り立たなくなる.
\end{tcolorbox}

\subsubsection{定義と特徴付け}

\begin{notation}
    $(S,\A,\mu)$を可測空間とする.
    $\mu$-単関数$f=\sum^N_{n=1}1_{A_n}\otimes x_n$について,
    \[\int_Sfd\mu:=\sum^N_{n=1}\mu(A_n)x_n\]
    とする.
\end{notation}

\begin{definition}[Bochner integral]
    $\mu$-強可測関数$f:S\to X$が\textbf{$\mu$についてBochner積分可能}であるとは,$\mu$-単関数列$(f_n)$が存在して次を満たすことをいう:
    \[\lim_{n\to\infty}\int_S\norm{f-f_n}d\mu=0.\]
    実際,この条件が成り立つとき,$\paren{\int_Sf_nd\mu}$は$\K$上のCauchy列をなす.この極限を$\int_Sfd\mu$とする.
\end{definition}

\begin{lemma}
    $f:S\to X$は$\mu$-強可測とする.
    \begin{enumerate}
        \item $f$がBochner可積分であることと,$\int_S\norm{f}d\mu<\infty$とは同値.
        \item $f$がBochner可積分であるとき,$\Norm{\int_Sfd\mu}\le\int_S\norm{f}d\mu(<\infty)$.
    \end{enumerate}
\end{lemma}

\subsubsection{作用素との可換性}

\begin{tcolorbox}[colframe=ForestGreen, colback=ForestGreen!10!white,breakable,colbacktitle=ForestGreen!40!white,coltitle=black,fonttitle=\bfseries\sffamily,
title=]
    Bochner積分により,線型汎関数$L^1(S,X)\to\K$が定まった.
    これと,作用素$T:L^1(S,X)\to L^2(S,Y)$との相互関係を考える.
\end{tcolorbox}

\begin{discussion}
    $f:S\to X$がBochner可積で,$T:X\to Y$が有界線型作用素であるとき,$Tf:S\to Y$もBochner可積で
    \[T\int_Sfd\mu=\int_STfd\mu.\]
    特に,$Y=\K$である場合について,
    \[\forall_{x^*\in E^*}\quad\Brac{\int_Sfd\mu,x^*}=\int_S\brac{f,x^*}d\mu.\]
    この結果を一般化する.
\end{discussion}

\begin{theorem}[Hille]
    $f:S\to X$をBochner可積で,$T:D(T)\to Y$を部分空間上の閉線型作用素とする.また,$f$は殆ど至る所$D(T)$に値を取り,殆ど至る所で定義された関数$Tf:S\to Y$もBochner可積であるとする.このとき,
    \begin{enumerate}
        \item $f:S\to D(T)$はBochner可積である.
        \item $\int_Sfd\mu\in D(T)$.
        \item $T\int_Sfd\mu=\int_STfd\mu$.
    \end{enumerate}
\end{theorem}

\subsection{Bochner空間}

\begin{tcolorbox}[colframe=ForestGreen, colback=ForestGreen!10!white,breakable,colbacktitle=ForestGreen!40!white,coltitle=black,fonttitle=\bfseries\sffamily,
title=]
    こうして,Lebesgue空間とは,スカラー値関数についてのBochner空間として相対化された.
    このとき,$\mu$-強可測性は,$\mu$-可測性に退化する.
\end{tcolorbox}

\begin{definition}\mbox{}
    \begin{enumerate}
        \item $L^p(S;X)\;(1\le p<\infty)$を,$\int_S\norm{f}^pd\mu<\infty$を満たす$\mu$-強可測関数$f:S\to X$の同値類がなす線型空間とする.
        \item $L^\infty(S;X)$を,$\exists_{r\ge0}\;\mu\Brace{\norm{f}>r}=0$を満たす$\mu$-強可測関数$f:S\to X$全体がなす線型空間とする.
    \end{enumerate}
    これらは$p$-ノルムについてBanach空間をなす.scaler-valued case $L^p(S):=L^p(S;\K)$をLebesgue空間という.
    また,$\sigma$-部分代数$\F$について,可測空間$(S,\F,\mu|_\F)$上のBochner空間を$L^p(S,\F;X)$と表すと,これは$L^p(S;X)$の$\F$上$\mu$-強可測な関数のなす部分空間である.
    $L^p(S,\F):=L^p(S,\F;\K)$と表す.
\end{definition}

\begin{definition}[convergence in measure]
    関数列$(f_n)$が\textbf{測度収束}するとは,$\forall_{r>0}\;\forall_{A\in\A}\;\mu(A)<\infty\Rightarrow\lim_{n\to\infty}\mu(A\cap\Brace{\abs{f_n-f}>r})=0$.
\end{definition}

\begin{lemma}
    $p\in[1,\infty]$とする.
    \begin{enumerate}
        \item $\mu$-単関数は$L^p(S;X)\;(1\le p<\infty)$上稠密である.特に,$L^p(S)\otimes X$は稠密である.
        \item $\mu$-単関数は$L^\infty(S;X)$上,測度収束の位相について稠密である.すなわち,任意の$f\in L^\infty(S;X)$と測度確定な$A\in\A$について,任意の$\ep>0$に対して$\mu$-単関数$g:S\to X$と$\mu(A\setminus A')<\ep$を満たす可測集合$A'\subset A$が存在して,$\norm{g}_\infty\le\norm{f}_\infty$かつ$\sup_{s\in A'}\norm{f(s)-g(s)}<\ep$を満たす.
    \end{enumerate}
\end{lemma}

\subsection{Pettis積分}\label{subsection-Pettis-integral}

\begin{tcolorbox}[colframe=ForestGreen, colback=ForestGreen!10!white,breakable,colbacktitle=ForestGreen!40!white,coltitle=black,fonttitle=\bfseries\sffamily,
title=]
    Pettis積分は,双対ペアを介した弱位相のアイデアを,積分の定義に用いたものである.
    導入したのはGelfandらしい.
\end{tcolorbox}

\subsubsection{弱積分の定義}

\begin{notation}
    $(S,\A,\mu)$を可測空間,$Y\subset X^*$を部分空間とする.
    関数$f:S\to X$は任意の$x^*\in Y$に対して$\brac{f,x^*}$が可積分だとする.
    この条件を,$Y=X^*$のとき\textbf{弱可積分}という.
    このとき,$T_f:Y\to L^1(S)$を$T_fx^*:=\brac{f,x^*}\;(x^*\in Y)$と定めると,これは有界線型作用素である.
    この随伴$T_f^*:L^\infty(S)\subset(L^1(S))^*\to Y^*$が鍵となる.なお,$\mu$が$\sigma$-有限のとき,$(L^1(S))^*=L^\infty(S)$.
\end{notation}

\begin{definition}[Pettis integrable]
    任意の$A\in\A$に対して,
    \[\tau(X,Y)\text{-}\int_Afd\mu:=T_f^*1_A\quad\in Y^*\]
    とする.これを,$f$の$A$上の\textbf{$\tau(X,Y)$-積分}と呼ぶ.
    $\tau(X,X^*)$-積分を弱積分という.さらにこの$Y=X^*$のとき,
    この$T_f^*$の像が$X(\subset X^{**})$に含まれるとき,$\mu$-弱可積分関数は\textbf{Pettis可積分}であるという.
\end{definition}

\begin{lemma}[弱積分の特徴付け]
    $\tau(X,Y)$-積分は,次の条件を満たすただ一つの$Y^*$の元である:
    \[\forall_{x^*\in Y}\quad\Brac{x^*,\tau(X,Y)\text{-}\int_Afd\mu}=\int_A\brac{f,x^*}d\mu.\]
\end{lemma}

\begin{proposition}
    弱積分可能な関数$f:S\to X$について,次の2条件は同値.
    \begin{enumerate}
        \item $f$は$\mu$についてPettis可積分.
        \item $\forall_{A\in\A}\;\exists_{x_A\in X}\;\forall_{x^*\in X^*}\;\brac{x_A,x^*}=\int_A\brac{f,x^*}d\mu$.
    \end{enumerate}
    この同値な条件が成り立つとき,$x_A=T^*_f1_A=:(P)\text{-}\int_Afd\mu$と表し,$f$の$A$上の\textbf{Pettis積分}という.
\end{proposition}
\begin{remarks}
    Bochner可積な関数はPettis可積で,各集合$A\in\A$上でその値は一致する.
\end{remarks}

\subsubsection{Pettis可積性の特徴付け}

\begin{tcolorbox}[colframe=ForestGreen, colback=ForestGreen!10!white,breakable,colbacktitle=ForestGreen!40!white,coltitle=black,fonttitle=\bfseries\sffamily,
title=]
    $\mu$-強可測かつ弱可積分ならば,ほぼPettis可積分であり,例外は$c_0$の非回帰性が引き起こすタイプのもののみである.
\end{tcolorbox}

\begin{theorem}[Pettis可積性の十分条件]
    $1<p\le\infty,1\le q<\infty$を共役指数とする.
    $\mu$-強可測関数$f:S\to X$が$\forall_{x^*\in X^*}\;\brac{f,x^*}\in L^p(S)$を満たすならば,任意の$\phi\in L^q(S)$に対して,関数$\phi\otimes f:S\to\K;s\mapsto \phi(s)f(s)$はPettis可積分である.
\end{theorem}

\begin{corollary}
    $(S,\A,\mu)$を有限測度空間とし,$p>1$とする.$f:S\to X$が強可測で$\forall_{x^*\in X^*}\;\brac{f,x^*}\in L^p(S)$が成り立つならば,$f$はPettis可積である.
\end{corollary}

\begin{example}[弱可積であるがPettis可積分でない例]
    $\{A_n\}\subset P((0,1))$を,互いに素な区間の列で,正の測度$\abs{A_n}>0$をもつものとする.
    関数$f:(0,1)\to c_0$を,$c_0$の標準Schauder基底$(e_n)$について
    \[f:=\sum_{n\in\N}\frac{1_{A_n}\otimes e_n}{\abs{A_n}}\]
    で定める.このとき,$f$は明らかに強可測で,弱可積分でもある.実際,任意の絶対収束列$a=(a_n)\in l^1=(c_0)^*$について,
    \[\int^1_0\brac{f(x),a}dx=\sum_{n\in\N}\abs{A_n}\frac{a_n}{\abs{A_n}}=\sum_{n\in\N}a_n<\infty\]
    より,$\brac{f(-),a}\in L^1(S;\R)$.
    ここで,$x_{**}:=T_f^*1_{(0,1)}$を$f$の$(0,1)$上の弱積分とする.
    すると,
    \[\forall_{n\in\N}\quad\brac{e_n^*,x^{**}}=\int^1_0\brac{f(x),e_n^*}dx=\int^1_0\sum_{m\in\N}\frac{\brac{1_{A_m}(x)e_m(x),e_n^*}}{\abs{A_n}}dx=1\]
    より,Fourier係数の一意性から$x^{**}=1\in l^\infty$である.しかし,これは有界列であっても,$0$へ収束はしない:$x^{**}\notin c_0$.よってPettis可積分ではない.
\end{example}
\begin{remarks}
    弱積分$\tau(X,Y)\td\int\in X^{**}$が$X$に入っていない例を作れ良いので,$c_0\subsetneq(c_\infty\subsetneq)(c_0)^{**}=l^\infty$を用いた.
\end{remarks}

\begin{proposition}
    $(S,\A,\mu)$を測度空間,$X$は$c_0$と同型な閉部分空間を持たないとする.このとき,$f:S\to X$が$\mu$-強可測で弱可積ならば,$f$はPettis可測である.
\end{proposition}
\begin{proof}
    \ref{thm-Bessaga-and-Pelczynski}による.
\end{proof}

\subsubsection{Analysis Now}

\begin{tcolorbox}[colframe=ForestGreen, colback=ForestGreen!10!white,breakable,colbacktitle=ForestGreen!40!white,coltitle=black,fonttitle=\bfseries\sffamily,
    title=Pettis積分]
    局所コンパクトハウスドルフ空間$X$からBanach空間$\fX$への関数$f:X\to\fX$を考える.
    $X$上のRadon積分$\int:C_c(X)\to\R$について,積分$\int f\in\fX$にあたる元を考えたい.
    $\R^n$-値積分が成分毎であるのと同様に,全ての連続な線型形式(射影みたいなもの)について,可換性
    \[\forall_{\varphi\in\fX^*}\;\Bracket{\int f,\varphi}=\int\brac{f(-),\varphi}\]
    を満たしてほしい.
    実はこの性質だけで積分は特徴付けられる,ということを議論する.
    1つ目の命題は双対ペア$(X^*,X)$についてで,2つ目は$(X,X^*)$についてである.
    ただし,左辺がこれから定義しようとしている$\fX$上の積分,右辺が$X$上のRadon積分となる.
\end{tcolorbox}

\begin{remark}[integrable norm]
    $\norm{f(-)}:X\to\R$は$X$上の可積分関数を定めるとする.このノルムを可積分ノルムと呼ぶ.
\end{remark}

\begin{lemma}
    関数$f:X\to\fX$に対して,$\fY_f:=\Brace{\varphi\in\fX^*\mid\brac{f(\cdot),\varphi}:X\to\fX\to\R は可測}$と定める.
    このとき,
    \begin{enumerate}
        \item $\fY_f$は$w^*$-閉,特にノルム閉な$\fX^*$の部分空間である.
        \item $\fY_f$が$\fX$の点を分離するならば,高々1つの元$\int f\in\fX$が存在して,
        \[\forall_{\varphi\in\fY_f}\;\Bracket{\int f,\varphi}=\int\brac{f(-),\varphi}\]
        を満たす.
    \end{enumerate}
\end{lemma}
\begin{proof}\mbox{}
    \begin{enumerate}
        \item $\fY_f$の列$(\varphi_n)$であって,$\varphi\in X^*$に$w^*$-収束するものを任意に取る.
        すると,任意の$x\in X$について,$\fX^*\ni\varphi_n\mapsto\brac{f(x),\varphi_n}\in\bF$は連続だから,
        $\{\brac{f(-),\varphi_n}\}\subset\Meas(X,\bF)$は$\brac{f(-),\varphi}$に各点収束する.
        このとき,極限である$\brac{f(-),\varphi}$は可測.
        よって,$\varphi\in\fY_f$.
        したがって,$\fY_f$は$w^*$-閉で,特にノルム閉である.
        線型部分空間であることは明らか.
        \item 作用素ノルムの定義から$\abs{\brac{f(-),\varphi}}\le\norm{f(-)}\norm{\varphi}$であり,右辺の関数は仮定より可積分であるから,$\brac{f(-),\varphi}:X\to\bF$も可積分である.
        すると,
        $\fY_f=\Brace{\varphi\in X^*\mid\brac{f(-),\varphi}\in\L^1(X)}$とも表せる.
        分離的な部分空間$\fY_f$は$w^*$-稠密である\ref{cor-separating-subspace-is-dense}から,
        $X^*$上への一意的な延長を持つ\ref{prop-extension-of-operator-on-dense-subset}.
        よって一意に定まる.
    \end{enumerate}
\end{proof}

\begin{proposition}
    $\fY$をBanach空間で,$f:X\to\fY^*$は$w^*$-可測で,ノルム$\norm{f(-)}$は可積分であるとする.
    このとき,ただ一つの元$\int f\in\fY^*$が存在して,
    \[\forall_{y\in\fY}\;\Bracket{y,\int f}=\int\brac{y,f(-)}\]
    を満たす.
\end{proposition}
\begin{proof}
    $\fX:=\fY^*$とすると,$f$は$w^*$-可測で,任意の$y\in\fY$に対して$\brac{y,-}:\fY^*\to\bF$も$w^*$-連続より$w^*$-可測だから,$\brac{y,f(-)}:X\to\bF$も可測.よって,$\fY\subset\fY_f$.
    したがって補題より,任意の$y\in\fY$について$\brac{f(-),y}\in\L^1(X)$だから,$\int\brac{y,f(-)}\in\bF$.
    この対応$\fY\to\R;y\mapsto\int\brac{y,f(-)}$は明らかに$\fY$上で線型である.この対応を$\int f$と表すと,一意性は補題より従うから,連続性すなわち$\int f\in\fY^*$を示せば良い.

    \[\Abs{\Bracket{y,\int f}}=\Abs{\int\brac{y,f(-)}}=\norm{y}\int\Bracket{\frac{y}{\norm{y}},f(-)}\le\norm{y}\int\norm{f(-)}\]
    より,線型汎函数$\int f$は有界である.よって,$\int f\in\fY^*$.
\end{proof}

\begin{proposition}
    $f:X\to\fX$を弱可測関数で,$\fX$を可分とする.$\norm{f(-)}\in\L^1(X)$とする.
    このとき,ただ一つの$\int f\in\fX$が存在して,
    \[\forall_{\varphi\in\fX^*}\;\Bracket{\int f,\varphi}=\int\brac{f(-),\varphi}\]
    を満たす.
\end{proposition}
\begin{proof}\mbox{}
    \begin{description}
        \item[方針] $f:X\to\fX$は弱可測としたから,$\brac{-,\varphi}:X\to\bF$は弱連続より弱可測であることと併せると,$\fY_f=\fX^*$.
        よって,$f:X\to\fX\mono\fX^{**}$も弱可測で特に$w^*$-可測だから,命題より,$\int f\in\fX^{**}$が存在して,
        \[\forall_{\varphi\in\fX^*}\;\Bracket{\int f,\varphi}=\int\brac{f(-),\varphi}\]
        を満たす.あとはこれが$\int f\in\fX$であること,すなわち,$\int f:\fX^*\to\bF$が$w^*$-連続であることを示せば良い.
        \item[証明] 
        $\fX$は可分としたから,$\fX^*$の単位閉球$B^*$は$w^*$-位相についても第2可算であり,距離化可能\ref{thm-metrizability-of-ball}.
        よって,$B^*$の列$(\varphi_n)$が$\varphi$に$w^*$-収束する,すなわち,$\L^1(X)$の列$(\brac{f(-),\varphi_n})$が$\brac{f(-),\varphi}$に各点収束する($\forall_{x\in\fX}\;\brac{f(x),-}:\fX^*\to\bF$は$w^*$-連続)と仮定して,$\paren{\int\brac{f(-),\varphi_n}}$も$\bF$上で$\int\brac{f(-),\varphi}$に収束することを示せば良い\ref{cor-characterization-of-weak-star-continuousness}.
        が,これは$\norm{f(-)}$という可積分な非負値の優関数が見つかるから,Lebesgueの優収束定理より従う.
    \end{description}
\end{proof}



\chapter{Hilbert Space}

\begin{quotation}
    完備な距離を内積が誘導する場合をHilbert空間という.
    これは,標準的なペアリングが存在するBanach空間論と見れる.
    些細な違いかもしれないが,理論がフルパワーを発するのはこの場合のみである.
    特に,Banach空間上の作用素にはほとんど一般論が成り立たない.
    例えばHilbert空間には,標準的なペアリングを通じてRieszの表現が存在する.

    Banach空間上に,内積$L^p(X)\times L^q(X)\ni(f,g)\mapsto\brac{f,g}:=\int f\o{g}$によって,
    $(L^p(X))^*\simeq L^q(X)$なる同型が誘導される\ref{thm-duality-of-Lp}.
    2次のノルムがHilbert空間として重要である理由は,自身と共役だからである.
\end{quotation}

\section{内積}

\begin{tcolorbox}[colframe=ForestGreen, colback=ForestGreen!10!white,breakable,colbacktitle=ForestGreen!40!white,coltitle=black,fonttitle=\bfseries\sffamily,
title=]
    Hilbert空間には内積という名前の標準的なペアリングが存在する.
    これを通じてRieszの表現が存在し,また直交性の概念が定義される.
    これは基底論にも影響を与え,Hilbert空間の分類は有限次元線型空間論の延長に落ちる.
\end{tcolorbox}

\subsection{半双線型形式とセミノルムの関係}

\begin{tcolorbox}[colframe=ForestGreen, colback=ForestGreen!10!white,breakable,colbacktitle=ForestGreen!40!white,coltitle=black,fonttitle=\bfseries\sffamily,
title=]
    Hilbert空間とは,Banach空間であって,ノルムが中線定理を満たすような空間である.

    双線型形式が関数$x^2$であるとしたら,半双線型形式がノルム$\abs{x^2}$を定める.
\end{tcolorbox}

\begin{definition}[sesquilinear form, adjoint form, self-adjoint, positive, semi-inner product, inner product]
    $\bF$-線型空間$X$について,
    \begin{enumerate}
        \item 写像$(-|-):X\times X\to\bF$が\textbf{半双線型形式}であるとは,第一引数について線型で,第二引数について共役線型であることをいう.\footnote{数理物理では逆.}$\bF=\R$のときは双線型性に同値.
        \item 半双線型形式$(-|-)$の\textbf{随伴形式}$(-|-)^*$とは,$(x|y)^*:=\o{(y|x)}$で定まる半双線型形式をいう.
        \item $(-|-)^*=(-|-)$を満たす半双線型形式を\textbf{自己共役}という.$\bF=\R$であるときは対称ともいう.
        \item $\forall_{x\in X}\;(x|x)\ge 0$を満たすとき,$(-|-)$を\textbf{半正定値}という.
        \item 半正定値な自己共役半双線型形式を,$X$上の\textbf{半内積}という.
        \item $(x|x)=0\Rightarrow x=0$を満たす半内積を,$X$上の\textbf{内積}という.
    \end{enumerate}
\end{definition}

\begin{lemma}[自己共役性の特徴付け]\label{lemma-characterization-of-self-adjointness}
    $\C$-線型空間上の半双線型形式について,
    \begin{enumerate}
        \item $4(x|y)=\sum^3_{k=0}i^k(x+i^ky|x+i^ky)$.
        \item $(-|-)$が自己共役であることは,$\forall_{x\in X}\;(x|x)\in\R$に同値.特に,$(-|-)$が正定値ならば自己共役である.\footnote{実数値の正定値双線型形式が対称とは限らない.}
    \end{enumerate}
\end{lemma}
\begin{proof}\mbox{}
    \begin{enumerate}
        \item \begin{align*}
            &(x+y|x+y)+i(x+iy|x+iy)+(-1)(x-y|x-y)+(-i)(x-iy|x-iy)\\
            =&(x+y|x+y)+(x+iy|y-ix)+(x-y|y-x)+(x-iy|y+ix)=4(x|y).
        \end{align*}
        \item $(-|-)$が自己共役とすると,$y=0$と任意の$x\in X$について,(1)を用いて展開すると
        \begin{align*}
            (x|0)&=(x|x)+i(x|x)+(-1)(x|x)+(-i)(x|x)\\
            \o{(0,x)}&=\o{(x|x)}+(-i)\o{(x|x)}+(-1)\o{(x|x)}+i\o{(x|x)}
        \end{align*}
        と展開できるから,$(x|x)\in\R$である.
        一方で$(x|x)\in\R$を仮定すると,(1)を用いて,$(x|x)\in\R$に注意して$a+ib\;(a,b\in\R)$の形に整理すると,
        \begin{align*}
            (x|y)&=(x+y|x+y)+i(x+iy|x+iy)+(-1)(x-y|x-y)+i(-x+iy|x-iy)\\
            &=(2y|2x)+i\underbrace{(2iy|2x)}_{\in\R}\\
            (y|x)&=(x+y|x+y)+i(y+ix|y+ix)+(-1)(y-x|y-x)+i(y-ix|ix-y)\\
            &=(2y|2x)+i\underbrace{(2y|2ix)}_{\in\R}
        \end{align*}
        となるが,それぞれの虚部について$(2iy|2x)=i(2y|ix),(2y|2ix)=-i(2y|2x)$だから,たしかに$(x|y)=\o{(y|x)}$を満たす.
    \end{enumerate}
\end{proof}
\begin{remarks}
    一変数じゃないが,鏡像の原理を奥に感じる.
\end{remarks}

\begin{proposition}[polarization identity, parallellogram law]\label{prop-polarization-identity}
    半内積$(-|-):X\times X\to\bF$について,
    \begin{enumerate}
        \item 関数$\norm{-}:X\to\R_+$を$\norm{x}:=(x|x)^{1/2}$で定めると,これは斉次関数である.
        \item 次の\textbf{極化恒等式}が成り立つ:
        \begin{enumerate}[(a)]
            \item $\bF=\C$のとき,$4(x|y)=\sum^3_{k=0}i^k\norm{x+i^ky}^2$.
            \item $\bF=\R$のとき,$4(x|y)=\norm{x+y}^2-\norm{x-y}^2$.
            \item $\bF=\C$のとき,実部と虚部に分けて$\norm{x+y}^2=\norm{x}^2+2\Re\brac{x,y}+\norm{y}^2$も極化恒等式と呼ぶ.
        \end{enumerate}
        \item (Cauchy-Bunyakowsky-Schwarz) $\abs{(x|y)}\le\norm{x}\norm{y}$.
        特に,$\norm{-}:X\to\R_+$は劣加法的であり,$X$上のセミノルムを定める.
        $(-|-)$が内積であるとき,$\norm{-}$はノルムを定める.
        \item セミノルム$\norm{-}$について中線定理が成り立つ:$\norm{x+y}^2+\norm{x-y}^2=2(\norm{x}^2+\norm{y}^2)$.
        \item ノルム$\norm{-}$が中線定理を満たすとき,(2)の極化恒等式によって定まる半双線型形式は内積を定める.
    \end{enumerate}
\end{proposition}
\begin{proof}\mbox{}
    \begin{enumerate}
        \item 半内積$(-|-):X\times X\to\bF$は特に半正定値だから,特に$\forall_{x\in X}\;(x|x)\ge 0$.よって確かに$\norm{-}:X\to\R_+$はwell-defined.
        $\forall_{a\in\bF}\;\norm{ax}=(\abs{a}^2(x|x))^{1/2}=\abs{a}\norm{x}$.
        \item (a)は補題(1)より.(b)は
        \begin{align*}
            (x+y|x+y)-(x-y|x-y)&=(x|x)+(x|y)+(y|x)+(y|y)-((x|x)-(y|x)-(x|y)+(y|y))=4(x|y)
        \end{align*}
        より.(c)は
        \begin{align*}
            \norm{x+y}^2&=\norm{x}^2+(y|x)+(x|y)+\norm{y}^2=\norm{x}^2+2\Re(x|y)+\norm{y}^2.
        \end{align*}
        \item 
        \begin{description}
            \item[Cauchy-Schwarz] 
        (2)(c)と同様にして,
        \[\forall_{\al\in\bF}\;\abs{\al}^2\norm{x}^2+2\Re\al(x|y)+\norm{y}^2=\norm{\al x+y}^2\ge 0\]
        を得る.これを$\al$の方程式と見ると,実数解は高々1つであるから,判別式は非正でなくてはならない.よって,
        \[(\Re(x|y))^2-\norm{x}^2\norm{y}^2\le 0\]
        であるが,ここで$(x|y)=be^{i\theta}\;(b\ge 0)$とおいたとき,$\al:=te^{-i\theta}\;(t\ge0)$を考えることで,
        \[\norm{x}^2t^2+2bt+\norm{y}^2\ge0\]
        を得る.特に,左辺を実数$t$についての二次方程式と見たときに解は高々1つだから,$b^2-\norm{x}^2\norm{y}^2\le 0$が必要.ここから$\abs{(x|y)}\le\norm{x}^2\norm{y}^2$を得る.
            \item[劣加法性] Cauchy-Schwarzの不等式より$\Re(x|y)\le\abs{(x|y)}\le\norm{x}\norm{y}$と(2)(c)を併せると,$\norm{x+y}^2\le(\norm{x}+\norm{y})^2$を得る.
            \item[斉次性] (1)で示した.
            \item[分離] $\norm{x}=0$とすると,$\norm{x}^2=(x|x)=0$.$(-|-)$が内積であるとき,これは$x=0$を導く.
        \end{description}
        \item 省略.
        \item 
    \end{enumerate}
\end{proof}
\begin{remarks}
    一般に(共役)対称形式は二次形式を定めるが,二次形式から対称性を復元する際に必要なノルムと内積を結ぶ条件を極化恒等式という.
\end{remarks}

\begin{definition}[orthogonal]
    $(-|-):X\times X\to\bF$を半双線型形式とする.
    \begin{enumerate}
        \item ベクトル$x,y$が直交する$x\perp y$とは,$(x|y)=0$を満たすことをいう.
        \item 部分集合$Y,Z\subset X$が直交する$Y\perp Z$とは,$\forall_{y\in Y,z\in Z}\;y\perp z$を満たすことをいう.
        \item 部分集合$X\subset H$について,$X^\perp:=\Brace{x^\perp\in H\mid x^\perp\perp X}$と表すと,$X^\perp$は閉部分空間である.
    \end{enumerate}
\end{definition}

\begin{lemma}[Pythagoras identity]\label{lemma-Pythagoras-identity}
    $X$が実線型空間であるとき,
    $x,y\in X$について,次の2条件は同値.
    \begin{enumerate}
        \item $x\perp y=0$.
        \item $\norm{x+y}^2=\norm{x}^2+\norm{y}^2$.
    \end{enumerate}
\end{lemma}
\begin{proof}
    (1)$\Rightarrow$(2)は極化恒等式(c)より.
    (2)$\Rightarrow$(1)は,極化恒等式(c)より$\Re(x|y)=0$と,$(x|iy)=0$より$\Im(x|y)=0$を,別々に得る.
\end{proof}

\subsection{二次形式と極化恒等式}

\begin{tcolorbox}[colframe=ForestGreen, colback=ForestGreen!10!white,breakable,colbacktitle=ForestGreen!40!white,coltitle=black,fonttitle=\bfseries\sffamily,
title=]
    内積をノルムで表した式を極化不等式と呼んだが,これには双線型形式と二次形式との間の一般論がある.
    これが,2という数字が重要たる所以であろうか?
\end{tcolorbox}

\begin{definition}[quadratic form, quadratic refinement]\mbox{}
    \begin{enumerate}
        \item $k$-加群$V$上の\textbf{二次形式}とは,関数$q:V\to k$であって,2次の斉次性をもち,さらにその極化$(v,w)\mapsto q(v+w)-q(v)-q(w)$が双線型形式を定めることをいう.
        \item 双線型形式$\brac{-,-}:V\otimes V\to k$が\textbf{定める二次形式}とは,$\forall_{v,w\in V}\;\brac{v,w}=q(v,w)-q(v)-q(w)+q(0)$を満たすものをいう.\footnote{$2\in k$が可逆ならば常に存在する.}
    \end{enumerate}
\end{definition}

$R$を可換環とし,$V,W$を$R$上の加群とする.
$m:V\times V\to W$を双線型写像とし,$Q(x):=m(x,x)$をこれが定める二次形式とする.
$m:V\otimes V\to W$は$R$-加群の準同型であるが,$Q:V\to W$はそうではない.

この設定の下で,次が成り立つ.
\begin{proposition}\mbox{}
    \begin{enumerate}
        \item 中線定理:$2Q(x)+2Q(y)=Q(x+y)+Q(x-y)$.
        \item 極化恒等式:$2m(x,y)+2m(y,x)=Q(x+y)-Q(x-y)$.
    \end{enumerate}
    中線定理と和差を取ったり,$W$上で2が可逆であるかによって,極化不等式は他の表現も持つ.
\end{proposition}

\begin{proposition}[極化不等式の一般形]
    $\sum_{i\in I}a_i^2c_i=\sum_{i\in I}b_i^2c_i=0$を満たす$R^3$の有限列$(a_i,b_i,c_i)$について,$k:=\sum_{i\in I}a_ib_ic_i$とおいたとき,
    \[kxy+kyx=\sum_{i\in I}c_i(a_ix+b_iy)^2.\]
\end{proposition}

$m$が対称であるとき,極化不等式を通じて,$Q(x)=x^2$から,$m(x,y)=xy$の値を復元できる.

\subsection{Hilbert空間の例}

\begin{definition}[pre-Hilbert space]
    内積空間$H$が,付随するノルムについてBanach空間をなすとき,これを\textbf{Hilbert空間}という.
    また内積空間を前Hilbert空間ともいう.
\end{definition}

\begin{example}[二乗可積分関数の空間:特殊から一般へ]\mbox{}
    \begin{enumerate}
        \item Euclid空間$\bF^n$は,通常の内積についてHilbert空間をなし,付随するノルムは2-ノルムである.
        \item $l^2(\Z):=\Brace{(a_n)_{n\in\Z}\in\prod_{n\in\Z}\bF\;\middle|\;\sum_{n\in\Z}\abs{a_n}^2<\infty}$は,内積$((a_n)|(b_n))=\sum_{n\in\Z}a_n\o{b_n}$に関してHilbert空間をなす.
        \item コンパクト台を持つ関数の空間$C_c(\R^n)$は,内積$(f|g):=\int f(x)\o{g(x)}dx$についてpre-Hilbert空間をなし,付随するノルムは2-ノルムである.これを完備化したものは$L^2(\R^n)$であった\ref{exp-Banach-spaces}が,これがHilbert空間である.
        \item 一般に,局所コンパクトハウスドルフ空間$X$上のRadon積分$\int$について二乗可積分な関数のなす空間の完備化$L^2(X)$\ref{exp-Banach-space-of-Radon-integrable-functions}は,内積$(f|g)=\int f\o{g}$についてHilbert空間となる.
    \end{enumerate}
\end{example}

\begin{definition}[orthogonal sum / direct sum]\label{def-orthogonal-sum-of-Hilbert-spaces}
    Hilbert空間の族$(H_j)_{j\in J}$について,
    \begin{enumerate}
        \item 代数的直和$\sum_{i\in J}H_j$には,$(x|y):=\sum_{j\in J}(\pr_jx|\pr_jy)$によって内積が定まる.
        \item この内積に付随するノルムは2-ノルムであり,これについての完備化を\textbf{(直交)直和}といい,$\oplus_{j\in J}H_j$と表す.
        \item 命題\ref{prop-completion-of-algebraic-direct-product}より,集合としては
        \[\bigoplus_{j\in J}=\Brace{x\in\prod_{j\in J}H_j\;\middle|\;\sum_{j\in J}\norm{\pr_j(x)}^2<\infty}\]
        と表せる.特に,Hilbert空間の直和$\oplus H_j$の元$x$は,可算個の$j\in J$を除いて$\pr_j(x)=0$である.
    \end{enumerate}
\end{definition}
\begin{remarks}
    これがHamel基底を超える,待ち望まれた無限次元の直和空間の構成である.
    有限の場合はHamel基底と変わらないが,無限の場合は,無限和が二乗収束するものからなる空間を考えると完備な内積が定まる.
\end{remarks}

\subsection{直交分解}

\begin{tcolorbox}[colframe=ForestGreen, colback=ForestGreen!10!white,breakable,colbacktitle=ForestGreen!40!white,coltitle=black,fonttitle=\bfseries\sffamily,
title=]
    直交分解を距離の言葉によって議論するところは,商ノルムの定義と同じ作戦である.
\end{tcolorbox}

\begin{lemma}
    $C$をHilbert空間$H$の非空な閉凸集合とする.このとき,任意の$y\in H$に対して,距離$d(y,C)$を最小にするときの点$x=\argmin_{x\in C}d(y,x)\in C$が唯一つ存在する.
\end{lemma}
\begin{proof}
    $C$を$C-y$と取り直すことで,$d(0,C)$を最小にする$x\in C$を考えても,一般性は失われない.
    \begin{description}
        \item[存在] $\al:=\inf\Brace{\norm{x}\ge0\int x\in C}$とおき,$\norm{x_n}\to\al$を満たす$C$の点列$(x_n)$を任意に取る.
        $C$の凸性より$\forall_{y,z\in C}\;(y+z)/2\in C$だから,中線定理\ref{prop-polarization-identity}より,
        \[\forall_{y,z\in C}\;\quad2(\norm{y}^2+\norm{z}^2)=\norm{y+z}^2+\norm{y-z}^2\ge4\al^2+\norm{y-z}^2.\]
        特に$y=x_n,z=x_m$の場合を考えることで,$(x_n)$はCauchy列であることが分かる.よって,$C$は閉集合だから,ある$x\in C$が存在して,$\lim_{n\to\infty}x_n=x,\norm{x}=\al$を満たす.
        \item[一意性] $z\in C$も$\norm{z}=\al$を満たすとする.すると,中線定理より,$4\al^2\ge4\al^2+\norm{x-z}^2$だから,$x=z$を得る.
    \end{description}
\end{proof}

\begin{theorem}
    任意の閉部分空間$X\subset H$について,
    \begin{enumerate}
        \item 任意の元$y\in H$は一意的な分解$y=x+x^\perp\in X\oplus X^\perp$を持つ.
        また,$H=X\oplus X^\perp$と直交直和で表せる.
        \item この$x\in X$は$y$に一番近い点$\argmin_{x\in X} d(y,x)$であり,$x^\perp\in X^\perp$も$y$に一番近い点$\argmin_{x^\perp\in X^\perp} d(y,x^\perp)$である.
        \item $(X^\perp)^\perp=X$が成り立つ.
    \end{enumerate}
\end{theorem}
\begin{proof}\mbox{}
    \begin{description}
        \item[(1)] 
        任意に$y\in H$をとり,$x:=\argmin_{x\in X}d(y,X)$とおく.
        \begin{description}
            \item[存在] $x^\perp:=y-x\in X^\perp$を示す.
            \begin{align*}
                \forall_{z\in X}\;\forall_{\ep>0}\quad\norm{x^\perp}^2&=\norm{y-x}^2\le\norm{y-(x+\ep x)}^2&\because xの取り方\\
                &=\norm{x^\perp-\ep z}^2=\norm{x^\perp}^2-2\ep\Re(x^\perp|z)+\ep^2\norm{z}^2&\because 極化恒等式\ref{prop-polarization-identity}
            \end{align*}
            より,$\forall_{z\in X}\;\forall_{\ep>0}\;2\Re(x^\perp|z)\le\ep\norm{z}^2$.
            よって,$\forall_{z\in X}\;\Re(x^\perp|z)\le 0$.$z,-z,iz,-iz$について考えることで,$\forall_{z\in X}\;(x^\perp|z)=0$を得る.
            以上より,対応$\Phi:H\to X\oplus X^\perp;y\mapsto x+x^\perp$が定まった.
            明らかに全射で,Pythagorasの恒等式\ref{lemma-Pythagoras-identity}より,等長写像であることも分かる.あとは単射性を示せば良い.
            \item[一意性] $y=z+z^\perp\;(z\in X,z\in X^\perp)$とも表せたとする.このとき$0=(x-z)+(x^\perp-z^\perp)$であるが,Pythagorasの恒等式\ref{lemma-Pythagoras-identity}より,$0=\norm{x-z}^2+\norm{x^\perp-z^\perp}^2$.
            すなわち,$x=z\land x^\perp=z^\perp$.
            よって,$\Phi:H\iso X\oplus X^\perp$は等長同型である.
        \end{description}
        \item[(3)] $X\perp X^\perp$より$X\subset (X^\perp)^\perp$であるから,$X\supset (X^\perp)^\perp$を示せば良い.
        任意に$y\in(X^\perp)^\perp$をとると,一意的な分解$y=z+z^\perp$を持つ.
        $x^\perp=y-x\in X^\perp\cap (X^\perp)^\perp$より,
        $x^\perp=0$が従い,$X=(X^\perp)^\perp$が分かる.
        \item[(2)] $(X^\perp)^\perp=X$より,(1)の証明を$X$を$X^\perp$として行うことより,$x^\perp=\argmin_{x^\perp\in X^\perp}d(y,x^\perp)$も分かる.
    \end{description}
\end{proof}

\begin{corollary}[closed linear span]\label{cor-expression-of-closed-linear-span}
    任意の部分集合$X\subset H$について,$X$を含む最小の閉部分空間は$(X^\perp)^\perp$である.
    特に,$X$が$H$の部分空間ならば,$\dbloverline{X}=(X^\perp)^\perp$.
\end{corollary}
\begin{proof}
    $X^\perp$は閉部分空間であるから,定理より$(X^\perp)^\perp$も閉部分空間である.これが$X$を含む最小の閉部分空間であることを示す.
    $X\subset Y$を$H$の閉部分空間とすると,$Y^\perp\subset X^\perp$で,$(X^\perp)^\perp\subset (Y^\perp)^\perp=Y$が従う.

    $H$の部分空間$X$を含む最小の閉集合が$\dbloverline{X}$で,$\dbloverline{X}$自身も部分空間であるから,これは最初の閉部分空間でもある.
\end{proof}

\begin{corollary}[部分空間の稠密性の特徴付け]\label{cor-dense-subspace}
    $X$を$H$の部分空間とする.次の2条件は同値.
    \begin{enumerate}
        \item $X$は$H$上稠密である.
        \item $X^\perp=0$.
    \end{enumerate}
\end{corollary}
\begin{proof}
    部分空間$X$について,$X^\perp$は$H$の閉部分空間だから,$X^\perp\oplus(X^\perp)^\perp=H$.系より$(X^\perp)^\perp=\dbloverline{X}$であるから,$\dbloverline{X}=H$は$X^\perp=0$と同値.
\end{proof}

\subsection{Rieszの表現定理}

\begin{tcolorbox}[colframe=ForestGreen, colback=ForestGreen!10!white,breakable,colbacktitle=ForestGreen!40!white,coltitle=black,fonttitle=\bfseries\sffamily,
title=]
    Rieszの表現定理は変種が多々あり,いずれもある種の位相線型空間とその双対空間との間に(反)同型を取る知識である.

    なお,
    Metの同型は全射な等距離写像である(全単射な等長写像は,等長な逆を持つ).
    これは擬距離空間やRiemann多様体では成り立たない.
\end{tcolorbox}

\begin{proposition}[Riesz representation theorem]\label{prop-isometry-of-Hilbert-dual}
    写像$\Phi:H\to H^*;x\mapsto(-|x)$は,共役線型な等長同型である.
\end{proposition}
\begin{proof}\mbox{}
    \begin{description}
        \item[共役線形性] $\Phi(ax)=(-|ax)=\o{a}(-|x)=\o{a}\Phi(x)$.
        \item[等長写像] Cauchy-Schwarzの不等式\ref{prop-polarization-identity}より,
        \[\forall_{x\in H}\;\norm{\Phi(x)}=\sup_{y\in B}\abs{(y|x)}\le\sup_{y\in B}\norm{y}\norm{x}=\norm{x}.\]
        また一方で,
        \[\forall_{x\in H}\;\norm{x}^2=\Phi(x)(x)=\Phi(x)\paren{\norm{x}\frac{x}{\norm{x}}}\le\norm{x}\norm{\Phi(x)}\]
        より$\norm{x}\le\norm{\Phi(x)}$.
        \item[全単射] 等長写像は単射であるから,全射性を示せば良い.
        任意に$\varphi\in H^*\setminus\{0\}$をとって逆像が像が空で無いことを示す.
        $X:=\Ker\varphi$とおくと,$\varphi\ne 0$よりこれは$H$の真の閉部分空間をなすから,ある$x\in X^\perp$が存在して$\varphi(x)=1$を満たす.
        すると,$\forall_{y\in H}\;y-\varphi(y)x\in X$より,
        \[(y|x)=(y-\varphi(y)x+\varphi(y)x|x)=\varphi(y)\norm{y}^2\]
        が任意の$y\in H$について成り立つから,$\Phi^{-1}(\varphi)\ni\frac{x}{\norm{x}^2}$.
    \end{description}
\end{proof}

\begin{corollary}[$L^2$上の有界線形関数の積分表現]
    $(X,\Om,\mu)$を測度空間とする.任意の有界線型汎函数$F:L^2(\mu)\to\bF$に対して,ただ一つの元$h_0\in L^2(\mu)$が存在して,
    \[\forall_{h\in L^2(\mu)}\quad F(h)=\int h\o{h_0}d\mu\]
    と表せる.
\end{corollary}

\subsection{弱位相とその特徴付け}

\begin{definition}[weak topology on Hilbert space]\label{def-weak-topology-on-H(B)}
    Hilbert空間$H$上の弱位相とは,有界な線型汎函数の集合$H^*$が定める始位相をいう.
    これは,標準的な同型$\Phi:H\to H^*$ \ref{prop-isometry-of-Hilbert-dual}により,$H^*$上の$w^*$-位相を引き戻したものに一致する.
    よって特に,Alaogluの定理\ref{thm-Alaoglu}より,$H$内の単位球は弱コンパクトである.
\end{definition}

\begin{lemma}\label{lemma-characterization-of-bounded-operator-on-Hilbert-space}
    任意のHilbert空間の作用素$T:H\to H$について,次の2条件は同値.
    \begin{enumerate}
        \item $T\in B(H)$である($H$のノルム位相について連続,すなわち,有界).
        \item $T$は弱位相について連続である(weak-weak continuous).
        \item $T$はノルム-弱連続である(norm-weak continuous).
    \end{enumerate}
\end{lemma}
\begin{proof}\mbox{}
    \begin{description}
        \item[(1)$\Rightarrow$(2)] 任意の$y\in H$を取れば,随伴作用素\ref{thm-existence-of-adjoint-operator}を考えることにより,$T^*y\in H$が存在して,次の図式は可換.
        \[\xymatrix{
            H\ar[r]^-T\ar[dr]_-{(-|T^*y)}&H\ar[d]^-{(-|y)}\\
            &\bF
        }\]
        $H$上に弱位相を考えるとき,$(-|T^*y),(-|y)$はいずれも連続だから,$T:H\to H$も連続である.
        \item[(2)$\Rightarrow$(1)] 
        グラフ$G(T):=\Brace{(x,y)\in H\times H\mid Tx=y}$がノルム閉集合であることを示せば良い\ref{thm-closed-graph-theorem}.
        任意に,$G(T)$の列$(x_n,Tx_n)$を取り,これは$(x,y)$にノルム収束するとする.
        すると,
        ノルム位相は弱位相よりも強いため,特に弱収束する.
        よって,$T$が弱位相について連続であるとすると,弱位相について$Tx_n\to Tx$であるから,これは$Tx=y$を含意する.すなわち,$(x,y)\in G(T)$.
        \item[(2)$\Rightarrow$(3)] ノルム位相は弱位相よりも強いため.
        \item[(3)$\Rightarrow$(1)] (2)$\Rightarrow$(1)と同様の議論によって示せる.
    \end{description}
\end{proof}

\begin{remarks}
    weak-norm連続な作用素は,有限なランクを持つ.この議論は,コンパクト作用素の特徴付けにつながる.
\end{remarks}

\subsection{正規直交系}

\begin{tcolorbox}[colframe=ForestGreen, colback=ForestGreen!10!white,breakable,colbacktitle=ForestGreen!40!white,coltitle=black,fonttitle=\bfseries\sffamily,
title=]
    任意の線型空間には,極大な線型独立系が存在するという意味で,Hamel基底を持つ.
    Hilbert空間では,内積から定まる基底が存在し,これは常にHamel基底とは異なる.
    正規直交基底を用いて,Hilbert空間の同型を特徴づけることが出来る.
\end{tcolorbox}

\begin{definition}[orthonormal, orhonormal basis]
    Hilbert空間$H$の部分集合$\{e_j\}_{j\in J}$について,
    \begin{enumerate}
        \item $\{e_j\}_{j\in J}$が\textbf{正規直交}であるとは,$\forall_{j\in J}\;\norm{e_j}=1$かつ$(e_j|e_i)=\delta_{ij}$を満たすことをいう.
        \item $\{e_j\}_{j\in J}$が\textbf{正規直交基底}であるとは,生成する部分空間が$H$上稠密であることをいう:$\dbloverline{\brac{e_j}_{j\in J}}=H$.これは,$\oplus_{j\in J}\bF e_j=H$に同値\ref{def-orthogonal-sum-of-Hilbert-spaces}.すなわち,2-ノルムで収束する表示$x=\sum_{j\in J}\al_j e_j$が存在する.
        \item (Parseval identity) $\norm{x}^2=\sum_{j\in J}\abs{\al_j}^2\;\paren{x=\sum_{j\in J}\al_jx_j}$が成り立つ.
    \end{enumerate}
\end{definition}
\begin{proof}
    (3)は,$(x|x)=\sum_{j\in J}\abs{\al_j}^2$であるが,内積$(-|-):H\times H\to\bF$が連続であることより,右辺は収束する.
\end{proof}

\begin{proposition}
    Hilbert空間$H$の任意の正規直交系は,正規直交基底へと拡大できる.
\end{proposition}
\begin{proof}\mbox{}
    \begin{description}
        \item[方針] $\{e_j\}_{j\in J_0}\subset H$を正規直交系とする.
        $\{e_j\}_{j\in J_0}$を含む$H$の正規直交系全体からなる集合は,包含関係について帰納的順序集合を定めるから,Zornの補題より極大元$\{e_j\}_{j\in J}\;J_0\subset J$が存在する.
        これが生成する閉部分空間を$X$としたとき,$X=H$を示せば良い.
        \item[証明] 
        $X\ne H$のとき,$X^\perp\ne 0$だから,ある$e\in X^\perp$が存在して,$\forall_{j\in J}\;(e_j|e)=0,(e|e)=1$を満たす.
        これは$\{e_j\}_{j\in J}$の極大性に矛盾する.
    \end{description}
\end{proof}

\begin{example}\mbox{}
    \begin{enumerate}
        \item $\{\varphi_n(t):=\exp(2\pi int)\}_{n\in\Z}$は複素Hilbert空間$L^2([0,1])$の正規直交基底である.
        この基底についての展開を\textbf{Fourier展開}といい,係数$\lambda_n=(f|\varphi_n)$を\textbf{Fourier係数}という.
        ノルムの等式$\norm{f}^2=\sum_{n\in\Z}\abs{\lambda_n}^2$を\textbf{Plancherelの式}という.
        この基底によるHilbert空間の同型$L^2([0,1])\iso l^2(\Z)$を\textbf{Fourier変換}という.\footnote{SchrödingerとHeisenbergによる2つの量子力学の定式化の等価性の証明もこれによる.物理的には,$f$が$[0,1]$上の波の形状を表す場合,$\lambda_n$は振動数$n$の波の成分を表し,$n$の符号は進行方向に対応する.}
        \item Hilbert空間$L^2(\R)$におけるFourier変換はどう違うのか?
    \end{enumerate}
\end{example}

\begin{theorem}[Gram-Schmidt Orthogonalization Process]
    $\{h_n\}_{n\in\N}$を線型独立な部分集合とする.このとき,正規直交系$\{e_n\}_{n\in\N}$が存在して,$\forall_{n\in\N}\;\brac{e_1,\cdots,e_n}=\brac{h_1,\cdots,h_n}$.
\end{theorem}

\begin{proposition}[Hilbert空間の同型類]\label{prop-characterization-of-isomorphism-of-Hilbert-spaces}
    $H,K$をHilbert空間とし,$\{e_i\}_{i\in I},\{f_j\}_{j\in J}$をそれぞれの正規直交基底とし,$I$と$J$は集合として同型であるとする.
    この時,等長同型$U:H\to K$が存在して,$\forall_{x,y\in H}\;(Ux|Uy)=(x|y)$を満たすものが存在する.
\end{proposition}
\begin{proof}\mbox{}
    \begin{description}
        \item[稠密部分集合上の作用素] 全単射$\gamma:I\to J$をとる.これを用いて,
        作用素$U_0:\sum_{i\in I}\bF_ie_i\to\sum_{j\in J}\bF_jf_j$を
        $U_0x:=\sum_{i\in I}a_if_{\gamma(i)}$と定めると,これは全射な線型作用素で,Parseval恒等式より等長写像である:$\norm{U_0x}=\sum_{i\in I}\abs{\al_i}=\norm{x}$.
        特にこれは連続写像であるから,一意な延長$U:H\to K$を持つ\ref{prop-extension-of-operator-on-dense-subset}.
        これは再び全射であり,極化恒等式\ref{prop-polarization-identity}より,
        \begin{align*}
            4(Ux|Uy)&=\sum_{k=0}^3i^k\norm{U(x+i^ky)}^2\\
            &=\sum_{k=0}^3i^k\norm{x+i^ky}^2=4(x|y).
        \end{align*}
    \end{description}
\end{proof}
\begin{remarks}
    無限次元の可分なHilbert空間は,全て可算な正規直交基底を持つため,全て同型である.
    標準的に$l^2$を考えると良い.これはEuclid空間の非常に自然な一般化となっている.
    Hilbert空間論の今後の主な対象は,その上の作用素となり,これが同型類よりもさらに細かい構成を特徴付ける.
\end{remarks}

\subsection{可分Hilbert空間の特徴付け}

\begin{tcolorbox}[colframe=ForestGreen, colback=ForestGreen!10!white,breakable,colbacktitle=ForestGreen!40!white,coltitle=black,fonttitle=\bfseries\sffamily,
title=]
    可分なHilbert空間では可算無限な正規直交系が取れる.
\end{tcolorbox}

\begin{theorem}[Bessel's inequality]
    $\{e_n\}_{n\in\N}\subset H$を正規直交系とする.$\forall_{h\in H}\;\sum^\infty_{n=1}\abs{\brac{h,e_n}}^2\le\norm{h}^2$.
\end{theorem}

\begin{corollary}\label{cor-well-definedness-of-Bessel's-identity}
    $Z\subset H$を正規直交系とする.$\forall_{h\in H}\;\Brace{e\in Z\mid\brac{h,e}\ne 0}\le\aleph_0$.
    特に,Besselの不等式は任意濃度の正規直交系について成り立つ.
\end{corollary}

\begin{definition}
    族$(h_i)_{i\in I}$の添字集合$I$について,その有限部分集合全体の集合を$\F$とする.$\F$は包含関係について帰納的順序を持つ有向集合である.
    $\paren{h_F:=\sum_{i\in F}h_i}_{F\in\F}$は$H$のネットである.このネットが収束するとき,無限和$\sum_{i\in I}h_i$は収束するとする.
    このように定義すると,系より,$\sum_{e\in Z}\abs{\brac{h,e}}^2\le\norm{h}^2$だから,これは「収束する」と言える.
\end{definition}
\begin{remarks}
    $I$が可算無限集合である場合も,通常の定義$\sum_{n=1}^\infty h_n=\lim_{N\to\infty}\sum_{n=1}^Nh_n$とは一般には一致しない.
\end{remarks}

\begin{theorem}[正規直交基底の特徴付け]
    正規直交系$Z\subset H$について,次の6条件は同値.
    \begin{enumerate}
        \item $Z$は正規直交基底である.
        \item $\forall_{h\in H}\;h\perp Z\Rightarrow h=0$.
        \item $Z$は極大な正規直交系である.
        \item $\forall_{h\in H}\;h=\sum_{e\in Z}\brac{h,e}e$.
        \item $\forall_{g,h\in H}\;\brac{g,h}=\sum_{e\in Z}\brac{g,e}\brac{e,h}$.
        \item (Parseval's identity) $\forall_{h\in H}\;\norm{h}^2=\sum_{e\in Z}\abs{\brac{h,e}}^2$.
    \end{enumerate}
\end{theorem}

\begin{definition}[dimension]
    Hilbert空間$H$の正規直交基底の濃度は等しい\ref{prop-characterization-of-isomorphism-of-Hilbert-spaces}.これを\textbf{次元}$\dim H$で表す.
\end{definition}

\begin{proposition}[the separable Hilbert space]
    $H$を無限次元Hilbert空間とする.次の2条件は同値.
    \begin{enumerate}
        \item $H$は可分である.
        \item $\dim H=\aleph_0$.
    \end{enumerate}
\end{proposition}

\subsection{同型なHilbert空間}

\begin{tcolorbox}[colframe=ForestGreen, colback=ForestGreen!10!white,breakable,colbacktitle=ForestGreen!40!white,coltitle=black,fonttitle=\bfseries\sffamily,
title=Hilbert空間の同型の典型的な例がFourier変換である]
    各基底との内積をFourier係数といい,成分表示をFourier展開という.
    Parsevalの恒等式はRiemann-Lebesgueの補題に特殊化される.

    Banと同様,等長同型(=全単射な等長写像)な線型写像を同型とする.
    なお,距離を保つことと内積を保つことは同値になる.
    等長写像はCauchy列を保つから,等長同型は完備性を保つ.
    Hilbの自己同型をユニタリー作用素という.
\end{tcolorbox}

\begin{theorem}
    $f:\partial\Delta\to\C$を連続関数とする.このとき,列$\{p_n(z,\o{z})\}\subset\C[z,\o{z}]$が存在して,$\partial\Delta$上で一様に$p_n(z,\o{z})\to f(z)$.
\end{theorem}
\begin{remarks}[trigonometric polynomials / Fourier polynomials]
    $\partial\Delta$上で$\o{z}=z^{-1}$だから,$p_n$は$\partial\Delta$で$\sum^n_{k=-m}\al_kz^k$という形の表示を持つ.
    特に,$z=e^{i\theta}$としたとき,この形の関数を三角多項式と呼ぶ.
\end{remarks}

\begin{theorem}
    $\Brace{e_n(t):=\frac{1}{\sqrt{2\pi}}e^{int}}_{n\in\Z}$は$L^2_\C([0,2\pi])$の基底である.
\end{theorem}

各係数$\hat{f}(n):=\brac{f,e_n}=\frac{1}{2\pi}\int^{2\pi}_0f(t)e^{-int}dt$をFourier係数という.
Fourier展開は$L^2$-ノルムで収束する.
\begin{theorem}[Carleson, Hunt]
    $f\in L^p_\C([0,2\pi])\;(1<p\le\infty)$について,$f$のFourier級数は$f$にほとんど至る所収束する.
\end{theorem}
\begin{remarks}
    一方で$L^1_\C([0,2\pi])$でほとんど至る所収束しない関数の例がKolmogorovによってたくさん発見されている.
\end{remarks}

\begin{lemma}[Riemann-Lebesgue lemma]
    $\forall_{f\in L^2_\C([0,2\pi])}\quad\int^{2\pi}_0f(t)e^{-int}dt\to 0\;(n\to\pm\infty)$.
\end{lemma}

\begin{theorem}[Fourier transform]
    $U:L^2_\C([0,2\pi])\to l^2(\Z)$を$Uf=\hat{f}$で定める.これは等長同型な線型作用素である.
\end{theorem}
\begin{remarks}
    $L^2_\C([0,2\pi])\simeq L^2_\C(\partial\Delta)$であるから,これは単位円上の関数のFourier変換についての結果とも読める.
\end{remarks}

\section{Hilbert空間上の作用素}

\begin{tcolorbox}[colframe=ForestGreen, colback=ForestGreen!10!white,breakable,colbacktitle=ForestGreen!40!white,coltitle=black,fonttitle=\bfseries\sffamily,
title=]
    $B(H)$は作用素ノルムと随伴について,単位的な$C^*$-環となる.
    実は$H$はEuclid空間の自然な一般化であまり面白くない.自己準同型の空間$B(H)$が主な対象であり,$H$の幾何学について多くのことを教えてくれる.
    \begin{enumerate}
        \item まず,$B(H)$の$C^*$-環としての構造を定式化し,$*:B(H)\to B(H)$に関する不変部分空間$B(H)_\sa$の元として自己共役作用素とその順序.
        そしてその内部に閉凸錐をなす正作用素とその例としての直交射影を調べる.直交射影への分解を対角化という.
        \item 引き続いて,Hilbert空間の距離構造に注目して,これを保つ自己射としてのユニタリ作用素と,これらすべてを含む正規作用素などのクラスを調べる.ユニタリ作用素への分解が可能であることをRusso-Dye-Gradner定理が保証する.
        \item 最後に,自己共役作用素を調べる際に頻出した構造である数域についてまとめ,そのスペクトルとの関係もみながら,作用素の具体例をまとめる.
    \end{enumerate}
    以上の$B(H)$と$B(H)_\sa$の構造は,一般の$C^*$-代数$A$と$\Re A$の骨組みでもある.
    Hilbert空間上の作用素を調べるのに必要な道具でもある.
\end{tcolorbox}

\begin{notation}
    $H$をHilbert空間,$(-|-)$をその内積,$B(H)$をその上の有界な自己準同型,$I\in B(H)$を恒等写像$\id_H$とする.
\end{notation}

\begin{history}
    この一般化に際して,Hilbertは二次形式/双線型形式の不変式論を04-10に研究していて,\footnote{不変式論のテーマは,BooleからCayleyに引き継がれてから,大陸を渡ってHilbertに届いた.2人ともlogicに入る前は不変式論の研究をしていた.}その時にスペクトル理論を構築したが,
    基底の選択と無限次元行列としての表現,そして積は畳み込みとして成分ごとに計算するというのはあまりにも煩雑であった.
    von Neumannの成功は,補題の同型を渡って,作用素の概念の方に注目したことが大きい.
    これが作用素の研究の第一歩となる.
\end{history}

\subsection{随伴作用素}

\begin{tcolorbox}[colframe=ForestGreen, colback=ForestGreen!10!white,breakable,colbacktitle=ForestGreen!40!white,coltitle=black,fonttitle=\bfseries\sffamily,
title=]
    Hilbert空間では,標準的なペアリングを通じて任意の有界線型作用素を表現できるのであった.
    これは随伴と呼ばれる対合$*:B(H)\to B(H)$を作用素の間に定める.
    こちらの構造はさらに奥が深い.
    これは行列の共役転置の一般化である.
    またこうして自己共役性の概念が内積から作用素へ流入する.
\end{tcolorbox}

\begin{lemma}[有界作用素の内積による特徴付け]\label{lemma-correspondence-between-sesquilinearform-and-operator}
    Hilbert空間$H$上の有界線形作用素と,これが内積$(-|-)$を通じて定める有界な半双線型形式との
    次の対応は等長同型である:
    \[\xymatrix@R-2pc{
        B(H)\ar[r]&\{\brac{-|-}\in\Map(H\times H,\bF)\mid\brac{-|-}は半双線型\}\\
        \rotatebox[origin=c]{90}{$\in$}&\rotatebox[origin=c]{90}{$\in$}\\
        T\ar@{|->}[r]&B_T(x,y):=(x|Ty)
    }\]
    ただし,終域となっている空間のノルムは作用素ノルム$\norm{B_T}:=\sup\Brace{\abs{B_T(x,y)}\in\R_+\mid\norm{x}\le 1,\norm{y}\le1}$とする.
\end{lemma}
\begin{proof}\mbox{}
    \begin{description}
        \item[$\to$がwell-definedな等長写像] 
        $T\in B(H)$ならば,$B_T:=(-,T-)$は明らかに半双線型形式である.
        あとは$B_T$が有界であり,$\norm{T}=\norm{B_T}$であることを示せば良い.

        まず,Cauchy-Schwarzの不等式より,
        \begin{align*}
            \norm{B_T}&=\sup\Brace{\abs{B_T(x,y)}\in\R\mid\norm{x}\le1,\norm{y}\le1}\\
            &=\sup\Brace{\abs{(x|Ty)}\in\R\mid\norm{x}\le1,\norm{y}\le1}\le\norm{T}
        \end{align*}
        よって$B_T$は有界である.次に,任意の$x\in H$について,
        \begin{align*}
            \norm{Tx}^2&=(Tx|Tx)=B_T(Tx,x)=B_T\paren{\norm{Tx}\frac{Tx}{\norm{Tx}},\norm{x}\frac{x}{\norm{x}}}\\
            &\le\norm{B_T}\norm{Tx}\norm{x}\le(\norm{B_T}\norm{T})\norm{x}^2
        \end{align*}
        と評価できるから,$\norm{T}^2\le\norm{B_T}\norm{T}$より,$\norm{T}\le\norm{B_T}$.
        \item[$\leftarrow$がwell-definedな切断] 
        $B$を$H$上の有界な半双線型形式とする.任意の$y\in H$に対して,$B(-,y)\in H^*$だから,Rieszの表現定理より,$\exists!_{Ty\in H}\;B(-,y)=(-|Ty)$.
        こうして写像$T:H\to H;y\mapsto Ty$が定まる.これが有界な線形作用素であることと,$\rightarrow\circ\leftarrow=\id$であることを示せば良い.

        線形性は明らか:$B(-,ay)=\o{a}B(-,y)=\o{a}(-|Ty)=(-|a\cdot Ty)$.
        有界性は,既に行った評価$\forall_{y\in H}\;\norm{Ty}^2\le\norm{B}\norm{T}\norm{y}^2$より,$\norm{T}\le\norm{B}$で抑えられる.
        よって,$T\in B(H)$.
        また,この有界作用素$T$が定める半双線型形式は$B_T=B$に他ならない.
    \end{description}
\end{proof}

\begin{theorem}[随伴作用素の存在]\mbox{}\label{thm-existence-of-adjoint-operator}
    \begin{enumerate}
        \item 任意の$T\in B(H)$に対して,$\forall_{x,y\in H}\;(Tx|y)=(x|T^*y)$を満たす$T^*\in B(H)$が唯一つ存在する.
        \item これが定める全単射な対応${}^*:B(H)\to B(H)$について,
        \begin{enumerate}[(a)]
            \item 対合的(周期2)である:$T^{**}=T$.
            \item 共役線型である:$(aT)^*=\o{a}T^*$.
            \item 乗法について反変的である:$(ST)^*=T^*S^*$.\footnote{antimultiplicativeと表現されている.homomorphismが積を保つのに対し,antihomomorphismとは積を逆にする.}
            \item 等長写像である:$\norm{T}=\norm{T^*}$.
            \item $\norm{T^*T}=\norm{T}^2$を満たす.
        \end{enumerate}
    \end{enumerate}
\end{theorem}
\begin{proof}\mbox{}
    \begin{enumerate}
        \item 任意の$T\in B(H)$に対して,$(T-|-):H^2\to\bF$は有界な半双線型形式であるから,ただ一つの$T^*\in B(H)$が存在して$(T-|-)=(-|T^*-)$を満たす.
        \item \begin{enumerate}[(a)]
            \item 内積は自己共役であるから,任意の$x,y\in H$について,
            \begin{align*}
                (x|T^{**}y)&=(T^*x|y)=\o{(y|T^*x)}\\
                &=\o{(Ty|x)}=(x|Ty).
            \end{align*}
            より,$T,T^{**}$は同一の半双線型形式を定める.よって,$T=T^**$.
            \item 同様にして,
            \begin{align*}
                \forall_{x,y\in H}\;\quad(x|(aT)^*y)=(aTx|y)=a(Tx|y)=(x|\o{a}T^*y)
            \end{align*}
            より,$(aT)^*=\o{a}T^*$.
            \item \begin{align*}
                \forall_{x,y\in H}\quad(x|(ST)^*y)&=(STx|y)=(Tx|S^*y)=(x|T^*(S^*(y)))
            \end{align*}
            \item 作用素ノルムの劣乗法性より,$\norm{T^*T}\le\norm{T}\norm{T^*}$である.また,
            \begin{align*}
                \forall_{x\in H}\quad\norm{Tx}^2=(Tx|T^{**}x)=(T^*Tx|x)\le\norm{T^*T}\norm{x}^2
            \end{align*}
            より,$\norm{T}^2\le\norm{T^*T}(\le\norm{T}\norm{T^*})$でもある.よって2つ併せて,$\norm{T}\le\norm{T^*}(\le\norm{T^{**}}=\norm{T})$を得るから,$\norm{T}=\norm{T^*}$.
            \item $\norm{T^*T}=\norm{T}^2$は,$B$を$H$の閉単位球として$K=T(B)$とおくと,
            $\sup_{y\in K}\abs{T^*y}=\sup_{y\in K}\abs{Ty}$と見ると明らか.
        \end{enumerate}
    \end{enumerate}
\end{proof}

\subsection{自己共役作用素}

\begin{tcolorbox}[colframe=ForestGreen, colback=ForestGreen!10!white,breakable,colbacktitle=ForestGreen!40!white,coltitle=black,fonttitle=\bfseries\sffamily,
title=]
    自己共役作用素はノルムに対して特殊な振る舞いをする.
    特に,作用素ノルムの特徴付け$\norm{T}=\sup_{x\in\partial B}\abs{(Ax|x)}\;(T\in B(H)_\sa)$は本質的であり,
    自己共役作用素を調べるのに,半内積$(T-|-)$の構造と,これについての一般化Cauchy-Schwartzの不等式$\abs{(Ax|y)}\le\sqrt{(Ax|x)(Ay|y)}$を多用する.
\end{tcolorbox}

\begin{definition}[self-adjoint / hermitian]\mbox{}
    \begin{enumerate}
        \item 乗法についての反準同型$*:B(H)\to B(H)$が対合な共役線型写像で,等長同型でもあるとき,$(B(H),*)$を$B^*$-代数という.$C^*$-性$\norm{T^*T}=\norm{T}^2$も満たすとき,$C^*$-代数という.\footnote{最初に定義したI. E. Segal in 1947で,$C$は"Closed"から取られた.\url{https://en.wikipedia.org/wiki/C*-algebra}}
        \item $T\in B(H)$が$T=T^*$を満たすとき,これを自己共役作用素という.これを$B_\sa(H)=B(H)^{\Brace{*}}$\footnote{群作用に対する不変式の記法を踏襲した.}と表すと,$B(H)$の閉な実部分空間となる.
    \end{enumerate}
\end{definition}

\begin{example}[随伴]\mbox{}
    \begin{enumerate}
        \item 乗算作用素\ref{operator-multiplication}については$M^*_\varphi=M_{\o{\varphi}}$となる.よって,$\varphi\in L^\infty(^mu)$が実数値である場合に限り,自己共役である.$\abs{\phi}=1$である場合に限り,ユニタリである.
        \item $k$を核とする積分作用素$K$\ref{operator-integral-transformation}については$K^*$は$k^*(x,y)=\o{k(y,x)}$を核とする積分作用素となる.よって,$k(x,y)=\o{k(y,x)}\;\ae[\mu\times\mu]$である場合に限り,自己共役である.
        \item shift作用素\ref{operator-unilateral-shift}は方向が逆になる:$S^*(\al_1,\al_2,\cdots)=(\al_2,\al_3,\cdots)$.これを\textbf{後方シフト(backward shift)}と呼ぶ.
        \item 複素Hilbert空間$H$上の作用素$A\in B(H)$について,$B:=\frac{A+A^*}{2},C:=\frac{A-A^*}{2}$をそれぞれ実部と虚部と呼び,自己共役である.
    \end{enumerate}
\end{example}

\begin{corollary}[自己共役作用素の特徴付け]
    $\bF=\C$のとき,次の2条件は同値.
    \begin{enumerate}
        \item $T=T^*$.
        \item $\forall_{x\in H}\;(Tx|x)\in\R$.
    \end{enumerate}
\end{corollary}
\begin{remark}
    極化恒等式による証明\ref{lemma-characterization-of-self-adjointness}参照.
    $H$が実Hilbert空間という仮定のみでは,その上の任意の作用素$A\in B(H)$に対して$\brac{Ah,g}\in\R$が常に成り立つので,特徴付けにならないことに注意.
\end{remark}

\begin{proposition}[自己共役作用素の作用素ノルム]\label{prop-operator-norm-of-self-adjoint-operator}
    $A$を自己共役作用素とする.このとき,作用素ノルムは
    \[\norm{A}=\sup\Brace{\abs{(Ax|x)}\in\R\mid\norm{x}=1}=\sup_{x\ne0}\frac{\abs{(Ax|x)}}{\norm{x}^2}.\]
    とも表せる.
\end{proposition}
\begin{remarks}[数域]
    これはRayleigh商の値域になっている.数域半径と深い関係にある.
\end{remarks}

\begin{corollary}[半内積の非退化性]\label{cor-nondegeneratedness-of-semi-inner-product}
    $A$を自己共役作用素とする.このとき,$\forall_{x\in H}\;(Ax|x)=0\Rightarrow A=0$.
    なお,$H$が複素Hilbert空間ならば,一般の$A$について成り立つ.
\end{corollary}
\begin{proof}\mbox{}
    \begin{enumerate}
        \item 自己共役作用素の作用素ノルムの性質\ref{prop-operator-norm-of-self-adjoint-operator}により,$\forall_{x\in H}\;(Ax|x)=0\Rightarrow\norm{A}=0\Rightarrow A=0$が従うことによる.
        \item ??
    \end{enumerate}
\end{proof}

\subsection{位相線形空間としての同型}

\begin{tcolorbox}[colframe=ForestGreen, colback=ForestGreen!10!white,breakable,colbacktitle=ForestGreen!40!white,coltitle=black,fonttitle=\bfseries\sffamily,
title=]
    ${}^*:B(H)\to B(H)$というのはある種${}^\op$のように使える.
    作用素の可逆性は,随伴の消息を以て特徴付けることが出来る.
\end{tcolorbox}

\begin{proposition}\label{prop-Ker-of-adjoint-operator}
    任意の$T\in B(H)$に対して,$\Ker T^*=(\Im T)^\perp$.
\end{proposition}
\begin{proof}
    $\forall_{x,y\in H}\;(x|T^*y)=(Tx|y)$の下で,
    $y\in\Ker T^*\Rightarrow (T(H)|y)=0$より,$y\in(T(H))^\perp$.
    逆に$y\in(T(H))^\perp\Rightarrow (H|T^*y)=0\Rightarrow T^*y\in H^\perp=\{0\}$.
\end{proof}
\begin{remark}
    双対的な関係は$(\Ker A)^\perp=\oo{\Im A^*}$までしか成り立たない.
\end{remark}

\begin{proposition}\label{prop-characterization-of-invertibleness-of-operator}
    $T\in B(H)$について,次の6条件は同値.
    \begin{enumerate}
        \item $T$は可逆:$T^{-1}\in B(H)$.
        \item $T^*$は可逆.
        \item $T,T^*$はbounded away from zero:$\exists_{\ep>0}\;\forall_{x\in H}\;\norm{Tx}\ge\ep\norm{x}$.%$\inf d(0,\Im T)>0$.
        \item $T,T^*$は単射で,$\Im T$はノルム閉.
        \item $T$は全単射.
        \item $T,T^*$は全射.
    \end{enumerate}
\end{proposition}
\begin{proof}\mbox{}
    \begin{description}
        \item[(1)$\Leftrightarrow$(2)] $(T^{-1}T)=(TT^{-1})=I$であるとき,${}^*:B(H)\to B(H)$の劣乗法性より$(T^{-1})^*$が$T^*$の逆射である.また$T^*T^{*-1}=T^{*-1}T^*=I$のときも同様に$*$を作用させれば良い.
        \item[(1)$\Rightarrow$(3)] 任意の$x\in H$について$\norm{x}=\norm{T^{-1}Tx}\le\norm{T^{-1}}\norm{Tx}$より,$\ep:=\norm{T^{-1}}^{-1}$と取れば良い.
        \item[(3)$\Rightarrow$(4)] $\exists_{\ep>0}\;\forall_{x\in H}\;\norm{Tx}\ge\ep\norm{x}$は特に$\norm{Tx-Ty}\ge\ep\norm{x-y}$より,単射性$Tx=Ty\Rightarrow x=y$を含意する.またこれより,$T(H)$の任意のCauchy列$(Tx_i)_{i\in\N}$は$H$上のCauchy列$(x_i)_{i\in H}$を定めることより,$Tx$に$T(H)$は完備で,特に閉集合.
        \item[(4)$\Rightarrow$(5)] ノルム閉包の直交補空間による特徴付け\ref{cor-expression-of-closed-linear-span}より,$T(H)=\dbloverline{T(H)}=(T(H)^\perp)^\perp$.命題より,$(T(H)^\perp)^\perp=(\Ker T^*)^\perp=0^\perp=H$.
        \item[(5)$\Rightarrow$(1)] 開写像定理の系\ref{cor-inverse-mapping-theorem}より.
        \item[(6)$\Rightarrow$(5)] $\Ker T=(T^*(H))^\perp=0$より.
        \item[(1)$\Rightarrow$(6)] 明らか.
    \end{description}
\end{proof}

\subsection{Hilbの射}

\begin{tcolorbox}[colframe=ForestGreen, colback=ForestGreen!10!white,breakable,colbacktitle=ForestGreen!40!white,coltitle=black,fonttitle=\bfseries\sffamily,
title=]
    Hilbの射は等長同型(=全写な等長写像)であった.
    一般にMetでは計量写像(short map)を射とする.
    そこでは,等長写像がモノ射となる.
\end{tcolorbox}

\begin{proposition}[モノ射の内積による特徴付け]\label{prop-characterization-of-isometry}
    $V:H\to K$を線型写像とする.
    次の2条件は同値.
    \begin{enumerate}
        \item $V$は等長写像である.
        \item $\forall_{h,g\in H}\;(Vh|Vg)=(h|g)$.
    \end{enumerate}
\end{proposition}
\begin{proof}\mbox{}
    \begin{description}
        \item[(2)$\Rightarrow$(1)] $\forall_{h\in H}\;\norm{Vh}^2=(Vh|Vh)=(h|h)=\norm{h}^2$より,$V$は等長である.
        \item[(1)$\Rightarrow$(2)] $\forall_{\lambda\in\bF}\;\norm{h+\lambda g}^2=\norm{Vh+\lambda Vg}^2$について,極化恒等式を用いると,
        \[\norm{h}^2+2\Re\o{\lambda}(h|g)+\abs{\lambda}^2\norm{g}^2=\norm{Vh}^2+2\Re\o{\lambda}(Vh|Vg)+\abs{\lambda}^2\norm{Vg}^2\]
        より,$\forall_{\lambda\in\bF}\;\Re\o{\lambda}(h|g)=\Re\o{\lambda}(Vh|Vg)$.
        $\bF=\R$のとき,$\lambda=1$とすると結論を得る.$\bF=\C$のとき,$\lambda=1,i$とすると結論を得る.
    \end{description}
\end{proof}
\begin{remarks}
    内積とノルムをつなぐものが極化恒等式である.
\end{remarks}

\subsection{正規作用素}

\begin{tcolorbox}[colframe=ForestGreen, colback=ForestGreen!10!white,breakable,colbacktitle=ForestGreen!40!white,coltitle=black,fonttitle=\bfseries\sffamily,
title=]
    正規であるという可換性条件は,距離的な特徴付けがある.
    随伴とノルム的に判別不可能ならば,正規である.
    スペクトル定理は,正規作用素なるクラスが正確に対角化可能であることを主張する.
\end{tcolorbox}

\begin{definition}[normal]
    作用素$T\in B(H)$について,次の2条件は同値.$\F=\C$のとき,(3)も同値.
    \begin{enumerate}
        \item $T^*T=TT^*$.
        \item (metrically identical) $\norm{Tx}=\norm{T^*x}$.
        \item $T$の実部と虚部は可換である.
    \end{enumerate}
    この同値な条件を満たすとき,$T$を\textbf{正規作用素}という.
\end{definition}
\begin{proof}\mbox{}
    \begin{description}
        \item[(1)$\Rightarrow$(2)] $\norm{Tx}=(T^*Tx|x)^{1/2}=(TT^*x|x)^{1/2}=\norm{T^*x}$.
        \item[(2)$\Rightarrow$(1)] 極化恒等式から$\forall_{x,y\in H}\;(T^*Tx|y)=(TT^*x|y)$を得る.具体的には,
        \[\norm{Tx}^2-\norm{T^*x}^2=(Tx|Tx)-(T^*x|T^*x)=((TT^*-T^*T)x|x)\]
        であるが,交換子作用素$TT^*-T^*T$は自己共役であるから,任意の$x\in H$について,この式が常に$0$になるのは$TT^*=T^*T$と等価.
        \item[(1)$\Leftrightarrow$(3)] $A=B+iC$とおく.
        \begin{align*}
            A^*A&=B^2+C^2+(BC-CB)i\\
            AA^*&=B^2+C^2+(CB-BC)i
        \end{align*}
        より,$A^*A=AA^*\Leftrightarrow BC=CB$.
    \end{description}
\end{proof}

\begin{example}
    自己共役作用素とユニタリ作用素は正規である.多くの書籍はスペクトル分解をこのクラスに限る.
    また,半正定値作用素\ref{def-positive-operator}も正規である.
\end{example}

\begin{proposition}[等長写像の特徴付け]\label{prop-characterization-of-isometry-2}
    $A\in B(H)$について,次の3条件は同値.
    \begin{enumerate}
        \item $A$は等長写像である.
        \item $A^*A=I$.
        \item $\forall_{h,g\in H}\;(Ah|Ag)=(h|g)$.
    \end{enumerate}
\end{proposition}
\begin{proof}\mbox{}
    \begin{description}
        \item[(1)$\Leftrightarrow$(3)] 等長写像の特徴付け\ref{prop-characterization-of-isometry}に他ならない.
        \item[(2)$\Leftrightarrow$(3)] $\forall_{h,g\in H}\;(A^*Ah|g)=(Ah,Ag)$より,$A^*A=I$と同値.
    \end{description}
\end{proof}

\begin{proposition}[同型の特徴付け]
    $A\in B(H)$について,次の3条件は同値.
    \begin{enumerate}
        \item $A^*A=AA^*=I$.
        \item $A$は全写な等長写像,すなわちユニタリである.
        \item $A$は正規な等長写像である.
    \end{enumerate}
\end{proposition}
\begin{proof}\mbox{}
    \begin{description}
        \item[(1)$\Rightarrow$(2)] 等長写像の特徴付け\ref{prop-characterization-of-isometry-2}より$A$は等長で,また可逆であるから特に全写.
        \item[(2)$\Rightarrow$(3)] $A$について,ユニタリならば等長だから,$A^*A=I$.また一般に,全写な等長写像$A$の逆写像$A^{-1}$も全写な等長写像である.よって,$(A^{-1})^*A^{-1}=I\Leftrightarrow(AA^*)^{-1}=I$.
        \item[(3)$\Rightarrow$(1)] 等長写像の特徴付け\ref{prop-characterization-of-isometry-2}より,$A^*A=I$.$A$は正規だから,$AA^*=I$も成り立つ.
    \end{description}
\end{proof}

\begin{corollary}
    $A$が自己共役ならば,$\forall_{h\in H}\;(Ah|h)=0\Rightarrow A=0$.
\end{corollary}

\begin{corollary}[正規作用素の可逆性]\label{prop-characterization-of-invertibleness-of-normal-operator}
    正規作用素$T\in B(H)$について,次の2条件は同値.
    \begin{enumerate}
        \item $T$は可逆.
        \item $T$ is bounded away from zero.
    \end{enumerate}
\end{corollary}
\begin{proof}
    作用素の可逆性の特徴付け\ref{prop-characterization-of-invertibleness-of-operator}と,正規性の特徴付け$\norm{Tx}=\norm{T^*x}$より.
\end{proof}

\subsection{半正定値作用素}

\begin{tcolorbox}[colframe=ForestGreen, colback=ForestGreen!10!white,breakable,colbacktitle=ForestGreen!40!white,coltitle=black,fonttitle=\bfseries\sffamily,
title=]
    半正定値作用素は,density matrix formalismを通じて,量子状態を表す.
    自己共役作用素のうち,特別なクラスである.
    大雑把には,複素平面内で,自己共役作用素は実数,正作用素は非負実数と見れる.
\end{tcolorbox}

\begin{definition}[positive (semi-definite)]\label{def-positive-operator}
    $T\in B(H)$が\textbf{半正定値}であるとは,$T=T^*$かつ$\forall_{x\in H}\;(Tx|x)\ge 0$を満たすことをいう.\footnote{2つ条件があるように見えるが,$\bF=\C$かつ$T$が$H$全域で定義されているとき,半正定値性は自己共役性を含意する\ref{lemma-characterization-of-self-adjointness}.これは極化恒等式による.}
    これを「\textbf{作用素$T$は正である}」と省略して良い,$T\ge 0$と表す.正定値であることは$T>0$で表す.
\end{definition}

\begin{lemma}[正作用素の閉凸錐]\label{lemma-positive-closed-cone-of-positive-operator}
    $T_1,T_2\in B(H)$を半正定値とする.
    \begin{enumerate}
        \item $T_1+T_2$も半正定値:半正定値は作用素は凸錐をなす.
        \item 正作用素の凸錐は閉集合である.
        \item $T_1T_2$は一般には半正定値とも自己共役とも限らないが,この積が自己共役であるとき,正である.
    \end{enumerate}
\end{lemma}
\begin{proof}\mbox{}
    \begin{enumerate}
        \item $\forall_{x\in H}\;(T_1x+T_2x|x)=(T_1x|x)+(T_2x|x)\ge0$.
        \item 閉性は,Cauchy列$(T_n)$を取ることによる.極限$T:=\lim_{n\to\infty}T_n$は正ではないと仮定する:$\exists_{x\in H}\;(Tx|x)<0$.$\ep:=-(Tx|x)>0$とおくと,$\exists_{N\in\N}\;\forall_{n\ge N}\;\norm{T_n-T}\le\ep/\norm{x}^2$だから,これについて,
        \[\Abs{(T_nx|x)-(Tx|x)}=\abs{((T_n-T)x|x)}\le\ep.\]
        よって,$T_n$が正であることに矛盾.
        \item 二乗根補題\ref{prop-square-root-lemma}より,$ST=(S^{1/2})^2T=S^{1/2}TS^{1/2}\ge S^{1/2}0S^{1/2}=0$.
        $\forall_{x\in H}\;(ABx|x)=(\sqrt{A}\sqrt{A}Bx|x)=(B\sqrt{A}x|\sqrt{A})\ge0$と言ってもよい.
    \end{enumerate}
\end{proof}


\subsection{自己共役作用素の順序}

\begin{tcolorbox}[colframe=ForestGreen, colback=ForestGreen!10!white,breakable,colbacktitle=ForestGreen!40!white,coltitle=black,fonttitle=\bfseries\sffamily,
title=]
    幾何学的には,自己共役作用素のなす閉・実・順序部分空間$B(H)_\sa$の内部で閉凸錐をなす.
    しかしこの順序は束にさえならず,より深い考察には二乗根補題を必要とする.
\end{tcolorbox}

\begin{proposition}[ordering by positive operator]\mbox{}
    \begin{enumerate}
        \item 自己共役な作用素は,$B(H)$の閉な実部分空間をなす.これを$B(H)_\sa$と表す.
        \item $B(H)_\sa$は,順序$S\le T:\Leftrightarrow T-S\ge 0$を備える.
    \end{enumerate}
\end{proposition}
\begin{proof}\mbox{}
    \begin{enumerate}
        \item $0\in B(H)$.$*:B(H)\to B(H)$は共役線型であるから,実数倍は関係$T=T^*$を変えず,和についても閉じている.
        また,ノルム閉性については,$(T_n)$を$B(H)_\sa$のCauchy列とすると,収束先$T:=\lim_{n\to\infty}T_n$を持つ.$*:B(H)\to B(H)$は等長写像だから,像$(T_n^*)$もCauchy列で,収束先$T^*$を持ち,$*:B(H)\to B(H)$は特に連続だから,$T=\lim_{n\to\infty}T_n=\lim_{n\to\infty}T_n^*=T^*$.
        \item \begin{enumerate}[(a)]
            \item $S\le S\Leftrightarrow S-S=0\ge 0$は,零写像は半正定値だから,常に成り立つ.
            \item $S\le T$かつ$T\le S$のとき,$\forall_{x\in H}\;(Tx|x)=(Sx|x)$であるから,$T=S$.これは,自己共役作用素の作用素ノルムの性質\ref{prop-operator-norm-of-self-adjoint-operator}により,$\forall_{x\in H}\;(Ax|x)=0\Rightarrow\norm{A}=0\Rightarrow A=0$が従うことによる.
            \item 推移律は明らか.
        \end{enumerate}
    \end{enumerate}
\end{proof}

\begin{lemma}[数域:自己共役作用素に特徴的な対象]\label{lemma-numerical-range}
    写像
    \[\xymatrix@R-2pc{
        B:B(H)_\sa\ar[r]&\Map(H,\R)\\
        \rotatebox[origin=c]{90}{$\in$}&\rotatebox[origin=c]{90}{$\in$}\\
        T\ar@{|->}[r]&(T-|-)=:B_T(-)
    }\]
    を考える.
    \begin{enumerate}
        \item これは線型社像ではない,計量写像である.特に,有界.
        \item 単射である.
    \end{enumerate}
\end{lemma}
\begin{proof}
    等長同型であることを示せば,特に有界であること,すなわちこの写像のwell-definednessが従う.

    まず,$\forall_{x\in B}\;\norm{(Tx|x)}\le\norm{Tx}\norm{x}\le\norm{x}^2\norm{T}\le\norm{T}$より,$\norm{(T-|-)}\le\norm{T}$.
    逆は,任意の$x\in H$について,$(Tx|Tx)$を合成$x\mapsto Tx\mapsto (Tx|Tx)$と見ると,
\end{proof}

\begin{proposition}[自己共役作用素の順序]\label{prop-order-of-self-adjoint-operator}
    $S,T\in B(H)_\sa$について,
    \begin{enumerate}
        \item $S\le T$ならば,$\forall_{A\in B(H)}\;A^*SA\le A^*TA$.
        \item $0\le S\le T$ならば,$\norm{S}\le\norm{T}$.特に,正作用素$S\ge$について,$S\le I$と$\norm{S}\le1$とは同値.
        \item $-I\le T\le I$と$\norm{T}\le 1$は同値.
    \end{enumerate}
\end{proposition}
\begin{proof}
    $\bF=\R,\C$のいずれの場合も,自己共役作用素$T$について,$\forall_{x\in H}\;(Tx|x)\in\R$を満たすことに注意.
    \begin{enumerate}
        \item \begin{align*}
            \forall_{y\in H}\;((T-S)y|y)\ge0&\Leftrightarrow\forall_{x\in H}\;((T-S)Ax|Ax)\ge0\\
            &\Leftrightarrow\forall_{x\in H}\;(A^*(T-S)Ax|x)\ge0.
        \end{align*}
        なお,$A^*TA-A^*SA=A^*(T-S)A$は線型写像についてしか成り立たないことに注意.
        \item 半内積$(S-|-):H\times H\to\bF$についてのCauchy-Schwartzの不等式,仮定$0\le S\le T$より,任意の単位ベクトル$x,y\in\partial B$について,
        \begin{align*}
            \abs{(Sx|y)}^2\le(Sx|x)(Sy|y)\le(Tx|x)(Ty|y)\le\norm{T}^2
        \end{align*}
        であるから,$\norm{S}^2\le\norm{T}^2$が従う.なお,左辺の上限が$\norm{S}^2$に等しいことは,等長同型\ref{lemma-correspondence-between-sesquilinearform-and-operator}による.
        特に,$T=I$とすると,$0\le S\le I\Leftrightarrow\norm{S}\le 1$.
        \item $\norm{T}\le 1$ならば,Cauchy-Schwartzの不等式より$\abs{(Tx|x)}\le\norm{x}^2=(x|x)$.
        絶対値の場合分けより,$-I\le T\le I$が従う.
        よって,$-I\le T\le I$を仮定して$\norm{T}\le 1$を導く.
        仮定より,
        \begin{align*}
            (T(x+y)|x+y)&\le(I(x+y)|x+y),&(T(x-y)|x-y)&\ge(-I(x-y)|x-y)\\
            \Leftrightarrow\quad(T(x+y)|x+y)&\le\norm{x+y}^2,&-(T(x-y)|x-y)&\le\norm{x-y}^2.
        \end{align*}
        これと中線定理を併せて,特に単位ベクトル$x,y\in\partial B$について,
        \begin{align*}
            &(T(x+y)|x+y)-(T(x-y)|x-y)&\le &2(\norm{x}^2+\norm{y}^2)\\
            \Leftrightarrow\quad&4\Re(Tx|y)&\le&4.
        \end{align*}
    \end{enumerate}
    $x,y\in\partial B$は任意に取ったから,等長同型対応\ref{lemma-correspondence-between-sesquilinearform-and-operator}を通じて,
    $\norm{T}\le 1$を得る.
\end{proof}

\begin{theorem}[有界単調収束定理\cite{Eidelman}]
    $A_0\le A_1\le\cdots\le A_n\le\cdots\le A$を$B(H)_\sa$の有界な単調増加列とする.
    このとき,ある有界作用素$B$が存在して,これに各点収束する.
\end{theorem}

\begin{proposition}[\cite{Eidelman}]
    $A\in B(H)_\sa$は$\exists_{m,M\in\R}\;m I\le A\le MI$を満たすとし,$P\in\R[x]$を$\forall_{z\in[m,M]}\;P(z)\ge0$を満たす多項式とする.
    このとき,$P(A)\ge0$.
\end{proposition}

\subsection{二乗根補題}

\begin{tcolorbox}[colframe=ForestGreen, colback=ForestGreen!10!white,breakable,colbacktitle=ForestGreen!40!white,coltitle=black,fonttitle=\bfseries\sffamily,
title=]
    正作用素は正規作用素の代表例だが,その正規たる所以を解明する.
    基本的にはスペクトル定理の帰結でもある.

    正作用素は凸錐をなすが,その中に二乗根が必ず見つかる.
    可逆性は,順序構造を用いて特徴付けることが出来る.
    また正作用素の逆は再び正である.
\end{tcolorbox}

\begin{lemma}
    正係数を持つ多項式の列$\{p_n\}\subset\R_{>0}[x]$であって,
    和$\sum_{n\in\N}p_n$が$[0,1]$上で関数$t\mapsto 1-(1-t)^{1/2}$に一様収束するものが存在する.
\end{lemma}

\begin{proposition}[square root lemma]\label{prop-square-root-lemma}
    半正定値な作用素$T\in B(H)$について,
    \begin{enumerate}
        \item ただ一つの半正定値な作用素$T^{1/2}\ge0$が存在して,$(T^{1/2})^2=T$を満たす.
        \item $A\in B(H)$が$T$と可換ならば,$T^{1/2}$と可換である.
    \end{enumerate}
\end{proposition}
\begin{proof}\mbox{}
    \begin{enumerate}
        \item 
        \begin{description}
            \item[存在] 任意の$\al\ge 0$に対して,$(\al T)^{1/2}:=\al^{1/2}T^{1/2}$と対応させれば良いから,$T\le I$を仮定して存在を示せば良い.
            $S:=I-T$とすると,$0\le S\le I$を満たす.
            いま.補題の多項式列$\{p_n\}\subset\R_{\ge 0}[x]$に対して$S_n:=p_n(S)$と定めると,和$\sum_{n\in\N}p_n$は一様収束するから,
            \[\forall_{\ep>0}\;\exists_{n_0\in\N}\;\forall_{m\ge n_0}\;\forall_{t\in[0,1]}\;0\le\sum_{n=n_0}^mp_n(t)=\sum\al_kt^k\le\ep.\]
            $t=S$を代入すると,
            \[\Abs{\sum^m_{n=n_0}S_n}\le\sum\al_k\norm{S^k}\le\sum\al_k\le\ep.\]
            よって,$\sum S_n$はCauchy列だから,$0\le R\le I$を満たすある作用素$R$に一様収束する.
            これについて,$T^{1/2}:=I-R$と定めれば良い.
            \item[一意性]
        \end{description}
        \item $T^{1/2}$は,$I,T$からなる多項式の一様収束極限として構成したから,$T$と可換ならば$T^{1/2}$とも可換である.
    \end{enumerate}
\end{proof}

\begin{corollary}
    $A\ge 0$かつ$(Ax,x)=0$ならば,$Ax=0$.
\end{corollary}
\begin{proof}
    $A$の二乗根$X:=\sqrt{A}$を取ると,$(Ax|x)=(Xx|Xx)=0$.内積の非退化性より,$Xx=0$.よって,$Ax=0$.
\end{proof}
\begin{remark}
    一般の場合は\ref{cor-nondegeneratedness-of-semi-inner-product}に示してあるが,正作用素については二乗根補題より直接従う.
\end{remark}

\begin{proposition}[正作用素の可逆性]
    半正定値な作用素$T\in B(H)$について,
    \begin{enumerate}
        \item $T$が可逆であることと,$\exists_{\ep>0}\;T\ge\ep I$は同値.
        \item $T$が可逆であるとき,逆作用素も正$T^{-1}\ge 0$で,$T^{1/2}$も可逆で,$(T^{-1})^{1/2}=(T^{1/2})^{-1}=:T^{-1/2}$.
        \item 可逆な$T$に対して$T\le S$ならば$S$も可逆で,$S^{-1}\le T^{-1}$.
    \end{enumerate}
\end{proposition}
\begin{proof}\mbox{}
    \begin{enumerate}
        \item \begin{description}
            \item[$\Leftarrow$] $T\ge\ep I$と仮定する.すると,2つの作用素$T-\ep I,T+\ep I$はいずれも正で,互いに可換であるから,積$T^2-\ep^2I$も正である:$T^2\ge\ep^2I$.
            よって,
            \[\norm{Tx}^2=(T^2x|x)\ge\ep^2(x|x)=\ep^2\norm{x}^2.\]
            すなわち,$0$に対して有界であるから,$T$は可逆\ref{prop-characterization-of-invertibleness-of-operator}.
            \item[$\Rightarrow$]
            $T$が可逆ならば$0$に対して有界であるから,$\exists_{\ep>0}\;(T^2x|x)\norm{Tx}^2\ge^2\norm{x}^2=\ep^2(x|x)$より,$T^2\ge\ep^2I$.
            いま,$\Leftarrow$より$T+\ep I$は可逆であり,その逆も正((2)の結果を先に用いた).
            すると,$T-\ep I=(T^2-\ep^2I)(T+\ep I)^{-1}$と見ると,2つの可換な正作用素の積だから,これも正である.
        \end{description}
        \item $\forall_{x\in H}\;(T^{-1}Tx|Tx)=(Tx|x)\ge0$であるから,$T$が可逆であるとき特に$\Im T=H$であることと併せると,$T^{-1}\ge0$を得る.
        (1)の$\Rightarrow$で$T^2\ge\ep^2I$から$T\ge\ep I$を導いた議論を繰り返すことにより,$T^{1/2}\ge\ep^{1/2}I$を得るから,(1)より可逆で正な逆作用素$(T^{1/2})^{-1}$を持つ.
        二乗根の一意性より,$(T^{1/2})^{-1}=(T^{-1})^{1/2}$.
        \item $T\le S$とすると,(1)より$S$も可逆で,また(2)より$S^{-1/2}$も存在する.
        自己共役作用素の順序の性質\ref{prop-order-of-self-adjoint-operator}(1)より,$S^{-1/2}TS^{-1/2}\le S^{-1/2}SS^{-1/2}=I$.
        $C^*$-性と自己共役作用素の順序の性質\ref{prop-order-of-self-adjoint-operator}(2)より,
        \[\norm{T^{1/2}S^{-1/2}}^2=\norm{(T^{1/2}S^{-1/2})^*T^{1/2}S^{-1/2}}=\norm{S^{-1/2}TS^{-1/2}}\le 1.\]
        すると再び$C^*$性と$*:B(H)\to B(H)$が等長写像であることより,
        \[\norm{T^{1/2}S^{-1}T^{1/2}}=\norm{S^{-1/2}T^{1/2}}^2=\norm{T^{1/2}S^{-1/2}}^2\le1.\]
        よって,$T^{1/2}S^{-1}T^{1/2}\le I$を得る.
        再び自己共役作用素の順序の性質\ref{prop-order-of-self-adjoint-operator}(1)より,$S^{-1}\le T^{-1/2}IT^{-1/2}=T^{-1}$.
    \end{enumerate}
\end{proof}
\begin{remarks}
    正規作用素の可逆性はbounded away from zeroのみであった\ref{prop-characterization-of-invertibleness-of-normal-operator}.
    正作用素については,これをさらに緩めることができる.
\end{remarks}

\subsection{直交射影と対角化}

\begin{tcolorbox}[colframe=ForestGreen, colback=ForestGreen!10!white,breakable,colbacktitle=ForestGreen!40!white,coltitle=black,fonttitle=\bfseries\sffamily,
title=対角化とは,直交射影への標準分解をいう.]
    直交射影は,正作用素の代表例であり,$0\le P\le I$に位置する.
    また作用素の中でも最も基本的なもので,実解析における特性関数のような役割を持つ.
    単関数は直交射影の有限和に当たる.
    これへの分解が対角化の理論である.
\end{tcolorbox}

\begin{definition}[orthogonal projection / orthoprojection]
        一般に,冪等律を満たす線型作用素$P:H\to H,P^2=P$を射影という.射影が自己共役であるとき,\textbf{直交射影}であるという.\footnote{前者を冪等作用素,後者を射影と呼び分けることもある.この場合直交射影とは,直交分解が定める射影のことを指す.}
\end{definition}
\begin{example}[直交射影の直交分解による特徴付け]\mbox{}
    \begin{enumerate}
        \item 閉部分空間$X\le H$について,直交分解$H=X+X^\perp$が導く作用素$P:H\to H;y\mapsto x$は$\norm{P}\le 1$かつ$P^2=P$を満たす半正定値作用素である.よってこれは\textbf{直交射影}である.
        \item 逆に,任意の自己共役な冪等作用素$P:H\to H$に対して,$X:=\Im P$とおくとこれは閉部分空間で,任意の$x^\perp\in X^\perp$に対して$\norm{Px^\perp}^2=(x^\perp|P^2x\perp)=0$より,$P$は直交射影である.$I-P$は$\Im(I-P)=X^\perp$を満たす直交射影である.
    \end{enumerate}
    以上の観察より,直交射影$P$について,$\Im P\perp\Ker P$すなわち$\Im P\perp\Im(I-P)$が成り立つ.
\end{example}
\begin{remarks}[diagonalizable]\mbox{}
    \begin{enumerate}
        \item 射影は実解析における特性関数にあたる.
        \item 単関数は$T=\sum_{i\in[n]}\lambda_iP_i$にあたる.
        \item ある正規直交基底$(e_j)_{j\in J}$と有界集合$\{\lambda_j\}_{j\in J}\subset\bF$が存在して,$\forall_{x\in H}\;Tx=\sum_{j\in J}\lambda_j(x|e_j)e_j=\sum_{j\in J}\lambda_jP_j\;(P_j:H\epi\bF e_j)$と表せるとき,$T$は\textbf{対角化可能}という.このとき,$(e_j)$を固有ベクトル,$(\lambda_j)$を固有値という.$(x|e_j)$は$x$の$j$-座標である.
        
        なお,無限和$T=\sum_{j\in J}\lambda_jP_j$は作用素の強位相については収束する(各点収束)が,そのほかは保証されない.
    \end{enumerate}
\end{remarks}

\begin{lemma}[一般の射影の性質]\label{lemma-property-of-projection}
    $P^2=P\in B(E)$を満たすとする.$E_1:=\Im P,E_2:=\Ker P$とする.
    \begin{enumerate}
        \item $P|_{E_1}=\id_{E_1}$.
        \item $Q:=I-P$は$\Im P=\Ker Q,\Ker P=\Im Q$を満たす射影である.
        \item $E=E_1\oplus E_2$.
    \end{enumerate}
\end{lemma}

\begin{lemma}[直交射影の特徴付け]
    $E$を射影とし,$E\ne 0$とする.次の6条件は同値.
    \begin{enumerate}
        \item $\Ker E=(\Im E)^\perp$.
        \item $E$はある閉部分空間$M$についての直交射影である.
        \item $\norm{E}=1$.
        \item $E$は自己共役である.
        \item $E$は正規である.
        \item $E$は正である.
    \end{enumerate}
\end{lemma}

\begin{lemma}[対角化可能作用素の性質]
    $T$が対角化可能であるとする:$T=\sum_{j\in J}\lambda_jP_j$.
    \begin{enumerate}
        \item $T^*$も対角化可能で,固有ベクトルは$(e_j)$,固有値は$(\o{\lambda_j})$である.
        \item $T$は正規:$TT^*=T^*T$.
        \item $T$が自己共役であることと,固有値が全て実数であることは同値.
        \item $T$が半正定値であることと,固有値が全て非負実数であることは同値.
    \end{enumerate}
\end{lemma}
\begin{proof}\mbox{}
    \begin{enumerate}
        \item $T^*x=\sum\o{\lambda_j}(x|e_j)e_j$より.
    \end{enumerate}
\end{proof}

\begin{proposition}[有限次元線型空間論]
    $H$が有限次元であるとき,任意の正規な作用素は対角化可能である.
    また,互いに可換な正規な作用素は,同時対角化可能である.
\end{proposition}
\begin{proof}
    有限次元線型空間論による.
\end{proof}

\subsection{Hilbの(自己)同型}

\begin{tcolorbox}[colframe=ForestGreen, colback=ForestGreen!10!white,breakable,colbacktitle=ForestGreen!40!white,coltitle=black,fonttitle=\bfseries\sffamily,
title=ユニタリ作用素]
    いままではHilbert空間を位相線形空間の延長として扱い,射は有界線型写像としてきたが,
    ここで距離構造に注目する.ユニタリ作用素は,真の意味での自己射である.
    可逆な有界線型写像がユニタリであるための必要十分条件は$V^{-1}=V^*$である.

    全ての基底変換はユニタリ作用素で記述されるのであった\ref{prop-characterization-of-isomorphism-of-Hilbert-spaces}.
    よって,この分だけ条件を緩めることに価値がある.
    ユニタリー部分群
    \[U(H)\mono\GL(H)=B(H)^\times\]
    が考えられ,この$B(H)$への作用による共役類を考える.
    これをユニタリー同値という.
\end{tcolorbox}

\begin{definition}[unitary operator / orthogonal operator]
    $H$の自己等長同型を\textbf{ユニタリ作用素}という(全写な等長写像であることと同値):$\forall_{x\in H}\;\norm{Ux}=\norm{x}$かつ$\Im U=H$.実線型空間については,直交作用素ともいう.
\end{definition}

\begin{lemma}[ユニタリ作用素の特徴付け]\label{lemma-characterization-of-unitary-operator}
    ユニタリ作用素$U\in B(H)$について,
    \begin{enumerate}
        \item $U$は内積を保つ.
        \item $U^*U=I$(実はこれは等長写像の特徴付け).
        \item 正規である:$UU^*=U^*U=I$.
        \item 一般の作用素$V\in B(H)$が可逆でかつ$V^{-1}=V^*$を満たすなら,$V$はユニタリである.
    \end{enumerate}
\end{lemma}
\begin{proof}\mbox{}
    \begin{enumerate}
        \item 極化恒等式\ref{prop-polarization-identity}より,等長写像ならば内積を保つこと\ref{prop-characterization-of-isometry}に注意.
        \item $(Ux|Uy)=(x|U^*Uy)=(x|y)$より,補題\ref{lemma-correspondence-between-sesquilinearform-and-operator}の全単射から,$U^*U=I$.
        \item 明らか.
        \item $\norm{Ux}^2=(U^*Ux|x)=\norm{x}^2$より等長.可逆なら全写.
    \end{enumerate}
\end{proof}

\begin{definition}[unitary equivalent]
    2つの作用素$S,T\in B(H)$について$\exists_{U\in U(H)}\;S=UTU^*$を満たすとき,これらは\textbf{ユニタリー同値}であるという.
\end{definition}

\begin{lemma}
    ユニタリー同値は,作用素ノルム,自己共役性,正規性,対角化可能性,ユニタリー性を保つ.
\end{lemma}

\subsection{部分等長作用素と極分解}

\begin{tcolorbox}[colframe=ForestGreen, colback=ForestGreen!10!white,breakable,colbacktitle=ForestGreen!40!white,coltitle=black,fonttitle=\bfseries\sffamily,
title=]
    自己共役作用素は実数,正作用素は正実数と見れるのであった.
    $C^*$-代数の元の標準形がほしい.
    $\forall_{\lambda\in\C}\;\lambda=\abs{\lambda}e^{i\theta}$に対応する作用素の概念は,$\abs{A}:=(A^*A)^{1/2}$と表せるが,$e^{i\theta}$に対応する作用素のクラスは,新たに用意する必要がある.
\end{tcolorbox}

\begin{definition}[partial isometry, initial subspace, final subspace]
    作用素$U\in B(H)$が\textbf{部分等長作用素}であるとは,閉部分集合$X=(\Ker U)^\perp=\oo{\Im U^*}$が存在して,制限$U|_X$は等長写像で,$U|_{X^\perp}=0$を満たすことをいう.
    $X$を\textbf{初期部分空間},$\Im U$を\textbf{最終部分空間}という.
\end{definition}
\begin{remarks}
    $\Ker U=X^\perp$は閉であるから,$X=(\Ker U)^\perp=\oo{\Im U^*}$は必然的に閉である.
\end{remarks}
\begin{lemma}[部分等長写像の特徴付け]\mbox{}
    \begin{enumerate}
        \item $P:=U^*U$とおくと,$\forall_{x\in X}\;Px=x,\;\forall_{x^\perp\in X^\perp}\;Px^\perp=0$.すなわち,$P$は直交射影である:$P^2=P\land P^*=P$.
        \item (1)は部分等長作用素を特徴付ける.すなわち,作用素$U\in B(H)$について$U^*U$が直交射影となるとき,$U,U^*$は部分等長作用素である.$U^*$を$U$の\textbf{部分逆}(partial inverse)という.
    \end{enumerate}
    結局,次の6条件は同値である.
    \begin{enumerate}
        \item $U$は部分等長作用素である.
        \item $U^*$は部分等長作用素である.
        \item $U^*U$は直交射影である.
        \item $UU^*$は直交射影である.
        \item $UU^*U=U$.
        \item $U^*UU^*=U^*$.
    \end{enumerate}
\end{lemma}
\begin{proof}\mbox{}
    \begin{enumerate}
        \item $x^\perp\in X^\perp=\Ker U$に対して,$U^*Ux=U^*0=0$は明らか.
        等長写像の特徴付け\ref{lemma-characterization-of-unitary-operator}より,$X$上で$U^*U=I$である.
        \item $X:=\Im U^*U$とおくと,射影はこの上で恒等だから,$\forall_{x\in X}\;\norm{Ux}^2=(Ux|Ux)=(U^*Ux|x)=(x|x)=\norm{x}^2$より,$X$上で等長.$X^\perp=(\Im U^*U)=\Ker U^*U$より,$\forall_{x\in X^\perp}\;(Ux|Ux)=(U^*Ux|x)=(0|x)=0$.
        
        一般に,$U$が部分等長写像であるとき,$U^*$も部分等長写像である.
        $U(I-P)=0$である.よって,$(UU^*)^2=UU^*UU^*=UPU^*=UU^*$から,$UU^*$も直交射影である.
        よって,前述の議論を繰り返せば良い.
    \end{enumerate}
\end{proof}
\begin{remarks}
    部分等長写像は,$0$と等長写像とに分解できる作用素をいう.
    等長写像はepi:$U^*U=I$として特徴付けられるから,射影の言葉で部分等長写像が特徴付けられることは自然である.
\end{remarks}

\begin{example}[unilateral shift operator]
    可分Hilbert空間$H$上のシフト作用素\ref{operator-unilateral-shift}
    \[S\paren{\sum\al_ne_n}=\sum\al_ne_{n+1}\]
    は等長作用素だが,ユニタリーではない(すなわち全射でない)例となっている.
    また$S,S^*$は部分等長作用素である.
    $S:H\to\{e_1\}^\perp$は等長作用素であり,$S^*S=I$(たしかに射影である).
    一方で,$SS^*=P_{\Brace{e_1}^\perp}$であり,$S^*$は$\{e_1\}^\perp$上では$H$に値を取る部分等長写像である.
\end{example}

\begin{theorem}[polar decomposition (von Neumann)]\mbox{}
    \begin{enumerate}
        \item 任意の作用素$T\in B(H)$に対して,ただ一つの半正定値作用素$\abs{T}:=(T^*T)^{1/2}\in B(H)$が存在して,$\forall_{x\in H}\;\norm{Tx}=\norm{\abs{T}x}$を満たす.特に,$\Ker T=\Ker\abs{T}$.
        \item ただ一つの部分等長写像$U$が存在して,$\Ker U=\Ker T,U\abs{T}=T$を満たす.
        \item $U^*U\abs{T}=U^*T=\abs{T},\;UU^*T=T$が成り立つ.
    \end{enumerate}
\end{theorem}
\begin{proof}\mbox{}
    \begin{enumerate}
        \item 一意性を示せば良い.
        ある正作用素$S\ge0$が$\forall_{x\in H}\;\norm{Tx}=\norm{Sxx}$を満たすならば,$(S^2x|x)=\norm{Sx}^2=\norm{Tx}^2=(T^*Tx|x)$である.これより,$T^*T$も正だから,$S^2=T^*T$が従う\ref{cor-nondegeneratedness-of-semi-inner-product}.
        二乗根の一意性より,$S=(T^*T)^{1/2}$である.
        \item $T,\abs{T}$は任意の$x\in H$に対して像のノルムが等しいから,核も等しい.よって,$\Ker T=\Ker\abs{T}=(\Im\abs{T})^\perp$\ref{prop-Ker-of-adjoint-operator}.
        \begin{description}
            \item[存在] まず,任意の$y=\abs{T}x\in\Im\abs{T}$に対して,$U_0y=U_0\abs{T}x:=Tx$と定めると,これは等長作用素$U_0:\Im\abs{T}\to\Im T$である:$\norm{U_0y}=\norm{Tx}=\norm{\abs{T}x}=\norm{y}$.
            連続延長\ref{prop-extension-of-operator-on-dense-subset}により,等長な延長$\o{U_0}:\oo{\Im\abs{T}}\to\oo{\Im T}$が得られる.
            さらに,$\Ker T=(\Im\abs{T})^\perp$上では零とすることで,$U\abs{T}=T$を満たす部分等長写像$U$を得る.これはたしかに$\Ker U=\Ker T$を満たす.
            \item[一意性] 
            $V$を,$\Ker V=\Ker T,V\abs{T}=T$を満たす部分等長写像とする.
            すると,まず$U=V\on\;\oo{\Im\abs{T}}$が必要で,その直交補空間ではいずれも$0$であるから,$U=V$が必要.
        \end{description}
        \item $U^*U$は構成から$\oo{\Im\abs{T}}$への直交射影だから,$U^*U\abs{T}=\abs{T}$.
        $T=U\abs{T}$より,$U^*T=\abs{T}$.
        この両辺に$U$を乗じると$UU^*T=U\abs{T}=T$.
    \end{enumerate}
\end{proof}

\begin{corollary}[随伴作用素の極分解]
    $T=U\abs{T}$を極分解とすると,$T^*=\abs{T}U^*=U^*(U\abs{T}U^*)$より,符号は$U^*$で,絶対値は$U\abs{T}U^*$である.
\end{corollary}
\begin{remark}
    和,積の極分解には殆ど法則がない.
\end{remark}

\begin{theorem}
    $T\in B(H)$について,次の2条件は同値.
    \begin{enumerate}
        \item あるユニタリ作用素$U$について,$T=U\abs{T}$と極分解される.
        \item $T,T^*$の核が同次元である:$\Ker T\simeq_\Hilb\Ker T^*$.
    \end{enumerate}
\end{theorem}

\begin{corollary}\label{cor-polar-decomposition-of-invertible-operators}
    $T\in B(H)$が可逆であるとき,極分解に含まれる部分等長写像$U$はユニタリである.
\end{corollary}
\begin{proof}
    ここでは定理に依らず,単独で証明する.

    $T=U\abs{T}$を極分解とすると,$\Ker U=\Ker T$で,$\Im U=\oo{\Im T}$.
    $T$が可逆のとき$\Ker U=0,\Im U=H$が従うから,部分等長写像$U$はユニタリである.
\end{proof}

\begin{corollary}
    $T\in B(H)$を正規とする.
    このとき,$T,T^*,\abs{T}$と可換なユニタリ作用素$W$が存在して,$T=W\abs{T}$と極分解できる.
\end{corollary}

\subsection{Russo-Dye-Gardner定理}

\begin{tcolorbox}[colframe=ForestGreen, colback=ForestGreen!10!white,breakable,colbacktitle=ForestGreen!40!white,coltitle=black,fonttitle=\bfseries\sffamily,
title=ユニタリ作用素への分解]
    極分解をしたなら,$\C\simeq\R^2$に対応する分解も欲しくなる.これは$B\cap B(H)_\sa$上のみで存在する.
    結局,$\norm{S}\le 1$を満たす有界作用素は,ユニタリ作用素の凸結合(特に平均)として表せる.
    さらに,$B(H)_\sa$は$B(H)$を生成するから,任意の作用素はユニタリ作用素の線型結合である.
\end{tcolorbox}

\begin{lemma}[単位球内の自己共役作用素の表示]
    $T=T^*$かつ$\norm{T}\le 1$ならば,作用素$U:=T+i(I-T^2)^{1/2}$はユニタリであり,$T=\frac{1}{2}(U+U^*)$と表せる.
\end{lemma}
\begin{proof}
    $I-T\ge0$かつ$I+T\ge0$で互いに可換であるから,$I-T^2\ge0$である\ref{lemma-positive-closed-cone-of-positive-operator}(2),\ref{prop-order-of-self-adjoint-operator}(3).
    よってたしかに二乗根を持つ.
    $UU^*=U^*U=I$がわかるから,$U$はユニタリである.
    正作用素は自己共役であることに注意すれば,$T=\frac{1}{2}(U+U^*)$は明らか.
\end{proof}

\begin{lemma}
    $S\in B(H)$かつ$\norm{S}<1$ならば,任意のユニタリ作用素$U$に対して,ユニタリ作用素$U_1,V_1$が存在して,$S+U=U_1+V_1$を満たす.
\end{lemma}
\begin{proof}
    $U=I$の場合について示せば,両辺に$U^*$を左から乗ずることで一般の結論を得る.

    $\norm{S}=\norm{S^*}<1$の仮定より,三角不等式$\norm{x}-\norm{y}\le\norm{x-y}$から,
    \[\norm{Sx+x}\ge\norm{x}-\norm{Sx}\ge(1-\norm{S})\norm{x}\]
    より,$I+S,I+S^*$はいずれも$0$に対して有界である.よって,作用素の可逆性の特徴付け\ref{prop-characterization-of-invertibleness-of-operator}より,
    $I+S$は可逆である.
    よって,極分解$I+S=V\abs{I+S}$において,$V$はユニタリである\ref{cor-polar-decomposition-of-invertible-operators}.
    $\norm{I+S}\le\norm{I}+\norm{S}<2$より,補題から,正作用素$\abs{I+S}$はあるユニタリ作用素$W$を用いて$\abs{I+S}=W+W^*$と表せる.
    以上より,
    \[S+I=V(W+W^*)=VW+VW^*.\]
\end{proof}

\begin{proposition}[Russo-Dye-Gardner theorem (66)]
    $T\in B(H)$かつ$\exists_{n>2}\norm{T}<1-\frac{2}{n}$ならば,ユニタリ作用素$U_1,\cdots,U_n$が存在して,
    \[T=\frac{1}{n}(U_1+U_2+\cdots+U_n).\]
\end{proposition}
\begin{proof}
    $S:=\frac{1}{n-1}(nT-I)$とおくと,$nT=(n-1)S+IT$で,
    \[\norm{S}\le\frac{1}{n-1}(n\norm{T}-1)<\frac{1}{n-1}(n-3)<1\]
    を満たすから,補題を$n-1$回繰り返し適用すると,
    \begin{align*}
        nT&=(n-1)S+I=(n-2)S+(S+I)=(n-2)S+(V_1+U_1)\\
        &=(n-3)S+(S+V_1)+U_1=(n-3)S+(V_2+U_2)+U_1\\
        &=(n-4)S+(S+V_2)+U_2+U_1\\
        &=(n-4)S+(V_3+U_3)+U_2+U_1=\cdots\\
        &=(S+V_{n-2})+U_{n-2}+\cdots+U_1=U_n+U_{n-1}+U_{n-2}+\cdots+U_1.
    \end{align*}
\end{proof}
\begin{remarks}
    $B(H)$の開球の元は,ユニタリ作用素の凸結合(特に平均)で表せる.
    球面上の元はこの限りではなく,シフト作用素が反例である.
\end{remarks}

\begin{corollary}[同じ論文で紹介されている系]
    $C^*$-代数$A$のユニタリ元の全体を$U(A)$で表す.ノルム空間$B$への線型写像$f:A\to B$について,$\sup_{U\in U(A)}\norm{f(U)}$.
\end{corollary}
\begin{remarks}
    the norm of an operator can be calculated using only the unitary elements of the algebra. \footnote{\url{https://en.wikipedia.org/wiki/Russo\%E2\%80\%93Dye_theorem}}
\end{remarks}

\subsection{数域半径}

\begin{tcolorbox}[colframe=ForestGreen, colback=ForestGreen!10!white,breakable,colbacktitle=ForestGreen!40!white,coltitle=black,fonttitle=\bfseries\sffamily,
title=]
    作用素ノルムと$(T-|-)$なる半内積(の制限)との関係の観察\ref{prop-operator-norm-of-self-adjoint-operator}から,
    $C^*$-代数の手法に数域$\Im(T-|-)$を考えることは自然である(参考:\ref{lemma-numerical-range}).
    その半径は作用素ノルムと同値なノルムを定め,正規作用素については作用素ノルムと同じノルムを定める.
    さらにスペクトル半径と数域半径が一致するとき,spectraloid作用素といい,正規作用素はこれにも当てはまる.
\end{tcolorbox}

\begin{definition}[numerical radius]
    $H$を複素Hilbert空間,$T\in B(H)$とする.$T$の\textbf{数域半径}とは,
    \[\nnorm{T}=\sup\Brace{\abs{(Tx|x)}\in\R_+\mid x\in H,\norm{x}\le 1}\]
    をいう.なお,$T\in B(H)_\sa$のとき,作用素ノルムは
    \[\norm{T}=\sup\Brace{\abs{(Tx|x)}\in\R_+\mid x\in H,\norm{x}=1}\]
    と表せること\ref{prop-operator-norm-of-self-adjoint-operator}に注意.
    一般に命題が成り立つ.
\end{definition}
\begin{remark}[numerical range]
    $W:=\Brace{(Tx|x)\in\C\mid x\in H,\norm{x}= 1}=\Brace{\frac{x^*Tx}{x^*x}\in\C\;\middle|\;x\in H,x\ne 0}$を数域という.
\end{remark}

\begin{notation}[自己共役部分]
    $\Re(T):=\frac{1}{2}(T+T^*)$.$T$が自己共役のとき,これは$T$自身に一致する.
\end{notation}

\begin{proposition}[数域半径は同値なノルムを定める]
    $\nnorm{\cdot}$は$B(H)$上のノルムで,$\forall_{T\in B(H)}\;\frac{1}{2}\norm{T}\le\nnorm{T}\le\norm{T}$を満たす.
\end{proposition}
\begin{proof}
    $\nnorm{T}$の斉次性と劣加法性は明らかで,非退化性は後半の主張から従う.また,Cauchy-Schwartzの不等式より$\nnorm{T}\le\norm{T}$もわかる.
    よって,$\norm{T}\le 2\nnorm{T}$を示せば良い.
    任意の単位ベクトル$x,y\in H$について,極化恒等式と中線定理より
    \begin{align*}
        4\abs{(Tx|y)}&=\Abs{\sum^3_{k=0}i^k(T(x+i^ky)|x+i^ky)}\\
        &\le\nnorm{T}(\norm{x+y}^2+\norm{x-y}^2+\norm{x+iy}^2+\norm{x-iy}^2)\\
        &=\nnorm{T}2(\norm{x}^2+\norm{y}^2+\norm{x}^2+\norm{iy}^2)=8\nnorm{T}
    \end{align*}
    これで,半双線型形式のノルムについて$\norm{(T-|-)}\le2\nnorm{T}$を得たが,等長同型\ref{lemma-correspondence-between-sesquilinearform-and-operator}より結論を得る.
\end{proof}
\begin{remarks}
    また,一般の線型写像について,有界であることと$\nnorm{T}<\infty$になることは同値になる.
\end{remarks}

\begin{proposition}
    $\forall_{T\in B(H)}\;\nnorm{T}=\max\Brace{\norm{\Re(\theta T)}\in\R_+\mid\theta\in\C,\abs{\theta}=1}$.
\end{proposition}

\begin{proposition}
    任意の$T\in B(H)$について,
    \begin{enumerate}
        \item $\nnorm{T^2}\le\nnorm{T}^2$.
        \item $T$が正規であるとき,等号成立.
    \end{enumerate}
\end{proposition}

\subsection{数域}

\begin{tcolorbox}[colframe=ForestGreen, colback=ForestGreen!10!white,breakable,colbacktitle=ForestGreen!40!white,coltitle=black,fonttitle=\bfseries\sffamily,
title=]
    自己共役作用素$T$に対して,$R(T,x):=\frac{(Tx|x)}{(x|x)}$をRayleigh商といい\ref{prop-operator-norm-of-self-adjoint-operator},固有値の数値計算に利用される.\footnote{\url{https://ja.wikipedia.org/wiki/レイリー商}}
    そしてこの値域を数域という.
    $T$がエルミート行列$M$のとき数域は実数に含まれ,$R(M,x)\in[\lambda_\mathrm{min},\lambda_\mathrm{max}]$となる.
\end{tcolorbox}

\begin{proposition}[Hausdorff-Toeplitz]
    $T\in B(H)$について,数域$W(T)$は凸集合である.$H$が有限次元のとき,$W(T)$はコンパクトである.
\end{proposition}

\begin{proposition}
    $T\in B(H)$を正規作用素とする.数域の閉包$\oo{W(T)}$とスペクトルの閉凸包$\oo{\Conv(\sigma(T))}$とは等しい.また,$\oo{W(T)}$の極点$x$について,次の2条件は同値.
    \begin{enumerate}
        \item $x\in W(T)$.
        \item $x$は$T$の固有値である.
    \end{enumerate}
\end{proposition}
\begin{remark}
    工学では,この定理を利用して,$T$の固有値を$W(T)$の形から推定する.\footnote{\url{https://en.wikipedia.org/wiki/Numerical_range}}
\end{remark}

\subsection{例}

\begin{tcolorbox}[colframe=ForestGreen, colback=ForestGreen!10!white,breakable,colbacktitle=ForestGreen!40!white,coltitle=black,fonttitle=\bfseries\sffamily,
title=]
    乗算作用素は対角行列の概念を一般化する.任意のHilbert空間上の自己共役作用素は,$L^2$空間上のある乗算作用素とユニタリ同値である,という主張がスペクトル定理である.
    核が超関数になることも許せば,全ての線型作用素は積分作用素として表せる,という主張がSchwartzの核定理である.
    そのほか,合成作用素と転送作用素の随伴,シフト作用素=平行移動作用素(時系列解析ではラグ作用素)などもある.

    また,作用素の随伴は,「転置」の要素は隠れて,複素共役を取ることに似る.
\end{tcolorbox}

\begin{theorem}[multiplication operator and its symbol]\label{operator-multiplication}
    $(X,\Om,\mu)$を$\sigma$-有限な測度空間とし,$H:=L^2(X,\Om,\mu)=:L^2(\mu)$とする.任意の$\varphi\in L^\infty(\mu)$に対して,これと積を取る写像
    \[\xymatrix@R-2pc{
        M_\varphi:L^2(\mu)\ar[r]&L^2(\mu)\\
        \rotatebox[origin=c]{90}{$\in$}&\rotatebox[origin=c]{90}{$\in$}\\
        f\ar@{|->}[r]&\varphi\cdot f
    }\]
    は,有界で$M_\varphi\in B(L^2(\mu))$,等長である:$\norm{M_\varphi}=\norm{\varphi}_\infty$.
    $\varphi$を\textbf{乗算作用素}$M_\varphi$の\textbf{記号}という.
\end{theorem}

\begin{theorem}[integral operator / transform and its kernel]\label{operator-integral-transformation}
    $(X,\Om,\mu)$を測度空間,$k:X\times X\to\bF$を次を満たす$\Om\times\Om$-可測関数とする:
    \begin{align*}
        \int_X\abs{k(x,y)}d\mu(y)\le c_1,&\ae[\mu],&\int_X\abs{k(x,y)}d\mu(x)\le c_2,&\ae[\mu].
    \end{align*}
    このとき,写像
    \[\xymatrix@R-2pc{
        K:L^2(\mu)\ar[r]&L^2(\mu)\\
        \rotatebox[origin=c]{90}{$\in$}&\rotatebox[origin=c]{90}{$\in$}\\
        f\ar@{|->}[r]&Kf(x):=\int k(x,y)f(y)d\mu(y)
    }\]
    は有界線型作用素で,ノルムは$\norm{K}\le(c_1c_2)^{1/2}$を満たす.
\end{theorem}

\begin{example}[Volterra operator]\label{operator-Volterra}
    $k:[0,1]\times[0,1]\to2\mono\R$を集合$\{(x,y)\in[0,1]\times[0,1]\mid y<x\}$の特性関数とする.
    これを核とする積分作用素$V:L^2(0,1)\to L^2(0,1);f\mapsto Vf(x)=\int^1_0k(x,y)f(y)dy=\int^x_0f(y)dy$をVolterra作用素という.
    これは不定積分を定める積分作用素である.\footnote{ヴォルテラ積分方程式は、人口学や、粘弾性物質の研究、保険数学に現れる再生方程式などへと応用されている。}
\end{example}

\begin{example}[matrix multiplication]
    行列乗算は,離散空間上での積分変換と捉えられる.
    これが行列という形式の普遍性を説明しているのではなかろうか?
    線形代数と微分積分の概念はここに交錯する.
    積分変換はより一般に多項式関手(polynomial functor)の特別な場合で,多項式関手とは,多項式概念の関手化である.
    Volterra作用素は,不定積分概念の作用素化であろうか.
\end{example}

\begin{example}[unilateral shift]\label{operator-unilateral-shift}
    \[\xymatrix@R-2pc{
        S:l^2\ar[r]&l^2\\
        \rotatebox[origin=c]{90}{$\in$}&\rotatebox[origin=c]{90}{$\in$}\\
        (\al_1,\al_2,\cdots)\ar@{|->}[r]&(0,\al_1,\al_2,\cdots)
    }\]
    は全射でない等長作用素となる(すなわち,ノルム1の有界線型写像).
    これをシフト作用素という.
\end{example}


\section{コンパクト作用素}

\begin{tcolorbox}[colframe=ForestGreen, colback=ForestGreen!10!white,breakable,colbacktitle=ForestGreen!40!white,coltitle=black,fonttitle=\bfseries\sffamily,
title=]
    ($X$が局所コンパクトハウスドルフ空間である時,)
    有界連続関数の中でコンパクト台を持つもの$C_c(X)\subset C_b(X)$の関係と,有界作用素の中で有限ランクを持つもの$B_f(H)\subset B(H)$の関係は非常に似ている.
    These classes describe local phenomena on $H$ and on $X$.\cite{AnalysisNow}
    そこで,$C_c(X)$の完備化として得た$C_0(X)$に対応するクラス"$B_0(H)$"を,作用素論でも構成することを考える.
    このクラスはある種の有限的な性質を持ち,$B(H)$の閉イデアルをなす.
\end{tcolorbox}

\subsection{定義}

\begin{definition}[finite rank]
    Hilbert空間$H$上の作用素$T:H\to H$について,
    \begin{enumerate}
        \item $T$が\textbf{有限ランク}であるとは,$\Im T$が$H$の有限次元部分空間であることをいう(したがって特に閉\ref{prop-finite-subspaces}).
        \item 有限ランクな有界作用素全体$B_f(H)$は,$B(H)$内の部分代数であり,かつイデアルである.
        \item $B_f(H)$はイデアルとして自己共役である:$(B_f(H))^*=B_f(H)$.
    \end{enumerate}
\end{definition}
\begin{proof}\mbox{}
    \begin{enumerate}\setcounter{enumi}{1}
        \item 任意の$S,T\in B_f(H),U\in B(H),a\in\bF$について,$aS,S+T,ST\in B_f(H)$である:$\Im aS=\Im S,\Im(S+T)\subset\Im S\oplus\Im T,\Im(SU)\subset\Im S,\dim\Im(US)\le\dim\Im(S)$.
        \item $T\in B_f(H)\Leftrightarrow T^*\in B_f(H)$を示す.直交分解$H=\Im T\oplus\Im T^\perp$を随伴によって表現することより,$H=\Im T\oplus\Ker T^*$の関係がある.よって,$\Im T^*=T^*(\Im T)$より,$T^*$も有限ランクである.
    \end{enumerate}
\end{proof}

\begin{lemma}[近似的単位元の構成]
    $B_f(H)$には射影からなるネット$(P_\lambda)_{\lambda\in\Lambda}$が存在し,$\forall_{x\in H}\;\norm{P_\lambda x-x}\to 0$を満たす.
\end{lemma}
\begin{proof}
    $H$の正規直交基底${(e_j)}_{j\in J}$を取り,$\Lambda:=\Brace{\lambda\in P(J)\mid \abs{\lambda}<\infty}$からのネット$(P_\lambda:=\pr\{\brac{e_j}_{j\in J})$を考える.
    すると,$\forall_{\lambda\in\Lambda}\;\dim(\Im(P_\lambda))<\infty$より確かに$B_f(H)$上のネットなっている.
    任意の$x=\sum_{j\in J}\al_je_j\in H$について,
    系\ref{cor-well-definedness-of-Bessel's-identity}より,任意の$0\in[0,\infty)$の開近傍の基本系の元$[0,\ep)$に対して,ある有限集合$J$が存在して,$\norm{P_\lambda x-x}^2=\sum_{j\in\Lmd}\abs{\al_j}^2<\ep$(Parseval's identity)が成り立つ.
    よって,ネットとして,$\norm{P_\lmd x-x}$は$0$に収束する.
\end{proof}

\begin{theorem}[stability of compact operator]
    $T\in B(H)$について,次の5条件は同値.
    \begin{enumerate}
        \item $T\in\dbloverline{B_f(H)}$.
        \item $T|_B:B\to H$は弱-ノルム連続な関数である.
        \item $T(B)$は$H$でコンパクトである.
        \item $\dbloverline{T(B)}$は$H$でコンパクトである.
        \item $B$上の任意のネットは,$T$での像が$H$上で収束するような部分ネットを持つ.
    \end{enumerate}
\end{theorem}
\begin{proof}\mbox{}
    \begin{description}
        \item[(1)$\Rightarrow$(2)] 
        $x$に弱収束する$B$のネット$(x_\lambda)_{\lambda\in\Lambda}$を任意に取る.任意の$\ep>0$に対して,仮定より,$S\in B_f(H)$が存在して$\norm{S-T}<\ep/3$を満たすから,
        \begin{align*}
            \norm{Tx_\lambda-Tx}&=\norm{(T-S)x_\lambda-(T-S)x+Sx_\lambda-Sx}\\
            &\le2\norm{T-S}+\norm{Sx_\lambda-Sx}\\
            &\le\frac{2}{3}\ep+\norm{Sx_\lambda-Sx}.
        \end{align*}
        いま,任意の$B(H)$の元は弱-弱連続\ref{lemma-characterization-of-bounded-operator-on-Hilbert-space}だから,$Sx_\lambda$は$Sx$に弱収束する.
        $\Im S$は有限次元であるから,この正規直交基底を$\{e_1,\cdots,e_n\}$とおくと,各線型汎関数$(-|e_j)$は弱連続だから,
        ノルムと内積の関係性
        \[\norm{Sx_\lambda-Sx}^2=\sum_{j\in[n]}\abs{(S(x_\lambda-x)|e_j)}^2\to0\]
        より,ノルムについても収束する.
        以上より,$\norm{Tx_\lambda-Tx}<\ep$であるから,$T$は弱-ノルム連続である.
        \item[(2)$\Rightarrow$(3)]
        任意のHilbert空間は回帰的で,$H$の単位球$B$は弱コンパクト\ref{def-weak-topology-on-H(B)}であったから,$T(B)$はノルムコンパクトである.
        \item[(3)$\Rightarrow$(4)]
        一般に,任意のHausdorff空間のコンパクト集合は閉であるから,$T(B)=\oo{T(B)}$.
        \item[(4)$\Rightarrow$(5)]
        $T(B)$は相対コンパクトであるから,任意の$T(B)$のネットは収束する部分ネットを持つ.したがって,その像も収束する.
        \item[(5)$\Rightarrow$(1)]
        補題の通りの射影からなるネット$(P_\lambda)_{\lambda\in\Lambda}$を取る.
        これについて,$(P_\lambda T)$は$B_f(H)$のネットになるが,これが$T$にノルム収束することを示す.
        仮に収束しないと仮定すると,$\ep>0$が存在して,任意の$\lambda\in\Lambda$に対して単位ベクトル$x_\lambda$が存在して,$\norm{(P_\lambda T-T)x_\lambda}\ge\ep$.
        仮定より,ネット$(Tx_\lambda)$はある極限$y$にノルム収束すると仮定してよく,このとき補題より,
        \begin{align*}
            \ep&\le\norm{(I-P_\lambda)Tx_\lambda}\le\norm{(I-P_\lambda)(Tx_\lambda-y)}+\norm{(I-P_\lambda)y}\\
            &\le\norm{Tx_\lambda-y}+\norm{(I-P_\lambda)y}\to0.
        \end{align*}
        よって矛盾.
    \end{description}
\end{proof}
\begin{remarks}
    $B_f(H)$の元の像は有限次元であるから,そこではノルム位相と弱位相が一致する.
    これが(1)の消息となる.
\end{remarks}

\begin{definition}[compact operator]
    定理の同値な条件を満たす作用素を\textbf{コンパクト作用素}という.
    無限遠で消えるため,コンパクト作用素の空間は$B_0(H)$と表すが,$K(H),C(H)$も一般的である.
\end{definition}

\begin{lemma}[コンパクト作用素の空間の描像]
    コンパクト作用素の空間$B_0(H)=\oo{B_f(H)}\subset B(H)$は
    \begin{enumerate}
        \item ノルム閉で自己共役なイデアルである.
        \item (1)の条件を満たすもので最小のものである.特に$H$が可分である場合は唯一の非自明な閉イデアルである.
        \item $H$が無限次元であるとき,非単位的な部分代数である:$I\notin B_0(H)$.が,その場合でも,有限ランクの射影からなる近似的単位元を持つ.
    \end{enumerate}
\end{lemma}
\begin{proof}\mbox{}
    \begin{enumerate}
        \item 自己共役性が保たれる.
        \item 
    \end{enumerate}
\end{proof}

\subsection{正規なコンパクト作用素}

\begin{tcolorbox}[colframe=ForestGreen, colback=ForestGreen!10!white,breakable,colbacktitle=ForestGreen!40!white,coltitle=black,fonttitle=\bfseries\sffamily,
title=]
    $B_0(H)$には射影からなる近似的単位元が存在することを確認したが,
    正規コンパクト作用素は,固有値の言葉で特徴付けることが出来る.
\end{tcolorbox}

\begin{lemma}
    $T\in B(H)$が対角化可能であるとき,
    \begin{enumerate}
        \item $T$はコンパクトである.
        \item 正規直交基底$(e_j)_{j\in J}$に対応する固有値$(\lambda_j)_{j\in J}$は$c_0(J)$の元である.
    \end{enumerate}
\end{lemma}

\begin{lemma}
    $x\in H$を正規作用素$T\in B(H)$の固有ベクトルとし,対応する固有値を$\lambda\in\C$とする.
    \begin{enumerate}
        \item $x$は$T^*$の固有ベクトルであり,対応する固有値は$\o{\lambda}$である.
        \item $T$の他の固有値に対応する固有ベクトルは$x$に直交する.
    \end{enumerate}
\end{lemma}

\begin{lemma}
    複素Hilbert空間$H$上の正規なコンパクト作用素$T$は,$\abs{\lambda}=\norm{T}$を満たす固有値$\lambda\in\C$を持つ.
\end{lemma}

\begin{theorem}[正規コンパクト作用素の特徴付け]
    $H$を複素Hilbert空間,$T\in B(H)$とする.次の2条件は同値.
    \begin{enumerate}
        \item $T$は正規なコンパクト作用素である.
        \item $T$は対角化可能で,その固有値は無限遠で消える.
    \end{enumerate}
\end{theorem}

\begin{notation}
    $x,y\in H$に対して,$x\odot y\in B(H)$を$(x\odot y)z=(z|y)x$で定まる$\Im x\odot y=\bF x$を満たす階数$1$の作用素とする.
    この構成は半双線型写像$H\times H\to B_f(H)$を定める.
    すると,$\norm{e}=1$について,$e\odot e$は$\C e$への射影である.
\end{notation}

\begin{remarks}[正規なコンパクト作用素についてのスペクトル写像定理の消息]
    正規なコンパクト作用素は,ある正規直交基底$(e_j)_{j\in J}$が存在して,ノルム収束する級数$T=\sum_{j\in J}\lambda_je_j\odot e_j$と表せるクラスである.
    これは,$J_0:=\Brace{j\in J\mid\lambda_j\ne0}$が有限集合であるか,係数列$(\lambda_j)_{j\in J_0}$は$0$に収束するかのいずれかであるからである.
    このとき,$T$のスペクトルは集積点$0$を併せて,$\Sp(T)=\{\lambda_j\}_{j\in J}\cup\{0\}$と表せる.
    ここで,$*$-代数の等長準同型
    \[\xymatrix@R-2pc{
        C(\Sp(T))\ar[r]&B(H)\\
        \rotatebox[origin=c]{90}{$\in$}&\rotatebox[origin=c]{90}{$\in$}\\
        f\ar@{|->}[r]&f(T):=\sum_{j\in J}f(\lambda_j)e_j\odot e_j
    }\]
    を考えると,$f(T)\in B_0(H)$であることは,$f(0)=0$であることに同値であることがわかる.
    特に$f(z)=\sum\al_{nm}z^n\o{z}^m$と表せる場合を考えると,$f(T)=\sum\al_{nm}T^nT^{*m}$となる.
\end{remarks}

\subsection{Fredholm作用素}

\begin{tcolorbox}[colframe=ForestGreen, colback=ForestGreen!10!white,breakable,colbacktitle=ForestGreen!40!white,coltitle=black,fonttitle=\bfseries\sffamily,
title=]
    Calkin代数$B(H)\epi B(H)/B_0(H)$上に同じ像を定める作用素は,同値類を定める.
    これは応用上重要である.
    この同値類についての不変量が重要な意味を持つこととなる.
\end{tcolorbox}

\begin{definition}[Calkin algebra]
    $B_0(H)$は閉イデアルであるから,商$B(H)/B_0(H)$は商ノルムについてBanach代数を定める.
    これを\textbf{Calkin代数}という.\footnote{Wikipediaによると$H$が可分である場合のみを指す,指数理論と作用素環論の対象である.}
\end{definition}

\begin{lemma}
    Calkin代数$B(H)/B_0(H)$は$C^*$-代数である.
\end{lemma}

\begin{definition}[compact perturbation]
    $S,T\in B(H)$が$S-T\in B_0(H)$を満たすとき,片方はもう片方の\textbf{コンパクトな摂動}であるという.
    すなわち,Calkin代数上に同じ像を定めることをいう.
\end{definition}

\begin{proposition}[Atkinson's theorem]
    $T\in B(H)$について,次の4条件は同値.これらを満たす作用素を\textbf{Fredholm作用素}といい,その全体を$F(H)$で表す.
    \begin{enumerate}
        \item ある作用素$S\in B(H)$が一意的に存在して$\Ker S=\Ker T^*$かつ$\Ker S^*=\Ker T$が成り立ち,$ST,TS$はそれぞれ$(\Ker T)^\perp,(\Ker T^*)^\perp$上への有限な余次元を持つ射影を定める.
        \item ある作用素$S\in B(H)$が存在して,$ST-I,TS-I$はいずれもコンパクトである.
        \item $T$は$B(H)/B_0(H)$で見れば可逆である.
        \item $\Ker T,\Ker T^*$は有限次元で,$\Im T$は閉である.
    \end{enumerate}
\end{proposition}
\begin{remarks}
    $F(H)$は積で閉じており,自己共役である($*$作用素について閉じている).
\end{remarks}

\begin{definition}
    $T\in F(H)$をFredholm作用素とする.
    $\Index T:=\dim\Ker T-\dim\Ker T^*$を\textbf{指数}という.
    このとき,射影$P,Q$を$ST=I-P,TS=I-Q$と定めると,$\Index T=\rank P-\rank Q$でもある.
    $F(H)$の部分集合を
    \[F_n(H):=\Brace{T\in F(H)\mid\Index T=n}\quad n\in\Z\]
    で定めると,$\forall_{n\in\Z}\;F_n(H)\ne\emptyset$.シフト作用素\ref{operator-unilateral-shift}$S$について,$\forall_{n\in\N}\;S^n\in F_{-n}(H),S^{*n}\in F_n(H)$が成り立つため.
\end{definition}

\begin{lemma}
    $T\in F(H)$とする.
    \begin{enumerate}
        \item $R\in B(H)$を可逆とすると,$\Index RT=\Index TR=\Index T$.
        \item $\Index T^*=-\Index T$.
        \item $T$の部分逆を$S$とすると,$\Index S=-\Index T$.
    \end{enumerate}
\end{lemma}

\subsection{Fredholmの交代定理}

\begin{lemma}
    $A\in B_f(H)$ならば,$I+A\in F_0(H)$.
\end{lemma}

\begin{lemma}\mbox{}
    \begin{enumerate}
        \item 任意の$T\in F_0(H)$について,部分等長作用素$V\in B_f(H)$が存在して,$T+V$は可逆である.
        \item $A\in B_0(H)$ならば,$T+A\in F_0(H)$.
    \end{enumerate}
\end{lemma}

\begin{corollary}[Fredholm alternative]
    $A\in B_0(H)$,$\lambda\in\C\setminus\{0\}$とする.
    このとき,$\lambda I-A\in B(H)$は可逆であるか,または,$\lambda$は$A$の有限な重複度を持った固有値である.
    後者の場合,$\o{\lambda}$は$A^*$の固有値で,同じ重複度を持つ.
\end{corollary}
\begin{remarks}
    コンパクト作用素のスペクトルは,$\{0\}$と,$T$の固有値のみからなる.
    特に,$\C$の可算部分集合で,$0$のみを集積点とする.
\end{remarks}

\subsection{Fredholm作用素の連結部分}

\begin{theorem}
    任意のFredholm作用素$T\in F(H)$とコンパクト作用素$A\in B_0(H)$について,$\Index(T+A)=\Index T$.
\end{theorem}

\begin{proposition}
    任意のFredholmクラス$F_n(H)$は$B(H)$の開集合である.
\end{proposition}

\begin{proposition}
    $T_1\in F_n(H),T_2\in F_m(H)$について,$T_1T_2\in F_{n+m}(H)$.
\end{proposition}

\begin{remarks}
    $G:=B(H)/B_0(H)$を乗法群とし,$G_0$を単位元を含む連結成分とする.すなわち,ある$A\in B(H)/B_0(H)$について$\exp A$と表せる元が生成する部分群とする.
    このとき,$G_0$は$G$内で開かつ閉で,$G/G_0$は離散群で$G$の連結成分を特徴付ける.
    今回,$G/G_0\simeq\Z$である.
    そして,Fredholm作用素$T$の指数とは,作用素$F(H)\epi G\epi G/G_0=\Z$による$T$の像である.
\end{remarks}

\section{跡}

\begin{quotation}
    関数論とヒルベルト空間上の作用素論との類比において,$C_0,B_0$と$C_c,B_f$は対応するが,
    $B(H)$は2つの役割を持つ.1つは$C_b$であるが,もう一つは$L^\infty(X)$である.そのためにはLebesgue測度にあたる概念が必要であるが,これが跡である.
    このような測度論との交差が特に美しい.いつしか積分もただの線型作用素であったし,級写像も積分作用素の特殊な例なのであった.

    跡が積分に当たることを強烈に示唆する例が,Shannonのエントロピーが事象$\rho$に対して$-E[\rho\log\rho]$であるのに対して,von Neumannエントロピーが密度行列$\rho$に対して$-\Tr(\rho\log\rho)$である.
\end{quotation}

\begin{tcolorbox}[colframe=ForestGreen, colback=ForestGreen!10!white,breakable,colbacktitle=ForestGreen!40!white,coltitle=black,fonttitle=\bfseries\sffamily,
title=]
    前節では$B_f(H)$を$C_c(X)$に,$B_0(H)$を$C_0(X)$に,$B(H)$を$C_b(X)$に見立てたが,$B(H)$は同時に$L^\infty(X)$に似た振る舞いもする.

    有限次元線型空間論では,跡だけがまるで異邦人のような特異な存在であった.
\end{tcolorbox}



\chapter{Spectral Theory}

\begin{quotation}
    解析学の理論であるが,幾何学と代数学の双対性であるIsbell双対性の一つの例・Gelfand dualityに到達する.
    任意の$C^*$-代数$\A$は,あるGelfandスペクトル$\Sp(A)$と呼ばれる位相空間上の連続関数のなす$C^*$-代数に等しい.
    群と体の間の写像を調べる営みが解析学であるとしたら,どこに存在するのか.

    Banach代数$\A$のfunction calculusとは,あるコンパクトハウスドルフ空間$X$上の連続関数のなす代数$C\subset C(X)$との間の同型$\Iso(C,A)$の部分集合をいう.
    $C$が$C(X)$の中で大きいほど,よいfunction calculusだとみなされる.

    ある同型$\Phi:C\to A$に対して,

    Spectra in algebraic geometry : 
    Grothendieck has defined a prime spectrum of commutative unital ring having in mind Gel'fand's spectrum of a commutative $C^*$-algebra
\end{quotation}

\section{Categorical Settings}

\begin{tcolorbox}[colframe=ForestGreen, colback=ForestGreen!10!white,breakable,colbacktitle=ForestGreen!40!white,coltitle=black,fonttitle=\bfseries\sffamily,
title=]
    Gelfand Duality\footnote{\url{https://ncatlab.org/nlab/show/Gelfand+duality}}
\end{tcolorbox}

\begin{notation}\mbox{}
    \begin{enumerate}
        \item $C^*\Alg$で,単位的な$C^*$-代数のなす圏を表す.
        \item $C^*\Alg_\nonunital$で,単位的でない$C^*$-代数のなす圏を表す.
        \item $\Top_\cpt$で,コンパクトハウスドルフ空間のなす$\Top_\Haus$の充満部分圏を表す.
        \item $*/\Top_\cpt$で,$\Top_\cpt$のpointed objectのなす圏とする.
        \item $\Top_{\lcpt,\infinity}$で,射を無限遠点で消える連続写像とする局所コンパクトハウスドルフ空間のなす圏とする.
        \item $\Top_{\lcpt,\proper}$で,射をproper mapとする局所コンパクトハウスドルフ空間のなす圏とする.
    \end{enumerate}
\end{notation}

\begin{notation}[functors]\mbox{}
    \begin{enumerate}
        \item 関手$C:\Top_\cpt\to C^*\Alg^\op_{\com}$を,$C(X):=\Brace{f\in\Map(X,\C)\mid f\text{\;continuous}}$に写す関手,
        ,$C_0:*/\Top_\cpt\to C^*\Alg^\op_{\com,\nonunital}$を,$x_0\in X$で消える連続関数のなす空間に写す関手とする.
        \item $\Spec:C^*\Alg^\op_\com\to\Top_\cpt$で,可換な$C^*$-代数$\A$を,スペクトル位相によって位相空間と見た指標の空間に対応させる関手とする.同様に$\Spec:C^*\Alg^\op_{\com,\nu}\to\Top_\lcpt$も定まる.
    \end{enumerate}
\end{notation}

\begin{theorem}[unital Gelfand duality theorem]
    $\Spec:C^*\Alg^\op_\com\to\Top_\cpt$は圏の同値であり,$C$が準逆である.
\end{theorem}
\begin{corollary}
    $\Spec:C^*\Alg^\op_{\com,\nu}\to */\Top_\cpt$は圏の同値であり,$C_0$が準逆である.
\end{corollary}

\begin{lemma}
    コンパクトハウスドルフ空間の開集合は,局所コンパクトである.
\end{lemma}
\begin{remarks}
    これより,一点コンパクト化によって,連続関数は「無限遠点で消える連続関数」に対応するから,圏の反変同値$\Top_{\lcpt,\infinity}\Leftrightarrow C^*\Alg_{\com,\nonunital}$が引き起こされる.
    この双対性は,基礎体が$\C$でも$\R$でも成り立つ.
\end{remarks}

\section{Banach代数}

\begin{tcolorbox}[colframe=ForestGreen, colback=ForestGreen!10!white,breakable,colbacktitle=ForestGreen!40!white,coltitle=black,fonttitle=\bfseries\sffamily,
    title=]
    Banach環は何故か解析学が展開される場として,$\C$の一般化としても扱える.
    \begin{quote}
        冪級数を介して定義されるいくつかの初等関数は、任意の単位的バナッハ環において定義されうる。そのような例として、指数関数や三角関数、さらに一般的な任意の整関数が挙げられる(特に、指数写像は抽象指数群(英語版)を定義するために用いられる)。幾何級数の公式は、一般の単位的バナッハ環においても依然として有効である。二項定理もまた、バナッハ環の二つの可換な元に対して成立する\footnote{\url{https://ja.wikipedia.org/wiki/バナッハ環}}。
    \end{quote}
\end{tcolorbox}

\subsection{定義}

\begin{tcolorbox}[colframe=ForestGreen, colback=ForestGreen!10!white,breakable,colbacktitle=ForestGreen!40!white,coltitle=black,fonttitle=\bfseries\sffamily,
title=]
    Banach代数はBanのモノイド対象であるが,定義に単位性は入らない(半群対象).
    しかし,標準的な単位化が存在するから,ここでは単位性を暗黙に仮定する.
    ということで,単位的なBanach代数とは,ノルムが$1$な単位元が存在するノルム代数を言う.
    ノルム代数とは劣乗法的なノルムを備えた代数をいう.
    代数とは体による作用を備えた環,または環の構造も持つ線型空間を言う.
\end{tcolorbox}

\begin{definition}[algebra, normed -, Banach -, unital, essential ideal]\mbox{}
    \begin{enumerate}
        \item \textbf{代数}$\A$とは,ある結合的な双線型写像$\cdot:\A\times \A\to \A$について(単位的とは限らない)環の構造も持つ線型空間$\A$をいう.
        \item 部分代数とは,乗法について閉じているような部分線型空間をいう.
        \item $\A$が\textbf{ノルム代数}であるとは,劣乗法性を満たすノルム$\norm{xy}\le\norm{x}\norm{y}$を備えた代数を言う.
        \item ノルム代数$\A$がそのノルム位相について完備であるとき,\textbf{Banach代数}という.
        \item ある元$I\in\A$が存在して$\forall_{A\in\A}\;IA=AI=A$を満たすとき,ノルム代数$\A$は単位的であるという.
        \item イデアル$\I\subset\A$が,$\A$の任意のイデアルに対して零でない共通部分を持つとき,これを\textbf{本質イデアル}という.
    \end{enumerate}
\end{definition}

\begin{lemma}[単位元の性質について]\mbox{}
    \begin{enumerate}
        \item 単位元$I$は一意的であり,$\A$が零でないならば$\norm{I}\ge 1$を満たす.
        \item 単位的なノルム代数$\A$について,$\norm{I}=1$ならば,写像$\bF\ni\al\mapsto\al I\in\A$は単射な準同型を定め,さらに等長写像である:$\norm{\al I}=\abs{\al}$.\footnote{これをBanach代数の指標の自動連続性という.Frechet代数の基礎体への準同型も自動的に連続であるかは未解決.}
    \end{enumerate}
\end{lemma}
\begin{proof}\mbox{}
    \begin{enumerate}
        \item ノルムの劣乗法性より,$\forall_{A\in\A}\;\norm{A}\le\norm{A}\norm{I}$.$\A$が零でないとき,$A\ne 0$に取れるから,$1\le\norm{I}$.
        \item 略.
    \end{enumerate}
\end{proof}
\begin{remarks}
    (2)は行列で言えばスカラー行列への埋め込みであるが,これが同型になる場合は限られることがGelfand-Mazur定理\ref{cor-Gelfand-Mazur-thm}となる.
\end{remarks}

\begin{lemma}[Banach代数のノルムの劣乗法性の十分条件]
    代数$\A$がBanach空間でもある(ノルムが定まっており,ノルム位相について完備)とする.このとき,(1)$\Rightarrow$(2)$\Rightarrow$(3)が成り立つ.
    \begin{enumerate}
        \item ノルムが劣乗法性を満たす:$\forall_{A,B\in\A}\;\norm{AB}\le\norm{A}\norm{B}$.
        \item 乗法$\A\times\A\to\A$はノルム位相について連続である.
        \item 劣乗法性を満たすノルムであって元のノルムと同値なものが存在する.
    \end{enumerate}
\end{lemma}
\begin{proof}\mbox{}
    \begin{description}
        \item[(1)$\Rightarrow$(2)] ノルム位相に関して収束する$\A\times\A$の点列$((A_n,B_n))_{n\in\N}\xrightarrow{n\to\infty}(A,B)\in\A\times\A$を取り,積もノルム収束すること:$\norm{AB-A_nB_n}\xrightarrow{n\to\infty}0$を示せば良い.
        \begin{align*}
            \norm{AB-A_nB_n}&=\norm{AB-A_nB+A_nB-A_nB_n}\\
            &\le\norm{(A-A_n)B}+\norm{A_n(B-B_n)}\\
            &\le\norm{A-A_n}\norm{B}+\norm{A_n}\norm{B-B_n}\xrightarrow{n\to\infty}0.&\because\text{劣乗法性}
        \end{align*}
        \item[(2)$\Rightarrow$(3)]
        $\A_1:=\A\times\bF$を単位化とする.
        $a\in\A$に対して,
        \[\xymatrix@R-2pc{
            L_a:\A_1\ar[r]&\A_1\\
            \rotatebox[origin=c]{90}{$\in$}&\rotatebox[origin=c]{90}{$\in$}\\
            (x,\zeta)\ar@{|->}[r]&(ax+\zeta a,0)
        }\]
        とすると,$L_a\in B(\A_1)$である:$\norm{L_a(x,\zeta)}=\norm{ax+\zeta a}\le\norm{a}\norm{(x,\zeta)}$.
        これに対して,$\nnorm{a}:=\norm{L_a}$と定めると,これは$\A$の元々のノルムと同値であり,$\A$はこのノルムについてBanach代数である(劣乗法性を満たす).
        実際,劣乗法性は作用素ノルムの劣乗法性から従い,ノルムの同値性も,
        $\A\to\A;x\mapsto ax$は連続であるから,$\exists_{C\in\R}\;\norm{ax}\le C\norm{x}$に注意すると,
        特に$(x,\zeta)=(0,1)$は$\norm{(0,1)}\le 1$を満たすから,
        \[\norm{a}=\norm{a\cdot 0+1\cdot a}\le\sup_{\norm{(x,\zeta)}\le 1}\norm{ax+\zeta a}=\nnorm{a}\]
        と,
        \begin{align*}
            \nnorm{a}&=\sup_{\norm{(x,\zeta)}\le 1}\norm{ax+\zeta a}\\
            &\le\sup_{\norm{(x,\zeta)}\le 1}(\norm{ax}+\abs{\zeta}\norm{a})\\
            &\le\sup_{\norm{(x,\zeta)}\le 1}(C\norm{x}+\abs{\zeta}\norm{a})\\
            &=\norm{a}\sup_{\norm{x}+\abs{\zeta}\le 1}\underbrace{\paren{\frac{C}{\norm{a}}\norm{x}+\abs{\zeta}}}_{\in\R}
        \end{align*}
        とより判る.
    \end{description}
\end{proof}
\begin{remarks}[作用素ノルムの立ち位置]
    後に議論する単位化によって$\norm{I}=1$として良いように,ノルムの劣乗法性は作用素ノルムが模範となっている.
\end{remarks}

\subsection{代数の射}

\begin{definition}
    代数$A,B$について,
    \begin{enumerate}
        \item $\varphi:A\to B$が代数の射であるとは,積を保つ線型写像であることをいう.
        \item 単位元を保つとき,$\varphi$を単位的という.
    \end{enumerate}
\end{definition}

\begin{lemma}
    代数の射$\varphi:A\to B$について,
    \begin{enumerate}
        \item $\Ker\varphi$は$A$のイデアルである.
        \item $\Im\varphi$は$B$の部分代数である.
    \end{enumerate}
\end{lemma}

\begin{lemma}
    ノルム代数$A,B$と,その間の2つの連続な代数の準同型$\varphi,\psi:A\to B$について,ある$A$をノルム代数として生成する集合$S$(すなわち,$S$の生成する閉部分代数が$A$である)上で一致するならば,これらは写像として等しい.
\end{lemma}

\begin{example}[disk algebra]
    円板代数$A(D)$に対して,$\lambda\in\o{D}$が定める評価写像$\ev_\lambda:A(D)\to\C$は連続な代数の準同型である.
    また,任意の連続な代数の準同型で零でないもの(これを指標というのであった\ref{def-character})はこの形を持つ.
    この条件を満たすBanach代数,すなわち指標がすべて評価者像となるBanach代数を,自然な一様環\ref{def-uniform-algebra}という.
    これは,恒等写像と包含写像$z:\o{D}\mono\C$とが生成する閉部分代数は$A(D)$自身であることによる.
\end{example}

\subsection{部分代数と生成}

\begin{lemma}\mbox{}
    \begin{enumerate}
        \item 部分代数の閉包は再び部分代数である.
        \item Banach代数の閉じた部分代数はBanach代数である.
    \end{enumerate}
\end{lemma}

\begin{lemma}
    $(B_\lambda)_{\lambda\in\Lambda}$を部分代数の族とする.
    \begin{enumerate}
        \item $\cap_{\lambda\in\Lambda}B_\lambda$も部分代数である.
        \item 任意の集合$S\subset A$に対して,これを含む最小の部分代数が定まる.
    \end{enumerate}
\end{lemma}

\begin{example}
    一点$a\in A$が生成する部分代数は$\Brace{a^n\in A\mid n\in\N}$によって与えられる.
\end{example}

\begin{example}[Stone-Weierstrassの定理]
    単位円周$\T:=\partial\Delta$上の連続関数全体の代数$C(\T)$内で,包含写像$z:\T\mono\C$とその共役$\o{z}:\T\mono\C$とが生成する閉部分代数は$C(\T)$自身になる.
    すなわち,この2つの生成する部分代数は$C(\T)$内で稠密である.
\end{example}

\subsection{Banach代数の単位化}

\begin{proposition}[Banach代数の単位化]
    $\A$を単位元のない零でないBanach代数とする.
    このとき,基礎体との直和空間$\A_I:=\A\oplus\bF$に次のように積を定めると,単位元$I:=(0,1)$を持つBanach代数となり,写像$i:\A\to\A_I:A\mapsto(A,0)$は余次元1を持つ本質閉イデアルに$\A$を等長に埋め込む.
    \begin{quote}
        $(A,\al)(B,\beta):=(AB+\al B+\beta A,\al\beta)$.
    \end{quote}
\end{proposition}
\begin{proof}
    $\A_I$が代数になること,$i:\A\to\A_I$が代数の埋め込みであり,等長写像になることは認める.
    \begin{description}
        \item[劣乗法性] \begin{align*}
            \norm{(A,\al)(B,\beta)}&=\norm{(AB+\al B+\beta A,\al\beta)}\\
            &=\norm{AB+\al B+\beta A}+\abs{\al\beta}\\
            &\le\norm{A}\norm{B}+\abs{\al}\norm{B}+\abs{\beta}\norm{A}+\abs{\al}\abs{\beta}\\
            &=(\norm{A}+\abs{\al})(\norm{B}+\abs{\beta})=\norm{(A,\al)}\norm{(B,\beta)}.
        \end{align*}
        \item[完備性] 命題\ref{prop-completion-of-algebraic-direct-product}より.
        \item[像は余次元$1$の本質イデアルである] 準同型定理より$\A_I/\A\simeq\bF$だから余次元$1$で,また$\A$は極大イデアルである.
        $\B$を零でないイデアルとする.
        $\A\cap\B=0$と仮定して矛盾を導く.
        このとき$\exists_{\al\in\bF^\times}\;(B,\al)\in\B$が成り立つが,$\A_1\B=\B$より,特に$\al=1$と考えて良い.
        また,$\forall_{A\in\A}\;(A,0)(B,1)=(AB+A,0)\in\B$であるが,このとき$AB+A\ne0$である.$AB+A=0$ならば,$\forall_{A\in\A}\;AB^2=A$が従い,$\A$に単位元がないことに矛盾.よって,$AB+A\ne0$であるが,これは$\A\cap\B=0$に矛盾.
        \footnote{$\B\subset\A$ならば明らかに$\A\cap\B\ne0$だから,$\B\setminus\A\ne\emptyset$とすると,$\exists_{\al\in\bF^\times,A\in\A}\;(A,\al)\in\B$であるが,$(A,\al^{-1})$との積を考えることより,特に$\al=1$として良い.
        すると,$(-A,0)+(A,1)=(0,1)\in \A+\B$より,$\A+\B=\A_I$だから,2つのイデアル$\A,\B$は互いに素である.
        よって中国剰余定理より,$\A_I/\A\cap\B\simeq\A_I/\A\times\A_I/\B\simeq\bF\times\A_I/\B$であり,$\B\ne0$より,$\A_I/\A\cap\B\subsetneq\A_I$.
        特に$\A\cap\B\ne0$.}
        \item[像はノルム閉である]
        明らか.また像は極大イデアルだから,極大イデアルは閉であること\ref{cor-maximal-ideal-is-closed}からもわかる.
    \end{description}
\end{proof}
\begin{remarks}[一点コンパクト化との対応]
    一点コンパクト化に次の図式を可換にするという意味で「対応」する操作である.
    \[\xymatrix{
        C^*\Alg_{\com,\nonunital}^\op\ar[r]^-{\Spec}\ar[d]_-{u}&\Top_{\lcpt,\infinity}\ar[d]^-c\\
        C^*\Alg^\op_\com\ar[r]^-{\Spec|}&\Top_\cpt
    }\]
    ただし,$u:C^*\Alg^\op_{\com,\nonunital}\to C^*\Alg^\op_{\com}$は$C^*$-代数の単位化が定める関手,$c:\Top_{\lcpt,\infinity}\to\Top_\cpt$は一点コンパクト化が定める関手とした.\footnote{\url{https://math.stackexchange.com/questions/2084557/why-is-adjoining-a-unit-the-algebraic-counterpart-to-the-one-point-compactificat}}
\end{remarks}

\begin{remark}[極大スペクトルとの関係]
    なお,指標$\A\to\C$の核も,余次元$1$の閉部分空間で,極大イデアルになる.
    特に,Gelfandスペクトルは,環や(離散)代数の極大スペクトル$\Spec_mA$の位相的な類比でもある.
    極大スペクトルの方について,$R$を体$k$上の有限生成な単位的で可換なNoether環で冪零元を持たないものとしたとき,$\Spec_mA$はZariski位相によってNoether位相空間となる.
    こちらでは,より一般の環について素スペクトルを考えていく.
\end{remark}

\begin{example}
    局所コンパクトハウスドルフ空間上の代数$C_0(X)$はBanach代数で,単位的であることは$X$がコンパクトであることに同値.
\end{example}

\subsection{$*$-代数の定義}

\begin{definition}[involution, anti-involution, star algebra, Banach star algebra]\mbox{}
    \begin{enumerate}
        \item 二乗が恒等写像となるような準同型を\textbf{対合}という.恒等射自身も対合である.反準同型でもある対合を\textbf{反対合}と呼ぶ.
        \item 対合を備えた可換環$K$について,$K$-$*$-代数$A$とは,$K$-双線型写像$\cdot:A\times A\to A$と$K$-反線型写像${}^*:A\to A$を備えた$K$-加群$A$であって,次の2条件を満たすものをいう:
        \begin{enumerate}[(a)]
            \item ${}^*$は台となる$K$-加群$A$上に対合を定める:$\forall_{x\in A}\;x^{**}=x$.
            \item ${}^*$は乗法$*$上に反準同型を定める:$\forall_{x,y\in A}\;(xy)^*=y^*x^*$.
        \end{enumerate}
        \item $K=\C$であり,$C$-$*$-代数$A$がBanach代数でもあるとき,これを\textbf{Banach $*$-代数}という.$*:A\to A$は等長同型である条件$\norm{xx^*}=\norm{x}^2$(これを$B^*$-条件という)も課すとき,$B^*$-代数ともいう.
        \item Banach $*$-代数$B$について,半対合${}^*:B\to B$が台となるBanach代数$B$上のノルムと両立するとき:$\norm{a^*a}=\norm{a}\norm{a^*}$(これを$C^*$-性という),これを特に$C^*$-代数と呼ぶ.
        \item 実は$C^*$-条件と$B^*$-条件とは同値になるため,現在は前者の用語しか使われない.
    \end{enumerate}
\end{definition}

\begin{example}\mbox{}
    \begin{enumerate}
        \item 
    \end{enumerate}
\end{example}

\subsection{例}

\begin{tcolorbox}[colframe=ForestGreen, colback=ForestGreen!10!white,breakable,colbacktitle=ForestGreen!40!white,coltitle=black,fonttitle=\bfseries\sffamily,
title=]
    ここでは$\bF=\C$とする.
\end{tcolorbox}

\subsubsection{可換な例}

\begin{example}[位相空間上の関数の代数]\mbox{}
    \begin{enumerate}
        \item 集合$S$上の有界な関数$l^\infty(S)$は,各点毎の和・積・スカラー倍と一様ノルムによって,単位的なBanach代数をなす.
        \item 位相空間$\Om$上の有界連続関数$C_b(\Om)$は,その閉部分代数をなす.よって,単位的Banach代数である.
        \item 局所コンパクトハウスドルフ空間$\Om$上の無限遠点で消える関数$C_0(\Om):=\Brace{f\in C_b(\Om)\mid\forall_{\ep>0}\;\{\abs{f(\om)}\ge\ep\}\text{ is compact}}$は,この閉部分代数で,よってBanach代数である.
        これが単位的であることは$\Om$がコンパクトであることに同値.これは$C_0(\Om)=C(\Om)$であることに同値.
        \item これらはいずれも,各点毎の複素共役を対合として,$C^*$-代数となる.
    \end{enumerate}
\end{example}

\begin{example}[可測関数の代数]\mbox{}
    \begin{enumerate}
        \item 測度空間$(\Om,\mu)$上の本質的に有界な可測関数全体の集合$L^\infty(\Om,\mu)$は,上述の代数と同じ演算と本質的上限ノルムによって,単位的Banach代数をなす.
        \item $\Om$を可測空間とすると,その上の有界な可測関数全体の空間$B_\infty(\Om)$は,一般の有界な関数の空間$l^\infty(\Om)$の中の閉部分代数をなす.よって,単位的Banach代数である.
    \end{enumerate}
\end{example}

\begin{example}[正則関数の代数(disk / disc algebra, Hardy space]\mbox{}
    \begin{enumerate}
        \item 単位開円板$D:=\Delta$上で正則な関数全体の空間$A(D)$は連続関数全体の空間$C(\o{D})$の閉部分代数で,したがって単位的Banach代数である.
        これを\textbf{円板代数}という.これが,Banach代数上の理論(Neumann級数展開など)の霊性源となる.
        実は,$A(D)=H^\infty(D)\cap C(\o{D})$が成り立ち,Hardy空間の閉部分代数でもある.
        \item 単位開円板$D$上の有界正則関数のなす空間$H^\infty(D)$は,一様ノルムについてBanach代数をなす.これを\textbf{Hardy代数}という.
    \end{enumerate}
\end{example}

\subsubsection{Gelfand変換の用語}

\begin{tcolorbox}[colframe=ForestGreen, colback=ForestGreen!10!white,breakable,colbacktitle=ForestGreen!40!white,coltitle=black,fonttitle=\bfseries\sffamily,
title=]
    測度のなす空間の弱位相はBanach空間の弱位相の定義の15年前にRadonが考えていた.
    この測度代数(measure algebra)は$C(X)$(の部分環)の双対空間の初期に考えられた例である.
\end{tcolorbox}

\begin{definition}[uniform algebra, natural, Banach function algebra]\label{def-uniform-algebra}
    $X$をコンパクトハウスドルフ空間とする.
    \begin{enumerate}
        \item $X$上の連続関数のなす一様ノルムについての$C^*$-環$C(X)=C_b(X)$の閉部分環(したがってBanach代数)であって,すべての定数関数を含み,$X$の点を分離するものを\textbf{一様環}という.
        \item $X$上の一様環$A$の極大イデアルが,ある点$x\in X$について,そこで消失する関数のイデアル$M_x$であるとき,これを\textbf{自然}であるという.
        \item 一般に,ノルムを考えず,$C(X)$の部分代数を\textbf{関数代数}といい,一様環は$C(X)$とその関数代数に一様ノルムを入れた場合に当たる.
    \end{enumerate}
\end{definition}

\begin{corollary}
    単位的な可換Banach代数$A$が$\forall_{a\in A}\;\norm{a^2}=\norm{a}^2$を満たすとき,あるコンパクトハウスドルフ空間$X$が存在して,$A$はその上のある一様環と同型になる.
\end{corollary}

\begin{example}[測度のなす代数(measure algebra)]
    局所コンパクトハウスドルフな群$G$上のRadon測度全体$M(G)$はBanach空間をなす.
    これは畳み込みを積とするとBanach代数となる.
\end{example}

\subsubsection{非可換な例}

\begin{example}[群環:可測関数の代数で積を合成ではなく畳み込みとしたもの]
    局所コンパクト群$G$上のHaar測度$\mu$について,$\mu$-可積分関数全体はBanach空間$L^1(G)$をなす.
    積を畳み込みとすると,Banach代数となる.
\end{example}

\begin{example}[Banの内部ホム]\mbox{}
    \begin{enumerate}
        \item ノルム空間$X$の有界自己作用素全体の空間$B(X)$は,和・スカラー倍を各点で定め,積を合成とすることで,作用素ノルムについてノルム代数となり,$X$がBanach代数であるとき$B(X)$もBanach代数である.コンパクト作用素はこの閉イデアルをなす.随伴を対合として$C^*$-代数をなす.
        \item 行列代数$M_n(\C)$は作用素の代数$B(\C^n)$に同型であるから,単位的Banach代数である.
        \item 上三角行列全体の空間は,$M_n(\C)$の部分代数をなす.
    \end{enumerate}
\end{example}



\subsection{イデアルと商}

\begin{tcolorbox}[colframe=ForestGreen, colback=ForestGreen!10!white,breakable,colbacktitle=ForestGreen!40!white,coltitle=black,fonttitle=\bfseries\sffamily,
title=]
    非単位的環を扱う際に重要な橋渡しをする概念として,モジュラーイデアルがある.
\end{tcolorbox}

\begin{definition}[modular / regular]
    イデアル$I$が\textbf{単模}であるとは,$\exists_{u\in A}\;\forall_{a\in A}\;a-au\in I\land a-ua\in I$を満たすことをいう.
\end{definition}
\begin{remarks}
    実は正しくは「単模イデアルを含む極大イデアルは存在する」であり,一般の単位的環$A$の任意のイデアルは単模であるから一般に極大イデアルを持つが,単位的とは限らない環については,必ずしも極大イデアルを持つとは限らない.
\end{remarks}

\begin{example}
    局所コンパクトハウスドルフ空間$\Om$上の任意の元$\om\in\Om$に対して,
    \[M_\om:=\Brace{f\in C_0(\Om)\mid f(\om)=0}\]
    はモジュラーイデアルである.
    実際,ある関数$u\in C_0(\Om)$で$u(\om)=1$を満たすものが存在するから,$\forall_{f\in C_0(\Om)}\;f-uf\in M_\om$.
    これは$M\oplus \C u=C_0(\Om)$を満たすから余次元$1$で,極大イデアルでもある.
\end{example}

\begin{lemma}[quotient]
    $\A$をBanach代数,$\I\subset \A$をイデアルとする.
    \begin{enumerate}
        \item $\I$が閉集合であるとき,商$\A/\I$は再びBanach代数である.
        \item イデアル$\I$について,次の2条件は同値である.
        \begin{enumerate}[(a)]
            \item $\I$は閉集合である.
            \item $\I$はあるノルム減少的な連続線型作用素$\Phi:\A\to \A/\I$の核である.
        \end{enumerate}
    \end{enumerate}
\end{lemma}
\begin{proof}\mbox{}
    \begin{enumerate}
        \item $\I$が閉集合であるとき,商$\A/\I$は商ノルムについて再びBanach空間である\ref{prop-quotient-Banach-space}.
        また,環$\A/\I$は積$(A+\I)(B+\I)=AB+\I$によって再び代数となる.よって,商ノルムの劣乗法性を示せば良い.
        \begin{align*}
            \norm{A+\I}\norm{B+\I}&=\inf_{S\in\I}\norm{A+S}\inf_{T\in\I}\norm{B+T}\\
            &\ge\inf_{S\in\I}\inf_{T\in\I}\norm{AB+(AT+SB+ST)}\ge\inf_{R\in\I}\norm{AB+R}=\norm{AB+\I}.
        \end{align*}
        \item 明らか.
    \end{enumerate}
\end{proof}

\begin{lemma}
    イデアル$I$がモジュラーであることと,$A/I$が単位的であることとは同値.
\end{lemma}

\subsection{正則表現}

\begin{tcolorbox}[colframe=ForestGreen, colback=ForestGreen!10!white,breakable,colbacktitle=ForestGreen!40!white,coltitle=black,fonttitle=\bfseries\sffamily,
title=]
    群のCayley表現のように,代数の埋め込み$\A\mono B(\A)$が標準的に存在する.
    これは埋め込みで,近似的単位元が存在するときは等長でもある.

    代数的存在は他の数学的対称に作用させて研究する指針がある(表現論).\footnote{\url{http://nlab-pages.s3.us-east-2.amazonaws.com/nlab/show/regular+representation}}
    特に代数は加群に作用する.このとき,代数は,加群の特別なクラスだと思える,ちょうど群は台集合をもち,代数は台加群を持つのと同様に.
    すると,代数に備わる追加構造である積は,自身を忘却して得る加群の上に標準的な作用を定め,これを\textbf{正則表現}という.
\end{tcolorbox}

\begin{definition}[(left) regular representation]
    Banach代数$\A$の\textbf{左正則表現}$\rho:\A\to B(\A)$とは,左移動$\forall_{A,B\in\A}\;\rho(A)(B)=AB$をいう.
\end{definition}
\begin{lemma}
    $\rho:\A\mono B(\A)$はノルム減少的な代数の準同型である.
    また,$\A$が単位的ならば,$\rho$は位相の埋め込みでもある.
\end{lemma}
\begin{proof}
    $\rho$が単射な代数準同型を定めることは明らか.
    \begin{description}
        \item[ノルム減少性] $\norm{\rho(A)}\le\norm{A}$は,作用素ノルムの定義
        \[\norm{\rho(A)}=\sup\Brace{\norm{\rho(A)(B)}=\norm{AB}\in\R_{\ge0}\mid B\in\A,\norm{B}\le 1}\]
        と劣乗法性より従う.
        \item[劣乗法性の保存] 作用素ノルムの劣乗法性より明らか:$\norm{\rho(AB)}=\norm{\rho(A)\rho(B)}\le\norm{\rho(A)}\norm{\rho(B)}$.
        が,$\A$が非単位的であるとき,位相の埋め込みとは限らないから,Banach代数の埋め込みであるとは言えない.
        \item[位相の埋め込み] 
        $\norm{A}=\norm{\rho(A)(I)}\le\norm{\rho(A)}\norm{I}\le\norm{A}\norm{I}$は$\rho:\A\to B(\A)$とその逆$\Im\rho\to\A$のLipschitz連続性を表していると読めるから,
        $\rho$は部分空間への位相同型である.
    \end{description}
\end{proof}
\begin{remark}[adjointing identity from the regular representation]
    Banach代数の埋め込み$\rho:\A\mono B(\A)$を用いて,$\A$に$B(\A)$の作用素ノルムと同値なノルムを入れることで,$\A$が零でないとき$\norm{I}=1$と仮定して良い.
    以降,これを暗黙の仮定とする.
\end{remark}

\subsection{近似的単位元}

\begin{definition}[approximate unit/ identity]
    Banach代数$\A$の\textbf{近似的単位元}とは,単位球$B\subset\A$上のネット$(E_\lambda)_{\lambda\in\Lambda}$で,$\forall_{A\in\A}\;\lim E_\lambda A=\lim AE_{\lambda}=A$を満たすものを言う.
\end{definition}
\begin{lemma}[近似的単位元の特徴付け]
    $\A$をBanach代数とする.次の2条件は同値である.
    \begin{enumerate}
        \item 近似的単位元$(E_\lambda)$が存在する.
        \item ある有界集合$\E\subset\A$が存在して,$\forall_{\ep>0}\;\forall_{A\in\A}\;\exists_{E\in\E}\;\norm{AE-A}+\norm{EA-A}<\ep$.
    \end{enumerate}
\end{lemma}
\begin{lemma}
    Banach代数に近似的単位元が存在するとき,左正則表現$\rho$は等長同型である.
\end{lemma}
\begin{proof}
    評価$\norm{AE_\lambda}\le\norm{\rho(A)}\norm{E_\lambda}\le\norm{A}\norm{E_\lambda}\le\norm{A}$の極限を考えることより従う.
\end{proof}
\begin{example}
    局所コンパクトハウスドルフ空間上の無限遠点で消失する連続関数全体の空間$C_0(X)$,
    Hilbert空間上のコンパクト作用素の空間$B_0(H)$,局所コンパクトな群$G$上の可積分関数全体のなす空間$L^1(G)$は近似的単位元を持つ.
    $P_r:\partial\Delta\to[0,\infty);z\mapsto\sum^\infty_{n=-\infty}r^{\abs{n}}z^n\;\;(r\in(0,1))$をPoisson核とすると,ネット$(P_r)_{r\in(0,1)}$は$L^1(\partial\Delta)$を畳み込みについてBanach代数とみたときの近似的単位元である.
\end{example}

\subsection{可逆性}

\begin{definition}[invertible]
    単位的なBanach代数$\A$において,元$A\in\A$が\textbf{可逆}であるとは,ある$B,C\in\A$が存在して$BA=AC=I$を満たすことを言う.
    このとき$B=C$が従うが,これを$A^{-1}$と表す.可逆元全体の集合を$\GL(\A)=\A^\times$で表す.
\end{definition}

\begin{lemma}[Neumann series]\label{lemma-Neumann-series}
    単位的Banach代数$\A$の元$A$が$\norm{A}<1$を満たすとき,
    \begin{enumerate}
        \item $I-A\in\GL(\A)$で,
        \item $(I-A)^{-1}=\sum^\infty_{n=0}A^n$と表せる.
    \end{enumerate}
\end{lemma}
\begin{proof}
    ノルムの劣乗法性より$\norm{A^n}\le\norm{A}^n$より,$\sum_{n=0}^\infty\norm{A^n}<\infty$.よって,Banach代数$\A$の完備性より,$\sum_{n=0}^{\infty}A^n$も,ある元$B\in\A$に収束する.
    $B_n:=\sum_{i=0}^nA^i$とおくと,$AB_n=B_{n+1}-I$より,$AB=\lim_{n\to\infty}AB_n=\limn B_{n+1}-I=B-I$.同様に$BA=B-I$.
    よって,$I-A$は可逆で,$(I-A)^{-1}=B$.
\end{proof}

\begin{proposition}
    単位的Banach代数$\A$において,乗法群$\GL(\A)$は開集合で,写像$A\mapsto A^{-1}$は$\GL(\A)$の位相同型を定める(よって$\GL(\A)$は位相群をなす).
\end{proposition}
\begin{proof}\mbox{}
    \begin{enumerate}
        \item 任意に$A\in\GL(\A)$を取る.
        \[\forall_{B\in\A}\quad B=A-(A-B)=A(I-A^{-1}(A-B))\]
        より,補題から,$\norm{A^{-1}(A-B)}<1$ならば,$B\in\GL(\A)$である.
        特に,$\ep<\norm{A^{-1}}^{-1}(>0)$を満たす$\ep>0$を取れば,$B(A,\ep)\subset\GL(\A)$を満たす.
        \item 任意の$A,B\in\GL(\A)$について,
        \[B^{-1}=(A(I-A^{-1}(A-B)))^{-1}=\paren{\sum^\infty_{n=0}(A^{-1}(A-B))^n}A^{-1}\]
        が成り立つ.よって,$B\to A$のとき,$B^{-1}\to A^{-1}$.
        よって,${}^{-1}:\GL(\A)\to\GL(\A)$は連続写像.これが対合であることより,同相写像でもある.
    \end{enumerate}
\end{proof}

\begin{corollary}\label{cor-maximal-ideal-is-closed}
    $\A$を単位的なBanach代数とする.
    \begin{enumerate}
        \item 真のイデアルの閉包は再び真のイデアルである.
        \item 極大イデアルは閉である.
    \end{enumerate}
\end{corollary}
\begin{proof}\mbox{}
    \begin{enumerate}
        \item 真のイデアル$\I\subset\A$を取ると,$\I\cap\GL(\A)=\emptyset$である.
        すなわち,$\I\subset\A\setminus\GL(\A)$であるが,$\GL(\A)$は閉であるから$\o{\I}\subset\A\setminus\GL(\A)$.
        よって,$\o{\I}$も真のイデアルである.
        \item 極大イデアル$\M$については極大性より$\M=\o{\M}$が成り立つため.
    \end{enumerate}
\end{proof}

\subsection{スペクトル}

\begin{tcolorbox}[colframe=ForestGreen, colback=ForestGreen!10!white,breakable,colbacktitle=ForestGreen!40!white,coltitle=black,fonttitle=\bfseries\sffamily,
title=]
    いままで代数っぽかったが,ここでいきなり幾何的な空間$\C$に落とす.
    このときの$\C\to X$への帰還.

    任意のBanach代数の元のスペクトルが空でないことを示すにあたって,
    以降,体を$\bF=\C$とする.
    すると,レゾルベントを通じて複素解析学の道具が流入する(Fredholm 1903).
    というよりむしろ,複素解析学が,$\C$の領域上のBanach空間値関数に一般化される.
\end{tcolorbox}

\begin{definition}[spectrum, spectral radius, resolvent set, resolvent]
    $\A$を単位的Banach代数とする.
    \begin{enumerate}
        \item 元$A\in\A$の\textbf{スペクトル}とは,複素数の部分集合$\Sp(A):=\Brace{\lambda\in\C\mid\lambda I-A\notin\GL(\A)}$をいう.
        \item 元$A\in\A$の\textbf{スペクトル半径}とは,実数$r(A):=\sup\Brace{\abs{\lambda}\in\R_{\ge 0}\mid\lambda\in\Sp(A)}=\min\Brace{r\in\R_{\ge0}\mid\Sp(A)\subset B(0,r)}$を指す.
        \item 補集合$\rho(A):=\C\setminus\Sp(A)$を\textbf{解核集合}という.
        \item 解核集合上の関数$R(A,\lambda):=(\lambda I-A)^{-1}:\C\setminus\Sp(A)\to\A$を\textbf{解核}という.
    \end{enumerate}
\end{definition}

\begin{lemma}[spectral mapping theorem]
    元$A\in\A$と$\norm{A}\le r$を満たす半径$r$を持つ閉円板$B(0,r)$を含む領域上で定義された正則関数$f(z)=\sum_{n=0}^\infty\al_nz^n$について,
    $f(A)=\sum_{n=0}^\infty\al_nA^n$によって定まる写像$\A\to\A$は,
    $\lambda\in B(0,r)$を満たす$\lambda\in\Sp(A)$に関して,$f(\lambda)\in\Sp(f(A))$を満たす:$f(\Sp(A)\cap \Delta(0,r))\subset\Sp(f(A))$.
\end{lemma}
\begin{proof}\mbox{}
    \begin{enumerate}
        \item $\sum_{n=0}^\infty\abs{\al_n}\norm{A^n}<\infty$と$\A$の完備性より,写像$f$はwell-definedである.
        \item 対偶を示す.任意の$\lambda\in\C$について,
        $P_{n-1}\in\C[A]$を$P_{n-1}(\lambda,A):=\sum_{k=0}^{n-1}\lambda^kA^{n-k-1}$と定めると,
        \[\norm{P_{n-1}(\lambda,A)}\le\sum^{n-1}_{k=0}\abs{\lambda}^k\norm{A}^{n-k-1}\le nr^{n-1}\]
        より,列$(P_{n-1}(\lambda,A))$は,ある$A$と可換な元$B\in\A$に収束する.
        また,
        \[f(\lambda)I-f(A)=\sum^\infty_{n=1}\al_n(\lambda^nI-A^n)=(\lambda I-A)\sum^\infty_{n=1}\al_nP_{n-1}(\lambda,A)=(\lambda I-A)B.\]
        よって,$f(\lambda)I-f(A)\in\GL(\A)$で逆元$C$を持つならば,$BC$は$\lambda I-A$の逆元となる.
    \end{enumerate}
\end{proof}

\begin{lemma}
    任意の$A\in\A$に関して,$r(A)\le\inf_{n\in\N}\norm{A^n}^{1/n}$.
    特に,$\Sp(A)$は有界である.
\end{lemma}
\begin{proof}\mbox{}
    \begin{enumerate}[(a)]
        \item 補題\ref{lemma-Neumann-series}より,
        \[\forall_{\lambda\in\C}\quad\abs{\lambda}>\norm{A}\Rightarrow(\lambda I-A)^{-1}=\lambda^{-1}(I-\lambda^{-1}A)^{-1}=\sum^\infty_{n=0}\lambda^{-n-1}A^n\]
        だから,$\abs{\lambda}>\norm{A}\Rightarrow\lambda I-A\in\GL(\A)$,すなわち,$\Sp(A)\subset\Delta(0,r)$,$r(A)<r$.
        よって,$r(A)\le\norm{A}$.
        \item 任意の$\lambda\in\Sp(A)$を取る.$x\mapsto x^n$は整関数だから,補題より,$\lambda^n\in\Sp(A^n)$.(a)での議論より,$\abs{\lambda}^n\le\norm{A^n}\Leftrightarrow\abs{\lambda}\le\norm{A^n}^{1/n}$.
        実際,$\abs{\lambda^n}>\norm{A^n}$ならば,$\lambda^nI-A^n\notin\GL(\A)$に矛盾.
    \end{enumerate}
\end{proof}

\begin{theorem}\label{thm-Spectrum-is-compact}
    単位的Banach代数の任意の元$A\in\A$に関して,
    \begin{enumerate}
        \item スペクトル$\Sp(A)\subset\C$は非空なコンパクト集合である.
        \item $A$のスペクトル半径は$r(A)=\lim_{n\to\infty}\norm{A^n}^{1/n}$と表せる.
    \end{enumerate}
\end{theorem}
\begin{proof}\mbox{}
    \begin{enumerate}
        \item 
        \begin{description}
            \item[コンパクト性] 任意の$A\in\A$を取り,$R:\C\setminus\Sp(A)\to\C$をそのレゾルベントとする.
            任意の$\lambda\notin\Sp(A)$について,$\abs{\zeta}<\norm{R(\lambda)}^{-1}$を満たすように取れば,補題\ref{lemma-Neumann-series}より,
            $\lambda-\zeta\notin\Sp(A)$で,その逆元は
            \begin{align*}
                R(\lambda-\zeta)&=(\lambda I-A-\zeta I)^{-1}\\
                &=((\lambda I-A)(I-R(\lambda)\zeta))^{-1}=\sum^\infty_{n=0}R(\lambda)^{n+1}\zeta^n
            \end{align*}
            と表せる.特に,$\rho(A)=\C\setminus\Sp(A)$は開集合であるから,$\Sp(A)$は閉集合である.補題と併せて,$\Sp(A)$はコンパクトである.
            \item[非空]
            \begin{enumerate}[(a)]
                \item 任意に有界連続汎関数$\varphi\in\A^*$を取り,$f(\lambda):=\varphi(R(\lambda))$と定めると,
                各点$\lambda\in\rho(A)$において,$r>0$が存在して,
                \[f(\lambda-\zeta)=\sum^\infty_{n=0}\varphi(R(\lambda)^{n+1})\zeta^n\quad(\zeta\in\Delta(\lambda,r))\]
                と冪級数表示できるから,$f:\C\setminus\Sp(A)\to\C$は正則関数である.

                $\abs{\lambda}>\norm{A}$とする.
                補題の(a)での議論の通り,$f(\lambda)=\sum_{n=0}^\infty\lambda^{-n-1}\varphi(A^n)$だから,
                \begin{align*}
                    \abs{f(\lambda)}&\le\sum^\infty_{n=0}\abs{\lambda}^{-n-1}\norm{A}^n\norm{\varphi}\\
                    &=\abs{\lambda}^{-1}\norm{\varphi}(1-\abs{\lambda}^{-1}\norm{A})^{-1}=\norm{\varphi}(\abs{\lambda}-\norm{A})^{-1}.
                \end{align*}
                これより,$\abs{f(\lambda)}\xrightarrow{\abs{\lambda}\to\infty}0$.
                \item $\Sp(A)=\emptyset$と仮定して矛盾を導く.このとき$f$は$C_0(\C)$に属する整関数であるが,Liouvilleの定理より,これは定数関数であることが必要だから,$f=0$である.
                したがって,$\forall_{\varphi\in\A^*}\;\varphi((\lambda I-A)^{-1})=0$.
                よって系\ref{cor-Hahn-Banach}より,$(\lambda I-A)^{-1}=0$が必要であるが,これは矛盾.
            \end{enumerate}
        \end{description}
        \item \begin{enumerate}[(a)]
            \item (1)で定義した正則関数$f:\C\setminus\Sp(A)\to\C$は,$\abs{\lambda}>\norm{A}$の範囲で局所的な冪級数展開$f(\lambda)=\sum^\infty_{n=0}\lambda^{-n-1}\varphi(A^n)$を持つ.
            正則関数$f(\lambda^{-1})$は少なくとも$\Delta(0,r(A)^{-1})$上で定義されており,対応する冪級数展開はCauchyの積分表示より,この領域内で広義一様収束する.
            したがって,任意の$r>r(A)$について,$f$の冪級数表示も$\Brace{\lambda\in\C\mid\abs{\lambda}>r}$上で一様収束する(Laurent展開の議論と並行).
            \item よって,$\lambda:=re^{i\theta}$とおくと,正則関数$\lambda^{n+1}f(\lambda)$は$\partial\Delta(0,r)$上で次のように項別積分出来る:
            \begin{align*}
                \int^{2\pi}_0r^{n+1}e^{i(n+1)\theta}f(re^{i\theta})d\theta&=\sum^\infty_{m=0}\int^{2\pi}_0r^{n-m}e^{i(n-m)\theta}\varphi(A^m)d\theta\\
                &=2\pi\varphi(A^n).&m=n\text{の時を除いて積分は}0
            \end{align*}
            また最左辺の積分は
            $M(r):=\sup_{\theta\in[0,2\pi]}\norm{R(re^{i\theta})}$とおくことで
            \begin{align*}
                \int^{2\pi}_0r^{n+1}e^{i(n+1)\theta}f(re^{i\theta})d\theta&\le r^{n+1}\sup_{\theta\in[0,2\pi]}\abs{\varphi(R(re^{i\theta}))}2\pi\\
                &\le r^{n+1}\norm{\varphi}M(r)2\pi
            \end{align*}
            と評価できるから,$\varphi(A^n)\le r^{n+1}M(r)\norm{\varphi}$を得る.
            \item (b)の議論は$\varphi\in\A^*$を任意としたから,特に$\norm{\varphi}=1,\norm{\varphi(A^n)}=\norm{A^n}$をみたすものについて(系\ref{cor-Hahn-Banach}より存在する),
            $\norm{A^n}\le r^{n+1}M(r)\Leftrightarrow\norm{A^n}^{1/n}\le r(rM(r))^{1/n}$.
            よって,$\limsup_{n\to\infty}\norm{A^n}^{1/n}\le r$.$r>r(A)$は任意に取ったから,
            $\limsup_{n\to\infty}\norm{A^n}^{1/n}\le r(A)$.補題より$r(A)\le\liminf_{n\to\infty}\norm{A^n}^{1/n}$と併せると,
            $\norm{A^n}^{1/n}$は収束して,$\lim_{n\to\infty}\norm{A^n}^{1/n}=r(A)$.
        \end{enumerate}
    \end{enumerate}
\end{proof}
\begin{remarks}
    乗法群$\GL(\A)$が開集合となるような単位的位相代数$\A$を$Q$-代数といい,一般に$Q$-代数の任意の元のスペクトルは非空でコンパクトになる.
\end{remarks}

\begin{corollary}[Gelfand-Mazur theorem (41)]\label{cor-Gelfand-Mazur-thm}
    $\A$が斜体であるとき,すなわち,$\GL(\A)=\A\setminus\{0\}$を満たすとき,$\A=\C$である.
\end{corollary}
\begin{proof}
    定理より,任意の$A\in\A$について,$\lambda\in\Sp(A)\ne\emptyset$.
    すなわち,$\lambda I-A\notin\GL(\A)$であるが,$\A$が可除環であるとき,これは$A=\lambda I$を意味する.
\end{proof}

\section{Gelfand変換}

\begin{tcolorbox}[colframe=ForestGreen, colback=ForestGreen!10!white,breakable,colbacktitle=ForestGreen!40!white,coltitle=black,fonttitle=\bfseries\sffamily,
title=]
    ここでは,Banach代数は複素係数で単位的で可換であるとする.
\end{tcolorbox}

\subsection{Gelfand変換}

\begin{tcolorbox}[colframe=ForestGreen, colback=ForestGreen!10!white,breakable,colbacktitle=ForestGreen!40!white,coltitle=black,fonttitle=\bfseries\sffamily,
title=]
    どうして$C(X)$と書いたときは体への写像であるかといえば,これは指標という概念で群にまで遡る.
\end{tcolorbox}

\begin{definition}[character, spectral topology]\label{def-character}
    $\A$を単位的$C^*$-代数とする.
    \begin{enumerate}
        \item $\A$の\textbf{指標}とは,全射(即ち零でない)連続線型準同型$\A\epi\C$のことをいう.\footnote{一般の群と体について定義される.}
        Banach代数については,任意の準同型は連続であることに注意.
        \item $\A$指標全体の空間$\hat{\A}:=\Brace{\gamma\in\Hom_{\Ban\Alg}(\A,\C)\mid\dim\Im\gamma=1}$を\textbf{指標空間}または\textbf{スペクトル}といい,この空間に定義される位相を\textbf{スペクトル位相}といい,この位相はコンパクトハウスドルフである.
        $\A$が単位的でないとき,これは局所コンパクトとなる.
    \end{enumerate}
\end{definition}
\begin{remark}
    冪集合$P(X)$を$2$上の線型空間だと思えば,集合論の特性関数の用語と一致する.
\end{remark}

\begin{notation}
    Banach代数$\A$の極大イデアル全体のなす空間を$M(\A)$で表す.
\end{notation}

\begin{proposition}
    $\A$を可換で単位的なBanach代数とする.
    \begin{enumerate}
        \item 次の対応は集合の同型を定める:
        \[\xymatrix@R-2pc{
            \hat{\A}\ar[r]&M(\A)\\
            \rotatebox[origin=c]{90}{$\in$}&\rotatebox[origin=c]{90}{$\in$}\\
            \gamma\ar@{|->}[r]&\Ker\gamma
        }\]
        \item 任意の$A\in\A$について,$\Sp(A)=\Brace{\brac{A,\gamma}\in\C\mid\gamma\in\hat{\A}}$.
    \end{enumerate}
\end{proposition}

\begin{theorem}
    $\A$を可換な単位的Banach代数とする.
    \begin{enumerate}
        \item 指標のなす集合$\hat{\A}$はコンパクトハウスドルフ位相を備える.
        \item 写像
        \[\xymatrix@R-2pc{
            \Gamma:\A\ar[r]&C(\A)\\
            \rotatebox[origin=c]{90}{$\in$}&\rotatebox[origin=c]{90}{$\in$}\\
            A\ar@{|->}[r]&\hat{A}
        }\]
        を$\hat{A}(\gamma):=\brac{A,\gamma}$で定めると,これはノルム減少的な代数の準同型である.
        \item 像$\Im\Gamma<C(\hat{\A})$は$\hat{\A}$の点を分離する.
        \item 任意の$A\in\A$について,$\hat{A}(\hat{\A})=\Sp(A)$かつ$\norm{\hat{A}}_\infty,r(A)$.
    \end{enumerate}
\end{theorem}

\begin{proposition}
    Gelfand変換の核は,$\A$の根基である:
    $\Ker\Gamma=R(\A)=\bigcap_{\I\in M(\A)}\I=\Brace{A\in\A\mid r(A)=0}$.
\end{proposition}
\begin{remark}
    ほとんどの古典的なBanach代数は半単純である($R(\A)=\{0\}$).
    さらには,$\Im\Gamma<C(\hat{\A})$を決定する問題は極めて難しい場合が多い.
    使える知識といえば,(3)の$\hat{\A}$の点を分離するということくらいである.
\end{remark}

\subsection{例}

\section{関数代数}

\begin{tcolorbox}[colframe=ForestGreen, colback=ForestGreen!10!white,breakable,colbacktitle=ForestGreen!40!white,coltitle=black,fonttitle=\bfseries\sffamily,
title=]
    $C^*$-代数は,Banach代数のうち特に振る舞いのよいもので,Hilbert空間上の作用素の理論を流入させることが出来る.
\end{tcolorbox}

\subsection{Stone-Weierstrassの定理}

\begin{tcolorbox}[colframe=ForestGreen, colback=ForestGreen!10!white,breakable,colbacktitle=ForestGreen!40!white,coltitle=black,fonttitle=\bfseries\sffamily,
title=]
    $\Gamma(\A)$は$\A$の点を分離するくらいには大きい.
    Stone-Weierstrassの定理は,さらに$C(\hat{\A})$の中で稠密であるための必要条件を調べるための道具となる.
\end{tcolorbox}

\begin{definition}[self-adjoint]
    複素関数の集合$\A\subset\Map(X,\C)$が\textbf{自己共役}であるとは,$\forall_{f\in\A}\;\o{f}\in\A$を満たすことをいう.
\end{definition}
\begin{remarks}
    $f=\frac{1}{2}(f+\o{f})+i\frac{f-\o{f}}{2i}$より,$\A_\sa:=\Brace{f\in\A\mid\Im f\subset\R}$を$\A$の実数値関数がなす部分集合とすると,$\A=\A_\sa+i\A_\sa$と表せる.
\end{remarks}

\begin{lemma}
    $\A\subset C(X,\R)$を,コンパクトハウスドルフ空間$X$上の実数値連続関数のなす線型空間とする.
    $\forall_{f,g\in\A}\;f\lor g\in\A\land f\land g\in\A$を満たすとき,
    任意の$X$上の連続関数で$X$の任意の2点において$\A$によって近似できるものは,$\A$によって一様に近似できる.
\end{lemma}

\begin{lemma}
    $\A\subset C_b(X,\R)$を,位相空間$X$上の実数値有界連続関数のなす一様に閉じた代数とする.
    $\A$は$C(X)$の束演算$f\lor g,f\land g$について閉じている.
\end{lemma}

\begin{theorem}[Stone (1948)]
    $X$をコンパクトハウスドルフ空間,$\A$を自己共役な$C(X)$の部分代数で定数を含み,$X$の点を分離するとする.
    このとき,$\A$は$C(X)$の中で一様に稠密である.
\end{theorem}
\begin{remark}
    Weierstrassの1895年の仕事「有界閉区間上の実数値連続関数は,多項式によって一様に近似できる」は系として導出される.
\end{remark}
\begin{example}
    $[0,2\pi]$上の任意の連続な周期関数は,三角多項式によって一様に近似できる(Fourier級数は必ずしも一様収束しないにも拘らず).
\end{example}
\begin{example}[Rungeの近似定理]
    単位閉円板$[\Delta]$上の開円板$\Delta$上で正則な関数全体のなす集合$H(\Delta)$は,$C(\Delta)$の真の閉部分代数であって,点を分離し,定数を含む.
\end{example}

\begin{corollary}
    $X$を局所コンパクトハウスドルフ空間,$\A$を$C_0(X)$の自己共役な部分代数で,$X$の点を分離し,$X$のどの点でも同時には消えないとする:$\forall_{x\in X}\;\exists_{f\in\A}\;f(x)\ne0$.
    このとき,$\A$は$C_0(X)$上で一様に稠密である.
\end{corollary}

\subsection{$C^*$-代数}

\begin{tcolorbox}[colframe=ForestGreen, colback=ForestGreen!10!white,breakable,colbacktitle=ForestGreen!40!white,coltitle=black,fonttitle=\bfseries\sffamily,
title=]
    対合と両立するBanach代数が$C^*$-代数である.
    対合と両立するHeyting代数がBoole代数であるのと同じように.
\end{tcolorbox}

\begin{definition}[involution, $C^*$-algebra, symmetric]\mbox{}
    \begin{enumerate}
        \item 代数$\A$の\textbf{対合}とは,周期2の写像$*:\A\to\A$であって,共役線型で乗法について反変的なものをいう.
        対合を備えた代数を$*$-代数という.
        \item ノルムについて$\forall_{A\in\A}\;\norm{A^*A}=\norm{A}^2$を満たす対合を備えたBanach代数を,\textbf{$C^*$-代数}という.
        \item 対合$*$が$\forall_{A\in\A}\;A=A^*\Rightarrow\Sp(A)\subset\R$を満たすとき,これを\textbf{対称的}であるという.
        \item 対合$*$が$\forall_{A\in\A}\;\Sp(A^*A)\subset\R_+$を満たすとき,これを\textbf{正}であるという.
    \end{enumerate}
\end{definition}
\begin{remarks}
    $\norm{A^*A}=\norm{A}^2$は,劣乗法性より$\norm{A}^2\le\norm{A^*}\norm{A}$から$\norm{A}\le\norm{A^*}$.対合であることより$\norm{A}=\norm{A^*}$を結局は含意する.
\end{remarks}
\begin{example}[involution]
    $C(X)$上の複素関数の共役や,$B(\H)$上の共役作用素.
\end{example}
\begin{example}\mbox{}
    \begin{enumerate}
        \item 任意の局所コンパクトハウスドルフ空間について,$C_0(X)$は複素共役について$C^*$-代数となる.
        \item $B(H)$は随伴について$C^*$-代数となる.
        \item (Gelfand and Naimark 43) $B(H)$の任意の閉な自己共役な部分代数は$C^*$-代数で,任意の$C^*$-代数は,なんらかのこの部分代数に$*$-等長同型になる.
    \end{enumerate}
\end{example}

\begin{lemma}
    任意の単位的でない$C^*$-代数$\A$に対して,単位的$C^*$-代数$\wt{\A}=\A+\C I$が存在して,$\A\mono\wt{\A}$は余次元$1$の極大イデアルとなる.
\end{lemma}

\begin{lemma}
    $A\in\A$が$C^*$-代数の正規な元であるならば,$r(A)=\norm{A}$.
\end{lemma}

\begin{lemma}\mbox{}
    \begin{enumerate}
        \item $A\in\A$が$C^*$-代数の自己共役な元であるならば,$\Sp(A)\subset\R$(すなわち,対合は対称的である).
        \item $\A$が単位的で$U$がユニタリであるならば,$\Sp(U)\subset\T$.
    \end{enumerate}
\end{lemma}

\subsection{Gelfandの定理}

\begin{theorem}
    任意の可換な単位的$C^*$-代数$\A$は,$\hat{\A}$を指標のなすコンパクトハウスドルフ空間として,$C(\hat{\A})$と等長$*$-同型である.
\end{theorem}

\begin{corollary}
    
\end{corollary}

\subsection{Gelfandスペクトルの対応}

\subsection{Stone-Cechコンパクト化}

\subsection{Tychonoff空間}

\section{スペクトル理論I}

\section{スペクトル理論II}

\section{作用素代数}

\section{極大可換代数}

\chapter{Unbounded Operators}

\section{始域・延長・グラフ}

\subsection{定義域}

\begin{definition}[operator in a Hilbert space, domain, densely defined, extension]
    Hilbert空間$H$\textbf{内}の作用素$T$とは,$H$の部分空間$D(T)$上の作用素$T:D(T)\to H$をいう.
    $D(T)$を\textbf{定義域}という.
    \begin{enumerate}
        \item $\oo{D(T)}=H$が成り立つとき,$T$は\textbf{稠密に定義されている}という.
        \item $D(S)\subset D(T),S=T\on D(S)$が成り立つとき,$T$は$S$の\textbf{延長}であるといい,$T\subset S$で表す.たしかに$T=S\Leftrightarrow T\subset S\land T\supset S$が成り立つ.
        \item 値域を$R(T):=T(D(T))$で表す.
    \end{enumerate}
\end{definition}

\begin{definition}[定義域の演算]
    作用素$S,T$に対して,$S+T,ST,T^{-1}$の定義域を次のように定める.
    \begin{enumerate}
        \item $D(S+T)=D(S)\cap D(T)$.
        \item $D(ST)=D(T)\cap T^{-1}(D(S))$.
        \item $D(T^{-1})=R(T),R(T^{-1})=D(T)$.
    \end{enumerate}
    とすると,和と積は結合的であるが,分配的ではない.
    また,$T$が単射でない限り,逆$T^{-1}$は集合論的な逆写像と相違する.?
\end{definition}

\begin{lemma}
    $S,T$が単射であるとき,$(ST)^{-1}=T^{-1}S^{-1}$が成り立つ.
\end{lemma}

\subsection{随伴}

\begin{definition}
    $H$内の稠密に定義された作用素$T:D(T)\to H$に対して,
    \[D(T^*):=\Brace{x\in H\mid y\mapsto(Ty|x)\in B(D(T),H)}\]
    とする.このとき,Rieszの表現定理より,$\forall_{y\in D(T)}\;\forall_{x\in D(T^*)}\;\exists!_{T^*x\in H}\;(y|T^*x)=(Ty|x)$.
    よって,$T^*:D(T^*)\to H$が定まる.これを$T$の\textbf{随伴}という.
\end{definition}

\begin{lemma}
    $S,T,S+T,ST$は稠密に定義されているとする.
    \begin{enumerate}
        \item $S\subset T\Rightarrow T^*\subset S^*$である.
        \item $S^*+T^*\subset(S+T)^*$.
        \item $T^*S^*\subset(ST)^*$.
    \end{enumerate}
\end{lemma}

\begin{lemma}
    $T$が稠密に定義されているとき,$\Ker T^*=(R(T))^\perp$.
    特に,$\Brace{x\in D(T^*)\mid T^*x=\o{\lambda}x}=R(T-\lambda I)^\perp$.
\end{lemma}

\subsection{対称作用素}

\begin{definition}[symmetric]
    稠密に定義された作用素$S$が\textbf{対称}であるとは,$\forall_{x,y\in D(S)}\;(Sx|y)=(x|Sy)$を満たすことをいう.
\end{definition}

\begin{lemma}[対称作用素の随伴による特徴付け]
    稠密に定義された作用素$S$について,次の2条件は同値.
    $\bF=\C$のとき,(3)とも同値.
    \begin{enumerate}
        \item $S$は対称である.
        \item $S\subset S^*$.
        \item $\forall_{x\in D(S)}\;(Sx|x)\in\R$.
    \end{enumerate}
\end{lemma}

\begin{lemma}
    対称作用素$S$と$\lambda=\al+i\beta\in\C$について,
    \begin{enumerate}
        \item $\forall_{x\in D(S)}\;\norm{(S-\lambda I)x}^2=\norm{(S-\al I)x}^2+\beta^2\norm{x}^2$.
        \item $\beta\ne0$ならば,$S-\lambda I$は単射で,$(S-\lambda I)^{-1}$は$\abs{\beta}^{-1}$によって抑えられる.
    \end{enumerate}
\end{lemma}

\begin{definition}[maximal symmetric, self-adjoint]
    対称作用素$S$であって,任意の対称作用素$T$に対して$S\subset T\Rightarrow S=T$を満たすものを,\textbf{極大対称作用素}という.
    $S=S^*$のとき,$S$を\textbf{自己共役}であるという.
\end{definition}

\begin{lemma}
    任意の自己共役作用素は極大対称的である.
\end{lemma}

\subsection{閉作用素}

\begin{definition}[closed operator]
    $H$内の作用素$T$のグラフ$G(T)=\Brace{(x,Tx)\in H\oplus H\mid x\in D(T)}$が$H\oplus H$内で閉集合であるとき,これを\textbf{閉作用素}という.
\end{definition}

\begin{lemma}
    $T$を閉作用素とする.
    $D(T)$内の$x$に収束する列$(x_n)$について,$(Tx_n)$が$y$に収束するならば,$x\in D(T)$かつ$Tx=y$が成り立つ.
\end{lemma}
\begin{example}
    全域で定義されている閉作用素は有界である.
    $D(T)=D(T^*)=H$を満たす作用素は有界である.
    特に,全域で定義されている対称作用素は,自己共役で有界である.
\end{example}

\begin{definition}[closable / preclosed, closure, core]\mbox{}
    \begin{enumerate}
        \item $H$内の作用素$T$が\textbf{可閉}または\textbf{前閉}であるとは,グラフ$G(T)$のノルム閉包が,ある作用素$\o{T}$のグラフになっていることをいう.
        このとき,$\o{T}$は$T$を含む最小の閉作用素であり,$T$の\textbf{閉包}という.
        \item 閉作用素$T$と$D(T)$の部分空間$D_0$について,$T_0:=T|_{D_0}$の閉包が$T$であるとき,$D_0$を$T$の\textbf{核}という.
    \end{enumerate}
\end{definition}

\begin{lemma}
    $H$内の作用素$T$について,次の2条件は同値.
    \begin{enumerate}
        \item $T$は可閉である.
        \item $D(T)$内の$0$に収束する列$(x_n)$について,$(Tx_n)$のただ一つの集積点は$0$である.
    \end{enumerate}
\end{lemma}

\section{Cayley変換}

\section{無制限スペクトル理論}

\chapter{Integration Theory}

\begin{quotation}
    積分とは,コンパクト台を持つ連続関数の空間上の,正な線型汎関数である.
    これをRudinではpositive linear functionalと呼び続ける.
    この見方をすれば,積分を完全に関数解析の言葉で定義でき,測度とは積分の特別な場合である.
    また積分と測度の双対性も明らかになる.

    理論の完成を知らせる定理は\textbf{表現定理}である.
    積分とは表現であり,2つのBanach空間をつなぐものである.

    \begin{tcolorbox}[colframe=blue, colback=blue!3!,breakable,colbacktitle=ForestGreen!40!white,coltitle=black,fonttitle=\bfseries\sffamily,
    title=]
        \begin{theorem*}[Riesz, Markov]
            $X$を局所コンパクトハウスドルフ空間とする.
            \begin{enumerate}
                \item 積分は正錐と拡張正錐の間に
                位相同型$(C_c(X,\C)^*)_+\simeq_\Top\o{\RM(X)}_+$を定める\ref{thm-Riesz-representation}.
                \item 積分は等長同型$C_0(X,\C)^*\simeq_\Ban\RM(X)$を定める\ref{prop-Riesz-Markov-3}.\footnote{\url{https://ncatlab.org/nlab/show/Riesz+representation+theorem}}
            \end{enumerate}
        \end{theorem*}
    \end{tcolorbox}
    いままでのが局所凸位相線型空間論だとしたら,ここからは錐の構造をよく見るために,
    Riesz空間=束線型空間を対象にした順序線型空間論のような理論展開をする.
\end{quotation}

\begin{definition*}[positive cone, extended positive cone]
    $C^*$-環の内部で,正作用素のなす閉凸錐の描像を思い出す.
    \begin{enumerate}
        \item 点付き順序集合$(V,0)$の正錐とは,$V_+:=\Brace{x\in V\mid x\ge0}$をいう.
        \item $W^*$-代数$V$のpredualを$V_*$とする:$V=\Hom(V_*,\R)$.$V$の拡張正錐$\o{V}_+$とは,正錐$(V_*)_+$から$\o{\R}_+=[0,\infty]$への下半連続な線型写像全体の空間をいう.
    \end{enumerate}
    $\R$の拡張正錐は$\oR_+$で,その一般化と考えられる.
\end{definition*}

\begin{definition*}[positive linear functional, finie Radon measure]
    複素数値関数/測度の空間に言及する言葉を整理する.
    \begin{enumerate}
        \item 正な連続線型汎関数の全体$C_c(X,\C)^*_+$は正錐をなす.
        \item 有限なRadon測度の全体$\RM(X,\C)$は全変動についてBanach空間をなす.
        その正錐は実数値の意味での有限Radon測度$\mu:\M\to\R_+$である.一般のRadon測度にも言及したいから,
        その拡張正錐$\o{\RM(X)}_+$を,正なRadon測度$\M\to\oR_+$全体の集合とする.
    \end{enumerate}
\end{definition*}

\section{Radon積分}

\begin{tcolorbox}[colframe=ForestGreen, colback=ForestGreen!10!white,breakable,colbacktitle=ForestGreen!40!white,coltitle=black,fonttitle=\bfseries\sffamily,
title=]
    積分自体を作用素として定義し,その定義域を十分広げることで,特殊な場合として測度を定義することを目指す.
\end{tcolorbox}

\subsection{関数束とRadon測度}

\begin{tcolorbox}[colframe=ForestGreen, colback=ForestGreen!10!white,breakable,colbacktitle=ForestGreen!40!white,coltitle=black,fonttitle=\bfseries\sffamily,
title=]
    まず,議論の対象となる「性質の良い関数と測度のクラス」を定義する.
    積分を定義する関数クラスは,束の構造を持つもの$C_c(X),C_c(X)^m,C_c(X)_m$に注目する.
    $X$上の測度はRadon測度(緊密かつ局所有限なもの)のみを考える.
    Radon測度は,$C_c(X)$上に正な線型汎関数を定める.
    これをRadon積分と呼び,逆にRadon測度を復元すること\ref{prop-recovery-of-Radon-measure}を考える.
    実は,$C_c(X)$が可積分関数$\L^1(\R^n)$の構造を支配する(稠密である\ref{prop-dense-subset-of-Lp}).
\end{tcolorbox}

\begin{notation}\mbox{}
    \begin{enumerate}
        \item $X$を局所コンパクトハウスドルフ空間とし,その上のコンパクト台を持つ実数値連続関数のなすノルム代数$C_c(X)$を考える.
        \item $X$のコンパクト部分集合の全体を$\cC\subset P(X)$で表す.
        \item $X$の可測集合全体を$\M$で表し,体積確定集合の全体を$\M^1$で表す\ref{def-measurable-sets}.
        \item $X$上のBorel関数のクラスを$\B(X)$,可測関数の全体を$\L(X)$で表す.$\B\subset\M$より$\B(X)\subset\L(X)$である\ref{def-measurable-function}.$X$上の局所可積分関数の空間を$\L^1_\loc(X)$で表す\ref{def-locally-integrable-function}.
        \item $X$上の有限なRadon電荷$\Phi$全体に全変動ノルム$\norm{\Phi}:=\abs{\Phi}(1)$を考えたBanach空間を$M(X)$で表す.
        \item $X$上の零関数を$\cN(X)$,零集合を$\cN$で表す\ref{def-null-set}.
        \item $C_c(X)$の単調ネットによって近似可能な関数全体の空間を
        \[C_c(X)^m:=\Brace{f:X\to\R\cup\{\infty\}\mid C_c(X)\text{上の単調増加ネット}(f_\lambda)_{\lambda\in\Lambda}\text{が存在して}\forall_{x\in X}\;f(x)=\sup f_\lambda(x)\text{を満たす}}\]
        とする.
        また,$C_c(X)_m:=\Brace{f:X\to\R\cup\{-\infty\}\mid\exists_{\{f_\lambda\}\subset C_c(X)}\;f_\lambda\searrow f}$とすると,$C_c(X)_m=-C_c(X)^m$であり,$C_c(X)^m\cap C_c(X)_m=C_c(X)$である.
        \item 関数クラス$M(X)\subset\Map(X,\R)$に対して$M(X)_+$とは,非負関数のなす部分空間$M(X)\cap\Map(X,\R_{\ge 0})$を表す.
        \item 特性関数を$[A]=\chi_A$で表す.
    \end{enumerate}
\end{notation}

\begin{lemma}[Riesz空間 / 束線型空間,関数束]\mbox{}
    \begin{enumerate}
        \item $C_c(X)$は束線型空間である:$C_c(X)$は束の演算$\lor,\land$について閉じており,$C_c(X)$は任意の部分集合の上限を取る操作について閉じている.
        \item $C_c(X)^m$は束である上に任意族の上限・下限に閉じており,additive cone(和と$\R_{\ge 0}$倍について閉じている)でもあり,任意の2つの正な元の積について閉じている:$C_c(X)^m_+\cdot C_c(X)^m_+\subset C_c(X)^m_+$.
        \item $C_c(X)^m$の元は全て下半連続で,非負な下半連続関数の全体を含む:$C^{1/2}(X)_+\subset C_c(X)^m$.
        \item $X$がコンパクトであることと,$-1\in C_c(X)^m$は同値.
        \item $X$が$\sigma$-コンパクトであることと,$1\in C_c(X)^m$は同値.
        \item 任意の$f\in C_c(X)^m$について,$\o{\Brace{f<0}}$はコンパクト.任意の$f\in C_c(X)_m$について,$\o{\Brace{f>0}}$はコンパクト.
        \item $X$が局所コンパクトな距離空間であるとき,任意のコンパクト集合$K\subset X$について,$\chi_K\in C_c(X)_m$.
    \end{enumerate}
\end{lemma}
\begin{proof}\mbox{}
    \begin{enumerate}
        \item a
        \item a
        \item \ref{prop-lower-semicontinuous-functions}より.
    \end{enumerate}
\end{proof}

\begin{definition}[Borel meaesure, inner regular, Radon measure, outer regular, essential measure]\mbox{}
    \begin{enumerate}
        \item Borel測度とは,Borel $\sigma$-代数$\B$を含む$\sigma$-代数上に定まる測度をいう.
        \item Borel測度が$\mu(A)=\sup\Brace{\mu(K)\in[0,\infty]\;\middle|\; K\overset{\mathrm{cpt}}{\subset}A}$を満たすとき,\textbf{内部正則}または\textbf{緊密}であるという.
        \item 内部正則なBorel測度で,\textbf{局所有限}性$\forall_{K\overset{\mathrm{cpt}}{\subset}X}\;\mu(K)<\infty$を満たすものを,\textbf{Radon測度}という\ref{def-Radon-measure}.
        \item Borel測度が\textbf{外部正則}であるとは,$\forall_{A\in\B}\;\mu(A)=\inf\Brace{\mu(B)\in[0,\infty]\mid A\subset B\osub X}$を満たすことをいう.
        \item Borel測度$m$が$M$に付随する\textbf{本質的測度}であるとは,
        \[m(A)=\sup\Brace{m^*(C)\in\R\mid C\subset A,m^*(C)<\infty},\quad\text{ただし}m^*(C)=\inf\Brace{m(B)\in[0,\infty]\mid C\subset B,B\in\B}\]
    \end{enumerate}
    結局,内部正則かつ局所有限なBorel測度をRadon測度といい,いま考えている局所コンパクトハウスドルフ空間$X$上で考えられる素性の良い測度のクラスである.
\end{definition}
\begin{example}
    $\R^n$上のLebesgue測度は,Radon測度である.
\end{example}

今後の目標の一つは次の定理を示すことである.

\begin{theorem}[Borel測度の特徴付け]
    Borel測度$\mu:\B\to\bF$について,
    次の2条件は同値.
    \begin{enumerate}
        \item $\mu$はRadon測度である.
        \item $\mu$は,$X$の任意の開集合上において,内部正則かつ局所有限である上に,外部正則でもある.
        \item $\mu$は,次を満たす2つのBorel測度$m,M$が$\M^1$の元上に取る同一の値を取る:
        $m$は$M$に付随する本質的測度であり,$M$は局所有限かつ$\B$上外部正則かつ内部正則で,$B$が開集合または体積確定$B\in\M^1$ならば$m(B)=M(B)$である.
    \end{enumerate}
    (3)はRadon積分の方を重視する見方であり,今後の議論で同値性を見る\ref{prop-recovery-of-Radon-measure}.
    さらに,台空間$X$が局所コンパクトハウスドルフ空間であるとき,次も同値.
    \begin{enumerate}\setcounter{enumi}{3}
        \item $\mu$は$C_c(X,\R)$上の正な線型汎関数である.
    \end{enumerate}
\end{theorem}

\begin{tbox}{red}{}
    唯一すべての用語が定義済みであるのは(4)のみである.そこで,
    以降の議論は(4)の立場から開始する.
    そうして,$M:=\mu^*,m:=\mu^{\o{*}}$として通常の意味でのRadon測度を復元していく.
\end{tbox}

\subsection{Daniell積分}

\begin{tcolorbox}[colframe=ForestGreen, colback=ForestGreen!10!white,breakable,colbacktitle=ForestGreen!40!white,coltitle=black,fonttitle=\bfseries\sffamily,
title=]
    Radon積分の延長を調べる前に,Daniell積分の消息を述べる.
    局所コンパクト空間上のRadon測度のPercy Daniell-style\footnote{Daniell, P. J. (1918), “A General Form of Integral”, Annals of Mathematics, Second Series (Annals of Mathematics) 19 (4): 279–294}の定義\ref{thm-extension-of-Radon-integral}はBourbakiのIntegration. Chapter IXでも探求されている.\footnote{\url{http://nlab-pages.s3.us-east-2.amazonaws.com/nlab/show/Radon+measure}}
    積分の公理化の試みのうち,これをDaniell積分という\footnote{\url{https://ja.wikipedia.org/wiki/ダニエル積分}}.
\end{tcolorbox}

\begin{definition}[monotone / positive functional]
    $L\subset\Map(X,\R)$を関数束とする.汎関数$T:L\to\R$が次の3条件を満たすとき,\textbf{単調}である,または\textbf{正}であるという.
    \begin{enumerate}
        \item 正錐の射:$T(f+g)=T(f)+T(g),\forall_{c\in\R_+}\;T(cf)=cT(f)$.
        \item 単調性:$\forall_{f,g\in L}\;f\ge g\Rightarrow T(f)\ge T(g)$.
        \item 単調収束性:任意の広義単調増加列$\{f_i\}\subset L$について,$\lim_{i\to\infty}f_i=:f\in L\Rightarrow T(f)=\lim_{i\to\infty}T(f_i)$.
    \end{enumerate}
    これらは積分の公理として非常に自然である.
\end{definition}

\begin{theorem}
    $L\subset\Map(X,\R)$を関数束,$T:L\to\R$を正な汎関数とする.
    このとき,$X$上の外測度$\mu$が存在して,
    \begin{enumerate}
        \item 任意の$f\in L$について$f$は$\mu$-可測で,$T(f)=\int_Xfd\mu$.
        \item 集合$A\subset X$は$\exists_{f\in L}\;\chi_A\le f$を満たすならば,$\mu(A)=\int_X\chi_Ad\mu=T(\chi_A)$の値は$T$によって一意に定まる.
    \end{enumerate}
\end{theorem}

\begin{definition}[Daniell integral]
    $L\subset\Map(X,\R)$を関数束とする.汎関数$T:L\to\R$が次の3条件を満たすとき,\textbf{Daniell積分}であるという.
    \begin{enumerate}
        \item 正錐の射:$T(f+g)=T(f)+T(g),\forall_{c\in\R_+}\;T(cf)=cT(f)$.
        \item $\forall_{f\in L_+}\;\sup_{0\le k\le f}T(k)<\infty$.
        \item $\{f_i\}\subset L$が広義単調増加ならば,$T(\lim_{i\to\infty}f)=\lim_{i\to\infty}T(f_i)$.
    \end{enumerate}
\end{definition}
\begin{remarks}
    Jordan分解の精神に従い,正な汎関数の差として表せる汎関数のクラスを考える.
    これにより,正汎関数の$\R_+$-線形性の非対称性が解消され,するとこれは積分になる.
\end{remarks}

\begin{theorem}
    $L\subset\Map(X,\R)$を関数束,汎関数$T:L\to\R$をDaniell積分とする.
    このとき,
    \begin{enumerate}
        \item $L_+$上に正な汎関数$T^+,T_-$が存在して,$\forall_{f\in L_+}\;T(f)=T^+(f)-T^-(f)$を満たす.
        \item $X$上に外測度$\mu^+,\mu^-$が存在して,$f\in L$が$\mu^+$-可測かつ$\mu^-$-可測ならば,$T(f)=\int fd\mu^+-\int fd\mu^-$.
    \end{enumerate}
\end{theorem}

\subsection{Rieszの表現定理}

\begin{tcolorbox}[colframe=ForestGreen, colback=ForestGreen!10!white,breakable,colbacktitle=ForestGreen!40!white,coltitle=black,fonttitle=\bfseries\sffamily,
title=]
    さきに,測度の定める積分と,$C_c(X)$上の正な線型汎関数の空間との間の同型対応に関する結果をまとめておく.
    次の章からは,積分の公理化ではなく,Radon測度から議論を始める.
\end{tcolorbox}

\begin{definition}[Baire sets]
    集合族$\{f>t\}_{f\in C_c(X),t\in\R}$が生成する$\sigma$-加法族$\A$を\textbf{Baire集合族}という.
    $\{f>t\}$は必ず開集合であるから,$\A\subset\B$である.なお,部分集合であるだけでなく,$\delta$-環をなす.
    また$\A$はコンパクトな$G_\delta$集合によっても生成される\ref{remark-Baire-function}.
    以降,$\A$がすべて$\mu$-可測であるとき,外測度$\mu$をBaire外測度と呼ぶ.
\end{definition}

\begin{lemma}
    $X$を局所コンパクトハウスドルフ空間とする.
    \begin{enumerate}
        \item (Urysohn) $K\subset U\osub X$をコンパクト集合とする.$\exists_{f\in C_c(X)}\;\chi_K\le f\le\chi_U$.
        \item $K$をコンパクト集合とし,これは開集合$V_1,\cdots,V_n$によって被覆されるとする:$K\subset V_1\cup\cdots\cup V_n$.このとき,各$V_i$内に台を持つ関数$0\le g_i\le1$が存在して,$\forall_{x\in K}\;g_1(x)+\cdots+g_n(x)=1$.
    \end{enumerate}
\end{lemma}


\begin{theorem}[Riesz]\label{thm-Riesz}
    局所コンパクトハウスドルフ空間上の関数束$C_c(X)$と,その上の汎関数$T:C_c(X)\to\R$であって$\forall_{f\in C_c(X)_+}\sup\Brace{T(g)\in\R\mid0\le g\le f}<\infty$を満たすものについて,
    2つのBaire外測度$\mu^+,\mu^-$が存在して,
    \[\forall_{f\in C_c(X)}\quad T(f)=\int_Xfd\mu^+-\int_Xfd\mu^-.\]
\end{theorem}
\begin{remark}
    符号付きBaire外測度$\mu:=\mu^+-\mu^-$と定めれば,
    $T(f)=\int_Xfd\mu$と表示出来るが,これは$\infty-\infty$の場合をのがしているのでステートメントには採用できない.
\end{remark}

\begin{corollary}
    特に$X$がコンパクトならば,ステートメントは次のようになる.
    有界線型汎関数$T:C(X)\to\R$に対して,符号付きBaire外測度$\mu$が一意に存在して,
    \[T(f)=\int_Xfd\mu.\]
    また,この対応は等長線型同型を引き起こす.
\end{corollary}

\begin{theorem}
    Rieszの表現定理\ref{thm-Riesz}の下で,ある符号付きBorel外測度$\o{\mu}$が存在して,
    \[\forall_{f\in C_c(X)}\quad\int fd\o{\mu}=T(f)=\int fd\mu\]
    が成り立つ.
    またとくに$\o{\mu}$はRadon外測度でもあり,Radon測度であることも示せる.
\end{theorem}

\subsection{単調極限上へのRadon積分の延長}

\begin{tcolorbox}[colframe=ForestGreen, colback=ForestGreen!10!white,breakable,colbacktitle=ForestGreen!40!white,coltitle=black,fonttitle=\bfseries\sffamily,
title=]
    積分の公理化から,Radon積分に話を戻そう.
    まず,ネットの極限としての閉包上に対する
    $C_c(X)$上の正な線型汎関数$\int$の延長を議論する.
\end{tcolorbox}

\begin{definition}[Radon integral, monotone limits]\mbox{}
    \begin{enumerate}
        \item \textbf{Radon測度}または\textbf{Radon積分}とは,線型汎函数$\int:C_c(X)\to\R$であって,\textbf{正}であるものをいう:$f\ge 0\Rightarrow\int f\ge 0$.
        \item 上積分$\int^*:C_c(X)^m\to\R\cup\{\infty\}$を,$\int^*f:=\sup\Brace{\int g\in\R\cup\{\infty\}\;\middle|\;g\in C_c(X),g\le f}$と定めると,これは線型汎函数である.
        \item 下積分$\int_*:C_c(X)_m\to\R\cup\{-\infty\}$を,$\int_*f:=\inf\Brace{\int g\in\R\cup\{-\infty\}\;\middle|\;g\in C_c(X),g\ge f}$と定めると,これは線型汎函数である.
    \end{enumerate}
\end{definition}
\begin{remarks}
    局所コンパクトハウスドルフ空間$X$を第2可算とすれば,$(f_n)$は単調列とすれば十分.
    この$C_c(X)$の単調極限なるクラスが大事である理由は,$X=\R$であるとき,区間の定義関数が入るためである:$\chi_{(a,b)}\in C_c(X)^m,\chi_{[a,b]}\in C_c(X)_m$.
    $C_c(X)^m,C_c(X)_m$は錐であったから,上積分と下積分は加法と$\R_{\ge 0}$倍に限っては明らかに「線型」であるが,実際に線型であることは非自明である.
\end{remarks}

\begin{lemma}\mbox{}
    \begin{enumerate}
        \item $\forall_{f\in C_c(X)_m}\;\int_*f=-\int^*(-f)$.
        \item $\forall_{f\in C_c(X)}\;\int_*f=\int^*f=\int f$.
    \end{enumerate}
\end{lemma}

\begin{lemma}\mbox{}
    \begin{enumerate}
        \item コンパクト台を持つ上半連続な正関数の単調減少ネット$(f_\lambda)$について,$\forall_{x\in X}\;f_\lambda(x)\searrow 0$ならば,$\norm{f_\lambda}_\infty\searrow0$.
        \item $C_c(X)_m$の単調減少ネット$(f_\lambda)$について,$f_\lambda\searrow0$ならば$\int_*f_\lambda\searrow0$である.
        \item $C_c(X)$の単調増加ネット$(f_\lambda)$について,$\exists_{f\in C_c(X)^m}\;f_\lambda\nearrow f$ならば$\int f_\lambda\nearrow\int^*f$である.
        \item $f,g\in C_c(X)^m$について,$\forall_{t>0}\;\int^*(tf+g)=t\int^*f+\int^*g$.
        \item $C_c(X)^m$の単調増加ネット$(f_\lambda)$について,ある関数$f$について$f_\lambda\nearrow f$ならば$\int^*f_\lambda\nearrow\int^*f$である.
        \item $f\in C_c(X)^m,g\in C_c(X)_m$に対して,$g\le f$ならば,$\int_*g\le\int^*f$である.
    \end{enumerate}
\end{lemma}
\begin{proof}\mbox{}
    \begin{enumerate}
        \item $f_\lambda$は上半連続だから,$\forall_{\ep>0}\;\{f_\lambda\ge\ep\}$は閉.仮定より台$\o{\Brace{f_\lambda>0}}$はコンパクトだから,$\{f_\lambda\ge\ep\}$はコンパクトでもある.$(f_\lambda)$は$0$に各点収束するから,$\cap_{\lambda\in\Lambda}\{f_\lambda\ge\ep\}=\emptyset$.
        コンパクト性の特徴付け\ref{thm-characterization-of-compactness}より,ある有限個を選び出せばやはり共通部分は空である.
    \end{enumerate}
\end{proof}

\subsection{可積分関数上への延長}

\begin{tcolorbox}[colframe=ForestGreen, colback=ForestGreen!10!white,breakable,colbacktitle=ForestGreen!40!white,coltitle=black,fonttitle=\bfseries\sffamily,
title=]
    セミパラ理論の外積分の定義は,一般の局所コンパクトハウスドルフ空間上のRadon積分について展開できる.
\end{tcolorbox}

\begin{definition}
    任意の実数値関数$f\in\Map(X,\R)$について,
    \begin{enumerate}
        \item 上積分を$\int^{**}f:=\inf\Brace{\int^*g\in\R\cup\{\infty\}\mid g\in C_c(X)^m,g\ge f}$と定める.
        \item 下積分を$\int_{**}f:=\sup\Brace{\int_*g\in\R\cup\{-\infty\}\mid g\in C_c(X)_m,g\le f}$と定める.
    \end{enumerate}
    以降,$\int^*,\int_*$と略記する.
    すると,可積分関数$f\in\L^1(X)$は,$\int^{*}f=\int_{*}f\in\R$によって特徴付けられる.
    次の特徴付けはよく用いる:
    \begin{quote}
        $\forall_{\ep>0}\;\exists_{g\in C_c(X)^m,h\in C_c(X)_m}\;h\le f\le g\land\int^*g-\int_*h<\ep$.
    \end{quote}
    特に,$f\in C_c(X)^m$について,可積分条件は$\int^*f<\infty$となる.
\end{definition}

\begin{theorem}[Daniell's extension theorem]\label{thm-extension-of-Radon-integral}
    $\int:C_c(X)\to\R$を局所コンパクトハウスドルフ空間$X$上のRadon積分とする.
    \begin{enumerate}
        \item 可積分関数の空間$\L^1(X)$は$C_c(X)$を部分空間にもつ線型空間である.
        \item 束演算$\land,\lor$について閉じている.
        \item $\int:\L^1(X)\to\R$は正な線型汎函数で,$C_c(X)$上のRadon積分の延長となっている.
    \end{enumerate}
\end{theorem}

\begin{corollary}[三角不等式]
    $f\in\L^1(X)$ならば,$\abs{f}\in\L^1(X)$でえ,$\Abs{\int f}\le\int\abs{f}$.
\end{corollary}

\subsection{Lebesgueの優収束定理}

\begin{tcolorbox}[colframe=ForestGreen, colback=ForestGreen!10!white,breakable,colbacktitle=ForestGreen!40!white,coltitle=black,fonttitle=\bfseries\sffamily,
title=]
    今回の定義では,積分の極限に対する保存性が極めて明瞭に,普遍的に議論できる.
\end{tcolorbox}

\begin{theorem}[monotone convergence theorem]
    関数$f:X\to\R$はある$\L^1(X)$の単調増加列$(f_n)$の各点収束極限であり,$\sup_{n\in\N}\int f_n<\infty$を満たすとする.
    このとき,$f\in\L^1(X)$で,$\int f=\lim_{n\to\infty}\int f_n$である.
\end{theorem}

\begin{lemma}[Fatou's lemma]
    $\L^1(X)_+$の列$(f_n)$は$\forall_{x\in X}\;\liminf_{n\to\infty}f_n(x)<\infty$かつ$\liminf_{n\to\infty}\int f_n<\infty$を満たすとする.
    このとき,$\liminf_{n\to\infty}f_n\in\L^1(X)$で,$\int\liminf_{n\to\infty}f_n\le\liminf_{n\to\infty}\int f_n$.
\end{lemma}

\begin{theorem}[Lebesgue convergence theorem]
    $\L^1(X)$の列$(f_n)$がある$g\in\L^1(X)_+$に関して$\forall_{n\in\N}\;\abs{f_n}\le g$を満たしながら
    $f:X\to\R$に各点収束するとする.このとき,$f\in\L^1(X)$で,$\int f=\lim_{n\to\infty}\int f_n$である.
\end{theorem}

\subsection{Stieltjes積分}

\begin{tcolorbox}[colframe=ForestGreen, colback=ForestGreen!10!white,breakable,colbacktitle=ForestGreen!40!white,coltitle=black,fonttitle=\bfseries\sffamily,
title=]
    $\R$上のRadon積分の古典的構成法を議論する.アイデアとしては,ある有界変動関数$\rho:[a,b]\to\R$を定めれば,これを累積分布関数として分布が定まり,それについての積分が定まる.そして,$C[a,b]$の双対空間の元は,このような積分で尽きる.
    これをStieltjes積分という.\footnote{Riemannの方法を一般化したStieltjes積分を,Lebesgue積分の方法で一般化させたLebesgue-Stieltjes積分のことを,歴史的には主な貢献者の名前をとってRadon積分と呼んだ.}
\end{tcolorbox}

\begin{definition}[Stieltjes integral]
    $m:\R\to\R$を単調増加な関数とすると,$m$の不連続点は高々可算個である.
    このとき,$m(x):=\sup\Brace{n(y)\in\R\mid y<x}$と定め直すことで,$m$は左半連続であると仮定しても一般性を失わない.

    任意の$f\in C_c(X)$について,区間$\supp f\subset[a,b]$の分割$\lambda=(x_i)_{i\in n+1}$をとり,
    \begin{align*}
        (Sf)_k&:=\sup\Brace{f(x)\in\R\mid x_{k-1}\le x<x_k},&(If)_k&:=\inf\Brace{f(x)\in\R\mid x_{k-1}\le x<x_k},
    \end{align*}
    と定め,これを用いて
    \begin{align*}
        {\sum_{\lambda}}^*f&=\sum_{k=0}^n(Sf)_k(m(x_k)-m(x_{k-1})),&{\sum_{\lambda}}_*f&=\sum_{k=0}^n(If)_k(m(x_k)-m(x_{k-1})),
    \end{align*}
    と定める.$[a,b]$の有限な分割全体のなす有向集合$\Lambda$について,2つのネット$\paren{{\sum_\lambda}^* f}_{\lambda\in\Lambda},\paren{{\sum_\lambda}_*f}_{\lambda\in\Lambda}$
    が定まる.それぞれ単調増加,単調減少で,$\forall_{\lambda\in\Lambda}\;{\sum_\lambda}_*f\le{\sum_\lambda}^*f$を満たす.
    $f$は特に一様連続であるから,$\forall_{\ep>0}\;\exists_{\delta>0}\;x_k-x_{k-1}<\delta\Rightarrow(Sf)_k-(If)_k<\ep$.
    よって,$\lambda_0$を長さ$(b-a)\delta^{-1}<n$の等分割とすると,$\forall_{\lambda\ge\lambda_0}\;{\sum_\lambda}^*f-{\sum_\lambda}_*f<\ep(m(b)-m(a))$.
    よって,2つのネットは同一の実数に収束する.
    こうして定まる積分$\int:C_c(\R)\to\R$を\textbf{Stieltjes積分}といい,$\int fdm$と表す.
\end{definition}
\begin{remarks}
    Riemann-Stieltjes積分はこの$\int:C_c(\R)\to\R$の延長であり,Lebesgue-Stieltjes積分はその$\L^1(\R)$への更なる延長である(多分).
    しかし,Riemann可積分関数のクラスは,測度の言葉で簡明な特徴付けはあるが,単調列の極限に関して安定でない.また,Riemann-Stieltjes積分は,$\R$の全順序性により過ぎているため,高次元に一般化できない.
\end{remarks}

\begin{lemma}
    Stieltjes積分$\int:C_c(\R)\to\R$はRadon積分である.
\end{lemma}

\begin{theorem}
    $\R$上のRadon積分$\int:C_c(\R)\to\R$について,ある単調増加関数$m:\R\to\R$が存在して,これについてのStieltjes積分と一致する.
\end{theorem}

\begin{theorem}[有界閉区間上の連続関数の空間の双対空間]
    $a<b\in\R$と$F:[a,b]\to\R$について,次の2条件は同値.
    \begin{enumerate}
        \item $F\in C[a,b]^*$.
        \item 有界変動関数$\rho:[a,b]\to\R$が存在して,$\forall_{u\in C[a,b]}\;F(u)=\int^b_au(x)d\rho(x)$が成り立つ.
    \end{enumerate}
    さらにこのとき,$\norm{F}=V(\rho)$となる.
\end{theorem}
\begin{proof}
    Hahn-Banachの定理による.
\end{proof}

\section{可測性}

\begin{notation}[characteristic function]
    特性関数を$[A]=\chi_A$で表す.
\end{notation}

\subsection{点列完備性}

\begin{definition}[(monotone) sequentially complete]
    クラス$\F\subset\Map(X,\R)$が\textbf{(単調)点列完備}とは,$\F$の(単調)列の各点収束極限が$\F$に属していることをいう.
\end{definition}

\begin{lemma}\label{lemma-sequential-completion}
    代数$\A\subset\Map(X,\R)$を$\land,\lor$について閉じているBoole代数とする.
    このとき,単調列完備化$\B(\A)$は再び$\land,\lor$について閉じているBoole代数である.
\end{lemma}

\subsection{集合代数}

\begin{definition}
    集合$M\subset P(X)$について,
    \begin{enumerate}
        \item 次の2条件を満たすとき,$M$を$\sigma$-環という:
        \begin{enumerate}[(a)]
            \item 任意の可算族$(A_n)$について,$\cup_{n}A_n\in M$.これは$\emptyset\in M$を含意する.
            \item 任意の$A,B\in M$について,$A\setminus B\in M$.
        \end{enumerate}
        $\sigma$-環は$\delta$-環である.
        有限合併についてのみ閉じているとき,$M$をBoole環という.
        これは,加法を対称差$A+B=(A\cup B)\setminus(A\cap B)=(A\setminus B)\cup(B\setminus A)$,積を共通部分として環をなすためである.
        \item $X\in M$を満たすBoole環をBoole代数といい,$X\in M$を満たす$\sigma$-環を$\sigma$-代数という.
        これは,$\sigma$-環が定める環$(M,+,\cap)$が単位的であることに同値.
    \end{enumerate}
\end{definition}
\begin{remarks}
    $\sigma$-環$M$の演算$+,\cdot$は,対称差$A+B=(A\setminus B)\cup(B\setminus A)$と共通部分$A\cdot B=A\cap B$である.
    環が代数になる条件は,積の単位元$X$を含むかどうかである.
    「環」の用語はBoole環の略であり,歴史的には環と言っても単位的であることを必要としなかった.
    したがって,単にBoole環$P(X)$の部分環を指す.
    $\sigma$はドイツ語のSumme,$\delta$はドイツ語のDurchschnittから来ている.
\end{remarks}

\begin{lemma}\label{lemma-function-space-defined-by-measurable-sets}
    $S\subset P(X)$を集合系とする.
    \begin{enumerate}
        \item $\F\subset\Map(X,\R)$を単調点列完備な代数とする.これが定める集合系$S:=\Brace{A\in P(X)\mid [A]\in\F}$は$\sigma$-環である.
        \item $S\subset P(X)$を$\sigma$-代数とする.$\F:=\Brace{f\in\Map(X,\R)\mid\forall_{t\in\R}\{f>t\}\in S}$と定めると,$\F$は点列完備な,$\land,\lor$について閉じている単位的代数であり,$\forall_{f\in\F}\;\forall_{p>0}\;\abs{f}^p\in\F$を満たす.
    \end{enumerate}
\end{lemma}

\subsection{Borel関数論}

\begin{tcolorbox}[colframe=ForestGreen, colback=ForestGreen!10!white,breakable,colbacktitle=ForestGreen!40!white,coltitle=black,fonttitle=\bfseries\sffamily,
title=単関数各点近似が出来るクラスが可測関数である]
    BaireとLebesgueによるBorel集合論.
    Borel関数の空間は$C_c(X)$の各点完備化として得られる.
    これが単関数の理論である.
\end{tcolorbox}

\begin{example}[Euclid空間のBorel集合]
    $\R^n$の開集合は,開矩形の可算積で表せる.開矩形は$2n$個の開半空間の共通部分として表せる.
    したがって,$\R^n$のBorel集合系は,すべての開半空間$\Brace{x\in\R^n\mid x_k>t}_{t\in\R,k\in[n]}$が生成する$\sigma$-環である.
\end{example}

\begin{definition}[Borel map]
    位相空間$X,Y$の間の写像$f:X\to Y$がBorel写像であるとは,$\forall_{B\in\B_Y}\;f^{-1}(B)\in\B_X$を満たすことをいう.
    連続写像はBorel写像である.Borel写像の合成はBorelである.
\end{definition}

\begin{corollary}
    位相空間$X$上のBorel関数のクラス$\B(X)$は,点列完備な単位的代数で,束演算$\land,\lor$について閉じている.
    また,$\forall_{f\in\B(X)}\;\forall_{p>0}\;\abs{f}^p\in\B(X)$.
\end{corollary}

\begin{lemma}[コンパクト集合の定義関数の近似]
    局所コンパクトハウスドルフ空間$X$について,任意のコンパクト集合$C\subset X$について,$C_c(X)$上の単調減少ネット$(f_\lambda)$が存在して,$f_\lambda\searrow[C]$が成り立つ.
    $X$が第2可算であるとき,数列についての議論で十分.
\end{lemma}

\begin{proposition}
    $X$を第2可算な局所コンパクトハウスドルフ空間とする.
    \begin{enumerate}
        \item $X$上のBorel関数の空間$\B(X)$は,$C_c(X)$の単調点列完備化である.
        \item $X$上の有界Borel関数の空間$\B_b(X)$は,$C_c(X)$の$l^\infty(X)$上での単調点列完備化である.
    \end{enumerate}
\end{proposition}

\begin{remark}[Baire function]\label{remark-Baire-function}
    第二可算とは限らない局所コンパクトハウスドルフ空間$X$については,一般に,単調点列完備化について$\B(C_c(X))\subset\B(X)$が成り立つ.
    ただし,$C_c(X)$の単調点列完備化を$\B(C_c(X))$で表した.
    この真に小さいかもしれないクラスを\textbf{Baire関数}という.
    またこのクラスは$\land,\lor$について閉じている(補題\ref{lemma-sequential-completion}).
    $1$がBaire関数であることは,$X$が$\sigma$-コンパクトであることに同値.
    定義関数$[B]$がBaire関数であるとき,$B$を\textbf{Baire集合}という.
    Baire集合はBorel $\sigma$-代数$\B_X$の中の$\delta$-環をなす.
    この$\delta$-環は$X$内のコンパクト$G_\delta$-集合によって生成される.
\end{remark}

\begin{definition}
    部分集合$C\subset X$がコンパクト$G_\delta$-集合であるとは,$\exists_{f\in C_c(X)}\;\exists_{\ep>0}\;C=\{f\ge \ep\}$が成り立つことをいう.
\end{definition}

\begin{remark}[Borel関数の空間の得方]
    一般の$X$について,$\B(X)$を完備化として得たいときは,$C^{1/2}_b(X)$を有界な下半連続関数の空間として,クラス
    \[\F:=C^{1/2}_b(X)-C^{1/2}_b(X)\]
    を考える.これは線型空間で,代数でもある.
    任意の開集合$A$について定義関数$[A]$を含むから,$\F$は$\B(X)$上で稠密である.
\end{remark}

\subsection{可測集合}

\begin{definition}\label{def-measurable-sets}
    $X$を局所コンパクトハウスドルフ空間とする.
    \begin{enumerate}
        \item $\cC$で$X$のコンパクト部分集合系とする.
        \item $\int:C_c(X)\to\R$をRadon積分(=非負性を保存する線型汎関数)とし,$\L^1(X)$をこれについての可積分関数とする\ref{thm-extension-of-Radon-integral}.
        \[\M^1:=\Brace{B\in P(X)\mid [B]\in\L^1(X)},\qquad\M:=\Brace{A\in P(X)\mid\forall_{C\in\cC}\;A\cap C\in\M^1}\]
        と定め,$\M$の元を\textbf{可測集合}という.
    \end{enumerate}
\end{definition}

\begin{proposition}
    局所コンパクトハウスドルフ空間$X$上のRadon積分$\int$が定める可測集合系$\M$は,Borel集合系$\B$を含む$\sigma$-代数で,$\cC\subset\M^1$を満たす.
\end{proposition}

\subsection{可測関数}

\begin{definition}\label{def-measurable-function}
    Radon積分$\int:C_c(X)\to\R$について,
    \begin{enumerate}
        \item 関数$f:X\to\R$が\textbf{可測}であるとは,$\forall_{t\in\R}\;\{f>t\}\in\M$を満たすことを言う.これは$\forall_{B\in\B_\R}\;f^{-1}(B)\in\M$と$\forall_{g\in\B(\R)}\;f\circ g\in\L(X)$を含意する.
        \item $X$上の可測関数の空間を$\L(X)$で表す.$\B\subset\M$より,$\B(X)\subset\L(X)$である.
    \end{enumerate}
\end{definition}

\begin{proposition}
    局所コンパクトハウスドルフ空間$X$上のRadon積分に関する可測関数の空間$\L(X)$は,
    \begin{enumerate}
        \item 点列完備な単位的代数で,
        \item $\lor,\land$について閉じており,
        \item $\forall_{f\in\L(X)}\;\abs{f}^p\in\L(X)$を満たす.
    \end{enumerate}
\end{proposition}
\begin{proof}
    補題\ref{lemma-function-space-defined-by-measurable-sets}(2)の具体例である.
\end{proof}

\begin{lemma}\label{lemma-regularity-of-Radon-integral}
    任意の部分集合$B\subset X$について,
    \[\int_*[B]=\sup\Brace{\int[C]\in\R_{\ge0}\;\middle|\;C\subset B,C\in\C},\]
    \[\int^*[B]=\inf\Brace{\int^*[A]\in\R_{\ge0}\;\middle|\;B\subset A,A:\text{open}}.\]
    また,$B\in\M^1$は,$B\in\M$かつ$\int^*[B]<\infty$に同値.
\end{lemma}

\begin{theorem}\label{thm-Radon-integrability}
    $\int:C_c(X)\to\R$を局所コンパクトハウスドルフ空間$X$上のRadon積分とする.
    \begin{enumerate}
        \item 任意の可積分関数は可測である:$\L^1(X)\subset\L(X)$.
        \item 任意の可測関数について,可積分であることと,$\int^*\abs{f}<\infty$は同値.
    \end{enumerate}
\end{theorem}

\section{測度}

\begin{tcolorbox}[colframe=ForestGreen, colback=ForestGreen!10!white,breakable,colbacktitle=ForestGreen!40!white,coltitle=black,fonttitle=\bfseries\sffamily,
title=]
    Rieszの表現定理より,測度と積分は数学的には等価であることがわかる.
\end{tcolorbox}

\subsection{Radon測度}

\begin{tcolorbox}[colframe=ForestGreen, colback=ForestGreen!10!white,breakable,colbacktitle=ForestGreen!40!white,coltitle=black,fonttitle=\bfseries\sffamily,
title=]
    測度は積分の特別な場合である.
\end{tcolorbox}

\begin{definition}[measure, Radon measure]\mbox{}\label{def-Radon-measure}
    \begin{enumerate}
        \item $\sigma$-環$S\subset P(X)$上の\textbf{測度}とは,関数$\mu:S\to[0,\infty]$であって,$\sigma$-加法的であるものをいう:$\mu\paren{\cup A_n}=\sum\mu(A_n)$.
        \item $X$が局所コンパクトハウスドルフ空間,$\B_X\subset S$を満たす$(X,S)$について,次の2条件を満たす測度を\textbf{Radon測度}という:
        \begin{enumerate}[(i)]
            \item (locally finite / locally integrable) $\forall_{C\in\cC}\;\mu(C)<\infty$.
            \item (inner regularity) $\forall_{A\in S}\;\mu(A)=\sup\Brace{\mu(C)\in\R_{\ge0}\mid C\subset A,C\in\cC}$.
        \end{enumerate}
    \end{enumerate}
\end{definition}
\begin{remark}
    $\B_X\subset S$であるから,Radon測度はBorel測度である.
    幾何学的に興味のあるほとんどの測度はRadonである.
\end{remark}

\begin{proposition}[Radon積分からRadon測度の復元]\label{prop-recovery-of-Radon-measure}
    局所コンパクトハウスドルフ空間$X$上のRadon積分$\int:C_c(X)\to\R$に,次のRadon積分が定める可測集合$\M$上の2つの測度$\mu^*,\mu_*:\M\to[0,\infty]$が対応する:
    \begin{enumerate}
        \item $\mu^*(A):=\int^*[A]$.
        \item $\mu_*(A):=\int_*[A]$.
    \end{enumerate}
    このとき,$\mu_*$はRadon測度で,$\mu^*$は外正則:
    \[\mu^*(A)=\inf\Brace{\mu^*(B)\in[0,\infty]\mid A\subset B,B:\text{open}}\]
    で$\mu_*\le\mu^*$を満たす.
    また,$A$が測度確定な可測集合$\M^1(\subset\M)$\ref{def-measurable-sets}の可算合併として表せるとき,$\mu_*(A)=\mu^*(A)$.
\end{proposition}
\begin{remarks}
    したがって,$X$が第2可算性などの条件を満たすならば,Radon測度とは正則Borel測度に他ならない.
\end{remarks}

\subsection{Rieszの表現定理}

\begin{tcolorbox}[colframe=ForestGreen, colback=ForestGreen!10!white,breakable,colbacktitle=ForestGreen!40!white,coltitle=black,fonttitle=\bfseries\sffamily,
title=]
    Radon積分からRadon測度を構成する手順を示したが,きれいに対応が付く.
\end{tcolorbox}

\begin{lemma}
    $\mu$を$\sigma$-代数$S\subset P(X)$上の測度,$\F$を$S$-可測関数全体からなる集合とする:$\forall_{t\in\R}\;\{f>t\}\in S$\ref{lemma-sequential-completion}.
    このとき,非負性を保存する斉次な加法的関数$\Phi:\F_+\to[0,\infty]$であって.次の2条件を満たすものが一意的に存在する:
    \begin{enumerate}
        \item $\forall_{A\in S}\;\Phi([A])=\mu(A)$.
        \item $\forall_{f_n,f\in\F_+}\;f_n\nearrow f\Rightarrow\Phi(f)=\lim\Phi(f_n)$.
    \end{enumerate}
\end{lemma}

\begin{theorem}[Riesz' representation theorem 09]\label{thm-Riesz-representation}
    局所コンパクトハウスドルフ空間$X$上のRadon積分と,$X$のBorel集合$\B_X$上のRadon測度との間に,
    \[\mu(A)=\int_*[A]\]
    によって定まる全単射が存在する.
\end{theorem}

\begin{remark}
    Rieszの表現定理によると,Daniellの延長定理\ref{thm-extension-of-Radon-integral}はまだ最大の延長ではないことがわかる.
\end{remark}

\begin{definition}[essential integral]\mbox{}
    \begin{enumerate}
        \item $\L^1_\ess(X):=\Brace{f\in\L(X)\;\middle|\;\int_*\abs{f}<\infty}$.
        \item $\L^1_\ess(X)$上の積分$\int_\ess$が,
        \[\int_\ess f:=\int_*f\lor 0+\int^*f\land 0\]
        によって定まり,これはLebesgueの優収束定理を満たすようなRadon積分$\int:C_c(X)\to\R$の延長であるような非負性を保存する線型汎関数である.
        $\mu_*\ne\mu^*$のとき,$\L^1(X)\subsetneq\L^1_\ess(X)$である.\footnote{$X$がパラコンパクトであるとき,すなわちコンパクト集合の可算合併であるとき,$\mu_*=\mu^*$であることに注意.}
    \end{enumerate}
\end{definition}

\begin{proposition}\label{prop-Radon-integral-on-paracompact-spaces}
    $X$を$\sigma$-コンパクトな局所コンパクトハウスドルフ空間とする.
    \begin{enumerate}
        \item 任意のRadon測度は外正則である.
        \item $\int:C_c(X)\to\R$をRadon積分とすると,これについての可測関数$f$が可積分であることは$\int_*\abs{f}<\infty$に同値.
    \end{enumerate}
\end{proposition}

\subsection{拡張積分}

\begin{definition}
    $X$を局所コンパクトハウスドルフ空間,$\int:C_c(X)\to\R$をRadon積分とする.
    \begin{enumerate}
        \item 可測関数$f\ge0$が拡張積分可能であるとは,$\int_*f=\int^*f\in\R\cup\{\infty\}$を満たすことをいう.
        \item 一般の可測関数$f$が拡張積分可能であるとは,$f\lor0$と$-(f\land0)$のいずれかが可積分で,もう一方が拡張積分可能であることをいう.
        \item $\int_\ext f=\int f\lor0-\int(-f\land 0)\in\R\cup\{\pm\infty\}$と表す.$\infty-\infty$の場合は定義できないので,拡張積分可能な関数全体の集合は線型空間ではない.
    \end{enumerate}
\end{definition}

\begin{lemma}
    正な拡張積分可能な関数は正錐をなし,Radon積分はその上に非負性を保存する斉次な$\sigma$-加法的関数として作用する.
\end{lemma}

\begin{lemma}
    $X$が$\sigma$-コンパクトであるとき,任意の正な可測関数は拡張可積分であり(命題\ref{prop-Radon-integral-on-paracompact-spaces}),
    任意の可測関数は$\int f\lor0=\infty,\int f\land0=-\infty$である場合を除いて拡張可積分である.
\end{lemma}

\section{$L^p$-空間}

\begin{tcolorbox}[colframe=ForestGreen, colback=ForestGreen!10!white,breakable,colbacktitle=ForestGreen!40!white,coltitle=black,fonttitle=\bfseries\sffamily,
title=]
    局所コンパクトハウスドルフ空間$X$とその上のRadon積分$\int:C_c(X)\to\R$を1つ固定し,
    可測関数のなす線型束$\L(X)$の各種部分空間を調べる.
\end{tcolorbox}

\subsection{零集合}

\begin{definition}\mbox{}\label{def-null-set}
    \begin{enumerate}
        \item $\cN(X):=\Brace{f\in\L(X)\;\middle|\;\int\abs{f}=0}$の元を\textbf{零関数}という.
        \item $\cN:=\Brace{N\in\M\mid[N]\in\cN(X)}$の元を\textbf{零集合}という.
    \end{enumerate}
\end{definition}

\begin{lemma}\mbox{}
    \begin{enumerate}
        \item $\cN(X)$は$\L(X)$内の点列完備なイデアルであり,束でもある.
        \item $\cN$はBoole環$\M$内のイデアルであり,$\sigma$-環でもある.
    \end{enumerate}
\end{lemma}

\begin{definition}
    今後,$X$上の関数とは,$X\to\R\cup\{\pm\infty\}$を指すとし,$f\in\L(X)\Leftrightarrow\exists_{N\in\cN}\;\forall_{x\notin N}\;f(x)\in\R\land[X\setminus N]f\in\L(X)$であるとする.
\end{definition}
\begin{remarks}
    代数としての,そして束としての構造が「殆ど至る所」しか成り立たなくなり,その構造を回復するには$\L(X)/\cN(X)=:L(X)$なる空間を考える必要があり,この空間はもはや直接の関数空間ではなくなる.
    これは一見不便であるが,次のBeppo-Leviの定理という名の単調収束定理の変種のように,関数の極限はこのように拡張実数値関数を自然に生んでしまう.
    その際には,個別に議論するよりかは,各積分に応じて「殆ど至る所」という言及の仕方が明瞭になりえる.
\end{remarks}

\begin{theorem}[Beppo-Levi's theorem]
    $\L^1(X)$の列$(f_n)$は,$\forall_{n\in\N}\;f_n(x)\le f_{n+1}(x)\;\ae$を満たし,$\lim\int f_n<\infty$であるとする.
    このとき,ある元$f\in\L^1(X)$が存在して,$\int f=\lim\int f_n$かつ$f(x)=\lim f_n(x)\;\ae$を満たす.
\end{theorem}

\subsection{Radon積分のLebesgue分解}

\begin{definition}[continuous / diffuse, atomic]\mbox{}
    \begin{enumerate}
        \item Radon積分$\int$が\textbf{連続}であるとは,$\forall_{x\in X}\;\int[\{x\}]=0$を満たすことをいう.
        \item Radon積分が\textbf{原子的}であるとは,ある集合$S\subset X$が存在して,$X\setminus S\in\cN$かつ$\forall_{C\in\cC}\;S\cap C$が可算であることをいう.
    \end{enumerate}
\end{definition}
\begin{remarks}
    連続な積分については,可算個の点で成り立たない命題も,殆ど至る所成り立ち得る.
    「連続」なる用語の由来は,$X=\R$の場合においてはRadon積分はStieltjes積分とも呼ばれるが,increment関数$m$が連続であることとRadon積分が連続であることが同値になることから来ている.
\end{remarks}
\begin{example}
    Dirac積分$\delta_x(f)=f(x)$は原子的なRadon積分である.
\end{example}

\begin{lemma}
    任意のRadon積分$\int$について,連続部分$\int_c$と原子的部分$\int_a$とが存在して,$\int=\int_c+\int_a$と表せる.
\end{lemma}

\subsection{Lebesgue空間}

\begin{definition}[Lebesgue space]
    実数$p\in[1,\infty)$について,位数$p$のLebesgue空間とは,
    \[\L^p(X):=\Brace{f\in\L(X)\;\middle|\;\abs{f}^p\in\L^1(X)}.\]
    また,$\L^p(X)$上の関数を$\norm{f}_p:=\paren{\int\abs{f}^p}^{1/p}$と定める.
\end{definition}

\begin{lemma}\mbox{}
    \begin{enumerate}
        \item $\L^p(X)$は線型空間である.
        \item $f\mapsto\norm{f}_p$は非負性を保存し,斉次である.
        \item $f\mapsto\norm{f}_p$はセミノルムである.
    \end{enumerate}
\end{lemma}

\begin{definition}[essential supremum]\mbox{}
    \begin{enumerate}
        \item 可測関数$f\in\L(X)$について,
        \[\esssup f:=\inf\Brace{t\in\R\;\middle|\;\int^*[\{f>t\}]=0}=\inf\Brace{t\in\R\;\middle|\;\int^*(f-f\land t)=0}.\]
        \item Lebesgue空間$\L^\infty(X)$で,$X$上の本質的に有界な可測関数のなす線型空間を表す.$\norm{f}_\infty:=\esssup\abs{f}$はその上にセミノルムを定める.
    \end{enumerate}
\end{definition}

\begin{lemma}[H\"{o}lder's inequality]
    共役指数$p,q$について,$f\in\L^p(X),g\in\L^q(X)$とする:$p^{-1}+q^{-1}=1$.
    このとき,$fg\in\L^1(X)$で,$\norm{fg}_1\le\norm{f}_p\norm{g}_q$.
\end{lemma}

\begin{lemma}[Minkowski's inequality]
    $f,g\in\L^p(X)$について$f+g\in\L^p(X)$で,$\norm{f+g}_p\le\norm{f}_p+\norm{g}_p$.
\end{lemma}

\subsection{完備化}

\begin{lemma}
    任意の$p\in[1,\infty]$について,
    \begin{enumerate}
        \item $\cN(X)\subset\L^p(X)$.
        \item $\cN(X)=\Brace{f\in\L^p(X)\mid\norm{f}_p=0}$.
        \item 商空間$L^p(X):=\L^p(X)/\cN(X)$は$\norm{-}_p$についてノルム空間をなす.
    \end{enumerate}
\end{lemma}

\begin{proposition}[Egoroff's theorem]
    $p\in[1,\infty)$とする.任意の$\L^p(X)$のCauchy列と任意の$\ep>0$について,部分列$(f_n)$と開集合$A$であって$\int[A]<\ep$を満たすものと零集合$N$が存在して,$(f_n)$は$X\setminus A$上一様に収束し,$X\setminus N$上各点収束する.
\end{proposition}

\begin{theorem}
    $L^p(X)$はBanach空間である.
\end{theorem}

\subsection{稠密部分集合}

\begin{proposition}\label{prop-dense-subset-of-Lp}
    任意の$p<\infty$について,$C_c(X)$(の商写像についての像)は$L^p(X)$上稠密である.
\end{proposition}
\begin{remarks}
    $p=\infty$の場合は,双対\ref{thm-duality-of-Lp}により,$(L^1(X))^*$の
    $w^*$-位相について,$C_c(X)\mono L^\infty(X)$は$w^*$-稠密である.従う.
\end{remarks}

\begin{corollary}
    $p\in[1,\infty)$とする.任意の$f\in\L^p(X)$と$\ep>0$について,開集合$A$であって$\int[A]<\ep$を満たすものが存在し,$f|_{X\setminus A}\in C_0(X\setminus A)$を満たす.
\end{corollary}

\subsection{Borel関数近似}

\begin{proposition}
    $p\in[1,\infty)$とする.
    \begin{enumerate}
        \item 任意の可測関数$f\in\L^p(X)$について,Borel関数$g\in\B(X)$が存在して,$f-g\in\cN(X)$を満たす.
        \item 同様のことが,$X$の$\sigma$-コンパクト部分集合$S$について$X\setminus S$上殆ど至る所$0$な可測関数$f\in\L(X)$についても成り立つ.
    \end{enumerate}
\end{proposition}

\subsection{複素関数について}

\begin{tcolorbox}[colframe=ForestGreen, colback=ForestGreen!10!white,breakable,colbacktitle=ForestGreen!40!white,coltitle=black,fonttitle=\bfseries\sffamily,
title=]
    基本はスムーズに証明されるが,次の命題だけ特別な証明が必要になる.
\end{tcolorbox}

\begin{proposition}
    可測関数$f:X\to\C$が可積分であることと$\int^*\abs{f}<\infty$を満たすことは同値.
    またこの条件下で$\Abs{\int f}\le\int\abs{f}$.
\end{proposition}
\begin{remarks}
    証明抽出により,$f:X\to\C$について,$f\in\L^p(X)\Leftrightarrow\abs{f}\in\L^p(X)$もわかる.
\end{remarks}

\subsection{Lebesgue空間の相互関係}

\begin{tcolorbox}[colframe=ForestGreen, colback=ForestGreen!10!white,breakable,colbacktitle=ForestGreen!40!white,coltitle=black,fonttitle=\bfseries\sffamily,
title=]
    この結果は$\bF=\R,\C$のいずれについても同様の証明によって成り立つ.
\end{tcolorbox}

\begin{proposition}
    $1\le p<r<q\le\infty$のとき,
    \begin{enumerate}
        \item $\L^p(X)\cap\L^q(X)\subset\L^r(X)$.
        \item $\forall_{f\in\L^p(X)\cap\L^q(X)}\;\norm{f}_r\le\norm{f}_p\lor\norm{f}_q$.
    \end{enumerate}
\end{proposition}

\begin{corollary}\mbox{}
    \begin{enumerate}
        \item $\limsup\norm{f}_r\le\norm{f}_\infty$.
        \item $\forall_{f\in\L^p(X)\cap\L^\infty(X)}\;\norm{f}_r\to\norm{f}_\infty$.
    \end{enumerate}
\end{corollary}

\subsection{包含関係}

\begin{discussion}
    一般に(特に$X=\R^n$でLebesgue積分を考えているとき)$\L^p(X)\not\subset\L^q(X)\;(p\ne q)$である.
    しかし,次の2つの場合に限って包含関係は成り立つ.
    \begin{enumerate}
        \item $\int$が有限である:$\int 1<\infty$.
        \item $\int$はある最小アトムに関して原子的である:$\exists_{\ep>0}\;\forall_{A\in\M}\;\int[A]=0\lor\int[A]\ge\ep$.
    \end{enumerate}
    (1)の場合は,議論を確率分布の場合$\int 1=1$に限っても一般性は失われない.
    (2)の場合は,$X$が$\sigma$-コンパクトならば,積分が$X$のある可算部分集合上に集中しており,各アトムの重さが$\ep$以上であることをいう.
    このとき,$L^p$空間は数列空間$l^p$に同型となる.
\end{discussion}

\begin{corollary}
    $\int 1=1$かつ$1\le p<q\le\infty$のとき,$\L^q(X)\subset\L^p(X)$かつ$\norm{-}_p\le\norm{-}_q$である.
\end{corollary}
\begin{remarks}[確率空間上のLebesgue空間]
    確率測度については,$p\in[1,\infty]$が小さいほど空間$\L^p(X)$は大きく,ノルム$\norm{f}_p$は小さい.
\end{remarks}

\begin{corollary}
    任意の$1\le p<q\le\infty$について,$l^p\subset l^q$かつ$\norm{-}_q\le\norm{-}_p$が成り立つ.
\end{corollary}

\section{双対理論}

\begin{tcolorbox}[colframe=ForestGreen, colback=ForestGreen!10!white,breakable,colbacktitle=ForestGreen!40!white,coltitle=black,fonttitle=\bfseries\sffamily,
title=]
    Radon-Nikodymの定理を考えるにあたって,
    位相的測度論では,Radon測度の内部正則性と本質的積分を用いて,$\sigma$-有限性の条件を取り除くことが可能だが,特にこれはしないこととする.
    すなわち,$X$を$\sigma$-コンパクトな局所コンパクトハウスドルフ空間とする.
    このとき,Radon測度は相対コンパクト集合に有限な測度を与えるから,$X$は$\sigma$-有限である(パラコンパクト性の特徴付け\ref{lemma-characterization-of-paracompactness}).
\end{tcolorbox}

\begin{remark}
    一般の局所コンパクトハウスドルフ空間$X$上の$\sigma$-有限なRadon積分を考えれば十分である.
    実際,$\{A_n\}\subset\M^1$かつ$\cup A_n=X$を満たすならば,補題\ref{lemma-regularity-of-Radon-integral}により,少し大きくして$A_n$はすべて開であると仮定して良い.
    次に,再び補題\ref{lemma-regularity-of-Radon-integral}より,各$A_n$に対して,相対コンパクトな開集合の列$(B_{nm})$であって$\forall_{m\in\N}\;\o{B_{nm}}\subset B_{nm+1}$を満たし,$A_n\setminus\cup B_{nm}\in\cN$を満たすものが取れる.
    すると,$Y:=\cup_{n,m\in\N}B_{nm}$は$\sigma$-コンパクトな開集合であり,$X\setminus Y\in\cN$を満たす.こうして$Y$について考えれば,$\sigma$-コンパクト性を仮定して議論していることと等価である.
\end{remark}

\subsection{絶対連続性}

\begin{definition}[absolutely continuous, equivalent]\mbox{}
    \begin{enumerate}
        \item $\int_0,\int:C_c(X)\to\R$が次の命題の同値な条件を満たすとき,$\int_0$は$\int$について\textbf{絶対連続}であるといい,$\int_0\ll\int$と表す.
        \item $\int_0\ll\int\land\int\ll\int_0$が成り立つとき,\textbf{同値}であるといい,$\int_0\sim\int$で表す.これは$\cN(X)=\cN_0(X)$と同値で,$L^\infty_0(X)=L^\infty(X)$と同値.
    \end{enumerate}
\end{definition}

\begin{proposition}
    $\int_0,\int:C_c(X)\to\R$を局所コンパクトハウスドルフ空間$X$上のRadon積分とする.次の3条件は同値である.
    \begin{enumerate}
        \item 任意の単調減少列$\{f_n\}\subset C_c(X)_+$について,$\lim\int f_n=0\Rightarrow\lim\int_0f_n=0$が成り立つ.
        \item 任意のBorel集合$N\in\B(X)$について,$\int[N]=0\Rightarrow\int_0[N]=0$.
        \item 任意の非負Borel関数$f\in\B(X)_+$について,$\int f=0\Rightarrow\int_0f=0$.
    \end{enumerate}
\end{proposition}

\subsection{Radon-Nikodymの定理}

\begin{definition}[locally integrable]\label{def-locally-integrable-function}
    関数$f$が$\forall_{C\in\cC}\;[C]f\in\L^1(X)$を満たすとき,\textbf{局所可積分}であるという.このとき$f$は可測である.
\end{definition}
\begin{remarks}
    これは可測集合$\M$と$\M^1$の別に一致する.集合の可測性には局所性が暗黙のうちに入っていたのである.
\end{remarks}

\begin{lemma}
    $X$上の局所可積分関数の空間$\L^1_\loc(X)$はRiesz空間である.
\end{lemma}

\begin{theorem}[Radon-Nikodym]
    $X$を$\sigma$-コンパクトな局所コンパクトハウスドルフ空間,$\int_0,\int:C_c(X)\to\R$をその上のRadon積分とする.
    このとき,次の2条件は同値.
    \begin{enumerate}
        \item $\int_0\ll\int$.
        \item Borel関数$m\ge0$が存在して,$\int$について局所可積分で,$\forall_{f\in\B(X)}\;\int_0f=\int fm$を満たす.
    \end{enumerate}
    なお,$m$は存在すれば零集合の差を除いて一意である.
\end{theorem}

\subsection{Jordan分解}

\begin{tcolorbox}[colframe=ForestGreen, colback=ForestGreen!10!white,breakable,colbacktitle=ForestGreen!40!white,coltitle=black,fonttitle=\bfseries\sffamily,
title=]
    $X$上のRadon積分の生成する複素線型空間を考えたい.これはRadon電荷$(C_c(X))^*$となる.
    これはHilbert空間上の作用素の分解(極表示)と同じ働きを持つ.
\end{tcolorbox}

\begin{definition}[Radon charge]
    \textbf{(複素)Radon電荷}とは,線型汎関数$\Phi:C_c(X)\to\C$であって,
    \[\forall_{f\in C_c(X)_+}\;\sup\Brace{\abs{\Phi(g)}\ge0\mid g\in C_c(X),\abs{g}\le f}<\infty\]
    を満たすものとする.
\end{definition}
\begin{lemma}
    $\tau$を弱位相,すなわちセミノルム$f\mapsto\Abs{\int f}$の誘導する位相として,位相線型空間$(C_c(X),\tau)$の双対空間が,Radon電荷全体のなす空間である.
\end{lemma}
\begin{proof}
    命題\ref{prop-separating-space-of-functionals}より.
\end{proof}

\begin{theorem}
    $X$を$\sigma$-コンパクトな局所コンパクトハウスドルフ空間,$\Phi:C_c(X)\to\R$をRadon電荷とする.
    このとき,Radon積分$\int$とBorel関数$u\in\B(X)$が存在して,$\abs{u}=1$と$\Phi=\int\cdot u\;\paren{\int\text{-}\ae}$を満たす.
\end{theorem}

\begin{corollary}[Jordan decomposition]
    任意のRadon電荷$\Phi:C_c(X)\to\R$について,2つのRadon積分$\int_+,\int_-$が存在して,互いに素なBorel集合内に台を持ち,$\Phi=\int_+-\int_-$と表せる.
\end{corollary}
\begin{remarks}[Hahn decomposition]
    証明中の分解$X=A_+\cup A_-\cup N$をHahn分解という.
\end{remarks}

\subsection{Radon電荷の空間}

\begin{definition}[total variation]\mbox{}
    \begin{enumerate}
        \item 定理から得られる積分$\int$を\textbf{全変動}といい,$\abs{\Phi}$で表す.
        $u$は「符号」にあたり,これは零関数の差を除いて一意である.
        \item Radon電荷$\Phi$が\textbf{有限}であるとは,全変動$\abs{\Phi}$が有限なRadon積分を定めることをいう.有限なRadon電荷全体の空間を$M(X)$で表し,$\norm{\Phi}=\abs{\Phi}(1)$をノルムとする.
    \end{enumerate}
\end{definition}
\begin{remark}
    全変動$\abs{\Phi}$は,
    \[\forall_{f\in C_c(X)}\;\abs{\Phi(f)}\le\int\abs{f}\]
    を満たすRadon積分$\int$の中で最小のものとして特徴付けられる.
    実電荷$\Phi$について,全変動とは$\abs{\Phi}=\Phi_++\Phi_-$に他ならない.
\end{remark}

\begin{proposition}[Riesz-Markov representation theorem]\label{prop-Riesz-Markov-3}
    $X$上の有限Radon電荷の空間$M(X)$にノルム$\norm{\Phi}=\abs{\Phi}(1)$を考えたものは,Banach空間$(C_0(X))^*$に等長同型である.
\end{proposition}

\begin{example}\label{exp-Riesz-Markov-theorem-to-sequence-spaces}
    $X=\N$のとき,$M(\N)=l^1$である.すなわち,$(c_0)^*\simeq l^1$.
\end{example}

\begin{proposition}[絶対連続測度の表現]\label{prop-Riesz-4}
    あるRadon積分$\int:C_c(X)\to\R$について,$\abs{\Phi}\ll f$を満たす有限なRadon電荷$\Phi$全体のなす空間は,次の写像によって$L^1(X)$に等長同型である:
    \[\xymatrix@R-2pc{
        L^1(X)\ar[r]&M(X)\\
        \rotatebox[origin=c]{90}{$\in$}&\rotatebox[origin=c]{90}{$\in$}\\
        f\ar@{|->}[r]&\Phi_f(g):=\int gf\;(f\in\L^1(X),g\in C_c(X))
    }\]
\end{proposition}

\subsection{Lebesgue空間の双対}

\begin{theorem}[双対]\label{thm-duality-of-Lp}
    $X$を$\sigma$-コンパクトな局所コンパクトハウスドルフ空間,$\int:C_c(X)\to\R$をその上のRadon積分とする.
    $p\in[1,\infty],q\in[1,\infty),p^{-1}+q^{-1}=1$について,双線型形式
    \[\brac{f,g}=\int fg\;(f\in\L^p(X),g\in\L^q(X))\]
    は等長同型$L^p(X)\simeq (L^q(X))^*$を引き起こす.
\end{theorem}
\begin{remark}
    $p\in(1,\infty)$の場合は$X$が$\sigma$-コンパクトでない場合にも一般化出来る.
\end{remark}

\begin{example}
    $(l^1)^*=l^\infty$である.なお,$l^1=L^1(\N,2^\N,\mu)$で,$\mu$は数え上げ測度である.
\end{example}

\section{積分の積}

\begin{tcolorbox}[colframe=ForestGreen, colback=ForestGreen!10!white,breakable,colbacktitle=ForestGreen!40!white,coltitle=black,fonttitle=\bfseries\sffamily,
title=]
    体$\R$上の各点積を,関数空間上に持ち上げることが出来るのは周知の事実で,これがどのような構造を引き起こすかを考える.
\end{tcolorbox}

\subsection{関数の積の定義と基本性質}

\begin{definition}[product]
    2つの関数$f:X\to\R,g:Y\to\R$について,\textbf{積}$f\otimes g:X\times Y\to\R$を$(f\otimes g)(x,y)=f(x)g(y)$で定める.
\end{definition}

\begin{lemma}\mbox{}
    \begin{enumerate}
        \item $X,Y$を局所コンパクトハウスドルフ空間とする.積空間$X\times Y$も局所コンパクトハウスドルフである.
        \item $X$と$Y$がいずれも$\sigma$-コンパクトであることと,$X\times Y$が$\sigma$-コンパクトであることは同値.
        \item $f\in C_c(X),g\in C_c(Y)\Rightarrow f\otimes g\in C_c(X\times Y)$.
        \item 同様に,$C_0(X)\otimes C_0(Y)\subset C_0(X\times Y)$かつ$C_b(X)\otimes C_b(Y)\subset C_b(X\times Y)$.
    \end{enumerate}
\end{lemma}

\begin{lemma}
    関数$f:X\times Y\to\R$と点$y\in Y$について,
    \begin{enumerate}
        \item $f\in C_c(X\times Y)\Rightarrow f(-,y)\in C_c(X)$.
        \item $f\in C_c(X\times Y)^m\Rightarrow f(-,y)\in C_c(X)^m$.
        \item $f\in\B(X\times Y)\Rightarrow f(-,y)\in\B(X)$.
    \end{enumerate}
\end{lemma}

\subsection{積分の積とFubiniの定理}

\begin{tcolorbox}[colframe=ForestGreen, colback=ForestGreen!10!white,breakable,colbacktitle=ForestGreen!40!white,coltitle=black,fonttitle=\bfseries\sffamily,
title=]
    Fubiniの定理は,積分の積を先に定義すると極めて見通しが良い.
    しかし,逐次積分が計算可能かどうかの判断の前に,積積分についての可積分性の判定が必要であるのが実用性にかけるが,$\sigma$-コンパクトの場合は抜け道がある.
\end{tcolorbox}

\begin{proposition}[product integral]
    $\int_x,\int_y$をそれぞれ局所コンパクトハウスドルフ空間$X,Y$上のRadon積分とする.
    このとき,次の条件を満たすRadon積分$\int_x\otimes\int_y:X\times Y\to\R$がただ一つ存在する:
    \[\forall_{f\in C_c(X),g\in C_c(Y)}\quad\int_x\otimes\int_yf\otimes g=\paren{\int_xf}\paren{\int_yg}.\]
\end{proposition}

\begin{lemma}
    $h\in C_c(X\times Y)$ならば,$x\mapsto\int_yh(x,-)\in C_c(X)$で,$\int_x\int_yh(-,-)=\int_x\otimes\int_yh$.
\end{lemma}

\begin{lemma}
    $h\in C_c(X\times Y)^m$ならば,$x\mapsto\int_y^*h(x,-)\in C_c(X)^m$である.
    また,$h\in\L^1(X\times Y)$ならば,$f\in\L^1(X)$かつ
    \[\int_xf=\int_x\int_y^*h(-,-)=\int_x\otimes\int_yh.\]
\end{lemma}

\begin{theorem}[Fubini]
    $\int_x,\int_y$をそれぞれ局所コンパクトハウスドルフ空間$X,Y$上のRadon積分とする.
    $h\in\B(X\times Y)$で,積$\int_x\otimes\int_y$に関して可積分ならば,Borel関数について$y\mapsto h(x,y)\in\L^1(Y)\;x\text{-}\ae$かつ殆ど至る所定義された関数について$x\mapsto\int_yh(x,-)\in\L^1(X)$かつ
    \[\int_x\int_yh(-,-)=\int_x\otimes\int_yh.\]
\end{theorem}
\begin{remarks}
    特に,$h\in\L^1(X\times Y)$ならば,次の逐次積分は存在しかつ等しい:\[\int_x\int_yh(-,-)=\int_x\otimes\int_yh=\int_y\int_xh(-,-).\]
\end{remarks}

\begin{corollary}[Tonelli]
    Borel関数$h\in\B(X\times Y)$はある$\sigma$-コンパクト集合の外では消えているとする.
    Borel関数が$y\mapsto\abs{h(x,y)}\in\L^1(Y)\;x\text{-}\ae$で,殆ど至る所定義された関数が$x\mapsto\int_y\abs{h(x,-)}\in\L^1(X)$を満たすならば,$h\in\L^1(X\times Y)$で,
    \[\int_x\int_yh(-,-)=\int_x\otimes\int_yh=\int_y\int_xh(-,-).\]
\end{corollary}

\subsection{位相群上の積分}

\begin{tcolorbox}[colframe=ForestGreen, colback=ForestGreen!10!white,breakable,colbacktitle=ForestGreen!40!white,coltitle=black,fonttitle=\bfseries\sffamily,
title=]
    $G$の2つの演算を連続にするような局所コンパクトハウスドルフ位相を備えた群$G$を考える.
\end{tcolorbox}

\begin{definition}[translated function, Haar integarl]
    位相群$G$上の関数$f:G\to\bF(=\R,\C)$について,
    \begin{enumerate}
        \item 左移動$G\times\Map(G,\bF)\to\Map(G,\bF)$を${}_xf(y):=f(x^{-1}y)$,右移動を${}^xf=f(yx)$で定める.このとき,${}_{xy}f={}_x({}_yf)$かつ${}^{xy}f={}^x({}^yf)$が成り立つ.
        \item $G$上の左\textbf{Haar積分}とは,零でないRadon積分であって,左移動不変であるものをいう:$\forall_{x\in G}\;\forall_{f\in C_c(G)}\;\int{}_xf=\int f$.
        \item このとき,移動不変性は実際は任意の$\L^1(G)$について成り立つ.特に,任意の$x\in G$とBorel集合$A\in\B_G$について,$\int[xA]=\int[A]$.
        \item 任意の非空な開集合$A$について,$\int[A]>0$である.
    \end{enumerate}
\end{definition}

\begin{lemma}
    $f\in C_0(G)$ならば,$x\mapsto {}_xf$と$x\mapsto{}^xf$によって定まる2つの写像$G\to C_0(G)$はいずれも一様連続である.
\end{lemma}

\begin{lemma}
    $G$の任意のRadon積分$\int:C_c(G)\to\R$と実数$p\in[1,\infty)$について,任意の$f\in C_c(G)$について,$x\mapsto{}_xf,x\mapsto{}^xf$によって定まる写像$G\to\L^p(G)$は一様連続である.
\end{lemma}

\begin{theorem}[Haar]
    局所コンパクトな位相群$G$において,左Haar積分は正のスカラー因数の別を除いて一意に定まる.
\end{theorem}
\begin{remark}
    全く対称的な議論により,右Haar積分も一意的に存在するが,左Haar積分と一致するとは限らない.
    \[G:=\Brace{\begin{pmatrix}a&b\\0&1\end{pmatrix}\in M_2(\R)\;\middle|\;a>0,b\in\R}\]
    と定めると,これは$\R$のaffine同型群で,それぞれの行列は変換$x(t)=at+b$に対応する.
    \[\int_lf=\iint f(a,b)a^{-2}dadb,\quad\int_rf=\iint f(a,b)a^{-1}dadb\]
    と定める.ただし,右辺は$\R_+\times\R$上のLebesgue積分で,$f\in C_c(G)$とした.
\end{remark}

\subsection{調和解析}

\begin{notation}
    位相群$G$上のHaar積分を一つ固定し,これを$\int f(x)dx$で表す.これは,調和解析特有の頻繁な変数変換を視覚的に表すためである.
\end{notation}

\begin{definition}[modular function, unimodular]
    任意の$x\in G$について,Radon積分$C_c(G)\to\R;f\mapsto\int f(yx)dy$は左不変だから,一意な実数$\Delta(x)>0$であって次を満たすものが定まる:$\forall_{f\in\L^1(G)}\;\Delta(x)\int f(yx)dy=\int f(y)dy$.
    \begin{enumerate}
        \item 関数$\Delta:G\to(0,\infty)$を\textbf{モジュラー関数}または\textbf{母数}という.
        \item $\Delta=1$であるとき,群$G$を\textbf{ユニモジュラー}または\textbf{単模}という.これは左右のHaar積分が一致することに同値.
    したがってモジュラー関数は左右のズレを測っているとも考えられる.
    \item $d(yx)=\Delta(x)dy$が成り立つ.
    \end{enumerate}
\end{definition}
\begin{remark}
    有限次元の場合,行列式が単元になるものをユニモジュラー行列という.
\end{remark}

\begin{proposition}
    モジュラー関数$\Delta:G\to(0,\infty)$は,$(0,\infty)=\exp(\R)$を乗法群とみると,連続な群順同型を定める.
    $G$がAbelである,または離散である,またはコンパクトであるならば,$G$は単模である.
\end{proposition}

\subsection{対合代数}

\begin{lemma}
    \[\forall_{f\in C_c(G)}\quad\int f(x^{-1})\Delta(x)^{-1}dx=\int f(x)dx.\]
\end{lemma}

\begin{discussion}
    $\check{f}(x):=f(x^{-1})$とすると,$f\mapsto\int\check{f}(x)dx$は右不変である.
    この間の関係は
    \[\int\check{f}(x)dx=\int f(x^{-1})\Delta(x)\Delta(x)^{-1}dx=\int f(x)\Delta(x)^{-1}dx.\]
    さらに,$f^*(x):=\o{f(x^{-1})}\Delta(x)^{-1}$と定めると,
    \[\forall_{f\in C_c(G)}\quad\norm{f^*}_1=\int\abs{f(x^{-1})}\Delta(x)^{-1}dx=\int\abs{f(x)}dx=\norm{f}\]
    で,結局${}^*$はBanach空間$L^1(G)$上に等長な対合を定める.
\end{discussion}

\begin{proposition}
    $\forall_{1\le p<\infty}$について,$f\in\L^p(G)$ならば,$x\mapsto{}_xf,x\mapsto{}^xf$は一様連続写像$G\to L^p(G)$を定める.
\end{proposition}

\begin{proposition}
    $\forall_{1\le p<\infty}$について,$f\in\L^1(G),g\in\L^p(G)$をそれぞれBorel関数とすると,$y\mapsto f(y)g(y^{-1}x)\in\L^1(G)\;x\text{-}\ae$で,殆ど至る所定義された関数について$x\mapsto\int f(y)g(y^{-1}x)dy\in\L^p(G)$で,
    \[\Norm{\int f(y)g(y^{-1}\cdot)dy}_p\le\norm{f}_1\norm{g}_p.\]
\end{proposition}

\begin{theorem}
    局所コンパクト群$G$とそのHaar積分$\int$について,空間$L^1(G)$は等長対合を備えたBanach代数である.ただし積と対合は次のように定める:
    \[(f\times g)(x)=\int f(y)g(y^{-1}x)dy,\quad f^*(x)=\o{f(x^{-1})}\Delta(x)^{-1}.\]
\end{theorem}

\subsection{対合代数の表現}

\begin{proposition}
    等長同型$\Phi:L^1(X)\mono M(X)$は,$L^1(G)$をある$M(G)$の$*$-不変な閉イデアル上に移す対合代数の$*$-同型である.
\end{proposition}

\chapter{Sobolev空間}

\begin{quotation}
    ノルムには関数の滑らかさに関する情報を取り入れることが出来る\ref{exp-Banach-space-with-derivative-norm}.
    すると,$C_c(X)$よりさらに小さいクラスを作り出す一つの方法となる.
    超関数とSobolev空間は,可微分関数のクラスを一般化して,古典的な解空間を拡張した.
\end{quotation}

\section{分布}

\begin{tcolorbox}[colframe=ForestGreen, colback=ForestGreen!10!white,breakable,colbacktitle=ForestGreen!40!white,coltitle=black,fonttitle=\bfseries\sffamily,
title=]
    Rieszの表現の考え方では,$C_c(X;\C)$の双対が有限Radon測度$\RM(X)$なのであった.
    そこで,$C_c(X;\C)$よりもさらに小さい関数のクラスを「試験関数」として定め,
    この双対を取ることで,種々の超関数が構成できる,というのがSchwartzのアイデアである.
    試験関数ありきの定義であるので,分布の空間には$w^*$-位相を考える.
\end{tcolorbox}

\subsection{アイデア}

簡単のため,開区間$I=(a,b)\osub\R\;(a,b\in\oR)$で考える.

\begin{discussion}[Rieszの表現流の発想の転換]
    $f\in\L^1_\loc(I)$を,$I\to\R$と捉えるのではなく,試験関数上の線型汎関数
    $C_c^\infty(I)\to\R;\varphi\mapsto\int f\varphi$とみる.
    するとクラス$C_c^\infty(I)$は十分に大きいので,この対応は単射となる:\ref{thm-uniqueness-of-weak-derivative}.
\end{discussion}

\begin{discussion}[微分の導入]
    するとこの枠組みを用いて,部分積分の公式から微分を一般化出来る.
    $\varphi\in C_c^\infty(I)$について,
    \[\forall_{f\in C^1(I)}\;\int f'\varphi=\Square{f\varphi}^b_a-\int_If\varphi'=-\int f\varphi'.\]
    $\supp(\varphi)\subsetneq I$より,
    $f\varphi(a)=f\varphi(b)=0$に注意.
    一般に,$f\in C^\infty(I)$ならば,部分積分の公式を繰り返し適用することより,
    \[\forall_{\varphi\in C_c^\infty(I)}\;\forall_{k\in\N}\;\int f^{(k)}\varphi=(-1)^k\int f\varphi.\]
    この右辺を通じて,一般の局所可積分関数に微分を導入することが出来る.
    これを\textbf{超関数の微分}または\textbf{弱微分}という.
\end{discussion}

弱微分は微分に持ってほしい性質の殆どを備えており,またFourier解析と相性が良い.

\subsection{基礎となるFrechet空間}

\begin{tcolorbox}[colframe=ForestGreen, colback=ForestGreen!10!white,breakable,colbacktitle=ForestGreen!40!white,coltitle=black,fonttitle=\bfseries\sffamily,
title=]
    ここに,位相線形空間論をまとめる\cite{Rudin}.
\end{tcolorbox}

\begin{definition}[Heine-Borel property]
    位相線形空間$X$がHeine-Borel性を持つとは,任意の有界閉集合がコンパクトであることをいう.
\end{definition}

\begin{example}
    Hardy空間$H(\Om)$と非空な内部を持つコンパクト集合$K$内に台を持つ滑らかな関数の空間$C^\infty_K=C^\infty_c(K)$とは無限次元Frechet空間でHeine-Borel性を持つ.
    したがってこれらは局所有界でなく,したがってノルム付け可能でない.
\end{example}

\begin{theorem}\label{thm-character-of-TVS}
    $X$を位相線形空間とする.
    \begin{enumerate}
        \item $X$が局所有界ならば$X$は第2可算である.
        \item $X$が距離化可能であることと,第2可算であることとは同値.
        \item $X$がノルム付け可能であることと,局所凸で局所有界であることと,$0$に有界凸近傍が存在することは同値.
        \item $X$が有限次元であることと,$X$が局所コンパクトであることとは同値.
        \item 局所有界な位相線形空間$X$について,Heine-Borel性を持つことと有限次元であることとは同値.
    \end{enumerate}
\end{theorem}

\begin{theorem}
    位相線形空間$X$上の零でない線型汎関数$\Lambda:X\to\K$について,次の4条件は同値:
    \begin{enumerate}
        \item $\Lambda$は連続.
        \item $N(\Lambda):=\Lambda^{-1}(0)$は閉である.
        \item $N(\Lambda)$は$X$上稠密でない.
        \item $\Lambda$は$0$のある近傍$V$上で有界である.
    \end{enumerate}
\end{theorem}

\begin{theorem}
    $X,Y$を位相線形空間,$\Lambda:X\to Y$を線型写像とする.
    (1)$\Rightarrow$(2)$\Rightarrow$(3)が成り立つ.
    $X$が距離化可能であるとき,(3)$\Rightarrow$(4)$\Rightarrow$(1)も成り立ち,4条件が同値になる.
    \begin{enumerate}
        \item $\Lambda$は連続である.
        \item $\Lambda$は有界である.
        \item $x_n\to 0$ならば$\{\Lambda x_n\}_{x\in\N}$は有界である.
        \item $x_n\to0$ならば$\Lambda x_n\to0$.
    \end{enumerate}
    特にFrechet空間は距離化可能な完備局所凸線型空間であるから,上の4条件は同値になる.
\end{theorem}

\begin{example}[滑らかな関数のFrechet空間とその部分Frechet空間]
    $K\subset\R^n$をコンパクトとする.
    \[D_K:=C_c^\infty(K)=\Brace{f\in C^\infty(\R^n)\mid \supp f\subset K}\]
    とすると,これは半ノルム列$(p_N)_{N\in\N}$に関するFrechet空間$C^\infty(\Om)\;(K\subset\Om)$の閉部分空間かつFrechet空間である.
    ただし,$\R^n$の$\sigma$-コンパクト性から$\Om$に届くコンパクト集合の増大列$(K_i)$について,
    \[p_N(f):=\max\Brace{\abs{D^\al f(x)}\in\R_+\mid x\in K_N,\abs{\al}\le N}\]
    と定める.
\end{example}

\subsection{試験関数の空間}

\begin{definition}
    $\Om\osub\R^n$とする.
    $D(\Om):=\bigcup_{K\compsub\Om}D_K$で定まるノルム空間を試験関数の空間という.
    $D_K=C^\infty_c(K)$であったから,これは$D(\Om)=C^\infty_c(\Om)$に等しい.
    \[\norm{\varphi}_N=\max\Brace{\abs{D^\al\varphi(x)}\in\R_+\mid x\in\Om,\abs{\al}\le N}\]
    はノルムを定め,各$D_K$への制限は半ノルムを定め,$p_N$と同値な位相を$D_K$に定める.
    $\norm{-}_N$は距離化可能であるが,完備ではない.
\end{definition}

\begin{lemma}
    次の位相$\tau$は$D(\Om)$を完備にするが,距離化可能でない.
    \begin{enumerate}
        \item $\tau_K\subset P(D_K)$をFrechet空間$D_K$の位相とする.
        \item $\beta:=\Brace{W\subset D(\Om)\mid W\text{は絶対凸集合で}\forall_{K\compsub\Om}\;D_K\cap W\in\tau_K}$.
        \item $W\in\beta$の$\varphi\in D(\Om)$-平行移動$\varphi+W$で表せる集合の合併で表せる集合の全体を$\tau$とする.
    \end{enumerate}
\end{lemma}

\begin{theorem}\mbox{}
    \begin{enumerate}
        \item $\tau$は$D(\Om)$に位相を定め,$\beta$は局所基底となる.
        \item $(D(\Om),\tau)$は局所凸空間である.
    \end{enumerate}
\end{theorem}

\begin{theorem}\mbox{}
    \begin{enumerate}
        \item 絶対凸集合$V\subset D(\Om)$について,開であることと$V\in\beta$であることとは同値.
        \item $\tau_K$は$\tau$の相対位相に等しい.
        \item $D(\Om)$の有界集合$E$について,あるコンパクト集合$K\subset\Om$が存在して$E\subset D_K$を満たし,また,正整数列$(M_N)_{N\in\N}$が存在して$\forall_{\varphi\in E}\;\forall_{N\in\N}\;\norm{\varphi}_N\le M_N$が成り立つ.
        \item $D(\Om)$はHeine-Borel性を持つ.
        \item $\{\varphi_i\}\subset D(\Om)$がCauchy列ならば,あるコンパクト集合$K\subset\Om$が存在して$\{\varphi_i\}\subset D_K$でもあり,$\forall_{N\in\N}\;\lim_{i,j\to\infty}\norm{\varphi_i-\varphi_j}_N=0$.
        \item $\varphi_i\to0$ならば,あるコンパクト集合$K\subset\Om$が存在して$\cup_{i\in\N}\supp\varphi_i\subset K$を満たし,$\forall_{\al\in\N^n}\;D^\al\varphi_i\to0$は一様収束する.
        \item $D(\Om)$は完備である.
    \end{enumerate}
\end{theorem}

\begin{theorem}
    $Y$を局所凸空間,$\Lambda:D(\Om)\to Y$を線型作用素とする.次の4条件は同値:
    \begin{enumerate}
        \item $\Lambda$は連続.
        \item $\Lambda$は有界.
        \item $\varphi_i\to0$ならば,$\Lambda\varphi_i\to0$.
        \item 任意の$K\compsub\Om$について,制限$\Lambda|_{D_K}$は連続.
    \end{enumerate}
\end{theorem}

\begin{definition}[distribution]
    $D(\Om)$上の連続な線型汎関数を\textbf{$\Om$上の分布}という.
    分布の全体を$\D'(\Om)$で表す.
\end{definition}

\begin{theorem}
    $\Lambda:D(\Om)\to\C$を線型汎関数とする.次の2条件は同値:
    \begin{enumerate}
        \item $\Lambda\in\D'(\Om)$.
        \item 任意のコンパクト集合$K\subset\Om$に対して自然数$N\in\N$と定数$C\in\R$が存在して,$\forall_{\varphi\in D_K}\;\abs{\Lambda\varphi}\le C\norm{\varphi}_N$を満たす.
    \end{enumerate}
\end{theorem}

\subsection{分布による解析学}

\begin{tcolorbox}[colframe=ForestGreen, colback=ForestGreen!10!white,breakable,colbacktitle=ForestGreen!40!white,coltitle=black,fonttitle=\bfseries\sffamily,
title=]
    関数は分布を定める.
    分布の微分を部分積分を通じて定める.
    その後,関数の弱微分を,その関数が定める分布の微分として定める.
\end{tcolorbox}

\begin{example}[関数は分布]
    $f\in L^1_\loc(\Om;\C)$は
    \[\Lambda_f(\varphi):=\int_\Om\varphi(x)f(x)dx\]
    によって分布$\Lambda_f\in\D'(\Om)$を定める.
\end{example}

\begin{example}[測度は分布]
    $\mu$を$\Om$上の複素Borel測度,または,局所有界な正測度$\Im\mu\subset[0,\infty]$とする.
    \[\Lambda_\mu(\varphi)=\int_\Om\varphi d\mu\]
    によって分布$\Lambda_\mu\in\D'(\Om)$を定める.
\end{example}

\begin{definition}[differentiation of distribution]
    $\al\in\N^n,\Lambda\in\D'(\Om),\Om\osub\R^n$とする.
    部分積分の公式$(D^\al\Lambda)(\varphi):=(-1)^{\abs{\al}}\Lambda(D^\al\varphi)$によって$D^\al\Lambda:\D(\Om)\to\K$を定めると,これは有界である$D^\al\Lambda\in\D'(\Om)$.
\end{definition}
\begin{remark}
    $D^\al,D^\beta$が$C^\infty(\Om)$上で可換であるから,分布に対しても$D^\al D^\beta\Lambda=D^{\al+\beta}\Lambda=D^\beta D^\al\Lambda$が成り立つ.
\end{remark}

\begin{definition}[distribution derivative of functions]
    $f\in L^1_\loc(\Om)$の\textbf{$\al$-次分布微分}または\textbf{$\al$-弱微分}とは,分布$D^\al\Lambda_f$をいう.
\end{definition}

\begin{lemma}\mbox{}
    \begin{enumerate}
        \item $f\in C^{\abs{\al}}$ならば,$D^\al\Lambda_f=\Lambda_{D^\al f}$である.
        \item この可換性は一般には成り立たない.
    \end{enumerate}
\end{lemma}

\begin{example}
    $\Om\subset\R$を線分とする.
    $f:\Om\to\R$を左連続で有界変動な関数とすると,$Df\in L^1$が成り立つ.
    このとき,$\mu([a,b)):=f(b)-f(a)$によって測度を定めると,$D\Lambda_f=\Lambda_\mu$が成り立つ:
    \begin{align*}
        D\Lambda_f(\varphi)&=-\Lambda_f(D\varphi)\\
        &=-\int_\Om f(x)\cdot D\varphi(x)dx\\
        &=\int_\Om Df(x)\cdot\varphi(x)dx=\Lambda_\mu(\varphi).
    \end{align*}
    したがって,$D\Lambda_f=\Lambda_{Df}$が成り立つことは,$\Lambda_{Df}=\Lambda_\mu$に同値.これは$d\mu=Df(x)dx$と同値だから,$f$が絶対連続であることに同値\ref{thm-absolutely-continuous}.
\end{example}

\begin{theorem}[絶対連続性の復習]\label{thm-absolutely-continuous}
    $f:[a,b]\to\R$について,次の条件は同値.
    \begin{enumerate}
        \item $\forall_{\ep>0}\;\exists_{\delta>0}\;\forall_{n\in\N}\;\forall_{\sqcup_{i\in[n]}\{(a_i,b_i)\}\subset[a,b]}\;\sum^n_{i=1}(b_i-a_i)<\delta\Rightarrow\sum^n_{i=1}\abs{f(b_i)-f(a_i)}<\ep$.
        \item $g\in L^1([a,b])$が存在して,$\forall_{x\in[a,b]}\;f(x)=f(a)+\int_{y=a}^xg(y)dy$.特に,$f$は$L^1$-有界な導関数を殆ど至る所で持つ.
        \item $f$は$[a,b]$上一様連続かつ有界変動かつ$[a,b]$の零集合の$f$に関する順像は零である.
        \item $f$に関するStieltjes測度$df$はLebesgue測度$dx$に関して絶対連続である:$df=gdx$.
    \end{enumerate}
\end{theorem}

\subsection{用語のまとめ}

\begin{discussion}[distributional density]
    密度(体積形式$\dvol$)を備えた可微分多様体$X$において,積分なる連続線型汎関数
    \[\xymatrix@R-2pc{
        C^\infty_c(X)\ar[r]&\R\\
        \rotatebox[origin=c]{90}{$\in$}&\rotatebox[origin=c]{90}{$\in$}\\
        b\ar@{|->}[r]&\int_{x\in X}b(x)\dvol(x)
    }\]
    が定まる.$C_c^\infty(X)$の元は\textbf{隆起関数}または\textbf{試験関数}と呼ばれることに注意.なお,$C_c^\infty(X)$は局所凸かつ完備で,LF-空間=Frechet空間の帰納極限である(それ自身がFrechet空間であるわけではない).
    一方で,$C_c^\infty(X)$の元のすべてが積分の形で表現できるわけではない.
    例えばデルタ分布$\delta_{x_0}:b\mapsto b(x_0)$である.
    これは隆起関数の積分として表せないが,その列(台が$x_0$に収束するような)の積分の極限と捉えられる.
    そこで,$C_c^\infty(X)^*$の元を\textbf{分布密度}という.
\end{discussion}
\begin{remarks}
    こうして,Sobolevが考え始めた「一般化関数」を,特定の関数クラス(今回だと$C_c^\infty(X)$)上の連続線型汎関数として定義することが,Schwartz超関数のアイデアであり,「分布」と名付けた.
    この観点は確率分布に象徴され,
    \[\brac{P,\varphi}=\int_\R\varphi dP\]
    とすると,テスト関数$\varphi$に関して線型かつ連続である.
    そう,確率分布なる概念には,関数を一般化する力がある!
\end{remarks}

\begin{definition}[compactly supported distribution, tempered distribution]\mbox{}
    \begin{enumerate}
        \item テスト関数を$C^\infty(X)$に広げると,\textbf{コンパクト台を持つ分布}という.この空間を$\E'(X)$で表す.
        実際に,$C^\infty(X)^*$の元は,分布として,コンパクト台を持つ:$\E'(\R^n)=(\E(\R^n))^*$.
        \item テスト関数を$\cS(X)$にすると,この空間を\textbf{緩増加超函数}という.この空間を$\cS'(\R)$で表す.
    \end{enumerate}
\end{definition}

\begin{example}
    $f\in L^p(\R^n)$に対して,$q$を共役指数として,$g\in L^q(\R^n)$に対する測度$f\dvol$に関する積分
    \[\xymatrix@R-2pc{
        L^q(\R^n)\ar[r]&\R\\
        \rotatebox[origin=c]{90}{$\in$}&\rotatebox[origin=c]{90}{$\in$}\\
        g\ar@{|->}[r]&\int_{x\in\R^n}g(x)f(x)\dvol(x)
    }\]
    は緩増加関数である.
\end{example}

\begin{theorem}
    包含$\E'(\R^n)\mono\S'(\R)$が存在する.
\end{theorem}

\begin{discussion}[双対ペアと微分]
    Bochner積分論\ref{subsection-Pettis-integral}と同様に,双対ペアは微分の定義を一意に示唆する.
    部分積分の公式
    \[\int_\R f'\varphi dx=-\int_\R f\varphi'dx\]
    は,$\brac{S',\varphi}=-\brac{S,\varphi'}$を意味している.これによると,任意のSchwartz超関数は無限階微分可能である.
\end{discussion}

\begin{discussion}[その他の操作と演算]
    他にも,分布の引き戻しや分布の積などの演算も,滑らかな関数の場合と整合的に,部分的に定義できるが,大域的には定義できない.
\end{discussion}

\subsection{定義1}

\begin{definition}[distribution on $X$]
    \textbf{$X$上の分布}とは,連続線型汎関数$C_c^\infty(X)\to\R$をいう.
    この空間を$\D'(X)$で表す.
    評価写像が定める双線型写像
    \[\xymatrix@R-2pc{
        \D'(X)\times C^\infty_c(X)\ar[r]&\R\\
        \rotatebox[origin=c]{90}{$\in$}&\rotatebox[origin=c]{90}{$\in$}\\
        (S,\phi)\ar@{|->}[r]&\brac{S,\phi}:=S(\phi)
    }\]
    は自然にペアリング$(\D'(X),C_c^\infty(X))$を与える.
    $\D'(X)$には,この対による$w^*$-位相を入れて考える.すなわち,$\brac{-,\phi}:\D'(U)\to\R$を連続にする最弱の位相を考える.
    なお,$C^\infty_c(X)$は回帰的であるから,これは弱位相でもある.
\end{definition}

\begin{example}
    任意の局所可積分な関数$f:X\to\R$は,任意の$\phi\in C_c^\infty(X)$に対して
    $\brac{f,\phi}=\int_Xf(x)\phi(x)dx$が定まるから,連続線型写像$C_c^\infty(X)\to\R$を誘導する.これが定める
    包含$C_c^\infty(X)\mono\D'(X)$は稠密である.
\end{example}

\begin{example}[Riesz-Markovの表現定理の消息]
    連続な線型汎関数$\mu:C_c(U)\to\R$とは,$U$上の符号付き測度とみなせるのであった(Riesz-Markov\ref{prop-Riesz-Markov-3}).
    これを$C_c^\infty(U)$上に制限することで,分布を定める.
\end{example}

\begin{example}[累積分布関数の消息]
    $U=\R$上の有界な単調増加関数$\al$を考える.このとき,Riemann-Stieltjes積分$\int_\R f(x)d\al(x)$が定まるが,これは暗黙のうちに測度,すなわち分布$\al$を定めている.
    特に,$H(x)=1\;(x>0),H(x)=0\;(x\le0)$なるHeaviside関数が定めるRiemann-Stieltjes積分を通じて,
    \[\brac{f,dH}=\int_\R f(x)dH(x)=f(0).\]
    これを理工学分野では$dH(x)=\delta_0(x)dx$で表す.
\end{example}

\subsection{軟化子のさらなる補題}

\begin{tcolorbox}[colframe=ForestGreen, colback=ForestGreen!10!white,breakable,colbacktitle=ForestGreen!40!white,coltitle=black,fonttitle=\bfseries\sffamily,
title=]
    $J_\ep:L^1_\loc(\R^n)\to C^\infty(\R^n)$の$\ep\to0$極限では,殆ど至る所恒等写像になる.
\end{tcolorbox}

\begin{definition}[Lebesgue point]\mbox{}
    \begin{enumerate}
        \item $f\in L^1_\loc(\R^n)$について,$x\in\R^n$が$f$の\textbf{Lebesgue点}であるとは,次を満たすことをいう:
        \[\lim_{r\to0^+}\frac{1}{\abs{B(x,r)}}\int_{B(x,r)}\abs{f(y)-f(x)}dy=0.\]
        \item $f\in L^1(\R^n)$の$x$における微分を$\lim_{B\to x}\frac{1}{\abs{B}}\int_Bfdm$と定めると,Lebesgue点では,$f$の$x$でのLebesgue積分の微分が$f$に等しい.これは三角不等式による.
    \end{enumerate}
\end{definition}
\begin{remarks}
    この量は$f$の$x$の近傍における振動の平均を測っている.
\end{remarks}

\begin{theorem}[Lebesgue differentiation theorem 1910]
    任意の$f\in L^1_\loc(\R^n)$について,
    $\R^n$の殆ど至る所Lebesgue点である.
\end{theorem}

\begin{theorem}
    $f\in L^1_\loc(\R^n)$について,
    \begin{enumerate}
        \item $x$を$f$のLebesgue点とすると,$\lim_{\ep\to0}J_\ep f(x)=f(x)$.
        \item $f$が連続ならば,$J_\ep f\in C^\infty(\R^n)$は$f$に広義一様収束する.
    \end{enumerate}
\end{theorem}

\subsection{定義2}

\begin{definition}[distribution, order]
    線型汎関数$T:C^\infty_c(X)\to\R$が\textbf{分布}であるとは,任意のコンパクト集合$K\in\cC$について,定数$C\in\R,N\in\N$が存在して,$\forall_{\varphi\in C^\infty(K)}\;\abs{T(\varphi)}\le C\sup_{x\in K}\sum_{\abs{\al}\le N}\abs{D^\al\varphi(x)}=C\norm{\varphi}_{K,N}$を満たすものをいう.
    $N$が$K\in\cC$に依らず一様に取れるとき,その最小な$N$を用いて,分布は\textbf{$N$次}であるという.
\end{definition}

\begin{theorem}
    $T$を分布とすると,次の2条件は同値.
    \begin{enumerate}
        \item $T$は$0$次の分布である.
        \item $T$はRadon測度である.
    \end{enumerate}
\end{theorem}

\begin{example}
    Radon測度は,$\delta_x$を含め,$0$次の分布である.
    実は$0$次の
\end{example}

\section{微分をFourier変換で緩める}

\begin{tcolorbox}[colframe=ForestGreen, colback=ForestGreen!10!white,breakable,colbacktitle=ForestGreen!40!white,coltitle=black,fonttitle=\bfseries\sffamily,
title=]
    $C^{m,p}(\Om)$が完備であるためには,微分できる関数の範囲を広げる必要がある.
    そこで,Fourier変換を用いて微分を緩めることを考える.
    こうして$L^2$-微分の概念に到達するが,この考え方はFourier変換と関係がない.
    
    そこで,次の節で$u\mapsto u_\al$の対応をFourier変換の言葉を使わずに一意に定まることを示し,
    $m$階$L^p$-微分可能な関数の空間$W^{m,p}(\Om)$の理論を立てる.
\end{tcolorbox}

\begin{notation}
    $\Om\osub\R^n$上の$C^m$-級関数であって,$m$階までの全ての偏導関数が,連続であるだけでなく$L^p$-可積分でもあるような関数の全体を$C^{m,p}(\Om)$で表す.
    \[\norm{u}_{C^{m,p}(\Om)}:=\paren{\sum_{\abs{\al}\le m}\norm{D^\al u}^p_{L^p(\Om)}}^{1/p}\]
    と定めると,これはノルムである.$C_b^m(\Om)$と異なり,完備にはならない.
\end{notation}

\begin{discussion}
    Fourier変換$\F:L^2(\R^n)\to L^2(\R^n)$は,複素Frechet空間$\S(\R^n)\subset C^\infty(\R^n),\S(\R^n)\subset L^p(\R^n)\;(p\in[1,\infty])$上に線型同型を定めるのであった.
    $u\in C^m_c(\R^n)\subset C^{m,2}(\R^n)$について,合成関数の微分法則より,
    \[\forall_{\abs{\al}\le m}\;(FD^\al u)(\xi)=(i\xi)^\al(Fu)(\xi)\]
    が成り立つ.故にParsevalの等式より
    \[\norm{u}_{C^{m,2}(\R^n)}=\paren{\sum_{\abs{\al}\le m}\norm{\xi^\al Fu}^2_{L^2(\R^n)}}^{1/2}=\paren{\int_{\R^n}\abs{(Fu)(\xi)}^2\paren{\sum_{\abs{\al}\le m}\abs{\xi^\al}^2}d\xi}^{1/2}\]
    が成り立つ.
    こうして,Fourier変換のことばで,$C^{m,p}(\Om)$のノルムに別の表示を得た.
    実は,右辺が有限になる関数$u$の全体は,同じノルムについてBanachになる.
\end{discussion}

\begin{proposition}
    $H^m(\R^n)\subset L^2(\R^n)$を
    \[H^m(\R^n):=\Brace{u\in L^2(\R^n)\mid\forall_{\abs{\al}\le m}\;\xi^\al(Fu)(\xi)\in L^2(\R^n)}\]
    と定める.
    \begin{enumerate}
        \item $\nnorm{u}_m:=\paren{\int_{\R^n}\abs{(Fu)(\xi)}^2\paren{\sum_{\abs{\al}\le m}\abs{\xi^\al}^2}d\xi}^{1/2}$はノルムを定め,これについて完備になる.
        \item $H^m(\R^n)$は内積$(u,v)_m:=\int_{\R^n}\paren{\sum_{\abs{\al}\le m}\abs{\xi^m}^2}(Fu)(\xi)\o{(Fv)(\xi)}d\xi$についてHilbert空間をなす.
    \end{enumerate}
\end{proposition}

\begin{example}[Sobolev空間の微分不可能な元]
    $u(x):=(1-\abs{x})\land0$とすると,$Fu(\xi)=\paren{\frac{2}{\pi}}^{1/2}\frac{1-\cos\xi}{\xi^2}$であるから,$u\in H^1(\R^1)$.
\end{example}

\begin{definition}
    $u\in L^2(\R^n)$が$L^2$の意味で$m$階微分可能であるとは,次を満たすことをいう:
    \[\forall_{\varphi\in C_c^\infty(\R^n)}\quad\int_{\R^n}u(x)D^\al\varphi(x)dx=(-1)^{\abs{\al}}\int_{\R^n}u_\al(x)\varphi(x)dx.\]
\end{definition}
\begin{lemma}
    $u\in H^m(\R^n)$に対し,$u_\al:=F^{-1}((i\xi)^\al Fu)$は$\abs{\al}$階の$L^2$-導関数である.
    実はこの逆も成り立ち,$H^m(\R^n)$は$L^2$の意味で$m$階微分可能な関数全体の空間である:$W^{m,2}(\R^n)=H^m(\R^n)$\ref{thm-Hilbert-Sobolev-space}.
\end{lemma}

\section{軟化作用素}

\subsection{変分法の基本補題}

\begin{tcolorbox}[colframe=ForestGreen, colback=ForestGreen!10!white,breakable,colbacktitle=ForestGreen!40!white,coltitle=black,fonttitle=\bfseries\sffamily,
title=]
    $L^p$の元を$C^\infty$の列で近似するときはいつでも畳み込みを用いる.
\end{tcolorbox}

\begin{lemma}
    次を満たす関数$j\in C^\infty_c(\R^m)$が存在する:
    \begin{enumerate}
        \item $j\ge0$.
        \item $\abs{x}\ge1\Rightarrow j(x)=0$.すなわち,$\supp j\subset B(0,1)$.
        \item $\int_{\R^n}j(x)dx=1$.
    \end{enumerate}
\end{lemma}

\begin{definition}[mollifier]
    補題を満たす関数$j\in C^\infty_c(\R^m)$を一つ取る.
    \begin{enumerate}
        \item $\ep>0$に対して,$j_\ep(x):=\frac{1}{\ep^n}j\paren{\frac{x}{\ep}}$とおくと,再び$j_\ep\in C_c^\infty(\R^n)$と(1),(3)を満たすが,(2)については$\abs{x}\ge\ep\Rightarrow j_\ep(x)=0$という形で満たす.
        \item 畳み込みを取る作用素を$J_\ep:L^1_\loc(\R^n)\to C^\infty(\R^n);u\mapsto J_\ep u=j_\ep*u$で定める.これを\textbf{Friedrichsの軟化子}という.
    \end{enumerate}
\end{definition}
\begin{lemma}
    $J_\ep:L^1_\loc(\R^n)\to C^\infty(\R^n)$について,$u\in L^1_\loc(\R^n)$が$K\subset\R^n$の外で$0$ならば,$J_\ep u$は
    \[K_\ep:=\Brace{x=x_1+x_2\in\R^n\mid x_1\in K,\abs{x_2}\le\ep}\]
    の外において$0$である.
\end{lemma}

\begin{lemma}
    $u\in L^p(\R^n)$ならば$J_\ep\in L^p(\R^n)$であり,
    \begin{enumerate}
        \item $\forall_{p\in[1,\infty]}\;\norm{J_\ep u}_{L^p(\R^n)}\le\norm{u}_{L^p(\R^n)}$.
        \item $\forall_{p\in[1,\infty)}\;\lim_{\ep\searrow0}\norm{J_\ep u-u}_{L^p(\R^n)}=0$.
    \end{enumerate}
\end{lemma}

\begin{theorem}[du Bois-Reymond lemma / 変分法の基本補題]
    任意の$1\le p<\infty$において,$C_c^\infty(\Om)\subset C_c(\Om)$も$L^p(\Om)$上稠密である.
\end{theorem}

\subsection{$L^p$-微分の一意性}

\begin{tcolorbox}[colframe=ForestGreen, colback=ForestGreen!10!white,breakable,colbacktitle=ForestGreen!40!white,coltitle=black,fonttitle=\bfseries\sffamily,
title=]
    微分可能性を積分を使って緩める手法を考えているが,$\varphi\in C_c^\infty(\Om)$はそのための関数クラスとして十分大きい.
    収束の概念を弱めるときと似ている.
\end{tcolorbox}

\begin{theorem}\label{thm-uniqueness-of-weak-derivative}
    $u\in L^1_\loc(\Om)$が
    $\forall_{\varphi\in C_c^\infty(\Om)}\;\int_\Om u(x)\varphi(x)dx=0$
    を満たすならば,$u=0\;\ae$である.
\end{theorem}

\section{弱微分}

\subsection{一般化された導関数}

\begin{definition}[weak derivative]
    $u\in L^1_\loc(\Om)$と$\al\in\N^m$について,$u$の\textbf{$\al$-弱微分}とは,
    \[\forall_{\varphi\in C_c^\infty(\Om)}\;\int_\Om u(x)D^\al\varphi(x)dx=(-1)^{\abs{\al}}\int_\Om u_\al(x)\varphi(x)dx\]
    を満たす$u_\al\in L^1_\loc(\Om)$をいう.
\end{definition}
\begin{remark}
    $u\in C^{\abs{\al}}(\Om)$ならば,$u_\al=D^\al u$である.
    以降,弱微分も$D^\al u$で表す.
    普通の導関数と区別するときは,これを$\partial^\al u$で表す.
\end{remark}

\begin{notation}
    $D_ju:=u_{e_j}$を
    \[\forall_{\varphi\in C_c^\infty(\Om)}\;\int_\Om u(x)\pp{\varphi(x)}{x_j}\varphi(x)dx=-\int_\Om (D_ju(x))\varphi(x)dx\]
    によって定めると,$D^{\al}u=D^{\al_1}_1\cdots D^{\al_n}_nu$と表せる.
\end{notation}

\begin{proposition}[コンパクト台を持つ$C^m$-級関数は滑らかな関数で一様近似可能,]\mbox{}
    \begin{enumerate}
        \item 任意の$\varphi\in C_c^m(\Om)$について,$C_c^\infty(\Om)$の列$(\varphi_n)$であって,任意の$\abs{\al}\le m$について$D^\al\varphi_n$が$D^\al\varphi$に一様収束するものが存在する.
        \item $u\in L^1_\loc(\Om)$に弱微分$D^\al u=u_\al$が存在するとする.このとき,任意の$\varphi\in C^{\abs{\al}}_c(\Om)$についても,関係
        \[\forall_{\varphi\in C_c^{\abs{\al}}(\Om)}\;\int_\Om u(x)D^\al\varphi(x)dx=(-1)^{\abs{\al}}\int_\Om u_\al(x)\varphi(x)dx\]
        を満たす.
    \end{enumerate}
\end{proposition}

\subsection{Sobolev空間}

\begin{tcolorbox}[colframe=ForestGreen, colback=ForestGreen!10!white,breakable,colbacktitle=ForestGreen!40!white,coltitle=black,fonttitle=\bfseries\sffamily,
title=]
    連続微分可能な関数の空間$C^m(\Om)$を一般化して,$L^p$-微分可能な関数の空間を考える.
\end{tcolorbox}

\begin{definition}[weak differentiable, Sobolev space]
    $p\in[1,\infty),m\in\N$とする.
    \begin{enumerate}
        \item $p$乗可積分関数$u\in L^p(\Om)$が$m$階$L^p$-微分可能であるとは,任意の$\abs{\al}\le m$について,一般化された導関数$u_\al$が存在し,これも$p$-乗可積分である$u_\al\in L^p(\Om)$ことをいう.
        \item $m$階$L^p$-微分可能な関数全体のなす$L^p(\Om)$の部分空間を$W^{m,p}(\Om)$で表し\textbf{Sobolev空間}という.このとき$C^{m,p}(\Om)\subset W^{m,p}(\Om)$.
    \end{enumerate}
\end{definition}

\begin{theorem}
    Sobolev空間$W^{m,p}(\Om)$について,
    \begin{enumerate}
        \item $\norm{u}_{W^{m,p}(\Om)}=\paren{\sum_{\abs{\al}\le m}\norm{D^\al u}^p_{L^p(\Om)}}^{1/p}$をノルムとしてBanach空間をなす.
        \item $p=2$のとき,同じノルムについてHilbert空間をなし,対応する内積は$(u,v)_{W^{m,2}(\Om)}=\sum_{\abs{\al}\le m}(D^\al u,D^\al v)_{L^2(\Om)}$である.
    \end{enumerate}
\end{theorem}

\subsection{Sobolev空間の例}

\begin{tcolorbox}[colframe=ForestGreen, colback=ForestGreen!10!white,breakable,colbacktitle=ForestGreen!40!white,coltitle=black,fonttitle=\bfseries\sffamily,
title=]
    滑らかさの尺度$p$の大きさを調節することで,多様な滑らかさを表現できる.
    強い順に,連続微分可能,Lipschitz連続,$\al$-Holder連続($0<\al\le 1$),絶対連続,一様連続,連続関数となる.
\end{tcolorbox}

\begin{example}[一次元の場合]
    $I\osub\R$とする.
    滑らかさの尺度$p$の大きさを調節することで,多様な滑らかさを表現できる.
    強い順に,連続微分可能,Lipschitz連続,$\al$-Holder連続($0<\al\le 1$),絶対連続,一様連続,連続関数となる.
    \begin{enumerate}
        \item $W^{1,1}(I)\subset C(I)$は絶対連続な関数のなす空間\ref{thm-absolutely-continuous}.
        \item $W^{1,2}(I)=H^1(I)$は$1/2$次Holder連続な関数のなす空間である.
        \item $W^{1,\infty}(I)=\Lip(I)$はLipschitz連続関数全体のなす空間となる.
    \end{enumerate}
    Cantor関数は絶対連続でないが,連続かつ殆ど至る所微分可能で,Lebesgue可積分な導関数を持つ.
\end{example}

\begin{proposition}
    高次の場合は,もはや連続でない関数を含む.
    しかし,次のような形一般化出来る.
    $\Om\osub\R^n$とする.
    \begin{enumerate}
        \item $f\in W^{1,p}(\Om)\;(1\le p\le\infty)$は$\R^n$の座標方向に平行な殆どすべての直線への制限が絶対連続であるような関数と,殆ど至る所等しい.
        \item $f\in W^{1,p}(\Om)\;(n\le p\le\infty)$は$\gamma:=1-\frac{n}{p}$次Holder連続な関数と殆ど至る所等しい.特に$p=\infty$のときは局所Lipschitzな関数と殆ど至る所等しい.
    \end{enumerate}
\end{proposition}

\begin{proposition}\mbox{}
    \begin{enumerate}
        \item $W^{k,p}(\Om)\;(1\le p<\infty)$は可分である.
        \item $W^{k,\infty}$はノルム代数となる.
    \end{enumerate}
\end{proposition}

\subsection{弱微分の性質}

\begin{tcolorbox}[colframe=ForestGreen, colback=ForestGreen!10!white,breakable,colbacktitle=ForestGreen!40!white,coltitle=black,fonttitle=\bfseries\sffamily,
title=]
    $D^\al$は線型作用素である.
\end{tcolorbox}

\begin{proposition}[線型作用素]
    $u,v\in L^1_\loc(\Om)$に対して$D^\al u,D^\al v$が存在するならば,$au+bv\;(a,b\in\C)$も弱微分可能で,$D^\al(au+bv)=aD^\al u+bD^\al v$.
\end{proposition}

\begin{theorem}[Leibniz則]
    $u\in L^1_\loc(\Om)$は1階の弱偏導関数$D_ju$を持つとする.
    $\psi\in C^1(\Om)$に対して,$D_j(\psi u)$が存在し,$D_j(\psi u)=D_j\psi\cdot u+\psi\cdot D_ju$.
\end{theorem}

\begin{proposition}[畳み込みとの可換性]
    $u\in L^1_\loc(\R^n)$に対して$D^\al u$が存在するならば,
    \[\forall_{\varphi\in C_c^{\abs{\al}}(\R^n)}\;D^\al(\varphi *u)=(D^\al\varphi)*u=\varphi *D^\al u.\]
\end{proposition}

\begin{proposition}
    領域$\Om\osub\R^n$上の$u\in L^1_\loc(\Om)$が,1階導関数$D_j(u)\;(j\in[n])$をすべて持ち,すべて$0$に等しいならば,$u$は殆ど至る所定数である:$\exists_{\al\in\C}\;u(x)=\al\;\ae$.
\end{proposition}

\subsection{Sobolev部分空間}

\begin{tcolorbox}[colframe=ForestGreen, colback=ForestGreen!10!white,breakable,colbacktitle=ForestGreen!40!white,coltitle=black,fonttitle=\bfseries\sffamily,
title=]
    $L^p(\Om)$において$C_c(\Om),C_c(\Om)^\infty$が稠密であるように,$W^{m,p}(\Om)$では$C^{m,p}(\Om)$が稠密である.
\end{tcolorbox}

\begin{theorem}
    $C^{m,p}(\Om)$は$W^{m,p}(\Om)$上稠密である.
\end{theorem}
\begin{remark}
    $u\in\o{C^{m,p}(\Om)}$に$L^p$-収束する$C^{m,p}(\Om)$の列$(u_n)$が存在する:$\norm{D^\al u_n-D^\al u}_{L^p(\Om)}\to0$.
    これを用いて,$D^\al u=\lim_{n\to\infty,L^p}\partial^\al u_n$が成り立つ.
    すなわち,通常の意味での連続な導関数の$L^p$極限として得られる導関数を\textbf{$\al$-強微分}と呼ぶとすると,
    $u\in\o{C^{m,p}(\Om)}$については,強微分と弱微分は一致する.
    上の定理は,強微分と弱微分が完全に一致することも含意している.
\end{remark}

\begin{definition}
    $W^{m,p}_0(\Om):=\o{C^\infty_c(\Om)}$と定めると,$C^m_c(\Om)$関数も$C^\infty_c(\Om)$によって一様近似可能だから,$W^{m,p}_0(\Om)=\o{C^m_c(\Om)}$でもある.
\end{definition}

\begin{theorem}
    $\forall_{p\in[1,\infty)}\;W_0^{m,p}(\Om)=W^{m,p}(\Om)$.
\end{theorem}

\subsection{Sobolev空間の描像}

\begin{tcolorbox}[colframe=ForestGreen, colback=ForestGreen!10!white,breakable,colbacktitle=ForestGreen!40!white,coltitle=black,fonttitle=\bfseries\sffamily,
title=]
    $W^{1,p}(a,b)$は,偏微分方程式の解空間みたいになっている.
    この範囲で探す解を,$C^{1,p}(a,b)$-古典解とは別に,弱解と呼ぶ.
    $W^{1,p}_0(a,b)$はそのうち,境界上で消えるもののなす部分空間となっている.
\end{tcolorbox}

\begin{theorem}
    $u\in L^p((a,b))\;(a<b\in\oR)$について,次の2条件は同値.
    \begin{enumerate}
        \item $u\in W^{1,p}(a,b)$.
        \item ある$v\in L^p(a,b),\al\in\C,c\in(a,b)$が存在して,$u(x)=\int^x_cv(y)dy+\al\;\ae$と表せる.
    \end{enumerate}
    このとき,$u$は至る所連続で,$u(c)=\al$かつ$v=Du$であることに注意.
\end{theorem}

\begin{definition}
    $a,b\in\R$で,区間が有界である場合を考える.
    このとき,$u(x)=\int^x_cDu(y)dy+\al$は$[a,b]$上に連続延長出来る.
\end{definition}

\begin{theorem}
    $(a,b)$が有界であるとき,$u\in W^{1,p}(a,b)$について,次の2条件は同値.
    \begin{enumerate}
        \item $u\in W^{1,p}_0(a,b)$.
        \item $u(a)=u(b)=0$.
    \end{enumerate}
\end{theorem}

\section{$p=2$の場合}

\begin{tcolorbox}[colframe=ForestGreen, colback=ForestGreen!10!white,breakable,colbacktitle=ForestGreen!40!white,coltitle=black,fonttitle=\bfseries\sffamily,
title=]
    Sobolev空間がHilbert空間にもなる$p=2$のとき,$H^m(\R^n):=W^{m,2}(\R^n)$で表す.
    この場合をFourier変換を使って調べる.
\end{tcolorbox}

\begin{theorem}\label{thm-Hilbert-Sobolev-space}
    $W^{m,2}(\R^n)\simeq_\Hilb H^m(\R^n)$.
\end{theorem}

\subsection{同値なノルムによる拡張}

\begin{tcolorbox}[colframe=ForestGreen, colback=ForestGreen!10!white,breakable,colbacktitle=ForestGreen!40!white,coltitle=black,fonttitle=\bfseries\sffamily,
title=]
    同値なノルムを用いると,$H^m(\R^n)\;(m\in\N)$だけでなく,一般の$s\in\R_+$についても拡張出来る.
    なお,$H^0(\R^n)=L^2(\R^n)$で,$s$が大きくなる毎にどんどん狭くなっていく.
\end{tcolorbox}

\begin{proposition}[同値なノルム]
    $\R^n$にEuclidノルム$\abs{\xi}$を考える.
    \[\norm{u}_m=\paren{\int_{\R^n}\abs{(Fu)(\xi)}^2(1+\abs{\xi}^2)^md\xi}^{1/2}\]
    は$u\in H^m(\R^n)$上の$\nnorm{u}_m=\norm{u}_{W^{m,2}}$と同値なノルムを定める.
\end{proposition}

\begin{proposition}
    $s\in\R_+$に対して
    \[H^s(\R^n):=\Brace{u\in L^2(\R^n)\mid(Fu)(\xi)(1+\abs{\xi}^2)^{s/2}\in L^2(\R^n)}\]
    とおくと,これは同じノルムと内積についてHilbert空間をなす.
\end{proposition}

\begin{lemma}
    $\forall_{s\le s'\in\R_+}\;\forall_{u\in H^{s'}(\R^n)}\;\norm{u}_s\le\norm{u}_{s'}$.
\end{lemma}

\begin{theorem}
    $C_c^\infty(\R^n)$は$H^s(\R^n)$上稠密である.
\end{theorem}

\subsection{Sobolevの埋蔵定理}

\begin{tcolorbox}[colframe=ForestGreen, colback=ForestGreen!10!white,breakable,colbacktitle=ForestGreen!40!white,coltitle=black,fonttitle=\bfseries\sffamily,
title=]
    前節の消息
    \[C_c^\infty(\R^n)\subset H^s(\R^n)\subset H^{t}(\R^n)\subset H^0(\R^n)=L^2(\R^n)\;(s\ge t)\]
    は,より$s$の大きい$H^s$に属することは,$l^2$の意味でより滑らかであることを映し出している.
\end{tcolorbox}

\begin{theorem}
    $s>n/2$ならば,任意の$u\in H^s(\R^n)$は$\R^n$上殆ど至る所一様連続であり,定数$c=c_{n,s}$が存在して次が成り立つ:$\exists_{u\in H^s(\R^n)}\;\sup_{x\in\R^n}\abs{u(x)}\le c\norm{u}_s$.
\end{theorem}
\begin{remarks}
    $C^m,C_b^m$は古典的な導関数の深さと$\sup$ノルムで関数の滑らかさを測ったが,$H^s$一般に$W^{m,p}$は一般化された導関数と$L^2$-ノルムの組み合わせで滑らかさを測っている.
    そしてこれらの間に,$H^s$の滑らかさは,$n/2$だけずらせば$C_b^m$の中に埋め込めることをいう.
    すなわち,$s=m+n/2$という対応関係である.
    これは$s$が半整数であるが,整数の$s$は$s-n/2$次のHolder連続性を表す.
\end{remarks}

\begin{definition}[Holder continuous]
    $\al\ge0$について,
    $f:\R^n\supset\Om\to\C$が$\al$-Holder連続であるとは,
    \[\exists_{C\in\R}\;\forall_{x,y\in\Om}\;\abs{f(x)-f(y)}\le C\norm{x-y}^\al\]
    を満たすことをいう.
    $0$-Holder連続性は有界性と同値で,$1$-Holder連続性はLipschitz連続性に同値.
\end{definition}

\begin{theorem}
    $\frac{n}{2}<s<\frac{n}{2}+1$とする.このとき,定数$c=c_{n,s}$が存在して,$u$は$\R^n$上$(s-n/2)$-次Holder連続である:
    \[\forall_{u\in H^s(\R^n)}\;\forall_{x,x'\in\R^n}\;\abs{u(x')-u(x)}\le c_{n,s}\norm{u}_s\abs{x'-x}^{s-n/2}.\]
\end{theorem}

\chapter{Fourier変換}

\chapter{von Neumann環}

\begin{quotation}
    量子力学の「自然数,実数などの古典的な数学的対象を,すべて複素Hilbert空間上の作用素を用いて考えよ」という天啓から始まった
    作用素環の理論は,当然のように,数学全体の基礎づけの役割も果たし得る.
    作用素環の数理構造を解析しているうちに,病理的な対象でさえも場の量子論に自然に現れることが判明するという,逆の結果さえ生まれた.

\end{quotation}

\section{作用素位相}

\begin{tcolorbox}[colframe=ForestGreen, colback=ForestGreen!10!white,breakable,colbacktitle=ForestGreen!40!white,coltitle=black,fonttitle=\bfseries\sffamily,
title=]
    von Neumannの量子力学の基礎づけに限らず,種々の応用において一様作用素位相は強すぎる.
    そこで,作用素弱位相が注目された.
\end{tcolorbox}

\begin{definition}[uniform operator topology, weak operator topology, strong operator topology]
    $V,W\in\TVS$を位相線形空間,$L(V,W):=\Hom_\TVS(V,W)$をその間の連続線型作用素全体のなす空間とする.
    \begin{enumerate}
        \item $V,W$をノルム空間とする.これが定める作用素ノルム$p(A):=\sup_{v\ne0}\frac{p_W(Av)}{p_V(v)}$が定める位相を,\textbf{一様作用素位相}という.
        \item $U(x,f):=\Brace{A\in L(V,W)\mid\abs{f(A(x))}<1}$として,$(U(x,f))_{x\in V,f\in W^*}$を$0$の開近傍系の基底として生成される位相を,\textbf{弱作用素位相}という.作用素列$(A_n)$が$A$に弱作用素位相で収束することは,任意の$x\in X$について$(A_n(x))$が$(A(x))$に$W$の弱位相に関して収束する\footnote{すなわち任意の有界線型汎関数$f\in W^*$について$f(A_n(x))\to f(A(x))$.}ことに同値.
        \item $N(x,U):=\Brace{A\in L(V,W)\mid Av\in U}$として,$(N(x,U))_{x\in V,0\in U\osub W}$を$0$の開近傍系の基底として生成される位相を,\textbf{作用素強位相}という.作用素列$(A_n)$が$A$に強作用素位相で収束することは,任意の$x\in X$について$(A_n(x))$が$(A(x))$に$W$にてノルム収束することに同値.
    \end{enumerate}
\end{definition}

\begin{remark}
    局所コンパクト位相群$G$のユニタリ表現$G\to U(H)$を考えるに際して,$U(H)$には作用素強位相を入れ,ノルム位相を用いないのは,$U(H)$にノルム位相を入れると多くの表現が連続ではなくなってしまうためである.\footnote{\url{http://nlab-pages.s3.us-east-2.amazonaws.com/nlab/show/operator+topology}}
\end{remark}

\section{von Neumann環の定義}

\begin{definition}
    $C^*$-環$B(H)$の部分$*$-代数であって,単位元$1$を含み,作用素弱位相で閉じたものを\textbf{von Neumann環}という.
    $C^*$-環であって,あるvon Neumann環と同型でもあるようなものを$W^*$-環という.
\end{definition}
\begin{example}\mbox{}
    \begin{enumerate}
        \item $L^\infty(X)$は$W^*$-代数である.
    \end{enumerate}
\end{example}

\chapter{非線形関数解析}

\begin{quotation}
    関数解析の主な対象は線型写像であったが,非線形な関数解析に必要な線型化の理論,すなわち微分理論を考える.
    非線型汎函数の最適化理論である変分法は,遥か昔からの基本的問題であった.
    Specifically, it studies the critical points , i.e. the points where the first variational derivative of a functional vanishes, for functionals on spaces of sections of jet bundles. The kinds of equations specifying these critical points are Euler-Lagrange equations.\footnote{\url{https://ncatlab.org/nlab/show/variational+calculus}}
\end{quotation}

\section{Fréchet空間}

\subsection{定義1と特徴付け}

\begin{tcolorbox}[colframe=ForestGreen, colback=ForestGreen!10!white,breakable,colbacktitle=ForestGreen!40!white,coltitle=black,fonttitle=\bfseries\sffamily,
title=局所凸な$F$-空間]
    Fréchet空間とは,斉次性を緩めたノルムを備えた空間($F$-ノルム空間という)で,そのノルム位相について完備であるものをいう.
    なお,セミノルム空間はHausdorffでないから,「完備なセミノルム空間」という概念は筋が悪い.が,セミノルム可算族が定めるセミノルム位相はHausdorff足り得る\ref{def-seminorm-topology}.
    そして,セミノルム可算族が定める位相について完備な空間はFréchet空間の特徴付けとなる.
    さらに,Fréchet空間とは,距離化可能な完備なHausdorff局所凸位相線形空間なるクラスに一致することもわかる(距離がノルムから来ているかは不問).なお,このとき距離関数は平行移動不変性を持つように選べる.
    こうして,Banach空間の一般化として極めて安定な概念が見つかった.
    Banach空間の性質の大部分はFréchet空間に一般化出来る.
\end{tcolorbox}

\begin{definition}[$F$-norm, $F$-space, Frechet space]
    $V$を$K$-線型空間とする.$\norm{-}:V\to K$を関数とする.
    \begin{enumerate}
        \item 次の3条件を満たすとき,$\norm{-}$を\textbf{$G$-[セミ]ノルム}という.\footnote{ノルム空間のunderlying spaceをアーベル群について一般化した時に適切となるノルム概念である.}
        \begin{enumerate}[(a)]
            \item 正:$\norm{0_V}=0$.$[\norm{x}=0\Rightarrow x=0_V]$
            \item \textbf{$-1$の作用}:$\forall_{x\in V}\;\norm{-x}=\norm{x}$.
            \item 三角不等式:$\forall_{x,y\in V}\;\norm{x+y}\le\norm{x}+\norm{y}$.
        \end{enumerate}
        \item $G$-[セミ]ノルムが$V$に定める擬距離が定める位相について,スカラー乗法$K\times V\to V;(a,x)\mapsto ax$が連続であるとき,これを\textbf{$F$-[セミ]ノルム}という.
        \item そのノルム位相が完備な$F$-ノルム空間を\textbf{$F$-空間}という:すなわち,任意のCauchyネットが収束する$F$-ノルム空間を$F$-空間という.
        \item $F$-空間のノルム位相が局所凸であるとき,これを\textbf{Fréchet空間}という.射を連続線型写像とすると,Fréchet空間は圏TVSの充満部分圏をなす.
    \end{enumerate}
\end{definition}

\begin{lemma}[ノルムの$F$-性は距離でいうと平行移動不変性]\mbox{}
    \begin{enumerate}
        \item $F$-ノルムの定める距離$d:V\times V\to \R,d(x,y):=\norm{x-y}$は,平行移動不変性を満たす:$\forall_{x,y,a\in V}\;d(x+a,y+a)=d(x,y)$.
        \item 逆に,平行移動不変な距離$d:V\times V\to \R$について,$\norm{x-y}:=d(x-y,0)$と定めると,これは$F$-ノルムである.
    \end{enumerate}
\end{lemma}
\begin{proof}\mbox{}
    \begin{enumerate}
        \item 明らか:$d(x+a,y+a)=\norm{(x+a)-(y+a)}=d(x,y)$.
        \item 正であることは明らか.$-1$による作用について,$\norm{-x}=d(-x,0)=d(0,x)=d(x,0)=\norm{x}$.
        三角不等式は
        \[\norm{x+y}=d(x,-y)\le d(x,0)+d(0,-y)=\norm{x}+\norm{y}.\]
    \end{enumerate}
\end{proof}
\begin{remarks}[距離の言葉による特徴付け]
    $F$-空間を距離の言葉で定義すると,平行移動不変距離$d:V\times V\to \R$について完備な位相線形空間を定めること($d$が定める位相についてスカラー乗法と加法が連続であること)をいう.
    さらにこれが局所凸であるとき,Frechet空間という.
\end{remarks}

\subsection{例}

\begin{example}[連続関数の空間]
    $\Om\osub\R^n$上の連続関数の空間$C(\Om)$はFrechet空間となる.
    $C(\Om)$は局所有界でないので,ノルム付け可能ではない\ref{thm-character-of-TVS}.
\end{example}

\begin{example}[$p<1$の場合のLebesgue空間]\mbox{}
    \begin{enumerate}
        \item Lebesgue空間$l^p\;(0<p<1)$はFrechet空間ではないが,$F$-空間である.なお,この空間の$p$-ノルムは,三角不等式を成り立たせるために$p$乗根を省いて,$\norm{x}_p=\sum_{i}\abs{x_i}^p$とする.
        \item Lebesgue空間$L^p(X)$も,$p<1$のときFréchet空間でない局所凸線型空間となる.
    \end{enumerate}
\end{example}

\begin{example}[滑らかな関数の空間]\mbox{}
    \begin{enumerate}
        \item コンパクトな可微分多様体$X$上の可微分写像の空間$C^\infty(X)$は,半ノルムの族$p^\al_K:C_c^\infty(\R^n)\to\R_+;\Phi\mapsto\sup_{x\in K}\abs{\partial^\al\Phi(x)}\;(K\compsub\R^n,\al\in\N^n)$についてFréchet空間である.Schwartzはこれを$\mathcal{E}(\R^n)$で表した.
        \item コンパクトでない可微分多様体でも,$C^\infty(\R)$などはFréchet空間となる.
    \end{enumerate}
\end{example}

\begin{example}[Hardy space]
    $\Om\osub\C$上の正則関数の空間$H(\Om)\subset C(\Om)$はFrechet空間の閉部分空間をなすから,再びFrechet空間である.
    $H(\Om)$はHeine-Borel性を持つ.局所有界でないので,ノルム付け可能ではない\ref{thm-character-of-TVS}.
\end{example}

\begin{example}[Euclid空間の延長線上としてのFrechet空間]
    射影極限$\R^\infty:=\varprojlim_n\R^n$(位相線型空間としての直積として得られる空間)はFréchet空間となる.つまり,射影$p^n:\R^\infty\to\R^n$と$\R^n$上のノルム$\R^n\to\R$との合成の族$(p^n\circ\norm{-}_n)_{n\in\N}$はセミノルムの族になるが,これが定める位相について完備である.
\end{example}

\begin{example}[the Schwartz space]
    急減少関数の空間$\cS$はFréchet空間である.
\end{example}

\subsection{定義2とBanach空間との関係}

\begin{definition}[ゲージ空間としての捉え方]
    $V$の点を分離するセミノルムの列$\F=(\norm{-}_n)_{n\in\N}$が線型空間$V$に定めるセミノルム位相$\O$を考える.
    \begin{enumerate}
        \item $\O$はHausdorffになり\ref{def-seminorm-topology},
        \item さらに$(V,\O)$は局所凸である\ref{prop-characterization-of-locally-convex-spaces}.
        \item $d(x,y):=\sup_{n\in\N}\norm{x-y}_n=0$とすると,これはこのセミノルム位相$\O$を定める距離であり,平行移動について不変である.すなわち,$(V,\O)$はFrechet空間である.
    \end{enumerate}
\end{definition}
\begin{proof}
    (3)を示す.
    \begin{enumerate}[(a)]
        \item $d$は非負関数であり,$d(x,y)=0$のとき$\forall_{n\in\N}\;\norm{x-y}=0$である.$\F$は$V$の点を分離することから,これは$x=y$を意味する.
        \item 対称律は明らか.
        \item \begin{align*}
            d(x,y)=\sup_{n\in\N}\norm{x-y}_n&\le\sup_{n\in\N}\paren{\norm{x-z}_n+\norm{z-y}_n}\\
            &\le d(x,z)+d(z,y).
        \end{align*}
        \item 平行移動不変性も明らか.
    \end{enumerate}
\end{proof}

\begin{proposition}[Banach空間との関係]
    Fréchet空間$X$について,次の2条件は同値.
    \begin{enumerate}
        \item $X^*$はFréchet空間である.
        \item $X$はBanach空間である.
    \end{enumerate}
\end{proposition}

\subsection{定義3と総覧}

\begin{tcolorbox}[colframe=ForestGreen, colback=ForestGreen!10!white,breakable,colbacktitle=ForestGreen!40!white,coltitle=black,fonttitle=\bfseries\sffamily,
title=]
    実は,Frechet空間の定義は次の形にまで簡潔になる.\footnote{\url{http://www2.math.uni-wuppertal.de/~vogt/vorlesungen/fs.pdf}}
    以上の議論をさらに洗練させると,Frechet空間の同値な4つの定義は,次の補題の形にまとまる.
    この消息は,Hahn-Banachの分離定理\ref{sub-Hahn-Banach-separation}で見え隠れした,Minkowski汎関数の言葉を通じて,半ノルムと絶対凸集合との対応\ref{remarks-seminorm-and-absolutely-convex-sets}による.
\end{tcolorbox}

\begin{definition}
    完備な局所凸線型空間であって,距離化可能であるものを\textbf{Frechet空間}という.
\end{definition}

\begin{lemma}[Frechet空間の特徴付け]
    局所凸空間$E$について,次の4条件は同値.
    \begin{enumerate}
        \item $E$は距離化可能である.
        \item $E$には,$0$の開近傍の基底であって,可算なものが存在する.
        \item $E$は可算なセミノルムの基本系が存在する.
        \item $E$の位相は,ある平行移動不変な距離によって与えられる.
    \end{enumerate}
\end{lemma}

\subsection{Banach空間論と並行な議論}

\begin{proposition}
    $E$をFrechet空間,$F\subset E$を閉部分空間とする.このとき,$F,E/F$はいずれもFrechet空間である.
\end{proposition}

\begin{theorem}[開写像定理]
    $E,F$をFrechet空間,$A:E\to F$を全射な連続線型作用素とする.このとき,$A$は開写像である.
\end{theorem}

\begin{corollary}[逆写像定理]
    $E,F$をFrechet空間,$A:E\to F$を全単射な連続線型作用素とする.このとき,線型写像$A^{-1}$も連続である.
\end{corollary}

\begin{theorem}[閉グラフ定理]
    $E,F$をFrechet空間,$A:E\to F$を線型作用素とする.グラフ$G:=\Brace{(x,Ax)\in E\times F\mid x\in E}$が閉集合であるならば,$A$は連続である.
\end{theorem}

\subsection{樽型空間と一様有界性の原理}

\begin{tcolorbox}[colframe=ForestGreen, colback=ForestGreen!10!white,breakable,colbacktitle=ForestGreen!40!white,coltitle=black,fonttitle=\bfseries\sffamily,
title=]
    位相線形空間のうち,樽型空間と呼ばれるクラスについては,一様有界性の原理(Banach-Steinhaus)が成り立つ.
    Frechet空間はこれに当てはまる.
    Bourbaki 1950考案.
\end{tcolorbox}

\begin{definition}[absorbant, barrelled]\label{def-barrel}
    $E$を線型空間とする.
    \begin{enumerate}
        \item $M\subset E$が\textbf{吸収的集合}または\textbf{併呑集合}であるとは,$\cup_{t>0}tM=E$を満たすことをいう.
        \item 任意の閉じた絶対凸な併呑集合(これを\textbf{樽型集合}という)が,$0$の近傍であるとき,局所凸線型空間$E$を\textbf{樽型空間}であるという.
    \end{enumerate}
\end{definition}

\begin{lemma}\mbox{}
    \begin{enumerate}
        \item 任意の位相線形空間において,$0$の近傍は併呑である.
        \item 併呑集合の有限共通部分は併呑である.
        \item 樽は任意の有界完備凸部分空間を併呑する.
    \end{enumerate}
\end{lemma}

\begin{lemma}[併呑性の特徴付け]\label{lemma-characterizing-absorbant}
    $A\subset E$を絶対凸集合とする.このとき,次の2条件は同値.
    \begin{enumerate}
        \item $A$の定める計量関数$m_A$について$m_A(x)<\infty$.
        \item $A$は$x$を併呑する:$x\in\Span\{A\}:=\cup_{t>0}tA$.
    \end{enumerate}
\end{lemma}

\begin{notation}
    $A$を絶対凸集合とし,$A=m_A^{-1}([0,1])$と表せるとする.このとき,$m_A$は$X$上のノルムである.
    $A$の定める完備線型空間を
    \[E_A:=\o{\paren{\Span(A)/\Ker m_A,m_A}}\]
    で表す.実際,$\forall_{y\in\Ker m_A}\;m_A(x+y)=m_A(x)$を満たすから,$m_A$は$\Span(A)/\Ker m_A$上にノルムを定める.
\end{notation}

\begin{proposition}
    任意のFrechet空間$E$はbarrelledである.
\end{proposition}
\begin{proof}
    $M$を絶対凸な吸収的閉集合とする:$E=\cup_{t>0}tM$.このとき,均衡性より$E=\cup_{n\in\N}nM$も成り立つ.
    すると,各$nM$も閉であるから,
    Baireのカテゴリー定理より,ある$n_0$が存在して$n_0M$は内点を持つ.すると均衡性より$M$も内点を持つ.
    $E=\cup_{t>0}tM$と併せると,$0$は$M$の内点である.
\end{proof}

\subsection{双対空間}

\begin{theorem}
    $E$を距離化可能な局所凸空間とする.このとき,双対空間$E^*$は完備である.
\end{theorem}

\begin{corollary}
    $E$を距離化可能な局所凸空間とする.
    \begin{enumerate}
        \item $E^*$は距離化可能である,すなわち再びFrechet空間である.
        \item $E$はノルム空間である,すなわちBanach空間である.
    \end{enumerate}
\end{corollary}

\begin{theorem}[Alaoglu-Bourbaki]
    $E$を局所凸,$U\subset E$を$0$の開近傍とする.極集合$U^\circ$は$\sigma(E^*,E)$-コンパクトである.
\end{theorem}

\subsection{回帰性}

\begin{tcolorbox}[colframe=ForestGreen, colback=ForestGreen!10!white,breakable,colbacktitle=ForestGreen!40!white,coltitle=black,fonttitle=\bfseries\sffamily,
title=]
    樽型同様,有界型空間というのも,Bourbakiによる.
    ある種の位相線形空間は,開集合の系を考えるのが有効であるのと同様に,有界集合の系を考えることも有効になる.
\end{tcolorbox}

\begin{theorem}
    $E$をFrechet空間とする.$M\subset E^*$について,次の4条件は同値.
    \begin{enumerate}
        \item $M$は$w^*$-有界である.
        \item $M$は$w^*$-相対コンパクトである.
        \item $M$は有界である.
        \item $\exists_{k\in\N}\;\exists_{C>0}\;M\subset C^\circ_k$.
    \end{enumerate}
    $M=E^*$がこの同値な条件を満たす時,$E$は\textbf{distinguished}という.
\end{theorem}

\begin{definition}[bornological space]
    $E$を局所凸位相線形空間とする.
    \begin{enumerate}
        \item 部分集合$M\subset E$が\textbf{有界型}であるとは,任意の有界集合$B\subset E$について$t>0$が存在して$B\subset tM$を満たすことをいう.
        \item 任意の絶対凸な有界型集合が$0$の近傍であるとき,$E$を\textbf{有界型空間}または\textbf{界相空間}という.
    \end{enumerate}
\end{definition}

\begin{theorem}
    Frechet空間$E$について,次の2条件は同値.
    \begin{enumerate}
        \item $E$は回帰的である.
        \item 任意の$E$の有界集合は,弱相対コンパクトである.
    \end{enumerate}
\end{theorem}

\subsection{Schwartz空間}

\begin{definition}[precompact]
    $X$を線型空間,$V,U\subset X$を絶対凸とする.
    \begin{enumerate}
        \item $V\prec U:\Leftrightarrow\exists_{t>0}\;V\subset tU$とき,$U$は$V$を併呑するという.
        \item $V\prec U$であるとする.このときさらに,$\forall_{\ep>0}\;\exists_{m\in\N}\;\exists_{x_1,\cdots,x_m\in X}\;V\subset\cup_{j\in[m]}(x_j+\ep U)$が成り立つ時,$V$は\textbf{$U$-プレコンパクト}であるという.
    \end{enumerate}
\end{definition}

\begin{example}
    the Schwartz space $\cS(\R^n)$はSchwartz空間である.
\end{example}

\begin{definition}
    局所凸空間$E$が\textbf{Schwartz空間}であるとは,任意の絶対凸な$0$の開近傍$U$に対して,ある$U$-プレコンパクトな$0$の絶対凸開近傍$V\prec U$が存在することをいう.
\end{definition}

\begin{lemma}
    完備なSchwartz空間$E$において,任意の有界集合は相対コンパクトである.
\end{lemma}

\begin{theorem}
    任意のFrechet-Schwartz空間は回帰的である.
\end{theorem}

\begin{theorem}
    $E$がFrechet-Schwartz空間ならば,$E^*$はSchwartz空間である.
\end{theorem}

\subsection{核型空間}

\begin{tcolorbox}[colframe=ForestGreen, colback=ForestGreen!10!white,breakable,colbacktitle=ForestGreen!40!white,coltitle=black,fonttitle=\bfseries\sffamily,
title=]
    この理論の多くはGrothendieck 55により,テンソル積の言葉で探求された.
    Banach空間が核型ならば,有限次元である.
    A nuclear vector space is a locally convex topological vector space that is as far from being a normed vector space as possible.
\end{tcolorbox}

\begin{definition}[nuclear]
    局所凸空間$E$が\textbf{核型}であるとは,任意の絶対凸な$0$の開近傍$U$に対して,$0$の開近傍$V$と$w^*$-コンパクトな集合$V^\circ$上の有限なRadon測度$\mu$が存在して,$\forall_{x\in E}\;\norm{x}_U\le\int_{V^\circ}\abs{y(x)}d\mu(y)$を満たすことをいう.
\end{definition}

\section{微分論}

\begin{tcolorbox}[colframe=ForestGreen, colback=ForestGreen!10!white,breakable,colbacktitle=ForestGreen!40!white,coltitle=black,fonttitle=\bfseries\sffamily,
title=]
    It is possible to generalize some aspects of analysis (differential calculus) to Fréchet spaces.\footnote{\url{https://ncatlab.org/nlab/show/Fr\%C3\%A9chet+space}}
    全微分に対応する概念をFréchet微分,方向微分に対応する概念をGâteaux微分という.
    定義そのものは全く似ているが,存在性の議論に入ると無限が顔を出す.
    強微分については,連鎖律も基本定理も成り立つ.

    実数値関数の微積分は,可分な場合に限ってよく理解されている.
    主に数学的に研究されているクラスはLipschitz関数である.
\end{tcolorbox}

\subsection{Banach空間間のLipschitz関数}

\begin{tcolorbox}[colframe=ForestGreen, colback=ForestGreen!10!white,breakable,colbacktitle=ForestGreen!40!white,coltitle=black,fonttitle=\bfseries\sffamily,
title=]
    非線形関数解析の道具の多くはこの研究から流入している.
    距離構造のみが目を向けるに足りるBanach空間の構造であるから,その一様構造に注目する.\cite{Lindenstrauss}
\end{tcolorbox}

\begin{theorem}[Mazur and Ulam]
    Banach空間の間の等長同型であって,$0$を$0$に移すものは線型である.
\end{theorem}
\begin{remarks}
    Banach空間の構造は,距離空間の構造だけで分類出来てしまい,線型同型はあとからついて来る.つまり,距離構造の方が豊かとわかる.
\end{remarks}

\begin{theorem}[Kadec]
    任意の可分な無限次元Banach空間は互いに同相である.特に,$\R^{\aleph_0}$は可分なBanach空間である.
\end{theorem}
\begin{remarks}
    Banach空間の位相構造は,線型構造について何も示唆しない.
\end{remarks}

\subsection{Path smoothness}

\begin{tcolorbox}[colframe=ForestGreen, colback=ForestGreen!10!white,breakable,colbacktitle=ForestGreen!40!white,coltitle=black,fonttitle=\bfseries\sffamily,
title=]
    まず,Frechet空間上の連続な線型汎関数$\mu\in V^*$が滑らかであるとはどういうことかを初等的に定義する.
    Frechet空間$V$をある種の多様体だと考えると,その上の任意の曲線$f\in C^\infty(\R,V)$について微分可能であることがまず考えられ,これは連続であることを含意する.
    次に,Gateaux微分を高階にしたものを用いれば,通常のsmoothnessも定義でき,したがって可微分Frechet多様体が定義できる.
\end{tcolorbox}

\begin{definition}
    Fréchet空間$V$上の線型汎関数$\mu\in V^*$が\textbf{path smooth}であるとは,
    任意の可微分写像$f:\R\to V$に対して,合成$\mu\circ f:\R\to\R$が可微分であることをいう.
\end{definition}

\begin{proposition}
    Fréchet空間$V$上のpath smoothな線型汎関数$\mu\in V^*$は連続である.
\end{proposition}

\begin{corollary}
    Schwartz超関数は滑らかである.
\end{corollary}

\begin{definition}[曲線の微分]
    Fréchet空間$V$上の連続道$\gamma:I:=[a,b]\to V;t\mapsto f(t)$について,導関数
    \[f'(t):=\lim_{h\to0}\frac{1}{h}(f(t+h)-f(t))\]
    が存在して連続であるとき,$C^1$級であるという.
\end{definition}

\subsection{偏微分}

\begin{tcolorbox}[colframe=ForestGreen, colback=ForestGreen!10!white,breakable,colbacktitle=ForestGreen!40!white,coltitle=black,fonttitle=\bfseries\sffamily,
title=]
    有限次元の場合と全く同様に微分を定義でき,$C^n(V)$の概念がある.
    すると,滑らかな関数という概念もあり,Fréchet多様体も同様に考えられる.
\end{tcolorbox}

\begin{definition}[directional / Gateaux derivative, Frechet derivative]
    $F,G$をFréchet空間,$P$を$U\osub F$上の連続な非線型写像$P:U\to G$とする.
    \begin{enumerate}
        \item 
    $f\in U$での$h\in F$方向微分を与える写像
    \[\xymatrix@R-2pc{
        DP:U\times F\ar[r]&G\\
        \rotatebox[origin=c]{90}{$\in$}&\rotatebox[origin=c]{90}{$\in$}\\
        (f,h)\ar@{|->}[r]&DP(f)h:=\lim_{t\to0}\frac{1}{t}(P(f+th)-P(f))
    }\]
    が存在して連続であるとき,$P$は$U$上$C^1$級であるという.
    $DP(f)=D_P(f)$を,$f\in F$における\textbf{弱導関数}または\textbf{Gâteaux導関数}という.\footnote{Lagrangeに従えば,$P$の$f$における第一変分という.}
    \item さらに,$DP(f)$が単位球$h\in B^\subset F$上一様に存在するとき,すなわち,
    \[\forall_{h\in F}\quad P(f+h)=P(f)+DP(f)h+o(\norm{h}).\]
    が成り立つとき,$P$は$f\in U$において\textbf{Frechet微分可能}といい,$DP(f)$を特に
    \textbf{強導関数}または\textbf{Fréchet導関数}という.
    \end{enumerate}
\end{definition}

\begin{lemma}
    連続作用素$P:F\to G$は$U\osub F$においてGateaux微分可能であるとする:$DP:U\times F\to G$は連続.
    \begin{enumerate}
        \item Gateaux導関数$DP(f):F\to G$は一意な連続線型作用素である:$DP(f)\in L(F,G)$.なお,連続写像$DP:U\times F\to G$の存在の仮定なくして,ある一点$f\in F$で弱微分可能なだけであるとき,Gateaux微分$DP(f):F\to G$が線型とは限らない.
        \item 連続とは限らない$P$について,$f\in F$においてFrechet微分可能ならば,$f\in F$において連続.なお,Gateaux微分可能性のみでは連続とは限らない.
    \end{enumerate}
\end{lemma}
\begin{proof}\mbox{}
    \begin{enumerate}
        \item 
        $\forall_{\ep>0}\;\exists_{\delta>0}\;\norm{h}<\delta\Rightarrow\norm{P(f+h)-P(f)-DP(f)h}\le\ep\norm{h}$がGateaux微分可能性の同値な条件である.これは明らかに$DP(f)$の連続性を含意する.
        また,$L_1,L_2$を2つのGateaux微分とすると,$\norm{L_1h-L_2h}=o(h)$であるが,これは$L_1=L_2$を含意する.
    \end{enumerate}
\end{proof}

\begin{theorem}[$C^1$級]
    写像$f:E\to F$は$x_0$の近傍でGateaux微分可能かつ,Gateaux導関数$Df(x)$は$x_0$にて連続とする:$\lim_{x\to x_0}\norm{Df(x)-Df(x_0)}=0$.このとき,$f$は$x_0$にてFrechet微分可能である.
    これが成り立つ時,写像$f$は$x_0$にて$C^1$級であるという.\footnote{Gateauxの意味か,Frechetの意味かは不問になる.}
\end{theorem}

\subsection{写像論}

\begin{tcolorbox}[colframe=ForestGreen, colback=ForestGreen!10!white,breakable,colbacktitle=ForestGreen!40!white,coltitle=black,fonttitle=\bfseries\sffamily,
title=]
    逆写像定理と陰関数定理がそのまま成り立つ.
\end{tcolorbox}

\begin{theorem}
    $E,F$をBanach空間,$f:E\supset U\to F$を開集合上の$C^1$級関数とする.$Df(x_0)$が可逆であるとき,$f$の$x_0$の近傍$U_0$への制限は$f|_{U_0}$は位相同型で,$f^{-1}$は$f(U_0)$上$C^1$級である.
\end{theorem}

\subsection{高階微分}

\begin{definition}
    $P$の$k$方向の2階微分を
    \[D^2P(f)(h,k)=\lim_{t\to0}\frac{1}{t}(DP(f+tk)h-DP(f)h)\]
    と定める.
\end{definition}

\section{積分論}

\begin{tcolorbox}[colframe=ForestGreen, colback=ForestGreen!10!white,breakable,colbacktitle=ForestGreen!40!white,coltitle=black,fonttitle=\bfseries\sffamily,
title=]
    確率積分と違って,考慮に足りる積分が複数ある.
    Bochner積分\ref{subsection-Pettis-integral}という.
\end{tcolorbox}

\subsection{Riemann積分の一般化}

\begin{notation}
    道
    $f:I:=[a,b]\to F$に沿った積分$\int^b_af(t)dt\in F$を定めたいが,自然な定め方はただ一つに定まる.
\end{notation}

\begin{theorem}
    次の4条件を満たす元$\int^b_af(t)dt\in F$は一意的である:
    \begin{enumerate}
        \item $\forall_{\phi\in F^*}\;\phi\paren{\int^b_af(t)dt}=\int^b_a\phi(f(t))dt$.
        \item $\forall_{\norm{-}:F\to\R:\text{seminorm}}\;\Norm{\int^b_af(t)dt}\le\int^b_a\norm{f(t)}dt$.
        \item 積分作用素$C(I,F)\ni f\mapsto \int^b_af(t)dt\in F$は線型である.
        \item 加法的である:$\int^b_af(t)dt+\int^c_bf(t)dt=\int^c_af(t)dt$.
    \end{enumerate}
\end{theorem}

\begin{theorem}[微積分学の基本定理]
    $P:F\to G$は$C^1$球で,$\forall_{t\in[0,1]}\;f+th\in\Dom(P)$ならば,
    \[P(f+h)-P(f)=\int^1_0DP(f+th)hdt.\]
\end{theorem}

\begin{theorem}[連鎖律]
    $P:F\to G,Q:G\to H$は$C^1$級であるとする.このとき,合成$Q\circ P$も$C^1$級で,
    \[D[Q\circ P](f)h=DQ(P(f))DP(f)h.\]
\end{theorem}

\section{幾何学}

\subsection{連続度}

\begin{definition}
    $X,Y$を距離空間,$f:X\to Y$を関数とする.
    \begin{enumerate}
        \item $\om_f(t):=\sup_{d(f(x),f(y))\ge0\mid d(x,y)\le t}$によって定まる写像$\om_f:\R_+\to\R_+$を\textbf{連続度}という.
        \item $\exists_{t_0>0}\;\forall_{t<t_0}\;\om_f(t)<\infty$かつ$\lim_{t\to0+}\om_f(t)=0$が成り立つ時,$f$は一様連続であるという.
        \item $\exists_{c\in\R}\;\om_f(t)\le ct$が成り立つとき,$f$は\textbf{Lipschitz写像}であるといい,条件を満たす最小の$c=\sup_{x\ne y}\frac{d(f(x),f(y))}{d(x,y)}$をLipschitz定数という.
        \item $\exists_{\al>0}\;\exists_{c\in\R}\;\om_f(t)\le ct^\al$が成り立つとき,$f$は\textbf{Holder$(\al)$-写像}であるという.
    \end{enumerate}
\end{definition}

\begin{lemma}
    $X$が凸集合であるとき,$\om_f$は劣加法的である:$\om_f(t+s)\le\om_f(t)+\om_f(s)$.
\end{lemma}

\subsection{不動点定理}

\begin{tcolorbox}[colframe=ForestGreen, colback=ForestGreen!10!white,breakable,colbacktitle=ForestGreen!40!white,coltitle=black,fonttitle=\bfseries\sffamily,
title=]
    Brouwerの不動点定理は,有限次元についての結果で,いくつかの同値な条件を持ったが,Banach空間ではそのいくつかは失敗する.
\end{tcolorbox}

\begin{theorem}[Schauder-Tychonoff (30)]
    局所凸位相線形空間$E$のコンパクトな凸集合$K$上の全射な連続写像$K\to K$は不動点を持つ.
\end{theorem}

\begin{theorem}[Banach (22)]
    $X$を完備距離空間,
    $f:X\to X$を縮小写像,すなわち$1$より小さいLipschitz定数をもつLipschitz写像とする.
    このとき,$f$はただ一つの不動点を持つ.
\end{theorem}

\chapter{最適化と変分法}

\begin{quotation}
    最適化とは極値問題のことであり,変分法とは汎関数の最適化である.
    この問題を関数解析の枠組みから捉え直したい.

    ニューラルネットワークをはじめ,統計モデルとは,関数のクラスに他ならない.
    位相解析の知見が使えないはずがない.
\end{quotation}

\section{枠組み}

\subsection{最適化問題}

\begin{definition}[mathematical programming / optimization, objective function, feasible region, constraint]
    数理計画問題または最適化問題は,次のように表す:
    \[\min\;f(x),\quad\st\;x\in S.\]
    $f$を目的関数,s.t.とはsubject to,$S$を実行可能領域または許容領域,条件$x\in S$を制約という.
    $S$を定める関数
    \[S:=\Brace{s\in\R^n\mid\forall_{i\in[m]}\;g_i(x)\le0,\forall_{j\in[l]}\;h_j(x)=0}\]
    を考える.
    \begin{enumerate}
        \item $f,g_i,h_j$がすべて1次関数であるとき,これを線型計画問題ともいう.そうとは限らないとき,非線形計画問題という.線型計画問題は非線形計画問題である.
        \item $f$が2次,$g_i,h_j$が1次であるとき,これを\textbf{2次計画問題}という.
        \item $f,g_i$が凸で$h_j$が1次であるとき,これを\textbf{凸計画問題}という.
        \item 制約条件の中に行列の半正定値条件を含むような問題を\textbf{半正定値計画問題}という.
    \end{enumerate}
\end{definition}

\subsection{基底についてのメモ}

\begin{tcolorbox}[colframe=ForestGreen, colback=ForestGreen!10!white,breakable,colbacktitle=ForestGreen!40!white,coltitle=black,fonttitle=\bfseries\sffamily,
title=]
    有限次元線型空間論では,基底論がほぼ全てであった.
    無限次元の場合,無限の切り取り方に任意性がある.
\end{tcolorbox}

\begin{definition}[Hamel, topological, Schauder basis]
    位相線型空間$V$と部分集合$B\subset V$について,
    \begin{enumerate}
        \item $B$が\textbf{Hamel基底}であるとは,任意の元が$B$の有限線型和で一意的に表せることをいう.すなわち,$\Span(B)=V$かつ,任意の$B$の真の部分集合はこれを満たさないことをいう.
        \item $B$が\textbf{位相基底}であるとは,任意の元が$B$の有限線型和の列の極限で表わせ,任意の$B$の元は他の$B$の元の有限線型和の列の極限で表せないことをいう.すなわち,$\o{\Span(B)}=V$(これをtotalという)かつ任意の$B$の真の部分集合はtotalでないことをいう.
        \item $B$が\textbf{Schauder基底}であるとは,任意の元が$B$の(無限足り得る)線型和で一意的に表せることをいう.
    \end{enumerate}
\end{definition}
\begin{remarks}
    Hamel基底の存在は選択公理と同値.Schauder基底は位相基底であり,位相基底には双対基底が存在する.
\end{remarks}

\begin{example}
    $c_0$と$l^p\;(1\le p<\infty)$には,標準的なSchauder基底$(e_i)_{i\in\N}$が存在する.
    なお,$l^\infty$は可分でない.
\end{example}

\section{凸関数論}

\begin{tcolorbox}[colframe=ForestGreen, colback=ForestGreen!10!white,breakable,colbacktitle=ForestGreen!40!white,coltitle=black,fonttitle=\bfseries\sffamily,
title=凸解析の整備]
    Hahn-Banachの分離定理は,位相線型空間上の互いに素な凸集合は,連続線型汎関数$\varphi$が定める超平面を用いて分離出来ることをいう.
    Hahn-Banachの定理から始まり,Frechet空間論も,位相線形空間の議論は凸集合への注目と切り離せない.
    特に,凸集合がノルム閉であることと弱閉であることが同値\ref{prop-closedness-of-convex-sets}であることが使える.
    特に最適化理論では強力な道具になる.
\end{tcolorbox}

\subsection{凸集合論}



\begin{lemma}\mbox{}
    \begin{enumerate}
        \item 任意個数の(閉)凸集合族$(S_i)_{i\in I}$の共通部分$\cap_{i\in I}S_i$は閉凸集合である.凸性とは$\forall_{u,v\in S}\;[u,v]\subset S$であるので,空集合は凸であることに注意.
        \item $\R^n$の$m$個の元の凸結合は,$m\ge n+2$ならば,$n+1$個の元を選んでその凸結合として表せる.
    \end{enumerate}
\end{lemma}

\begin{theorem}[Caratheodory]
    任意の集合$S\subset\R^n$の凸包$\Conv S$は,
    \begin{enumerate}
        \item $S$の高々$n+1$個の点の凸結合全体の集合として表せる.
        \item $S$が連結ならば,高々$n$個の点の凸結合全体の集合として表せる.
    \end{enumerate}
\end{theorem}

\begin{theorem}
    有界閉集合$S\subset\R^n$の凸包$\Conv S$は閉集合である.
    すなわち,任意の集合$S\subset\R^n$について$\Conv(\o{S})\subset\o{\Conv S}$であり,$S$が有界であるとき等号成立.
\end{theorem}

\begin{definition}
    集合$S\subset V$について,
    \begin{enumerate}
        \item $S$を含む最小のaffine空間を$S$の\textbf{affine包}と呼び,$\Aff S$で表す.
        \item $x\in S$が\textbf{相対的内点}であるとは,$\exists_{\ep>0}\;(B(x,\ep)\cap\Aff S)\subset S$を満たすことをいう.
        \item 有限個の半平面の共通部分を\textbf{多面体}という.
    \end{enumerate}
\end{definition}

\subsection{錐論}

\begin{tcolorbox}[colframe=ForestGreen, colback=ForestGreen!10!white,breakable,colbacktitle=ForestGreen!40!white,coltitle=black,fonttitle=\bfseries\sffamily,
title=]
    閉凸錐の構造がよく現れる.
\end{tcolorbox}

\begin{definition}[cone, poluhydron]
    非空集合$C\subset V$について,
    \begin{enumerate}
        \item $\forall_{x\in C}\;\R_+x\subset C$が成り立つ時,\textbf{錐}であるという.集合$\R_+x$を\textbf{射線}という.
        \item $\R_+$を係数とした線型和のことを,\textbf{錐結合}という.錐結合全体の集合は閉凸錐になり,これを生成される錐という.\textbf{錐包}を$\Cone S$で表す.
        \item 有限個の半平面の共通部分として定まる集合を\textbf{多面体}という:$\exists_{s_1,\cdots,s_m\in\R^n}\;\exists_{r_1,\cdots,r_m\in\R}\;P=\Brace{x\in\R^n\mid\forall_{i\in[m]} s_i^\top x\le r_i}$.
        \item $\forall_{i\in[m]}\;r_i=0$のときの多面体を,\textbf{多面錐}という.すなわち,$0$を通る有限個の半平面の共通部分として定まる非有界集合である.
    \end{enumerate}
\end{definition}

\begin{example}\mbox{}
    \begin{enumerate}
        \item $C_p:=\Brace{(t,x)\in\R\times\R^n\mid t\ge\norm{x}_p}$は凸錐である.$C_2$を\textbf{二次錐}という.
        \item 対称行列の空間$\R^{n(n+1)/2}$内で,半正定値行列が作る集合は凸錐である.一般に,自己共役作用素$B(H)_\sa$内で半正定値作用素は閉凸錐をなす.実軸の非負部分$\R_+$は凸錐であるが,この例は退化したものと捉えられる.
    \end{enumerate}
\end{example}


\begin{lemma}[多面錐の特徴付け]
    部分集合$P\subset\R^n$について,次の2条件は同値.
    \begin{enumerate}
        \item $P$は多面錐である.
        \item $\exists_{m\in\N}\;\exists_{x_1,\cdots,x_m}\;P=\Cone(\{x_1,\cdots,x_m\})$.
    \end{enumerate}
\end{lemma}

\begin{lemma}[多面体の特徴付け]
    部分集合$P\subset\R^n$について,次の2条件は同値.
    \begin{enumerate}
        \item $P$は多面体である.
        \item $\exists_{k,k'\in\N}\;\exists_{x_1,\cdots,x_k,y_1,\cdots,y_{k'}\in\R^n}\;P=\Cone(\{x_1,\cdots,x_k\})+\Conv(\{y_1,\cdots,y_{k'}\})$.
    \end{enumerate}
\end{lemma}

\begin{definition}[polar cone, dual cone, self-adjoint cone]
    錐$C\subset\R^n$に対して,
    \begin{enumerate}
        \item $C^\circ:=\Brace{s\in\R^n\mid\forall_{x\in C}\;s^\top x\le0}$を\textbf{極錐}という.これは錐$C$の極集合である\ref{lemma-polar}.
        \item $C^*:=-C^\circ=\Brace{s\in\R^n\mid\forall_{x\in C}\; s^\top x\ge0}$を\textbf{双対錐}という.
        \item $C^*=C$を満たすとき,錐$C$を\textbf{自己共役錐}という.
    \end{enumerate}
\end{definition}

\begin{example}[自己共役錐]
    $C_2$,半正定値作用素のなす凸錐は,自己共役錐である.
\end{example}

\begin{corollary}[双極定理]
    錐$C,C_1,C_2\subset\R^n$について,
    \begin{enumerate}
        \item $C^\circ$は閉凸錐.
        \item $C_1\subset C_2\Rightarrow C_1^\circ\supset C_2^\circ$.
        \item $C$が凸のとき,$C^{\circ\circ}=\Cl\;C$.
    \end{enumerate}
\end{corollary}
\begin{proof}
    双極定理\ref{thm-polar-theorem}の錐に関する特殊化である.錐は必ず$0$を含むことに注意.
\end{proof}

\begin{corollary}[Krein-Milman]
    有界閉な凸集合$C\subset\R^n$に対して,$C$の極点全体がなす集合の凸包は$C$に等しい:$C=\Conv(\Ex(C))$.
\end{corollary}

\begin{definition}
    集合$S\subset\R^n$とその点$x\in S$について,
    \begin{enumerate}
        \item $d\in\R^n$が$x$における$S$の\textbf{接ベクトル}であるとは,$x$に収束する$S$の列$\{x_k\}\subset S$と$0$に上から収束する数列$(t_k)$が存在して,$d=\lim_{k\to\infty}\frac{x_k-x}{t_k}$と表せることをいう.
        \item 接ベクトル全体の集合$T_S(x)$は閉錐をなし,\textbf{接錐}という.$x\in\Int S\Rightarrow T_S(x)=\R^n$に注意.
        \item 接錐の極錐$N_S(x):=T_S(x)^\circ$を\textbf{法線錐}という.
    \end{enumerate}
\end{definition}

\subsection{凸集合の分離}

\begin{definition}[closed hyperplane, separate]\mbox{}
    \begin{enumerate}
        \item 連続線型汎関数$f\in X^*$と実数$\al\in\R$を用いて
        $H:=\Brace{u\in X\mid f(u)=\al}$と表せる集合を,\textbf{閉超平面}という.
        \[H_\le:=\Brace{u\in X\mid f(u)\le\al}\quad H_>:=\Brace{u\in X\mid f(u)>\al}\]
        と表す.
        \item 集合$A,B\subset X$について,$A\subset H_\le\land B\subset H_>$を満たすとき,\textbf{真に分離する}といい,$A\subset H_\le\land B\subset H_\ge$を満たすとき分離するという.
    \end{enumerate}
\end{definition}

\begin{lemma}[Mazur's lemma:弱閉凸集合はノルム閉である]
    $V$をノルム空間とし,
    $(u_n)_{n\in\N}$を$\o{u}$に弱収束する点列とする.
    このとき,凸結合$v_n=\sum^N_{k=n}\lambda_ku_k,\sum^N_{k=n}\lambda_k=1,\lambda_k\ge0$の列が存在して,$\o{u}$にノルム収束する.
\end{lemma}

\subsection{凸関数}

\begin{definition}[convex function, strictly convex function]
    線型空間$V$の凸部分空間$W$上の関数$F:W\to\o{\R}$が\textbf{凸関数}であるとは,
    \[\forall_{u,v\in W}\;\forall_{\lambda\in[0,1]}\;F(\lambda u+(1-\lambda)v)\le\lambda F(u)+(1-\lambda)F(v)\]
    が(右辺が意味を持つ限り)成り立つことをいう.
    $u=v$または$\lambda\in\partial[0,1]$である場合を除き,常に不等号が真に成り立つとき,$F$は\textbf{狭義凸}または\textbf{厳密に凸}であるという.
\end{definition}
\begin{remarks}[effective domain, proper]
    定義域に$+\infty$を許すことで,任意の部分関数を,$V$全体に凸なまま延長して考えることができる.
    $\dom F:=\Brace{u\in V\mid F(u)<\infty}$を\textbf{実効定義域}という.
    また,集合の定義関数を$\{0,+\infty\}$に値を取ることにすれば,$A$が凸集合であることと,$\chi_A$が凸関数であることとが同値になる.
    したがって,以降は凸関数のみを扱えばよい.
    一方で,$F(u_0)=-\infty$なる点$u_0$が存在すると,任意の$u_0$を端点とする半直線上で,恒等的に$-\infty$であるか,ある一点で$+\infty$に切り替わる振る舞いをするかになる.
    以降,\textbf{真凸関数}といったとき,$F$は$-\infty$を取らず,恒等的に$+\infty$ではないとする.
\end{remarks}

\begin{definition}[epigraph]
    \[\Epi F:=\Brace{(u,a)\in V\times\R\mid f(u)\le a}\]
    を関数$F:V\to\o{\R}$のエピグラフという.$\pr_1(\Epi F)=\dom F$である.
\end{definition}

\begin{proposition}[凸関数の特徴付け]
    関数$F:V\to\o{\R}$について,次の2条件は同値.
    \begin{enumerate}
        \item $F$は凸である.
        \item $\Epi F$は凸である.
    \end{enumerate}
\end{proposition}

\begin{proposition}[凸関数の凸錐]
    関数$F,G:V\to\o{\R}$について,
    \begin{enumerate}
        \item $F$が凸ならば,$\lambda>0$について$\lambda F$も凸である.
        \item $F,G$が凸ならば,$F+G$も凸である.ただし,$\infty-\infty=+\infty$とする.
        \item 凸関数族$(F_i)_{i\in I}$について,各点で上限を取るもの$F:=\sup_{i\in I}F_i$は凸である.
    \end{enumerate}
    特に,凸関数は凸錐をなす.
\end{proposition}

\subsection{凸関数の変種}

\begin{definition}
    \[\exists_{\al\in\R}\;\forall_{\lambda\in[0,1]}\;\lambda f(x)+(1-\lambda)f(y)\ge f(\lambda x+(1-\lambda)y)+\frac{1}{2}\al\lambda(1-\lambda)\norm{x-y}^2\]が成り立つ関数$f$を\textbf{強凸関数}という.
\end{definition}

\begin{definition}[level set]\label{def-level-set}
    $\Epi f$の各$a\in\R$に関する切断面
    $S_\al:=\Brace{x\in\R^n\mid f(x)\le\al}$を
    等位集合という.凸関数の等位集合は凸集合である.有効定義域も凸である.
    等位集合がすべて凸集合となる関数を\textbf{準凸関数}という.これは凸関数とは限らない.
\end{definition}

\begin{definition}
    微分可能な凸関数$f:\R^n\to\R$は$\brac{\nabla f(x),y-x}\ge0\Rightarrow f(y)\ge f(x)$を満たす.
    すなわち,方向微分が非負である方向に関して,関数値は減少しない.
    この条件を満たす微分可能な関数$f$を\textbf{擬凸関数}という.
    擬凸関数は準凸関数である.
\end{definition}

\subsection{凸関数の例}

\begin{example}[perspective]
    凸関数$f:\R^n\to\R$の\textbf{錐拡張}とは,$g:\R^{n+1}\to\R\cup\{\infty\}$であって,
    \[g(x,t):=\begin{cases}
        tf(x/t),&t>0,\\
        +\infty,t\le0,
    \end{cases}\]
    として定まるものをいう.
    これはっ塔である.
\end{example}

\begin{corollary}
    関数$f:\R^n\to\R$について,
    \begin{enumerate}
        \item $f$が微分可能であるとき,$f$が凸であることと$\forall_{x,y\in\R^n}\;f(y)\ge f(x)+\brac{\nabla f(x),y-x}$とは同値.
        \item $f$が$C^2$級であるとき,$f$が凸であることと,任意の点$x\in\R^n$でHesse行列$\nabla^2f(x)$が半正定値であることとは同値.
    \end{enumerate}
\end{corollary}

\begin{definition}
    作用素$h:V\to V$が単調であるとは,$\forall_{x,y\in V}\;\brac{h(x)-h(y),x-y}\ge0$が成り立つことをいう.
\end{definition}

\begin{proposition}
    $C^1$級関数$f:\R^n\to\R$について,$f$が凸であることと,勾配$\nabla f:\R^n\to\R^n;x\mapsto\nabla f(x)$が単調であることとは同値.
\end{proposition}

\subsection{下半連続関数}

\begin{tcolorbox}[colframe=ForestGreen, colback=ForestGreen!10!white,breakable,colbacktitle=ForestGreen!40!white,coltitle=black,fonttitle=\bfseries\sffamily,
title=]
    凸関数の特徴付けにおいて,$\Epi F$が凸であるという部分を閉であるとすると,これは下半連続性の特徴付けとなる.
\end{tcolorbox}

\begin{definition}
    関数$F:V\to\o{\R}$が下半連続であるとは,次の2つの同値な条件を満たすことをいう:
    \begin{enumerate}
        \item $\forall_{a\in\R}\;\Brace{u\in V\mid F(u)\le a}$ is clsoed.
        \item $\forall_{\o{u}\in V}\;\liminf_{u\to\o{u}}F(u)\ge F(\o{u})$.
    \end{enumerate}
    $V=\R$のときの特徴付け\ref{prop-characterization-of-lsc}参照.
\end{definition}

\begin{proposition}[下半連続関数の特徴付け]
    関数$F:V\to\o{\R}$について,
    \begin{enumerate}
        \item $F$は下半連続である.
        \item $\Epi F$は閉である.
    \end{enumerate}
\end{proposition}
\begin{remarks}
    特に,閉集合の定義関数は下半連続である.
    すなわち,連続な関数のうち,いくつかの点が下に飛び出していてもエピグラフは閉のままであり,このようなクラスが下半連続関数である.
    そう考えると,$f,-f$がいずれも下半連続であることが$f$の連続性を特徴付ける.
\end{remarks}

\begin{definition}[regularization of mappings]
    任意の下半連続関数族の下限は下半連続であった\ref{prop-subalgebra-of-lsc-function}.
    そこで,任意の関数$F\in\Map(V,\o{\R})$に対して,
    束$C^{1/2}(V)$の中で下限を取ることが考えられる.これを$F$の\textbf{下半連続正則化}といい,$\o{F}$で表す:
    \[\o{F}:=\sup\Brace{f\in C^{1/2}(V)\mid f\le F}.\]
\end{definition}

\begin{corollary}
    $F:V\to\o{\R}$を関数,$\o{F}$をその下半連続正則化とする.このとき,
    \begin{enumerate}
        \item $\Epi\o{F}=\o{\Epi F}$.
        \item $\forall_{u\in V}\;\o{F}(u)=\liminf_{v\to u}F(v)$.
    \end{enumerate}
\end{corollary}

\begin{corollary}
    下半連続な凸関数$F:V\to\o{\R}$は,弱位相$\sigma(V,V^*)$についても下半連続である.
\end{corollary}
\begin{remarks}
    凸集合についてノルム閉であることと弱閉であることとが同値であるという消息は,ここまで換言できる.
\end{remarks}

\subsection{凸関数の連続性}

\begin{tcolorbox}[colframe=ForestGreen, colback=ForestGreen!10!white,breakable,colbacktitle=ForestGreen!40!white,coltitle=black,fonttitle=\bfseries\sffamily,
title=]
    凸関数が連続であることと,局所Lipschitzであることと,ある開集合が存在してその上で有界であることとは同値である.
\end{tcolorbox}

\begin{lemma}
    凸関数$F:V\to\oR$が$u\in V$の近傍において有界である$\exists_{U\in\O(u)}\;\exists_{M\in\R}\;\abs{F|_U}\le M\;\on U$とき,$F$は$u$において連続である.
\end{lemma}

\begin{proposition}
    凸関数$F:V\to\oR$について,次の2条件は同値.
    \begin{enumerate}
        \item 非空な開集合$U$が存在して,$F|_U$は真の関数で\footnote{$-\infty$を取らない},ある定数$M\in\R$より大きくならない.
        \item $F$は真の関数で,$\Dom F$の内部で連続である.
    \end{enumerate}
\end{proposition}

\begin{corollary}[有限次元空間上の凸関数は連続]
    有限次元空間上の真の凸関数は,$\Dom F$の内部で連続である.
\end{corollary}

\begin{corollary}
    $F:V\to\oR$をノルム空間上の真の凸関数とする.
    次の2条件は同値.
    \begin{enumerate}
        \item 非空な開集合$U$が存在して,$F$は$U$上有界である.
        \item $\Dom F$の内部は非空で,$F$はその上で局所Lipschitzである.
    \end{enumerate}
\end{corollary}

\begin{corollary}
    樽型空間(特にBanach空間)上の下半連続な凸関数は,$\Dom F$の内部において連続である.
\end{corollary}

\subsection{連続affine関数の上限としての特徴付け}

\begin{tcolorbox}[colframe=ForestGreen, colback=ForestGreen!10!white,breakable,colbacktitle=ForestGreen!40!white,coltitle=black,fonttitle=\bfseries\sffamily,
title=閉凸関数への注目]
    任意の下半連続関数は,ある連続関数の族の上限として表現できる\ref{prop-lower-semicontinuous-functions}.
    凸な下半連続関数は特に,連続なaffine関数を選べ,これが特徴付けになることを示す.
    以降,凸な下半連続関数を\textbf{閉凸関数}という.これは,定義関数が閉凸関数であることと,そのエピグラフが集合として閉凸であることとが同値であることから.
\end{tcolorbox}

\begin{definition}[continuous affine function]
    ある連続線型汎関数$l\in V^*$と実数$\al$を用いて$l+\al$と表せる関数を\textbf{連続affine関数}という.
    連続affine関数の像の各点上限として得られる関数全体を$\Gamma(V)$で表し,$\Gamma_0(V):=\Gamma(V)\setminus\{\pm\infty\}$とする.
\end{definition}

\begin{proposition}
    $F:V\to\oR$について,次の2条件は同値.
    \begin{enumerate}
        \item $F\in\Gamma(V)$.
        \item $F$は凸な下半連続関数であり,$-\infty$を取るならば恒等関数である.
    \end{enumerate}
\end{proposition}

\subsection{$\Gamma$-正則化}

\begin{tcolorbox}[colframe=ForestGreen, colback=ForestGreen!10!white,breakable,colbacktitle=ForestGreen!40!white,coltitle=black,fonttitle=\bfseries\sffamily,
title=]
    $C^{1/2}(X)$内での下限を正則化というのであった.$\Gamma(X)$はそのうち凸でもある関数のなす部分空間である.
    $\Gamma(X)$での下限は$C^{1/2}(X)$での下限より小さくなる.
    実はこれが双極定理に繋がる.
\end{tcolorbox}

\begin{definition}[$\Gamma$-regularization]
    $F,G:V\to\oR$について,次の同値な条件を満たすとき,$G$を$F$の\textbf{$\Gamma$-正則化}であるという.
    \begin{enumerate}
        \item $G$は,$F$より小さい連続affine関数全体の各点上限関数である.
        \item $G$は$\Gamma(V)$における$F$の下限である.
    \end{enumerate}
\end{definition}

\begin{proposition}[$\Gamma$-正則化のエピグラフ]\label{prop-epigraph-of-gamma-regularization}
    $F:V\to\oR$を関数,$G$をその$\Gamma$-正則化とする.このとき,$F$より小さい連続affine関数が存在するならば,$\Epi G=\o{\Conv\Epi F}$が成り立つ.
\end{proposition}

\begin{proposition}
    $F:V\to\oR$を関数,$G$をその$\Gamma$-正則化とする.
    \begin{enumerate}
        \item $G\le\o{F}\le F$である.
        \item $F$が凸で,連続affine関数によって下から抑えられるとき,$\o{F}=G$.
    \end{enumerate}
\end{proposition}

\section{極関数と微分論}

\begin{notation}
    双線型写像$\brac{-,-}$が定めるペアリングを$(V,V^*)$で表す.それぞれに$\sigma(V,V^*),\sigma(V^*,V)$-位相を考えると,これらはそれぞれ局所凸位相線形空間である.
    ここで,$l:V\to\oR$が連続な線型汎関数であるといったとき,これはある$u^*\in V^*$を用いて$l=\brac{-,u^*}$なる表現を持つ.
\end{notation}

\subsection{極関数}

\begin{tcolorbox}[colframe=ForestGreen, colback=ForestGreen!10!white,breakable,colbacktitle=ForestGreen!40!white,coltitle=black,fonttitle=\bfseries\sffamily,
title=]
    なんだか極めて外積分の議論に似ている.作用素の議論のパターンだろうか.
\end{tcolorbox}

\begin{definition}[polar function / conjugate function]
    関数$F:V\to\oR$の\textbf{極関数}または\textbf{共役関数}とは,
    \[F^*(u^*):=\sup_{u\in V}\Brace{\brac{u,u^*}-F(u)}\]
    によって定まる関数$F^*:V^*\to\oR$をいう.
\end{definition}
\begin{remarks}[極関数は$\Gamma$-正則化の定数項を与える]
    $u^*\in V^*,\al\in\R$に関して,$u\mapsto\brac{u,u^*}-\al$は,連続affine関数で,
    至る所$F$より小さいことと,条件
    \[\forall_{u\in V}\;\al\ge\brac{u,u^*}-F(u)\quad\Leftrightarrow\quad\al\ge F^*(u^*)\]
    とは同値.よって,$u^*\in V^*$が定める連続affine関数で$F$より小さいもののうち最大のものの定数項が$-F^*(u^*)$で,すなわち,関数$F$の連続affine関数による下限は$u\mapsto\brac{u,u^*}-F^*(u^*)$である.
\end{remarks}

\begin{lemma}
    極関数は$F^*\in\Gamma(V^*)$を満たす.特に,$F^*$は凸な下半連続関数である.
\end{lemma}
\begin{proof}
    $F^*$は連続affine関数族$(\brac{u,-}-F(u))_{u\in\Dom F}$の上限であるため.
\end{proof}

\begin{corollary}\mbox{}
    \begin{enumerate}
        \item $F^*(0)=-\inf_{u\in V}F(u)$.
        \item $F\le G\Rightarrow F^*\ge G^*$.
        \item $(\inf_{i\in I}F_i)^*=\sup_{i\in I}F^*_i$.
        \item $(\sup_{i\in I}F_i)^*\le\inf_{i\in I}F^*_i$.
        \item $\forall_{\{F_i\}_{i\in I}\subset\Map(V,\R)}\;\forall_{\lambda>0}\;(\lambda F)^*(u^*)=\lambda F^*(u^*/\lambda)$.
        \item $\forall_{\{F_i\}_{i\in I}\subset\Map(V,\R)}\;\forall_{\al\in\R}\;(F+\al)^*=F^*-\al$.
        \item $F_a(v):=F(v-a)\;(a\in V)$を平行移動とする.$(F_a)^*(u^*)=F^*(u^*)+\brac{a,u^*}$.
    \end{enumerate}
\end{corollary}

\subsection{双極}

\begin{tcolorbox}[colframe=ForestGreen, colback=ForestGreen!10!white,breakable,colbacktitle=ForestGreen!40!white,coltitle=black,fonttitle=\bfseries\sffamily,
title=]
    極めてきれいな理論が出来上がった.
\end{tcolorbox}

\begin{proposition}[bipolar theorem]
    $F:V\to\oR$を関数とする.このとき,$F^{**}$は$F$の$\Gamma$-正則化である.
    特に,$\forall_{F\in\Gamma(V)}\;F^{**}=F$.
\end{proposition}

\begin{corollary}
    $\forall_{F\in\Map(V,\oR)}\;F^*=F^{***}$.
\end{corollary}

\begin{definition}[duality]
    極化の操作は2つの束$\Gamma(V)$と$\Gamma(V^*)$の全単射を引き起こす.
    この全単射によって対応する元,$F\in\Gamma(V),G\in\Gamma(V^*)$であって$F=G^*,G=F^*$を満たすものを,互いに双対であるという.
\end{definition}
\begin{remark}
    $\pm\infty$は$\pm\infty$に対応するから,この全単射の$\Gamma_0(V)$への制限はそのまま$\Gamma_0(V^*)$への全単射である.
\end{remark}

\subsection{極集合への応用}

\begin{tcolorbox}[colframe=ForestGreen, colback=ForestGreen!10!white,breakable,colbacktitle=ForestGreen!40!white,coltitle=black,fonttitle=\bfseries\sffamily,
title=]
    極関数の結果が整備できたいま,特性関数について特殊化させれば,極化集合論である.
    凸解析もマジできれいな理論で,関数解析に応用可能だな.
\end{tcolorbox}

\begin{example}[indicator function, support function]
    $A\subset V$を部分集合とし,$\chi_A$をその標示関数とする.
    その極関数は\textbf{$A$の支持関数}と呼ばれ,
    \[\chi_A^*(u^*)=\sup_{u\in V}\Brace{\brac{u,u^*}-\chi_a(u)}=\sup_{u\in A}\brac{u,u^*}.\]
    これは凸な下半連続関数で,また$V^*$上正斉次性を持つ.
    また,$\Gamma$-正則化の表現より,$\chi_A^{**}=\chi_{\o{\Conv}A}$.
    特に,$A$と$\o{\Conv}A$は同じ支持関数を持つ.
\end{example}

\begin{example}
    ノルム空間$(V,\norm{-})$とその双対空間$(V^*,\norm{-}^*)$について,$\varphi\in\Gamma_0(\R)$を偶関数とし,$\varphi^*\in\Gamma_0(\R)$をその極関数とする.
    このとき,
    \[F(u):=\varphi(\norm{u}),\quad G(u^*):=\varphi^*(\norm{u^*}^*)\]
    と定めると,これらは双対である.
\end{example}

\subsection{劣微分可能性}

\begin{tcolorbox}[colframe=ForestGreen, colback=ForestGreen!10!white,breakable,colbacktitle=ForestGreen!40!white,coltitle=black,fonttitle=\bfseries\sffamily,
title=]
    劣微分の概念は,通常の微分の定義(これはGateaux微分が引き継ぐ)より直観的に「affine関数による近似」として定まる.
    そして通常の微分概念とは違って常に存在するので,応用可能性が高い.
    したがって,極値問題を劣微分の言葉で捉え直すのである:
    \[F(u)=\min_{v\in V}F(v)\quad\Leftrightarrow\quad 0\in\partial F.\]
\end{tcolorbox}

\begin{definition}[exact]
    写像$F:V\to\oR$と,至る所$F$以下な連続affine関数$l:V\to\R,\forall_{u\in V}\;l(u)\le F(u)$に関して,
    \begin{enumerate}
        \item $l$は点$u\in V$において\textbf{完全}であるとは,$l(u)=F(u)$を満たすことをいう.
        \item $u$で完全なaffine関数$l:V\to\oR$は,ある$u^*\in V^*$を用いて,$l(v)=\brac{v-u,u^*}+F(u)=\brac{v,u^*}+F(u)-\brac{u,u^*}$なる表示を持つ必要がある.
        \item この完全な$l$は$F$を超えないものの中で極大である.したがって,定数項は極大であるから極関数$F^*$を用いて:$F(u)-\brac{u,u^*}=-F^*(u^*)$.
        \item $F$が$u\in V$において\textbf{劣微分可能}であるとは,$u$において完全な連続affine下限$l$が存在することをいう.
        \item $l$の傾き$u^*\in V^*$を$F$の$u$における\textbf{劣勾配}といい,劣勾配全体の集合を\textbf{劣微分}といい,$\partial F(u)$で表す.$\partial F(u)=\emptyset$と劣微分不可能であることとは同値.
    \end{enumerate}
\end{definition}

\begin{lemma}[劣勾配の特徴付け]\mbox{}
    \begin{enumerate}
        \item $u^*\in\partial F(u)$.
        \item $F(u)<\infty$かつ$\forall_{v\in V}\;\brac{v-u,u^*}+F(u)\le F(v)$.
        \item $F(u)+F^*(u^*)=\brac{u,u^*}$.
    \end{enumerate}
\end{lemma}

\begin{corollary}[劣微分は弱閉凸集合]
    劣微分$\partial F(u)\subset V^*$は,$\sigma(V^*,V)$-閉な凸集合である.
\end{corollary}

\begin{corollary}
    \[\forall_{F\in\Map(V,\R)}\;u^*\in\partial F(u)\Rightarrow u\in\partial F^*(u^*).\]
    $F\in\Gamma(V)$のとき,$\Leftarrow$も成り立つ.
\end{corollary}

\begin{lemma}[劣微分可能性の必要条件]
    $u\in V$について,
    \begin{enumerate}
        \item $\partial F(u)\ne\emptyset\Rightarrow F(u)=F^{**}(u)=l(u)$.
        \item $F(u)=F^{**}(u)\Rightarrow\partial F(u)=\partial F^{**}(u)$.
    \end{enumerate}
\end{lemma}

\begin{proposition}[凸関数の劣微分可能性の特徴付け]
    $F:V\to\R$を有限な凸関数とする.ある点$u\in V$において連続ならば,$\forall_{v\in(\Dom F)^\circ}\;\partial F(v)\ne\emptyset$.
\end{proposition}
\begin{remark}
    Banach空間上の下半連続な真凸関数は,$(\Dom F)^\circ$の稠密部分集合において劣微分可能である\ref{cor-subdifferentiable-points-of-Banach-operator}.
\end{remark}

\subsection{Gateaux微分可能性}

\begin{tcolorbox}[colframe=ForestGreen, colback=ForestGreen!10!white,breakable,colbacktitle=ForestGreen!40!white,coltitle=black,fonttitle=\bfseries\sffamily,
title=]
    凸関数において,劣微分可能性は通常の微分可能性の自然な一般化にもなっていることを見る.
    Gateaux微分のことばで,初等的に学ぶ極値問題が,無限次元にも通用する一般な形で,凸解析に応用が出来た.
\end{tcolorbox}

\begin{definition}
    ここでは,$DP(f)h\in\o{\R}$が存在する時,方向微分可能といい,これが第二変数について線型になるとき$\exists_{u^*\in V^*}\;\forall_{v\in V}\;DP(u)v=\brac{v,u^*}$,Gateaux微分可能であると言い分けることとする.
    そして,このときの$u^*=:F'(u)$をGateaux微分係数という.
\end{definition}

\begin{proposition}[Gateaux微分と劣微分の関係]
    $F:V\to\oR$を凸関数とする.
    \begin{enumerate}
        \item $F$が$u\in V$にてGateaux微分可能ならば,それは劣微分可能で,$\partial F(u)=\Brace{F'(u)}$が成り立つ.
        \item $F$が$u\in V$にて連続かつ有限で,一点集合の劣微分を持つとき,$F$はGateaux微分可能で,$\partial F(u)=\Brace{F'(u)}$である.
    \end{enumerate}
\end{proposition}

\begin{proposition}[Gateaux微分可能な関数が凸であることの特徴付け]
    $F$を凸関数$A\subset V$上のGateaux微分可能な実関数とする.このとき,次の2条件は同値.
    \begin{enumerate}
        \item $F$は$A$上凸である.
        \item $\forall_{u,v\in A}\;F(v)\ge F(u)+\brac{F'(u),v-u}$.
        \item Gateaux導関数$F':V\to V^*$は単調である:$\forall_{u_1,u_2\in V}\;\brac{u_1-u_2,F'(u_1)-F'(u_2)}\ge0$.\footnote{「変曲点がない」ことを見事に捉えている.}
    \end{enumerate}
\end{proposition}

\subsection{劣微分演算}

\begin{tcolorbox}[colframe=ForestGreen, colback=ForestGreen!10!white,breakable,colbacktitle=ForestGreen!40!white,coltitle=black,fonttitle=\bfseries\sffamily,
title=]
    劣微分とは,導関数を集合値関数として拡張する.
    値が一点集合に退化しているときが,通常の微分概念だと理解するのである.
\end{tcolorbox}

\begin{lemma}
    $F:V\to\oR,\lambda>0$について,
    $\partial:\Map(V,\o{\R})\to\Map(V,P(V^*))$は次を満たす:
    \begin{enumerate}
        \item 線形性:$\forall_{u\in V}\;\partial(\lambda F)(u)=\lambda \partial F(u)$.
        \item 優加法性:$\forall_{u\in V}\;\partial(F_1+F_2)(u)\supset\partial F_1(u)+\partial F_2(u)$.
    \end{enumerate}
\end{lemma}

\begin{proposition}[等号成立十分条件]
    $F_1,F_2\in\Gamma(V)$かつある点$\o{u}\in\Dom F_1\cap\Dom F_2$において$F_1$が連続であるとき,$\forall_{u\in V}\;\partial(F_1+F_2)(u)=\partial F_1(u)+\partial F_2(u)$.
\end{proposition}

\begin{proposition}
    局所凸空間$V,Y$と連続線型作用素$\Lambda:V\to Y$と関数$F\in\Gamma(Y)$について,$F$はある点$\Lambda\o{u}$において連続かつ有限であるとする.
    このとき,$\forall_{u\in V}\;\partial(F\circ\Lambda)(u)=\Lambda^*\partial F(\Lambda u)$.
\end{proposition}

\subsection{}

\begin{notation}
    $V$をBanach空間とする.
    $C(m):=\Brace{(u,a)\in V\times\R\mid a+m\norm{u}\le 0}$と定めると,これは非空な内部を持つ閉凸錐である.
    これが$V\times\R$上に定める関係$(u,a)\le(v,b):\Leftrightarrow(v-u,b-a)\in C(m)$は順序関係を定め,この順序について$C(m)$は正錐となる.
\end{notation}

\begin{proposition}
    部分集合$S\subset V\times\R$が$\inf\Brace{a\in\R\mid(u,a)\in S}>-\infty$を満たすならば,$S$は上述の順序における極大元を持つ.
\end{proposition}

\subsection{非凸関数への応用}

\begin{theorem}
    $F:V\to\oR$を下半連続関数で,$\inf F\in\R$を満たし,$\exists_{u\in V,\ep\in\R}\;F(u)\le\inf F+\ep$を満たすとする.
    このとき,任意の$\lambda>0$に対して,$u_\lambda\in V$が存在して,次の2条件を満たす
    \begin{enumerate}
        \item $\norm{u-u_\lambda}\le\lambda\land F(u_\lambda)\le F(u)$.
        \item $\Epi F\cap\Brace{(u_\lambda,F(u_\lambda))+C(\ep/\lambda)}=(u_\lambda,F(u_\lambda))$.
    \end{enumerate}
\end{theorem}

\subsection{凸関数への応用}

\begin{definition}
    \[\partial_\ep F(u):=\Brace{u^*\in V^*\mid 0\le F(u)+F^*(u^*)-\brac{u,u^*}\le\ep}\]
    を$F$の$u\in V$における\textbf{$\ep$-劣微分}という.
\end{definition}

\begin{theorem}
    
\end{theorem}

\begin{corollary}\label{cor-subdifferentiable-points-of-Banach-operator}
    $V$をBanach空間,$F\in\Gamma_0(V)$とする.$F$が劣微分可能な点の集合は$\Dom F$上稠密である.
\end{corollary}

\section{凸関数の最小化と変分不等式}

\begin{tcolorbox}[colframe=ForestGreen, colback=ForestGreen!10!white,breakable,colbacktitle=ForestGreen!40!white,coltitle=black,fonttitle=\bfseries\sffamily,
title=]
    極値問題のうち,汎関数の最小化の問題を変分問題という.
    古来から物理法則の多くは変分問題の形で定式化出来る.
    一般に最適化問題は有限次元の場合,すなわちパラメトリックな場合を指す.
\end{tcolorbox}

\begin{problem}
    回帰的Banach空間$V$の空でない閉凸集合$C\subset V$上の閉真凸関数(下半連続な真凸関数)$F:C\to\R$を考える.
    $F(u)=\inf_{v\in C}F(v)$を満たす元$u\in C$を\textbf{解}と呼ぶ.
    一般に,$V\setminus C$上では$+\infty$を取るとして延長した閉凸関数\footnote{エピグラフは変わらないので閉凸なままである.}$\wh{F}:V\to\oR$の$V$上での最小化を考えればよい.
\end{problem}

\subsection{最小値の存在}

\begin{proposition}
    開集合は$C$の閉凸集合である(空集合含む).
\end{proposition}
\begin{proof}
    開集合はある$\al\in\R$を用いて$\Brace{u\in V\mid\wh{F}(u)\le\al}$という等位集合の形で表せるので\ref{def-level-set}.
\end{proof}

\begin{proposition}
    次の(1)または(2)を満たすとき,解は存在する.(3)も満たすとき,解は一意である.
    \begin{enumerate}
        \item $C$は有界である.
        \item $F$は$C$上強圧的(coercive)である:$\forall_{u\in C}\;\norm{u}\to\infty\Rightarrow\lim F(u)=+\infty$.
        \item $F$は$C$上狭義凸である.
    \end{enumerate}
\end{proposition}

\subsection{解の特徴付け}

\begin{proposition}
    閉真凸関数$F:V\to\R$は連続なGateaux導関数$F'$を持つとする.$u\in C$について,次の3条件は同値.
    \begin{enumerate}
        \item $u$は解である.
        \item $\forall_{v\in C}\;\brac{F'(u),v-u}\ge0$.
        \item $\forall_{v\in C}\;\brac{F'(v),v-u}$.
    \end{enumerate}
\end{proposition}

\begin{proposition}
    閉真凸関数$F_1,F_2:V\to\R$について,$F_1$はGateaux導関数$F'$を持つとし,$F:=F_1+F_2$も閉真凸とする.$u\in C$について,次の3条件は同値.
    \begin{enumerate}
        \item $u$は解である.
        \item $\forall_{v\in C}\;\brac{F'_1(u),v-u}+F_2(v)-F_2(u)\ge0$.
        \item $\forall_{v\in C}\;\brac{F'_1(v),v-u}+F_2(v)-F_2(u)\ge0$.
    \end{enumerate}
\end{proposition}

\subsection{変分不等式}

\begin{problem}
    今度は,回帰的Banach空間$V$において,作用素$A:V\to V^*$と真の閉凸汎関数$\varphi:V\to\oR$を考える.
    任意の$f\in V^*$に対して,
    \[\forall_{v\in V}\quad\brac{Au-f,v-u}+\varphi(v)-\varphi(u)\ge0\]
    を満たす元$u\in V$を求めることを考える.
\end{problem}

\begin{theorem}
    $A,\varphi$はさらに次を満たすとする.
    \begin{enumerate}
        \item $V$の任意の有限次元部分空間上において$A$は弱連続.
        \item $A$は単調:$\forall_{u,v\in V}\;\brac{Au-Av,u-v}\ge0$.
        \item $\exists_{v_0\in\Dom\varphi}\;\norm{v}\to\infty\Rightarrow\frac{\brac{Av,v-v_0}+\varphi(v)}{\norm{v}}\to\infty$.
    \end{enumerate}
    このとき,任意の$f\in V^*$に対して,変分不等式の解$u\in V$は存在する.
\end{theorem}

\section{凸最適化の双対理論}

\begin{tcolorbox}[colframe=ForestGreen, colback=ForestGreen!10!white,breakable,colbacktitle=ForestGreen!40!white,coltitle=black,fonttitle=\bfseries\sffamily,
title=]
    双対理論はRochafellerによる.
    Fenchelに始まり,Rochafellerは共役関数の方法で双対理論を構築した.
    一方で,数理経済学で主流なのはLagrangianによるものである.Hurwicz and 宇沢など.
\end{tcolorbox}

\begin{notation}
    $X$を実ノルム空間とし,連続な双線型写像$\brac{-,-}:X\times X^*\to\R$を評価写像として定める.
    一般の双線型形式が定めるペアリングについて成り立つ.
\end{notation}

\subsection{双対理論一般}

\begin{definition}[primal problem, dual problem]
    $F:V\to\oR$に対して,
    $\inf_{u\in V}F(u)=\inf P\in\R$を主問題$P$とする.これに対して
    \[\sup\brac{u^*,u_0}=\beta,\quad u^*\in L^\perp,\norm{u^*}\le1\]
    を双対問題という.
\end{definition}

\begin{theorem}
    $L$を実ノルム空間$X$の部分空間とする.$u_0\in X$とする.次が成り立つ.
    \begin{enumerate}
        \item $\al=\beta$.
        \item 双対問題は解$u^*$を持つ.
        \item $u\in L$が主問題の解であることと,$\brac{u^*,u_0-u}=\norm{u-u_0}$が成り立つことは同値.
    \end{enumerate}
\end{theorem}

\begin{corollary}
    $L$が有限次元のとき,主問題は解を持つ.
\end{corollary}

\begin{remark}
    $L$は無限次元だが,回帰的Banach空間$X$の閉部分空間であるならば,主問題は解を持つ.
\end{remark}

\subsection{修正理論一般}

\begin{tcolorbox}[colframe=ForestGreen, colback=ForestGreen!10!white,breakable,colbacktitle=ForestGreen!40!white,coltitle=black,fonttitle=\bfseries\sffamily,
title=]
    minimum norm problemを$X^*$上で考える道もある.
\end{tcolorbox}

\begin{definition}[modified primal problem]
    $\inf\norm{u^*-u_0^*}=\al,\;u^*\in L^\perp$を\textbf{修正された主問題}という.
    これに対する双対問題は
    \[\sup\brac{u^*_0,u}=\beta,\quad u\in L,\norm{u}\le 1\]
    となる.
\end{definition}

\begin{theorem}
    $X$を実ノルム空間,$L$をその線型部分空間とする.$u_0^*\in X^*$とすると,次が成り立つ.
    \begin{enumerate}
        \item $\al=\beta$.
        \item 主問題は解$u^*$を持つ.
        \item $u\in L,\norm{u}\le 1$が双対問題の解であることは,$\brac{u_0^*-u^*,u}=\norm{u^*_0-u^*}$を満たすことに同値.
    \end{enumerate}
\end{theorem}

\subsection{応用:Cebysev近似}

\begin{tcolorbox}[colframe=ForestGreen, colback=ForestGreen!10!white,breakable,colbacktitle=ForestGreen!40!white,coltitle=black,fonttitle=\bfseries\sffamily,
title=]
    最適化理論はこのような数学的応用を持つ.
\end{tcolorbox}

\begin{definition}
    連続関数$u_0:[a,b]\to\R$と有限次元部分空間$L:=\Brace{P\in\R[x]\mid\deg P\le N}$
    に対して,近似問題
    \[\max_{a\le x\le b}\abs{u_0(x)-u(x)}=\min!,\quad u\in L\]
    を考える.
\end{definition}

\begin{proposition}
    上述の近似問題は解$u$を持ち,$\abs{u_0(x)-u(x)}$は$[a,b]$上で少なくとも$N+2$個の最大値を取る点を持つ.
\end{proposition}

\section{変分法}

\begin{tcolorbox}[colframe=ForestGreen, colback=ForestGreen!10!white,breakable,colbacktitle=ForestGreen!40!white,coltitle=black,fonttitle=\bfseries\sffamily,
title=]
    ノンパラメトリックな最適化法を議論しよう.
    言うならば,現代の統計学者が向き合っているのは時代の最先端の変分法である.
\end{tcolorbox}

\subsection{変分原理}

\begin{tcolorbox}[colframe=ForestGreen, colback=ForestGreen!10!white,breakable,colbacktitle=ForestGreen!40!white,coltitle=black,fonttitle=\bfseries\sffamily,
title=]
    無限次元空間では,ほとんどの集合がコンパクトにならないことが,変分問題の難しい点である.
    そこで,回帰的Banach空間では(特にHilbert空間では),有界列は弱点列コンパクトである\ref{thm-characterization-of-reflexive-Banach-spaces}ことに注目する.
\end{tcolorbox}

\begin{theorem}[generalized Weierstrass theorem]
    回帰的Banach空間$X$の単位閉球$B$上の
    最小化問題$F(u)=\min!,u\in B$は,汎関数$F:B\to\R$が,弱位相について点列下半連続であるとき,解を持つ.

    このことは,単位閉球$B$上だけでなく,任意の非空な有界閉凸集合$M\subset X$上で成り立つ.
\end{theorem}

\begin{remark}
    変分問題で現れる汎関数は$C^1$の$1$-ノルム$\norm{y}_1=\max_{x\in[a,b]}\abs{y(x)}+\max_{x\in[a,b]}\abs{y'(x)}$では連続だが,$C$の一様ノルムでは連続ではないことが多い.
\end{remark}

\subsection{変分法の基本補題}

\begin{tcolorbox}[colframe=ForestGreen, colback=ForestGreen!10!white,breakable,colbacktitle=ForestGreen!40!white,coltitle=black,fonttitle=\bfseries\sffamily,
title=]
    $C([a,b])$においては,$\Brace{h\in C([a,b])\mid h(a)=h(b)=0}$が「基底」のようなもので,この関数との内積がすべて$0$であるならば,$\al=0$に限る.
\end{tcolorbox}

\begin{lemma}
    $\al\in C([a,b])$について,
    \[\forall_{h\in C([a,b])}\;h(a)=h(b)=0\Rightarrow\int^b_a\al(x)h(x)dx=0\Rightarrow\al=0.\]
\end{lemma}

\begin{lemma}
    $\al\in C([a,b])$について,任意の$h(a)=h(b)=0$を満たす$C^1$級関数$h\in C^1([a,b])$に対して
    $\int^b_a\al(x)h'(x)dx=0$を満たすならば,$\al$は定数関数である.
\end{lemma}

\begin{lemma}[部分積分の公式の逆]
    $\al,\beta\in C([a,b])$が,任意の$h(a)=h(b)=0$を満たす$h\in C^1([a,b])$について
    \[\int^b_a\paren{\al(x)h(x)+\beta(x)h'(x)}=0\]
    を満たすならば,$\beta$は微分可能であり,$\al=\beta'$.
\end{lemma}

\subsection{$n$次変分}

\begin{tcolorbox}[colframe=ForestGreen, colback=ForestGreen!10!white,breakable,colbacktitle=ForestGreen!40!white,coltitle=black,fonttitle=\bfseries\sffamily,
title=]
    変分とは方向微分に他ならない.
\end{tcolorbox}

\begin{definition}[$n$-th variation]
    ノルム空間$X$上の点$u_0\in X$の開近傍$U$上で定義された汎関数$F:U\to\R$について,
    \begin{enumerate}
        \item $h\in X$について,$t=0\in\R$の近傍で定義された関数を$\phi_h(t):=F(u_0+th)$で表す.$\Delta F[h]$とも表す.
        \item $n$次変分$\delta^nF(u_0;h)$を,$\delta^nF(u_0;h):=\phi^{(n)}(0)$で定義する.
    \end{enumerate}
\end{definition}

\begin{lemma}[微分との関係]\mbox{}
    \begin{enumerate}
        \item $F$が$u_0\in X$においてGateaux微分可能であることと,任意の$h\in X$について変分$\delta F(u_0;h)$が存在して対応$\delta F(u_0;h)=F'(u_0)(h)$の$F'$が線型になることと同値.
        \item さらに$F(u_0+h)-F(u_0)=F'(u_0)(h)+o(\norm{h})$も成り立つとき,$F$は$u_0$においてFrechet微分可能であるという.
        \item 任意の$u\in U$においてFrechet微分が存在し,対応$F':U\to X^*$が連続になる時,$F$を$C^1$級という.
    \end{enumerate}
\end{lemma}

\begin{definition}\mbox{}
    \begin{enumerate}
        \item $F$が$u_0\in U$にて極小値を取るとは,ある開近傍$u_0\in V\subset U$が存在して,$\forall_{u\in V}\;F(u)\ge F(u_0)$を満たすことをいう.
        \item $u_0$にて臨界値を取るとは,$\forall_{h\in X}\;\delta F(u_0;h)=0\lor\delta F(u_0;h)$ is undefinedをいう.この条件は,$F$にGateaux導関数が存在するとき,$F'(u_0)=0$に同値.
    \end{enumerate}
\end{definition}

\subsection{極値の存在条件}

\begin{theorem}[極値の特徴付け]\mbox{}
    \begin{description}
        \item[必要条件] $F$が$u_0$にて極値を取るならば,$u_0$は臨界点である:任意の$h\in X$について,1次変分$\delta F(u_0;h)$が存在するならば$0$である.$F$のGateaux導関数$F'(u_0)$が存在するとき,同値な条件$F'(u_0)=0$はEuler方程式と呼ばれる.
        \item[十分条件] $F$が次の3条件を満たすとき,$u_0$にて極小値を取る.
        \begin{enumerate}
            \item (1次の最適性条件) $u_0$は$F$の臨界点である.
            \item (Legendre:2次の最適性条件) ある$u_0$の開近傍$V\subset U$について,任意の$u\in V$で2次変分$\delta^2F(u;h)$が存在し,$\exists_{c>0}\;\forall_{h\in X}\;\delta^2 F(u_0;h)\ge c\norm{h}^2$を満たす.
            \item (Jacobi) $\forall_{\ep>0}\;\exists_{\eta>0}\;\forall_{u,h\in X}\;\norm{u-u_0}<\eta\Rightarrow\abs{\delta^2F(u;h)-\delta^2F(u_0;h)}\le\ep\norm{h}^2$.
        \end{enumerate}
    \end{description}
\end{theorem}

\subsection{変分法の基本問題}

\begin{problem}
    $F\in C^2(\R^3)$に対して,空間
    \[Y(a,b):=\Brace{y\in C^1([a,b])\mid y(a)=A,y(b)=B}\]
    内で汎関数
    \[J[y]=\int^b_aF(x,y(x),y'(x))dx\]
    の$C^1$-極値点となるようなものを求めよ.
    $C$-極値は$C^1$-極値であることに注意.
\end{problem}

\begin{proposition}[基本問題の変分とEuler方程式]\mbox{}
    \begin{enumerate}
        \item \[\delta J=\int^b_a\paren{F_y(x,y,y')h+F_{y'}(x,y,y')h'}dx.\]
        \item $y$が$J[y]$の極値点であるための必要十分条件は,
        \[F_y-\dd{}{x}F_{y'}=0.\]
    \end{enumerate}
    (2)の方程式の積分曲線を\textbf{停留曲線}(extremals)といい,(2)の左辺を$F$の\textbf{変分導関数}という.
\end{proposition}

\begin{remark}
    $y\in C^1([a,b])$は2階導関数をもたないが,$C^1$-極値点になる,という状況がありえる.
\end{remark}

\subsection{双対理論}

\begin{tcolorbox}[colframe=ForestGreen, colback=ForestGreen!10!white,breakable,colbacktitle=ForestGreen!40!white,coltitle=black,fonttitle=\bfseries\sffamily,
title=]
    Friedrichs変換.
\end{tcolorbox}

\subsection{Ritz法}

\begin{definition}
    (連続な)目的汎関数$J(u)$の値域の下限$d:=\inf\Im J$について,許容関数の列$(\wh{u}_i)$で$J$での値が$d$にしゅうそくするものが存在する.
    このような関数列を,変分問題の極小列という.
    Euler方程式を導く解析的な手法ではなく,極小列を構成することで近似的に解く解法を\textbf{直接法}という.
\end{definition}

\begin{example}
    試験関数の空間$D$を決め,$D$の完全系$(\varphi_k)$を1つ取る.
    極小列の$m$番目を$u_m=c_0+c_1\varphi_1+\cdots+c_m\varphi_m$であって,係数$c_1,\cdots,c_m$は$J(u_m)$を最小にするものとすれば良い.これは有限次元の最適化問題に帰着されている.
\end{example}

\subsection{有限要素法}



\section{関数機械}

\begin{tcolorbox}[colframe=ForestGreen, colback=ForestGreen!10!white,breakable,colbacktitle=ForestGreen!40!white,coltitle=black,fonttitle=\bfseries\sffamily,
title=]
A function machine is a generalization of neural networks to potentially infinite dimensional layers, motivated by the study of universal approximation of operators and functionals over abstract Banach spaces.\footnote{\url{https://ncatlab.org/nlab/show/function+machine}}
\end{tcolorbox}

\begin{definition}[function machine, operator layer, functional layer, basis layer, fully-connected layer]
    $K\subset\R^d,K'\subset\R^{d'}$をコンパクト集合,$T$をaffine写像とする.
    \begin{enumerate}
        \item 作用素層とは,affine写像$T^o:\L^1(K,\mu)\to\L^1(K',\mu)$であって,測度の連続族$(W_t\ll\mu)_{t\in K'}$と関数$b\in\L^1(K,\mu)$が存在して,$T^0:f\mapsto\paren{t\mapsto\int_KfdW_t+b(t)}$と表せるものをいう.
    \end{enumerate}
\end{definition}

\section{ニューラルネットワーク}

\begin{tcolorbox}[colframe=ForestGreen, colback=ForestGreen!10!white,breakable,colbacktitle=ForestGreen!40!white,coltitle=black,fonttitle=\bfseries\sffamily,
title=]
A neural network is \textbf{a class of functions} used in both supervised and unsupervised machine learning to approximate a correspondence between samples in a dataset and their associated labels.\footnote{\url{https://ncatlab.org/nlab/show/neural+network}}
\end{tcolorbox}

\section{テンソルネットワーク}

\begin{tcolorbox}[colframe=ForestGreen, colback=ForestGreen!10!white,breakable,colbacktitle=ForestGreen!40!white,coltitle=black,fonttitle=\bfseries\sffamily,
title=]
    モノイド圏論におけるストリング図式と等価な概念(with an attitude)で,はじめは量子物理学で台頭した.
\end{tcolorbox}

\subsection{カーネル法}

\begin{tcolorbox}[colframe=ForestGreen, colback=ForestGreen!10!white,breakable,colbacktitle=ForestGreen!40!white,coltitle=black,fonttitle=\bfseries\sffamily,
title=]
    関数族ではなく,距離関数$d:D\times D\to\R$の指定によってデータ集合$D$を解析することをいう.
    $d$が定める積分核$\exp(-\lambda\cdot d(-,-))$を適切なクラスから選ぶことを考える.
\end{tcolorbox}

\begin{thebibliography}{99}
    \bibitem{AnalysisNow}
    Gert K. Pederson "Analysis Now"
    \bibitem{作用素環}
    生西明夫,中神祥臣『作用素環入門I』
    \bibitem{John Conway}
    John B. Conway "A Course in Functional Analysis"
    \bibitem{Brezis}
    Haïm Brezis 『関数解析』
    \bibitem{増田久弥}
    増田久弥『非線型数学』
    \bibitem{Kolmogorov}
    Kolmogorov, Fomin 『関数解析の基礎』
    \bibitem{藤田}
    藤田宏,黒田成俊,伊藤清三『関数解析』
    \bibitem{Eidelman}
    Eidelman, Yuli, Vitali D. Milman, and Antonis Tsolomitis. 2004. Functional analysis: an introduction. Providence (R.I.): American mathematical Society.
    \bibitem{Lindenstrauss}
    Joram Lindenstrauss, and Yoav Benyamini "Geometric Nonlinear Functional Analysis" Volume 1
    \bibitem{Ekeland and Temam}
    Ekeland and Temam -  Convex Analysis and Variational Problems
    \bibitem{Zeldler}
    Eberhard Zeldler "Applied Functional Analysis"
    \bibitem{Gelfand}
    I. M. Gelfand and S. V. Fomin "Calculus of Variations"
    \bibitem{Hytonen}
    Tuomas Hytonen, Jan van Neerven, Mark Veraar, Lutz Weis "Analysis in Banach Spaces"
    \bibitem{Rudin}
    Rudin "Functional Analysis"
\end{thebibliography}

\end{document}