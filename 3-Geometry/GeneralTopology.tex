\documentclass[uplatex,dvipdfmx]{jsreport}
\title{General Topology}
\author{司馬博文}
\date{\today}
\pagestyle{headings} \setcounter{secnumdepth}{4}
%\usepackage{mathtools}
%\mathtoolsset{showonlyrefs=true} %labelを附した数式にのみ附番される設定.
%\usepackage{amsmath} %mathtoolsの内部で呼ばれるので要らない.
\usepackage{amsfonts} %mathfrak, mathcal, mathbbなど.
\usepackage{amsthm} %定理環境.
\usepackage{amssymb} %AMSFontsを使うためのパッケージ.
\usepackage{ascmac} %screen, itembox, shadebox環境.全てLATEX2εの標準機能の範囲で作られたもの.
\usepackage{comment} %comment環境を用いて,複数行をcomment outできるようにするpackage
\usepackage{wrapfig} %図の周りに文字をwrapさせることができる.詳細な制御ができる.
\usepackage[usenames, dvipsnames]{xcolor} %xcolorはcolorの拡張.optionの意味はdvipsnamesはLoad a set of predefined colors. forestgreenなどの色が追加されている.usenamesはobsoleteとだけ書いてあった.
\setcounter{tocdepth}{2} %目次に表示される深さ.2はsubsectionまで
\usepackage{multicol} %\begin{multicols}{2}環境で途中からmulticolumnに出来る.

\usepackage{url}
\usepackage[dvipdfmx,colorlinks,linkcolor=blue,urlcolor=blue]{hyperref} %生成されるPDFファイルにおいて、\tableofcontentsによって書き出された目次をクリックすると該当する見出しへジャンプしたり、さらには、\label{ラベル名}を番号で参照する\ref{ラベル名}やthebibliography環境において\bibitem{ラベル名}を文献番号で参照する\cite{ラベル名}においても番号をクリックすると該当箇所にジャンプする.囲み枠はダサいので,colorlinksで囲み廃止し,リンク自体に色を付けることにした.
\usepackage{pxjahyper} %pxrubrica同様,八登崇之さん.hyperrefは日本語pLaTeXに最適化されていないから,hyperrefとセットで,(u)pLaTeX+hyperref+dvipdfmxの組み合わせで日本語を含む「しおり」をもつPDF文書を作成する場合に必要となる機能を提供する
\definecolor{花緑青}{cmyk}{0.52,0.03,0,0.27}
\definecolor{サーモンピンク}{cmyk}{0,0.65,0.65,0.05}
\definecolor{暗中模索}{rgb}{0.2,0.2,0.2}

\usepackage{tikz}
\usetikzlibrary{positioning,automata} %automaton描画のため
\usepackage{tikz-cd}
\usepackage[all]{xy}
\def\objectstyle{\displaystyle} %デフォルトではxymatrix中の数式が文中数式モードになるので,それを直す.\labelstyleも同様にxy packageの中で定義されており,文中数式モードになっている.

\usepackage[version=4]{mhchem} %化学式をTikZで簡単に書くためのパッケージ.
\usepackage{chemfig} %化学構造式をTikZで描くためのパッケージ.
\usepackage{siunitx} %IS単位を書くためのパッケージ

\usepackage{ulem} %取り消し線を引くためのパッケージ
\usepackage{pxrubrica} %日本語にルビをふる.八登崇之(やとうたかゆき)氏による.

\usepackage{graphicx} %rotatebox, scalebox, reflectbox, resizeboxなどのコマンドや,図表の読み込み\includegraphicsを司る.graphics というパッケージもありますが,graphicx はこれを高機能にしたものと考えて結構です(ただし graphicx は内部で graphics を読み込みます)

\usepackage[breakable]{tcolorbox} %加藤晃史さんがフル活用していたtcolorboxを,途中改ページ可能で.
\tcbuselibrary{theorems} %https://qiita.com/t_kemmochi/items/483b8fcdb5db8d1f5d5e
\usepackage{enumerate} %enumerate環境を凝らせる.
\usepackage[top=15truemm,bottom=15truemm,left=10truemm,right=10truemm]{geometry} %足助さんからもらったオプション

%%%%%%%%%%%%%%% 環境マクロ %%%%%%%%%%%%%%%

\usepackage{listings} %ソースコードを表示できる環境.多分もっといい方法ある.
\usepackage{jvlisting} %日本語のコメントアウトをする場合jlistingが必要
\lstset{ %ここからソースコードの表示に関する設定.lstlisting環境では,[caption=hoge,label=fuga]などのoptionを付けられる.
%[escapechar=!]とすると,LaTeXコマンドを使える.
  basicstyle={\ttfamily},
  identifierstyle={\small},
  commentstyle={\smallitshape},
  keywordstyle={\small\bfseries},
  ndkeywordstyle={\small},
  stringstyle={\small\ttfamily},
  frame={tb},
  breaklines=true,
  columns=[l]{fullflexible},
  numbers=left,
  xrightmargin=0zw,
  xleftmargin=3zw,
  numberstyle={\scriptsize},
  stepnumber=1,
  numbersep=1zw,
  lineskip=-0.5ex
}
%\makeatletter %caption番号を「[chapter番号].[section番号].[subsection番号]-[そのsubsection内においてn番目]」に変更
%    \AtBeginDocument{
%    \renewcommand*{\thelstlisting}{\arabic{chapter}.\arabic{section}.\arabic{lstlisting}}
%    \@addtoreset{lstlisting}{section}
%    }
%\makeatother
\renewcommand{\lstlistingname}{算譜} %caption名を"program"に変更

\newtcolorbox{tbox}[3][]{%
colframe=#2,colback=#2!10,coltitle=#2!20!black,title={#3},#1}

%%%%%%%%%%%%%%% フォント %%%%%%%%%%%%%%%

\usepackage{textcomp, mathcomp} %Text Companionとは,T1 encodingに入らなかった文字群.これを使うためのパッケージ.\textsectionでブルバキに!
\usepackage[T1]{fontenc} %8bitエンコーディングにする.comp系拡張数学文字の動作が安定する.

%%%%%%%%%%%%%%% 数学記号のマクロ %%%%%%%%%%%%%%%

\newcommand{\abs}[1]{\lvert#1\rvert} %mathtoolsはこうやって使うのか!
\newcommand{\Abs}[1]{\left|#1\right|}
\newcommand{\norm}[1]{\|#1\|}
\newcommand{\Norm}[1]{\left\|#1\right\|}
%\newcommand{\brace}[1]{\{#1\}}
\newcommand{\Brace}[1]{\left\{#1\right\}}
\newcommand{\paren}[1]{\left(#1\right)}
\newcommand{\bracket}[1]{\langle#1\rangle}
\newcommand{\brac}[1]{\langle#1\rangle}
\newcommand{\Bracket}[1]{\left\langle#1\right\rangle}
\newcommand{\Brac}[1]{\left\langle#1\right\rangle}
\newcommand{\Square}[1]{\left[#1\right]}
\renewcommand{\o}[1]{\overline{#1}}
\renewcommand{\u}[1]{\underline{#1}}
\renewcommand{\iff}{\;\mathrm{iff}\;} %nLabリスペクト
\newcommand{\pp}[2]{\frac{\partial #1}{\partial #2}}
\newcommand{\ppp}[3]{\frac{\partial #1}{\partial #2\partial #3}}
\newcommand{\dd}[2]{\frac{d #1}{d #2}}
\newcommand{\floor}[1]{\lfloor#1\rfloor}
\newcommand{\Floor}[1]{\left\lfloor#1\right\rfloor}
\newcommand{\ceil}[1]{\lceil#1\rceil}

\newcommand{\iso}{\xrightarrow{\,\smash{\raisebox{-0.45ex}{\ensuremath{\scriptstyle\sim}}}\,}}
\newcommand{\wt}[1]{\widetilde{#1}}
\newcommand{\wh}[1]{\widehat{#1}}

\newcommand{\Lrarrow}{\;\;\Leftrightarrow\;\;}

%ノルム位相についての閉包 https://newbedev.com/how-to-make-double-overline-with-less-vertical-displacement
\makeatletter
\newcommand{\dbloverline}[1]{\overline{\dbl@overline{#1}}}
\newcommand{\dbl@overline}[1]{\mathpalette\dbl@@overline{#1}}
\newcommand{\dbl@@overline}[2]{%
  \begingroup
  \sbox\z@{$\m@th#1\overline{#2}$}%
  \ht\z@=\dimexpr\ht\z@-2\dbl@adjust{#1}\relax
  \box\z@
  \ifx#1\scriptstyle\kern-\scriptspace\else
  \ifx#1\scriptscriptstyle\kern-\scriptspace\fi\fi
  \endgroup
}
\newcommand{\dbl@adjust}[1]{%
  \fontdimen8
  \ifx#1\displaystyle\textfont\else
  \ifx#1\textstyle\textfont\else
  \ifx#1\scriptstyle\scriptfont\else
  \scriptscriptfont\fi\fi\fi 3
}
\makeatother
\newcommand{\oo}[1]{\dbloverline{#1}}

\DeclareMathOperator{\grad}{\mathrm{grad}}
\DeclareMathOperator{\rot}{\mathrm{rot}}
\DeclareMathOperator{\divergence}{\mathrm{div}}
\newcommand{\False}{\mathrm{False}}
\newcommand{\True}{\mathrm{True}}
\DeclareMathOperator{\tr}{\mathrm{tr}}
\newcommand{\M}{\mathcal{M}}
\newcommand{\cF}{\mathcal{F}}
\newcommand{\cD}{\mathcal{D}}
\newcommand{\fX}{\mathfrak{X}}
\newcommand{\fY}{\mathfrak{Y}}
\newcommand{\fZ}{\mathfrak{Z}}
\renewcommand{\H}{\mathcal{H}}
\newcommand{\fH}{\mathfrak{H}}
\newcommand{\bH}{\mathbb{H}}
\newcommand{\id}{\mathrm{id}}
\newcommand{\A}{\mathcal{A}}
% \renewcommand\coprod{\rotatebox[origin=c]{180}{$\prod$}} すでにどこかにある.
\newcommand{\pr}{\mathrm{pr}}
\newcommand{\U}{\mathfrak{U}}
\newcommand{\Map}{\mathrm{Map}}
\newcommand{\dom}{\mathrm{Dom}\;}
\newcommand{\cod}{\mathrm{Cod}\;}
\newcommand{\supp}{\mathrm{supp}\;}
\newcommand{\otherwise}{\mathrm{otherwise}}
\newcommand{\st}{\;\mathrm{s.t.}\;}
\newcommand{\lmd}{\lambda}
\newcommand{\Lmd}{\Lambda}
%%% 線型代数学
\newcommand{\Ker}{\mathrm{Ker}\;}
\newcommand{\Coker}{\mathrm{Coker}\;}
\newcommand{\Coim}{\mathrm{Coim}\;}
\newcommand{\rank}{\mathrm{rank}}
\newcommand{\lcm}{\mathrm{lcm}}
\newcommand{\sgn}{\mathrm{sgn}}
\newcommand{\GL}{\mathrm{GL}}
\newcommand{\SL}{\mathrm{SL}}
\newcommand{\alt}{\mathrm{alt}}
%%% 複素解析学
\renewcommand{\Re}{\mathrm{Re}\;}
\renewcommand{\Im}{\mathrm{Im}\;}
\newcommand{\Gal}{\mathrm{Gal}}
\newcommand{\PGL}{\mathrm{PGL}}
\newcommand{\PSL}{\mathrm{PSL}}
\newcommand{\Log}{\mathrm{Log}\,}
\newcommand{\Res}{\mathrm{Res}\,}
\newcommand{\on}{\mathrm{on}\;}
\newcommand{\hatC}{\hat{\C}}
\newcommand{\hatR}{\hat{\R}}
\newcommand{\PV}{\mathrm{P.V.}}
\newcommand{\diam}{\mathrm{diam}}
\newcommand{\Area}{\mathrm{Area}}
\newcommand{\Lap}{\Laplace}
\newcommand{\f}{\mathbf{f}}
\newcommand{\cR}{\mathcal{R}}
\newcommand{\const}{\mathrm{const.}}
\newcommand{\Om}{\Omega}
\newcommand{\Cinf}{C^\infty}
\newcommand{\ep}{\epsilon}
\newcommand{\dist}{\mathrm{dist}}
\newcommand{\opart}{\o{\partial}}
%%% 解析力学
\newcommand{\x}{\mathbf{x}}
%%% 集合と位相
\renewcommand{\O}{\mathcal{O}}
\renewcommand{\S}{\mathcal{S}}
\renewcommand{\U}{\mathcal{U}}
\newcommand{\V}{\mathcal{V}}
\renewcommand{\P}{\mathcal{P}}
\newcommand{\R}{\mathbb{R}}
\newcommand{\N}{\mathbb{N}}
\newcommand{\C}{\mathbb{C}}
\newcommand{\Z}{\mathbb{Z}}
\newcommand{\Q}{\mathbb{Q}}
\newcommand{\TV}{\mathrm{TV}}
\newcommand{\ORD}{\mathrm{ORD}}
\newcommand{\Tr}{\mathrm{Tr}\;}
\newcommand{\Card}{\mathrm{Card}\;}
\newcommand{\Top}{\mathrm{Top}}
\newcommand{\Disc}{\mathrm{Disc}}
\newcommand{\Codisc}{\mathrm{Codisc}}
\newcommand{\CoDisc}{\mathrm{CoDisc}}
\newcommand{\Ult}{\mathrm{Ult}}
\newcommand{\ord}{\mathrm{ord}}
\newcommand{\maj}{\mathrm{maj}}
%%% 形式言語理論
\newcommand{\REGEX}{\mathrm{REGEX}}
\newcommand{\RE}{\mathbf{RE}}

%%% Fourier解析
\newcommand*{\Laplace}{\mathop{}\!\mathbin\bigtriangleup}
\newcommand*{\DAlambert}{\mathop{}\!\mathbin\Box}
%%% Graph Theory
\newcommand{\SimpGph}{\mathrm{SimpGph}}
\newcommand{\Gph}{\mathrm{Gph}}
\newcommand{\mult}{\mathrm{mult}}
\newcommand{\inv}{\mathrm{inv}}
%%% 多様体
\newcommand{\Der}{\mathrm{Der}}
\newcommand{\osub}{\overset{\mathrm{open}}{\subset}}
\newcommand{\osup}{\overset{\mathrm{open}}{\supset}}
\newcommand{\al}{\alpha}
\newcommand{\K}{\mathbb{K}}
\newcommand{\Sp}{\mathrm{Sp}}
\newcommand{\g}{\mathfrak{g}}
\newcommand{\h}{\mathfrak{h}}
\newcommand{\Exp}{\mathrm{Exp}\;}
\newcommand{\Imm}{\mathrm{Imm}}
\newcommand{\Imb}{\mathrm{Imb}}
\newcommand{\codim}{\mathrm{codim}\;}
\newcommand{\Gr}{\mathrm{Gr}}
%%% 代数
\newcommand{\Ad}{\mathrm{Ad}}
\newcommand{\finsupp}{\mathrm{fin\;supp}}
\newcommand{\SO}{\mathrm{SO}}
\newcommand{\SU}{\mathrm{SU}}
\newcommand{\acts}{\curvearrowright}
\newcommand{\mono}{\hookrightarrow}
\newcommand{\epi}{\twoheadrightarrow}
\newcommand{\Stab}{\mathrm{Stab}}
\newcommand{\nor}{\mathrm{nor}}
\newcommand{\T}{\mathbb{T}}
\newcommand{\Aff}{\mathrm{Aff}}
\newcommand{\rsub}{\triangleleft}
\newcommand{\rsup}{\triangleright}
\newcommand{\subgrp}{\overset{\mathrm{subgrp}}{\subset}}
\newcommand{\Ext}{\mathrm{Ext}}
\newcommand{\sbs}{\subset}\newcommand{\sps}{\supset}
\newcommand{\In}{\mathrm{In}}
\newcommand{\Tor}{\mathrm{Tor}}
\newcommand{\p}{\mathfrak{p}}
\newcommand{\q}{\mathfrak{q}}
\newcommand{\m}{\mathfrak{m}}
\newcommand{\cS}{\mathcal{S}}
\newcommand{\Frac}{\mathrm{Frac}\,}
\newcommand{\Spec}{\mathrm{Spec}\,}
\newcommand{\bA}{\mathbb{A}}
\newcommand{\Sym}{\mathrm{Sym}}
\newcommand{\Ann}{\mathrm{Ann}}
%%% 代数的位相幾何学
\newcommand{\Ho}{\mathrm{Ho}}
\newcommand{\CW}{\mathrm{CW}}
\newcommand{\lc}{\mathrm{lc}}
\newcommand{\cg}{\mathrm{cg}}
\newcommand{\Fib}{\mathrm{Fib}}
\newcommand{\Cyl}{\mathrm{Cyl}}
\newcommand{\Ch}{\mathrm{Ch}}
%%% 数値解析
\newcommand{\round}{\mathrm{round}}
\newcommand{\cond}{\mathrm{cond}}
\newcommand{\diag}{\mathrm{diag}}
%%% 確率論
\newcommand{\calF}{\mathcal{F}}
\newcommand{\X}{\mathcal{X}}
\newcommand{\Meas}{\mathrm{Meas}}
\newcommand{\as}{\;\mathrm{a.s.}} %almost surely
\newcommand{\io}{\;\mathrm{i.o.}} %infinitely often
\newcommand{\fe}{\;\mathrm{f.e.}} %with a finite number of exceptions
\newcommand{\F}{\mathcal{F}}
\newcommand{\bF}{\mathbb{F}}
\newcommand{\W}{\mathcal{W}}
\newcommand{\Pois}{\mathrm{Pois}}
\newcommand{\iid}{\mathrm{i.i.d.}}
\newcommand{\wconv}{\rightsquigarrow}
\newcommand{\Var}{\mathrm{Var}}
\newcommand{\xrightarrown}{\xrightarrow{n\to\infty}}
\newcommand{\au}{\mathrm{au}}
\newcommand{\cT}{\mathcal{T}}
%%% 情報理論
\newcommand{\bit}{\mathrm{bit}}
%%% 積分論
\newcommand{\calA}{\mathcal{A}}
\newcommand{\calB}{\mathcal{B}}
\newcommand{\D}{\mathcal{D}}
\newcommand{\Y}{\mathcal{Y}}
\newcommand{\calC}{\mathcal{C}}
\renewcommand{\ae}{\mathrm{a.e.}\;}
\newcommand{\cZ}{\mathcal{Z}}
\newcommand{\fF}{\mathfrak{F}}
\newcommand{\fI}{\mathfrak{I}}
\newcommand{\E}{\mathcal{E}}
\newcommand{\sMap}{\sigma\textrm{-}\mathrm{Map}}
\DeclareMathOperator*{\argmax}{arg\,max}
\DeclareMathOperator*{\argmin}{arg\,min}
\newcommand{\cC}{\mathcal{C}}
\newcommand{\comp}{\complement}
\newcommand{\J}{\mathcal{J}}
\newcommand{\sumN}[1]{\sum_{#1\in\N}}
\newcommand{\cupN}[1]{\cup_{#1\in\N}}
\newcommand{\capN}[1]{\cap_{#1\in\N}}
\newcommand{\Sum}[1]{\sum_{#1=1}^\infty}
\newcommand{\sumn}{\sum_{n=1}^\infty}
\newcommand{\summ}{\sum_{m=1}^\infty}
\newcommand{\sumk}{\sum_{k=1}^\infty}
\newcommand{\sumi}{\sum_{i=1}^\infty}
\newcommand{\sumj}{\sum_{j=1}^\infty}
\newcommand{\cupn}{\cup_{n=1}^\infty}
\newcommand{\capn}{\cap_{n=1}^\infty}
\newcommand{\cupk}{\cup_{k=1}^\infty}
\newcommand{\cupi}{\cup_{i=1}^\infty}
\newcommand{\cupj}{\cup_{j=1}^\infty}
\newcommand{\limn}{\lim_{n\to\infty}}
\renewcommand{\l}{\mathcal{l}}
\renewcommand{\L}{\mathcal{L}}
\newcommand{\Cl}{\mathrm{Cl}}
\newcommand{\cN}{\mathcal{N}}
\newcommand{\Ae}{\textrm{-a.e.}\;}
\newcommand{\csub}{\overset{\textrm{closed}}{\subset}}
\newcommand{\csup}{\overset{\textrm{closed}}{\supset}}
\newcommand{\wB}{\wt{B}}
\newcommand{\cG}{\mathcal{G}}
\newcommand{\Lip}{\mathrm{Lip}}
\newcommand{\Dom}{\mathrm{Dom}}
%%% 数理ファイナンス
\newcommand{\pre}{\mathrm{pre}}
\newcommand{\om}{\omega}

%%% 統計的因果推論
\newcommand{\Do}{\mathrm{Do}}
%%% 数理統計
\newcommand{\bP}{\mathbb{P}}
\newcommand{\compsub}{\overset{\textrm{cpt}}{\subset}}
\newcommand{\lip}{\textrm{lip}}
\newcommand{\BL}{\mathrm{BL}}
\newcommand{\G}{\mathbb{G}}
\newcommand{\NB}{\mathrm{NB}}
\newcommand{\oR}{\o{\R}}
\newcommand{\liminfn}{\liminf_{n\to\infty}}
\newcommand{\limsupn}{\limsup_{n\to\infty}}
%\newcommand{\limn}{\lim_{n\to\infty}}
\newcommand{\esssup}{\mathrm{ess.sup}}
\newcommand{\asto}{\xrightarrow{\as}}
\newcommand{\Cov}{\mathrm{Cov}}
\newcommand{\cQ}{\mathcal{Q}}
\newcommand{\VC}{\mathrm{VC}}
\newcommand{\mb}{\mathrm{mb}}
\newcommand{\Avar}{\mathrm{Avar}}
\newcommand{\bB}{\mathbb{B}}
\newcommand{\bW}{\mathbb{W}}
\newcommand{\sd}{\mathrm{sd}}
\newcommand{\w}[1]{\widehat{#1}}
\newcommand{\bZ}{\mathbb{Z}}
\newcommand{\Bernoulli}{\mathrm{Bernoulli}}
\newcommand{\Mult}{\mathrm{Mult}}
\newcommand{\BPois}{\mathrm{BPois}}
\newcommand{\fraks}{\mathfrak{s}}
\newcommand{\frakk}{\mathfrak{k}}
\newcommand{\IF}{\mathrm{IF}}
\newcommand{\bX}{\mathbf{X}}
\newcommand{\bx}{\mathbf{x}}
\newcommand{\indep}{\raisebox{0.05em}{\rotatebox[origin=c]{90}{$\models$}}}
\newcommand{\IG}{\mathrm{IG}}
\newcommand{\Levy}{\mathrm{Levy}}
\newcommand{\MP}{\mathrm{MP}}
\newcommand{\Hermite}{\mathrm{Hermite}}
\newcommand{\Skellam}{\mathrm{Skellam}}
\newcommand{\Dirichlet}{\mathrm{Dirichlet}}
\newcommand{\Beta}{\mathrm{Beta}}
\newcommand{\bE}{\mathbb{E}}
\newcommand{\bG}{\mathbb{G}}
\newcommand{\MISE}{\mathrm{MISE}}
\newcommand{\logit}{\mathtt{logit}}
\newcommand{\expit}{\mathtt{expit}}
\newcommand{\cK}{\mathcal{K}}
\newcommand{\dl}{\dot{l}}
\newcommand{\dotp}{\dot{p}}
\newcommand{\wl}{\wt{l}}
%%% 函数解析
\renewcommand{\c}{\mathbf{c}}
\newcommand{\loc}{\mathrm{loc}}
\newcommand{\Lh}{\mathrm{L.h.}}
\newcommand{\Epi}{\mathrm{Epi}\;}
\newcommand{\slim}{\mathrm{slim}}
\newcommand{\Ban}{\mathrm{Ban}}
\newcommand{\Hilb}{\mathrm{Hilb}}
\newcommand{\Ex}{\mathrm{Ex}}
\newcommand{\Co}{\mathrm{Co}}
\newcommand{\sa}{\mathrm{sa}}
\newcommand{\nnorm}[1]{{\left\vert\kern-0.25ex\left\vert\kern-0.25ex\left\vert #1 \right\vert\kern-0.25ex\right\vert\kern-0.25ex\right\vert}}
\newcommand{\dvol}{\mathrm{dvol}}
\newcommand{\Sconv}{\mathrm{Sconv}}
\newcommand{\I}{\mathcal{I}}
\newcommand{\nonunital}{\mathrm{nu}}
\newcommand{\cpt}{\mathrm{cpt}}
\newcommand{\lcpt}{\mathrm{lcpt}}
\newcommand{\com}{\mathrm{com}}
\newcommand{\Haus}{\mathrm{Haus}}
\newcommand{\proper}{\mathrm{proper}}
\newcommand{\infinity}{\mathrm{inf}}
\newcommand{\TVS}{\mathrm{TVS}}
\newcommand{\ess}{\mathrm{ess}}
\newcommand{\ext}{\mathrm{ext}}
\newcommand{\Index}{\mathrm{Index}}
\newcommand{\SSR}{\mathrm{SSR}}
\newcommand{\vs}{\mathrm{vs.}}
\newcommand{\fM}{\mathfrak{M}}
\newcommand{\EDM}{\mathrm{EDM}}
\newcommand{\Tw}{\mathrm{Tw}}
\newcommand{\fC}{\mathfrak{C}}
\newcommand{\bn}{\mathbf{n}}
\newcommand{\br}{\mathbf{r}}
\newcommand{\Lam}{\Lambda}
\newcommand{\lam}{\lambda}
\newcommand{\one}{\mathbf{1}}
\newcommand{\dae}{\text{-a.e.}}
\newcommand{\td}{\text{-}}
\newcommand{\RM}{\mathrm{RM}}
%%% 最適化
\newcommand{\Minimize}{\text{Minimize}}
\newcommand{\subjectto}{\text{subject to}}
\newcommand{\Ri}{\mathrm{Ri}}
%\newcommand{\Cl}{\mathrm{Cl}}
\newcommand{\Cone}{\mathrm{Cone}}
\newcommand{\Int}{\mathrm{Int}}
%%% 圏
\newcommand{\varlim}{\varprojlim}
\newcommand{\Hom}{\mathrm{Hom}}
\newcommand{\Iso}{\mathrm{Iso}}
\newcommand{\Mor}{\mathrm{Mor}}
\newcommand{\Isom}{\mathrm{Isom}}
\newcommand{\Aut}{\mathrm{Aut}}
\newcommand{\End}{\mathrm{End}}
\newcommand{\op}{\mathrm{op}}
\newcommand{\ev}{\mathrm{ev}}
\newcommand{\Ob}{\mathrm{Ob}}
\newcommand{\Ar}{\mathrm{Ar}}
\newcommand{\Arr}{\mathrm{Arr}}
\newcommand{\Set}{\mathrm{Set}}
\newcommand{\Grp}{\mathrm{Grp}}
\newcommand{\Cat}{\mathrm{Cat}}
\newcommand{\Mon}{\mathrm{Mon}}
\newcommand{\CMon}{\mathrm{CMon}} %Comutative Monoid 可換単系とモノイドの射
\newcommand{\Ring}{\mathrm{Ring}}
\newcommand{\CRing}{\mathrm{CRing}}
\newcommand{\Ab}{\mathrm{Ab}}
\newcommand{\Pos}{\mathrm{Pos}}
\newcommand{\Vect}{\mathrm{Vect}}
\newcommand{\FinVect}{\mathrm{FinVect}}
\newcommand{\FinSet}{\mathrm{FinSet}}
\newcommand{\OmegaAlg}{\Omega$-$\mathrm{Alg}}
\newcommand{\OmegaEAlg}{(\Omega,E)$-$\mathrm{Alg}}
\newcommand{\Alg}{\mathrm{Alg}} %代数の圏
\newcommand{\CAlg}{\mathrm{CAlg}} %可換代数の圏
\newcommand{\CPO}{\mathrm{CPO}} %Complete Partial Order & continuous mappings
\newcommand{\Fun}{\mathrm{Fun}}
\newcommand{\Func}{\mathrm{Func}}
\newcommand{\Met}{\mathrm{Met}} %Metric space & Contraction maps
\newcommand{\Pfn}{\mathrm{Pfn}} %Sets & Partial function
\newcommand{\Rel}{\mathrm{Rel}} %Sets & relation
\newcommand{\Bool}{\mathrm{Bool}}
\newcommand{\CABool}{\mathrm{CABool}}
\newcommand{\CompBoolAlg}{\mathrm{CompBoolAlg}}
\newcommand{\BoolAlg}{\mathrm{BoolAlg}}
\newcommand{\BoolRng}{\mathrm{BoolRng}}
\newcommand{\HeytAlg}{\mathrm{HeytAlg}}
\newcommand{\CompHeytAlg}{\mathrm{CompHeytAlg}}
\newcommand{\Lat}{\mathrm{Lat}}
\newcommand{\CompLat}{\mathrm{CompLat}}
\newcommand{\SemiLat}{\mathrm{SemiLat}}
\newcommand{\Stone}{\mathrm{Stone}}
\newcommand{\Sob}{\mathrm{Sob}} %Sober space & continuous map
\newcommand{\Op}{\mathrm{Op}} %Category of open subsets
\newcommand{\Sh}{\mathrm{Sh}} %Category of sheave
\newcommand{\PSh}{\mathrm{PSh}} %Category of presheave, PSh(C)=[C^op,set]のこと
\newcommand{\Conv}{\mathrm{Conv}} %Convergence spaceの圏
\newcommand{\Unif}{\mathrm{Unif}} %一様空間と一様連続写像の圏
\newcommand{\Frm}{\mathrm{Frm}} %フレームとフレームの射
\newcommand{\Locale}{\mathrm{Locale}} %その反対圏
\newcommand{\Diff}{\mathrm{Diff}} %滑らかな多様体の圏
\newcommand{\Mfd}{\mathrm{Mfd}}
\newcommand{\LieAlg}{\mathrm{LieAlg}}
\newcommand{\Quiv}{\mathrm{Quiv}} %Quiverの圏
\newcommand{\B}{\mathcal{B}}
\newcommand{\Span}{\mathrm{Span}}
\newcommand{\Corr}{\mathrm{Corr}}
\newcommand{\Decat}{\mathrm{Decat}}
\newcommand{\Rep}{\mathrm{Rep}}
\newcommand{\Grpd}{\mathrm{Grpd}}
\newcommand{\sSet}{\mathrm{sSet}}
\newcommand{\Mod}{\mathrm{Mod}}
\newcommand{\SmoothMnf}{\mathrm{SmoothMnf}}
\newcommand{\coker}{\mathrm{coker}}

\newcommand{\Ord}{\mathrm{Ord}}
\newcommand{\eq}{\mathrm{eq}}
\newcommand{\coeq}{\mathrm{coeq}}
\newcommand{\act}{\mathrm{act}}

%%%%%%%%%%%%%%% 定理環境(足助先生ありがとうございます) %%%%%%%%%%%%%%%

\everymath{\displaystyle}
\renewcommand{\proofname}{\bf [証明]}
\renewcommand{\thefootnote}{\dag\arabic{footnote}} %足助さんからもらった.どうなるんだ?
\renewcommand{\qedsymbol}{$\blacksquare$}

\renewcommand{\labelenumi}{(\arabic{enumi})} %(1),(2),...がデフォルトであって欲しい
\renewcommand{\labelenumii}{(\alph{enumii})}
\renewcommand{\labelenumiii}{(\roman{enumiii})}

\newtheoremstyle{StatementsWithStar}% ?name?
{3pt}% ?Space above? 1
{3pt}% ?Space below? 1
{}% ?Body font?
{}% ?Indent amount? 2
{\bfseries}% ?Theorem head font?
{\textbf{.}}% ?Punctuation after theorem head?
{.5em}% ?Space after theorem head? 3
{\textbf{\textup{#1~\thetheorem{}}}{}\,$^{\ast}$\thmnote{(#3)}}% ?Theorem head spec (can be left empty, meaning ‘normal’)?
%
\newtheoremstyle{StatementsWithStar2}% ?name?
{3pt}% ?Space above? 1
{3pt}% ?Space below? 1
{}% ?Body font?
{}% ?Indent amount? 2
{\bfseries}% ?Theorem head font?
{\textbf{.}}% ?Punctuation after theorem head?
{.5em}% ?Space after theorem head? 3
{\textbf{\textup{#1~\thetheorem{}}}{}\,$^{\ast\ast}$\thmnote{(#3)}}% ?Theorem head spec (can be left empty, meaning ‘normal’)?
%
\newtheoremstyle{StatementsWithStar3}% ?name?
{3pt}% ?Space above? 1
{3pt}% ?Space below? 1
{}% ?Body font?
{}% ?Indent amount? 2
{\bfseries}% ?Theorem head font?
{\textbf{.}}% ?Punctuation after theorem head?
{.5em}% ?Space after theorem head? 3
{\textbf{\textup{#1~\thetheorem{}}}{}\,$^{\ast\ast\ast}$\thmnote{(#3)}}% ?Theorem head spec (can be left empty, meaning ‘normal’)?
%
\newtheoremstyle{StatementsWithCCirc}% ?name?
{6pt}% ?Space above? 1
{6pt}% ?Space below? 1
{}% ?Body font?
{}% ?Indent amount? 2
{\bfseries}% ?Theorem head font?
{\textbf{.}}% ?Punctuation after theorem head?
{.5em}% ?Space after theorem head? 3
{\textbf{\textup{#1~\thetheorem{}}}{}\,$^{\circledcirc}$\thmnote{(#3)}}% ?Theorem head spec (can be left empty, meaning ‘normal’)?
%
\theoremstyle{definition}
 \newtheorem{theorem}{定理}[section]
 \newtheorem{axiom}[theorem]{公理}
 \newtheorem{corollary}[theorem]{系}
 \newtheorem{proposition}[theorem]{命題}
 \newtheorem*{proposition*}{命題}
 \newtheorem{lemma}[theorem]{補題}
 \newtheorem*{lemma*}{補題}
 \newtheorem*{theorem*}{定理}
 \newtheorem{definition}[theorem]{定義}
 \newtheorem{example}[theorem]{例}
 \newtheorem{notation}[theorem]{記法}
 \newtheorem*{notation*}{記法}
 \newtheorem{assumption}[theorem]{仮定}
 \newtheorem{question}[theorem]{問}
 \newtheorem{counterexample}[theorem]{反例}
 \newtheorem{reidai}[theorem]{例題}
 \newtheorem{ruidai}[theorem]{類題}
 \newtheorem{problem}[theorem]{問題}
 \newtheorem{algorithm}[theorem]{算譜}
 \newtheorem*{solution*}{\bf{[解]}}
 \newtheorem{discussion}[theorem]{議論}
 \newtheorem{remark}[theorem]{注}
 \newtheorem{remarks}[theorem]{要諦}
 \newtheorem{image}[theorem]{描像}
 \newtheorem{observation}[theorem]{観察}
 \newtheorem{universality}[theorem]{普遍性} %非自明な例外がない.
 \newtheorem{universal tendency}[theorem]{普遍傾向} %例外が有意に少ない.
 \newtheorem{hypothesis}[theorem]{仮説} %実験で説明されていない理論.
 \newtheorem{theory}[theorem]{理論} %実験事実とその(さしあたり)整合的な説明.
 \newtheorem{fact}[theorem]{実験事実}
 \newtheorem{model}[theorem]{模型}
 \newtheorem{explanation}[theorem]{説明} %理論による実験事実の説明
 \newtheorem{anomaly}[theorem]{理論の限界}
 \newtheorem{application}[theorem]{応用例}
 \newtheorem{method}[theorem]{手法} %実験手法など,技術的問題.
 \newtheorem{history}[theorem]{歴史}
 \newtheorem{usage}[theorem]{用語法}
 \newtheorem{research}[theorem]{研究}
 \newtheorem{shishin}[theorem]{指針}
 \newtheorem{yodan}[theorem]{余談}
 \newtheorem{construction}[theorem]{構成}
% \newtheorem*{remarknonum}{注}
 \newtheorem*{definition*}{定義}
 \newtheorem*{remark*}{注}
 \newtheorem*{question*}{問}
 \newtheorem*{problem*}{問題}
 \newtheorem*{axiom*}{公理}
 \newtheorem*{example*}{例}
 \newtheorem*{corollary*}{系}
 \newtheorem*{shishin*}{指針}
 \newtheorem*{yodan*}{余談}
 \newtheorem*{kadai*}{課題}
%
\theoremstyle{StatementsWithStar}
 \newtheorem{definition_*}[theorem]{定義}
 \newtheorem{question_*}[theorem]{問}
 \newtheorem{example_*}[theorem]{例}
 \newtheorem{theorem_*}[theorem]{定理}
 \newtheorem{remark_*}[theorem]{注}
%
\theoremstyle{StatementsWithStar2}
 \newtheorem{definition_**}[theorem]{定義}
 \newtheorem{theorem_**}[theorem]{定理}
 \newtheorem{question_**}[theorem]{問}
 \newtheorem{remark_**}[theorem]{注}
%
\theoremstyle{StatementsWithStar3}
 \newtheorem{remark_***}[theorem]{注}
 \newtheorem{question_***}[theorem]{問}
%
\theoremstyle{StatementsWithCCirc}
 \newtheorem{definition_O}[theorem]{定義}
 \newtheorem{question_O}[theorem]{問}
 \newtheorem{example_O}[theorem]{例}
 \newtheorem{remark_O}[theorem]{注}
%
\theoremstyle{definition}
%
\raggedbottom
\allowdisplaybreaks
\usepackage{mathtools}
%\mathtoolsset{showonlyrefs=true} %labelを附した数式にのみ附番される設定.
%\usepackage{amsmath} %mathtoolsの内部で呼ばれるので要らない.
\usepackage{amsfonts} %mathfrak, mathcal, mathbbなど.
\usepackage{amsthm} %定理環境.
\usepackage{amssymb} %AMSFontsを使うためのパッケージ.
\usepackage{ascmac} %screen, itembox, shadebox環境.全てLATEX2εの標準機能の範囲で作られたもの.
\usepackage{comment} %comment環境を用いて,複数行をcomment outできるようにするpackage
\usepackage{wrapfig} %図の周りに文字をwrapさせることができる.詳細な制御ができる.
\usepackage[usenames, dvipsnames]{xcolor} %xcolorはcolorの拡張.optionの意味はdvipsnamesはLoad a set of predefined colors. forestgreenなどの色が追加されている.usenamesはobsoleteとだけ書いてあった.
\setcounter{tocdepth}{2} %目次に表示される深さ.2はsubsectionまで
\usepackage{multicol} %\begin{multicols}{2}環境で途中からmulticolumnに出来る.

\usepackage{url}
\usepackage[dvipdfmx,colorlinks,linkcolor=blue,urlcolor=blue]{hyperref} %生成されるPDFファイルにおいて、\tableofcontentsによって書き出された目次をクリックすると該当する見出しへジャンプしたり、さらには、\label{ラベル名}を番号で参照する\ref{ラベル名}やthebibliography環境において\bibitem{ラベル名}を文献番号で参照する\cite{ラベル名}においても番号をクリックすると該当箇所にジャンプする.囲み枠はダサいので,colorlinksで囲み廃止し,リンク自体に色を付けることにした.
\usepackage{pxjahyper} %pxrubrica同様,八登崇之さん.hyperrefは日本語pLaTeXに最適化されていないから,hyperrefとセットで,(u)pLaTeX+hyperref+dvipdfmxの組み合わせで日本語を含む「しおり」をもつPDF文書を作成する場合に必要となる機能を提供する
\definecolor{花緑青}{cmyk}{0.52,0.03,0,0.27}
\definecolor{サーモンピンク}{cmyk}{0,0.65,0.65,0.05}
\definecolor{暗中模索}{rgb}{0.2,0.2,0.2}

\usepackage{tikz}
\usetikzlibrary{positioning,automata} %automaton描画のため
\usepackage{tikz-cd}
\usepackage[all]{xy}
\def\objectstyle{\displaystyle} %デフォルトではxymatrix中の数式が文中数式モードになるので,それを直す.\labelstyleも同様にxy packageの中で定義されており,文中数式モードになっている.

\usepackage[version=4]{mhchem} %化学式をTikZで簡単に書くためのパッケージ.
\usepackage{chemfig} %化学構造式をTikZで描くためのパッケージ.
\usepackage{siunitx} %IS単位を書くためのパッケージ

\usepackage{ulem} %取り消し線を引くためのパッケージ
\usepackage{pxrubrica} %日本語にルビをふる.八登崇之(やとうたかゆき)氏による.

\usepackage{graphicx} %rotatebox, scalebox, reflectbox, resizeboxなどのコマンドや,図表の読み込み\includegraphicsを司る.graphics というパッケージもありますが,graphicx はこれを高機能にしたものと考えて結構です(ただし graphicx は内部で graphics を読み込みます)

\usepackage[breakable]{tcolorbox} %加藤晃史さんがフル活用していたtcolorboxを,途中改ページ可能で.
\tcbuselibrary{theorems} %https://qiita.com/t_kemmochi/items/483b8fcdb5db8d1f5d5e
\usepackage{enumerate} %enumerate環境を凝らせる.
\usepackage[top=15truemm,bottom=15truemm,left=10truemm,right=10truemm]{geometry} %足助さんからもらったオプション

%%%%%%%%%%%%%%% 環境マクロ %%%%%%%%%%%%%%%

\usepackage{listings} %ソースコードを表示できる環境.多分もっといい方法ある.
\usepackage{jvlisting} %日本語のコメントアウトをする場合jlistingが必要
\lstset{ %ここからソースコードの表示に関する設定.lstlisting環境では,[caption=hoge,label=fuga]などのoptionを付けられる.
%[escapechar=!]とすると,LaTeXコマンドを使える.
  basicstyle={\ttfamily},
  identifierstyle={\small},
  commentstyle={\smallitshape},
  keywordstyle={\small\bfseries},
  ndkeywordstyle={\small},
  stringstyle={\small\ttfamily},
  frame={tb},
  breaklines=true,
  columns=[l]{fullflexible},
  numbers=left,
  xrightmargin=0zw,
  xleftmargin=3zw,
  numberstyle={\scriptsize},
  stepnumber=1,
  numbersep=1zw,
  lineskip=-0.5ex
}
%\makeatletter %caption番号を「[chapter番号].[section番号].[subsection番号]-[そのsubsection内においてn番目]」に変更
%    \AtBeginDocument{
%    \renewcommand*{\thelstlisting}{\arabic{chapter}.\arabic{section}.\arabic{lstlisting}}
%    \@addtoreset{lstlisting}{section}
%    }
%\makeatother
\renewcommand{\lstlistingname}{算譜} %caption名を"program"に変更

\newtcolorbox{tbox}[3][]{%
colframe=#2,colback=#2!10,coltitle=#2!20!black,title={#3},#1}

%%%%%%%%%%%%%%% フォント %%%%%%%%%%%%%%%

\usepackage{textcomp, mathcomp} %Text Companionとは,T1 encodingに入らなかった文字群.これを使うためのパッケージ.\textsectionでブルバキに!
\usepackage[T1]{fontenc} %8bitエンコーディングにする.comp系拡張数学文字の動作が安定する.

%%%%%%%%%%%%%%% 数学記号のマクロ %%%%%%%%%%%%%%%

\newcommand{\abs}[1]{\lvert#1\rvert} %mathtoolsはこうやって使うのか!
\newcommand{\Abs}[1]{\left|#1\right|}
\newcommand{\norm}[1]{\|#1\|}
\newcommand{\Norm}[1]{\left\|#1\right\|}
%\newcommand{\brace}[1]{\{#1\}}
\newcommand{\Brace}[1]{\left\{#1\right\}}
\newcommand{\paren}[1]{\left(#1\right)}
\newcommand{\bracket}[1]{\langle#1\rangle}
\newcommand{\brac}[1]{\langle#1\rangle}
\newcommand{\Bracket}[1]{\left\langle#1\right\rangle}
\newcommand{\Brac}[1]{\left\langle#1\right\rangle}
\newcommand{\Square}[1]{\left[#1\right]}
\renewcommand{\o}[1]{\overline{#1}}
\renewcommand{\u}[1]{\underline{#1}}
\renewcommand{\iff}{\;\mathrm{iff}\;} %nLabリスペクト
\newcommand{\pp}[2]{\frac{\partial #1}{\partial #2}}
\newcommand{\ppp}[3]{\frac{\partial #1}{\partial #2\partial #3}}
\newcommand{\dd}[2]{\frac{d #1}{d #2}}
\newcommand{\floor}[1]{\lfloor#1\rfloor}
\newcommand{\Floor}[1]{\left\lfloor#1\right\rfloor}
\newcommand{\ceil}[1]{\lceil#1\rceil}

\newcommand{\iso}{\xrightarrow{\,\smash{\raisebox{-0.45ex}{\ensuremath{\scriptstyle\sim}}}\,}}
\newcommand{\wt}[1]{\widetilde{#1}}
\newcommand{\wh}[1]{\widehat{#1}}

\newcommand{\Lrarrow}{\;\;\Leftrightarrow\;\;}

%ノルム位相についての閉包 https://newbedev.com/how-to-make-double-overline-with-less-vertical-displacement
\makeatletter
\newcommand{\dbloverline}[1]{\overline{\dbl@overline{#1}}}
\newcommand{\dbl@overline}[1]{\mathpalette\dbl@@overline{#1}}
\newcommand{\dbl@@overline}[2]{%
  \begingroup
  \sbox\z@{$\m@th#1\overline{#2}$}%
  \ht\z@=\dimexpr\ht\z@-2\dbl@adjust{#1}\relax
  \box\z@
  \ifx#1\scriptstyle\kern-\scriptspace\else
  \ifx#1\scriptscriptstyle\kern-\scriptspace\fi\fi
  \endgroup
}
\newcommand{\dbl@adjust}[1]{%
  \fontdimen8
  \ifx#1\displaystyle\textfont\else
  \ifx#1\textstyle\textfont\else
  \ifx#1\scriptstyle\scriptfont\else
  \scriptscriptfont\fi\fi\fi 3
}
\makeatother
\newcommand{\oo}[1]{\dbloverline{#1}}

\DeclareMathOperator{\grad}{\mathrm{grad}}
\DeclareMathOperator{\rot}{\mathrm{rot}}
\DeclareMathOperator{\divergence}{\mathrm{div}}
\newcommand{\False}{\mathrm{False}}
\newcommand{\True}{\mathrm{True}}
\DeclareMathOperator{\tr}{\mathrm{tr}}
\newcommand{\M}{\mathcal{M}}
\newcommand{\cF}{\mathcal{F}}
\newcommand{\cD}{\mathcal{D}}
\newcommand{\fX}{\mathfrak{X}}
\newcommand{\fY}{\mathfrak{Y}}
\newcommand{\fZ}{\mathfrak{Z}}
\renewcommand{\H}{\mathcal{H}}
\newcommand{\fH}{\mathfrak{H}}
\newcommand{\bH}{\mathbb{H}}
\newcommand{\id}{\mathrm{id}}
\newcommand{\A}{\mathcal{A}}
% \renewcommand\coprod{\rotatebox[origin=c]{180}{$\prod$}} すでにどこかにある.
\newcommand{\pr}{\mathrm{pr}}
\newcommand{\U}{\mathfrak{U}}
\newcommand{\Map}{\mathrm{Map}}
\newcommand{\dom}{\mathrm{Dom}\;}
\newcommand{\cod}{\mathrm{Cod}\;}
\newcommand{\supp}{\mathrm{supp}\;}
\newcommand{\otherwise}{\mathrm{otherwise}}
\newcommand{\st}{\;\mathrm{s.t.}\;}
\newcommand{\lmd}{\lambda}
\newcommand{\Lmd}{\Lambda}
%%% 線型代数学
\newcommand{\Ker}{\mathrm{Ker}\;}
\newcommand{\Coker}{\mathrm{Coker}\;}
\newcommand{\Coim}{\mathrm{Coim}\;}
\newcommand{\rank}{\mathrm{rank}}
\newcommand{\lcm}{\mathrm{lcm}}
\newcommand{\sgn}{\mathrm{sgn}}
\newcommand{\GL}{\mathrm{GL}}
\newcommand{\SL}{\mathrm{SL}}
\newcommand{\alt}{\mathrm{alt}}
%%% 複素解析学
\renewcommand{\Re}{\mathrm{Re}\;}
\renewcommand{\Im}{\mathrm{Im}\;}
\newcommand{\Gal}{\mathrm{Gal}}
\newcommand{\PGL}{\mathrm{PGL}}
\newcommand{\PSL}{\mathrm{PSL}}
\newcommand{\Log}{\mathrm{Log}\,}
\newcommand{\Res}{\mathrm{Res}\,}
\newcommand{\on}{\mathrm{on}\;}
\newcommand{\hatC}{\hat{\C}}
\newcommand{\hatR}{\hat{\R}}
\newcommand{\PV}{\mathrm{P.V.}}
\newcommand{\diam}{\mathrm{diam}}
\newcommand{\Area}{\mathrm{Area}}
\newcommand{\Lap}{\Laplace}
\newcommand{\f}{\mathbf{f}}
\newcommand{\cR}{\mathcal{R}}
\newcommand{\const}{\mathrm{const.}}
\newcommand{\Om}{\Omega}
\newcommand{\Cinf}{C^\infty}
\newcommand{\ep}{\epsilon}
\newcommand{\dist}{\mathrm{dist}}
\newcommand{\opart}{\o{\partial}}
%%% 解析力学
\newcommand{\x}{\mathbf{x}}
%%% 集合と位相
\renewcommand{\O}{\mathcal{O}}
\renewcommand{\S}{\mathcal{S}}
\renewcommand{\U}{\mathcal{U}}
\newcommand{\V}{\mathcal{V}}
\renewcommand{\P}{\mathcal{P}}
\newcommand{\R}{\mathbb{R}}
\newcommand{\N}{\mathbb{N}}
\newcommand{\C}{\mathbb{C}}
\newcommand{\Z}{\mathbb{Z}}
\newcommand{\Q}{\mathbb{Q}}
\newcommand{\TV}{\mathrm{TV}}
\newcommand{\ORD}{\mathrm{ORD}}
\newcommand{\Tr}{\mathrm{Tr}\;}
\newcommand{\Card}{\mathrm{Card}\;}
\newcommand{\Top}{\mathrm{Top}}
\newcommand{\Disc}{\mathrm{Disc}}
\newcommand{\Codisc}{\mathrm{Codisc}}
\newcommand{\CoDisc}{\mathrm{CoDisc}}
\newcommand{\Ult}{\mathrm{Ult}}
\newcommand{\ord}{\mathrm{ord}}
\newcommand{\maj}{\mathrm{maj}}
%%% 形式言語理論
\newcommand{\REGEX}{\mathrm{REGEX}}
\newcommand{\RE}{\mathbf{RE}}

%%% Fourier解析
\newcommand*{\Laplace}{\mathop{}\!\mathbin\bigtriangleup}
\newcommand*{\DAlambert}{\mathop{}\!\mathbin\Box}
%%% Graph Theory
\newcommand{\SimpGph}{\mathrm{SimpGph}}
\newcommand{\Gph}{\mathrm{Gph}}
\newcommand{\mult}{\mathrm{mult}}
\newcommand{\inv}{\mathrm{inv}}
%%% 多様体
\newcommand{\Der}{\mathrm{Der}}
\newcommand{\osub}{\overset{\mathrm{open}}{\subset}}
\newcommand{\osup}{\overset{\mathrm{open}}{\supset}}
\newcommand{\al}{\alpha}
\newcommand{\K}{\mathbb{K}}
\newcommand{\Sp}{\mathrm{Sp}}
\newcommand{\g}{\mathfrak{g}}
\newcommand{\h}{\mathfrak{h}}
\newcommand{\Exp}{\mathrm{Exp}\;}
\newcommand{\Imm}{\mathrm{Imm}}
\newcommand{\Imb}{\mathrm{Imb}}
\newcommand{\codim}{\mathrm{codim}\;}
\newcommand{\Gr}{\mathrm{Gr}}
%%% 代数
\newcommand{\Ad}{\mathrm{Ad}}
\newcommand{\finsupp}{\mathrm{fin\;supp}}
\newcommand{\SO}{\mathrm{SO}}
\newcommand{\SU}{\mathrm{SU}}
\newcommand{\acts}{\curvearrowright}
\newcommand{\mono}{\hookrightarrow}
\newcommand{\epi}{\twoheadrightarrow}
\newcommand{\Stab}{\mathrm{Stab}}
\newcommand{\nor}{\mathrm{nor}}
\newcommand{\T}{\mathbb{T}}
\newcommand{\Aff}{\mathrm{Aff}}
\newcommand{\rsub}{\triangleleft}
\newcommand{\rsup}{\triangleright}
\newcommand{\subgrp}{\overset{\mathrm{subgrp}}{\subset}}
\newcommand{\Ext}{\mathrm{Ext}}
\newcommand{\sbs}{\subset}\newcommand{\sps}{\supset}
\newcommand{\In}{\mathrm{In}}
\newcommand{\Tor}{\mathrm{Tor}}
\newcommand{\p}{\mathfrak{p}}
\newcommand{\q}{\mathfrak{q}}
\newcommand{\m}{\mathfrak{m}}
\newcommand{\cS}{\mathcal{S}}
\newcommand{\Frac}{\mathrm{Frac}\,}
\newcommand{\Spec}{\mathrm{Spec}\,}
\newcommand{\bA}{\mathbb{A}}
\newcommand{\Sym}{\mathrm{Sym}}
\newcommand{\Ann}{\mathrm{Ann}}
%%% 代数的位相幾何学
\newcommand{\Ho}{\mathrm{Ho}}
\newcommand{\CW}{\mathrm{CW}}
\newcommand{\lc}{\mathrm{lc}}
\newcommand{\cg}{\mathrm{cg}}
\newcommand{\Fib}{\mathrm{Fib}}
\newcommand{\Cyl}{\mathrm{Cyl}}
\newcommand{\Ch}{\mathrm{Ch}}
%%% 数値解析
\newcommand{\round}{\mathrm{round}}
\newcommand{\cond}{\mathrm{cond}}
\newcommand{\diag}{\mathrm{diag}}
%%% 確率論
\newcommand{\calF}{\mathcal{F}}
\newcommand{\X}{\mathcal{X}}
\newcommand{\Meas}{\mathrm{Meas}}
\newcommand{\as}{\;\mathrm{a.s.}} %almost surely
\newcommand{\io}{\;\mathrm{i.o.}} %infinitely often
\newcommand{\fe}{\;\mathrm{f.e.}} %with a finite number of exceptions
\newcommand{\F}{\mathcal{F}}
\newcommand{\bF}{\mathbb{F}}
\newcommand{\W}{\mathcal{W}}
\newcommand{\Pois}{\mathrm{Pois}}
\newcommand{\iid}{\mathrm{i.i.d.}}
\newcommand{\wconv}{\rightsquigarrow}
\newcommand{\Var}{\mathrm{Var}}
\newcommand{\xrightarrown}{\xrightarrow{n\to\infty}}
\newcommand{\au}{\mathrm{au}}
\newcommand{\cT}{\mathcal{T}}
%%% 情報理論
\newcommand{\bit}{\mathrm{bit}}
%%% 積分論
\newcommand{\calA}{\mathcal{A}}
\newcommand{\calB}{\mathcal{B}}
\newcommand{\D}{\mathcal{D}}
\newcommand{\Y}{\mathcal{Y}}
\newcommand{\calC}{\mathcal{C}}
\renewcommand{\ae}{\mathrm{a.e.}\;}
\newcommand{\cZ}{\mathcal{Z}}
\newcommand{\fF}{\mathfrak{F}}
\newcommand{\fI}{\mathfrak{I}}
\newcommand{\E}{\mathcal{E}}
\newcommand{\sMap}{\sigma\textrm{-}\mathrm{Map}}
\DeclareMathOperator*{\argmax}{arg\,max}
\DeclareMathOperator*{\argmin}{arg\,min}
\newcommand{\cC}{\mathcal{C}}
\newcommand{\comp}{\complement}
\newcommand{\J}{\mathcal{J}}
\newcommand{\sumN}[1]{\sum_{#1\in\N}}
\newcommand{\cupN}[1]{\cup_{#1\in\N}}
\newcommand{\capN}[1]{\cap_{#1\in\N}}
\newcommand{\Sum}[1]{\sum_{#1=1}^\infty}
\newcommand{\sumn}{\sum_{n=1}^\infty}
\newcommand{\summ}{\sum_{m=1}^\infty}
\newcommand{\sumk}{\sum_{k=1}^\infty}
\newcommand{\sumi}{\sum_{i=1}^\infty}
\newcommand{\sumj}{\sum_{j=1}^\infty}
\newcommand{\cupn}{\cup_{n=1}^\infty}
\newcommand{\capn}{\cap_{n=1}^\infty}
\newcommand{\cupk}{\cup_{k=1}^\infty}
\newcommand{\cupi}{\cup_{i=1}^\infty}
\newcommand{\cupj}{\cup_{j=1}^\infty}
\newcommand{\limn}{\lim_{n\to\infty}}
\renewcommand{\l}{\mathcal{l}}
\renewcommand{\L}{\mathcal{L}}
\newcommand{\Cl}{\mathrm{Cl}}
\newcommand{\cN}{\mathcal{N}}
\newcommand{\Ae}{\textrm{-a.e.}\;}
\newcommand{\csub}{\overset{\textrm{closed}}{\subset}}
\newcommand{\csup}{\overset{\textrm{closed}}{\supset}}
\newcommand{\wB}{\wt{B}}
\newcommand{\cG}{\mathcal{G}}
\newcommand{\Lip}{\mathrm{Lip}}
\newcommand{\Dom}{\mathrm{Dom}}
%%% 数理ファイナンス
\newcommand{\pre}{\mathrm{pre}}
\newcommand{\om}{\omega}

%%% 統計的因果推論
\newcommand{\Do}{\mathrm{Do}}
%%% 数理統計
\newcommand{\bP}{\mathbb{P}}
\newcommand{\compsub}{\overset{\textrm{cpt}}{\subset}}
\newcommand{\lip}{\textrm{lip}}
\newcommand{\BL}{\mathrm{BL}}
\newcommand{\G}{\mathbb{G}}
\newcommand{\NB}{\mathrm{NB}}
\newcommand{\oR}{\o{\R}}
\newcommand{\liminfn}{\liminf_{n\to\infty}}
\newcommand{\limsupn}{\limsup_{n\to\infty}}
%\newcommand{\limn}{\lim_{n\to\infty}}
\newcommand{\esssup}{\mathrm{ess.sup}}
\newcommand{\asto}{\xrightarrow{\as}}
\newcommand{\Cov}{\mathrm{Cov}}
\newcommand{\cQ}{\mathcal{Q}}
\newcommand{\VC}{\mathrm{VC}}
\newcommand{\mb}{\mathrm{mb}}
\newcommand{\Avar}{\mathrm{Avar}}
\newcommand{\bB}{\mathbb{B}}
\newcommand{\bW}{\mathbb{W}}
\newcommand{\sd}{\mathrm{sd}}
\newcommand{\w}[1]{\widehat{#1}}
\newcommand{\bZ}{\mathbb{Z}}
\newcommand{\Bernoulli}{\mathrm{Bernoulli}}
\newcommand{\Mult}{\mathrm{Mult}}
\newcommand{\BPois}{\mathrm{BPois}}
\newcommand{\fraks}{\mathfrak{s}}
\newcommand{\frakk}{\mathfrak{k}}
\newcommand{\IF}{\mathrm{IF}}
\newcommand{\bX}{\mathbf{X}}
\newcommand{\bx}{\mathbf{x}}
\newcommand{\indep}{\raisebox{0.05em}{\rotatebox[origin=c]{90}{$\models$}}}
\newcommand{\IG}{\mathrm{IG}}
\newcommand{\Levy}{\mathrm{Levy}}
\newcommand{\MP}{\mathrm{MP}}
\newcommand{\Hermite}{\mathrm{Hermite}}
\newcommand{\Skellam}{\mathrm{Skellam}}
\newcommand{\Dirichlet}{\mathrm{Dirichlet}}
\newcommand{\Beta}{\mathrm{Beta}}
\newcommand{\bE}{\mathbb{E}}
\newcommand{\bG}{\mathbb{G}}
\newcommand{\MISE}{\mathrm{MISE}}
\newcommand{\logit}{\mathtt{logit}}
\newcommand{\expit}{\mathtt{expit}}
\newcommand{\cK}{\mathcal{K}}
\newcommand{\dl}{\dot{l}}
\newcommand{\dotp}{\dot{p}}
\newcommand{\wl}{\wt{l}}
%%% 函数解析
\renewcommand{\c}{\mathbf{c}}
\newcommand{\loc}{\mathrm{loc}}
\newcommand{\Lh}{\mathrm{L.h.}}
\newcommand{\Epi}{\mathrm{Epi}\;}
\newcommand{\slim}{\mathrm{slim}}
\newcommand{\Ban}{\mathrm{Ban}}
\newcommand{\Hilb}{\mathrm{Hilb}}
\newcommand{\Ex}{\mathrm{Ex}}
\newcommand{\Co}{\mathrm{Co}}
\newcommand{\sa}{\mathrm{sa}}
\newcommand{\nnorm}[1]{{\left\vert\kern-0.25ex\left\vert\kern-0.25ex\left\vert #1 \right\vert\kern-0.25ex\right\vert\kern-0.25ex\right\vert}}
\newcommand{\dvol}{\mathrm{dvol}}
\newcommand{\Sconv}{\mathrm{Sconv}}
\newcommand{\I}{\mathcal{I}}
\newcommand{\nonunital}{\mathrm{nu}}
\newcommand{\cpt}{\mathrm{cpt}}
\newcommand{\lcpt}{\mathrm{lcpt}}
\newcommand{\com}{\mathrm{com}}
\newcommand{\Haus}{\mathrm{Haus}}
\newcommand{\proper}{\mathrm{proper}}
\newcommand{\infinity}{\mathrm{inf}}
\newcommand{\TVS}{\mathrm{TVS}}
\newcommand{\ess}{\mathrm{ess}}
\newcommand{\ext}{\mathrm{ext}}
\newcommand{\Index}{\mathrm{Index}}
\newcommand{\SSR}{\mathrm{SSR}}
\newcommand{\vs}{\mathrm{vs.}}
\newcommand{\fM}{\mathfrak{M}}
\newcommand{\EDM}{\mathrm{EDM}}
\newcommand{\Tw}{\mathrm{Tw}}
\newcommand{\fC}{\mathfrak{C}}
\newcommand{\bn}{\mathbf{n}}
\newcommand{\br}{\mathbf{r}}
\newcommand{\Lam}{\Lambda}
\newcommand{\lam}{\lambda}
\newcommand{\one}{\mathbf{1}}
\newcommand{\dae}{\text{-a.e.}}
\newcommand{\td}{\text{-}}
\newcommand{\RM}{\mathrm{RM}}
%%% 最適化
\newcommand{\Minimize}{\text{Minimize}}
\newcommand{\subjectto}{\text{subject to}}
\newcommand{\Ri}{\mathrm{Ri}}
%\newcommand{\Cl}{\mathrm{Cl}}
\newcommand{\Cone}{\mathrm{Cone}}
\newcommand{\Int}{\mathrm{Int}}
%%% 圏
\newcommand{\varlim}{\varprojlim}
\newcommand{\Hom}{\mathrm{Hom}}
\newcommand{\Iso}{\mathrm{Iso}}
\newcommand{\Mor}{\mathrm{Mor}}
\newcommand{\Isom}{\mathrm{Isom}}
\newcommand{\Aut}{\mathrm{Aut}}
\newcommand{\End}{\mathrm{End}}
\newcommand{\op}{\mathrm{op}}
\newcommand{\ev}{\mathrm{ev}}
\newcommand{\Ob}{\mathrm{Ob}}
\newcommand{\Ar}{\mathrm{Ar}}
\newcommand{\Arr}{\mathrm{Arr}}
\newcommand{\Set}{\mathrm{Set}}
\newcommand{\Grp}{\mathrm{Grp}}
\newcommand{\Cat}{\mathrm{Cat}}
\newcommand{\Mon}{\mathrm{Mon}}
\newcommand{\CMon}{\mathrm{CMon}} %Comutative Monoid 可換単系とモノイドの射
\newcommand{\Ring}{\mathrm{Ring}}
\newcommand{\CRing}{\mathrm{CRing}}
\newcommand{\Ab}{\mathrm{Ab}}
\newcommand{\Pos}{\mathrm{Pos}}
\newcommand{\Vect}{\mathrm{Vect}}
\newcommand{\FinVect}{\mathrm{FinVect}}
\newcommand{\FinSet}{\mathrm{FinSet}}
\newcommand{\OmegaAlg}{\Omega$-$\mathrm{Alg}}
\newcommand{\OmegaEAlg}{(\Omega,E)$-$\mathrm{Alg}}
\newcommand{\Alg}{\mathrm{Alg}} %代数の圏
\newcommand{\CAlg}{\mathrm{CAlg}} %可換代数の圏
\newcommand{\CPO}{\mathrm{CPO}} %Complete Partial Order & continuous mappings
\newcommand{\Fun}{\mathrm{Fun}}
\newcommand{\Func}{\mathrm{Func}}
\newcommand{\Met}{\mathrm{Met}} %Metric space & Contraction maps
\newcommand{\Pfn}{\mathrm{Pfn}} %Sets & Partial function
\newcommand{\Rel}{\mathrm{Rel}} %Sets & relation
\newcommand{\Bool}{\mathrm{Bool}}
\newcommand{\CABool}{\mathrm{CABool}}
\newcommand{\CompBoolAlg}{\mathrm{CompBoolAlg}}
\newcommand{\BoolAlg}{\mathrm{BoolAlg}}
\newcommand{\BoolRng}{\mathrm{BoolRng}}
\newcommand{\HeytAlg}{\mathrm{HeytAlg}}
\newcommand{\CompHeytAlg}{\mathrm{CompHeytAlg}}
\newcommand{\Lat}{\mathrm{Lat}}
\newcommand{\CompLat}{\mathrm{CompLat}}
\newcommand{\SemiLat}{\mathrm{SemiLat}}
\newcommand{\Stone}{\mathrm{Stone}}
\newcommand{\Sob}{\mathrm{Sob}} %Sober space & continuous map
\newcommand{\Op}{\mathrm{Op}} %Category of open subsets
\newcommand{\Sh}{\mathrm{Sh}} %Category of sheave
\newcommand{\PSh}{\mathrm{PSh}} %Category of presheave, PSh(C)=[C^op,set]のこと
\newcommand{\Conv}{\mathrm{Conv}} %Convergence spaceの圏
\newcommand{\Unif}{\mathrm{Unif}} %一様空間と一様連続写像の圏
\newcommand{\Frm}{\mathrm{Frm}} %フレームとフレームの射
\newcommand{\Locale}{\mathrm{Locale}} %その反対圏
\newcommand{\Diff}{\mathrm{Diff}} %滑らかな多様体の圏
\newcommand{\Mfd}{\mathrm{Mfd}}
\newcommand{\LieAlg}{\mathrm{LieAlg}}
\newcommand{\Quiv}{\mathrm{Quiv}} %Quiverの圏
\newcommand{\B}{\mathcal{B}}
\newcommand{\Span}{\mathrm{Span}}
\newcommand{\Corr}{\mathrm{Corr}}
\newcommand{\Decat}{\mathrm{Decat}}
\newcommand{\Rep}{\mathrm{Rep}}
\newcommand{\Grpd}{\mathrm{Grpd}}
\newcommand{\sSet}{\mathrm{sSet}}
\newcommand{\Mod}{\mathrm{Mod}}
\newcommand{\SmoothMnf}{\mathrm{SmoothMnf}}
\newcommand{\coker}{\mathrm{coker}}

\newcommand{\Ord}{\mathrm{Ord}}
\newcommand{\eq}{\mathrm{eq}}
\newcommand{\coeq}{\mathrm{coeq}}
\newcommand{\act}{\mathrm{act}}

%%%%%%%%%%%%%%% 定理環境(足助先生ありがとうございます) %%%%%%%%%%%%%%%

\everymath{\displaystyle}
\renewcommand{\proofname}{\bf [証明]}
\renewcommand{\thefootnote}{\dag\arabic{footnote}} %足助さんからもらった.どうなるんだ?
\renewcommand{\qedsymbol}{$\blacksquare$}

\renewcommand{\labelenumi}{(\arabic{enumi})} %(1),(2),...がデフォルトであって欲しい
\renewcommand{\labelenumii}{(\alph{enumii})}
\renewcommand{\labelenumiii}{(\roman{enumiii})}

\newtheoremstyle{StatementsWithStar}% ?name?
{3pt}% ?Space above? 1
{3pt}% ?Space below? 1
{}% ?Body font?
{}% ?Indent amount? 2
{\bfseries}% ?Theorem head font?
{\textbf{.}}% ?Punctuation after theorem head?
{.5em}% ?Space after theorem head? 3
{\textbf{\textup{#1~\thetheorem{}}}{}\,$^{\ast}$\thmnote{(#3)}}% ?Theorem head spec (can be left empty, meaning ‘normal’)?
%
\newtheoremstyle{StatementsWithStar2}% ?name?
{3pt}% ?Space above? 1
{3pt}% ?Space below? 1
{}% ?Body font?
{}% ?Indent amount? 2
{\bfseries}% ?Theorem head font?
{\textbf{.}}% ?Punctuation after theorem head?
{.5em}% ?Space after theorem head? 3
{\textbf{\textup{#1~\thetheorem{}}}{}\,$^{\ast\ast}$\thmnote{(#3)}}% ?Theorem head spec (can be left empty, meaning ‘normal’)?
%
\newtheoremstyle{StatementsWithStar3}% ?name?
{3pt}% ?Space above? 1
{3pt}% ?Space below? 1
{}% ?Body font?
{}% ?Indent amount? 2
{\bfseries}% ?Theorem head font?
{\textbf{.}}% ?Punctuation after theorem head?
{.5em}% ?Space after theorem head? 3
{\textbf{\textup{#1~\thetheorem{}}}{}\,$^{\ast\ast\ast}$\thmnote{(#3)}}% ?Theorem head spec (can be left empty, meaning ‘normal’)?
%
\newtheoremstyle{StatementsWithCCirc}% ?name?
{6pt}% ?Space above? 1
{6pt}% ?Space below? 1
{}% ?Body font?
{}% ?Indent amount? 2
{\bfseries}% ?Theorem head font?
{\textbf{.}}% ?Punctuation after theorem head?
{.5em}% ?Space after theorem head? 3
{\textbf{\textup{#1~\thetheorem{}}}{}\,$^{\circledcirc}$\thmnote{(#3)}}% ?Theorem head spec (can be left empty, meaning ‘normal’)?
%
\theoremstyle{definition}
 \newtheorem{theorem}{定理}[section]
 \newtheorem{axiom}[theorem]{公理}
 \newtheorem{corollary}[theorem]{系}
 \newtheorem{proposition}[theorem]{命題}
 \newtheorem*{proposition*}{命題}
 \newtheorem{lemma}[theorem]{補題}
 \newtheorem*{lemma*}{補題}
 \newtheorem*{theorem*}{定理}
 \newtheorem{definition}[theorem]{定義}
 \newtheorem{example}[theorem]{例}
 \newtheorem{notation}[theorem]{記法}
 \newtheorem*{notation*}{記法}
 \newtheorem{assumption}[theorem]{仮定}
 \newtheorem{question}[theorem]{問}
 \newtheorem{counterexample}[theorem]{反例}
 \newtheorem{reidai}[theorem]{例題}
 \newtheorem{ruidai}[theorem]{類題}
 \newtheorem{problem}[theorem]{問題}
 \newtheorem{algorithm}[theorem]{算譜}
 \newtheorem*{solution*}{\bf{[解]}}
 \newtheorem{discussion}[theorem]{議論}
 \newtheorem{remark}[theorem]{注}
 \newtheorem{remarks}[theorem]{要諦}
 \newtheorem{image}[theorem]{描像}
 \newtheorem{observation}[theorem]{観察}
 \newtheorem{universality}[theorem]{普遍性} %非自明な例外がない.
 \newtheorem{universal tendency}[theorem]{普遍傾向} %例外が有意に少ない.
 \newtheorem{hypothesis}[theorem]{仮説} %実験で説明されていない理論.
 \newtheorem{theory}[theorem]{理論} %実験事実とその(さしあたり)整合的な説明.
 \newtheorem{fact}[theorem]{実験事実}
 \newtheorem{model}[theorem]{模型}
 \newtheorem{explanation}[theorem]{説明} %理論による実験事実の説明
 \newtheorem{anomaly}[theorem]{理論の限界}
 \newtheorem{application}[theorem]{応用例}
 \newtheorem{method}[theorem]{手法} %実験手法など,技術的問題.
 \newtheorem{history}[theorem]{歴史}
 \newtheorem{usage}[theorem]{用語法}
 \newtheorem{research}[theorem]{研究}
 \newtheorem{shishin}[theorem]{指針}
 \newtheorem{yodan}[theorem]{余談}
 \newtheorem{construction}[theorem]{構成}
% \newtheorem*{remarknonum}{注}
 \newtheorem*{definition*}{定義}
 \newtheorem*{remark*}{注}
 \newtheorem*{question*}{問}
 \newtheorem*{problem*}{問題}
 \newtheorem*{axiom*}{公理}
 \newtheorem*{example*}{例}
 \newtheorem*{corollary*}{系}
 \newtheorem*{shishin*}{指針}
 \newtheorem*{yodan*}{余談}
 \newtheorem*{kadai*}{課題}
%
\theoremstyle{StatementsWithStar}
 \newtheorem{definition_*}[theorem]{定義}
 \newtheorem{question_*}[theorem]{問}
 \newtheorem{example_*}[theorem]{例}
 \newtheorem{theorem_*}[theorem]{定理}
 \newtheorem{remark_*}[theorem]{注}
%
\theoremstyle{StatementsWithStar2}
 \newtheorem{definition_**}[theorem]{定義}
 \newtheorem{theorem_**}[theorem]{定理}
 \newtheorem{question_**}[theorem]{問}
 \newtheorem{remark_**}[theorem]{注}
%
\theoremstyle{StatementsWithStar3}
 \newtheorem{remark_***}[theorem]{注}
 \newtheorem{question_***}[theorem]{問}
%
\theoremstyle{StatementsWithCCirc}
 \newtheorem{definition_O}[theorem]{定義}
 \newtheorem{question_O}[theorem]{問}
 \newtheorem{example_O}[theorem]{例}
 \newtheorem{remark_O}[theorem]{注}
%
\theoremstyle{definition}
%
\raggedbottom
\allowdisplaybreaks
\usepackage[math]{anttor}
\begin{document}
\tableofcontents

\chapter{集合と写像}

\begin{notation}\mbox{}
    \begin{enumerate}
        \item 位相空間論では集合の概念については次の3点のみを使用し,その定義については抽象化してinformalに言及する.
        \begin{enumerate}[(1)]
            \item 像写像は合併は保つが,共通部分は不完全にしか保たない(命題\ref{prop-functoriality-of-image-and-inverse-image-mappings}).
            \item 逆像写像は合併も共通部分も保つ(命題\ref{prop-functoriality-of-image-and-inverse-image-mappings}).
            \item de Morganの法則(命題\ref{prop-de-Morgan's-law}).
        \end{enumerate}
        \item 集合$X$に対し,その有限部分集合全体からなる集合を
        \[ F(X):=\{A\in P(X)\mid|A|<\infty\} \]
        と置く.
    \end{enumerate}
\end{notation}

\section{写像の定める関手}

\begin{tcolorbox}[colframe=ForestGreen, colback=ForestGreen!10!white,breakable,colbacktitle=ForestGreen!40!white,coltitle=black,fonttitle=\bfseries\sffamily,
title=]
    位相では,集合$f:X\to Y$の定める対応$f^*:P(X)\to P(Y)$が肝要になる.
    この関手の性質を調べる.
    $P(X)$には,完備Boole代数である側面と,完備束である側面とがある.
\end{tcolorbox}

\subsection{随伴性}

\begin{tcolorbox}[colframe=ForestGreen, colback=ForestGreen!10!white,breakable,colbacktitle=ForestGreen!40!white,coltitle=black,fonttitle=\bfseries\sffamily,
title=]
    $f^*,f_*$は互いに随伴をなす($f_*$が左随伴:$f_*\dashv f^*$).これを用いて$P(X)$と$P(Y)$を往来して調べることが数学の基礎の最も基本的な手法となる.
    Lawvereによると,論理とは,2つの互いに随伴な射の組である.
\end{tcolorbox}

\begin{proposition}[adjunction]\label{prop-adjunction}
    $f:X\to Y$を写像,$A\subset X,B\subset Y$とする.
    \begin{enumerate}
        \item $f(A)\subset B\Leftrightarrow A\subset f^{-1}(B)$.
        \item \begin{enumerate}[(a)]
            \item $A\subset f^{-1}(f(A))$.
            \item $f(f^{-1}(B))\subset B$.
        \end{enumerate}
        \item $f$が単射のとき(a)は等号成立.また,(a)が任意の$A\in P(X)$について成り立つならば$f$は単射.
        \item $f$が全射のとき(b)は等号成立.特に,$f(f^{-1}(B))=f(X)\cap B$.
    \end{enumerate}
\end{proposition}
\begin{proof}
    どちらも,$x\in A\Rightarrow f(x)\in B$という論理式の表現である.
\end{proof}

\begin{remarks}[随伴の向きについて]
    なお,向きは,包含写像$i:f(A)\to B$の向きで左右が決まっている.
    これは内積空間での関係$\langle Ax,y\rangle=\langle x,By\rangle$から来たものである.
    あえて書くなら,$\bracket{f(A),B}=\bracket{A,f^{-1}(B)}$である.ただし$\bracket{S,T}$とは,$S\subset T$とした.
\end{remarks}

\subsection{像の関手性の破れ}

\begin{tcolorbox}[colframe=ForestGreen, colback=ForestGreen!10!white,breakable,colbacktitle=ForestGreen!40!white,coltitle=black,fonttitle=\bfseries\sffamily,
title=]
    非対称性が破れる理由は,終域に「制限」をかけられないことが1つある.
    これは$f(f^{-1}(B))=f(X)\cap B$なる消息からもわかる(projection formula)という.
    これは相対位相のようなものを考える必然性を生み出す.
\end{tcolorbox}

\begin{problem}[像の値域]
    $f:X\to Y$を写像とする.
    \begin{enumerate}
        \item $\Im f_*=P(\Im f)$.
        \item $\forall_{S\subset X,T\subset Y}\;f(S)\cap T=f(S\cap f^{-1}(T))$.
    \end{enumerate}
\end{problem}
\begin{proof}
    略.
\end{proof}

\begin{proposition}[像は和のみを保つ]\label{prop-image-preserve-unions}
    $f:X\to Y$を写像,$\{S_i\}_{i\in I}\subset P(X)$を族とする.像$f_*:P(X)\to P(Y)$は次を満たす:
    \begin{enumerate}
        \item $f_*(\cup_{i\in I}S_i)=\bigcup_{i\in I}f_*(S_i)$.
        \item $f_*(\cap_{i\in I}S_i)\subset\bigcap_{i\in I}f_*(S_i)$.
    \end{enumerate}
    $I\ne\emptyset$かつ$f$が単射のとき,(2)の等号は成立する.
    また,任意の$\{S_i\}_{i\in I}\subset P(X)$について(2)の等号が成立するならば,$f$は単射である.
\end{proposition}
\begin{proof}\mbox{}
    \begin{enumerate}
        \item a
        \item $\forall_{i\in I}\;\cap_{i\in I}S_i\subset S_i$だから,$\forall_{i\in I}\;f_*(\cap_{i\in I}S_i)\subset f_*(S_i)$.
    \end{enumerate}
\end{proof}

\subsection{逆像の関手性}

\begin{tcolorbox}[colframe=ForestGreen, colback=ForestGreen!10!white,breakable,colbacktitle=ForestGreen!40!white,coltitle=black,fonttitle=\bfseries\sffamily,
title=]
    $f^*$は完備Boole代数$P(X)$の射として完成されている.
\end{tcolorbox}

\begin{proposition}[順序の保存]
    $f:X\to Y$を写像とする.$f_*:P(X)\to P(Y),f^*:P(Y)\to P(X)$はいずれも関手である.
    \begin{enumerate}
        \item $\forall_{A,A'\subset X}\;A'\subset A\Rightarrow f_*(A')\subset f_*(A)$.
        \item $\forall_{B,B'\subset Y}\;B'\subset B\Rightarrow f^*(B')\subset f^*(B)$.
    \end{enumerate}
\end{proposition}

\begin{proposition}[逆像は和と積を保つ]
    $f:X\to Y$を写像,$\{T_i\}_{i\in I}\subset P(Y)$を族とする.像$f^*:P(Y)\to P(X)$は次を満たす:
    \begin{enumerate}
        \item $f^*(\cup_{i\in I}T_i)=\bigcup_{i\in I}f^*(T_i)$.
        \item  $f^*(\cap_{i\in I}T_i)=\bigcap_{i\in I}f^*(T_i)$.
    \end{enumerate}
\end{proposition}

\begin{proposition}[逆像は補集合を保つ]
    $f:X\to Y$を写像,$B\subset Y$を部分集合とする.
    この時,$f^{-1}(Y\setminus B)=X\setminus f^{-1}(B)$.
\end{proposition}

\subsection{可逆性の特徴付け}

\begin{tcolorbox}[colframe=ForestGreen, colback=ForestGreen!10!white,breakable,colbacktitle=ForestGreen!40!white,coltitle=black,fonttitle=\bfseries\sffamily,
title=]
    $P(X)$はBoole代数であるだけでなく,順序による圏とみなせる.
    このときの米田の補題が効いている.
    これが束の構造である.
\end{tcolorbox}

\begin{proposition}[可逆射の普遍性による特徴付け:集合を,他の集合への写像についての述語で特徴付けること]
    $f:X\to Y$を写像とする.次の2条件は同値である.
    \begin{enumerate}
        \item $f$は可逆である.
        \item 任意の集合$Z$に対して,写像$f^*:\Map(Y,Z)\to\Map(X,Z)$は可逆である.
    \end{enumerate}
\end{proposition}
\begin{proof}
    略.
\end{proof}

\begin{remarks}
    圏$P(X)$での米田の補題は,$(A=B)\Leftrightarrow(\forall_{T\in P(X)}\;A\le T\Leftrightarrow B\le T)$という主張になる.
    この完備性を利用したのが「上限」の概念であり,この完備性を利用して有理数を完備化する手法がDedekindの切断でもある.
\end{remarks}

\subsection{全射・単射性の特徴付け}

\begin{tcolorbox}[colframe=ForestGreen, colback=ForestGreen!10!white,breakable,colbacktitle=ForestGreen!40!white,coltitle=black,fonttitle=\bfseries\sffamily,
title=]
    ここは線形代数の理論の力を借りる.$P(X)$を特性関数のなす線型空間と同一視し,$f^*,f_*$をその間の線型写像とみる.
\end{tcolorbox}

\begin{proposition}[全射・単射の特徴付け]\label{prop-dual-maps}
    $f:X\to Y$を写像,$f^*,f_*$を関手とする.
    \begin{enumerate}
        \item $f$が単射であることと,$f_*$が単射であることと,$f^*$が全射であることは同値である.
        \item $f$が全射であることと,$f_*$が全射であることと,$f^*$が単射であることは同値である.
    \end{enumerate}
\end{proposition}
\begin{proof}
    まず,$f^*$について示す.$f_*$についても同様.
    集合$S$に対して,全単射
    \[\xymatrix@R-2pc{
        \paren{\bF_2^{\oplus S}}^\vee\ar[r]&P(S)\\
        \rotatebox[origin=c]{90}{$\in$}&\rotatebox[origin=c]{90}{$\in$}\\
        g\ar@{|->}[r]&\{s\in S\mid g([s])=1\}
    }\]
    が存在することに着目する.ただし,$(-)^\vee$は$\bF_2$-線型空間の双対を表す.

    写像$f:A\to B$は$\bF_2$-線型写像$\varphi_f:\bF_2^{\oplus A}\to\bF_2^{\oplus B}$を誘導し,
    $f$の単射性・全射性は$\varphi_f$の単射性・全射性と同値である.
    また,双対写像$\varphi_f:\paren{\bF_2^{\oplus B}}^\vee\to\paren{\bF_2^{\oplus A}}^\vee$は上記の全単射により$f^*:P(B)\to P(A)$に対応する.
    よって主張は次の補題から従う.
\end{proof}

\begin{lemma}
    $k$を体とする.$\varphi:V\to W$を$k$-線型空間の間の線型写像とし,$\varphi^*:W^\vee\to V^\vee$をその双対写像とする.
    この時,
    \begin{enumerate}
        \item $\varphi$が単射であることは,$\varphi^*$が全射であることと同値である.
        \item $\varphi$が全射であることは,$\varphi^*$が単射であることと同値である.
    \end{enumerate}
\end{lemma}
\begin{proof}\mbox{}
    \begin{enumerate}
        \item $0\to\Ker\varphi\to V\to W\to\Coker\varphi\to 0$は完全列である.
        \item $k$は$k$加群として入射的.即ち,関手$\Hom(-,k)$は完全となる.
        \item $0\to(\Coker\varphi)^\vee\to W^\vee\to V^\vee\to(\Ker\varphi)^\vee\to 0$は完全列である.
    \end{enumerate}
\end{proof}

\subsection{2つの関手の関係}

\begin{tcolorbox}[colframe=ForestGreen, colback=ForestGreen!10!white,breakable,colbacktitle=ForestGreen!40!white,coltitle=black,fonttitle=\bfseries\sffamily,
title=]
    $f$が全射または単射のとき,$f^*,f_*$は互いに部分的に逆になる.
\end{tcolorbox}

\begin{proposition}[全射の双対写像]\label{prop-dual-of-epi}\mbox{}
    \begin{enumerate}
        \item $f$が全射の時,定理\ref{thm-epi}より,split epiだから,右逆射(section) $f^{-1}:Y\to X$が存在して,\[f\circ f^{-1}=\id_Y\Rightarrow (f^{-1})^*\circ f^*=\id^*_Y=\id_{P(Y)}.\]よって,$(f^{-1})^*$は$f^*$の左逆射(retraction)であるという意味で$(f^*)^{-1}$とも表し得る.
        \item $f$が全射の時,同じくsplit monoだから,あるいは命題\ref{prop-functoriality-of-image-and-inverse-image-mappings}.3より,任意の部分集合$B\subset Y$について$f(f^{-1}(B))=\id_Y(B)=B=B\cap f(X)$.従って,$f_*\circ f^*=\id_{P(Y)}$である.
        \item 以上のことを象徴的に表せば,\[((f^*)^{-1}=(f^{-1})^*=)f^{*-1}=f_*.\]
    \end{enumerate}
\end{proposition}

\begin{proposition}[単射の双対写像]\label{prop-dual-of-mono}\mbox{}
    \begin{enumerate}
        \item $f$が単射の時,定理\ref{thm-mono}より,split monoだから,左逆射(retraction) $f^{-1}:Y\to X$が存在して,\[f^{-1}\circ f=\id_X\Rightarrow f^*\circ (f^{-1})^*=\id^*_X=\id_{P(X)}.\]よって,$(f^{-1})^*$は$f^*$の右逆射(section)であるという意味で$(f^*)^{-1}$とも表し得る.
        \item $f$が単射の時,同じくsplit monoだから,任意の部分集合$A\subset X$について$f^{-1}(f(A))=\id_X(A)=A$.従って,$f^*\circ f_*=\id_{P(X)}$である.
        \item 以上のことを象徴的に表せば,\[((f^*)^{-1}=(f^{-1})^*=)f^{*-1}=f_*.\]
    \end{enumerate}
\end{proposition}


\subsection{双対性概観}

\begin{tcolorbox}[colframe=ForestGreen, colback=ForestGreen!10!white,breakable,colbacktitle=ForestGreen!40!white,coltitle=black,fonttitle=\bfseries\sffamily,
title=]
    $2$への線型写像を通じて,集合論と線形代数との双対理論が相即相入する.
\end{tcolorbox}

\begin{tcolorbox}[breakable,coltitle=white,fonttitle=\bfseries\sffamily,
    title=双対写像への全射と単射の持ち越し]
    前の節の逆像写像についての内容を一般化する.
    定理\ref{thm-mono},\ref{thm-epi}より,次のHom関手について,
    \begin{align*}
        fが単射(左簡約可能)&\Leftrightarrow f_*が単射(左簡約可能),&fが全射(右簡約可能)&\Leftrightarrow f^*が単射\\
        &\Leftrightarrow f^*が全射,&&\Leftrightarrow f_*が全射
    \end{align*}
    は,1行目がすぐに判り\footnote{\url{https://ncatlab.org/nlab/show/monomorphism}での特徴付け4つのうちの1つに含まれている},2行目は双対原理から来る.
    $f^*$が$C$でmonic / epicであることと,$f^*$が$C^{op}$でepic / monicであることが同値なのである.$C^{op}$での$f^*$とは,postcomposition $f_*$に他ならない.
    よって,このことは単射/全射を,monic / epicに変えても一般の圏にて成り立つ.
    \[\xymatrix@R-2pc{
        f_*:\Map(Z,X)\ar[r]&\Map(Z,Y)&f^*:\Map(Y,Z)\ar[r]&\Map(X,Z)\\
        \rotatebox[origin=c]{90}{$\in$}&\rotatebox[origin=c]{90}{$\in$}&\rotatebox[origin=c]{90}{$\in$}&\rotatebox[origin=c]{90}{$\in$}\\
        g\ar@{|->}[r]&f\circ g&g\ar@{|->}[r]&g\circ f
    }\]
\end{tcolorbox}

\begin{theorem}
    $f:X\to Y$を写像とする.$Z$を任意の集合として,$f^*:\Map(Y,Z)\to\Map(X,Z)$を反変Hom関手,$f_*:\Map(Z,X)\to\Map(Z,Y)$を共変Hom関手とする.
    \begin{enumerate}
        \item $f:X\to Y$が単射である$\quad\Leftrightarrow\quad f^*$が全射である.
        \item $f:X\to Y$が全射である$\quad\Leftrightarrow\quad f^*$が単射である.
        \item $f:X\to Y$が単射である$\quad\Leftrightarrow\quad f_*$が単射である.
        \item $f:X\to Y$が全射である$\quad\Leftrightarrow\quad f_*$が全射である.
    \end{enumerate}
\end{theorem}
\begin{proof}\mbox{}
    \begin{enumerate}
        \item \begin{align*}
            fが単射&\quad\Leftrightarrow\quad fが左簡約可能&(定理\ref{thm-mono})\\
            &\quad\Leftrightarrow\quad f^*が右簡約可能&(後述)\\
            &\quad\Leftrightarrow\quad f^*が全射&(定理\ref{thm-epi})
        \end{align*}
        であるが,$\Rightarrow$は,$r\circ f=\id_X$を満たす$f$のretraction $r$に対して,$f^*\circ r^*=\id_{\Map(X,Z)}$を満たす$r^*:\Map(X,Z)\to\Map(Y,Z)$が見つかる.
        $\Leftarrow$は,$Z=X$とし,$f^*\circ r=\id_{P(X)}$を満たす$r:\Map(X,X)\to\Map(Y,X)$に対して,$r(\id_X)$が$f$のretractionとなる.実際,$f(r(\id_X))=\id_X$より,$r(\id_X)\circ f=\id_X$を得る.
        \item すでに述べた.
        \item すでに述べた.
        \item 
    \end{enumerate}
\end{proof}
\begin{remarks}
    おそらくこういうことであろう.しかし,epimorphismの定義は,任意の対象$Z$と平行な射の組$g_1,g_2:Y\to Z$について,$(g_1\circ f=g_2\circ f)\Rightarrow(g_1=g_2)$であり,これはhom-functor $\Hom(-,Z)$が単射であることの定義に他ならない.
    Sets上では単射とepimorphismが同値なので,上の定理が成り立つ.

\end{remarks}

$2$の役割を入れ替えることによる双対が起こる.
de Morgan dualityが開集合の閉集合の双対を引き起こしていて,
第\ref{sec-duality-of-opens-and-closed}節の主眼である.
これと,反対圏が生み出す双対とどのような関係があるのだろうか?

\section{普遍構成}

\subsection{和と積}

\begin{proposition}[積の普遍性]
    $(X_i)_{i\in I}$を集合族,$X:=\prod_{i\in I}X_i$とする.任意の集合$T\in\Set$と写像の族$(f_i:T\to X_i)_{i\in I}$に対して,ある$f:T\to X$がただ一つ存在して,次を満たす:$\forall_{i\in I}\;f_i=\pr_i\circ f$.
\end{proposition}

\subsection{等化子}

\begin{proposition}[等化子の普遍性:単射と一般の写像]
    $i:X\to Y$を単射,$T$を勝手な集合,$f:T\to Y$を写像とする.次の2つの条件は同値である.
    \begin{enumerate}
        \item $f(T)\subset i(X)$である.
        \item 下の図式を可換にする写像$g:T\to X$が一意的に存在する.
        \[\xymatrix{
            X\;\ar@{^{(}->}[r]^-i&Y\\
            T\ar@{.>}[u]^-g\ar[ur]_-f
        }\]
    \end{enumerate}
\end{proposition}
\begin{remarks}
    $f(T)\supsetneq i(X)$の時,$g$をどう取っても$f(T)\setminus i(X)\ne\varnothing$となってしまうため,写像として一致し得ない.
\end{remarks}

\subsection{余等化子}

\begin{proposition}[全射と一般の写像]\label{prop-induced-mapping}
    $X,Y,Z$を集合,$p:X\to Y$を全射,$f:X\to Z$を写像とする.
    \begin{enumerate}
        \item 次の条件(1)と(2)は同値である.
        \begin{enumerate}[(1)]
            \item $f=g\circ p$を満たす写像$g:Y\to Z$が存在する.\begin{center}\begin{tikzcd}
                X \ar[r, twoheadrightarrow, "p"] \ar[dr, "f"'] & Y \ar[d, dashed, "g"]\\
                & Z
            \end{tikzcd}\end{center}
            \item 全射$p$が定める同値関係$R_p$は,写像$f$が定める同値関係$R_f$よりも細かい:$C_{R_p}\subset C_{R_f}$.
        \end{enumerate}
        \item いま,$R_p$が$R_f$よりも細かいとする.この時,次の2つの条件は同値である.
        \begin{enumerate}[(1)]
            \item $f=g\circ p$を満たすこの$g:Y\to Z$は単射である.
            \item $R_p$と$R_f$は同値である.
        \end{enumerate}
    \end{enumerate}
\end{proposition}
\begin{proof}
    \begin{description}
        \item[1. ] $(1)\Rightarrow (2)$は\[ \forall x,x'\in X ,\; p(x)=p(x')\Rightarrow f(x)=f(x') \]を示せば良い.いま,実際$p(x)=p(x')$を満たす$x,x'\in X$について,$q(p(x))=q(p(x'))$であるから,$f(x)=f(x')$が従う.

        次に$(2)\Rightarrow (1)$を考える.写像$g$を構成するために,写像
        \begin{center}\begin{tikzcd}
            (p,f):X \ar[r] \ar[d, phantom, "\rotatebox{90}{$\in$}"] & Y\times Z \ar[d, phantom, "\rotatebox{90}{$\in$}"] \\
            x \ar[r, mapsto] & (p(x),f(x))
        \end{tikzcd}\end{center}
        を考える.この値域$(p,f)(X)=\{ (p(x),f(x))\mid x\in X \}=:\Gamma_g$は(A)写像のグラフとなっており,そして(B)このグラフが定める写像$(Y,Z,\Gamma_g)=:g$が求める唯一つの写像であることを示す.
    
        (B)については,全ての$x\in X$について,$g$の定め方より$g(p(x))=f(x)$が成り立つから,確かにこれは$f=g\circ p$を満たす写像である.
    
        (A) $\Gamma_g$が写像のグラフとなっていることの証明を,$\mathrm{pr}_1:Y\times Z\to Y$を第一射影として,$\mathrm{pr}_1|_{\Gamma_g}$が全単射であることを示すことによって行う.
        $\mathrm{pr}_1|_{\Gamma_g}\circ (p,f)=id_Y\circ p=p$より,$p$は全射であるから$\mathrm{pr}_1|_{\Gamma_g}$も全射である.また,$(y,z),(y',z')\in\Gamma_g$について$\mathrm{pr}_1(y,z)=\mathrm{pr}_1(y',z')$即ち$y=y'$即ち
        $\exists x,x'\in X \,\mathrm{s.t.}\, p(x)=p(x')$ならば,$R_p\subset R_f$より,$f(x)=f(x')$即ち$z=z'$より,$\mathrm{pr}_1|_{\Gamma_g}$は単射でもある.
        \item[2. ] $(2)\Rightarrow(1)$. $R_p=R_f$の時,$X/R_p=X/R_f$であるから,$p,f$の標準分解は,可逆写像$\tilde{p}:X/R_p\to Y$と単射$\overline{f}:X/R_p\to Z$を定める.
        \begin{center}\begin{tikzcd}
            X \ar[r, "p"] \ar[d, "q"'] \ar[dr, "f"' near end, "\circlearrowright"' near start] & Y \ar[d, "g"] \\
            X/R_p \ar[ur, "\tilde{p}" near end, "\circlearrowright"' near start] \ar[r, "\overline{f}"'] \ar[d, dashed, "\tilde{f}"] & Z \\
            f(X) \ar[ur, dashed, "i"']
        \end{tikzcd}\end{center}
        この図式は結局全体として可換であり($f=g\circ p$かつ$f=\overline{f}\circ q$より,$\overline{f}\circ q=g\circ p$を得る.これと$p=\tilde{p}\circ q$より),$\overline{f}\circ\tilde{p}^{-1}=g$となる.従って$g$は全射である.
    
        $(1)\Rightarrow(2)$.$g$が単射ならば,$g(y)=g(y')\Rightarrow y=y'$より,
        \begin{eqnarray*}
            p(x)=p(x') &\Leftrightarrow& g(p(x))=g(p(x')) \\
            &\Leftrightarrow& f(x)=f(x')
        \end{eqnarray*}
        より,$R_f=R_p$である.
    \end{description}
\end{proof}

\subsection{商集合の普遍性}

\begin{corollary}[商集合の普遍性]\label{cor-universality-of-quotient-set}
    $R$を集合$X$上の同値関係とし,$q:X\to X/R$を商写像とする.
    \begin{enumerate}
        \item 写像$f:X\to Y$について,次の2条件は同値である.
        \begin{enumerate}[(1)]
            \item 次の図式を可換にする写像$g:X/R\to Y$が存在する.これは$f$によって引き起こされた写像である.\begin{center}\begin{tikzcd}
                X \ar[r, "q"] \ar[dr, "f"'] & X/R \ar[d, "g"] \\
                & Y
            \end{tikzcd}\end{center}
            \item $R$は,$f$が定める同値関係$R_f$より細かい.
        \end{enumerate}
        \item $R'$を$Y$の同値関係とし,$q':Y\to Y/R'$を商写像とする.写像$f:X\to Y$に対して,次の2条件は同値である.
        \begin{enumerate}[(1)]
            \item 写像$g:X/R\to Y/R'$で,次の図式を可換にするものが存在する.
            \begin{center}\begin{tikzcd}
                X \ar[r, "f"] \ar[d, "q'"'] & Y \ar[d, "q'"] \\
                X/R \ar[r, "g"'] & Y/R'
            \end{tikzcd}\end{center}
            \item $C\subset X\times X$を$R$のグラフとし,$C'$を$R'$のグラフとすると,$C\subset (f\times f)^{-1}(C')$である.
        \end{enumerate}
    \end{enumerate}
\end{corollary}
\begin{proof}
    1. 全射$p$について命題\ref{prop-induced-mapping}を適用して得る主張である.なお,$q$が定める同値関係$R_q$とは$R$に他ならない.
    
    2. 全射$q'\circ f$について命題\ref{prop-induced-mapping}を適用して得る主張である.
\end{proof}

\subsection{貼り合わせ}

\begin{theorem}
    $X$を集合,$(X_i)_{i\in I}$を被覆とする.
    ある集合$Y$への写像の族$(f_i:X_i\to Y)_{i\in I}$について,次の2条件は同値.
    \begin{enumerate}
        \item ある$f:X\to Y$が存在して,$\forall_{i\in I}\;f|_{X_i}=f_i$.
        \item $\forall_{i,j\in I}\;f_i|_{X_i\cap X_j}=f_J|_{X_i\cap X_j}$.
    \end{enumerate}
    このとき,$f$は一意的に存在する.
\end{theorem}

\section{写像の標準分解}

\subsection{標準分解}

\begin{proposition}[canonical decomposition]
    $f:X\to Y$を写像とする.
    \begin{enumerate}
        \item 次の図式を可換にする写像$\overline{f}$が唯一つ存在する.この分解$f=i\circ\overline{f}\circ q$を\textbf{$f$の標準分解}という.\begin{center}\begin{tikzcd}
            X \ar[r, "f"] \ar[d, "q"'] & Y\\
            X/R_f \ar[r, dotted, "\overline{f}"] & f(X)\ar[u, "i"']
        \end{tikzcd}\end{center}
        \item このとき,写像$\overline{f}$は可逆である.この$\overline{f}$を\textbf{$f$によって引き起こされる可逆写像}と呼ぶ.
        \item $f$が定める同値関係$R_f$についての商集合$X/R_f$を,\textbf{$f$の余像}と呼ぶ.
    \end{enumerate}
\end{proposition}

\subsection{単射と全射の特徴付け}

\begin{proposition}[全射と単射の特徴付け]
    次の3条件は同値である.
    \begin{enumerate}
        \item $f$は単射である.
        \item $f$が定める同値関係$R_f$は相等関係と同値である.
        \item $f$が定める写像$X\to f(X)$は可逆である.
    \end{enumerate}
    次の3条件は同値である.
    \begin{enumerate}
        \item $f$は全射である.
        \item $f(X)=Y$である.
        \item $X$の同値関係$R$と商集合からの可逆写像$\overline{f}:X/R\to Y$で,$q:X\to X/R$を商写像とすると,$f=\overline{f}\circ q$を満たすものが存在する.
    \end{enumerate}
\end{proposition}

\subsection{単射と全射の十分条件}

\begin{tcolorbox}[colframe=ForestGreen, colback=ForestGreen!10!white,breakable,colbacktitle=ForestGreen!40!white,coltitle=black,fonttitle=\bfseries\sffamily,
title=]
単射は左$q$退化の事象,全射は右$i$退化の事象だと知っていれば,次は明らかに思えてくる.
\end{tcolorbox}

\begin{lemma}[単射は左$q$退化の事象,全射は右$i$退化の事象]
    $f:X\to Y, g:Y\to Z$を写像とする.
    次の条件について,1$\Rightarrow$2$\Rightarrow$3が成り立つ.
    \begin{enumerate}
        \item $f$と$g$は単射である.
        \item $g\circ f$は単射である.
        \item $f$は単射である.
    \end{enumerate}
    次の条件について,1$\Rightarrow$2$\Rightarrow$3が成り立つ.
    \begin{enumerate}
        \item $f$と$g$は全射である.
        \item $g\circ f$は全射である.
        \item $g$は全射である.
    \end{enumerate}
\end{lemma}

\begin{proposition}
    写像$f:X\to Y,g:Y\to Z$について,次の2条件は同値である.
    \begin{enumerate}
        \item $f\circ g$は可逆である.
        \item $g$が単射で$f$が全射である.
    \end{enumerate}
\end{proposition}

\begin{tbox}{red}{モノ射とエピ射まとめ}
    \begin{theorem}[mono]\label{thm-mono}
        以下は全て写像$f:X\to Y$が単射であることの同値な定義である.
        \begin{enumerate}
            \item [像/逆像の言葉]$\forall x\in X f^{-1}(f(x))=\{x\}$.
            \item [その論理的変形・大域化]$\forall A\subset X f^{-1}(f(A))=A$(雪江群論).
            \item [左一意性]$f$が定める同値関係$R_f$は相等関係と同値である.(関係が一致するとはグラフが一致することと定義した).
            \item [標準分解の言葉]$f$が定める写像$X\to f(X)$は可逆になる.
            \item [左簡約可能:monic]$g\circ f=\id_X$を満たす写像$g:Y\to X$が存在する.または,$X=\emptyset$である.
            \item [関手性]任意の$A_1,A_2\subset X$に対して,$f_*(A_1\cap A_2)=f_*(A_1)\cap f_*(A_2)$\ref{prop-image-preserve-unions}.
            \item [像関手の左簡約可能性]任意の${A\subset X}$に対して,$f^{-1}(f(A))=A$\ref{prop-adjunction}.
        \end{enumerate}
    \end{theorem}
    \begin{theorem}[epi (AC)]\label{thm-epi}
        以下は全て写像$f:X\to Y$が全射であることの同値な定義である.
        \begin{enumerate}
            \item [逆像の言葉]$\forall y\in Y f^{-1}(y)\ne\emptyset$.
            \item [右全域性]$f(X)=Y$.
            \item [標準分解の言葉]$f=\overline{f}\circ q$となる可逆写像$\overline{f}$が存在する.
            \item [右簡約可能:epic]$f\circ g=\id_Y$を満たす写像$g:Y\to X$が存在する.
        \end{enumerate}
    \end{theorem}
\end{tbox}

\section{実数の位相}

\subsection{点列の収束}

\begin{tcolorbox}[colframe=ForestGreen, colback=ForestGreen!10!white,breakable,colbacktitle=ForestGreen!40!white,coltitle=black,fonttitle=\bfseries\sffamily,
title=]
    Dedekindの構成により,$P(\R)$上に$\inf,\sup:P(\R)\to\o{\R}$が定まる.
    ここから,まず点列の収束が定義できる.
    すると,距離空間の点列の収束も定義できたことになる!
\end{tcolorbox}

\begin{definition}
    数列$\{x_n\}\subset\R$が$a\in\R$に収束するとは,
    \[\inf_{m\ge0}\paren{\sup_{n\ge m}\abs{x_n-a}}=0\]
    を満たすことをいう.
    これを用いて,$\R^n$上の収束は,距離$d:\R^n\times\R^n\to\R_+$を用いて定義できる.
\end{definition}

\begin{proposition}[点列の収束の位相的特徴付け]\label{prop-characterization-of-convergence-in-metric-spaces}
    $(x_m)\in{}^{<\omega}\mathbb{R}^n, a\in\mathbb{R}^n$とする.次の3条件は同値である.
    \begin{enumerate}
        \item $\lim_{m\to\infty}x_m=a$.すなわち,$\lim_{m\to\infty}d(x_m,a)=0$.
        \item $\forall r\in\mathbb{R}_{>0}\;\exists l\in\mathbb{N}\;\forall m\in\mathbb{N}_{\ge n}\; d(x_m,a)<r$.
        \item $a$を元として含む任意の開集合$U\subset\mathbb{R}^n$について,$\{m\in\mathbb{N}\mid x_m\notin U\}$は有限集合である.
    \end{enumerate}
    条件3.を「十分大きな$n$について$x_m\in U (m\ge n)$である,ということがある.
\end{proposition}

\subsection{連続写像}

\begin{tcolorbox}[colframe=ForestGreen, colback=ForestGreen!10!white,breakable,colbacktitle=ForestGreen!40!white,coltitle=black,fonttitle=\bfseries\sffamily,
title=]
    $\R$の位相が構成できたならば,任意の距離空間上において,次のような方法で写像の連続性が定義できる.
\end{tcolorbox}

\begin{definition}
    開集合$U\subset \R^n$について,
    \begin{enumerate}
        \item $f:U\setminus\{a\}\to\R^m$を写像とする.$b=\lim_{x\to a}f(x)$とは,次のことをいう:
        \[ \inf_{r>0}\left(\sup_{x\in U,\\0<d(x,a)<r}d(f(x),b)\right)=0. \]
        \item 写像$f:U\to\R^m$が$a\in U$で\textbf{連続}であるとは,$f(a)=\lim_{x\to a}f(x)$であることをいう.
        \item $f:U\to\R^m$が\textbf{連続写像}であるとは,全ての$x\in U$において$f$が連続であることをいう.
    \end{enumerate}
\end{definition}

\begin{proposition}\mbox{}
    \begin{enumerate}
        \item 加法$+:\R^2\to\R$と乗法$\cdot:\R^2\to\R$と逆元${}^{-1}:\R\setminus\{0\}\to\R$は連続である.
        \item 射影$\pr_i:\R^m\to\R\;(i\in[m])$は連続である.次の2条件は同値である.
        \begin{enumerate}[(1)]
            \item 写像$f:U\to\R^m$は$a$で連続である.
            \item $f_i:=\pr_i\circ f:U\to\R\;(i\in[m])$はそれぞれ$a$で連続である.
        \end{enumerate}
    \end{enumerate}
\end{proposition}

\begin{proposition}[連続写像の開集合による特徴付け]\label{prop-characterization-of-continuous-map-in-metric-spaces}
    $f:U\to\R^m$を開集合上の写像とする.
    \begin{enumerate}
        \item $a\in U$に対し,次の3条件は同値である.
        \begin{enumerate}[(1)]
            \item 写像$f:U\to\R^m$は$a$で連続である.
            \item $\forall q>0,\;\exists r>0,\; \forall x\in U,\; d(x,a)<r\Rightarrow d(f(x),d(a))<q$.
            \item $f(a)\in V$を満たす任意の開集合$V\subset\R^m$に対し,$a\in W\subset f^{-1}(V)$を満たす開集合$W\subset\R^n$が存在する.
        \end{enumerate}
        \item  次の2条件は同値である.
        \begin{enumerate}[(1)]
            \item $f:U\to\R^m$は連続である.
            \item 任意の開集合$V\subset\R^m$について,逆像$f^{-1}(V)\subset U$は$\R^n$の開集合である.
        \end{enumerate}
    \end{enumerate}
\end{proposition}

\begin{proposition}[連続写像の点列による特徴付け]
    $\R^n$-開集合上の写像$f:U\to\R^m$と$a\in U$に対し,次の2条件は同値.
    \begin{enumerate}
        \item $f:U\to\R^m$は$a$で連続である.
        \item $U$の点列$(x_k)$が$a$に収束するならば,$\R^m$の点列$(f(x_k))$は$f(a)$に収束する.
    \end{enumerate}
\end{proposition}

\section{関数の連続性}

\subsection{連続性の特徴付け}

\begin{tcolorbox}[colframe=ForestGreen, colback=ForestGreen!10!white,breakable,colbacktitle=ForestGreen!40!white,coltitle=black,fonttitle=\bfseries\sffamily,
title=]
    凸関数の特徴付けと似た消息が成り立つ.
\end{tcolorbox}

\begin{proposition}
    一般の位相空間$X$上の
    関数$f:X\to\R$について,次の2条件は同値:
    \begin{enumerate}
        \item $f$は連続.
        \item $U:=\Brace{(x,y)\in X\times\R\mid f(x)<y}$と$V:=\Brace{(x,y)\in X\times\R\mid f(x)>y}$はいずれも$X\times\R$の開集合.
    \end{enumerate}
\end{proposition}
\begin{proof}
    $\{(-\infty,c)\}_{c\in\R}\cup\{(c,\infty)\}_{c\in\R}$は$\R$の位相の準基であるためである.
    Borel可測性が成り立つには,さらに$\{(-\infty,c)\}_{c\in\R}$で十分である.
\end{proof}

\subsection{上・下半連続性}

\begin{tcolorbox}[colframe=ForestGreen, colback=ForestGreen!10!white,breakable,colbacktitle=ForestGreen!40!white,coltitle=black,fonttitle=\bfseries\sffamily,
title=]
    ハイパーグラフが凸集合な関数が凸関数,閉集合な関数が下半連続関数である.
    すでにFatouの補題のような消息が現れている.
\end{tcolorbox}

\begin{definition}[upper semicontinuous, right continuous]
    $X$を位相空間とする.
    \begin{enumerate}
        \item 実数値関数$f:X\to\R$が\textbf{上半連続}であるとは,$\R$の位相$\{(-\infty,t)\mid t\in\R\}\cup\{\R,\emptyset\}$について連続であることをいう.
        \item 実数上の関数$f:\R\to X$が\textbf{右連続}であるとは,$\Brace{[s,t)\mid s,t\in\R}$が生成する$\R$の位相について連続であることをいう.
    \end{enumerate}
\end{definition}
\begin{remark}
    上半連続性の定義は,$f^{-1}((-\infty,t))=\Brace{x\in X\mid f(x)<t}$が開であること.
    右連続性の定義は$\Brace{[s,t)\mid s,t\in\R}$は,$s$に右から近づく点列を収束と判定するためである.
\end{remark}

\begin{proposition}[特性関数の連続性]
    $A\subset X$について,
    \begin{enumerate}
        \item $A$が開であることと,$1_A$が下半連続であることとは同値.
        \item $A$が閉であることと,$1_A$が上半連続であることとは同値.
        \item $A$が開かつ閉であることと,$1_A$が連続であることとは同値.
    \end{enumerate}
    ただし,$2$には離散位相を入れた.
\end{proposition}

\begin{proposition}[上半連続性の特徴付け]
    関数$f:X\to\R$について,次の3条件は同値:
    \begin{enumerate}
        \item $f$は上半連続である.
        \item $\forall_{a\in X}\;\forall_{r>0}\;\exists_{U\in\O(a)}\;U\subset\Brace{x\in X\mid f(x)<f(a)+r}$.
        \item $X$の収束ネット$(x_\lambda)$について,$\limsup_\lambda f(x_\lambda)\le f(\lim x_\lambda)$.
    \end{enumerate}
    $X$は$C(X;[0,1])$によって分離されるとき(例えば正規空間のとき),次も同値:
    \begin{enumerate}\setcounter{enumi}{2}
        \item $f$はある連続関数の族の下限である.
    \end{enumerate}
\end{proposition}

\begin{theorem}[下半連続性の特徴付け]
    関数$f:X\to\R$について,次の2条件は同値:
    \begin{enumerate}
        \item $f$は下半連続である.
        \item ハイパーグラフ$G^+:=\Brace{(x,t)\in X\times\R\mid f(x)\le t}$は閉集合である.
        \item $X$の収束ネット$(x_\lambda)$について,$\limsup_\lambda f(x_\lambda)\le f(\lim x_\lambda)$.
    \end{enumerate}
    $X$は$C(X;[0,1])$によって分離されるとき(例えば正規空間のとき),次も同値:
    \begin{enumerate}\setcounter{enumi}{2}
        \item $f$はある連続関数の族の下限である.
    \end{enumerate}
\end{theorem}

\begin{proposition}[上半連続性の保存]
    $(f_i:X\to\R)_{i\in I}$は上半連続な関数の族とする.
    $f:X\to\R$が$f(x):=\inf_{i\in I}f_i(x)$によって定まるならば,$f$も上半連続である.
\end{proposition}

\subsection{下半連続関数の空間}

\begin{tcolorbox}[colframe=ForestGreen, colback=ForestGreen!10!white,breakable,colbacktitle=ForestGreen!40!white,coltitle=black,fonttitle=\bfseries\sffamily,
    title=]
    $X$上の下半連続な関数全体の集合を$C^{1/2}(X)$とする.
    すると,frameの構造が受け継がれ,$\Op(X)$を関数空間として実現できる.
\end{tcolorbox}

\begin{proposition}\mbox{}
    \begin{enumerate}
        \item $\paren{C^{1/2}(X),\bigvee,\land}$はframeである:任意の各点上限と,有限個の各点下限について閉じている.
        \item $C^{1/2}(X)$は和と$\R$-倍について閉じている.
        \item $C^{1/2}(X)$は一様収束位相について閉じている.
    \end{enumerate}
\end{proposition}

\section{実数上の関数の連続性}

\begin{tcolorbox}[colframe=ForestGreen, colback=ForestGreen!10!white,breakable,colbacktitle=ForestGreen!40!white,coltitle=black,fonttitle=\bfseries\sffamily,
title=]
    実数上の関数には,$\al\in(0,1)$について,
    \[C^1([a,b])\subset\Lip[a,b]\subset\Lip^\al[a,b]\subset UC([a,b])\subset C([a,b])\]
    なる関係と
    \[\Lip[a,b]\subset\AC[a,b]\subset\BV[a,b]\subset\{f\text{は殆ど至る所微分可能}\}\]
    なる関係がある.
\end{tcolorbox}

\begin{definition}
    $f:X\to Y$を実数の部分集合の間の関数とする.
    $\om_f(t):=\sup_{d(x,y)<t}d(f(x),f(y))$によって
    定まる関数$\om_f:[0,\infty]\to[0,\infty]$を
    \textbf{連続度}という.
\end{definition}
\begin{remarks}\mbox{}
    \begin{enumerate}
        \item $\om_f(0)=0$が常に成り立つが,この点において連続であることが,関数$f$が一様連続であることに同値.
        \item 連続度が線型に抑えられるとき,$f$をLipschitz連続関数という.
        \item さらに弱く,$F(x)=x^\al\;(\al\in(0,1))$で抑えられるとき,$f$を$\al$-Holder連続関数という.
    \end{enumerate}
\end{remarks}

\subsection{右・左連続性}

\subsection{Lipschitz連続関数の$C$上稠密性}

\begin{tcolorbox}[colframe=ForestGreen, colback=ForestGreen!10!white,breakable,colbacktitle=ForestGreen!40!white,coltitle=black,fonttitle=\bfseries\sffamily,
title=]
    これは,一般の距離空間の間の写像について定義される連続性概念である.
\end{tcolorbox}

\begin{definition}[Lipschitz continuous function]
    $0<\al\le1,f:[a,b]\to\C$とする.
    \begin{enumerate}
        \item 一様連続度を$\om_\delta(f):=\sup\Brace{\abs{f(x)-f(y)}\in\R_+\mid\forall_{x,y\in I}\;\abs{x-y}\le\delta}$と定める.
        \item $L(f):=\sup_{s\ne t\in[a,b]}\frac{\abs{f(s)-f(t)}}{\abs{s-t}^\al}=\sup_{\delta>0}\frac{\om_\delta(f)}{\delta^\al}<\infty$が成り立つとき,$f$は\textbf{$\al$-Holder連続}であるといい,$L(f)$を$\al$-次のLipschitz定数という.その関数全体の空間を$\Lip^\al([a,b])$で表す.
    \end{enumerate}
\end{definition}

\begin{proposition}[Lipschitz norm]
    次のノルムは互いに同値で,$\Lip^\al([a,b])$はBanach空間をなす.
    \begin{enumerate}
        \item $\norm{f}:=L(f)+\sup_{x\in[a,b]}\abs{f(x)}$.
        \item $\nnorm{f}:=L(f)+\abs{f(a)}$.
    \end{enumerate}
    また,次は$\Lip^\al([a,b])$の閉部分空間となる.
    \[\lip^\al([a,b]):=\Brace{f\in\Lip^\al\mid\lim_{\delta\to0}\frac{\om_\al(f)}{\delta^\al}=0}.\]
\end{proposition}

\begin{proposition}
    $f_n:S\to T$を距離空間の間のLipschitz連続写像の列で,$\exists_{M\in\R}\;\forall_{n\in\N}\;L(f_n)\le K$が成り立つとする.
    \begin{enumerate}
        \item $(f_n)$がある$f$に一様収束するならば,$f$もまた$L(f)<M$を満たすLipschitz写像である.
        \item $S$がコンパクトならば,$\Brace{f\in\Lip^1(S;T)\mid L(f)\le M}$なる集合は,Banach空間$C(S;T)$の閉凸で局所コンパクトな部分集合となる.
        \item また,$S$がコンパクトならば,$\Lip^1(S;T)$は$C(S;T)$上稠密である.
    \end{enumerate}
\end{proposition}
\begin{remarks}
    この命題の条件は,族$\{f_n\}$が一様に同程度連続であるための十分条件を与えるためである.
\end{remarks}


\begin{corollary}
    $(f_a)$は共通のLipschitz定数を持つLipschitz連続関数の族とする.
    $\sup_a f_a,\inf_af_a$は$\{\infty,-\infty\}$に値を取る定値関数でない限り,やはり同じLipschitz定数を持つLipschitz連続関数となる.
\end{corollary}

\subsection{Lipschitz連続関数は本質的に有界な導関数を持つ}

\begin{proposition}[至る所微分可能な関数がLipschitz連続であることの特徴付け]
    $f:\R\to\R$は至る所微分可能とする.このとき,次の2条件は同値:
    \begin{enumerate}
        \item $f$はLipschitz連続である.
        \item $f$の導関数は有界である.
    \end{enumerate}
\end{proposition}

\begin{proposition}[Lipschitz連続関数の導関数はLipschitz定数をノルムとして本質的有界]
    $f:\R\to\R$をLipschitz連続とする.
    \begin{enumerate}
        \item $f$は絶対連続であり,したがって殆ど至る所微分可能である.
        \item $f'\in L^\infty(\R)$で,$\norm{f'}_{L^\infty(\R)}=L(f)$である.
    \end{enumerate}
\end{proposition}

\subsection{絶対連続関数}

\begin{tcolorbox}[colframe=ForestGreen, colback=ForestGreen!10!white,breakable,colbacktitle=ForestGreen!40!white,coltitle=black,fonttitle=\bfseries\sffamily,
title=]
    絶対連続関数は殆ど至る所微分可能であり,さらにLipschitz連続であるためにはその導関数が有界であることが必要十分.
    なお,絶対連続な関数は,不定積分として表現される関数のクラスにほかならない.
\end{tcolorbox}

\begin{definition}
    $f:[a,b]\to\R$が\textbf{絶対連続}であるとは,
    \[\forall_{\ep>0}\;\exists_{\delta>0}\;\forall_{n\in\N}\;\forall_{\{x_i\}_{i\in[2n]}\subset [a,b]}\;\sum_{i=1}^n(x_{2i}-x_{2i-1})<\delta\Rightarrow\sum_{i=1}^n\abs{f(x_{2i})-f(x_{2i-1})}<\ep\]
    が成り立つことをいう.
\end{definition}

\begin{theorem}
    $f:[a,b]\to\R$について,次の4条件は同値.
    \begin{enumerate}
        \item $f$は絶対連続である.
        \item $\exists_{g\in L^1([a.b])}\forall_{x\in[a,b]}\;f(x)=f(a)+\int_a^xg(t)dt$.
        \item $f$は一様連続かつ有界変動で,$[a,b]$の零集合を零集合に写す.
        \item Stieltjes測度$df$はLebesgue測度$l$に関して絶対連続:$df\ll l$.
    \end{enumerate}
    また,この同値な条件が成り立つとき,Lebesgue (1904)より$\paren{\int^x_ag(t)dt}'=g(x)$だから,$f$は殆ど至る所微分可能であることが従う.
\end{theorem}

\begin{example}
    Cantor関数は連続で殆ど至る所微分可能で可積分な導関数を持つが,絶対連続ではない.
\end{example}

\subsection{一様連続関数と連続度}

\begin{tcolorbox}[colframe=ForestGreen, colback=ForestGreen!10!white,breakable,colbacktitle=ForestGreen!40!white,coltitle=black,fonttitle=\bfseries\sffamily,
title=]
    連続度が線型に抑えられるとき,$f$をLipschitz連続関数という.
    さらに弱く,$F(x)=x^\al\;(\al\in(0,1))$で抑えられるとき,$f$を$\al$-Holder連続関数という.
\end{tcolorbox}

\begin{definition}\mbox{}
    \begin{enumerate}
        \item 関数$f:I\to\R$が一様連続であるとは,$\forall_{\ep>0}\;\exists_{\delta>0}\;\forall_{x,y\in I}\;\abs{x-y}<\delta\Rightarrow\abs{f(x)-f(y)}<\ep$が成り立つことをいう.
        \item $\om_f(t):=\sup_{d(x,y)<t}d(f(x),f(y))$によって
        定まる単調増加関数$\om_f:[0,\infty]\to[0,\infty]$を
        \textbf{連続度}という.
        \item これによると,一様連続であるとは,$\om_f$は$0$にて連続であることに同値:$\lim_{\delta\to0}\om_f(\delta)=0$.
    \end{enumerate}
\end{definition}
\begin{remarks}
    $\forall_{x,y\in I}\;\abs{f(x)-f(y)}\le CF(\abs{x-y})$と,ある単調な関数$F$で一様に評価出来ているとき,$\delta:=F^{-1}\paren{\frac{\ep}{C}}$と定めれば良い.
    $F(x)=x^\al$は,$0<\al<1$のときは,$0$に十分近い$x$では接線の傾きが$1$より大きいから,線型の場合よりも収束が遅い.
    よって,Lipschitz連続ならば,$\al$-Holder連続$(0<\al<1)$である.
\end{remarks}



\section{空間上の関数}


\subsection{全微分可能性}

\begin{tcolorbox}[colframe=ForestGreen, colback=ForestGreen!10!white,breakable,colbacktitle=ForestGreen!40!white,coltitle=black,fonttitle=\bfseries\sffamily,
title=]
    $C^1$-級関数は全微分可能であった.この十分条件は弱められる.
    Lipschitz連続ならば,導関数は存在すれば有界になる.
\end{tcolorbox}

\begin{definition}[totally differentiable, gradient]
    $f:\R^d\to\R$が$x_0\in\R^d$において\textbf{全微分可能}とは,
    \[\exists_{\nabla f(x_0)\in\R^d}\;\lim_{0\ne h\to 0}\frac{f(x_0+h)-f(x_0)-h\cdot\nabla f(x_0)}{\norm{h}}=0\]
    が成り立つことをいう.$\nabla f(x_0)$は\textbf{勾配}という.
\end{definition}

\begin{theorem}[Rademacher differentiation theorem]
    Lipschitz連続関数$f:\R^d\to\R$は殆ど至る所全微分可能である.
\end{theorem}


\section{次元論}

\begin{tcolorbox}[colframe=ForestGreen, colback=ForestGreen!10!white,breakable,colbacktitle=ForestGreen!40!white,coltitle=black,fonttitle=\bfseries\sffamily,
title=]
    現代では殆ど廃れた問題意識であるが,コンパクト集合を考えるにあたって,開被覆がどれだけ自然な発想であるかがわかる.
\end{tcolorbox}

\begin{history}
    Cantorの1870年以降の論文で,初めて$\R^n$の任意の集合の位相が考えられた.
    そこで,内点,集積点,境界点,開・閉集合などのEuclid空間的な概念はすべてCantorが与えた.
    Cantor (1878)は線分から正方形への全単射を構成し,Peano (1890)はそれが連続に取れることがわかった.
    そこで,位相不変量としての次元の概念の精緻化が要請された.
    次の定義のような次元の概念が位相不変量であることを示すためにBrouwerが用いた手法は,代数的位相幾何学において主流となった.

    一方でFr\`{e}chetは,関数空間においても類似の位相的問題を考える仮定で,距離空間の概念を定義した(1905).
    そしてその様子から,「近傍」なる概念の持つべき性質を4公理にまとめあげて「位相空間」(今日のHausdorff空間に当たる)と呼んだのはHausdorff (1914)である.
    こうして,位相幾何学の議論の対象は真に広がった.

    位相空間の公理は,Kuratowski (1933)が「閉包の公理」を提案し,Bourbakiが開集合族,またはフィルターによるものがある.
    
\end{history}

\begin{definition}[Poincar\`{e} 1912, Brouwer 1913, Lebesgue 1921]
    コンパクト部分集合$K\subset\R^n$の\textbf{次元}とは,
    次の条件を満たす自然数$m$のうち最小のものをいう:
    \begin{quote}
        任意の$\ep>0$に対して,$K$の各点が$m+1$回以上は覆われないような,直径が$\ep$を超えない開集合の族が取れる.
    \end{quote}
\end{definition}

\section{整列集合論}

\begin{tcolorbox}[colframe=ForestGreen, colback=ForestGreen!10!white,breakable,colbacktitle=ForestGreen!40!white,coltitle=black,fonttitle=\bfseries\sffamily,
title=]
    任意の集合にどこまで性質の良い順序が定められるか,の問題を考える.
    「帰納法」を用いるためには,全ての集合に整列順序を入れる必要があった(Cantor).
    全順序は可能である.さらに条件を強めて,整列順序を考える.
    Zermelo (1904)が選択公理と同値であることを示し,Zorn (1935)がさらに使いやすい同値命題を付け加えた.なお,実際はKuratowski (1922)で出現する.
\end{tcolorbox}

\begin{definition}[well-ordered set]
    順序集合$(X,\le)$が\textbf{整列集合}であるとは,任意の空でない部分集合が最小元を持つことをいう.
    \begin{enumerate}
        \item 任意の二元集合を取れば,整列集合は明らかに全順序であることが判る.
    \end{enumerate}
\end{definition}

\begin{theorem}
    次の3条件は同値.
    \begin{enumerate}
        \item 任意の非空集合の族$(A_\lambda)_{\lambda\in\Lambda}$について,$\prod_{\lambda\in\Lambda}A_\lambda\ne\emptyset$.
        \item 任意の集合$A$に,順序関係$\le$が存在して,$(A,\le)$は整列集合である.
        \item 帰納的な順序集合には極大元が存在する.
    \end{enumerate}
\end{theorem}
\begin{remarks}
    いずれも,可算集合については明らかであるが,非可算無限集合について,それぞれの事実を直接示す方法は知られていない.
    どの条件も,「帰納法の適用限界」というような認識論的限界を提起していて,ネットも同じ構造に注目する.
\end{remarks}

\begin{theorem}[超限帰納法]
    $W$を整列集合,$a_0$をその最小元とする.命題$P:W\to2$は
    \begin{enumerate}
        \item $P(a_0)=1$.
        \item $\forall_{a_0\ne a\in W}\;\forall_{x\in W}\;x<a\Rightarrow [P(x)=1\Rightarrow P(a)=1]$.
    \end{enumerate}
    を満たすならば,$P=1$.
\end{theorem}

\begin{example}
    $L:=\R\cup\{\infty\}$n次のように$<\subset L\times L$を定めると,$(L,<)$は整列集合である:
    \begin{enumerate}
        \item $K:=\R\setminus\N$には,任意に整列順序$(K,<)$を入れる.
        \item $(\N,<)$には,通常の順序を入れると,これは整列順序.
        \item $\forall_{n\in\N}\;\forall_{a\in K}\;n<a$とする.
        \item $\forall_{\al\in\R}\;\al<\infty$とする.
    \end{enumerate}
    このとき,$L_1:=\Brace{\al\in L\mid\Brace{\beta<\al}\text{は非可算}}\ne\emptyset$より,最小元$\om_1\in L_1$が取れる.これについて,
    \[W:=\Brace{\al\in L\mid\al<\om_1}\]
    とすると,これは非可算な整列集合になる.
\end{example}

\begin{proposition}
    $(W,<)$は次を満たす:
    \begin{enumerate}
        \item $\forall_{\al\in W}\;\Brace{\beta<\al}$は可算.
        \item 任意の$W$の列は上に有界である.
        \item $\Brace{\beta\in W\mid\al<\beta}$も空でない.最小元を$\al+1$で表す.
    \end{enumerate}
\end{proposition}

\begin{definition}\mbox{}
    \begin{enumerate}
        \item 順序同型$f:X\to Y$とは,$x_1\le x_2\Leftrightarrow f(x_1)\le f(x_2)$を満たす全単射をいう.
        \item 整列集合の順序同型類を\textbf{順序数}という.
    \end{enumerate}
\end{definition}

\begin{proposition}
    順序数,基数のクラスは整列的(well-ordered)である.
\end{proposition}

\section{濃度論}

\begin{tcolorbox}[colframe=ForestGreen, colback=ForestGreen!10!white,breakable,colbacktitle=ForestGreen!40!white,coltitle=black,fonttitle=\bfseries\sffamily,
title=]
    集合には必ず整列順序が入り,順序数を持つことを見た.
    これをせずとも,集合に同型類が得られ,これを基数という.
\end{tcolorbox}

\begin{proposition}[Bernstein, 1897]
    $f:X\to Y,g:Y\to X$を単射とする.このとき,全単射$h:X\to Y$が存在する.
\end{proposition}

\chapter{位相空間とその射}

\begin{quotation}
    また位相空間は普遍構成ができる.
    が,連続全単射が位相同型になるにはコンパクトハウスドルフ性,指数対象が存在するためには局所コンパクト性,さらに$\Hom(X\times Y,Z)\simeq\Hom(X,Z^Y)$が成り立つには局所コンパクトハウスドルフ性が必要にいなる.

    「位」「相」の名前は,Alexandorff, P. (1928). \textit{Gestalt und Lage}(形態と位置) からである.
    phaseの方が位相らしい.
\end{quotation}

\section{開集合系による定義}

\subsection{公理と特徴付け}

\begin{tcolorbox}[colframe=ForestGreen, colback=ForestGreen!10!white,breakable,colbacktitle=ForestGreen!40!white,coltitle=black,fonttitle=\bfseries\sffamily,
title=]
    開集合系による定義では,開集合の全体$\Op(X)$が持つ$P(X)$の部分代数としての性質に注目する.
    特に束の言葉を利用して,枠と呼ばれる.
\end{tcolorbox}

\begin{definition}[frame, filter]
    $\O\subset P(X)$が
    \begin{enumerate}
        \item \textbf{枠}であるとは,次の3条件を満たすことをいう:
        \begin{enumerate}[(a)]
            \item $\bigcup$について閉じている.
            \item $\cap$について閉じている.
            \item 2つが両立する:任意の族$\{Y_i\}_{i\in I}\subset\O$について,$X\cap\paren{\bigcup_{i\in I}Y_i}=\bigcup_{i\in I}X\cap Y_i$.
        \end{enumerate}
        \item \textbf{フィルター}であるとは,次の3条件を満たすことをいう:
        \begin{enumerate}[(a)]
            \item 非空:$X\in\O$.
            \item 上方閉:$\forall_{A\in\O}\;\forall_{B\in P(X)}\;A\le B\Rightarrow B\in\O$.
            \item 下方有向:$\forall_{A,B\in\O}\;\exists_{C\in\O}\;C\le A\land C\le B$.
        \end{enumerate}
    \end{enumerate}
\end{definition}

\begin{definition}[topology, topological space]
    集合$X$と次の条件を満たす$\Op(X)\subset P(X)$との組$(X,\Op(X))$を\textbf{位相空間}という:
    \begin{enumerate}
        \item $(U_i)_{i\in I}$が$\O$の元の族ならば,$\cup_{i\in I}U_i\in\O$.
        \item $(U_i)_{i\in I}$が$\O$の元の\textbf{有限}族ならば,$\cap_{i\in I}U_i\in\O$.
    \end{enumerate}
    $\O_X:=\Op(X),\O(x):=\Brace{U\in\O_X\mid x\in U}$と表す.
    (2)の有限条件を落としても成り立つとき,$\O$を\textbf{Alexandroff位相}という.
\end{definition}

\begin{example}[近傍フィルター]
    $\Op(X)$や,一点$x\in X$の近傍系または開近傍系$\O(x)$や,$X$のコンパクト集合の全体はフィルターをなす.
\end{example}

\begin{proposition}[開集合の公理の特徴付け]\label{prop-characterization-of-open-sets}
    $X$を集合とし,$\O\subset P(X)$を集合とする.
    \begin{enumerate}
        \item 次の条件は同値である.\begin{description}
            \item[(1)] $(U_i)_{i\in I}$が$\O$の元の族ならば,$\cup_{i\in I}U_i\in\O$.
            \item[(1')] $X$の部分集合$U$であって,次を満たすものは全て$\O$の元である:\[\forall x\in U,\; \exists V\in\O,\;x\in V\subset U.\]
        \end{description}
        \item 次の条件は同値である.\begin{description}
            \item[(2)] $(U_i)_{i\in I}$が$\O$の元の\textbf{有限}族ならば,$\cap_{i\in I}U_i\in\O$.
            \item[(2')] $X\in\O$である.かつ,$U,V\in\O\Rightarrow U\cap V\in\O$である.
        \end{description}
    \end{enumerate}
\end{proposition}
\begin{proof}\mbox{}
    \begin{description}
        \item[(1)$\Rightarrow$(1')] (1')の論理式は,\textbf{$U$内部の開集合全体の集合$\U_U:=\{V\in\O\mid V\subset U\}$の和が$U$自身になる$\cup_{V\in\U_U}V=U$という条件}を表しており,(1')全体ではこれを満たすならば$U\in\O$という主張と同値である.従って,$\U_U$は$\O$の開集合の族だから,(1)$\Rightarrow$(1').
        \item[(1')$\Rightarrow$(1)] $\O$の族$(U_i)_{i\in I}$を取る.$\cup_{i\in I}U_i=:U$と置くと,任意の元$x\in U$に対して$i\in I$が存在して$x\in U_i\subset U$が成り立つから,$U\in\O$.
        \item[(2)$\Rightarrow$(2')] $I=\emptyset$とすれば,$(U_i)_{i\in I}=X\in\O$.$I=2$とすれば,$U,V\in\O\Rightarrow U\cap V\in\O$.
        \item[(2')$\Rightarrow$(2)] $I$の濃度についての数学的帰納法より.
    \end{description}
\end{proof}

\subsection{射が引き起こす位相}

\begin{tcolorbox}[colframe=ForestGreen, colback=ForestGreen!10!white,breakable,colbacktitle=ForestGreen!40!white,coltitle=black,fonttitle=\bfseries\sffamily,
title=]
    $f^*$は枠の射であるから,これによって位相を引き戻せる.
    一方で,終位相は$f_*(\O_X)$のような代数的な表現を必ずしも持たない.
    $f$が全単射であることが十分条件で,$f$が全射であることが必要条件である.
\end{tcolorbox}

\begin{proposition}[$f^*$は枠の射である]\label{prop-pullback-and-image-topology}
    $f:X\to Y$を写像とする.
    \begin{enumerate}
        \item $\O$が$Y$の位相ならば,引き戻し$f^*\O:=\{f^{-1}(U)\mid U\in\O\}=f^*(\O)$は$X$の位相になる.
        \item $\O$が$X$の位相ならば,像位相$\O':=\{V\in P(Y)\mid  f^{-1}(V)\in\O\}=(f^*)^{-1}(\O)$は$Y$の位相である.
    \end{enumerate}
\end{proposition}
\begin{proof}
    関手$f^*$が任意の$\cap,\cup$演算を保つこと
    (命題\ref{prop-functoriality-of-image-and-inverse-image-mappings})により成り立つ性質である.
    \begin{enumerate}
        \item 「像空間」$f^*\O\subset P(X)$の任意の合併は,$\O\subset P(Y)$の族$(U_i)_{i\in I}\;(U_i\subset Y)$(これは再び$\O$の元)を用いて$\cup_{i\in I}f^{-1}(U_i)$と表せるから,\[\cup_{i\in I}f^{-1}(U_i)=f^{-1}(\cup_{i\in I}U_i)\in f^*\O.\]同様に,$|J|<\infty$として,\[\cap_{i\in J}f^{-1}(U_i)=f^{-1}(\cap_{i\in J}U_i)\in f^*\O.\]
        \item 「逆像空間」$\O'=f^{*-1}(\O)\subset P(Y)$の任意の合併$\cup_{i\in I}V_i$は,$V_i\in\O\subset P(X)$であって$\cup_{i\in I}f^{-1}(V_i)\in \O$従って$f^{-1}(\cup_{i\in I}V_i)\in\O$が成り立つから,$\cup_{i\in I}V_i\in\O'=f^{*-1}(\O)$が従う.有限共通部分についても同様だから,$\O'=f^{*-1}(\O)$は任意の合併と任意の有限共通部分について閉じている.
    \end{enumerate}
\end{proof}

\begin{definition}[initial topology, relative topology / subspace topology, final topology]
    $f:X\to Y$を写像とする.
    \begin{enumerate}
        \item $X$の位相$f^*\O_Y$を\textbf{$\O_Y$の$f$による引き戻し}または\textbf{始位相}という.
        \item 部分空間$A\subset X$上の,$X$の位相$\O_X$の包含写像$i:A\to X$による引き戻し$i^*\O_X$のことを,$A$の\textbf{相対位相}という.この時,$\O_A=i^*\O_X=\{U\cap A\mid U\in\O_X\}$となる.
        \item $Y$の位相$f^{*-1}(\O_X):=\Brace{V\in P(Y)\mid f^{-1}(V)\in\O_X}$を$f$による\textbf{像位相}または\textbf{終位相}という.
    \end{enumerate}
\end{definition}

\begin{lemma}[押し出しが像位相になるとき]
    $f:X\to Y$を写像とする.
    \begin{enumerate}
        \item $f$が全単射でない場合は,$f_*(\O_X)\subsetneq (f^*)^{-1}(\O_X)$となることも,$(f^*)^{-1}(\O_X)\supsetneq f_*(\O_X)$となることもある.
        \item 一般には$f^*\circ f_*=\id_{P(X)},f_*\circ f^*=\id_{P(Y)}$は成り立たない.
        \item $f$が全単射の時,$f^*$も全単射で,$(f^*)^{-1}=(f^{-1})^*=f_*$である.
    \end{enumerate}
\end{lemma}
\begin{proof}\mbox{}
    \begin{enumerate}
        \item $f_*(\O_X)\subsetneq (f^*)^{-1}(\O_X)$は$f$が全射でないときに起こる.$a\in Y\setminus f(X)$を取る.$U\in f_*(\O_X)$とすると,$U\cup\{a\}$は$(f^*)^{-1}(\O_X)$の元であるが,像にはなり得ないので$f_*(\O_X)$の元ではない.
        これを単純化すると次の例のようになる.$1:\S\to 2$を定値写像とする.$f_*(\Op(\S))=\Op(\S)$となるが,$(f^*)^{-1}(\S)=P(2)$となるので後者は離散位相を定める.

        一方,$(f^*)^{-1}(\O_X)\supsetneq f_*(\O_X)$は$f$が単射でないがために起こる.
        恒等写像$\id_2$と$2\mapsto 1$の和写像$3\to 2$は全射だが単射ではない.$3$の位相$\Op(3)=\{0,3,\{1\}\}$を考えると,$f_*(\Op(3))=\Op(\S)$となるが,$(f^*)^{-1}(\Op(3))=\{0,2\}$となり,後者は密着位相を定める.
        \item 1の例を引き合いに出せば,そこ$\O_X\in P(X)$において恒等性は崩れる.
        \item 命題\ref{prop-dual-of-mono},\ref{prop-dual-of-epi}より.
    \end{enumerate}
\end{proof}
\begin{remarks}
    命題\ref{prop-functoriality-of-image-and-inverse-image-mappings}を見るとおり,一般の$f$について$A\subset f^{-1}(f(A)),f(f^{-1}(B))\subset B$が成り立つ.
    前者は非単射性から生じ,後者は非全射性から生じる.反例構成ではこれしか使っていない.
\end{remarks}

\begin{proposition}
    写像$f:X\to Y$と$X$の位相$\O$に対して,$f_*\O:=\{f(U)\in P(Y)\mid U\in\O\}$と定める.
    \begin{enumerate}
        \item $f$が全射でなければ,$f_*\O$は$Y$の位相になり得ない.
        \item $f:X\to Y$が全射であっても,$f_*\O$は$Y$の位相であるとは限らない.
    \end{enumerate}
\end{proposition}
\begin{proof}\mbox{}
    \begin{enumerate}
        \item $f$が全射でない限り$Y\in f_*\O$を満たさないので.
        \item 集合演算の$f(\cap_{i\in I}U_i)\subset \cap_{i\in I}f(U_i)$(命題\ref{prop-functoriality-of-image-and-inverse-image-mappings})をhackする.$f:4\to 3$を,$\Op(4)=\{0,4,\{0,1\},\{2,3\}\}$とし,$f(0)=0,f(1)=f(2)=1,f(3)=2$とすると,$f_*(\Op(4))=\{0,3,\{0,1\},\{1,2\}\}$より,これは位相にはならない.
        この原理は,$f(\{0,1\}\cap\{2,3\})=f(\emptyset)=\emptyset\ne\{1\}=\{0,1\}\cap\{1,2\}=f(\{0,1\})\cap f(\{2,3\})$ということである.
    \end{enumerate}
\end{proof}

\subsection{相対位相の性質}

\begin{tcolorbox}[colframe=ForestGreen, colback=ForestGreen!10!white,breakable,colbacktitle=ForestGreen!40!white,coltitle=black,fonttitle=\bfseries\sffamily,
title=]
    閉包演算子が良い振る舞いを続けるが,開集合では対称性が崩れる.
\end{tcolorbox}

\begin{proposition}[閉集合の特徴付け]
    $Y\subset X$を位相空間とする.任意の$A\subset Y$について$A$内での閉包は$Y\cap\o{A}$に等しく,さらに,次の2条件は同値:
    \begin{enumerate}
        \item $A$は$Y$-閉である.
        \item ある$X$-閉集合$F$が存在して,$A=F\cap Y$.
    \end{enumerate}
\end{proposition}

\begin{example}[開集合についてはどちらの方向も成り立たない]
    一方で,$A$の$Y$内での開核は,$Y\cap A^\circ$とは限らず,境界演算子もこのようになる.
    $X=\R^2,Y=\R\times\{0\}$とし,$A:=(a,b)\times\{0\}$を考えると,
    $A,Y$はいずれも開集合でないが,$A$は$Y$-開である.
\end{example}

\begin{proposition}[開埋め込み・閉埋め込みの重要性]\mbox{}
    \begin{enumerate}
        \item $Y$が開集合ならば,$A\osub Y\Leftrightarrow A\osub X$.
        \item $Y$が閉集合ならば,$A\csub Y\Leftrightarrow A\csub X$.
    \end{enumerate}
\end{proposition}

\begin{proposition}
    位相空間の部分集合$A\subset X$について,次の2条件は同値:
    \begin{enumerate}
        \item $\o{A}\setminus A$は閉集合.
        \item ある開集合$G$と閉集合$F$とが存在して,$A=G\cap F$と表せる.
    \end{enumerate}
\end{proposition}

\subsection{位相の生成}

\begin{tcolorbox}[colframe=ForestGreen, colback=ForestGreen!10!white,breakable,colbacktitle=ForestGreen!40!white,coltitle=black,fonttitle=\bfseries\sffamily,
    title=]
    位相の全体も有界束をなし,最大元が余離散位相$P(X)$,最小元が離散位相$\{\emptyset,X\}$となる.
    このうち,両端に位置する病的な位相を脱落させていくのが分離公理である.
\end{tcolorbox}

\begin{example}[trait / Sierpi\'{n}ski space, cofinite topology, Zariski topology]\mbox{}
    \begin{enumerate}
        \item $\bS:=(2,\{\emptyset,\{1\},2\})$を\textbf{線}または\textbf{Sierpi\'{n}ski空間}という.
        \item $\O=\Brace{X\setminus S\in P(X)\mid \abs{S}<\infty}\cup\{\emptyset\}$を\textbf{余有限位相}という.$T_1$位相のうち最も粗いものとして特徴付けられる.
        \item $K$を可換体,$R:=K[X_1,\cdots,X_n]$とし,$I(R)$を$R$のイデアル全体からなる集合とする.
        \begin{align*}
            \cR(A)&:=\Brace{x\in K^n\mid\forall_{f\in A}\;f(x)=0}\;(A\in I(R)),&\cC&:=\Brace{S\in P(K^n)\mid\exists_{A\in I(R)}\;S=\cR(A)}
        \end{align*}
        とすると,$\cC$は閉集合系としての公理を満たす.これを$K^n$の\textbf{Zariski位相}といい,$n$次元Affine $K$空間という.
    \end{enumerate}
\end{example}

\begin{proposition}[位相の完備束]
    $(\tau_j)_{j\in J}$を$X$上の位相の族とする.
    包含関係$\subset$に関する上限$\lor\tau_j$と下限$\land\tau_j$が存在する.
\end{proposition}
\begin{proof}\mbox{}
    \begin{enumerate}
        \item $\land\tau_j:=\Brace{A\in P(X)\mid \forall_{j\in J}\;A\in\tau_j}$と定めると,これは位相である.
        \item $T$を$X$上の位相であって任意の$\tau_j$より強いものとすると,$\CoDisc(X)\in T$より,$T\ne\emptyset$.これについて,$\lor\tau_j:=\land_{\tau\in T}\tau$と定めれば良い.
    \end{enumerate}
\end{proof}

\begin{proposition}[生成される位相の具体的構成]
    $\U\subset P(X)$とする.
    \[\O_\U:=\left\{U\in P(X)\;\middle|\;\begin{array}{l}x\in Uならば,\U の元の有限族(U_i)_{i\in n}で\\x\in\cap_{i\in n}U_i\subset Uを満たすものが存在する\end{array}\right\}.\]
    とすると,これは$\U\subset\O_\U$を満たす位相となり,$\U$を含む位相のうち最も小さい
\end{proposition}

\subsection{準基の性質}

\begin{proposition}
    $\U\subset\O$を準基とする:$\O_\U=\O$.
    このとき,任意の位相空間$(T,\O_T)$からの任意の写像$f:T\to X$について,次の2条件は同値.
    \begin{enumerate}
        \item $f$は連続.
        \item $f^*(\U)\subset\O_T$.
    \end{enumerate}
\end{proposition}

\begin{example}[実数の位相]\label{exp-subbasis-of-R}
    無限半開区間
    $\Brace{(-\infty,t)\mid t\in\R}\cup\Brace{(t,\infty)\mid t\in\R}$は$\R$の準基である.
\end{example}

\section{位相空間の射}

\subsection{連続写像の特徴付け:圏論と閉包作用素によるもの}

\begin{proposition}
    $f:X\to Y$を写像とする.
    \begin{enumerate}
        \item 次の同値な2条件を満たすとき,$f$は連続であるという:
        \begin{enumerate}[(a)]
            \item $f^*(\O_Y)\subset\O_X$.
            \item $\O_Y\subset f_*(\O_X)$.
        \end{enumerate}
        \item 次の2条件は同値.
        \begin{enumerate}[(a)]
            \item $f$は連続.
            \item 任意の点$x\in X$に対して,$\forall_{V\in\O(f(x))}\;\exists_{U\in\O(x)}\;U\subset f^{-1}(V)$.
        \end{enumerate}
    \end{enumerate}
\end{proposition}

\begin{theorem}[圏論的特徴付け]
    写像$f:X\to Y$について,次の3条件は同値.
    \begin{enumerate}
        \item $f$は連続である.
        \item $\forall_{g\in C(Y;\bS)}\;g\circ f\in C(X;\bS)$.
        \item $\forall_{Z\in\Top}\;\forall_{g\in C(Y;Z)}\;g\circ f\in C(X;Z)$.
    \end{enumerate}
\end{theorem}

\begin{proposition}[閉集合と閉包による特徴付け]
    写像$f:X\to Y$について,次の3条件は同値.
    \begin{enumerate}
        \item $f$は連続である.
        \item 任意の閉集合$B\subset Y$に対して,$f^{-1}(B)$は閉.
        \item $\forall_{A\in P(X)}\;f(\o{A})\subset\o{f(A)}$.
    \end{enumerate}
    また,次も同値になる.
    \begin{enumerate}\setcounter{enumi}{3}
        \item $\forall_{B\in P(Y)}\;f^{-1}(\Int B)\subset\Int f^{-1}(B)$.
        \item $\forall_{B\in P(Y)}\;f^{-1}(\o{B})\supset\o{f^{-1}(B)}$.
        \item $\forall_{B\in P(Y)}\;\partial f^{-1}(B)\subset f^{-1}(\partial B)$.
    \end{enumerate}
\end{proposition}

\subsection{開・閉写像の特徴付け}

\begin{tcolorbox}[colframe=ForestGreen, colback=ForestGreen!10!white,breakable,colbacktitle=ForestGreen!40!white,coltitle=black,fonttitle=\bfseries\sffamily,
title=]
    写像の連続性の閉包作用素による特徴付け$\forall_{A\in P(X)}\;f(\o{A})\subset\o{f(A)}$には,逆向きも考え得る.
    このように,
    議論の非対称性を補完する概念が開写像と閉写像である.
    また,$\o{f(A)}\subset f(\o{A})$はFubiniの定理と同じ向きとも思える.
\end{tcolorbox}

\begin{lemma}\mbox{}
    \begin{enumerate}
        \item 開写像・閉写像は合成しても保たれる.
        \item 2つの開写像の積写像は開だが,閉写像では成り立たない.
    \end{enumerate}
\end{lemma}

\begin{proposition}[演算子による開写像の特徴付け]
    $f:X\to Y$について,次の3条件は同値:
    \begin{enumerate}
        \item $f$が開写像である.
        \item $\forall_{A\in P(X)}\;f(A^\circ)\subset f(A)^\circ$.
        \item $\forall_{B\in P(Y)}\;f^{-1}(\o{B})\subset\o{f^{-1}(B)}$.
        \item $\forall_{F\csub X}\;\Brace{y\in Y\mid f^{-1}(y)\subset F}\csub Y$.
        \item $\forall_{x\in X}\;\forall_{N\in\O(x)}\;\exists_{V\in\O(f(x))}\;V\subset f(N)$.
    \end{enumerate}
\end{proposition}

\begin{proposition}[演算子による閉写像の特徴付け]
    $f:X\to Y$について,次の3条件は同値:
    \begin{enumerate}
        \item $f$は閉写像である.
        \item $\forall_{A\in P(X)}\;\o{f(A)}\subset f(\o{A})$
        \item $\forall_{F\csub X}\;\o{f(F)}\subset f(F)$.なお,$=$でも良い.
        \item $\forall_{U\in\O_X}\;\Brace{y\in Y\mid f^{-1}(y)\subset U}\in\O_Y$.
    \end{enumerate}
\end{proposition}

\begin{proposition}[全単射な開写像]
    $f:X\to Y$を全単射とする.
    \begin{enumerate}
        \item 開であることと閉であることとは同値.このとき,逆写像は連続である.
        \item $f$は連続でもあるとする.逆写像$f^{-1}$は開かつ閉であるが,連続とは限らない.
    \end{enumerate}
\end{proposition}

\subsection{連続な開写像}

\begin{tcolorbox}[colframe=ForestGreen, colback=ForestGreen!10!white,breakable,colbacktitle=ForestGreen!40!white,coltitle=black,fonttitle=\bfseries\sffamily,
title=]
    このクラスの連続写像の逆像関手は,開核作用素・閉包作用素と可換になる.
    例えば連続な閉写像は,$\forall_{A\in P(X)}\;f(\o{A})=\o{f(A)}$を満たす.
\end{tcolorbox}

\begin{proposition}[連続開写像の性質]
    $f:X\to Y$を連続な開写像とする.
    \begin{enumerate}
        \item $f$が全射ならば,$\forall_{B\in P(Y)}\;(f^{-1}(B))^\circ=f^{-1}(U^\circ)$.
        \item $\forall_{B\in P(Y)}\;\o{f^{-1}(B)}=f^{-1}(\o{B})$.
        \item $\forall_{B\in P(Y)}\;f^{-1}(\partial B)=\partial f^{-1}(B)$.
        \item $X$が第1可算・第2可算ならば,$f(X)$も第1可算・第2可算である.
    \end{enumerate}
\end{proposition}

\begin{theorem}
    $f:X\to Y$を開/閉な連続写像とする.
    \begin{enumerate}
        \item $f$が全射ならば,商写像である\ref{cor-open-continuous-surjection-is-quotient-map}.
        \item $f$が単射ならば,埋め込み$\quad\Leftrightarrow\quad f$である.
        \item $f$が全単射であることと,同相写像であることとは同値である.また,$f$が全単射の時,$f$が開写像であることと閉写像であることが同値で,$f$が全単射でない場合はそのどちらも一般的には成り立たない.
    \end{enumerate}
\end{theorem}
\begin{proof}\mbox{}
    \begin{enumerate}
        \item $f$が全射ならば,section $g:Y\to X$が存在して,$f\circ g=\id_Y$を満たす.
        即ち,
        ちなみに$\forall U\in P(X),\;g(U)=f^{-1}(U)$より,$f$の開閉に拘らず$g$は開写像になる.
        \item 任意の部分集合$U\subset X$について\[f(X\setminus U)\overset{単射の時}{=}f(X)\setminus f(U)\overset{全射の時}{=}Y\setminus f(U)\]
        が成り立つため,$f$が単射の時は,部分空間への同相写像となっている.
        \item $f^*\O_Y=\O_X$が成り立つため.上の関係$f(X\setminus U)=Y\setminus f(U)$は全射性と単射性のどちらか1つでも欠けると通れない.
    \end{enumerate}
\end{proof}
\begin{remarks}
    いずれの場合も,起こっている現象は$f^*\O_Y=\O_X$である.それぞれに対する表現が違うだけである.足元に誘発される場合はfinal topology,頭上に誘発される場合はinitial topology,全単射である場合は同相写像.
    この聖域は,コンパクト集合からHausdorff空間への射で起こる.
\end{remarks}

\subsection{完全写像}

\begin{definition}[perfect map]
    連続写像$p:X\to Y$が全射かつ閉で,さらに任意のファイバー$p^{-1}(y)$がコンパクトならば,\textbf{完全写像}であるという.
    完全写像は商写像である\ref{cor-open-continuous-surjection-is-quotient-map}ことに注意.
\end{definition}

\begin{proposition}
    $p:X\to Y$を連続かつ全射な閉写像とする.
    このとき,$X$が正規ならば,$Y$も正規である.
\end{proposition}

\begin{proposition}
    $p:X\to Y$が完全であるとする.
    \begin{enumerate}
        \item $Y$がコンパクトならば,$X$はコンパクトである.
        \item $Y$がパラコンパクトならば,$X$もパラコンパクトである.
        \item $X$がHausdorffならば,$Y$もHausdorffである.
        \item $X$がパラコンパクトハウスドルフならば,$Y$もパラコンパクトハウスドルフである.
        \item $X$が正則ならば,$Y$も正則である.
        \item $X$が局所コンパクトならば,$Y$も局所コンパクトである.
        \item $X$が第2可算ならば,$Y$も第2可算である.
    \end{enumerate}
\end{proposition}

\begin{example}
    $X$に作用するコンパクト位相群$G$について,商写像$X\epi X/G$は完全写像である.
\end{example}

\subsection{固有写像}

\begin{definition}[proper map]\mbox{}
    \begin{enumerate}
        \item 写像$f:X\to Y$が\textbf{固有写像}であるとは,任意のコンパクト集合の逆像がコンパクトになることをいう.
        \item 位相空間$X$の点列$(x_n)$が無限遠へ消失するとは,任意のコンパクト集合$C\subset X$について,$\abs{n\in\N\mid x_n\in S}<\infty$が成り立つことをいう.
    \end{enumerate}
\end{definition}

\begin{proposition}[終域が局所コンパクトハウスドルフであるとき,本質的に完全写像に同値]
    $Y$を局所コンパクトハウスドルフ空間とする.このとき,写像$f:X\to Y$について,次の2条件は同値.
    \begin{enumerate}
        \item $f$は固有写像である.
        \item $f$は連続な閉写像で,任意のファイバーはコンパクトになる.
    \end{enumerate}
\end{proposition}

\begin{proposition}
    $X,Y$を距離空間,$f:X\to Y$を連続写像とする.次の2条件は同値:
    \begin{enumerate}
        \item $f$は固有写像である.
        \item $X$上の無限遠へ消失する点列は,$Y$上の無限遠へ消失する点列へと写される.
    \end{enumerate}
\end{proposition}

\begin{proposition}
    位相空間$X$について,次の2条件は同値:
    \begin{enumerate}
        \item $X$はコンパクトである.
        \item 写像$X\to\{*\}$は固有である.
    \end{enumerate}
\end{proposition}

\subsection{連続写像の構成}

\begin{proposition}
    $f:X\to Y$を連続写像とし,$A,B\subset X$は$A\cup B=X$を満たす開集合,または共に閉集合とする.このとき,$f|_A,f|_B$がいずれも連続ならば,$X$上でも連続.
\end{proposition}

\begin{proposition}
    $f,g\in C(X;Y)$に対して,$f+g,f\cdot g,\abs{f},1/f,\sup(f,g)$は連続である.
\end{proposition}

\begin{proposition}
    $f,g\in C(X)$が稠密部分集合$D$上で一致するなら,$f=g$である.
    すなわち,制限$i^*:C(X)\to C(D)$は単射である.
\end{proposition}

\begin{corollary}
    $C(\R)$は連続体濃度である.
\end{corollary}
\begin{proof}
    命題より,単射$i^*:C(\R)\to C(D)$が取れる.
    一方で,$\ev:\R\to C(\R)$を値写像とすると,これも単射である.
\end{proof}

\subsection{位相同型の特徴付け}

\begin{tcolorbox}[colframe=ForestGreen, colback=ForestGreen!10!white,breakable,colbacktitle=ForestGreen!40!white,coltitle=black,fonttitle=\bfseries\sffamily,
title=]
    全単射な連続関数は,開・閉写像ならば位相同型である.
\end{tcolorbox}

\begin{proposition}
    $f:X\to Y$を連続写像とする.次の4条件は同値.
    \begin{enumerate}
        \item $f$は位相同型である.
        \item $f$が全単射であり,位相を引き戻す:$f^*(\O_Y)=\O_X$.
        \item $f$は全単射な開写像である.または,全単射な閉写像である.
        \item 任意の$T\in\Top$について,$f_*:C(T;X)\to C(T;Y)$は全単射を定める.
    \end{enumerate}
\end{proposition}
\begin{remarks}
    これより,位相空間の性質で,開集合のみを用いて表される性質は,全て位相不変量となる.
\end{remarks}

\subsection{局所同相写像}

\begin{tcolorbox}[colframe=ForestGreen, colback=ForestGreen!10!white,breakable,colbacktitle=ForestGreen!40!white,coltitle=black,fonttitle=\bfseries\sffamily,
title=]
    多様体の座標近傍や,被覆写像などは局所同相写像である.
\end{tcolorbox}

\begin{definition}
    連続写像$p:E\to B$が\textbf{局所同相写像}であるとは,
    $\forall_{e\in E}\;\exists_{U\in\O(e)}\;p_*(U)\in\O_B\land p|_U:U\simeq p_*(U)$を満たすことをいう.
\end{definition}

\begin{proposition}
    局所同相写像は開写像である.
\end{proposition}

\subsection{埋め込み}

\begin{tcolorbox}[colframe=ForestGreen, colback=ForestGreen!10!white,breakable,colbacktitle=ForestGreen!40!white,coltitle=black,fonttitle=\bfseries\sffamily,
title=]
    Topの切断を埋め込みという.
    単に連続な単射が,切断であるとは限らない.
\end{tcolorbox}

\begin{definition}[imbedding, retraction]
    $f:X\to Y$を連続写像とする.
    \begin{enumerate}
        \item $f$が単射かつ$\O_X=f^*(\O_Y)$を満たすとき,これを\textbf{埋め込み}という.このとき,$f:X\to f(X)$は位相同型である.
        \item $f$の左逆射$r\circ f=\id_X$を(連続な)\textbf{レトラクション}という.
    \end{enumerate}
\end{definition}

\begin{proposition}[埋め込みであることの十分条件]
    連続写像$f:X\to Y$について,
    \begin{enumerate}
        \item レトラクション$r:Y\to X$を持つならば,埋め込みである.
        \item 単射な開/閉写像ならば,開/閉な埋め込みである.
    \end{enumerate}
\end{proposition}

\subsection{像位相}

\begin{tcolorbox}[colframe=ForestGreen, colback=ForestGreen!10!white,breakable,colbacktitle=ForestGreen!40!white,coltitle=black,fonttitle=\bfseries\sffamily,
title=]
    ある程度の双対命題が成り立つ.
    そう,埋め込みの双対概念は商写像である.
\end{tcolorbox}

\begin{definition}
    $f:X\to Y$を連続写像とする.$f$の右逆射$f\circ s=\id_Y$を(連続な)\textbf{セクション}という.
\end{definition}

\begin{proposition}
    $f:X\to Y$を連続写像とする.$f$にセクションが存在すれば,$Y$の位相は$f$による像位相である.
\end{proposition}

\section{同値な公理}

\begin{tcolorbox}[colframe=ForestGreen, colback=ForestGreen!10!white,breakable,colbacktitle=ForestGreen!40!white,coltitle=black,fonttitle=\bfseries\sffamily,
    title=]
    いずれの命題も,開集合の公理と同値になる.
\end{tcolorbox}

\subsection{触点と閉包の定義}

\begin{tcolorbox}[colframe=ForestGreen, colback=ForestGreen!10!white,breakable,colbacktitle=ForestGreen!40!white,coltitle=black,fonttitle=\bfseries\sffamily,
title=]
    極限点(limit point)とは,集積点を指すのか触点を指すのか不明瞭である:$A$の触点は$A$のあるネットの極限であることと同値であり,$A$の集積点は$A$が連続延長し得る定義域でもある.
    集積点は,触点であって,さらに近傍によって$A$の他の点と分離不可能な点をいう.
\end{tcolorbox}

\begin{definition}[isolated point, adherent point, accumulated point, closure, dense]
    $X$を位相空間,$x\in X,A\subset X$とする.
    \begin{enumerate}
        \item $\{x\}\in\O(x)$のとき,\textbf{孤立点}という.これは全空間$X$の集積点でないことに同値.
        \item $\forall_{U\in\O(x)}\;U\cap A\ne\emptyset$のとき,$A$の\textbf{触点}という.
        \item $\forall_{U\in\O(x)}\;(U\cap A)\setminus\{x\}\ne\emptyset$のとき,$A$の\textbf{集積点}という.
        \item $A$の触点全体の集合を\textbf{閉包}$\o{A}$という.
        \item $\o{A}=X$のとき,\textbf{$X$上稠密}であるという.
    \end{enumerate}
\end{definition}

\begin{proposition}[開核,閉包,稠密性の特徴付け]
    $A\subset X$とする.
    \begin{enumerate}
        \item $A^\circ$は$A$に含まれる開集合のうち最大のものである.
        \item $\o{A}$は$A$を含む閉集合のうち最小のものである.
        \item $A$が$X$上稠密であることと,$\forall_{U\in\O_X}\;U\ne\emptyset\Rightarrow A\cap U\ne\emptyset$.
    \end{enumerate}
\end{proposition}

\subsection{閉包による閉集合の特徴付け}

\begin{tcolorbox}[colframe=ForestGreen, colback=ForestGreen!10!white,breakable,colbacktitle=ForestGreen!40!white,coltitle=black,fonttitle=\bfseries\sffamily,
title=]
    触点と集積点とは,孤立点の分しか相違がない.
    集積点は$T_1$空間でさらに直感的な特徴付け\ref{prop-property-of-T1-spaces}を持つ.
\end{tcolorbox}

\begin{proposition}[集積点の特徴付け]
    次の3条件は同値.
    \begin{enumerate}
        \item $x$は$A$の集積点である.
        \item $x\in X$が部分空間$A\cup\{x\}$の孤立点ではない.
        \item $x\in\overline{A\setminus\{x\}}$.
    \end{enumerate}
\end{proposition}

\begin{proposition}
    $A\subset X$について,次の3条件は同値.
    \begin{enumerate}
        \item $A$は閉集合.
        \item $\o{A}\subset A$.
        \item $A'\subset A$.
    \end{enumerate}
\end{proposition}

\subsection{集積点と導来集合}

\begin{tcolorbox}[colframe=ForestGreen, colback=ForestGreen!10!white,breakable,colbacktitle=ForestGreen!40!white,coltitle=black,fonttitle=\bfseries\sffamily,
title=]
    集積点なる概念は,孤立点と併せれば触点となる.
    そこで,導来演算子$'$は,閉包演算子より弱い概念になる.
    しかし導来演算子は原初Cantor (1872)が$\R$の研究に用い,集合論を展開した.
    Cantorの観点からは完全集合こそが振る舞いのよい集合であったが,さらにexoticな例が見つかっていくのであった.
\end{tcolorbox}

\begin{definition}[perfect set, derived set]
    部分集合$S\subset X$が
    \begin{enumerate}
        \item \textbf{完全}であるとは,閉集合であり,かつ孤立点を持たないことをいう.これは$S=S'$と同値.
        \item $S$の集積点全体の集合を\textbf{導来集合}といい,$S'$で表す.
    \end{enumerate}
\end{definition}

\begin{example}
    Cantor集合$2^\N$は完全であり,完全集合の中で唯一の全不連結でコンパクトで距離化可能な空間である.
\end{example}

\begin{proposition}[導来作用素の性質]\mbox{}
    \begin{enumerate}
        \item 閉包の特徴付け:$\o{A}=A\cup A'$.
        \item $\o{A}\setminus A'$は$A$の孤立点のみからなる.
    \end{enumerate}
\end{proposition}

\begin{proposition}
    $X$が距離空間ならば,任意の導来集合$A'$は閉集合である.
\end{proposition}

\subsection{基底の定義}

\begin{tcolorbox}[colframe=ForestGreen, colback=ForestGreen!10!white,breakable,colbacktitle=ForestGreen!40!white,coltitle=black,fonttitle=\bfseries\sffamily,
title=]
    任意の$\A\subset P(X)$について,有限共通部分についての閉包を取ると,これは基底になる.
\end{tcolorbox}

\begin{definition}
    $\U\subset\O_X$について,
    \begin{enumerate}
        \item $\U$が$\O$の\textbf{基底}であるとは,次を満たすことをいう:$\forall_{U\in\O_X}\;\forall_{x\in U}\;\exists_{V\in\U}\;x\in V\subset U$.
        \item $\U$が$\O(x)$の\textbf{基本系}であるとは,$\U\subset\O(x)$であって,次を満たすことをいう:$\forall_{U\in\O(x)}\;\exists_{V\subset\U}\;V\subset U$.
    \end{enumerate}
\end{definition}

\subsection{閉包の公理}

\begin{tcolorbox}[colframe=ForestGreen, colback=ForestGreen!10!white,breakable,colbacktitle=ForestGreen!40!white,coltitle=black,fonttitle=\bfseries\sffamily,
title=]
    Kuratowski, C.による.
    位相の概念は,閉包演算子だけで表すこともできることが分かる.
\end{tcolorbox}

\begin{proposition}
    $A,B\subset X$とする.次の4条件が成り立つ:
    \begin{enumerate}
        \item $A\subset\o{A}$.
        \item $\o{\emptyset}=\emptyset$.
        \item 冪等:$\oo{A}=\o{A}$.
        \item $\o{A\cup B}=\o{A}\cup\o{B}$.
    \end{enumerate}
\end{proposition}

\begin{theorem}\label{thm-Kuratowski}
    $u:P(X)\to P(X)$が上述の4条件を満たすとする.このとき,
    \[\Brace{X\setminus A\in P(X)\mid u(A)=A}\]
    は位相を定める.
\end{theorem}

\subsection{正則な集合}

\begin{definition}[regular open / closed set]\mbox{}
    \begin{enumerate}
        \item 集合$G\subset X$が\textbf{正則開集合}であるとは,$\Int(\Cl(G))=G$を満たすことをいう.
        \item 集合$F\subset X$が\textbf{正則閉集合}であるとは,$\Cl(\Int(F))=F$を満たすことをいう.
    \end{enumerate}
\end{definition}

\begin{proposition}\mbox{}
    \begin{enumerate}
        \item 正則開集合は開集合で,正則閉集合は閉集合である.
        \item $A$が閉集合ならば,$A^\circ$は正則開集合である.$B$が開集合ならば,$\o{B}$は正則閉集合である.
        \item $U,V$が正則開集合ならば,$U\cap V$もそうである.$F,G$が正則閉集合ならば,$F\cup G$もそうである.
    \end{enumerate}
\end{proposition}

\subsection{近傍系の公理}

\begin{tcolorbox}[colframe=ForestGreen, colback=ForestGreen!10!white,breakable,colbacktitle=ForestGreen!40!white,coltitle=black,fonttitle=\bfseries\sffamily,
title=]
    Hilbert, D. (1902)はHilbert幾何学に続いて,平面の位相的定義を与えた.
    この定義に現れた「近傍」の概念を発展させて,Hausdorff, F. (1914)は近傍の概念を確立し,現在ではHausdorff空間にあたる位相空間概念を公理化した.
\end{tcolorbox}

\begin{notation}
    ここでのみ,$\O(x)$を$x$の近傍のなすフィルターとする.開近傍に限らないことに注意.
\end{notation}

\begin{proposition}
    任意の$x\in X$について,
    次の5条件を満たす.
    \begin{enumerate}
        \item $X\in\O(x)$.
        \item $\forall_{V\in\O(x)}\;\forall_{W\in P(X)}\;V\subset W\Rightarrow W\in\O(x)$.
        \item $\forall_{V,W\in\O(x)}\;V\cap W\in\O(x)$.
        \item $\forall_{V\in\O(x)}\;x\in V$.
        \item $\forall_{V\in\O(x)}\;\exists_{W\in\O(x)}\;\forall_{y\in W}\;V\in\O(y)$.
    \end{enumerate}
    前の3条件は,$\O(x)$がフィルターであることを主張している.
\end{proposition}

\begin{proposition}[近傍系の定める位相]
    $A\in\Op(X):\Leftrightarrow\forall_{x\in A}\;\exists_{U\in\O(x)}U\subset A$とすれば良い.
\end{proposition}

\subsection{基底の公理}

\begin{proposition}
    $\U\subset\O_X$を基底とする.次の2条件が成り立つ.
    \begin{enumerate}
        \item $\forall_{U,V\in\U}\;\forall_{x\in U\cap V}\;\exists_{W\in\U}\;x\in W\subset U\cap V$.
        \item $\forall_{x\in X}\;\exists_{U\in\U}\;x\in U$.
    \end{enumerate}
\end{proposition}

\section{積空間と始位相}

\subsection{積位相の特徴付け:2つの場合}

\begin{definition}
    $X\times Y$の位相のうち,$\U:=\Brace{U\times V\in P(X\times Y)\mid U\in\O_X,V\in\O_Y}$が生成する位相$\O_X\times\O_Y$を\textbf{積位相}という.
    $\pr_1^*(\O_X)\cup\pr_2^*(\O_Y)$を準基として生成される位相と言っても良い.
\end{definition}

\begin{proposition}[積位相の特徴付け]
    $X,Y$を位相空間とする.
    \begin{enumerate}
        \item $\U$は$\O_X\times\O_Y$の基底である.実際,次の2条件が同値:
        \begin{enumerate}
            \item $W\in\O_X\times\O_Y$.
            \item $\forall_{(x,y)\in W}\;\exists_{U\in\O(x),V\in\O(y)}\;U\times V\subset W$.
        \end{enumerate}
        \item 射影$\pr_1:X\times Y\to X,\pr_2:X\times Y\to Y$はいずれも連続な開写像になり,$\O_X\times\O_Y$はこの性質を満たす最弱の位相である.
    \end{enumerate}
\end{proposition}
\begin{proof}\mbox{}
    \begin{enumerate}
        \item a
        \item 連続であり,$\O_X\times\O_Y$がこれについての始位相に一致することは明らか.像は合併を保つから,$\U$の元を開に移すことを確かめれば良い.これは定め方から明らか.
    \end{enumerate}
\end{proof}

\begin{proposition}[積写像の連続性の特徴付け]
    任意の$T\in\Top$と写像$f:T\to X,g:T\to Y$について,次の2条件は同値;
    \begin{enumerate}
        \item 積$(f,g):T\to X\times Y$が連続.
        \item $f,g$はそれぞれ連続.
    \end{enumerate}
\end{proposition}

\begin{proposition}[連続写像のグラフの埋め込み]
    $f:X\to Y$が連続ならば,積$g:=(\id_X,f):X\to X\times Y$は埋め込みで,$\Im g=\Gamma_f\subset X\times Y$.
\end{proposition}

\begin{example}[距離空間の積]
    2つの距離空間$(X,d_X),(Y,d_Y)$について,$L^1$-距離,$L^2$-距離,$L^\infty$-距離(正確にはManhattan距離,Euclid距離,一様距離)のいずれも同値で,$X\times Y$の積位相を定める.
    また,距離関数$d:X\times X\to d$は連続になる.
\end{example}

\begin{example}[二項演算]
    実数の加法$+:\R\times\R\to\R$と乗法$\times:\R\times\R\to\R$も連続である.
    行列積も連続である.
\end{example}

\subsection{積位相の特徴付け:一般の場合}

\begin{tcolorbox}[colframe=ForestGreen, colback=ForestGreen!10!white,breakable,colbacktitle=ForestGreen!40!white,coltitle=black,fonttitle=\bfseries\sffamily,
title=]
    一般の場合の積位相とは,写像の空間$\Map(I,X)$の各点収束位相である.
\end{tcolorbox}

\begin{definition}
    位相空間の族$(X_i,\O_i)_{i\in I}$の積$\Map(I,X)=X:=\prod_{i\in I}X_i$に定まる位相のうち,
    射影の族$(\pr_i)_{i\in I}$の定める始位相,すなわち,
    射影の引き戻し$\bigcup_{i\in I}\pr_i^*(\O_i)$によって生成される位相を,$X$の\textbf{積位相}または\textbf{Tychonoff位相}という.
\end{definition}

\begin{proposition}[積位相の特徴付け]
    \[ \U:=\left\{\prod_{i\in I}U_i\;\middle|\;\begin{array}{l}(U_i)_{i\in I}は開集合U_i\subset X_iの族であり,\\\{i\in I\mid U_i\subsetneq X_i\}は有限集合である\end{array}\right\} \]
    とすると,これは$\prod_{i\in I}\O_i$の基底である.すなわち,
    $U\subset X$について,次の2条件は同値.
    \begin{enumerate}
        \item $U\in\O_X$.
        \item $\forall_{x\in U}\;\exists_{\{U_i\}_{i\in I}\;(U_i\in\O_i)}\;U_i=X_i\;\fe\land\prod_{i\in I}U_i\subset U$.
    \end{enumerate}
\end{proposition}

\begin{proposition}[積写像の連続性の特徴付け]
    任意の$T\in\Top$と写像の族$(g_i:T\to X_i)$について,次の2条件は同値:
    \begin{enumerate}
        \item $g:=\prod_{i\in I}g_i:T\to X$は連続.
        \item $\forall_{i\in I}\;g_i\in C(T;X_i)$.
    \end{enumerate}
\end{proposition}

\begin{proposition}
    $X,Y,Z$を位相空間とする.(1)$\Rightarrow$(2)は成り立つが,逆は必ずしも成り立たない.
    \begin{enumerate}
        \item $f:X\times Y\to Z$は連続.
        \item それぞれの成分が連続.
    \end{enumerate}
\end{proposition}

\begin{example}[$m$-進小数展開]
    任意の自然数$m\ge 2$について,$m$-進小数展開$e_m:m^\N\to[0,1]$,すなわち
    \[e_m((x_n)):=\frac{1}{m}\sum^\infty_{n=0}\frac{x_n}{m^n}\]
    は連続.ただし,$2$は離散空間とした.Tychonoffの定理より$2^\N$はコンパクトだから,$e_m$は全射より$[0,1]$がコンパクトであること(Heine-Borel)もわかる.
\end{example}

\subsection{積空間の閉包作用素}

\begin{proposition}
    $X,Y$を位相空間とし,$A\subset X,B\subset Y$とする.
    \begin{enumerate}
        \item $\o{A\times B}=\o{A}\times\o{B}$.
        \item $(A\times B)^\circ=A^\circ\times B^\circ$.
    \end{enumerate}
\end{proposition}

\subsection{始位相の普遍性}

\begin{tcolorbox}[colframe=ForestGreen, colback=ForestGreen!10!white,breakable,colbacktitle=ForestGreen!40!white,coltitle=black,fonttitle=\bfseries\sffamily,
title=]
    積位相を調べるにあたって,始位相の基底の探し方と,積写像の連続性の特徴付けは,一般の始位相について使える手法である.
\end{tcolorbox}

\begin{proposition}[誘導位相の普遍性]\label{prop-universality-of-final-topology}
    $(X_i)_{i\in I}$を位相空間の族,$(f_i:X\to X_i)_{i\in I}$を写像の族とし,$X$を誘導位相により位相空間と考える.
    \begin{enumerate}
        \item $X$の部分集合$U$について,次の2条件は同値.
        \begin{enumerate}[(1)]
            \item $U$は$X$の開集合である.
            \item $x\in U$ならば,有限部分集合$J\subset I$と$f_i(x)\in X_i$の開近傍の有限族$(U_i)_{i\in J}$で,$\cap_{i\in J}f^{-1}_i(U_i)\subset U$を満たすものが存在する.
        \end{enumerate}
        \item 任意の位相空間$T$と写像$g:T\to X$に対し,次の2条件は同値.
        \begin{enumerate}[(1)]
            \item $g$は連続である.
            \item 任意の$i\in I$に対して,$f_i\circ g$が連続.
        \end{enumerate}
    \end{enumerate}
\end{proposition}

\subsection{始位相のネット}

\begin{tcolorbox}[colframe=ForestGreen, colback=ForestGreen!10!white,breakable,colbacktitle=ForestGreen!40!white,coltitle=black,fonttitle=\bfseries\sffamily,
title=]
    位相は収束するフィルターを指定することでも定まる.
    始位相は,収束ネットを$\F$によって指定しているとみなせる.
    積空間とは,各点収束の位相であることがよく分かる.
\end{tcolorbox}

\begin{proposition}
    $X$には$\F:=\{f:X\to Y_f\}_{f\in\F}$による始位相が入っているとする.このとき,ネット$(x_\lambda)$について,次の2条件は同値:
    \begin{enumerate}
        \item $(x_\lambda)\to x\in X$.
        \item $\forall_{f\in\F}\;(f(x_\lambda))\to f(x)$.
    \end{enumerate}
\end{proposition}

\begin{corollary}
    $g:Z\to X$を写像,$X$を$\F$が定める始位相によって位相空間とみると,次の2条件は同値:
    \begin{enumerate}
        \item $g$は連続である.
        \item $\forall_{f\in\F}\;f\circ g:Z\to Y_f$は連続である.
    \end{enumerate}
\end{corollary}

\section{商空間と像位相}

\subsection{商空間の普遍性}

\begin{proposition}[universality of quotient space]
    写像$f:X\to Y$が写像$g:X/R\to Y$を引き起こすならば,$R_f\subset R$である\ref{cor-universality-of-quotient-set}.
    このとき,$f$が連続なことと$g$が連続なことは同値.
\end{proposition}
\begin{proof}
    $f$が連続であることと,像位相$f_*\O_X$について$\O_Y\subset f_*\O_X$が成り立つこととは同値.
    $X/R$の位相は$p_*\O_X$であるから,$f=g\circ p$のとき,これは$\O_Y\subset g_*(p_*(\O_X))$に同値.
\end{proof}

\subsection{商写像}

\begin{tcolorbox}[colframe=ForestGreen, colback=ForestGreen!10!white,breakable,colbacktitle=ForestGreen!40!white,coltitle=black,fonttitle=\bfseries\sffamily,
title=]
    連続写像が終位相を定めているとき,これを商写像という.
\end{tcolorbox}

\begin{definition}[quotient map, saturated]
    $p:X\to Y$を全射とする.
    \begin{enumerate}
        \item 全射$p$が\textbf{商写像}であるとは,$U\in\O_Y\Leftrightarrow p^{-1}(U)\in\O_X$が成り立つことをいう.
        \item 集合$C\subset X$が全射$p$に関して\textbf{飽和集合}であるとは,$\forall_{y\in f(C)}\;f^{-1}(y)\subset C$が成り立つことをいう.
    \end{enumerate}
\end{definition}

\begin{proposition}[全射が商写像になるための条件]
    全射$p$について,次の2条件は同値.
    \begin{enumerate}
        \item $p$は商写像である.
        \item $p$は連続で,かつ飽和開集合を開集合に移す.
    \end{enumerate}
\end{proposition}

\begin{corollary}\mbox{}\label{cor-open-continuous-surjection-is-quotient-map}
    \begin{enumerate}
        \item 連続全射が開写像または閉写像ならば,商写像である.
        \item 開でも閉でもない商写像が存在する.
    \end{enumerate}
\end{corollary}

\subsection{商空間}

\begin{proposition}
    $Q:X\to\wt{X}$を商写像とする.次の2条件は同値.
    \begin{enumerate}
        \item $\wt{X}$は$T_1$である:任意の一点集合が閉.
        \item $X$の同値類のそれぞれが閉集合である.
    \end{enumerate}
\end{proposition}

\begin{proposition}
    $Q:X\to\wt{X}$を商写像とする.次の2条件は同値.
    \begin{enumerate}
        \item $\wt{X}$は$T_2$である.
        \item $\Delta\subset X\times X$が閉集合.
    \end{enumerate}
\end{proposition}


\subsection{直和位相の特徴付け}

\begin{tcolorbox}[colframe=ForestGreen, colback=ForestGreen!10!white,breakable,colbacktitle=ForestGreen!40!white,coltitle=black,fonttitle=\bfseries\sffamily,
title=]
    射影による始位相が直和位相で,包含による終位相が直和位相である.いずれも連続な開写像になる.包含は埋め込みでもある.
\end{tcolorbox}

\begin{definition}[direct sum topology]
    直和$\coprod_{i\in I}X_i$の,標準単射の族$j_i:X_i\to\coprod_{i\in I}X_i$による像位相を\textbf{直和位相}という.
\end{definition}

\begin{lemma}
    任意の$i\in I$に対して標準単射$j_i:X_i\to\coprod_{i\in I}X_i$は開埋め込みであり,$(j_i(X_i))_{i\in I}$は$\coprod_{i\in I}X_i$の開被覆である.
\end{lemma}

\subsection{開被覆の性質}

\begin{tcolorbox}[colframe=ForestGreen, colback=ForestGreen!10!white,breakable,colbacktitle=ForestGreen!40!white,coltitle=black,fonttitle=\bfseries\sffamily,
title=]
    非常に非自明な事実だが,開被覆がその空間の位相を殆ど規定してしまう.
\end{tcolorbox}

\begin{proposition}[開被覆の像位相としての見方]
    $X$を位相空間とし,$(U_i)_{i\in I}$を$X$の開被覆とする.
    \begin{enumerate}
        \item $X$の位相は,包含写像の族$(j_i:U_i\to X)_{i\in I}$による像位相である.
        \item $\U_i$を$U_i$の開集合系とすると,$X$の位相は$\U=\cup_{i\in I}\U_i$によって生成位される位相である.
    \end{enumerate}
\end{proposition}
\begin{proof}
    $\O$を$X$の位相とし,$\O_\U$を$\U$が生成する位相,$\O'$を像位相とする.
    $\O_\U\subset\O\subset\O'\subset\O_\U$を示す.
    最左辺は$\U\subset\O$による.中辺は$\O'$が包含写像を連続にする最も細かい位相であることによる.
    最後に最右辺は,任意の$U\in\O$に対して,$U=\cup_{i\in I}(U\cap U_i)$と$\O_\U$の元の合併として表せるから,$U\in\O_\U$.
\end{proof}


\begin{proposition}
    $f:X\to Y$を連続写像,$(V_i)_{i\in I}$を$Y$の開被覆とする.次の2条件は同値:
    \begin{enumerate}
        \item $Y$の位相は$f$の像位相である.
        \item 任意の$i\in I$に対して,$V_i$の位相は,制限$f_i:f^{-1}(V_i)\to V_i$による像位相である.
    \end{enumerate}
\end{proposition}

\subsection{貼り合わせ}

\begin{theorem}[位相の族の貼り合わせの条件]
    $X$を集合,$(X_i)$を$X$の被覆とする.$(X_i,\O_i)$は位相空間であるとする.
    次の2条件は同値.
    \begin{enumerate}
        \item 包含の族$(j_i:X_i\to X)$による像位相によって$(X_i)$は$X$の開被覆となり,$j_i$は開埋め込みになる.
        \item 任意の$X_i\cap X_j$は$X_i,X_j$の開部分空間となり,$X_i$からの相対位相と$X_j$からの相対位相とは等しい.
    \end{enumerate}
\end{theorem}

\subsection{終位相の普遍性}

\begin{tcolorbox}[colframe=ForestGreen, colback=ForestGreen!10!white,breakable,colbacktitle=ForestGreen!40!white,coltitle=black,fonttitle=\bfseries\sffamily,
title=]
    連続全射な開・閉写像が存在するならば,終域の位相は像位相である.
\end{tcolorbox}

\begin{proposition}[像位相の特徴付け]
    $(X_i)_{i\in I}$を位相空間の族,$(f_i:X_i\to X)_{i\in I}$を写像の族とする.
    \[ \O=\{U\in P(X)\mid \forall_{i\in I}\; f^{-1}_i(U)\in\O_{X_i}\} \]
    は$X$の位相である.この位相を,写像の族$(f_i)_{i\in I}$による\textbf{像位相}という.
\end{proposition}
\begin{proof}
    開集合の公理を満たすことを示す.$(U_j)_{j\in J}$を$\O$の族とする.
    \begin{enumerate}
        \item 任意の$i\in I$について$f^{-1}_i(\cup_{j\in J}U_j)=\cup_{j\in J}f^{-1}_i(U_j)\in\O_{X_i}$が成り立つから,$\cup_{j\in J}U_j\in\O$.
        \item $\abs{J}<\infty$とする.任意の$i\in I$について$f^{-1}_i(\cap_{j\in J}U_j)=\cap_{j\in J}f^{-1}_i(U_j)\in\O_{X_i}$が成り立つから,$\cap_{j\in J}U_j\in\O$.
    \end{enumerate}
\end{proof}

\begin{proposition}[終位相の普遍性]
    $X$を写像の族$(f_i:X_i\to X)_{i\in I}$による像位相によって位相空間と見做す.
    任意の位相空間$Y$と任意の写像$g:X\to Y$に対し,次の2条件は同値.
    \begin{enumerate}
        \item $g:X\to Y$は$\O$に関して連続である.
        \item 任意の$i\in I$に対し,$g\circ f_i:X_i\to Y$が連続である.
    \end{enumerate}
\end{proposition}
\begin{proof}\mbox{}
    \begin{description}
        \item[(1)$\Rightarrow$(2)] 像位相は各$f_i$を連続にする最も細かい位相であるから,$f_i$は連続.(1)より$g$も連続であるから,$g\circ f_i$は連続.
        \item[(2)$\Rightarrow$(1)] 任意の$U\in\O_Y$に対して,$g^{-1}(U)\in\O$を示せば良い.
        \begin{align*}
            &(g\circ f_i)^{-1}(U)\in\O_{X_i}\\
            \Leftrightarrow\quad&f^{-1}_i(g^{-1}(U))\in\O_{X_i}\\
            \Leftrightarrow\quad&g^{-1}(U)\in\O.
        \end{align*}
    \end{description}
\end{proof}

\begin{proposition}
    $f:X\to Y$を全射な連続写像とする.
    \begin{enumerate}
        \item $f$が開写像ならば,$\O_Y=f^*(\O_X)=\Brace{f(U)\in P(Y)\mid U\in\O_X}$であり,$f$による像位相である.
        \item $f_!(A):=\Brace{y\in Y\mid f^{-1}(y)\subset A}$として$f_i:P(X)\to P(Y)$を定める.$f$が閉写像ならば,$\O_Y=\Brace{f_!(U)\in P(Y)\mid U\in\O_X}$であり,$f$による像位相である.
    \end{enumerate}
\end{proposition}

\subsection{終位相のネット}

\begin{proposition}
    $Y$に,写像の族$\F:=\{f:X_f\to Y\}_{f\in\F}$によって終位相を入れる.このとき,次の2条件は同値:
    \begin{enumerate}
        \item $g:Y\to Z$は連続.
        \item $\forall_{f\in\F}\;g\circ f:X_f\to Z$は連続.
    \end{enumerate}
\end{proposition}

\subsection{帰納極限}

\begin{definition}
    位相空間の列$(X_n)$は,埋め込み$f_n:X_n\mono X_{n+1}$で結ばれているとする.
    空間$X:=\cup_{n\in\N}X_n$に$(f_n)$による終位相を入れた空間を\textbf{帰納的極限}と呼び,$\varinjlim X_n$と表す.
    一般の図式$\{X_\lambda\}_{\lambda\in\Lambda}$についても同様に定義できる.
\end{definition}

\begin{theorem}[帰納極限の普遍性]
    帰納系$(X_n)$と位相空間$Y$と連続写像の列$\varphi_n:X_n\to Y$が存在し,図式が可換になるとする.
    このとき,唯一つの連続写像$F:\varinjlim\to Y$が存在し,図式を可換にする.
\end{theorem}

\begin{example}
    $(\R^n)$を通常の包含によって帰納系と見る.
    \[X:=\Brace{x\in\R^n\mid x_n=0\;\fe}\]
    となるが,これは$\R^N$の積位相が定める相対位相よりも強い.
\end{example}

\section{ネット}

\begin{tcolorbox}[colframe=ForestGreen, colback=ForestGreen!10!white,breakable,colbacktitle=ForestGreen!40!white,coltitle=black,fonttitle=\bfseries\sffamily,
title=]
    点列の収束を,開近傍のフィルター$\O(x)$について,$\forall_{U\in\O(x)}\;x_n\in U\;\fe$と特徴付けられることに注目すると,
    一般の上に有向な集合からの写像について,収束の概念を定義できる.
    そして,収束するフィルターを指定することで,位相を指定することが出来るという同値な構成を得る.
\end{tcolorbox}

\subsection{ネットの収束}

\begin{tcolorbox}[colframe=ForestGreen, colback=ForestGreen!10!white,breakable,colbacktitle=ForestGreen!40!white,coltitle=black,fonttitle=\bfseries\sffamily,
title=]
    点列$(x_n):\N\to X$が収束することは,終局フィルター$F_{(x_n)}:=\Brace{A\in P(X)\mid x_n\in A\;\fe}$が$\O(x)$を含むことに同値.
\end{tcolorbox}

\begin{definition}[direction, directed set, net / generalized sequences]
    $(D,\le)$を前順序集合とする.すなわち,$\le$は反射的で,推移的である.
    \begin{enumerate}
        \item 前順序$\le$が任意の2元について上界を持つとき,\textbf{方向}であるという.
        \item 組$(D,\le)$を\textbf{有向集合}という.
        \item 有向集合$D$からの写像$i:D\to X$を\textbf{$X$上のネット}または\textbf{有向系}という.ネットも$(x_n)_{n\in D}$と表し,$x_n=i(n)$とする.
    \end{enumerate}
\end{definition}

\begin{definition}[eventuality filter / cofinitely often]
        $\nu:D\to X$を集合$X$上のネットとする.\textbf{ネット$\nu$が定めるフィルター}$F_\nu$とは,
        \[F_\nu:=\Brace{A\in P(X)\mid \exists_{i\in D}\;\forall_{j\ge i}\;\nu_j\in A}\]
        のことである.この条件を$\nu$は$A$に\textbf{終局}するという.
        $D=\N$のとき,$\nu\in A\;\fe$と同値.
\end{definition}

\begin{definition}[convergence of net, limit point, cluster / accumulation point, universal / ultranet]
    $X$を位相空間,$F$を$S$上のフィルターとする.
    \begin{enumerate}
        \item ネット$n:D\to X$が$x\in X$に\textbf{収束}するとは,任意の$x$の開近傍(従って近傍)$A\in\O(x)$に,$n$が終局する$A\in F_n$ことをいう.このとき,$x$を\textbf{極限点}という.$X$がHausdorffのとき,一意に定まる.
        \item ネット$n:D\to X$が$x\in X$に\textbf{集積}するとは,任意の$x$の開近傍(従って近傍)$A\in\O(x)$に,$n$が無限回入ることをいう:$\forall_{i\in D}\;\exists_{j\ge i}\;n_j\in A$.このとき,$x$を集積点という.実は,任意の集積点は必ず部分ネットの極限点になる.
        \footnote{これは英語でclusterと\url{https://ncatlab.org/nlab/show/filter}に乗っているが,filterについてはclusterとaccumulateは同じ定義になるのだろうか?}
        \item ネット$n:D\to X$が\textbf{普遍的}であるとは,任意の集積点が極限点であることをいう.
        同値だが,任意の$Y\in P(X)$について,$Y\in F_\nu\lor X\setminus Y\in F_\nu$が成り立つ.
        これは,終局フィルターが極大であることに同値.
    \end{enumerate}
\end{definition}

\begin{proposition}
    $(x_\lambda)$を普遍ネットとする.任意の写像$f:X\to Y$に関して,$f(x_\lambda)$も普遍ネットである.
\end{proposition}

\subsection{部分ネットと普遍ネット}

\begin{tcolorbox}[colframe=ForestGreen, colback=ForestGreen!10!white,breakable,colbacktitle=ForestGreen!40!white,coltitle=black,fonttitle=\bfseries\sffamily,
title=]
    ネットについては,任意の空間のコンパクト集合が「ネットコンパクト」になる.
    $A$-集積点であることと$A$-ネットの極限点であることとが同値になる.
\end{tcolorbox}

\begin{definition}[subnet (Kelley 1955)]\label{def-subnet-Kelley}
    $y:=(y_\beta)_{\beta\in B}$が$x:=(x_\al)_{\al\in A}$の部分ネットであるとは,次の2条件を満たす写像$f:B\to A$が存在することをいう:
    \begin{enumerate}
        \item $y=x\circ f$,すなわち,$\forall_{\beta\in B}\;x_{f(\beta)}=y_\beta$.
        \item strongly cofinal:$\forall_{\alpha\in A}\;\exists_{\beta\in B}\;\forall_{\beta_1\ge\beta}\;f(\beta_1)\ge\alpha$.
    \end{enumerate}
    $f:B\to A$を単調に取ったならば,あとは$\forall_{\alpha\in A}\;\exists_{\beta\in B}\;\alpha\le f(\beta)$を示せば,(2)を満たす.
\end{definition}

\begin{remark}
    WillardもKelley \ref{def-subnet-Kelley}も,(ネット$n:D\to X$同様)$f$に単射性を要求していない点に注意.したがって,$A=\N$としても,通常の部分列の定義よりは一般的である.
\end{remark}

\begin{lemma}[集積点の特徴付け]
    $\B\subset P(X)$は包含順序について下に有向とする:$\forall_{A,B\in\B}\;\exists_{C\in\B}\;C\subset A\cap B$とする.
    ネット$(x_\lambda)$は$\B$の各元$A\in\B$に集積するとする:
    $\forall_{A\in\B}\;\forall_{\lambda\in\Lambda}\;\exists_{\lambda\le\nu\in\Lambda}\;x_\nu\in A$.
    このとき,ある部分ネット$(x_{h(\mu)})_{\mu\in M}$が存在して,$\B$の任意の元に収束する:$\B\subset F_{(x_{h(\mu)})}$.
\end{lemma}
\begin{proof}\mbox{}
    \begin{enumerate}[(a)]
        \item $M:=\Brace{(\lambda,A)\in\Lambda\times\B\mid x_\lambda\in A}$は積順序に関して,上に有向である.
        任意の$(\lambda,A),(\mu,B)\in M$に関して,$\exists_{C\in\B}\;C\subset A\cap B$で,$\exists_{\nu\in\Lambda}\;\lambda\le\nu\land\mu\le\nu$.
        $C\in\B$に集積することより,$\exists_{\nu'\ge\nu}\;x_{\nu'}\in C$だから,$(\nu',C)$が取れて,$(\lambda,A)\le(\nu',C)$かつ$(\mu,B)\le(\nu',C)$.
        \item $h:M\to\Lambda$を第一射影によって定めると,$(x_{h(\mu)})_{\mu\in M}$は部分ネットであり,$\B$の任意の元に終局する.
    \end{enumerate}
\end{proof}
\begin{remarks}
    フィルターは$\B$たる要件を満たす.
\end{remarks}

\begin{corollary}
    位相空間$X$上のネットの任意の集積点$x$について,ある部分ネットが存在して$x$に終局する.
\end{corollary}
\begin{proof}
    $\B$を$\O(x)$と取れば良い.
\end{proof}

\begin{theorem}[existence of universal nets (AC)]
    任意のネットは,部分ネットとして普遍ネットを持つ.
\end{theorem}
\begin{proof}\mbox{}
    \begin{enumerate}[(a)]
        \item Zornの補題より,ネット$(x_\lambda)$の終局フィルターのうち,極大なものが取れる.
        \item 任意の$A\in P(X)$について,
        \[\forall_{\lambda\in\Lambda}\;\forall_{E,F\in\F}\;\exists_{\mu\ge\lambda}\;x_\mu\in E\cap A\lor x_\mu\in F\setminus A.\]
        が示せる.
        \item これは,$\forall_{A\in P(X)}\;A\in\F\lor X\setminus A\in\F$を意味する.
        \item 補題を$\B=\F$として適用すると,$\F$を終局フィルターとするネットが取れ,このネットは普遍的である.
    \end{enumerate}
\end{proof}
\begin{remarks}
    本質的には,任意のフィルターに,それを含む超フィルターが存在することによる.
\end{remarks}

\subsection{ネットによる閉包演算子の特徴付け}

\begin{tcolorbox}[colframe=ForestGreen, colback=ForestGreen!10!white,breakable,colbacktitle=ForestGreen!40!white,coltitle=black,fonttitle=\bfseries\sffamily,
title=]
    これにより,どのようなネットが収束するかを指定することは,位相を指定することに等価であることが分かる.
    またこの手法は,空間が第1可算のとき,数列のみによって殆ど同様に展開できる.
    これは,任意の数列の集積点に対して,これに収束する部分列が取れるためである.
\end{tcolorbox}

\begin{proposition}[(AC)]
    任意の位相空間$Y\subset X$と点$x\in X$について,次の2条件は同値.
    \begin{enumerate}
        \item $x\in\o{Y}$.
        \item $\forall_{A\in\O(x)}\;A\cap Y\ne\emptyset$.
        \item $x$に収束する$Y$のネットが存在する.
    \end{enumerate}
\end{proposition}
\begin{proof}
    (1)$\Leftrightarrow$(2)は良い.
    \begin{description}
        \item[(1),(2)$\Rightarrow$(3)] 選択公理より,$(x_A)_{A\in\O(x)}\in\prod_{A\in\O(x)}A\cap Y$が得られる.
        これは$x$に収束するネットである.実際,任意の開近傍$U\in\O(x)$に対して,任意の$U\subset V\in\O(x)$は$x_V\in V$を満たす.
        \item[(2)$\Rightarrow$(1)] $(x_\lambda)_{\lambda\in\Lambda}$を$x$に収束するネットとすると,$\forall_{A\in\O(x)}\;\exists_{\lambda\in\Lambda}\;\forall_{\lambda'}\;\forall_{\lambda'\ge\lambda}\;x_{\lambda'}\in A$.すなわち,$x_{\lambda'}\in A\cap Y\ne\emptyset$.
    \end{description}
\end{proof}

\subsection{フィルターの収束}

\begin{definition}
    フィルター$F\subset P(X)$が$x\in X$に収束するとは,$\O(x)\subset F$を満たすことをいう.
    このとき,ネット$n:D\to X$が収束することと,その終局フィルター$F_n$が収束することとは同値.
\end{definition}

\begin{proposition}[開集合の特徴付け]
    部分集合$Y\subset X$について,次の2条件は同値:
    \begin{enumerate}
        \item $Y$は開集合である.
        \item 任意の$\exists_{x\in Y}\;\O(x)\subset F$を満たすフィルター$F\subset P(X)$について,$Y\in F$.
    \end{enumerate}
\end{proposition}

\subsection{極大フィルターの特徴付け}

\begin{theorem}
    $\F\subset P(X)$が\textbf{超フィルター}であるとは,次の同値な条件を満たすもののことをいう:
    \begin{enumerate}
        \item 真のフィルターのうち,極大なものである.
        \item $\forall_{A\in P(X)}\;A\in\F\lor X\setminus A\in\F$.
        \item 真のフィルターであって,次を満たす:$\forall_{B\in\F}\;A\cap B\ne\emptyset\Rightarrow A\in\F$.
        \item $A\in\F\Leftrightarrow\forall_{B_1,\cdots,B_n\in\F}\;A\cap B_1\cap\cdots\cap B_n\ne\emptyset$.
    \end{enumerate}
\end{theorem}

\begin{proposition}[(AC)]
    任意のフィルターに対して,これを含む極大フィルターが存在する.
\end{proposition}

\subsection{ネットによる連続写像の特徴付け}

\begin{proposition}
    写像$f:X\to Y$と点$x\in X$について,次の2条件は同値.
    \begin{enumerate}
        \item $f$は$x$で連続である:$f^*(\O(f(x)))\subset\O(x)$.
        \item 任意の$x$に収束するネット$(x_\lambda)$について,$f(x_\lambda)\to f(x)$.
    \end{enumerate}
\end{proposition}
\begin{proof}\mbox{}
    \begin{description}
        \item[(1)$\Rightarrow$(2)] 任意の$x$に収束するネット$(x_\lambda)$を取る.$\forall_{A\in\O(f(x))}\;\exists_{B\in\O(x)}\;f(B)=A$だから,このネットは$f^{-1}(A)$に終局する.よって,ネット$(f(x_\lambda))$は$A$に終局する.
        \item[(2)$\Rightarrow$(1)] $\exists_{A\in\O(f(x))}\;f^{-1}(A)\notin\O(x)$とすると,$x\in X\setminus f^{-1}(A)$となり,ネットによる閉包の特徴付けより,ある$x$に収束する$X\setminus f^{-1}(A)$のネットが存在してしまう.これは$\forall_{\lambda\in\Lambda}\;f(x_\lambda)\notin A$を満たすネットだから,$f(x)$に収束せず,矛盾.
    \end{description}
\end{proof}

\chapter{距離空間}

\begin{quotation}
    抽象位相空間論を展開したが,元々位相空間論の端緒は,Fr\`{e}chet (1906)が距離空間の概念を定義してCantorの概念を一般化したことに始まる.
    これがそもそも,Euclid空間から離れて抽象的に空間を把握した最初である.
    その際のモチベーションは関数空間であった.
    そこで,関数空間の例を意識しながら,距離空間の扱い方を考える.
    \begin{enumerate}
        \item 距離空間のコンパクト性は,点列コンパクト性によって特徴付けられ,そのときに完備性の概念が見つかる.
        \item 可分位相空間の距離化可能性は,第2可算性と正規性によって特徴付けられる.このときに,可算性・可分性が問題になるのである.
    \end{enumerate}
\end{quotation}

\section{距離関数の扱い}

\subsection{同値な距離}

\begin{proposition}
    $(X,\rho)$を距離空間とする.
    \begin{enumerate}
        \item $\wt(\rho)(x,y):=\min(\rho(x,y),1)$.
        \item $\wt(\rho)(x,y):=\frac{\rho(x,y)}{1+\rho(x,y)}$
    \end{enumerate}
    とすると,$\wt{\rho}$も距離空間で,$(X,\rho)$と$(X,\wt{\rho})$の開集合系は一致する.
\end{proposition}

\subsection{開集合の特徴付け}

\begin{proposition}
    $X$を距離空間,$U\subset X$とする.次の2条件は同値:
    \begin{enumerate}
        \item $U$は開集合.
        \item $\exists_{f\in C(X)}\;U=\Brace{x\in X\mid f(x)\ne0}$.
    \end{enumerate}
\end{proposition}

\subsection{距離空間は正規}

\begin{proposition}
    $A,B\subset X$を互いに素な閉集合とする.このとき,$\exists_{f\in C(X)}\;A=f^{-1}(0)\land B=f^{-1}(1)$.
\end{proposition}

\subsection{閉包の特徴付け}

\begin{proposition}[閉包の距離関数による特徴付け]\label{prop-characterization-of-closure-in-terms-of-metric-function}
    $d(-,A):X\to\R$を$\emptyset\ne A\subset X$からの距離とする.
    \begin{enumerate}
        \item $|d(x,A)-d(y,A)|\le d(x,y)$が成り立つ.また関数$d(-,A):X\to\R$は連続である.
        \item $\overline{A}=\{x\in X\mid d(x,A)=0\}$である.
    \end{enumerate}
\end{proposition}
\begin{proof}\mbox{}
    \begin{enumerate}
        \item 三角不等式より,\[\forall x,y\in X,\forall a\in A,\;d(x,a)\le d(x,y)+d(y,a)\]
        である.両辺の下限をとって,$d(x,A)\le d(x,y)+d(y,A)\Rightarrow d(x,A)-d(y,A)\le d(x,y)$.同様に$d(y,A)-d(x,A)\le d(x,y)$も成り立つから,$|d(x,A)-d(y,A)|=\max\{d(x,A)-d(y,A),d(y,A)-d(x,A)\}\le d(x,y)$.
        すると,$d_\R=|xーy|$に注意して,系\ref{cor-characterization-distance-function}.1より,$d(-,A):X\to\R$は連続.
        \item $\overline{A}=\{x\in X\mid \forall r>0,\;U_r(x)\cap A\ne 0\}$より.
    \end{enumerate}
\end{proof}

\subsection{等距離写像}

\begin{tcolorbox}[colframe=ForestGreen, colback=ForestGreen!10!white,breakable,colbacktitle=ForestGreen!40!white,coltitle=black,fonttitle=\bfseries\sffamily,
title=]
    距離空間の射は等長写像という.
\end{tcolorbox}

\begin{definition}[isometry, global isometry]\mbox{}
    \begin{enumerate}
        \item $(X,d_X),(Y,d_Y)$を距離空間とする.写像$f:X\to Y$が$d_X=f^*d_Y$,すなわち,
        $\forall x,x'\in X,\; d_Y(f(x),f(x'))=d_X(x,x')$を満たす時,これを\textbf{等長写像}という.
        \item 等長写像が全単射であり,逆写像も等長写像であるとき,\textbf{大域的等長写像}という.
    \end{enumerate}
\end{definition}
\begin{remark}
    距離空間の等長写像は,全単射ならば大域的であるが,
    Riemann多様体では,全単射な等長写像と大域的等長写像とは違う概念になる.
\end{remark}

\begin{proposition}[部分距離空間のwell-definedness]\label{prop-部分距離空間のwell-definedness}\mbox{}
    \begin{enumerate}
        \item $f:X\to Y$が等長写像ならば,$f$は埋め込みである.
        \item 部分距離空間$A\subset X$の定める位相は,$X$の部分位相としての$A$の位相と等しい.
    \end{enumerate}
\end{proposition}
\begin{proof}\mbox{}
    \begin{enumerate}
        \item 系\ref{cor-characterization-distance-function}より,$f$は特に連続である.
        また,$f(x)=f(x')$とすると,$d_X(x,x')=d_Y(f(x),f(x'))=0$だから,$x=x'$が従うので$f$は単射である.
        あとは$f^*\O_Y=\O_X$を示す.連続性より,$f^*\O_Y\supset\O_X$を示せば良い.
        $f$の連続性より,$\U_X\subset f^*\U_Y\subset f^*\O_Y$.命題\ref{prop-generated-topology}より$\O_X\subset f^*\O_Y$が従う.
        \item 包含写像$i:A\to X$は等長写像だから,埋め込みである.よって,$i:A\to X$は部分距離空間の$A$と部分空間$A$は同相である.
    \end{enumerate}
\end{proof}

\subsection{ノルム空間の等長写像}

\begin{definition}[strictly convex]
    ノルム空間$F$が\textbf{厳密に凸}であるとは,$\forall_{u,v\in\partial B}\;\forall_{t\in(0,1)}\;tu+(1-t)v\in B^\circ$が成り立つことをいう.
    $L^p\;(1<p<\infty)$空間は厳密に凸である.
\end{definition}

\begin{theorem}
    ノルム空間の等長写像$f:E\to F$は,$F$が厳密に凸のとき,affineである.
\end{theorem}
\begin{proof}
    連続性より,
    \[f\paren{\frac{1}{2}(x+y)}=\frac{1}{2}(f(x)+f(y))\]
    を示せば十分.
\end{proof}

\begin{theorem}[Mazur-Ulam]
    $E,F$を実ノルム空間,$f:E\to F$を等長な線形写像とする.このとき,$f$はaffineである.
\end{theorem}

\subsection{非Archimedesな距離空間の例}


\begin{tcolorbox}[colframe=ForestGreen, colback=ForestGreen!10!white,breakable,colbacktitle=ForestGreen!40!white,coltitle=black,fonttitle=\bfseries\sffamily,
    title=]
        絶対値の概念を,ノルムに一般化するのではなく,一般の体に一般化する道もある.
\end{tcolorbox}

\begin{definition}[valuation]
    体$K$上の写像$v:K\to\R$が\textbf{付値}であるとは,次の3条件を満たすことをいう:
    \begin{enumerate}
        \item $v\ge0$かつ$v^{-1}(0)=\{0\}$.
        \item 積の分解:$\forall_{x,y\in K}\;v(xy)=v(x)v(y)$.
        \item 三角不等式:$\forall_{x,y\in K}\;v(x+y)\le v(x)+v(y)$.
    \end{enumerate}
    また,$v_p(x+y)\le\max\Brace{v_p(x),v_p(y)}$なる関係が成り立つ付値を\textbf{非Archimedes的付値}という.
\end{definition}

\begin{example}[$p$-adic valuation]
    $p$を素数とする.
    \begin{enumerate}
        \item $x\ne0$が整数のとき,$e\in\N$と$p$と互いに素な整数$r$を用いて$x=\pm p^er$と表せる.$e=:e_p(x)$を\textbf{$p$-指数}という.
        \item $x\ne0$が一般の有理数であるとき,$\exists_{r,s\in\Z\setminus\{0\}}\;x=r/s$.このとき,$e_p(x):=e_p(r)-e_p(s)$とおく.
        \item 最後に,$x\in\Q\setminus\{0\}$に対して,$v_p(x):=p^{-e_p(x)}$とする.
    \end{enumerate}
    このとき,$v_p(x+y)\le\max\Brace{v_p(x),v_p(y)}$が成り立ち,付値となる.
    これを\textbf{$p$進付値}という.
    $\Q$の絶対値による付値を\textbf{無限付値}または\textbf{Archimedes的付値}という.
    $\Q_p$はこの距離について完備で,\textbf{$p$-進体}という.
    \[\Z_p:=\Brace{a\in\Q_p\mid v_p(a)\le1}\]
    は$\Q_p$の部分環となる.
\end{example}

\begin{theorem}[付値が定める距離空間]
    
\end{theorem}

\begin{example}[Baireの(0次元)空間]
    集合$\Om$上の点列の空間$\Om^\N$を距離関数$\rho(x,y)=\sum_{n\in\N}\frac{1}{n}1_{\forall_{i\in n}x_i=y_i\land x_n\ne x_n}$によって距離空間と見ると,
    $\rho(x,z)\le\max(\rho(x,y),\rho(y,z))$が成り立つ.
    これを$B(\Om)$とも表す.
\end{example}

\begin{proposition}[非Archimedesな距離空間の開球は閉]
    $\rho$が$X$上の非Archimedes的距離関数とする.このとき,開球$U(x;\ep)$は閉集合でもある.
    すなわち,$(X,\rho)$は開かつ閉な集合からなる開基を持つ.このような位相空間を\textbf{$0$次元}であるという.
\end{proposition}

\subsection{距離概念の一般化}

\begin{tcolorbox}[colframe=ForestGreen, colback=ForestGreen!10!white,breakable,colbacktitle=ForestGreen!40!white,coltitle=black,fonttitle=\bfseries\sffamily,
    title=]
        Lawvere距離空間の観点からすると,short mapが正確にenriched functorとなる.
        これが,Metの射は連続写像でも,一様連続写像でもなくて,short mapが圏論的には良さそうだという指針になる.
\end{tcolorbox}

\begin{definition}[extended, pseudo-, quasi-, Lawvere]\mbox{}
    \begin{enumerate}
        \item 距離関数$d$の終域を$d:X\to[0,\infty]$としたものを,\textbf{拡大距離空間}という.終域を一般の有向集合とすると,一様空間の概念を得る.
        \item 公理(1) Separationを除いたものを,\textbf{擬距離}(または半距離)という.関数空間上の半ノルムが生成する.擬距離が距離であることと,生成する位相が$T_0$であることは同値.
        \item 公理(3) Symmetryを除いたものを,\textbf{準距離}という.\footnote{一方通行や坂道の概念など,日常にはありふれている距離概念である.}
        \item 擬距離であり,準距離でもある拡張距離空間を,\textbf{Lawvere距離空間}という.
    \end{enumerate}
\end{definition}

\begin{lemma}
    Lewvere距離空間$M$は,順序集合$([0,\infty],\le)$を加法$+$によってモノイド圏とみなして,この上の豊穣圏とみなせる.
    順序集合自体が$2$を積$\cdot$によってモノイド圏とみなした上の豊穣圏とみなせるから,$M$での射の合成は三角不等式と一致する.\footnote{Thus generalized, many constructions and results on metric spaces turn out to be special cases of yet more general constructions and results of enriched category theory. }
\end{lemma}

\subsection{一様空間の2つの定義}


\begin{tcolorbox}[colframe=ForestGreen, colback=ForestGreen!10!white,breakable,colbacktitle=ForestGreen!40!white,coltitle=black,fonttitle=\bfseries\sffamily,
    title=一様収束の概念による距離空間の一般化である]
    entourageによる定義はBourbaki流であり,いまだに一番一般的に見られる.FinSetsから開集合系の圏Opへの反変関手を記述しているとみなせる.
    一様空間の概念はAndré Weilが提案したもので,一様連続写像に自然に言及できる空間である.これは距離空間と位相群の概念の拡張とみなせる.\footnote{\url{https://ncatlab.org/nlab/show/uniform+space}}
\end{tcolorbox}

\begin{definition}[entourage / vicinities, uniform structure / uniformity, uniform space (Bourbaki)]
    $X$を集合とする.次の6公理を満たすような$X$上の\textbf{近縁}と呼ばれる二項関係$U\subset X\times X$のフィルター$\U\subset P(X\times X)$を\textbf{一様構造}と呼び,この組$(X,\U)$を\textbf{一様空間}という.
    \begin{enumerate}
        \item (reflexive) $\forall_{U\in\U}\;\Delta_X\subset U$.
        \item (transitive) $\forall_{U\in\U}\;\exists_{V\in\U}\;V\circ V\subset U$.即ち,$\forall_{U\in\U}\;\exists_{V\in\U}\;\forall_{x,y,z\in X}\;x\approx_V y\approx_V z\Rightarrow x\approx_U z$.
        \item (symmetric) $\forall_{U\in\U}\;\exists_{V\in\U}\;V^\op\subset U$.即ち,$\forall_{U\in\U}\;\exists_{V\in\U}\;\forall_{x,y\in X}\;y\approx_V x\Rightarrow x\approx_Uy$(6と合わせると,$V\subset U^\op\in\U$が必要).
        \item (inhabited) $\U\ne\emptyset$.6と合わせると,$X\times X\in\U$が必要.
        \item (downward-directed) $\forall_{U,V\in\U}\;\exists_{W\in\U}\;W\subset U\cap V$.6の下では$W=U\cap V\in\U$が必要.
        \item (upward-closed) $\forall_{U\in\U}\;U\subset V\subset X\times X\rightarrow V\in\U$.
    \end{enumerate}
    関係$(x,x')\in U$を,\textbf{$x\in X$と$x'\in X$とは$U$の程度に近い}という.
    (1)から(5)を満たす関係の族を,近縁の基と呼ぶ.
\end{definition}
\begin{remarks}
    (4)から(6)は$\U$がフィルターになる条件である.
\end{remarks}

\begin{example}[加法構造]\mbox{}
    \begin{enumerate}
        \item $\R$上の族$V_\al:=\Brace{(x,y)\in\R\times\R\mid\abs{x-y}<\al}_{\al>0}$は一様構造となる.これを\textbf{加法一様構造}という.
        \item 集合$X$上の同値関係$R$は一様構造の近縁の基本系となる.$R$が自明であるとき$R=\Delta_X$,これは一様構造を定め,\textbf{離散一様構造}という.
    \end{enumerate}
\end{example}

\begin{definition}[被覆]\mbox{}
    \begin{enumerate}
        \item 位相空間$X$の被覆$C\subset P(X)$とは,$\cup C=X$を満たす$P(X)$の族をいう.
        \item $C_1$は$C_2$の細分である$C_1<C_2$とは,次が成り立つことをいう:$\forall_{A\in C_1}\;\exists_{B\in C_2}\;A\subset B$.
        \item $C_1,C_2$の結びとは,$C_1\land C_2:=\Brace{A\cap B\in P(X)\mid A\in C_1,B\in C_2}$のことを指す.これは再び被覆となる.
        \item $A\in P(X)$について,$C[A]:=\bigcup\Brace{B\in C\mid A\cap B\ne\emptyset}$と定める.
        \item $C^*:=\Brace{C[A]\subset P(X)\mid A\in C}$と定める.
        \item 近縁$U$について,$U[x]:=\Brace{y\in X\mid x\approx_U y}$と定める.
    \end{enumerate}
\end{definition}

\begin{definition}[Covering uniformities]
    位相空間$X$の一様被覆$\cC$とは,次を満たす被覆の族$\cC$のことをいう:
    \begin{enumerate}
        \item $\forall_{C\in\cC}\;\exists_{C'\in\cC}\;(C')^*<C$.
        \item (inhabited) $\U\ne\emptyset$.(4)の下では,$\{X\}\in\cC$.
        \item (downward-directed) $\forall_{C_1,C_2\in\cC}\;\exists_{C_3\in\cC}\;C_3<C_1\land C_2$.
        \item (upward-closed) $\forall_{C\in\cC}\;C<C'\Rightarrow C'\in\cC$.
    \end{enumerate}
\end{definition}
\begin{remarks}
    公理(2)から(4)が$\cC$がフィルターであるための条件である.
    (1)だけが本質的で,これを満たすものを準基,(4)以外を満たすものを基という.
\end{remarks}

\begin{lemma}[2つの定義の同値性]\mbox{}
    \begin{enumerate}
        \item 近縁系$\U$に対して,族$C\subset P(X)$が,ある近縁$U\in\U$について$\{U[x]\in P(X)\mid x\in X\}<C$を満たすとき$C\in\cC$とすると,$\cC$は一様構造を定める.
        \item 一様被覆系$\cC$に対して,$C\in\cC$を一様被覆として$\bigcup\Brace{A\times A\mid A\in C}$が生成する近縁系$\U$は一様構造を定める.
        \item (1),(2)の対応は全単射を定める.
    \end{enumerate}
\end{lemma}

\subsection{一様構造の定める位相}


\begin{definition}[uniform topology]
    近縁系$\U$について,$U[x]=\Brace{y\in X\mid x\approx_Uy}$を$x\in X$の近傍のフィルターとする位相はただ一つ存在し,これを\textbf{一様構造$\U$が定める位相}という.
\end{definition}

\begin{proposition}
    一様空間$(X,\U)$を一様位相によって位相空間とみなす.$X$は正規かつ完備正規(completely regular)である.
\end{proposition}

\begin{definition}[uniformly continous function]
    $(X,\U_X),(Y,\U_Y)$を一様空間とする.
    \begin{enumerate}
        \item 関数$f:X\to Y$が一様連続であるとは,$\forall_{U\in\U_Y}\;(f\times f)^{-1}(V)\in\U_X$すなわち$(f\times f)^*\U_Y\subset\U_X$が成り立つことをいう.
        \item 一様空間と一様連続写像は具体圏をなし,これをUnifと書く.
    \end{enumerate}
\end{definition}

\begin{lemma}
    $(X,\U_X),(Y,\U_Y)$を一様空間とする.
    \begin{enumerate}
        \item 一様連続関数$f:X\to Y$は,対応する一様位相について連続である.
        \item 一様位相について連続だが,一様連続ではない関数$f:X\to Y$が存在する.
    \end{enumerate}
\end{lemma}

\subsection{位相群}

\section{コンパクト距離空間}

\begin{tcolorbox}[colframe=ForestGreen, colback=ForestGreen!10!white,breakable,colbacktitle=ForestGreen!40!white,coltitle=black,fonttitle=\bfseries\sffamily,
title=]
    コンパクトの概念は,Frechét, M.が距離空間について初めて導入した.
\end{tcolorbox}

\subsection{定義}

\begin{tcolorbox}[colframe=ForestGreen, colback=ForestGreen!10!white,breakable,colbacktitle=ForestGreen!40!white,coltitle=black,fonttitle=\bfseries\sffamily,
title=]
    コンパクト性も,連結性と同様,全体空間$X$に依らずに定義出来る概念であるとわかる.
\end{tcolorbox}

\begin{definition}
    部分空間$A\subset X$が\textbf{コンパクト}であるとは,$X$の任意の開集合族が$A$の開被覆となっているならば,$A$の有限な部分被覆が選び出せることをいう.
\end{definition}

\subsection{閉集合による特徴付け}

\begin{proposition}[コンパクト性の特徴付け:閉集合の言葉による双対命題]\label{prop-characterization-of-compactness-as-space}
    位相空間$X$の部分集合$A$に対して,次の3つの条件は同値である.
    \begin{enumerate}
        \item $A$はコンパクトである.
        \item $X$の閉集合の任意の族$(F_i)_{i\in I}$について,次の条件(C')が成り立つ.\begin{quote}
            (C') $I$の任意の有限部分集合$\{i_1,\cdots,i_n\}$に対して,$A\cap F_{i_1}\cap\cdots\cap F_{i_n}\ne\emptyset$ならば,$A\cap\bigcap_{i\in I}F_i\ne\emptyset$である.
        \end{quote}
        \item $X$の部分位相空間としての$A$はコンパクトである.
    \end{enumerate}
\end{proposition}
\begin{proof}
    \begin{description}
        \item[(1)$\Leftrightarrow$(2)] 条件(C)の対偶「有限部分被覆が存在しないならば,それはそもそも開被覆ではない」は
        \begin{quote}
            (C'') 任意の$I$の有限部分集合$\{i_1,\cdots,i_n\}$について$A\not\subset U_{i_1}\cup\cdots\cup U_{i_n}$ならば,$A\not\subset \cup_{i\in I}U_i$である.
        \end{quote}
        となる.$F_i:=X\setminus U_i$と定めるとこれは閉集合で,
        \begin{align*}
            A\not\subset U_{i_1}\cup\cdots\cup U_{i_n}&\Leftrightarrow A\cap F_{i_1}\cap\cdots\cap F_{i_n}\ne\emptyset,\\
            A\not\subset \cup_{i\in I}U_i&\Leftrightarrow A\cap\bigcap_{i\in I}F_i.
        \end{align*}
        \item[(1)$\Leftrightarrow$(3)]
        \begin{align*}
            A\subset\bigcup_{i\in I}U_i&\Leftrightarrow A=A\cap\bigcup_{i\in I}U_i\Leftrightarrow \bigcup_{i\in I}(A\cap U_i)
        \end{align*}
        より.
    \end{description}
\end{proof}

\subsection{最大値の定理による特徴付け}

\begin{proposition}[maximum value theorem]\label{prop-maximum-value-theorem}
    $X$を位相空間とし,$f:X\to\R$を連続関数とする.$A$が$X$のコンパクト集合であり空でないならば,$f$の$A$への制限には最大値が存在する.
\end{proposition}
\begin{proof}
    $\forall x\in A,\;\exists y\in A,\; f(x)<f(y)$と仮定して,矛盾を導く.
    これは,$A$の元$y\in A$に対して$U_y:=f^{-1}((-\infty,f(y)))$と定めると,$A=\cup_{y\in A}U_y$となる,と翻訳できる.
    任意の$x\in A$に対して,$y$が存在して$x\in U_y$を満たす,ということである.

    こうして$A$の開被覆$(U_y)_{y\in A}$を得たが,$A$はコンパクトであるから,有限個の元$y_1,\cdots,y_n\in A$で$A\subset U_{y_1}\cup\cdots\cup U_{y_n}$を満たすものが存在する.
    すると,このうちで$f(y_i)$を最大にする$y_i$が定まるから,$y\in A\subset U_{y_1}\cup\cdots\cup U_{y_n}=U_{y_i}$となってしまうが,これは$y_i\in f^{-1}(-\infty,f(y_i))=f(y_i)\in(-\infty,f(y_i))$を意味し,矛盾.
\end{proof}

\begin{proposition}
    $X$を距離空間とする.次の4条件は同値である:
    \begin{enumerate}
        \item $X$はコンパクト.
        \item $X$は点列コンパクト.
        \item $X$の離散な閉部分集合は有限である.
        \item $X=\emptyset$または$\forall_{f\in C(X)}$は最大値を持つ.
    \end{enumerate}
\end{proposition}

\subsection{実数のコンパクト部分集合}

\begin{corollary}[距離空間のコンパクト集合]\label{cor-compact-sets-in-metric-space-is-bounded}\mbox{}
    \begin{enumerate}
        \item $X$を距離空間とする.$A$が$X$のコンパクト部分集合ならば,$A$は有界である.
        \item $A$が$\R$の空でないコンパクト集合ならば,$A$は最大元を持つ.
    \end{enumerate}
\end{corollary}
\begin{proof}\mbox{}
    \begin{enumerate}
        \item $A$が空でない場合について示せば良い.勝手に$a\in A$を取り,連続関数$d(-,a):A\to\R$について最大値の定理\ref{prop-maximum-value-theorem}を適用すれば,最大値$M$を得る.三角不等式より,任意の$x,y\in A$について,$d(x,y)\le d(x,a)+d(y,a)\le 2M$.
        \item 恒等写像$A\to A\subset\R$は連続だから,これに最大値の定理\ref{prop-maximum-value-theorem}を適用すれば良い.
    \end{enumerate}
\end{proof}
\begin{theorem}[Heine-Borel / Borel-Lebesgue]\label{thm-Heine-Borel}
    $a\le b$を実数とする.閉区間$[a,b]$は$\R$のコンパクト集合である.
\end{theorem}

\begin{corollary}
    $f:[a,b]\to\R$を連続関数とする.$f$の最大値$M$と最小値$m$が存在し,$f([a,b])=[m,M]$である.
\end{corollary}
\begin{proof}
    最大値の定理より,$f$の最大値$M$が存在する.関数$-f$も最大値$M'$をもち,$m:=-M'$が$f$の最小値である.
    よって,$f([a,b])\subset[m,M]$.一方で,系\ref{cor-monotone-map-over-intervals}より,$[m,M]\subset f([a,b])$でもある.
\end{proof}

\subsection{全有界性による特徴付け}

\begin{tcolorbox}[colframe=ForestGreen, colback=ForestGreen!10!white,breakable,colbacktitle=ForestGreen!40!white,coltitle=black,fonttitle=\bfseries\sffamily,
title=]
    任意の大きさの開球によって有限被覆ができ,それらを十分小さく取ったときに任意の開被覆を細分出来る(Lebesgue数を持つ)ならば,コンパクトである.
\end{tcolorbox}

\begin{definition}[totally bounded, Lebesgue number]\label{def-Lebesgue-number}
    $X$を距離空間とする.
    \begin{enumerate}
        \item $X$が\textbf{全有界}であるとは,任意の実数$r>0$に対して,$X=U_r(a_1)\cup\cdots\cup U_r(a_n)$を満たす有限個の点$a_1,\cdots,a_n\in X$が存在することをいう.全有界ならば有界である.
        \item $(U_i)_{i\in I}$を$X$の開被覆とする.実数$r>0$が次の条件(L)を満たす時,$r$は$(U_i)_{i\in I}$の\textbf{ルベーグ数}であるという.
        \begin{quote}
            (L) 任意の空でない部分集合$A\subset X$に対して,$A$の直径が$r$以下ならば,$A\subset U_i$を満たす$i\in I$が存在する.
        \end{quote}
    \end{enumerate}
\end{definition}

\begin{proposition}[距離空間がコンパクトであることの特徴付け]\label{prop-距離空間がコンパクトであることの特徴付け}
    $X$を距離空間とする.次の2条件は同値である.
    \begin{enumerate}
        \item $X$はコンパクトである.
        \item $X$は全有界であり,$X$の任意の開被覆$(U_i)_{i\in I}$に対し,$(U_i)_{i\in I}$のルベーグ数が存在する.
    \end{enumerate}
\end{proposition}
\begin{proof}\mbox{}
    \begin{description}
        \item[(1)$\Rightarrow$(2)] $X$をコンパクトとする.
        \begin{enumerate}
            \item 任意の$r>0$に対して,$r$-開球$(U_r(x))_{x\in X}$は$X$の開被覆だから,$X=U_r(a_1)\cup\cdots\cup U_r(a_n)$を満たす$a_1,\cdots,a_n\in X$が存在する.
            \item $(U_i)_{i\in I}$を$X$の任意の開被覆とし,そのLebesgue数が存在することを示す.$X=\emptyset$ならば,$r\in\R$は任意に取ればいい.$X\ne\emptyset$とする.
            これに対して,\[B:=\{(x,r)\in X\times(0,\infty)\mid \exists i\in I,\;U_{2r}(x)\subset U_i\}\]
            とすると,$(U_i)_{i\in I}$が開被覆であることから,$(U_r(x))_{(x,r)\in B}$も開被覆である.従って$X$がコンパクトであることより,$(x_1,r_1),\cdots,(x_n,r_n)\in B\;(n\ge 1)$が存在して,$X\subset U_{r_1}(x_1)\cup\cdots\cup U_{r_n}(x_n)$が成り立つ.
            \item $r:=\min(r_1,\cdots,r_n)>0$とすれば,これが開被覆$(U_i)_{i\in I}$のLebesgue数であること,即ち条件(L)を満たすことを示す.
            $A\subset X$を,任意の,直径$r$以下の空でない部分集合とする.勝手な$a\in A$に対して,$a\in U_{r_j}(x_j)$を満たす$j\in[n]$が存在するから,$\exists i\in I,\;A\subset\overline{U_r(a)}\subset U_{2r}(x_j)\subset U_i$.
        \end{enumerate}
        \item[(2)$\Rightarrow$(1)]
        $(U_i)_{i\in I}$を任意の開被覆とし,$r>0$をそのLebesgue数とする.これの有限被覆を構成すれば良い.
        \begin{enumerate}
            \item まず,全有界より,$X=U_{\frac{r}{2}}(a_1)\cup\cdots\cup U_{\frac{r}{2}}(a_n)$を満たす$a_1,\cdots,a_n\in X$が存在する.
            \item 各開球の直径は$r$以下だから,$U_{\frac{r}{2}}(a_1)\subset U_{i_1},\cdots,U_{\frac{r}{2}}(a_n)\subset U_{i_n}$を満たす部分族$i_1,\cdots,i_n\in I$が存在する.
            \item この時,$X=U_{i_1}\cup\cdots\cup U_{i_n}$である.
        \end{enumerate}
    \end{description}
\end{proof}

\subsection{完備距離空間での消息}

\begin{tcolorbox}[colframe=ForestGreen, colback=ForestGreen!10!white,breakable,colbacktitle=ForestGreen!40!white,coltitle=black,fonttitle=\bfseries\sffamily,
title=]
    コンパクト性と全有界性が同値になるのは,距離空間が完備なときに限る.
\end{tcolorbox}

\begin{theorem}
    $X$を完備距離空間,$K\subset X$を閉部分集合とする.このとき,次の3条件は同値.
    \begin{enumerate}
        \item $K$はコンパクト.
        \item $K$の任意の無限部分集合は,$K$内に集積点を持つ.
        \item $K$は全有界である.
    \end{enumerate}
    (3)$\Rightarrow$(1)のみ$X$の完備性が必要で,(1)$\Leftrightarrow$(2)は任意の距離空間で成り立つ.
\end{theorem}

\section{写像の極限}

\begin{tcolorbox}[colframe=ForestGreen, colback=ForestGreen!10!white,breakable,colbacktitle=ForestGreen!40!white,coltitle=black,fonttitle=\bfseries\sffamily,
title=]
    極限とは連続延長可能性である.
    これによって,コンパクト性が特徴付けることができる.
    ネットはその特別で理解しやすい例であり,また距離空間におけるコンパクト性は点列で十分である.
\end{tcolorbox}

\subsection{連続延長による定義}

\begin{definition}[accumulation point]
    $A\subset X$を部分空間,$a\in X$を$A$の集積点とする.写像$f:A\to Y$の延長$\o{f}:A\cup\{a\}\to Y$が$a$で連続であるとき,$f(a)$を\textbf{極限}であるといい,$f(a)=\lim_{x\to a}f(x)$と表す.
\end{definition}

\subsection{極限の一意性}

\begin{tcolorbox}[colframe=ForestGreen, colback=ForestGreen!10!white,breakable,colbacktitle=ForestGreen!40!white,coltitle=black,fonttitle=\bfseries\sffamily,
title=]
    点列だけでなく,一般の写像の極限に関して,Hausdorff空間上では一意になる.
\end{tcolorbox}

\begin{proposition}[Hausdorff空間では極限は一意的]\label{prop-uniqueness-of-limit-in-Hausdorff-space}
    $X,Y$を位相空間とし,$A$を$X$の部分集合,$a\in X\setminus A$を$A$の集積点とする.
    \begin{enumerate}
        \item (収束の特徴付け) 写像$f:A\to Y$と$b\in Y$に対し,次の2条件は同値.
        \begin{enumerate}[(1)]
            \item $x\in A$が$a$に近づく時,$f(x)$は$b$に収束する.
            \item $b$の任意の開近傍$V$に対し,$a\in X$の開近傍$U$で,$f(U\cap A)\subset V$を満たすものが存在する.
        \end{enumerate}
        \item $Y$をハウスドルフ空間とし,$f:A\to Y$を写像とする.$x\in A$が$a$に限りなく近づく時$f(x)$が$b,c\in Y$の両方に収束するならば,$b=c$.
    \end{enumerate}
\end{proposition}
\begin{proof}\mbox{}
    \begin{enumerate}
        \item $f$の延長$\widetilde{f}:\widetilde{A}:=A\sqcup\{a\}\to Y$を$\widetilde{f}(a)=b$で定めると,(1)は,$\widetilde{f}$が$a\in\widetilde{A}$にて連続であることと同値だから,任意の$b$の開近傍$V$に対して,$a\in\widetilde{A}$の開近傍$U$であって
        \[U\subset\widetilde{f}^{-1}(V)\Leftrightarrow \widetilde{f}(U)\subset V\]
        を満たすものが存在するということと同値である.
        これは,写像$f$が$f(U\setminus\{a\})\subset V$であることに同値.
        \item 
        写像を$\widetilde{f}(a)=b$としても,$\widetilde{f}(a)=c$としても$x=a\in\widetilde{A}$にて連続であるということである.
        この下で$b\ne c$と仮定して矛盾を導く.
        $Y$はHausdorffだから,$V\cap W=\emptyset$を満たす$b,c$の開近傍$V,W$が存在する.
        すると,1より,$f(U\cap A)\subset V$かつ$f(U\cap A)\subset W$を満たす$a$の開近傍$U$が存在する.
        このとき,$f(U\cap A)\subset V\cap W=\emptyset$より,$U\cap A=\emptyset$である.
        $U$は$a$の開近傍であったから,この条件を満たす$U$が存在することは,$a$が$A$の集積点であることに矛盾する.
    \end{enumerate}
\end{proof}

\subsection{超フィルターによるコンパクト性の特徴付け}

\begin{notation}
    集合$A$に対して$\widetilde{A}:=A\sqcup\{\infty\}$とし,
    \[ S_A:=\left\{\O\in P(P(\widetilde{A}))\;\middle|\; \begin{array}{l}
        \O は\widetilde{A}の位相であり,\\\infty はAの集積点である
    \end{array}\right\} \]
    と置く.$\infty$が$A$の集積点であるとは,$A$が$\widetilde{A}$で稠密であることと同値.
\end{notation}

\begin{lemma}
    $S_A$は包含関係に関して帰納的半順序集合である.
\end{lemma}

\begin{proposition}[$S_A$を使った位相空間のコンパクト性の判定]\label{prop-characterization-compact-space-in-terms-of-filters}
    位相空間$X$について,次の3条件は同値である.
    \begin{enumerate}
        \item $X$はコンパクト空間である.
        \item $A$を任意の集合とし,$\O$を$S_A$の任意の元とする.$\widetilde{A}$を$\O$によって位相空間と考え,$A$をその部分空間と考える.任意の写像$f:A\to X$に対し,$f$のグラフ$\Gamma_f=\{(a,f(a))\mid a\in A\}$の積位相に関する閉包$\overline{\Gamma_f}\subset\widetilde{A}\times X$と$\{\infty\}\times X$の共通部分は空ではない.
        \item $A$を任意の集合とし,$\O$を$S_A$の任意の極大元とする.$\widetilde{A}$を$\O$によって位相空間と考える.任意の写像$f:A\to X$に対し,$f$の延長である連続写像$g:\widetilde{A}\to X$が存在する.(即ち,$f(a)$の$a\to\infty$の極限が存在する)
    \end{enumerate}
\end{proposition}

\subsection{Tychonoffの定理}

\begin{tcolorbox}[colframe=ForestGreen, colback=ForestGreen!10!white, breakable ,colbacktitle=ForestGreen!40!white, coltitle=black,fonttitle=\bfseries\sffamily
    ,title=Tychonoffの定理]
    Alexandroffのコンパクト化のように,filterの言葉をハードに使う.
    20世紀の香りがする.
    こうして,選択公理とZornの補題と同値な命題を,Topの中にも翻訳できたことになる.
\end{tcolorbox}

\begin{remark}
    Tychonoffの定理(定理\ref{thm-Tychonoff})の証明には選択公理とZornの補題を使うが,選択公理はTychonoffの定理から導ける.この3つは同値な命題である.
\end{remark}

\begin{theorem}[Tychonoff]
    $(X_i)_{i\in I}$をコンパクト空間の族とする.この時,積空間$\prod_{i\in I}X_i$はコンパクトである.
\end{theorem}

\section{点列の収束によるコンパクト距離空間の特徴付け}

\begin{tcolorbox}[colframe=ForestGreen, colback=ForestGreen!10!white,breakable,colbacktitle=ForestGreen!40!white,coltitle=black,fonttitle=\bfseries\sffamily,
title=完備性の発見]
    写像$\N\to X$が収束することは,ネットとして収束すること(任意の開近傍に終局すること)に同値.
    これにより,写像の連続性から,閉包・コンパクト性を調べることができる.
    このとき,距離空間において,完備性という点列に関する性質が見つかる.
\end{tcolorbox}

\subsection{収束の特徴付け}

\begin{definition}
    位相空間$X$の点列$(x_n)$が$a\in X$に収束するとは,$\forall_{U\in\O(a)}\;x_n\in U\;\fe$が成り立つことをいう.
\end{definition}

\begin{proposition}[収束の特徴付け]\label{prop-characterization-of-convergence-in-terms-of-continuousness}
    $X$を位相空間とし,$(x_n)$を$X$の点列,$a\in X$とする.
    \begin{enumerate}
        \item $\widetilde{\N}=\N\coprod\{\infty\}$を離散空間$\N$の一点コンパクト化とする.写像$\tilde{x}:\widetilde{\N}\to X$を\[\tilde{x}(n)=\begin{cases}
            x_n,&n\in\N,\\a,&n=\infty,
        \end{cases}\]と定める.次の2条件は同値.
        \begin{enumerate}[(1)]
            \item 点列$(x_n)$は$a$に収束する.
            \item 写像$\tilde{x}:\widetilde{\N}\to X$は連続.
        \end{enumerate}
        \item $X$が距離空間ならば,次の条件とも同値である.
        \begin{enumerate}[(1)]\setcounter{enumii}{2}
            \item $\lim_{n\to\infty}d(x_n,a)=0$である.
        \end{enumerate}
        \item $X$が距離空間であるとする.$(x_n)$が収束するならば,$(x_n)$は有界である.
    \end{enumerate}
\end{proposition}
\begin{proof}\mbox{}
    \begin{enumerate}
        \item 各$n\in\N\widetilde{\N}$は孤立点だから,$\tilde{x}|_{\N}$は連続.従って,(2)は$\tilde{x}$が$\tilde{x}(\infty)=a$にて連続であることに同値.
        いま,一般に$X$の部分集合$a\in U$に対し,$\infty\in V\subset\tilde{x}^{-1}(U)$を満たす$\infty$の開近傍$V\subset\widetilde{\N}$即ち余有限集合$V$が存在するとは,$\tilde{x}^{-1}(U)\subset\widetilde{\N}$自体も余有限であることに同値.これは(1)の定義である.
        \item 命題\ref{prop-characterization-of-convergence-in-metric-spaces}と同様.
        \item $(x_n)$が収束するならば,これが定める写像$\tilde{x}:\widetilde{\N}\to X$は連続である.よって,像$\tilde{x}(\widetilde{\N})$はコンパクトである.距離空間のコンパクト集合は有界である(系\ref{cor-compact-sets-in-metric-space-is-bounded}).従ってその部分集合$\{x_n\}_{n\in\N}\subset\tilde{x}(\widetilde{\N})$も有界である.
    \end{enumerate}
\end{proof}

\subsection{写像の連続性の特徴付け}

\begin{corollary}[連続性の極限の言葉による特徴付け]\label{cor-characterization-of-continuousness-in-terms-of-limits}
    $X,Y$を位相空間とし,$f:X\to Y$を写像とする.
    \begin{enumerate}
        \item $f$が$a\in X$で連続ならば,$X$の任意の$a$に収束する点列に対して,$(f(x_n))$は$f(a)\in Y$に収束する.
        \item $X$が距離空間であるとき,逆も成り立つ.
    \end{enumerate}
\end{corollary}
\begin{proof}\mbox{}
    \begin{enumerate}
        \item 点列$(x_n)$が収束するので,点列$(x_n)$の定める写像$\tilde{x}:\widetilde{N}\to X$は連続である.従って,合成$f\circ\tilde{x}:\w{\N}\to Y$も連続であるから,特に$\infty$でも連続.よって,点列$(f(x_n))$は$f(a)\in Y$に収束する.
        \item 点列$x:\N\to X$の,$\w{\N}$への延長は,$\N$は$\w{\N}$の稠密な部分集合だから,$X$がHausdorffであるとき一意的である(系\ref{cor-Hausdorff空間への連続写像は,稠密な部分集合への制限で一意的に定まる}).
    \end{enumerate}
\end{proof}

\subsection{閉包の点列による特徴付け}

\begin{corollary}[点列による閉包の特徴付け]\label{cor-characterization-of-closure-in-terms-of-limits}
    $X$を位相空間とし,$A$を$X$の部分集合とする.
    \[ B=\{a\in X\mid aに収束するAの点列(x_n)が存在する\} \]
    と定める.
    \begin{enumerate}
        \item $\overline{A}\supset B$.
        \item (AC) $X$が距離空間ならば,$\overline{A}=B$.
    \end{enumerate}
\end{corollary}
\begin{proof}\mbox{}
    \begin{enumerate}
        \item $A$の点列$(x_n)$が存在して$a\in X$に収束するならば,$a\in\overline{A}$であることを示せば良い.
        収束列$(x_n)$が定める連続写像$\tilde{x}:\w{\N}\to X$は,命題\ref{prop-continuous-map-and-closure}(3)より,$\tilde{x}(\w{\N})\subset\o{A}$を満たすので,$\x(\infty)=a\in\o{A}$を満たす.
        \item 
        $\o{A}\subset B$を示す.$a\in\o{A}$を任意にとり,これに収束する$X$の点列を構成すれば良い.
        いま,$a\in\o{A}$より,$a$の任意の近傍は$A$と共通部分を持つ(定義\ref{def-closure}).
        特に,任意の$n\ge 1$に対して,$U_{\frac{1}{n}}(a)\cap A\ne\emptyset$である.
        従って,選択公理を認めれば,点列$(x_n)\in\prod_{n\ge 1}(U_{\frac{1}{n}}(a)\cap A)$が取れる.
        これについて,$\lim_{n\to\infty}d(x,a)=0$より,$a=\lim_{n\to\infty}x_n\in B$である(定義\ref{def-convergence}).
    \end{enumerate}
\end{proof}

\subsection{コンパクト距離空間の点列による特徴付け}

\begin{tcolorbox}[colframe=ForestGreen, colback=ForestGreen!10!white,breakable,colbacktitle=ForestGreen!40!white,coltitle=black,fonttitle=\bfseries\sffamily,
title=]
    点列コンパクトと同値であることを示すには,完備性の概念が効いてくる.
\end{tcolorbox}

\begin{lemma}[部分列が収束することの特徴付け]\label{lemma-部分列が収束することの特徴付け}
    $X$を距離空間とする.$X$の点列$(x_n)$と$X$の点$a$について,次の3条件は同値である.
    \begin{enumerate}
        \item $(x_n)$の部分列$(x_{n_m})$で,$a$に収束するものが存在する.
        \item $A=\{(x_n,n)\mid n\in\N\}\subset X\times\widetilde{\N}$とし,$\overline{A}$を積位相に関する閉包とすると,$(a,\infty)\in\overline{A}$である.
        \item 任意の自然数$m$に対し,$\inf_{n\ge m}d(x_n,a)=0$である.
    \end{enumerate}
\end{lemma}
\begin{proof}\mbox{}
    \begin{description}
        \item[(1)$\Rightarrow$(2)] 
        部分列$(x_{m_n})$が$a$に収束するならば,$\w{\N}$の点列である$(m_n)_{n\in\N}$は$\infty$に収束するから,
        $A$の点列$(x_{m_n},m_n)$は$(a,\infty)\in X\times\w{\N}$に収束する.
        連続写像$\w{\N}\to X$と$\w{\N}\to\w{\N}$との積も連続であるためである.
        よって,系\ref{cor-characterization-of-closure-in-terms-of-limits}より,$(a,\infty)\in\o{A}$である.
        \item[(2)$\Rightarrow$(3)]
        $(a,\infty)\in\o{A}$とする.任意の自然数$m$に対して,$\inf_{n\ge m}d(x_n,a)=0$を示す.
        任意の実数$r>0$に対して,$(0\le)\inf_{n\ge m}d(x_n,a)<r$を導く.
        いま,任意の実数$r>0$に対して,それぞれは$a\in X,\infty\in\w{\N}$の開近傍なので,$U_r(a)\times\w{\N}_{\ge m}$は$(a,\infty)\in X\times\w{\N}$の開近傍である(積位相の基底\ref{prop-characterization-product-topology}).
        よって,$(a,\infty)\in\o{A}$は閉包の点なので,$A\cap(U_r(a)\times\w{\N}_{\ge m})=\{(x_n,n)\mid d(x_n,a)<r,n\ge m\}$は空でない.
        従って,$d(x_n,a)<r$を満たす自然数$n\ge m$が存在して,$\inf_{n\ge m}d(x_n,a)<r$を満たす.
        \item[(3)$\Rightarrow$(1)]
        次のように帰納的に構成した部分列$(x_{m_n})$は,$\lim_{n\to\infty}d(x_{m_n},a)=0$より,$a$に収束する.
        \begin{enumerate}
            \item $m_0:=0$.
            \item $n\ge 1$について,$\inf_{m\ge m_{n-1}+1}d(x_m,a)=0$が成り立つから,特に$d(x_m,a)<\frac{1}{n}$を満たす自然数$m>m_{n-1}$が存在する.そのような$m$のうち最小のものを$m_n:=m$とする.
        \end{enumerate}
    \end{description}
\end{proof}

\begin{proposition}[コンパクト距離空間は点列コンパクトである]\mbox{}\label{prop-Bolzano-Weierstrass}
    \begin{enumerate}
        \item $X$がコンパクト距離空間ならば,$X$は点列コンパクトである.
        \item (Bolzano-Weierstrass) $(x_n)$を$\R^m$の有界な点列とすると,収束する部分列$(x_{m_n})$が存在する.
    \end{enumerate}
\end{proposition}
\begin{proof}\mbox{}
    \begin{enumerate}
        \item 任意に$X$の点列$(x_n)$をとり,この収束する部分列が存在することを示せば良い.
        $A:=\{(x_n,n)\mid n\in\N\}\subset X\times\w{\N}$の閉包を$\o{A}$とし,$\exists a\in X,\;(a,\infty)\in\o{A}$を示せれば,補題\ref{lemma-部分列が収束することの特徴付け}より収束する部分列が存在する.
        いま,$X$はcompactだから,特徴付け\ref{prop-characterization-of-compactness-in-terms-of-projection}より,射影$\pr_2;X\times\w{\N}\to\w{\N}$は閉写像である.
        よって,$\pr_2(\o{A})\subset\w{\N}=\o{\pr_2(A)}$(命題\ref{prop-continuous-map-and-closure})は,$\N\subset\pr_2(A)\subset\pr_2(\o{A})$を満たす閉集合だから,$\pr_2(\o{A})=\w{\N}$である.即ち,$\exists a\in X,\;(a,\infty)\in\o{A}$.
        \item 
        $(x_n)$が$\R^m$の有界な点列ならば,実数$M>0$であって,$\{x_n\}\subset[-M,M]^m$を満たすものが存在する.命題\ref{prop-compact-sets-in-R^n}より,有界閉集合$[-M,M]^m$はコンパクトだから,1.より,$(x_n)$には収束する部分列が存在する.
    \end{enumerate}
\end{proof}

\begin{theorem}[AC]\label{thm-characterization-of-compact-metric-space}
    距離空間$X$について,次の3条件は同値である.
    \begin{enumerate}
        \item $X$はコンパクト.
        \item $X$は点列コンパクト.
        \item $X$は完備かつ全有界.
    \end{enumerate}
\end{theorem}
\begin{proof}\mbox{}
    \begin{description}
        \item[(1)$\Rightarrow$(2)] 
        命題\ref{prop-Bolzano-Weierstrass}.1.
        \item[(2)$\Rightarrow$(3),AC] \mbox{}\\
        \begin{description}
            \item[完備性について] 
            Cauchy列$(x_n)$を任意に取る.$X$は点列コンパクトなので,収束する部分列$(x_{m_n})$が存在する.補題\ref{lemma-Cauchy-sequence}より,Cauchy列$(x_n)$は部分列の収束先$\lim_{n\to\infty}x_{m_n}$に収束する.
            \item[全有界性について] 
            対偶命題:$X$が全有界でないならば,点列コンパクトでないことを示す.即ち,実数$r>0$であって,任意の有限部分集合$A\subset X$に対して,$X\supsetneq\cup_{a\in A}U_r(a)$が成り立つものが存在すると仮定する.
            これに対して次のように$X$の点列$(x_n)$を帰納的に定める:
            \begin{enumerate}
                \item $x_0\in X\ne\emptyset$を任意に取る.
                \item $x_0,\cdots,x_n\in X$に対して,$x_{n+1}\in X\setminus(U_r(x_0)\cup\cdots\cup U_r(x_n))\ne\emptyset$と取る.
            \end{enumerate}
        すると,この点列は$d(x_m,x_n)\ge r\;(0\le m<n)$を満たすので,収束する部分列を持たない.
        \end{description}
        \item[(3)$\Rightarrow$(1),AC] $X$を全有界かつ完備とし,ここからコンパクト性を示す.
        \begin{description}
            \item[コンパクト空間$A$の構成] 
            $X$は全有界で$\frac{1}{2^n}>0\;(n\in\N)$だから,任意の$n\in\N$に対して,$\forall x\in X\;\exists a\in A_n,\; d(x,a)<\frac{1}{2^n}$を満たす離散有限集合$A_n$が存在する($X=\cup_{a\in A_n}U_{\frac{1}{2^n}}(a)$を満たす$A_n$が存在するため).
            選択公理より,これらからなる,$X$の離散有限集合の列$(A_n)$が取れる.
            これらの無限積空間$A:=\prod_{n\in\N}A_n$はTychonoffの定理\ref{thm-Tychonoff}より,compactである.
            \item[コンパクト部分空間$C$の構成]
            $A$の部分空間$C$を
            \[C:=\left\{(a_n)\in A\;\middle|\;\forall n\in\N,\;d(a_n,a_{n+1})\le\frac{3}{2^{n+1}}\right\}\]
            で定める.するとこの$C$は
            \[C=\bigcap_{n\in\N}\left\{(a_m)\in A\;\middle|\;\forall n\in\N,\;d(a_n,a_{n+1})\le\frac{3}{2^{n+1}}\right\}=\bigcap_{n\in\N}(\pr_n\times\pr_{n+1})^{-1}\paren{\left[0,\frac{3}{2^{n+1}}\right]}\]
            をみたし,任意の$n\in\N$について射影の積$\pr_n\times\pr_{n+1}:A\to A_n\times A_{n+1}$は連続だから,$C$はコンパクト空間$A$の閉部分空間である.
            従って$C$もコンパクトである(系\ref{cor-union-intersection-of-compact-sets-is-compact}).
            \item[全射$l:C\to X$の構成]
            $(a_n)\in C$は
            \begin{align*}
                \forall n\le m\qquad d(a_n,a_m)&\le d(a_n,a_{n+1})+\cdots+d(a_{m-1},a_m)\\
                &\le \frac{3}{2^{n+1}}+\cdots+\frac{3}{2^m}\le\frac{3}{2^n}
            \end{align*}
            より,$X$のCauchy列である.$X$は完備だから,極限$\lim_{n\to\infty}a_n\in X$が存在し,写像
            \[\xymatrix@R-2pc{
                l:C\ar[r]&X\\
                \rotatebox[origin=c]{90}{$\in$}&\rotatebox[origin=c]{90}{$\in$}\\
                (a_n)\ar@{|->}[r]&\lim_{n\to\infty}a_n
            }\]
            が定まる.この$l$が全射であることを示す.

            任意の$x\in X$に対して,$l^{-1}(x)$の元を構成する.
            点列$a=(a_n)\in A$であって,$d(a_n,x)<\frac{1}{2^n}\;(n\in\N)$を満たすものが存在するように,空間$A$を定義したのであった.選択公理より,このような$a\in A$が取れる.
            ここでこの点列は,
            \[\forall n\in\N,\quad d(a_n,a_{n+1})\le d(a_n,x)+d(a_{n+1},x)<\frac{1}{2^n}+\frac{1}{2^{n+1}}=\frac{3}{2^{n+1}}\]
            を満たすから,$a\in C$でもある.この点列は$\lim_{n\to\infty}a_n=x$より,$l(a)=x$を満たす.
            \item[$l$は連続である]
            任意に点$a=(a_n)\in C$を取り,任意の$r>0$について,逆像$l^{-1}(U_r(l(a)))$に含まれる$a\in C$の開近傍の基本系の元(即ち基底)を構成すれば良い(命題\ref{prop-continuousness-in-terms-of-basis}).\footnote{$l$の終域の基底は開球で,始域の基底は}

            任意の$n\in\N$について$d(a_n,l(a))\le\frac{3}{2^n}$である.
            よって任意の実数$r>0$に対して,$m$を$\frac{6}{2^n}<r$を満たす最小の自然数$n$とすれば,
            $a=(a_n),b=(b_n)\in C$が$a_m=b_m$をみたすならば,$d(l(a),l(b))\le d(a_m,l(a))+d(b_m,l(b))\le\frac{6}{2^m}<r$である.
            従って,$\pr^{-1}_m(\pr_m(a))\subset l^{-1}(U_r(l(a)))$である.
            よって,全射$l:C\to X$は連続である.\footnote{???}
            $C$はコンパクトだから,\ref{cor-image-of-compact-set-is-compact}より,$X$もコンパクトである.
        \end{description}
    \end{description}
\end{proof}

\section{完備距離空間}

\subsection{閉集合との関係}

\begin{proposition}[距離空間内の閉集合]\label{prop-complete-sets-in-metric-space}
    $X$を距離空間,$A$を部分空間とする.次の2条件を考える.
    \begin{enumerate}
        \item $A$は完備.
        \item $A$は$X$の閉集合.
    \end{enumerate}
    1$\Rightarrow$2である.$X$が完備ならば逆も成り立つ.
\end{proposition}
\begin{proof}\mbox{}
    \begin{description}
        \item[(1)$\Rightarrow$(2),AC] $A\supset\o{A}$を示せば良い.任意に$a\in\o{A}$を取る.系\ref{cor-characterization-of-closure-in-terms-of-limits}.2(AC)より,
        $a$に収束する$A$の点列$(x_n)$が存在する.収束する列$(x_n)$はCauchy列で(補題\ref{lemma-Cauchy-sequence}.1),$A$は完備だから,$(x_n)$は収束する.系\ref{cor-characterization-of-closure-in-terms-of-limits}.2より,それは$a$で,よって$a\in A$.
        \item[(2)$\Rightarrow$(1)]
        $(x_n)$を$A$のCauchy列とし,これが収束することを示せば良い.$X$が完備ならば,極限を持つ:$\lim_{n\to\infty}x_n=:a\in X$.系\ref{cor-characterization-of-closure-in-terms-of-limits}より,$a\in\o{A}=A$.従って,$A$は完備.
    \end{description}
\end{proof}

\subsection{完備化}

\begin{definition}[completion]
    $X$を距離空間,$Y$を完備距離空間とする.
    $f:X\to Y$が等長写像であり,像$f(X)$が$Y$で稠密である時,$Y$は$f$に関して$X$の\textbf{完備化}であるという.
\end{definition}

\begin{proposition}
    $X$を距離空間とする.$x\in X$に対し,$d(x,-):X\to\R$を$y\mapsto d(x,y)$と定める.
    \[\xymatrix@R-2pc{
        D_X:X\ar[r]&\Map(X,\R)\\
        \rotatebox[origin=c]{90}{$\in$}&\rotatebox[origin=c]{90}{$\in$}\\
        x\ar@{|->}[r]&d(x,-)
    }\]
    $a\in X$とする.
    \begin{enumerate}
        \item $x\in X$ならば,$D_X(x)-D_X(a)\in B(X)$である.$x\in X$を$D_X(x)-D_X(a)\in B(X)$に写す写像$i:X\to B(X)$は等長写像である.
        \item 等長写像$i:X\to B(X)$の像の閉包$\overline{i(X)}$が$X$の完備化である.
        \item (一意性) $Y$を距離空間,$f:X\to Y$を等長写像とし,$Y$は$f$に関して$X$の完備化であるとする.$y\in Y$に対し,
        \[\xymatrix@R-2pc{
            \widetilde{D}_X(y):X\ar[r]&\R&\tilde{i}:Y\ar[r]&B(X)\\
            \rotatebox[origin=c]{90}{$\in$}&\rotatebox[origin=c]{90}{$\in$}&\rotatebox[origin=c]{90}{$\in$}&\rotatebox[origin=c]{90}{$\in$}\\
            x\ar@{|->}[r]&d(y,f(x))&y\ar@{|->}[r]&\widetilde{D}_X(y)-D_X(a)
        }\]
        と定めると,写像$\tilde{i}$は可逆な等長写像$Y\to\overline{i(X)}$を定め,$\tilde{i}\circ f=i$である.
        \[\xymatrix{
            X\ar[r]^-f\ar[d]_-i&Y\ar[dl]^-{\o{i}}\\
            B(X)
        }\]
    \end{enumerate}
\end{proposition}
\begin{proof}\mbox{}
    \begin{enumerate}
        \item \begin{enumerate}[(a)]
            \item 命題\ref{prop-characterization-of-closure-in-terms-of-metric-function}より,
            \[\abs{D_X(x)(y)-D_X(x)(z)}=\abs{d(x,y)-d(x,z)}\le d(y,z)\]
            より,$y\to x$の極限を考えると,$D_X(x)\in C(X)$である.
            \item 再び命題\ref{prop-characterization-of-closure-in-terms-of-metric-function}より,
            \[\forall z\in X,\;\abs{D_X(x)(z)-D_X(y)(z)}=\abs{d(x,z)-d(y,z)}\le d(x,y)\]
            より,さらに踏み込んで$D_X(x)-D_X(y)\in B(X)$である.
            \item 
            いま,
            \[\|D_X(x)-D_X(y)\|_\infty=\sup_{z\in X}\abs{d(x,z)-d(y,z)}\le d(x,y)\]
            であるが,$z=y$とすると,$\abs{d(x,y)-d(y,y)}=d(x,y)$なので,$d(i(x),i(y))=\|D_X(x)-D_X(y)\|_\infty=d(x,y)$より,$i$は等長写像である.
        \end{enumerate}
        \item 
        $\o{i(X)}\subset B(X)$は完備距離空間$B(X)$の部分閉集合だから,命題\ref{prop-complete-sets-in-metric-space}より完備である.
        $i$は等長写像で,$i(X)$は$\o{i(X)}$上で稠密に決まっているので,$\o{i(X)}$は$X$の完備化である.
        \item 

    \end{enumerate}
\end{proof}

\subsection{不動点定理}

\begin{definition}[contracting mapping]
    $X$を距離空間とし,$f:X\to X$を写像とする.
    Lipschitz定数$r\in(0,1)$を持つLipschitz連続関数$f$を\textbf{縮小写像}という:$\exists_{r\in(0,1)}\forall_{x,y\in X}\;d(f(x),f(y))\le r\cdot d(x,y)$.
\end{definition}

\begin{proposition}[完備距離空間の縮小写像には不動点が存在する]
    $X$を距離空間とし,$f:X\to X$を縮小写像とする.
    \begin{enumerate}
        \item $f(a)=a$を満たす$a\in X$は,存在すればただ一つである.
        \item $X$が完備であり空でなければ,$f(a)=a$を満たす$a\in X$が存在する.
    \end{enumerate}
\end{proposition}
\begin{proof}
    $f$は縮小写像だから,条件を満たす$r\in(0,1)$が存在する.
    \begin{enumerate}
        \item $a,b\in X$が$f(a)=a,f(b)=b$を満たすとする.$d(a,b)\le rd(a,b)$より,$r\in(0,1)$だから,$d(a,b)=0\Leftrightarrow a=b$.
        \item 任意に$x\in X\ne\emptyset$を取り,$X$の点列をこれから$(x_n:=f^n(x))$と定める.
        すると,
        \begin{align*}
            \forall m\in\N,\;\forall n\ge m,\quad d(x_n,x_m)&\le \sum^{n-1}_{i=m}d(x_{i+1},x_i)\\
            &\le \sum^{m-1}_{i=m}r^id(x_1,x_0)\\
            &=\frac{r^m-r^n}{1-r}d(x_1,x_0)\\
            &\le \frac{r^m}{1-r}d(x_1,x_0)\xrightarrow{m\to\infty}0.
        \end{align*}
        より,これはCauchy列である.$X$は完備だから,極限$\lim_{n\to\infty}x_n=:a\in X$が存在する.これについて,$f(a)=\lim_{n\to\infty}f(x_n)=\lim_{x\to\infty}x_{n+1}=a$である.
    \end{enumerate}
\end{proof}

\subsection{可算共通部分}

\begin{proposition}[AC, 完備距離異空間の稠密開集合の可算共通部分は稠密]
    $X$を完備距離空間とする.$(U_n)_{n\in\N}$が$X$の稠密な開集合の列ならば,共通部分$\cap_{n=0}^\infty U_n$も稠密である.
\end{proposition}

\begin{corollary}[Baire, AC]
    $X$を空でない完備距離空間とし,$(A_n)_{n\in\N}$を$X$の閉集合の列とする.$X=\cup_{n=0}^\infty A_n$ならば,内部$A^\circ_n$が空でない$n$が存在する.
\end{corollary}

\subsection{Baire空間}

\begin{tcolorbox}[colframe=ForestGreen, colback=ForestGreen!10!white,breakable,colbacktitle=ForestGreen!40!white,coltitle=black,fonttitle=\bfseries\sffamily,
    title=]
        3つのBanach空間上の作用素の基本結果は,全てBaireの範疇定理の上に拠って立つ.
        これは,完備距離空間の稠密開集合の可算交叉は再び稠密である(くらいに「濃い」)ことを主張している.
\end{tcolorbox}

\begin{definition}[Baire space, nowhere dense]\mbox{}
    \begin{enumerate}
        \item 閉包が内点を持たない集合を\textbf{疎集合}という.集合が疎であることと,その補集合が稠密であることは同値.
        \item 可算個の疎集合の合併として表せる集合を\textbf{第一類}という.
        \item そうでない集合,すなわち,任意の稠密開集合の可算共通部分は稠密であるような位相空間を\textbf{第二類}または\textbf{Baire空間}という.
    \end{enumerate}
\end{definition}
\begin{proof}
    任意の半径$r>0$の閉球$B_0$を取り,これと$\cap_{n=1}^\infty A_n$との共通部分が空でないことを示せば,$\cap_{n=1}^\infty A_n$の稠密性が示せる.

    いま,$A_1\cap B_0^\circ$は空でない開集合だから,ある半径$2^{-1}r$より小さい閉球$B_1$が取れる.
    これを繰り返すことで,$B_n\subset A_n\cap B^\circ_{n-1},r(B_n)<2^{-n}r$を満たす閉球の列$(B_n)$が取れる.
    $X$は完備だから,$\exists_{x\in X}\;\{x\}=\cap_{n=1}^\infty B_n\subset B_0\cap\paren{\cap_{n=1}^\infty A_n}$が成り立ち,共通部分が空でないことがわかった.
\end{proof}

\begin{proposition}[Baire category theorem 1]
    任意の完備距離空間$X$はBaire空間である.
    
    すなわち,$(A_n)$を$X$の稠密開集合の列とすると,この共通部分$\cap_{n\in\N}A_n$は$X$で稠密である.
    また双対命題は,閉集合列を用いて$X=\cup_{n=1}^\infty F_n$と表せたとき,少なくとも一つの$F_n$は疎でない(内点を持つ).
\end{proposition}
\begin{remark}
    実はZFの下で従属選択公理と呼ばれる弱い選択公理と同値になる.
\end{remark}

\begin{proposition}[BCT2]
    任意の局所コンパクトハウスドルフ空間はBaire空間である.
\end{proposition}
\begin{remarks}
    こちらは函数解析学では使わないが,任意の有限次元多様体がBaire空間であることがわかる.多様体がパラコンパクトでない場合でも成り立つ.
    なお,局所コンパクトでない完備距離空間も,距離化可能でない局所コンパクトハウスドルフ空間も存在することに注意.
\end{remarks}

\section{距離化可能性}

\begin{tcolorbox}[colframe=ForestGreen, colback=ForestGreen!10!white,breakable,colbacktitle=ForestGreen!40!white,coltitle=black,fonttitle=\bfseries\sffamily,
title=]
    一般の距離空間が距離化可能であるための条件は,可算性で捉えられる.
\end{tcolorbox}

\subsection{可算性・可分性}

\begin{definition}[first-countable, second-countable, separable, $\sigma$-compact]
    位相空間$X$について,
    \begin{enumerate}
        \item 任意の点$x\in X$の近傍系$\O(x)$が可算な基本形を持つとき,\textbf{第1可算}であるという.
        \item 開集合系$\O_X$が可算な基底$\U$を持つとき,$X$は\textbf{第2可算}であるという.これは可算な準基を持てば十分.
        \item 稠密な可算部分集合が存在する時,\textbf{可分}であるという.
        \item $X$は局所コンパクトであるとする.
        $X$のコンパクト集合の列$(A_n)_{n\in\N}$で,$X=\cup_{n\in\N}A_n$を満たすものが存在する時,$X$は\textbf{$\sigma$-コンパクト}であるという.
    \end{enumerate}
\end{definition}

\begin{proposition}
    位相空間$X$について,
    \begin{enumerate}
        \item 第2可算ならば第1可算である.
        \item 第2可算ならば可分である.
    \end{enumerate}
\end{proposition}

\subsection{距離空間の可算・可分性}

\begin{tcolorbox}[colframe=ForestGreen, colback=ForestGreen!10!white,breakable,colbacktitle=ForestGreen!40!white,coltitle=black,fonttitle=\bfseries\sffamily,
title=]
    距離空間では可分性の概念が十分強く,第2可算性と同値になる.
\end{tcolorbox}

\begin{proposition}[距離空間では全有界なら可分]\label{prop-sep-metric-space}
    距離空間$X$について,
    \begin{enumerate}
        \item $X$は第1可算である.
        \item 稠密な部分集合$A$に対して,次のようにして開集合の基底が構成できる:$\U_A=\{U_{1/n}(a)\mid a\in A,n\in\N,n>0\}$は$X$の開集合系の基底である.
        \item (AC) $X$が全有界ならば,$X$は可分である.
    \end{enumerate}
\end{proposition}
\begin{proof}\mbox{}
    \begin{enumerate}
        \item 命題\ref{prop-open-balls-is-basis}と実数のArchimedes性より,任意の$x\in X$に対し,$\{U_{\frac{1}{n}}(a)\}_{n\in\N_{>0}}$は開近傍の基本系であり,可算である.
        \item $X$の開集合$U$とその点$x\in U$を任意に取り,この間に挟まる$\U_A$の元を構成すれば良い.
        開球は基底だから,$U_r(x)\subset U$を満たす実数$r>0$が存在し,実数のArchimedes性より,$n\ge\frac{2}{r}$を満たす$n\in\N$も存在する.$A$は稠密だから,この$n$に対しても$U_{1/n}(x)\cap A\ne\emptyset$で,$a\in U_{1/n}(x)\cap A$が取れる.
        $\frac{1}{n}+d(x,a)<\frac{2}{n}\le r$が成り立つから,$x\in U_{1/n}(a)\subset U_r(x)\subset U$が成り立つ.
        \item 
        任意の自然数$n\ge 1$に対して,$\forall x\in X,\;\exists a\in A_n,\;d(x,a)\le 1/n$を満たす有限集合$A_n\subset X$が存在するから,選択公理より列$(A_n)$を得る.$A=\cup_{n\in\N}A_n$とするとこれは可算集合であり,$X$上稠密である.
    \end{enumerate}
\end{proof}

\begin{corollary}\label{cor-separability-and-second-countability}
    $X$を距離空間とすると,次の2条件は同値.
    \begin{enumerate}
        \item $X$は可分である.
        \item $X$は第2可算である.
    \end{enumerate}
\end{corollary}
\begin{proof}\mbox{}
    \begin{description}
        \item[(1)$\Rightarrow$(2)] $X$の稠密な可算部分集合を$A$とする.命題\ref{prop-sep-metric-space}の通りに$\U_A$を定めると,これは可算である.
        \item[(2)$\Rightarrow$(1),AC] 命題\ref{prop-sep-general-space}.3より.
    \end{description}
\end{proof}

\subsection{Frechet空間$\R^\infty$の距離付け可能性}

\begin{tcolorbox}[colframe=ForestGreen, colback=ForestGreen!10!white,breakable,colbacktitle=ForestGreen!40!white,coltitle=black,fonttitle=\bfseries\sffamily,
title=]
    $[0,1]^\N$に通常の積位相を入れたものは,コンパクトハウスドルフ空間になる(いずれも任意の積空間に伝播する性質である).
    これを\textbf{Tychonoff cube}といい,実は距離化可能でもある.
\end{tcolorbox}

\begin{proposition}[モデル空間での例]\mbox{}
    \begin{enumerate}
        \item 積空間$\R^\N$は距離付け可能である.
        \item 積空間$\R^\N$は可分であり,第2可算である.
    \end{enumerate}
\end{proposition}
\begin{proof}\mbox{}
    \begin{enumerate}
        \item \begin{description}
            \item[証明方針] $\R$は空でない開区間と同相だから,積空間$A:=\prod_{n\in\N}\paren{0,\frac{1}{n}}$が距離付け可能であることを示せば良い.
            また$l^\infty\subset\subset\R^\N$は距離空間だから(例\ref{exp-uniform-norm}),包含写像\[\xymatrix@R-2pc{
                i:A\ar[r]&l^\infty\\
                \rotatebox[origin=c]{90}{$\in$}&\rotatebox[origin=c]{90}{$\in$}\\
                (a_n)\ar@{|->}[r]&a:\N\to\R
            }\]
            が埋め込みであること:$i^*(\O_{l^\infty})=\O_A$,即ち$l^\infty$空間の部分空間であることを示せば良い.
            \item[証明] \begin{align*}
                \forall r>0,\;\forall a\in A,\quad i^{-1}(U_r(a))&=U_r(a)\cap A\\
                &=\prod_{n\le 1/r}U_r(a_n)\times\prod_{n>1/r}\paren{0,\frac{1}{n}}
            \end{align*}
            は族の積位相の特徴付けより開集合系の基底である.基底と基底が通じ合っているため,$i^*$が任意の合併を保つことより(命題\ref{prop-functoriality-of-image-and-inverse-image-mappings}),
            $i^*(\O_{l^\infty})=\O_A$である.
        \end{description}
        \item 
        $\R^\N$の可算部分集合
        \[\Q^{(\N)}:=\left\{x=(x_n)\in\R^\N\;\middle|\;x_n\ne 0を満たすnは高々有限個\right\}\]
        が稠密であるため.系\ref{cor-separability-and-second-countability}より.
    \end{enumerate}
\end{proof}

\subsection{一般の距離付け可能性}

\begin{tcolorbox}[colframe=ForestGreen, colback=ForestGreen!10!white,breakable,colbacktitle=ForestGreen!40!white,coltitle=black,fonttitle=\bfseries\sffamily,
title=]
    第2可算な正規空間が距離付け可能で,可分距離空間を定める.これは埋め込み$X\to\R^\infty$が構成できることによる.
    さらに,コンパクト距離空間は$[0,1]^\infty$の閉部分空間に埋め込める.
\end{tcolorbox}

\begin{theorem}[Urysohnの距離付け定理]
    位相空間$X$について,次の3条件は同値である.
    \begin{enumerate}
        \item $X$は可分かつ距離付け可能.
        \item $X$は第2可算かつ正規.
        \item 埋め込み$X\to\R^\N$が存在する.
    \end{enumerate}
\end{theorem}

\begin{corollary}[AC, コンパクト距離空間は閉区間の可算積の閉部分空間と同相である]
    位相空間$X$について,次の3条件は同値である.
    \begin{enumerate}
        \item $X$はコンパクトかつ距離付け可能である.
        \item $X$は第2可算かつコンパクト・ハウスドルフである.
        \item 閉埋め込み$X\to[0,1]^\N$が存在する.
    \end{enumerate}
\end{corollary}

\chapter{位相的構造}

\section{Hausdorff空間}

\subsection{定義}

\begin{definition}
    位相空間$X$が,$\forall_{x,y\in X}\;x\ne y\Rightarrow\exists_{U\in\O(x),V\in\O(y)}\;U\cap V=\emptyset$を満たすとき,\textbf{Hausdorff}または\textbf{分離空間}または\textbf{$T_2$-空間}であるという.
\end{definition}

\subsection{積位相による特徴付け}

\begin{proposition}[Hausdorff空間の特徴付け:対角集合が閉集合となる空間]\label{prop-characterization-Hausdorff}
    $X$を位相空間とする.
    \begin{enumerate}
        \item 次の2条件は同値である.
        \begin{enumerate}[(1)]
            \item $X$はHausdorffである.
            \item 対角集合$\Delta_X$が積空間$X\times X$の閉集合である.
        \end{enumerate}
        \item $X$がHausdorffならば,任意の$x\in X$について$\{x\}$は$X$の閉集合である.
    \end{enumerate}
\end{proposition}
\begin{proof}\mbox{}
    \begin{enumerate}
        \item 次の条件は全て同値である.
        \begin{description}
            \item[(2)] $\Delta_X$が閉集合である.
            \item[(1.75)] $(X\times X)\setminus\Delta_X$が開集合である.
            \item[(1.5)] 任意の点$(x,y)\in(X\times X)\setminus\Delta_X$に対して,$x$の開近傍$U$と$y$の開近傍$V$が存在して,$((x,y)\in )U\times V\subset (X\times X)\setminus\Delta_X$を満たすものが存在する(積位相の特徴付け\ref{prop-characterization-product-topology}より).
            \item[(1.25)] 任意の点$(x,y)\in(X\times X)\setminus\Delta_X$に対して,$x$の開近傍$U$と$y$の開近傍$V$が存在して,$U\times V\cap\Delta_X=\emptyset$を満たすものが存在する.
            \item[(1)] 任意の点$(x,y)\in(X\times X)\setminus\Delta_X$に対して,$x$の開近傍$U$と$y$の開近傍$V$が存在して,$U\cap V=\emptyset$を満たすものが存在する.
        \end{description}
        \item 任意の元$x\in X$に対して,写像$i_x:X\to X\times X$を$i_x(y):=(x,y)$と定めると,これは連続である.\footnote{例えば,$\pr_2:X\times X\to X$が開写像であるため(積位相の特徴付け\ref{prop-characterization-product-topology}).}$\{x\}$はこれによる閉集合$\Delta_X$の逆像$\{x\}=i^{-1}_x(\Delta_X)$である.
    \end{enumerate}
\end{proof}

\subsection{連続単射の逆像への遺伝}

\begin{tcolorbox}[colframe=ForestGreen, colback=ForestGreen!10!white,breakable,colbacktitle=ForestGreen!40!white,coltitle=black,fonttitle=\bfseries\sffamily,
title=]
    部分空間にHausdorff性は引き継がれる.
\end{tcolorbox}

\begin{proposition}\label{prop-Hausdorff性は連続単射によって足に感染る}
    $f:X\to Y$を連続単射とする.$Y$がHausdorffならば,$X$もHausdorffである.
\end{proposition}
\begin{proof}\mbox{}
    \begin{enumerate}
        \item $f$が連続ならば,$f\times f$は連続である(積位相の特徴付け\ref{prop-characterization-product-topology}.2).
        \item $f$が単射ならば,$(f\times f)^{-1}(\Delta_Y)=\Delta_X$である.
        \item 連続写像$f\times f$による逆像$(f\times f)^{-1}(\Delta_Y)=\Delta_X$は閉集合である.
    \end{enumerate}
\end{proof}

\begin{corollary}\label{cor-subspace-of-Hausdorff-is-Hausdorff}
    $X$をHausdorff空間とする.$X$の部分空間$A$はHausdorffである.
\end{corollary}
\begin{proof}
    包含写像$i:A\to X$は連続単射である.
\end{proof}

\begin{proposition}
    $X$は$\F$により始位相が定まっており,$\F$は$X$の点を分離するとする.
    各$Y_f\;(f\in\F)$がHausdorffならば,$X$もHausdorffである.
\end{proposition}

\begin{corollary}[直積へのHausdorff性の伝播]\mbox{}
    \begin{enumerate}
        \item $(X_i)_{i\in I}$をHausdorff空間の族とすると,積空間$\prod_{i\in I}X_i$もHausdorffである.
        \item $X$と$Y$を位相空間とし,$X\times Y$がHausdorffであるとする.$Y$が空でなければ$X$はHausdorffである.
    \end{enumerate}
\end{corollary}
\begin{proof}\mbox{}
    \begin{enumerate}
        \item $\Delta_X=\cap_{i\in I}(\pr_i\times\pr_i)^{-1}(\Delta_{X_i})$.仮定より,$\Delta_{X_i}$は閉集合である.$\pr_i$は$\prod_{i\in I}X_i$の積位相について連続写像であるから,$\pr_i\times\pr_i$も連続である(積位相の特徴付け\ref{prop-characterization-product-topology}).
        \item 各$b\in Y$に対して,写像$f:X\to X\times Y$を$f(x):=(x,b)$と定めると,これは連続で単射である.よって,命題\ref{prop-Hausdorff性は連続単射によって足に感染る}より.
    \end{enumerate}
\end{proof}

\subsection{等化子による特徴付け}

\begin{tcolorbox}[colframe=ForestGreen, colback=ForestGreen!10!white,breakable,colbacktitle=ForestGreen!40!white,coltitle=black,fonttitle=\bfseries\sffamily,
title=]
    非常に閉グラフ定理に繋がる見方である.
\end{tcolorbox}

\begin{proposition}[Hausdorff性の特徴付け:等化子]\label{prop-characterization-of-Hausdorff-in-terms-of-equalizer}
    位相空間$X$について,次の3条件は同値である.
    \begin{enumerate}
        \item $X$はHausdorffである.
        \item 任意の位相空間$T$と任意の連続写像$f,g:T\to X$に対して,$\Ker(f,g)=\{t\in T\mid f(t)=g(t)\}$は$T$の閉集合である.
        \item 任意の位相空間$T$と任意の連続写像$f:T\to X$に対し,$f$のグラフ$\Gamma=\{(t,x)\in T\times X\mid f(t)=x\}$は$T\times X$の閉集合である.
    \end{enumerate}
\end{proposition}
\begin{proof}\mbox{}
    \begin{description}
        \item[(1)$\Rightarrow$(2)] 
        $\Ker(f,g)=(f,g)^{-1}(\Delta_X)$である.$f,g$が連続であることと$(f,g)$が連続であることは同値(積位相の特徴付け\ref{prop-characterization-product-topology}).よって,$\Delta_X$が閉集合ならば,$\Ker(f,g)$は閉集合である.
        \item[(2)$\Rightarrow$(3)] 
        $\Gamma=\Ker(f\circ\pr_1,\pr_2)$である.
        \item[(3)$\Rightarrow$(1)] 
        対角集合$\Delta_X$とは,$\id_X:X\to X$のグラフである.
    \end{description}
\end{proof}

\subsection{Hausdorff空間値写像は連続延長する}

\begin{corollary}[Hausdorff空間への連続写像は,稠密な部分集合への制限で一意的に定まる]\label{cor-Hausdorff空間への連続写像は,稠密な部分集合への制限で一意的に定まる}
    $X$を位相空間とし,$Y$をHausdorff空間とする.
    $A$が$X$の稠密な部分集合ならば,制限写像$i^*:C(X,Y)\to C(A,Y)$は単射である.
\end{corollary}
\begin{proof}
    任意の$f,g\in C(X,Y)$について,$f|_A=g|_A\Rightarrow f=g$を示す.
    仮定は即ち$A\subset\Ker(f,g)$と同値であるが,$Y$がHausdorffならば,$\Ker(f,g)$は閉集合である.よって,$\Ker(f,g)=X$.
    即ち,$f=g$.
\end{proof}

\subsection{ネットによる特徴付け}

\begin{proposition}
    位相空間$X$について,次の2条件は同値.
    \begin{enumerate}
        \item $X$はHausdorffである.
        \item $X$の任意のネットは高々1つの収束点を持つ.
    \end{enumerate}
\end{proposition}

\section{連結空間}

\begin{tcolorbox}[colframe=ForestGreen, colback=ForestGreen!10!white,breakable,colbacktitle=ForestGreen!40!white,coltitle=black,fonttitle=\bfseries\sffamily,
title=]
    Hausdorff-Lenneによる概念である.
\end{tcolorbox}

\subsection{定義と特徴付け}

\begin{definition}[connected]\label{def-連結性}
    $X$を位相空間とし,$\emptyset\subsetneq A\subset X$について,
    \begin{enumerate}
        \item $A$が\textbf{連結}であるとは,$X$の開集合$U,V$であって$A\subset U\cup V,A\cap U\cap V=\varnothing, A\cap U\ne\varnothing, A\cap V\ne\varnothing$を満たすものは存在しないことをいう.すなわち,部分空間としての$A$が2つの開集合の非交和として表せないことをいう.
        \item $A$が\textbf{弧状連結}であるとは,任意の2点$x,y\in A$に対して,これらを結ぶ$[0,1]$からの連続写像$f:[0,1]\to A, f(0)=x, f(1)=y$が存在することをいう.
    \end{enumerate}
\end{definition}

\begin{proposition}[連結性の特徴付けとintermediate value theorem]\label{prop-charactorization-of-connectedness}
    空でない位相空間$X$に対し,次の5条件は同値である.
    \begin{enumerate}
        \item $X$は連結である.
        \item \textbf{中間値の定理}:$\forall_{f\in C(X)}\;\forall_{u,v\in X}\;\forall_{c\in\R}\;f(u)\le c\le f(v)\Rightarrow[\exists t\in X\;(c=f(t))]$.
        \item $f:X\to 2$が離散位相空間$2=\{0,1\}$への連続写像ならば,$f$は定数関数である.
        \item $U$が$X$の開集合であり閉集合でもあるならば,$U=X\lor U=\varnothing$.
        \item $p:X\to 1$を定値写像とすると,任意の離散位相空間$Y$に対して,$p^*:C(1,Y)\to C(X,Y)$が可逆である.
    \end{enumerate}
\end{proposition}
\begin{proof}
    \begin{description}
        \item[(1)$\Rightarrow$(2)] 
        $f\in C(X,\R)$を任意の連続関数とする.任意の$f(u)<c<f(v)$を満たす$u,v,c$について$f(t)=c$を満たす$c\in X$を構成すれば良い.
        いまこの$c\in\R$について,
        \begin{align*}
            U&=\{x\in X\mid f(x)<c\},&V&=\{x\in X\mid f(x)>c\},
        \end{align*}
        と定めるといずれも開区間の連続写像による逆像であるから,開集合である.
        また,$U\cap V=\emptyset$で,少なくとも$u\in U,v\in V$であるからいずれも空でない.
        $X$が連続ならば,$X\subset U\cup V$も満たす開集合$U,V$は存在しないから,$X\not\subset U\cup V$即ち$X\setminus(U\cup V)\ne\emptyset$である.
        この任意の元$t\in X\setminus(U\cup V)$を取ると,$f(t)\in\R\setminus((-\infty,c)\cup(c,\infty))$なので,
        $c=f(t)$である.
        \item[(2)$\Rightarrow$(3)]
        $f\in C(X,2)$を連続写像とする.$2$を離散位相空間と見ているので,$\R$への関数$f:X\to\R$と考えたものも連続である(離散位相空間$2$は$\R$の部分空間とみなせるので,$i:2\to\R$とすると連続で,$f'=i\circ f$と考えれば良い).
        従って,(2)より,$f(X)=2$ならば,$f(t)=1/2$を満たす$t\in X$が存在するはずだが,これは矛盾.従って,$f(X)\subsetneq 2$即ち$f$は定数関数である.
        \item[(3)$\Rightarrow$(4)]
        $U\subset X$は開かつ閉とする.すると,特性関数$\chi_U:X\to 2$は,$2$の離散位相に対しても連続になる(通常はSierpiński位相についてのみ連続).
        従って,$U=\emptyset\lor X$.
        \item[(4)$\Rightarrow$(1)]
        $X$の開集合$U$が最初の2条件$X\subset U\cup V,X\cap U\cap V=\emptyset$を満たすとは,無縁和$X=U\coprod V$を意味する.すると,$U,V$はいずれも開かつ閉であり,(4)より,$U=\emptyset\lor V=\emptyset$.
        よって,条件を満たす分割はなく,$X$は連結である.
        \item[(5)$\Leftrightarrow$(2)]
        任意の離散空間$Y$への連続写像$f\in C(X,Y)$は,$y\in Y$が存在して定値写像$f=y$に限る,ということである.これは,埋め込み$i:Y\to\R$が取れる場合については(2)$\Rightarrow$(3)の時と同様に示せる.
    \end{description}
\end{proof}

\subsection{閉包と交叉和への遺伝}

\begin{tcolorbox}[colframe=ForestGreen, colback=ForestGreen!10!white,breakable,colbacktitle=ForestGreen!40!white,coltitle=black,fonttitle=\bfseries\sffamily,
title=]
    連結な稠密部分集合を持つならば,連結である.
    Hausdorff性とは違って,部分集合への遺伝は難しい.
\end{tcolorbox}

\begin{corollary}[連結性の伝播]\label{cor-connectedness-over-union}
    $X$を位相空間とする.$A,B\subset X$とする.
    \begin{enumerate}
        \item $A$が連結とする.このとき,$B$が$A\subset B\subset\overline{A}$を満たすならば,$B$も連結である.特に,$\overline{A}$も連結である.
        \item $A,B$が連結とする.$A\cap B\ne\varnothing$ならば,$A\cup B$も連結である.
    \end{enumerate}
\end{corollary}
\begin{proof}\mbox{}
    \begin{enumerate}
        \item 任意に$f\in C(B,2)$を取り,これが定数関数であることを示す.
        勝手に$a\in A$を取って$c:=f(a)$と置くと,$f|_A=c$である.
        離散位相空間$2$はHausdorff($0,1$が近傍によって分離可能)で,$A$は$B$上稠密だから,系\ref{cor-Hausdorff空間への連続写像は,稠密な部分集合への制限で一意的に定まる}より,
        $f=c$を得る.
        \item 任意に$f\in C(A\cup B,2)$を取り,これが定数関数であることを示す.
        $A\cap B\ne\emptyset$なので,勝手に$a\in A\cap B$を取って$c:=f(a)$と置くと,$f|_A=f|_B=c$である.よって,$f=c$を得る.
    \end{enumerate}
\end{proof}

\subsection{弧状連結性}

\begin{corollary}[弧状連結ならば連結]
    位相空間$X$が弧状連結ならば,$X$は連結である.
\end{corollary}
\begin{proof}
    任意に連続関数$f\in C(X,2)$を取り,これが定数関数であることを示す.
    $a\in X$を勝手に取る.
    $X$は弧状連結だから,任意の$x\in X$に対して,連続写像$g:[0,1]\to X$であって$g(0)=a,g(1)=x$を満たすものが存在する.
    すると,$f\circ g:[0,1]\to 2$は連続であるが,$[0,1]$は連結であるから(命題\ref{prop-closed-interval-is-connected}),
    命題\ref{prop-charactorization-of-connectedness}より,定値写像$f\circ g=f(a)$である.よって,$f(x)=f(a)$.
    以上より,$f=f(a)$である.
\end{proof}

\begin{definition}[path component]
    $\PConn_x(X):=\Brace{y\in X\mid y,x\text{は道で結べる}}$を\textbf{弧状連結成分}という.
    この二項関係は$X$上に同値関係を定める.弧状連結成分の全体を$\pi_0(X)$で表す.
\end{definition}

\begin{lemma}
    弧状連結成分の全体は,連結成分の全体よりも細かい.
\end{lemma}

\begin{remark}
    位相幾何学者の正弦曲線の,$y=\sin(1/x)$上の点の弧状連結成分は,$\Brace{(x,y)\in\R^2\mid y=\sin(1/x),0<x\le 1}$となり,開集合である.
    連結成分はただ1つである.
\end{remark}

\subsection{実数の連結性}

\begin{tcolorbox}[colframe=ForestGreen, colback=ForestGreen!10!white,breakable,colbacktitle=ForestGreen!40!white,coltitle=black,fonttitle=\bfseries\sffamily,
title=]
    実数の連結部分集合を区間という.
\end{tcolorbox}

\begin{proposition}\label{prop-closed-interval-is-connected}
    $a<b$を実数とする.閉区間$[a,b]$は連結である.
\end{proposition}
\begin{proof}
    連続関数$f\in C([a,b],2)$を任意に取り,これが定数関数$f(a)$となることを背理法で示す.
    $f(c)\ne f(a)$となる$c\in[a,b]$が存在したと仮定する.

    いま,$A:=f^{-1}(f(a))\cap[a,c]$とすると,$\{f(a)\}$が閉より,$A$は少なくとも$a\in A$を持つ閉集合となるから,$s:=\sup A\in A$が存在する.$s\in A$は命題\ref{prop-characterization-of-supremum}による.
    これについて,$f(s)=f(a)\ne f(c)$より,$s<c$.よって,$(s,c]\cap A=\emptyset$より,勝手に取った$t\in(s,c]$に対し,$f(t)=f(c)$($f$は二値写像であることに注意).
    しかし$f$は連続だから,$[s,c]$上で$f=f(c)$.よって,$f(s)$の値について矛盾.
    従って,$f$は定数関数である.
\end{proof}

\begin{corollary}[区間上の単調写像は開埋め込み]\label{cor-monotone-map-over-intervals}
    $a<b$を実数とする.
    \begin{enumerate}
        \item $f:[a,b]\to\R$を連続関数とする.$f(a)\le f(b)$ならば,$[f(a),f(b)]\subset f([a,b])$である.
        \item $f:(a,b)\to\R$を連続関数とする.任意の$a<s<t<b$に対し,$f(s)<f(t)$ならば,$f:(a,b)\to\R$は開埋め込みである.
    \end{enumerate}
\end{corollary}
\begin{proof}\mbox{}
    \begin{enumerate}
        \item 閉区間$[a,b]$は連結であるから,$f(a)<c<f(b)$を満たす$c\in[f(a),f(b)]$について,$t\in[a,b]$が存在して$c=f(t)\in f([a,b])$である(命題\ref{prop-charactorization-of-connectedness}).
        \item 連続写像$f$はこのとき単射だから,あとは開写像であることを示せば良い.
        $U\subset(a,b)$を開集合としたとき,$f(U)$が$\R$の開集合であることを示す.
        任意に$y\in f(U)$を取ると,$\exists x\in U,\; f(x)=y$である.
        よって,$x\in(s,t)\subset U$を満たす実数$a<s<x<t<b$が存在する($\epsilon$-近傍などを取れば良い,$\epsilon$-近傍は$(a,b)$の位相の基底であるため).
        $f$は単射だから$f((s,t))=f([s,t])\setminus\{f(s),f(t)\}$で,単調性と1.より,$[f(s),f(t)]=f([s,t])$から,右辺は$(f(s),f(t))$とわかる.
        こうして,$f(y)\in(f(s),f(t))\subset U$と開近傍が取れたことになる.
        以上より,$f(U)$は開集合である.
    \end{enumerate}
\end{proof}

\begin{proposition}[$\R$の連結集合の分類]
    $\R$の部分集合$A$に対し,次の3条件は同値である.
    \begin{enumerate}
        \item $A$は連結である.
        \item $A$は弧状連結である.
        \item 次の条件のどれか1つが成り立つ.
        \begin{enumerate}[(i)]
            \item $A=\R$である.
            \item $a\in R$であって,$A$が$[a,\infty),(a,\infty),(-\infty,a),(-\infty,a]$のどれかと等しくなるようなものが存在する.
            \item 実数$a<b$であって,$A$が$(a,b),(a,b],[a,b),[a,b]$のどれかと等しくなるものが存在する.
            \item $a\in R$であって,$A=\{a\}$となるものが存在する.
        \end{enumerate}
    \end{enumerate}
\end{proposition}

\subsection{連結性の連続像への伝播}

\begin{proposition}[連続写像は連結性を保存する]\label{prop-connectedness-over-morphism}
    $f:X\to Y$を連続写像とする.$A$が$X$の連結部分集合ならば,$f(A)$は$Y$の連結部分集合である.
\end{proposition}
\begin{proof}
    $A$が連結ならば$A\ne\emptyset$としたから,$f(A)\ne\emptyset$より,$g\in C(f(A),2)$が取れる.これが定数関数であることを示す.
    すると,$g\circ f|_A\in C(A,2)$であるから,$g\circ f|_A$は定数関数.
    $f|_A$は全射だから$g$も定数関数と判る.
    \[\xymatrix{
        X\ar[r]^-f\ar@{.>}[dr]&Y\ar[d]^-g\\
        &2
    }\]
\end{proof}

\begin{corollary}[積への伝播]
    $X,Y$を位相空間とする.次の2条件は同値である.
    \begin{enumerate}
        \item $X,Y$はそれぞれ連結である.
        \item $X\times Y$は連結である.
    \end{enumerate}
\end{corollary}
\begin{proof}\mbox{}
    \begin{description}
        \item[(1)$\Rightarrow$(2)] 
        任意に取った$(a,b)\in X\times Y$に対して,任意の$f\in C(X\times Y,2)$が$f=f(a,b)$を満たす定数関数だと示せば良い.
        任意の$(x,y)\in X\times Y$に対して,$X\simeq X\times\{y\},Y\simeq\{a\}\times Y$なので,命題\ref{prop-connectedness-over-morphism}より,これらも連結.
        よって,$f|_{X\times\{y\}},f|_{\{a\}\times Y}$は定数関数.よって,$f(x,y)=f(a,y)=f(a,b)$.$(x,y)\in X\times Y$は任意に定めたから,$f=f(a,b)$.
        \item[(2)$\Rightarrow$(1)]
        連続写像$\pr_1:X\times Y\to X,\pr_2:X\times Y\to Y$の像である(命題\ref{prop-connectedness-over-morphism}).
    \end{description}
\end{proof}

\begin{proposition}[AC]
    一般に,連結空間の族$(X_i)$に対して,積空間$\prod_{i\in I}X_i$も連結である.
\end{proposition}

\subsection{連結成分}

\begin{proposition}[連結成分は閉集合]\label{prop-connected-component}
    $X$を位相空間とする.
    \begin{enumerate}
        \item $x,y\in X$に対し,$x$と$y$を元として含む連結な部分集合$A\subset X$が存在するという条件は,$X$上の同値関係$R$を定める.
        \item $A$をこの同値関係$R$に関する同値類の1つ(すなわち連結成分)とすると,$A$は連結な閉集合である.
    \end{enumerate}
\end{proposition}
\begin{proof}\mbox{}
    \begin{enumerate}
        \item 各$x\in X$に対して$\{x\}$は連結だから,反射律は成り立つ.対象律は明らか.連結部分集合$A,B$が存在して$x,y\in A,y,z\in B$の時,$A\cap B\ne\emptyset$より,$x,y,z\in A\cup B$も連結(系\ref{cor-connectedness-over-union})で,推移律も成り立つ.
        \item $a\in A$を任意に取る.$f\in C(A,2)$を任意に取り,$c:=f(a)$を置く.
        任意の$x\in A$に対して,$\{x,a\}\subset B$を満たす連結部分集合$B\subset A$が存在するが,これについて$f|_B=c$.よって,$f(x)=c$を得るが,$x\in A$は任意に取ったから$f=c$.
        よって,$A$は連結.系\ref{cor-connectedness-over-union}より$\overline{A}$も連結であるが,$A$の極大性より$\overline{A}\subset A$.よって,$A$は閉.
    \end{enumerate}
\end{proof}

\subsection{連結成分と開かつ閉集合}

\begin{tcolorbox}[colframe=ForestGreen, colback=ForestGreen!10!white,breakable,colbacktitle=ForestGreen!40!white,coltitle=black,fonttitle=\bfseries\sffamily,
title=]
    連結成分は閉集合だから,連結成分が高々有限個であるとき,連結成分は開かつ閉である.
    連結成分が無限個であるときもこの性質が成り立つためには,空間に局所連結性を課せば良い.
\end{tcolorbox}

\begin{definition}[quasi-component]
    同値関係$x\sim_qy:\Leftrightarrow \forall_{f\in C(X;2)}\;f(x)=f(y)$による同値類を,\textbf{擬成分}という.
    これは,$x$を含む全ての開かつ閉集合の共通部分として特徴付けられる.
\end{definition}

\begin{lemma}[極大性:quasi-component]
    $\Gamma_x$を$x$の連結成分,$\Gamma'_x$を$x$の擬成分含むとする.
    \begin{enumerate}
        \item $\Gamma_x\subset\Gamma'_x$である.
        \item $X$がcompact Hausdorffまたは局所連結ならば,$\Gamma'_x\subset\Gamma_x$でもある.
    \end{enumerate}
\end{lemma}
\begin{proof}\mbox{}
    \begin{enumerate}
        \item 任意の$X$のclopen set $U$について,$U\subsetneq\Gamma_x$ならば,$\Gamma_x=(U\cap\Gamma_x)\coprod(\Gamma_x\setminus U)$と2つの開集合で分割できてしまい,矛盾.
    \end{enumerate}
\end{proof}

\section{局所連結空間}

\begin{tcolorbox}[colframe=ForestGreen, colback=ForestGreen!10!white,breakable,colbacktitle=ForestGreen!40!white,coltitle=black,fonttitle=\bfseries\sffamily,
title=]
    一般に,任意の位相空間$X$が,その連結成分$(U_i)$の直和$\coprod_{i\in I}U_i$に等しいとは限らない.
    任意の開部分空間がこの性質を満たす空間を,\textbf{局所連結}であるという.
\end{tcolorbox}

\subsection{定義と特徴付け}

\begin{definition}
    位相空間$X$が\textbf{局所連結}であるとは,任意の点$x\in X$の(開)近傍系$\O(x)$が,連結な開集合のみからなる基本系を持つことをいう.
\end{definition}
\begin{remark}
    連結空間は局所連結とは限らない!
    位相幾何学者の正弦曲線は,連結であるが局所連結ではない.
\end{remark}

\begin{example}
    任意の連結成分が一点集合であるとき,\textbf{完全不連結空間}という.
    完全不連結空間が局所連結であるとは,連結集合が一点集合しかないのだからこれが基底になるしかない,すなわち,離散であることは同値である.$\Q$は離散ではなく,局所連結ではない,完全不連結空間である.
\end{example}

\begin{proposition}
    $X$を位相空間とする.次の3条件は同値:
    \begin{enumerate}
        \item $X$は局所連結である.
        \item 任意の$X$の開部分空間$U$の任意の連結成分は$U$-開である.
        \item 任意の開部分空間は,その連結成分の直和空間に等しい.
    \end{enumerate}
    特に,局所連結空間の任意の連結成分は開かつ閉である.
\end{proposition}

\subsection{伝播}

\begin{proposition}
    局所連結空間の商空間は再び局所連結である.
\end{proposition}



\section{一般のコンパクト性}

\subsection{積位相による特徴付け}

\begin{proposition}[コンパクト性の積位相の言葉による特徴付け]\label{prop-characterization-of-compactness-in-terms-of-product-topology}
    $X$を位相空間とし,$A$を$X$の部分集合とする.次の3条件は同値である.
    \begin{enumerate}
        \item $A$はコンパクトである.
        \item $Y$を任意の位相空間とし,$y\in Y$を任意の点とする.$A\times\{y\}$の任意の開近傍$W\subset X\times Y$に対し,$A$の開近傍$U\subset X$と$y$の開近傍$V\subset Y$で,$U\times V\subset W$を満たすものが存在する.
        \item $Y$を任意の位相空間とし,$y\in Y$を任意の点とする.$A\times\{y\}$の任意の開近傍$W\subset X\times Y$に対し,$y$の開近傍$V\subset Y$で,$A\times V\subset W$を満たすものが存在する.
    \end{enumerate}
\end{proposition}
\begin{proof}\mbox{}
    \begin{description}
        \item[(1)$\Rightarrow$(2)] 
        \begin{enumerate}
        \item 
        $W\subset X\times Y$を$A\times\{y\}$の任意の開近傍として,$U\times V\subset W$を満たす$A$の開近傍$U$と$y$の開近傍$V$を構成すれば良い.
        積位相の特徴付けより,任意の$x\in A$に対して,$(x,y)\in U_x\times V_x\subset W$を満たす$x$の開近傍$U_x$と$y$の開近傍$V_y$が存在するから,これらの族$(U_x,V_x)_{x\in A}$を取ると,
        $A\times\{y\}\subset\cup_{x\in A}(U_x\times V_x)\subset W$を満たす.が,$\cup_{x\in A}(U_x\times V_x)\subset (\cup_{x\in A}U_x)\times(\cup_{x\in A}V_x)$であって,等号は成り立たないので,
        まだ$U,V$は構成できない.\textbf{ここで$A$がコンパクトであることが効いてくる}のである.
        \item
        いま,族の定め方より$A\subset\cup_{x\in A}U_x$で,$A$はコンパクトであるために,族$(U_x,V_x)_{x\in A}$のうち$(U_1,V_1),\cdots,(U_n,V_n)$が存在して,$A\subset\cup_{i\in[n]}U_i$を満たす.
        この時,$U:=\cup_{i\in[n]}U_i,V:=\cap_{i\in[n]}V_i$とすれば,いずれも開集合で,
        \[A\times\{y\}\subset U\times V=\cup_{i\in[n]}U_i\times V\subset\cup_{i\in[n]}(U_i\times V_i)\subset W\]
        を満たす.
        \end{enumerate}
        \item[(2)$\Rightarrow$(3)]
        $U\times V\subset W$を満たす$A$の開近傍$U$が存在するならば,$A\times V\subset U\times V\subset W$である.
        \item[(3)$\Rightarrow$(1)] $(U_i)_{i\in I}$を$X$の開集合の族で,$A\subset\cup_{i\in I}U_i$を満たすものとする.このうちの有限部分被覆を構成する算譜を与えれば良い.
        \begin{description}
            \item[場面設定して(3)を使う] 
            位相空間$Y$を積空間$Y:=\S^I$とし,点$y:=(1)_{i\in I}\in Y$を定数関数$1:I\to 2$とする.
            族$(V_i)_{i\in I}$を逆像$V_i:=\pr_i^{-1}(1)$と定めるとこれは開集合の族で,
            $W:=\cup_{i\in I}(U_i\times V_i)$とすると$X\times Y$の開集合の族が定まる.
            これについて,
            \begin{align*}
                (X\times\{y\})\cap W&=\cup_{i\in I}(U_i\times(\{y\}\cap V_i))&(\forall i\in I,\;X\cap U_i=U_i)\\
                &=\cup_{i\in I}U_i\times\{y\}&(\forall i\in I,\;\{y\}\cap V_i=\{y\})
            \end{align*}
            が成り立つから,
            \[A\times\{y\}\subset\cup_{i\in I}U_i\times\{y\}=(X\times\{y\})\cap W\subset W\]
            がわかり,$W$は$A\times\{y\}$の開近傍である.よって,(3)より,$A\times V\subset W$を満たす開近傍$1\in V\subset Y$が存在する.
            \item[構成]
            $I$の有限部分集合$J$に対して,$V_J:=\{1\}^J\times\S^{I\setminus J}$と書くこととする.
            すると,族の積位相の特徴付け(命題\ref{prop-universality-of-product-space})より,
            開集合$V$に対して,$y\in V_J\subset V$となる$J$が存在する.
            \textbf{ここで,空間$Y$の決め方が効いてくる}.
            $\chi_J\in V_J$が成り立つ.これより,
            \[A\times\{\chi_J\}\subset A\times V_J\subset A\times V\subset W\]
            である.これと$(X\times\{\chi_J\})\cap W=\cup_{i\in J}U_i\times\{\chi_J\}$と併せて,$A\subset\cup_{i\in J}U_i$を得る.
        \end{description}
    \end{description}
\end{proof}

\subsection{閉集合との交叉と有限合併への遺伝}

\begin{corollary}[有限個の合併,閉集合との共通部分はコンパクトである]\label{cor-union-intersection-of-compact-sets-is-compact}
    $X$を位相空間とする.
    \begin{enumerate}
        \item $A\subset X$をコンパクト集合とする.$B$が$X$の閉集合ならば,$A\cap B$もコンパクト集合である.特に$X$がコンパクト空間ならば,$X$の閉集合$B=X\cap B$はコンパクト集合である.
        \item $A_1,\cdots,A_n$が$X$の有限個のコンパクト集合ならば,合併$A_1\cup\cdots\cup A_n$もコンパクト集合である.
    \end{enumerate}
\end{corollary}
\begin{proof}
    任意の位相空間$Y$とその点$y\in Y$を用意する.
    \begin{enumerate}
        \item $(A\cap B)\times\{y\}$の開近傍$W\subset X\times Y$を取り,$(A\cap B)\times\{y\}\subset (A\cap B)\times V\subset W$を満たす$y$の開近傍$V$が構成できることを示せば良い.
        いま$W\cup((X\setminus B)\times Y)$を考えれば,$B$は$X$-閉集合だから,これは$A\times\{y\}$を含む開集合,即ち開近傍で,$A$はコンパクトだから,$y$の開近傍$V\subset Y$が存在して,$A\times\{y\}\subset A\times V\subset W\cup((X\setminus B)\times Y)$を満たす.これについて,
        $(A\cap B)\times\{y\}\subset(A\cap B)\times V\subset W$である.
        \item 
        $(A_1\cup\cdots\cup A_n)\times\{y\}$の開近傍$W\subset X\times Y$を取り,
        \[(A_1\cup\cdots\cup A_n)\times\{y\}\subset (A_1\cup\cdots\cup A_n)\times V\subset W\]
        を満たす$y$の開近傍$V$を構成すれば良い.
        各$A_i\;(i\in[n])$はコンパクトで,$W$は$A_i\times\{y\}$の開近傍でもあるから,$A_i\times\{y\}\subset A_i\times V_i\subset W$を満たす$y$の開近傍の族$(V_i)_{i\in[n]}$が見つかる.これに対して,$V:=\cap_{i\in[n]}V_i$と定めれば良い.
        実際,これは再び$y$の開近傍で,
        \[(A_1\cup\cdots\cup A_n)\times\{y\}\subset (A_1\cup\cdots\cup A_n)\times V\subset\cup_{i\in[n]}(A_i\times V_i)\subset W\]
        を満たす.
    \end{enumerate}
\end{proof}

\subsection{連続像への伝播}

\begin{corollary}[連続写像はコンパクト性を保存する]\label{cor-image-of-compact-set-is-compact}
    $f:X\to Y$を連続写像とする.$A\subset X$をコンパクト集合とすると,$f(A)\subset Y$もコンパクト集合である.
\end{corollary}
\begin{proof}
    任意の位相空間$Z$とその点$z\in Z$を用意する.
    $f(A)\times\{z\}$の開近傍$W\subset Y\times Z$を任意に取り,$f(A)\times\{z\}\subset f(A)\times V\subset W$を満たす$z$の開近傍$V\subset Z$を構成すれば良い.

    $f\times 1:X\times Z\to Y\times Z$は連続写像より,$W':=(f\times 1)^{-1}(W)\subset X\times Z$は$A\times\{z\}$の開近傍である.$A$はコンパクトであるから,
    $A\times\{z\}\subset A\times V\subset W'$を満たす$z$の開近傍$V\subset Z$が存在する.これについて,
    \begin{align*}
        (f\times 1)(A\times\{z\})\subset(f\times 1)(A\times V)\subset(f\times 1)(W')\\
        \Leftrightarrow\quad&f(A)\times\{z\}\subset f(A)\times V\subset W
    \end{align*}
    である.
\end{proof}

\subsection{射影による特徴付け}

\begin{proposition}[コンパクト性の射影の言葉による特徴付け]\label{prop-characterization-of-compactness-in-terms-of-projection}
    位相空間$X$について,次の2条件は同値.
    \begin{enumerate}
        \item $X$はコンパクトである.
        \item 任意の位相空間$Y$に対して,$\pr_2:X\times Y\to Y$は閉写像である.
    \end{enumerate}
\end{proposition}
\begin{proof}\mbox{}
    \begin{description}
        \item[(1)を積位相の言葉に翻訳] 
        $Y$を任意の位相空間,$y\in Y$を任意の点とすると,$X$がコンパクトであることは,任意の$X\times\{y\}$の開近傍$W\subset X\times Y$に対して,$X\times\{y\}\subset X\times V\subset W$を満たす$y$の開近傍$V\subset Y$が存在することに同値.
        \item[さらに閉集合の言葉に翻訳]
        $F:=(X\times Y)\setminus W$と置いて,$X\times Y$の開集合$W$を考える代わりに,閉集合$F$を考える.
        すると,$X\times\{y\}=\pr_2^{-1}(y)$より,射影の言葉を使うと,
        \begin{enumerate}
            \item $X\times\{y\}\subset W\Leftrightarrow\pr_2^{-1}(y)\cap F=\emptyset\Leftrightarrow y\in Y\setminus\pr_2(F)$.
            \item $X\times V\subset W\Leftrightarrow\pr_2^{-1}(V)\cap F=\emptyset\Leftrightarrow V\subset Y\setminus\pr_2(F)$.
        \end{enumerate}
        と書き換えられる.以上をまとめると,(1)は
        \begin{quote}
            (3') 任意の位相空間$Y$に対して,$X\times Y$の任意の閉集合$F$と任意の点$y\in Y\setminus\pr_2(F)$に対し,$y$の開近傍$V\subset Y$で,$V\subset Y\setminus\pr_2(F)$を満たすものが存在する.
        \end{quote}
        と同値.
        \item[さらに点$y\in Y$を抽象化する]
        (3')の任意にとった$Y,F$に対する条件は,$\pr_2(F)$が$Y$の閉集合であるということと同値.よって(2)と同値.
    \end{description}
\end{proof}

\subsection{積への遺伝}

\begin{tcolorbox}[colframe=ForestGreen, colback=ForestGreen!10!white,breakable,colbacktitle=ForestGreen!40!white,coltitle=black,fonttitle=\bfseries\sffamily,
title=]
    Kelleyは,Tychonoffの定理が選択公理と同値であることを示した.
\end{tcolorbox}

\begin{corollary}[積もコンパクト]\label{cor-product-of-compact-sets-is-compact}
    $X,Y$を位相空間とする.$A\subset X,B\subset Y$がコンパクト集合ならば,$A\times B\subset X\times Y$もコンパクト集合である.
\end{corollary}
\begin{proof}
    命題\ref{prop-characterization-of-compactness-as-space}より,$A=X,B=Y$の場合について示せば十分である.
    $Z$を任意の位相空間とする.
    $X$がコンパクトという仮定より,射影$\pr_{23}:X\times(Y\times Z)\to Y\times Z$は閉写像で,$Y$がコンパクトより,射影$\pr_2:Y\times Z\to Z$は閉写像である.
    合成$\pr_3=\pr_2\circ\pr_{23}$も閉写像であるから,$X\times Y$はコンパクトである.
\end{proof}

\begin{theorem}[Tychonoff (AC)]\label{thm-Tychonoff}
    $(X_i)_{i\in I}$をコンパクト空間の族とする.この時,積空間$X:=\prod_{i\in I}X_i$はコンパクトである.
\end{theorem}
\begin{proof}\mbox{}
    \begin{enumerate}[(a)]
        \item $(x_\lambda)$を$X$の普遍ネットとする.
        \item 普遍ネットの任意の写像による像は普遍だから,$(\pr_i(x_\lambda))$は$X_i$の普遍ネットである.$X_i$のコンパクト性より,収束点$x_i$を持つ.
        \item $x\in X$であって$\pr_i(x)=x_i$を満たす点について,$x_\lambda\to x$が成り立つ.
    \end{enumerate}
\end{proof}

\subsection{ネットによる特徴付け}

\begin{theorem}[(AC)]
    位相空間$X$について,次の5条件は同値.
    \begin{enumerate}
        \item $X$はコンパクトである.
        \item $X$の任意の閉集合系$\Delta$が,任意の有限部分集合の共通部分が非空ならば,$\Delta$全体の共通部分は非空である.
        \item $X$の任意のネットは集積点を持つ.
        \item $X$の任意の普遍ネットは収束する.
        \item $X$の任意のネットは収束する部分ネットを持つ.
    \end{enumerate}
\end{theorem}

\section{コンパクトハウスドルフ空間}

\begin{tcolorbox}[colframe=ForestGreen, colback=ForestGreen!10!white,breakable,colbacktitle=ForestGreen!40!white,coltitle=black,fonttitle=\bfseries\sffamily,
title=]
    コンパクトハウスドルフ空間の位相は,これ以上弱くなるとハウスドルフではなくなり,これ以上強くなるとコンパクトではなくなる.
\end{tcolorbox}

\subsection{連続写像の標準分解}

\begin{tcolorbox}[colframe=ForestGreen, colback=ForestGreen!10!white, breakable ,colbacktitle=ForestGreen!40!white, coltitle=black,fonttitle=\bfseries\sffamily
    ,title=コンパクト空間とハウスドルフ空間との間のHom集合]
    コンパクト空間とハウスドルフ空間との間の射は,閉写像で,かつproperである.また,従って,全単射であることと同相であることが一致する.
\end{tcolorbox}

\begin{proposition}[閉写像補題]\label{prop-continuous-map-from-compact-to-Hausdorff}
    $X$を位相空間,$Y$をHausdorff空間とし,$f:X\to Y$を連続写像とする.
    \begin{enumerate}
        \item $X$がコンパクトならば,$f$は閉写像である.
        \item $A$が$X$のコンパクト集合ならば,$f(A)$は$Y$の閉集合である.
    \end{enumerate}
\end{proposition}
\begin{proof}\mbox{}
    \begin{enumerate}
        \item \begin{description}
            \item[場面設定] 任意の閉集合$A\subset X$について,像$f(A)\subset Y$が閉であることを示せば良い.$f$のグラフ$\Gamma\subset X\times Y$を考えると,$f(A)=\pr_2(\pr_1^{-1}(A)\cap\Gamma)$である.
            \item[終域Hausdorff性] $Y$がHausdorffであるため,$\Gamma$は閉である(Hausdorff性の特徴付け\ref{prop-characterization-of-Hausdorff-in-terms-of-equalizer}).
            \item[挿入曲] $\pr_1^{-1}(A)$が閉集合であるため,$\pr_1^{-1}(A)\cap\Gamma$も閉集合である(積位相の特徴付け\ref{prop-characterization-product-topology}より,射影は連続).
            \item[始域compact性] $X$がcompactであるため,$\pr_2:X\times Y\to Y$は閉写像である.よって,$f(A)=\pr_2(\pr_1^{-1}(A)\cap\Gamma)$は閉(コンパクト性の特徴付け\ref{prop-characterization-of-compactness-in-terms-of-projection}).
        \end{description}
        \item 
        1.より,制限$f|_A:A\to Y$は閉写像である.その閉集合$A$の像$f(A)$は閉である.
    \end{enumerate}
\end{proof}

\begin{corollary}\label{cor-Compact-Hausdorff}
    $X$をコンパクト空間,$Y$をハウスドルフ空間とし,$f:X\to Y$を連続写像とする.
    \begin{enumerate}
        \item $f$が全単射ならば,$f$は同相写像である.
        \item $f$が単射ならば,$f:X\to Y$は閉埋め込みである.
        \item 像$f(X)$は$Y$の閉集合であり,$f$が引き起こす可逆写像$\overline{f}:X/R_f\to f(X)$は商空間から部分空間への同相写像である.
    \end{enumerate}
\end{corollary}
\begin{proof}\mbox{}
    \begin{enumerate}
        \item 逆写像$g:Y\to X$が連続だと示せば良い.任意の開集合$U\subset X$について,$F:=X\setminus U$は閉より,$g^{-1}(U)=g^{-1}(X\setminus F)=Y\setminus g^{-1}(F)=Y\setminus f(F)$は$Y$の開集合である.
        \item $f:X\to Y$は閉写像でもあるから,閉埋め込みになる.
        \item 命題\ref{prop-continuous-map-from-compact-to-Hausdorff}より,$f(X)$は閉集合になる.
        系\ref{cor-image-of-compact-set-is-compact}より,商空間$X/R_f=p(X)$はコンパクトである.
        系\ref{cor-subspace-of-Hausdorff-is-Hausdorff}より,Hausdorff空間$Y$の部分空間$f(X)$はHausdorffである.よって,1.より,全単射$\overline{f}$は同型である.
    \end{enumerate}
\end{proof}

\begin{proposition}[閉グラフ定理]
    $f:X\to Y$をコンパクトハウスドルフ空間の間の写像とする.グラフが閉であるとき,$f$は連続である.
\end{proposition}

\subsection{Hausdorff空間のコンパクト集合}

\begin{tcolorbox}[colframe=ForestGreen, colback=ForestGreen!10!white,breakable,colbacktitle=ForestGreen!40!white,coltitle=black,fonttitle=\bfseries\sffamily,
title=]
    コンパクト集合の閉部分集合はコンパクトであり,Euclid空間のコンパクト集合は有界閉集合と等価である.
\end{tcolorbox}

\begin{lemma}
    $X$をHausdorff空間,$C\subset X$をコンパクト集合とする.任意の$x\notin C$について,互いに素な開集合$A,B$を用いて$C\subset A,x\in B$と出来る.
\end{lemma}

\begin{proposition}[Hausdorff空間内でのコンパクト集合]\label{prop-compact-set-in-Hausdorff-spaces}
    $X$をハウスドルフ空間とする.
    \begin{enumerate}
        \item $X$の部分集合$A$について,(1)$\Rightarrow$(2)が成り立つ,$X$がコンパクト空間である時に逆も成り立つ.
        \begin{enumerate}[(1)]
            \item $A$はコンパクト集合である.
            \item $A$は閉集合である.
        \end{enumerate}
        \item $A,B$が$X$のコンパクト集合ならば,$A\cap B$も$X$のコンパクト集合である.
    \end{enumerate}
\end{proposition}
\begin{proof}\mbox{}
    \begin{enumerate}
        \item $\Rightarrow$は包含写像$i:A\to X$の像$i(A)=A$は閉であることから従う.$\Leftarrow$は系\ref{cor-union-intersection-of-compact-sets-is-compact}より,$X$がコンパクト空間であるとき,閉集合$A(=X\cap A)$はコンパクトである.
        \item 系\ref{cor-product-of-compact-sets-is-compact}より,コンパクト集合の積$A\times B$は積空間$X\times X$のコンパクト集合である.
        $X$はHausdorff空間であるから,特徴付け\ref{prop-characterization-Hausdorff}より,$\Delta_X\subset X\times X$は閉集合である.よって,$(A\times B)\cap\Delta_X=(A\cap B)\times(A\cap B)$は$X\times X$のコンパクト集合である.
        連続写像$\pr_2:X\times X\to X$の像$\pr_2((A\cap B)\times (A\cap B))=A\cap B$もコンパクトである.
        なお,この事実は,積位相の特徴付け\ref{prop-characterization-product-topology}により,
        対角写像$X\to X\times X$が埋め込みであり,その制限により定まる同型$A\cap B\to(A\times B)\cap\Delta_X$からも分かる.
    \end{enumerate}
\end{proof}

\begin{corollary}[コンパクトではない空間でのコンパクト集合]
    $X$を距離空間とし,$X$の部分集合$A$について,次の2条件
    \begin{enumerate}[(1)]
        \item $A$はコンパクトである.
        \item $A$は有界閉集合である.
    \end{enumerate}
    は,(1)$\Rightarrow$(2)が成り立つ,$X=\R^n$の時に逆も成り立つ.
\end{corollary}
\begin{proof}\mbox{}
    \begin{description}
        \item[(1)$\Rightarrow$(2)] 
        \begin{enumerate}
            \item 距離空間内のコンパクト集合は有界である(系\ref{cor-compact-sets-in-metric-space-is-bounded}).距離函数に最大値の定理を適用することからえる.
            \item 距離空間はハウスドルフである(命題\ref{prop-metric-space-is-Hausdorff}).距離函数は実数に値をとっているために,自由に点を近傍によって分離できるのである.
            \item ハウスドルフ空間のコンパクト集合は閉集合である(命題\ref{prop-compact-set-in-Hausdorff-spaces}).
        \end{enumerate}
        \item[(2)$\Rightarrow$(1)]
        命題\ref{prop-compact-sets-in-R^n}.$\R^n$では有界閉集合は,Heine-Borelの定理の帰結だが,コンパクト空間の閉集合となる.
    \end{description}
\end{proof}

\subsection{商空間への遺伝}

\begin{tcolorbox}[colframe=ForestGreen, colback=ForestGreen!10!white,breakable,colbacktitle=ForestGreen!40!white,coltitle=black,fonttitle=\bfseries\sffamily,
title=]
    コンパクトハウスドルフ空間のコンパクト部分集合を一点に潰して得られる空間は再びコンパクトハウスドルフ空間である.
\end{tcolorbox}

\begin{proposition}[ハウスドルフはコンパクト集合を一点に潰してもハウスドルフである]\label{prop-compact-Hausdorff-after-one-point-quotient-map}
    $X$をハウスドルフ空間とし,$A\subset X$を空でないコンパクト集合とする.
    $A$を一点に潰して得られる空間$Y$はハウスドルフである.
    さらに,$X$がコンパクトなら,$Y$もコンパクトである.
\end{proposition}
\begin{proof}\mbox{}
    \begin{description}
        \item[商空間の復習] 
        $R_A:=(A\times A)\cup\Delta_X$とする.$q:X\to X/R_A=Y$を商写像とし,$q(A)=:\{c\}$とする.
        この一点集合は像位相について閉だから,$Y\setminus\{c\}$は開である.
        実際,Hausdorff空間内のコンパクト集合$A\subset X$は閉より,$X\setminus A$は開であり,$q^{-1}(Y\setminus\{c\})=X\setminus q^{-1}(c)=X\setminus A$を満たす集合$Y\setminus\{c\}$は開である.
        従って,制限$q|_{X\setminus A}:X\setminus A\to Y\setminus\{c\}$は同相写像である.
        \item[命題の言い換え]
        この値域$Y\setminus\{c\}$は,Hausdorff空間$X$の部分空間(これもまたHausdorff)と同相だから,Hausdorffである.
        証明としては,逆写像について命題\ref{prop-Hausdorff性は連続単射によって足に感染る}による.
        以下,任意の点$x\in X\setminus A$に対して,$c$の開近傍$U\subset Y$と$q(x)$の開近傍$V\subset Y$で互いに素なものを構成すれば良い.
        \item[$A$がcompactだと$q(A)=c$が近傍によって分離できる理由] \mbox{}\\
        コンパクト性の特徴付け\ref{prop-characterization-of-compactness-in-terms-of-product-topology}を用いる.
        $W:=(X\times X)\setminus\Delta_X$を考えると,$X$がHausdorffより,これは開で,$x\notin A$より,これは$A\times\{x\}$の開近傍である.
        $A$はコンパクトだから,
        \[A\times\{x\}\subset\tilde{U}\times\tilde{V}\subset W\]
        を満たす$A$の開近傍$\tilde{U}\subset X$と$x$の開近傍$\tilde{V}\subset X$が存在する.$\tilde{U}\times\tilde{V}\subset W$とは,任意の$u\in\tilde{U}$と$v\in\tilde{V}$について$u\ne v$ということだから,
        $\tilde{U}\cap\tilde{V}=\emptyset$である.

        この像を$U:=q(\tilde{U}),V:=q(\tilde{V})$と置けば,$q$は$A$を潰す商写像だから,$q^{-1}(\tilde{U})=U,q^{-1}(\tilde{V})=V$が成り立つ.
        よって$q^{-1}(\tilde{U}\cap\tilde{V})=U\cap V=\emptyset$で,また,$U\subset Y$は$q(A)=c$の開近傍で,$V\subset Y$は$q(x)$の開近傍である.
        よって,$Y$はHausdorffである.
        \item[コンパクト性] $X$がコンパクトなら,その$q$による像である$Y$もコンパクトである.
    \end{description}
\end{proof}

\subsection{コンパクトハウスドルフならば正規}

\begin{corollary}[コンパクト・ハウスドルフ空間の互いに交わらない閉集合は,開集合で分離できる]\label{cor-compact-Hausdorff-space-is-normal}
    $X$がコンパクト・ハウスドルフ空間ならば,$X$は正規である.
\end{corollary}
\begin{proof}
    $A,B\subset X$を互いに素な空でない閉集合とする.
    $A$の開近傍$U\subset X$と,$B$の開近傍$V\subset X$で,$U\cap V=\emptyset$を満たすものを構成すれば良い.
    \begin{description}
        \item[空間$Z$の構成] 
        compact-Hausdorff空間の閉集合とはcompact集合であるから(命題\ref{prop-compact-set-in-Hausdorff-spaces}),
        これを1点に潰して得られる空間$Y:=X\setminus(A\times A)\cup\Delta_X$はcompact-Hausdorffである(命題\ref{prop-compact-Hausdorff-after-one-point-quotient-map}).この商写像を$q:X\to Y$とすると,これは連続だから,$B':=q(B)$もcompactである(系\ref{cor-image-of-compact-set-is-compact}).
        こうして,$Y$を1点に潰して得られる空間$Z:=Y\setminus(B'\times B')\cup\Delta_Y$もcompact-Hausdorffである.
        この商写像を$q':Y\to Z$とする.
        \item[Hausdorff性から結論を得る]
        $p:=q'\circ q:X\to Z$とし,$a,b\in Z$を$p^{-1}(a)=A,p^{-1}(b)=B$で定まる1点の名前とする($A\cap B=\emptyset$よりwell-definedである).
        $a\ne b$だから,$a$の開近傍$U'\subset Z$と$b$の開近傍$V'\subset Z$で$U'\cap V'=\emptyset$を満たすものが存在する.
        $U:=p^{-1}(U'),V:=p^{-1}(V')$と置けば良い.
    \end{description}
\end{proof}

\section{正則空間と正規空間}

\begin{tcolorbox}[colframe=ForestGreen, colback=ForestGreen!10!white,breakable,colbacktitle=ForestGreen!40!white,coltitle=black,fonttitle=\bfseries\sffamily,
title=]
    Hahn-Banachの定理のように,$C(X;[0,1])$が$X$上に分離族を定める空間の一つの十分条件が正規性である.
    第2可算な正規空間が距離付け可能で,可分距離空間を定める.
\end{tcolorbox}

\subsection{定義}

\begin{definition}[regular, normal]
    Hausdorff空間$X$について,
    \begin{enumerate}
        \item 任意の非空閉集合$A$と$x\notin A$とが開近傍によって分離出来るとき,\textbf{正則}であるという.
        \item 任意の非空閉集合$A,B$が互いに素ならば,近傍によって分離出来るとき,\textbf{正規}であるという.
    \end{enumerate}
\end{definition}

\subsection{特徴付け}

\begin{tcolorbox}[colframe=ForestGreen, colback=ForestGreen!10!white,breakable,colbacktitle=ForestGreen!40!white,coltitle=black,fonttitle=\bfseries\sffamily,
title=]
    正規性の特徴付けに,Lebesgue測度の性質の萌芽を見ることが出来る.
\end{tcolorbox}

\begin{proposition}[正則性の特徴付け]
    $X$をHausdorff空間とする.次の3条件は同値:
    \begin{enumerate}
        \item 閉集合$A$と一点$x\notin A$が開近傍によって分離できる.
        \item 任意の点$x\in X$の近傍$x\in V$について,$x$の近傍$x\in U$が存在して$\overline{U}\subset V$を満たす.
        \item 空でない任意の閉集合$A$について,これを一点に潰して得られる空間$X/(\Delta_X\cup(A\times A))$もHausdorffである(正規性の命題\ref{prop-one-point-compactification}に当たる見方).
    \end{enumerate}
\end{proposition}

\begin{proposition}[正規性の特徴付け]\label{prop-characterizatioin-of-normality}
    $X$をHausdorff空間とする.次の3条件は同値:
    \begin{enumerate}
        \item 互いに交わらない任意の閉集合$A,B$について,開近傍によって分離できる.
        \item $F\subset G$を満たす任意の閉集合$F$と開集合$G$に付いて,$F\subset H\land \o{H}\subset G$を満たす開集合$H$が存在する.
        \item 任意の閉集合$A$の近傍$A\subset V$について,$A$の開近傍$A\subset U$が存在して$\overline{U}\subset V$を満たす.
        \item 互いに交わらない任意の閉集合$A,B$について,Urysohnの距離関数によって分離できる(補題\ref{lemma-Urysohn's-lemma}).
        \item 任意の閉集合$F\subset X$上の任意の連続写像$f:F\to I$は$X$上に延長する.
    \end{enumerate}
\end{proposition}
\begin{proof}\mbox{}
    \begin{description}
        \item[(1)$\Rightarrow$(2)] 
        $B:=X\setminus V$と定めると,$A\cap B\ne\emptyset$より,$U\cap W=\emptyset$を満たす開近傍$A\subset U,B\subset W$が存在する.再び$C:=X\setminus W$と置くと,$A\subset U\subset C\subset V$が成り立つが,$C$は閉集合より,$\o{U}\subset C$を得る.2つ併せて,$A\subset\o{U}\subset V$である.
        \item[(1)$\Rightarrow$(3)] Urysohnの補題\ref{lemma-Urysohn's-lemma}

    \end{description}
\end{proof}

\subsection{正規であるための十分条件}

\begin{proposition}
    コンパクトハウスドルフ空間($T_3$)は正規であるが,パラコンパクトなHausdorff空間も正規である.
    一方で,局所コンパクトな空間は正規とは限らない.
\end{proposition}

\begin{proposition}
    距離空間は$T_4$である.
\end{proposition}

\begin{proposition}
    Lindelofな正則空間は正規である.
\end{proposition}

\subsection{正規性の弱遺伝性}

\begin{proposition}
    正規性は弱遺伝的である.すなわち,任意の正規空間の閉部分空間は正規である.
\end{proposition}

\subsection{Urysohnの補題}

\begin{lemma}[Urysohn's lemma]\label{lemma-Urysohn's-lemma}
    $X$を正規空間とする.$A$を$X$の閉集合,$U$を$X$の開集合で,$A\subset U$を満たすものとする.
    \begin{enumerate}
        \item $X$の閉集合$B$と,$X$の開集合$V$で,$A\subset V\subset B\subset U$を満たすものが存在する.
        \item (Urysohn's lemma) 連続関数$f:X\to[0,1]$で,$A\subset f^{-1}(0)\subset f^{-1}([0,1))\subset U$を満たすものが存在する.
    \end{enumerate}
\end{lemma}
\begin{proof}\mbox{}
    \begin{enumerate}
        \item 正規性の特徴付け\ref{prop-characterizatioin-of-normality}より.
        \item 
    \end{enumerate}
\end{proof}

\subsection{正規空間上の有界連続関数}

\begin{tcolorbox}[colframe=ForestGreen, colback=ForestGreen!10!white,breakable,colbacktitle=ForestGreen!40!white,coltitle=black,fonttitle=\bfseries\sffamily,
title=]
    正規空間内の閉集合上に定義された有界連続関数は,一様ノルムを変えないような全空間への延長が存在する.
\end{tcolorbox}

\begin{theorem}[Tietzeの延長定理 (AC)]
    $X$を正規空間,$A\subset X$を閉とする.
    \begin{enumerate}
        \item 制限$F_A:C_b(X)\to C_b(A)$は全射であり,特に任意の$g\in C_b(A)$に対して$f\in F_A^{-1}(g)$は$\norm{f}_\infty=\norm{g}_\infty$に選べる.
        \item 制限$C(X)\to C(A)$も全射である.
    \end{enumerate}
\end{theorem}

\begin{corollary}[有界連続関数の空間が可分になる条件 (AC)]
    $X$を正規空間とする.次の2条件は同値:
    \begin{enumerate}
        \item $X$はコンパクトかつ距離付け可能である.
        \item $C_b(X)$は一様位相について可分である.
    \end{enumerate}
\end{corollary}

\subsection{1の分解定理}

\begin{definition}
    被覆$(A_n)$が局所有限であるとは,$\forall_{x\in X}\;\exists_{A\in\O(x)}\;A_n\cap A=\emptyset\;\fe$を満たすことをいう.
\end{definition}

\begin{theorem}
    正規空間の任意の局所有限な開被覆$(A_n)$について,これに従属する1の分割が存在する.
    すなわち,連続関数の列$\{f_n\}\subset C(X;[0,1])$であって,次を満たす:
    \begin{enumerate}
        \item $f|_{X\setminus A_n}=0$.
        \item $\sum_{n\in\N}f_n=1$.
    \end{enumerate}
\end{theorem}
\begin{remark}
    Zornの補題によれば,任意の開被覆に関して,これに従属する1の分割が取れる.
\end{remark}

\section{局所コンパクトハウスドルフ空間}

\begin{tcolorbox}[colframe=ForestGreen, colback=ForestGreen!10!white,breakable,colbacktitle=ForestGreen!40!white,coltitle=black,fonttitle=\bfseries\sffamily,
title=]
    Hausdorff空間が局所コンパクトであることと,一点コンパクト化可能であることとは同値.
    すなわち,コンパクトハウスドルフ空間の開部分空間として得られる空間にほかならない.
    また,コンパクトハウスドルフ空間では連続全単射は位相同型であったが,局所コンパクトハウスドルフ空間に限ると,空間はcurryingを許すようになる(コンパクト開位相).
    したがって,このクラスの位相空間が最終的に良い指標になり,Gelfand双対もこのクラスで成り立つ.
\end{tcolorbox}

\subsection{定義}

\begin{tcolorbox}[colframe=ForestGreen, colback=ForestGreen!10!white,breakable,colbacktitle=ForestGreen!40!white,coltitle=black,fonttitle=\bfseries\sffamily,
title=]
    次のような理由で,局所コンパクトハウスドルフ空間というとき,暗黙にHausdorff性を仮定する.
\end{tcolorbox}

\begin{definition}[locally compact]
    位相空間$X$が\textbf{局所コンパクト}であるとは,任意の点$x\in X$が相対コンパクトな近傍を持つことをいう.
    これは,位相$\O$が相対コンパクト集合からなる開基を持つことに同値.
\end{definition}

\begin{proposition}[Hausdorffな場合の特徴付け]
    位相空間$X$について,(1)$\Rightarrow$(2)が成り立つ.$X$がHausdorffのとき,(2)$\Rightarrow$(1)も成り立つ.
    \begin{enumerate}
        \item 任意の点$x\in X$の任意の開近傍$U\in\O(x)$に対して,より小さな開近傍$V\in\O(x),V\subset U$で相対コンパクトなものが存在し,$\o{V}\subset U$が成り立つ.
        \item 任意の点$x\in X$の(開)近傍系$\O(x)$がコンパクト集合からなる基本系を持つ.
    \end{enumerate}
    また,$X$がHausdorffのとき,次の2条件も同値になる.
    \begin{enumerate}\setcounter{enumi}{2}
        \item 任意の点$x\in X$に,相対コンパクトな開近傍が存在する.
        \item 任意の点$x\in X$に,コンパクトな近傍が存在する.
    \end{enumerate}
\end{proposition}

\begin{example}[コンパクトハウスドルフ空間の開部分集合は局所コンパクト]
    コンパクトハウスドルフ空間の任意の開部分集合は局所コンパクトハウスドルフである.
    そして,任意の局所コンパクトハウスドルフ空間はこのように構成される.
\end{example}

\begin{example}[局所コンパクト性は弱遺伝的である]
    局所コンパクト空間の任意の閉部分空間と開部分空間は局所コンパクト.
\end{example}

\subsection{一般化された閉写像補題}

\begin{proposition}[一般化された閉写像補題]
    $X$が位相空間,$Y$が局所コンパクトハウスドルフ空間ならば,連続写像$f:X\to Y$がproperならば閉である.
\end{proposition}

\subsection{Hausdorff空間からの1点コンパクト化による特徴付け}

\begin{tcolorbox}[colframe=ForestGreen, colback=ForestGreen!10!white,breakable,colbacktitle=ForestGreen!40!white,coltitle=black,fonttitle=\bfseries\sffamily,
title=]
    まずは,Hausdorff空間が局所コンパクトであることを特徴付ける.
    コンパクトハウスドルフ空間の開部分空間への同相写像を\textbf{コンパクト化}という.
\end{tcolorbox}

\begin{definition}[relatively compact, locally compact, compactification, (Alexandroff) one-point compactification]
    $X$をハウスドルフ空間,$Y$をコンパクトハウスドルフ空間とする.
    \begin{enumerate}
        \item 開埋め込み$f:X\to Y$が存在するとき,$Y$は$X$の\textbf{コンパクト化}であるという.
        \item さらに,$Y\setminus f(X)$が一点集合である時,$Y$は$X$の\textbf{一点コンパクト化}または\textbf{アレクサンドロフのコンパクト化}であるといい,この点を\textbf{無限遠点}という.
    \end{enumerate}
\end{definition}

\begin{proposition}[局所コンパクトなハウスドルフ空間ならば,1点コンパクト化可能である]\label{prop-one-point-compactification}
    $X$をハウスドルフ空間とする.次の3条件は同値である.
    \begin{enumerate}
        \item $X$は局所コンパクトである.
        \item $X$のコンパクト化が存在する.
        \item $X$の1点コンパクト化が存在する.
    \end{enumerate}
\end{proposition}
\begin{proof}\mbox{}
    \begin{description}
        \item[(3)$\Rightarrow$(2)] 1点コンパクト化は,コンパクト化の特別な場合である.
        \item[(2)$\Rightarrow$(1)] \mbox{}\\
        \begin{description}
            \item[証明方針] 
            コンパクト化$f:X\to Y$が存在する時,$Y$がcompact Hausdorff,$f$が開埋め込みだから,閉集合$A\subset Y$が存在して$f(X)=Y\setminus A$と同相.よって,$X\simeq Y\setminus A$として,$X$が局所コンパクトであると示せば良い.
            
            任意に$x\in X$を取る.$x\notin A$であるから,$Y$が正規より(系\ref{cor-compact-Hausdorff-space-is-normal}),$x$の開近傍$U$と$A$の開近傍$V$が存在して$U\cap V=\emptyset$を満たす.
            この時,
            \[U\subset K:=Y\setminus V\subset X=Y\setminus A\]
            であるが,$K\subset Y$は閉集合であるから,compactである(命題\ref{prop-compact-set-in-Hausdorff-spaces}).よって,$U$は相対コンパクトである.
        \end{description}
        \item[(1)$\Rightarrow$(3)] $X$を局所コンパクト空間として,その1点コンパクト化$i:X\to Y:=X\coprod\{c\}$を構成する.
        \begin{description}
            \item[証明方針] $\O_X\subset P(X)$を$X$の開集合系とし,新たに開集合とすべき集合を$\mathcal{V}:=\{V\in\O_X\mid X\setminus VはXのコンパクト集合\}$とする.
            $\O\subset P(Y)$を,$\O:=\O_X\cup\{V\coprod\{c\}\mid V\in\mathcal{V}\}$とおくと,$\mathcal{V}$がfilterの公理を満たすため,$\O$は$Y$の位相であり(問題\ref{problem-construction-of-topology-in-terms-of-filter-on-onepoint-compactification}.4),$i:X\to Y$が開埋め込みであることを示す.続いて,こうして得た$(Y,\O)$がcompact Hausdorffであることを示す.
            \item[$i,\O$の定義]
            \begin{enumerate}[(1)]
                \item $U\in\O_X,V\in\mathcal{V}$について,$V\subset U$とする.このとき$X\setminus U\subset X\setminus V$はcompact集合の閉部分集合だからコンパクトであり(命題\ref{cor-union-intersection-of-compact-sets-is-compact}),$U\in\mathcal{V}$である.
                \item $(V_i)_{i\in I}$を$\mathcal{V}$の有限族とすると,$X\setminus\cap_{i\in I}V_i=\cup_{i\in I}(X\setminus V_i)$は系\ref{cor-union-intersection-of-compact-sets-is-compact}よりコンパクトである.よって,$\cap_{i\in I}V_i\in\mathcal{V}$.
            \end{enumerate}
            以上より,問題\ref{problem-construction-of-topology-in-terms-of-filter-on-onepoint-compactification}.4より,$\O$は$Y$の位相であり,$i:X\to Y$は開埋め込みである.
            \item[Hausdorff性]
            $X$はHausdorffであるから,あとは任意の$x\in X$に対して$c$と近傍によって分離できることを示せば良い.
            $X$は局所コンパクトであるから,$x$の開近傍$U$とコンパクト集合$K$であって,$x\in U\subset K\subset X$を満たすものが存在する.$V:=Y\setminus K\in\O$とすると,これは$c$の開近傍であり,$U\cap V=\emptyset$である.
            \item[compact性]
            $Z$を任意の位相空間とし,$z\in Z$を任意の点とする.任意の$Y\times\{z\}$の開近傍$W\subset Y\times Z$を取る.

            今,積位相の特徴付け\ref{prop-characterization-product-topology}への,$c\in Y$の開近傍$U\subset Y$と$z\in Z$の開近傍$V\subset Z$であって,$U\times V\subset W$を満たすものが存在する.
            $U\cap X\in\mathcal{V}$より,補集合$A:=Y\setminus U=i(X\setminus(U\cap X))$はcompactである.$W$は$A\times\{z\}$の開近傍でもあるから,$A\times\{z\}\subset A\times V'\subset W$を満たす$z$の開近傍$V'\subset Z$が存在する.
            すると,$z$の開近傍$V\cap V'$について,
            \begin{align*}
                Y\times\{z\}&\subset Y\times(V\cap V')\\
                &\subset (U\times V)\cup(A\times V')&\because Y=U\cap A,V\cap V'\subset V,V'\\
                &\subset W
            \end{align*}
            が成り立つ.よって,コンパクト性の特徴付け\ref{prop-characterization-of-compactness-in-terms-of-product-topology}より,$Y$はコンパクトである.
        \end{description}
    \end{description}
\end{proof}

\subsection{1点コンパクト化の位相}

\begin{tcolorbox}[colframe=ForestGreen, colback=ForestGreen!10!white,breakable,colbacktitle=ForestGreen!40!white,coltitle=black,fonttitle=\bfseries\sffamily,
title=]
    このとき,$\infty$上に$f(\infty)=0$として延長する連続関数を$f\in C_0(X)$とすると,$\o{C_c(X)}=C_0(X)$となるのであった.
\end{tcolorbox}

\begin{lemma}[1点コンパクト化の位相の定め方のwell-definedness]
    $X$を局所コンパクト空間,$Y$をコンパクトハウスドルフ空間とし,$f:X\to Y$を$X$の一点コンパクト化とする.
    $\{c\}=Y\setminus f(X)$と置くと,次が成り立つ.
    \begin{enumerate}
        \item $\{U\in P(Y)\mid UはYの開集合でc\notin U\}=\{f(V)\mid VはXの開集合\}$,
        \item $\{U\in P(Y)\mid UはYの開集合でc\in U\}=\{Y\setminus f(A)\mid AはXのコンパクト集合\}$.
    \end{enumerate}
\end{lemma}
\begin{proof}\mbox{}
    \begin{enumerate}
        \item $f$は開埋め込みだから,$\overline{f}:X\to Y\setminus\{c\}$は同相写像.$c\notin U$を満たす部分集合$U\subset f(X)$が開集合であることは,$U$がある$X$の開集合の像であることに同値である.
        \item $Y$はcompact Hausdorffだから,命題\ref{prop-compact-set-in-Hausdorff-spaces}より,$U$が開集合であることと,$Y\setminus U$がcompactであることは同値.特に$c\in U$の時は補集合$Y\setminus U$が$f^{-1}(Y\setminus U)=X\setminus f^{-1}(U)$と同相だから,これは$U$があるcompact集合$A:=f^{-1}(U)$の像の補集合であることと同値.
    \end{enumerate}
\end{proof}

\begin{problem}[一点コンパクト化の位相の構成(フィルターの言葉で)]\label{problem-construction-of-topology-in-terms-of-filter-on-onepoint-compactification}
    $Y$を位相空間,$X\subset Y$を部分空間とし,補集合は一点集合:$Y\setminus X=\{c\}$とする.$i:X\to Y$を包含写像とし,$Y$の開集合系を$\O_Y$,$X$の開集合系を$\O_X$とする.
    $\mathcal{V}:=\{i^*(U)=U\cap X\mid U\in\O_Yはcの開近傍\}$とする.
    \begin{enumerate}
        \item $\mathcal{V}$は次の2条件を満たす.
        \begin{enumerate}[(1)]
            \item (upward-closed) $U\in\O_X,V\in\mathcal{V}$について,$V\subset U\Rightarrow U\in\mathcal{V}$.
            \item (downward-directed) $(V_i)_{i\in I}$が$\mathcal{V}$の有限族ならば,$\cap_{i\in I}V_i\in\mathcal{V}$.
        \end{enumerate}
        \item 次の条件は,$c$が$Y$の孤立点でないことに同値.
        \begin{enumerate}[(1)]\setcounter{enumii}{2}
            \item (proper) $\emptyset\notin\mathcal{V}$.
        \end{enumerate}
        \item $X$が$Y$の開部分空間であることと,$Y$の開集合系が$\O_Y=\O_X\coprod\{V\coprod\{c\}\mid V\in\mathcal{V}\}$であることとが同値である.
        \item $X$を位相空間とし,$\O_X$を$X$の開集合系,$\mathcal{V}\subset\O_X$を(1),(2)を満たす部分集合とする.無縁和$Y=X\coprod\{c\}$に対して,$\O\subset P(Y)$を$\O=\O_X\coprod\{V\coprod\{c\}\mid V\in\mathcal{V}\}$と定めるとこれは$Y$の位相であることを示せ.
    \end{enumerate}
\end{problem}
\begin{proof}\mbox{}
    \begin{enumerate}
        \item (1)は,$V\in\mathcal{V}$ならば,$V\cup\{c\}\in\O_Y$は$c$の開近傍.$V\subset U$の時,$U\cup\{c\}=(U\cup V)\cup\{c\}$も$c$の開近傍だから,$U\in\mathcal{V}$.(2)は,各$V_i\cup\{c\}$が$c$の開近傍だから,$\cap_{i\in I}(V_i\cup\{c\})=\paren{\cap_{i\in I}V_i}\cup\{c\}$も$c$の開近傍である.よって,$\cap_{i\in I}V_i\in\mathcal{V}$.
        \item $c$が$Y$の孤立点であることは,$\{c\}\in\O_Y$が$c$の開近傍であることだから,これは$\{\}\in\mathcal{V}$であることに同値.
        \item $X\in\O_Y$であるとき,$X$は開部分空間である.$X$が開部分空間であるとき,$i:X\to Y$が開埋め込みであるから,$\O_X=i^*\O_Y$であるが,$\O_Y\setminus\O_X$は全て$c$の開近傍であるから,$\O_Y\setminus\O_X=\{V\coprod\{c\}\mid V\in\mathcal{V}\}$である.
        \item $(U_i)_{i\in I}$を$\O_X$の族,$(V_j)_{j\in J}$を$\{V\coprod\{c\}\mid V\in\mathcal{V}\}$の族として,族$(W_i)_{i\in I\coprod J}:=(U_i)_{i\in I}\coprod(V_j)_{j\in J}$を考えても一般性は失われない.$\cup_{i\in I\coprod J}W_i=(\cup_{i\in I}U_i)\cup(\cup_{j\in J}V_j)$は$\mathcal{V}$の条件(1)より,$\O$の元である.同様にして有限族の共通部分についても$\O$の元であることがわかる.
    \end{enumerate}
\end{proof}

\subsection{一般理論}

\begin{tcolorbox}[colframe=ForestGreen, colback=ForestGreen!10!white,breakable,colbacktitle=ForestGreen!40!white,coltitle=black,fonttitle=\bfseries\sffamily,
title=]
Hausdorffに(1点)コンパクト化が可能な空間が局所コンパクトハウスドルフ空間である.
コンパクト化の概念を一般化し,Hausdorffにコンパクト化が可能な空間とは,全正規空間であり,これは正規空間を含む.
\end{tcolorbox}

\begin{definition}
    一般の位相空間$X$のコンパクト化とは,コンパクト集合$\wt{X}$と像が$\wt{X}$上稠密な埋め込み$i:X\mono\wt{X}$の組をいう.
\end{definition}

\begin{proposition}\mbox{}
    \begin{enumerate}
        \item 任意の位相空間$X$は一点コンパクト化を持つ.
        \item 一点コンパクト化$\wt{X}=X\cup\{\infty\}$がHausdorffであることと,$X$が局所コンパクトHausdorffであることとは同値.
    \end{enumerate}
\end{proposition}

\subsection{隆起関数}

\begin{tcolorbox}[colframe=ForestGreen, colback=ForestGreen!10!white,breakable,colbacktitle=ForestGreen!40!white,coltitle=black,fonttitle=\bfseries\sffamily,
title=]
    局所コンパクトハウスドルフ空間は正規空間とは限らないが,正規空間で成り立つところのUrysohnの補題と似た消息が,一歩引いて成り立つ.
    これは,関数族$C_c(X)$は,任意の互いに素なコンパクト集合と開集合とを分離できるくらいに内容が濃いことを言っている.
\end{tcolorbox}

\begin{proposition}
    局所コンパクトハウスドルフ空間$X$の任意のコンパクト部分集合$C\subset X$と開集合$C\subset A\osub X$とについて,コンパクト台を持つ連続関数$f:X\to[0,1]$であって,
    $f|_C=1$かつ$\supp f\subset A$を満たすものが存在する.
\end{proposition}

\subsection{弱遺伝性}

\begin{proposition}[局所コンパクト性は弱遺伝的である]
    局所コンパクト空間の任意の閉部分空間と開部分空間は局所コンパクト.
\end{proposition}

\subsection{直積への伝播}

\begin{proposition}
    局所コンパクト空間の有限直積は局所コンパクト.
\end{proposition}


\subsection{距離付け可能である十分条件}

\begin{theorem}
    $X$を局所コンパクト空間とする.
    次の5条件は同値:
    \begin{enumerate}
        \item $X$は第2可算である.
        \item $X$は可分かつ距離付け可能.
        \item $X$は$\sigma$-コンパクトかつ距離付け可能.
        \item $X$は$\sigma$-コンパクトかつ,任意の$x\in X$について第2可算な開近傍$U\in\O(x)$が存在する.
        \item $X$の一点コンパクト化が第2可算である.
    \end{enumerate}
\end{theorem}

\section{開被覆による局所コンパクト空間の精緻化}

\begin{tcolorbox}[colframe=ForestGreen, colback=ForestGreen!10!white,breakable,colbacktitle=ForestGreen!40!white,coltitle=black,fonttitle=\bfseries\sffamily,
    title=]
    距離付け可能性の議論で,$\sigma$-コンパクト性が出現した.これが何を意味するかを考える.
    \textbf{局所コンパクト空間について},次の関係が成り立つ:
    \[\xymatrix{
        \text{局所連続}\ar[ddrr]&\text{距離化可能}\ar[ddr]&\text{第2可算}\ar[d]\\
        &&\sigma\text{-コンパクト}\ar[d]&\text{Lindelöf}\ar@/^3pt/@{.>}[dl]^-{\text{正規}}\\
        &&\text{パラコンパクト}\ar@/^3pt/@{.>}[ur]^-{\text{Hausdorff}}
    }\]
    パラコンパクト性が第2可算性を含意して,真ん中の3概念Lindelöf性とが縮退するのは,局所コンパクト空間がさらに,Hausdorffかつ局所Euclidかつ連結成分が高々可算個であるときである.
    これらを全て湛えた概念が,位相多様体である.
    実はこのとき,局所連続であり,距離化可能でもある.
    また全ての可微分多様体はRiemann計量を持つ.
\end{tcolorbox}

\subsection{定義}

\begin{tcolorbox}[colframe=ForestGreen, colback=ForestGreen!10!white,breakable,colbacktitle=ForestGreen!40!white,coltitle=black,fonttitle=\bfseries\sffamily,
    title=]
    位相多様体($C^0$-多様体)は明らかに局所コンパクトである.
    局所コンパクトな空間に第2可算性を与えると,パラコンパクト性を持つ.
    この2条件が同値になるのは,連結成分が高々可算個であるときである.
    これで,可微分多様体の定義には第2可算性が入るのである.
    パラコンパクトな多様体は,任意の開被覆に関して1の分割を持つ.
\end{tcolorbox}

\begin{definition}[manifold]
    Hausdorff空間$M$と$C^r$-局所座標系$\{U_\al,\varphi_\al\}_{\al\in A}$との組を\textbf{$C^r$-多様体}という.
    $C^0$-多様体は局所的にEuclid空間と同相な空間のことであり,明らかに局所コンパクトかつ局所(弧状)連結である.
\end{definition}

\begin{definition}[paracompact, Lindelöf]
    位相空間$M$について,
    \begin{enumerate}
        \item 開被覆$(U_\al)_{\al\in A}$が\textbf{局所有限}であるとは,$\forall_{x\in M}\;\exists_{U\in\O(x)}\;\Abs{\Brace{\al\in A\mid U_\al\cap U\ne\emptyset}}\in\N$.
        \item 被覆$(U_\al)_{\al\in A}$の\textbf{細分}とは,被覆$(V_\beta)_{\beta\in B}$であって,$\forall_{\beta\in B}\;\exists_{\al\in A}\;V_\beta\subset U\al$を満たすものをいう.全ての元$U_\al$を縮めて作った被覆をいう.
        \item 任意の開被覆が,局所有限な開細分\footnote{元すべて開集合であるような被覆を開被覆ということと同様.}を持つとき,$M$を\textbf{パラコンパクト}であるという.
        \item 位相空間が\textbf{Lindelöf}または\textbf{可算被覆性}を持つとは,任意の開被覆が可算部分被覆を持つことをいう.
    \end{enumerate}
\end{definition}

\begin{example}[Sorgenfrey line]
    \[\U(x):=\Brace{[x,y)\in P(\R)\mid x<y\in\R}\]
    とし,$\U:=\cup_{x\in\R}\U(x)$とすると,これは$\R$の近傍系の公理を満たす.
    換言すれば,半開区間$[a,b)$の全体は基底をなし,これが生成する位相を\textbf{右半開区間位相}といい,この位相空間を\textbf{Sorgenfrey直線}といい,$\R_l$で表す.
    これはEuclid位相よりも強い位相になる(第2可算でない).
    積空間$\R_l\times\R_l$はSorgenfrey平面という.
    \begin{enumerate}
        \item $\R_l$の基底$[a,b)$は閉集合でもある.
        \item $\R_l$は可分かつ第1可算だが,第2可算ではない.
        \item $\R_l$はLindelofである.実は,$\sigma$を$[x,y)$の形をした半開区間の族とすると,$\cup_{A\in\sigma}A=\cup_{A\in\sigma'}A$を満たす可算部分集合$\sigma'\subset\sigma$が存在する.
        \item 正規・パラコンパクト性の積への伝播の反例:$\R_l$は$T_1$かつ正規でパラコンパクトでもあるが,$\R_l\times\R_l$は正規でなく,従ってパラコンパクトでもない.
        \item 可分性の部分集合への遺伝の反例:$\R_l\times\R_l$は可分であるが,非可算な離散部分集合(したがって可分でない部分集合)を持つ.
        すなわち,部分集合$\Brace{(x,y)\in\R^2\mid x+y=0}$は相対位相について離散であり,可分でないが,$\R^2$-閉である.
        \item $\R_l\times\R_l$はLindelofでない.
    \end{enumerate}
\end{example}

\subsection{パラコンパクト性}

\begin{tcolorbox}[colframe=ForestGreen, colback=ForestGreen!10!white,breakable,colbacktitle=ForestGreen!40!white,coltitle=black,fonttitle=\bfseries\sffamily,
title=]
    連結性の仮定を落としたとき,(3)$\Rightarrow$(4),(3)$\Rightarrow$(1)は残る.
\end{tcolorbox}

\begin{theorem}[局所コンパクト空間のパラコンパクト性の特徴付け]
    $M$を連結な局所コンパクト空間とする.次の4条件は同値.
    \begin{enumerate}
        \item $M$はパラコンパクト.
        \item $M$には局所有限で局所コンパクトな開被覆$(U_\al)_{\al\in A}$を持つ:各$\o{U_\al}$はコンパクト.
        \item $M$は$\sigma$-コンパクト:$M$は高々可算個のコンパクト集合$(S_i)_{i\in\N}$の和集合として表せる.
        \item $M$は高々可算個の開集合$(U_i)_{i\in\N}$の和で表せ,かつ,各$U_i$は相対コンパクトで$\forall_{i\in\N}\;\o{U_i}\subset U_{i+1}$を満たす.
    \end{enumerate}
\end{theorem}

\begin{theorem}[パラコンパクト多様体の開被覆]\label{thm-open-cover-of-paracompact-locally-Euclidean-space}
    $M^n$をパラコンパクトな位相多様体とし,$\U=(\wt{U}_i)$を$M^n$の局所有限な開被覆とする.このとき,$\U$の局所有限な細分$\V=(V_j)$であって,次の条件を満たすものが存在する:
    $\V$を$M$上の開集合の族と考えたとき,$\V=\V_0\cup\V_1\cup\cdots\V_n$と分解でき,各$\V_k$の元は互いに素な開集合族となっている.
\end{theorem}

\subsection{第2可算ならパラコンパクト}

\begin{tcolorbox}[colframe=ForestGreen, colback=ForestGreen!10!white,breakable,colbacktitle=ForestGreen!40!white,coltitle=black,fonttitle=\bfseries\sffamily,
title=]
    第2可算な空間が,局所コンパクトであるか,または,正規ならば,パラコンパクトである.
\end{tcolorbox}

\begin{corollary}
    局所コンパクトハウスドルフな局所Euclid的な空間$M$について,次の2条件は同値.
    \begin{enumerate}
        \item $M$は第2可算である.
        \item $M$はパラコンパクトかつ連結成分が高々可算個である.
    \end{enumerate}
    この同値な条件を満たすとき位相多様体という.
\end{corollary}
\begin{proof}\mbox{}
    \begin{description}
        \item[(1)$\Rightarrow$(2)] まず,連結成分が非可算個であった場合は,全ての連結成分からなる開被覆については可算な部分被覆を持たないので矛盾.よって連結成分は高々可算個である.
        あとは,それぞれの連結成分がパラコンパクトであることを示せば良い.任意の開被覆に対して,可算な部分被覆が存在する.
    \end{description}
\end{proof}

また,正規な第2可算空間がパラコンパクトであることは,次の補題からすぐに従う.

\begin{lemma}[Michael's theorem (53)]
    $X$が次を満たす正規空間ならば,パラコンパクトである.
    \begin{quote}
        every open cover of $X$ has a refinement by a union of a countable set of locally finite sets of open subsets (not necessarily covering).
    \end{quote}
\end{lemma}

\subsection{パラコンパクト空間が第2可算になるとき}

\begin{tcolorbox}[colframe=ForestGreen, colback=ForestGreen!10!white,breakable,colbacktitle=ForestGreen!40!white,coltitle=black,fonttitle=\bfseries\sffamily,
title=]
    先程の局所Euclid性はどのように外せるか?
    プレコンパクト性という,位相空間論的ではなく,どちらかというと一様空間的な概念が出現してしまう.
\end{tcolorbox}

\begin{proposition}
    $X$を局所コンパクトでパラコンパクトな空間とする.
    プレコンパクトで弧状連結な開集合による開被覆を持つとき,第2可算である.
\end{proposition}

\subsection{$\sigma$-コンパクトならばパラコンパクト}

\begin{tcolorbox}[colframe=ForestGreen, colback=ForestGreen!10!white,breakable,colbacktitle=ForestGreen!40!white,coltitle=black,fonttitle=\bfseries\sffamily,
title=]
    局所コンパクト空間が$\sigma$-コンパクトになると,パラコンパクトである.
    また,局所コンパクト空間が$\sigma$-コンパクトになる条件が,第2可算である.
\end{tcolorbox}

\begin{lemma}
    $X$を局所コンパクト空間,
    $\U$を$\O_X$の基底とする.
    $\U':=\Brace{U\in\U\mid U\text{は相対コンパクト}}$も$\O_X$の基底である.
\end{lemma}

\begin{proposition}
    $X$を局所コンパクト空間,$Y=X\cup\{b\}$を一点コンパクト化とする.
    次の2条件は同値.
    \begin{enumerate}
        \item $X$は$\sigma$-コンパクトである.
        \item $\O(b)$に,可算な基本系が存在する.
    \end{enumerate}
\end{proposition}

\begin{proposition}
    $X$を局所コンパクト空間とする.
    $\sigma$-コンパクトならば,パラコンパクトである.
    $X$が連結ならば,逆も言える.
\end{proposition}

\subsection{第2可算ならばLindelöf}

\begin{lemma}[second-countable then Lindelöf]
    第2可算な位相空間は,可算被覆性を持つ.すなわち,任意の開被覆に対して,可算な部分被覆が存在する.
\end{lemma}
\begin{proof}\mbox{}
    \begin{description}
        \item[私の案] 可算な開基$\U:=(U_i)_{i\in\N}$を1つ取る.
        $(V_\al)_{\al\in A}$を任意の開被覆とする.
        任意の$x\in M$について,$\exists_{\al}\;x\in V_\al$であるが,$\U$は$M$の開基だから,ある$i\in\N$が存在して,$x\in U_i\subset V_\al$を満たす.
        こうして$i$と結びつけられる$V_\al$を集めれば,可算な部分被覆となっている. 
        \item[stackexchange] ある種の背理法と言える論理マジック
    \footnote{\url{
        https://math.stackexchange.com/questions/2257690/if-x-is-second-countable-then-x-is-lindel%C3%B6f
    }}
    \end{description}
\end{proof}

\subsection{Lindelöf性とパラコンパクト性}

\begin{tcolorbox}[colframe=ForestGreen, colback=ForestGreen!10!white,breakable,colbacktitle=ForestGreen!40!white,coltitle=black,fonttitle=\bfseries\sffamily,
title=]
    Lindelöf空間が正規ならばパラコンパクトで,パラコンパクト空間がHausdorffならばLindelöf空間である.
    またパラコンパクト空間が可算鎖条件を満たしても,Lindelöfである.
\end{tcolorbox}

\begin{definition}
    $P$を$\bot$を持つ順序集合とする.
    \begin{enumerate}
        \item $P$の\textbf{反鎖}とは,部分集合$A\subset P$であって,$\forall_{a,b\in A}\;a=b\lor0=a\land b$を満たすものをいう.
        \item $P$が$\bot$を持たない場合は,人工的に最小元を付け足して,同様に定義する.
        \item $P$の任意の反鎖が可算であるとき,$P$は可算鎖条件を満たすという.
    \end{enumerate}
\end{definition}

\subsection{局所連結ならばパラコンパクト}

\begin{tcolorbox}[colframe=ForestGreen, colback=ForestGreen!10!white,breakable,colbacktitle=ForestGreen!40!white,coltitle=black,fonttitle=\bfseries\sffamily,
title=]
    結局,局所コンパクト空間の問題は,その大きさと連結成分の数である.
\end{tcolorbox}

\begin{proposition}
    局所コンパクト空間$X$が
    \begin{enumerate}
        \item 連結ならば$\sigma$-コンパクトである.
        \item 局所連結ならばパラコンパクトである.
    \end{enumerate}
\end{proposition}

\subsection{$\sigma$-コンパクトならば正規}

\begin{tcolorbox}[colframe=ForestGreen, colback=ForestGreen!10!white,breakable,colbacktitle=ForestGreen!40!white,coltitle=black,fonttitle=\bfseries\sffamily,
title=]
    $\sigma$-コンパクトな局所コンパクトハウスドルフ空間というのが,なんとも$\sigma$-有限性に似ている.
\end{tcolorbox}

\begin{proposition}
    局所コンパクト空間$X$が$\sigma$-コンパクトでもあるならば,正規である.
\end{proposition}

\begin{proposition}
    局所コンパクト空間$X$が$\sigma$-コンパクトでもあるならば,相対コンパクトな開集合の増大列$(U_n)$であって,$\o{U_n}\subset U_{n+1}$かつ$\cup_{n\in\N}U_n=X$を満たすものが存在する.
\end{proposition}

\subsection{パラコンパクトハウスドルフ空間は従属する1の分割を持つ}

\begin{tcolorbox}[colframe=ForestGreen, colback=ForestGreen!10!white,breakable,colbacktitle=ForestGreen!40!white,coltitle=black,fonttitle=\bfseries\sffamily,
title=]
    CW-複体はパラコンパクトハウスドルフ空間である.
\end{tcolorbox}

\begin{proposition}[パラコンパクトならば,任意の開被覆は1の分割を持つ (AC)]
    $(X,\O_X)$を位相空間とする.次の2条件は同値.
    \begin{enumerate}
        \item $X$はパラコンパクトなHausdorff空間である.
        \item 任意の開被覆は従属する1の分割を持つ.
    \end{enumerate}
\end{proposition}

\begin{theorem}
    第2可算な
    $C^r\;(r\in[1,\infty])$-多様体$M$の局所有限かつ相対コンパクトな開被覆$(O_\al)_{\al\in A}$について,これに属する1の分解$\{f_\al\}_{\al\in A}\subset C^r(M)$で,各$f_\al$が$M$上$C^r$級であるものが存在する.
    \begin{enumerate}
        \item $\{f_\al\}_{\al\in A}\subset C^r(M)$.
        \item $0\le f_\al\le 1$.
        \item $\supp f_\al=\o{x\in M\mid f_\al(x)\ne 0}\subset\O_\al$.
        \item $\sum_{\al\in A}f_\al=1$.
    \end{enumerate}
\end{theorem}

\section{分離公理}

\subsection{Kolmogorov}

\begin{definition}
    空間$X$が$T_0$または\textbf{Kolmogorov}であるとは,$\forall_{x_0\ne x_1\in X}\;\exists_{i\in2}\;\exists_{U\in\O(x_i)}\;x_{1-i}\notin U$を満たすことをいう.
\end{definition}

\subsection{Frechet}

\begin{definition}
    空間$X$が$T_1$または\textbf{Frechet}であるとは,任意の異なる2点に対して,お互いが互いを含まないような開近傍を持つことをいう.
\end{definition}

\begin{proposition}[$T_1$-性の特徴付け]
    位相空間$X$において,次の2条件は同値.
    \begin{enumerate}
        \item $X$は$T_1$である.
        \item $X$の任意の一点集合は閉である.
    \end{enumerate}
\end{proposition}

\begin{proposition}\label{prop-property-of-T1-spaces}
    $T_1$空間において,次の2条件は同値.
    \begin{enumerate}
        \item $x$は$A$の集積点である.
        \item $x$の任意の近傍は,$A$の点を無限個含む.
    \end{enumerate}
\end{proposition}

\subsection{相互関係}

\begin{tcolorbox}[colframe=ForestGreen, colback=ForestGreen!10!white,breakable,colbacktitle=ForestGreen!40!white,coltitle=black,fonttitle=\bfseries\sffamily,
title=]
    次のような理由から,
    Munkres\cite{Munkres}とnLabでは$T_3,T_4$の定義に$T_1$を仮定している.
    が,これは結局Hausdorff性を仮定しているに等しい.
    以降,ここでもこれに従う:$T_3$とは正則Hausdorff空間で,$T_4$とは正規Hausdorff空間である.
\end{tcolorbox}

\begin{proposition}
    正規な$T_1$空間($T_4$とする)ならば正則な$T_1$空間ならば$T_2$である.
\end{proposition}

\subsection{完全正則}

\begin{definition}[perfectly regular (Tychonoff)]
    空間$X$が$T_{3\frac{1}{2}}$または\textbf{完全正則}とは,
    \begin{enumerate}
        \item 距離化可能な空間は完全正規である.
    \end{enumerate}
\end{definition}

\begin{proposition}\mbox{}
    \begin{enumerate}
        \item 局所コンパクトハウスドルフ空間は完全正則である.
        \item $T_4$ならば完全正則である.
        \item 完全正則ならば正則である.
    \end{enumerate}
\end{proposition}

\subsection{分離性の遺伝}

\begin{proposition}\mbox{}
    \begin{enumerate}
        \item Frechet, Hausdorff, Vietoris (正則), Tychonoff (完全正則)の公理は,いずれも遺伝的である(部分空間に遺伝する).
        \item 同様の4公理は,任意の積空間に伝播する.
    \end{enumerate}
\end{proposition}

\subsection{全正規}

\begin{definition}[completely normal]
    空間$X$が
    \begin{enumerate}
        \item \textbf{全正規}とは,任意の分離された集合は,開近傍で分離されることをいう.
        \item 全正規かつHausdorffな空間を$T_5$という.
    \end{enumerate}
\end{definition}

\begin{proposition}
    次の3条件は同値.
    \begin{enumerate}
        \item $X$は全正規である.
        \item $X$の任意の部分集合は正規である.
        \item $X$の任意の開集合は正規である.
        \item $Y,Z\subset X$が$\o{Y}\cap Z=Y\cap\o{Z}=\emptyset$ならば,互いに素な開近傍で分離できる.
    \end{enumerate}
\end{proposition}

\subsection{完全正規}

\begin{definition}[perfectly normal, perfectly normal Hausdorff]
    空間$X$が
    \begin{enumerate}
        \item \textbf{完全正規}とは,
        任意の互いに素な閉集合が,関数でちょうど分離されることをいう.
        \item 完全正規かつHausdorffな空間を$T_6$という.
    \end{enumerate}
\end{definition}

\begin{proposition}
    完全正規ならば,全正規である.
\end{proposition}

\begin{proposition}
    正規空間$X$について,次の2条件は同値.
    \begin{enumerate}
        \item $T_6$である.
        \item 任意の閉集合は$G_\delta$集合である.
    \end{enumerate}
\end{proposition}

\begin{proposition}
    距離化可能な空間は$T_6$である.
\end{proposition}

\chapter{写像空間}

\section{逆写像定理}

\subsection{Jacobi行列}

\begin{definition}[Jacobi行列]\mbox{}
    \begin{enumerate}
        \item $C^r$級同相写像は,行列を定める:$J:C^r(U,V)\times U\to M_{np}(\R);(f,x)\mapsto J(f)(x)$.
        \item 各点$x\in U$に対し,$J:\Diff^r(U)\to\GL_n(\R)$は群準同型を定める.
        \item $\rank_xf:=\rank(J(f)(x))$と定める.
    \end{enumerate}
\end{definition}

\begin{lemma}
    自然な写像$\Phi(\varphi,\psi):C^r(U,V)\to C^r(U_1,V_1)$について,
    \[\forall_{x\in U}\;\rank_{\varphi(x)}\Phi(\varphi,\psi)f=\rank_xf.\]
\end{lemma}

\subsection{逆写像定理}

\begin{tcolorbox}[colframe=ForestGreen, colback=ForestGreen!10!white,breakable,colbacktitle=ForestGreen!40!white,coltitle=black,fonttitle=\bfseries\sffamily,
title=]
    Euclid空間の射には,局所的には,奇妙な可逆性の特徴付けがある.
    この局所性が多様体の発想となる.
    可微分写像は線型写像の空間に写される事実を使えば,ランクの半連続性から直ちに従う消息である.

    逆写像定理も陰関数定理も,証明には縮小写像の原理を用いてみる.
    距離空間のこの構造が情報の根源であると信じる.

    複素関数論の論理展開がその特殊化であるという見方が肝要.
    それぞれの可逆性の特徴付けがある.
\end{tcolorbox}

\begin{definition}[局所同相]
    次の2条件を満たす写像$\varphi:\R^n\to\R^n$を,\textbf{原点の周りの$C^r$級の局所同相写像}という.
    \begin{enumerate}
        \item $\exists_{U\in\O(0)}\;\varphi\in C^r(U,\R^n),\varphi(0)=0$.
        \item $\exists_{W\in\O(0)}\;W\subset U\land \varphi|_W\in\Diff^r(W,\varphi(W))$.
    \end{enumerate}
\end{definition}

\begin{theorem}[逆写像定理]
    $f$を$\R^n$の原点の近傍で定義された$C^r\;(r\in[1,\infty])$級写像$f:U\to\R^n$で,$f(0)=0$とする.
    この時,次の2条件は同値.
    \begin{enumerate}
        \item $f$は原点の周りで$C^r$級局所同相.
        \item $\rank_0f=n$.
    \end{enumerate}
\end{theorem}
\begin{proof}\mbox{}
    \begin{description}
        \item[方針] 
        (2)$\Rightarrow$(1)を示せば良い.
        $A:=J(f)(0)\in M_n(\R)$とおくと,$\rank A=n$である.
        この時,$A\times$は線型な可微分同相写像$\R^n\iso\R^n$を定めるから,$\o{f}:=A^{-1}\circ f$も$\o{f}(0)=0$を満たす$C^r$級の写像$U\to\R^n$となり,追加で$J(\o{f})(0)=I_n$を満たす.
        これについて(1)を導けば,一般の$f=A\circ\o{f}$についても,原点の周りの$C^r$級の局所同相写像となる.
        \item[可逆射の構成] \mbox{}\\
        \begin{enumerate}
            \item まず,原点の近傍で定まる$C^r$級の写像$g:U\to\R^n,g(x):=f(x)-x$を考える.
            これは$f$による移動の変位ベクトルを表す.
            以下,$\R^n$の距離$d$をノルム$\abs{x}:=\max_{1\le i\le n}\abs{x_i}$が定めるものとする.

            $g$は特に$C^1$級で($g'$が距離$d$に関して連続で),仮定より$g'(0)=1-1=0$であるから,$\ep>0$が存在して,任意の$i,j\in[n]$に対し,$\o{U_\ep(0)}\subset (Dg)^{-1}(U_{1/2n}(0))$,すなわち
            \[\Abs{\pp{g_i}{x_j}}<\frac{1}{2n}\quad\on\o{U_\ep(0)}.\]
            するとこの範囲では,
            \begin{align*}
                \forall_{x,x'\in U_\ep(0)}\;\exists_{\theta\in(0,1)}\quad\abs{g_i(x)-g_i(x')}&=\Abs{\sum^n_{i=1}\pp{g_i}{x_j}(x+\theta(x'-x))(x_j-x'_j)}&\because 多変数の平均値の定理\\
                &\le n\max_{x\in U_\ep(0),i,j\in[n]}\Abs{\pp{g_i}{x_j}(x)}\abs{x-x'}\le\frac{1}{2}\abs{x-x'}.
            \end{align*}
            と評価できる.
            $i\in[n]$は任意だったから,$\abs{g(x)-g(x')}\le\frac{1}{2}\abs{x-x'}$.
            特に,$g(x)\le\frac{\abs{x}}{2}$.
            ここで,$g:U_\ep(0)\to\R^n$は,Lipschitz係数$\frac{1}{2}$の縮小写像であると言いたいが,値域についてはまだよくわからない.
            同様に,$x,f(x),f(f(x)),\cdots$という点列の挙動も不明である.そこで,$g(x)\le\frac{\abs{x}}{2}$を用いて,値域mの方を先に制限することを考える.
            \item そこで,$U_{\ep/2}(0)$内の点$y$に至る点$\o{x}$を逆算する.
            \[\xymatrix@R-2pc{
                h:U_{\ep/2}(0)\ar[r]&C^r(U,\R^n)\\
                \rotatebox[origin=c]{90}{$\in$}&\rotatebox[origin=c]{90}{$\in$}\\
                y\ar@{|->}[r]&h_y(x):=y-g(x)=y+x-f(x)
            }\]
            を考えると,$h_y$は実は$h_y:U_\ep(0)\to U_\ep(0)$(また$\o{U_\ep(0)}\to U_\ep(0)$)を定める.
            実際,
            \begin{align*}
                \forall_{x\in U_\ep(0)}\quad\abs{h_y(x)}&\le\abs{y}+\abs{g(x)}\\
                &\le\abs{y}+\frac{\abs{x}}{2}<\ep.
            \end{align*}
            そして,$h_y$は完備距離空間$(\o{U_{\ep}(0)},d)$上の縮小写像である:
            \begin{align*}
                \forall_{x,x'\in U_\ep(0)}\quad\abs{h_y(x)-h_y(x')}=\abs{g(x)-g(x')}\le\frac{1}{2}\abs{x-x'}.
            \end{align*}
            したがって,不動点$\o{x}=h_y(\o{x})\in\o{U}_\ep(0)$がただ一つ存在する.

            さらに,$y\in U_{\ep/2}(0)$だから,$\o{x}\in U_\ep(0)$である.
            実際,$x=\o{x},x'=0$を考えると,$\abs{h_y(\o{x})-y}=\abs{\o{x}-y}\le\frac{\abs{\o{x}}}{2}$より,$\o{x}\le 2\abs{y}<\ep$が従う.
            \item 以上より,$W:=f^{-1}(U_{\ep/2}(0))\cap U_\ep(0)$とおくと,$f|_W:W\to U_{\ep/2}(0)$は全単射となる.
            全射性は構成より,単射性は不動点の一意性から従う.$W$は確かに$0$の開近傍となっている.
        \end{enumerate}
    \end{description}
\end{proof}
\begin{remarks}\mbox{}
    \begin{description}
        \item[方針] $A\times$という同型については,hom関手の消息で局所同相性は簡単に伝播するので,これについては無視してよく,仮定$J(f)(0)=I_n$も追加して良い.つまり,原点では微分が$0$でほぼ変化しないことを仮定して示せば良い.
        \item[] 
    \end{description}
\end{remarks}

\subsection{系としての陰関数定理}

\begin{tcolorbox}[colframe=ForestGreen, colback=ForestGreen!10!white,breakable,colbacktitle=ForestGreen!40!white,coltitle=black,fonttitle=\bfseries\sffamily,
title=逆関数定理の系としての陰関数定理]
    逆写像定理は同じ次元の間のEuclid空間の射の局所的可逆性を言った.
    これの系としての陰関数定理は,次元が違えど,低い方の次元に揃えれば局所的に「可逆」であることを言っていると捉えられる.
\end{tcolorbox}

\begin{corollary}
    $f\in C^r(U,V)\;(U,V\subset\R^n)$とする.このとき,次の2条件は同値.
    \begin{enumerate}
        \item $f$は可逆である:$f\in\Diff^r(U,V)$.
        \item $f$は全単射で,$U$の各点$x\in U$で$J(f)(x)$が正則.
    \end{enumerate}
\end{corollary}
\begin{remarks}
    層でやったような,局所的な可逆写像を貼り合わせた感覚.
    しかし複素関数論では,正則性と複素微分可能性が同値であるから,(2)の後半の条件が退化して見えなくなる.
\end{remarks}

\begin{corollary}[陰関数定理]\label{cor-implicit-function}
    $r\in[1,\infty],n\le p$について,$f:U\osub\R^n\to\R^p$を$\R^n$の原点の周りで定義された$C^r$級写像で,$f(0)=0$を満たすものとする.
    このとき,次の2条件は同値.
    \begin{enumerate}
        \item $\R^p$の原点の周りの$C^r$級局所同相写像$\varphi$が存在して,$\varphi\circ f(x_1,\cdots,x_n)=(x_1,\cdots,x_n,0,\cdots,0)$が成り立つ.
        \item $\rank_0f=n$.
    \end{enumerate}
\end{corollary}

\begin{corollary}[低次元への写像の場合]\label{cor-implicit-function-2}
    $r\in[1,\infty],n\ge p$について,$f:U\osub\R^n\to\R^p$を$\R^n$の原点の周りで定義された$C^r$級写像で,$f(0)=0$を満たすものとする.
    このとき,次の2条件は同値.
    \begin{enumerate}
        \item $\R^n$の原点の周りの$C^r$級局所同相写像$\psi$が存在して,$f\circ\psi(x_1,\cdots,x_n)=(x_1,\cdots,x_p)$が成り立つ.
        \item $\rank_0f=p$.
    \end{enumerate}
\end{corollary}


\subsection{陰関数定理}

\begin{tcolorbox}[colframe=ForestGreen, colback=ForestGreen!10!white,breakable,colbacktitle=ForestGreen!40!white,coltitle=black,fonttitle=\bfseries\sffamily,
title=]
    写像空間の縮小写像の不動点として,陰関数や常微分方程式の解が得られる.
\end{tcolorbox}

\begin{quotation}
    一般的な位相空間論を準備し,距離空間での例を見て,さらにこれを特徴づける位相的構造を見てきた.
    これで,位相空間論の発展を促した源流である,関数空間に取り組める.
\end{quotation}

\section{写像空間の各点収束位相}

\begin{tcolorbox}[colframe=ForestGreen, colback=ForestGreen!10!white,breakable,colbacktitle=ForestGreen!40!white,coltitle=black,fonttitle=\bfseries\sffamily,
title=]
    各点収束位相とは,積位相である.
    これはノルム化どころか,距離化可能でない.
\end{tcolorbox}

\subsection{準基の表示}

\begin{proposition}
    $\Map([0,1],\R)$上に,
    \[U(f;x_1,\cdots,x_n;\ep):=\Brace{g\in\Map([0,1];\R)\mid\forall_{i\in[n]}\;\abs{g(x_i)-f(x_i)}<\ep}\quad(n\in\N,x_1,\cdots,x_n\in[0,1],\ep>0)\]
    で定まる集合族が生成する位相は,$\R^{[0,1]}$の積位相である.
\end{proposition}

\subsection{一般の関数に対して稠密}

\begin{tcolorbox}[colframe=ForestGreen, colback=ForestGreen!10!white,breakable,colbacktitle=ForestGreen!40!white,coltitle=black,fonttitle=\bfseries\sffamily,
title=]
    任意の関数は,連続関数の各点収束極限として表せる.
\end{tcolorbox}

\begin{proposition}
    $\Map([0,1],\R)$上で$C([0,1])$は稠密である.
\end{proposition}

\section{$C$の一様ノルム位相}

\begin{tcolorbox}[colframe=ForestGreen, colback=ForestGreen!10!white,breakable,colbacktitle=ForestGreen!40!white,coltitle=black,fonttitle=\bfseries\sffamily,
title=]
    結局はコンパクト開位相であるが,コンパクト空間$X$上の$C(X)$の一様ノルムを調べる.
\end{tcolorbox}

\subsection{全有界集合:Ascoliの定理}

\begin{tcolorbox}[colframe=ForestGreen, colback=ForestGreen!10!white,breakable,colbacktitle=ForestGreen!40!white,coltitle=black,fonttitle=\bfseries\sffamily,
title=]
    完備距離空間の閉部分集合が,全有界であることとコンパクトであることは同値であることに注意.

    Heine-Borelの定理は$\R^n$のコンパクト集合は有界閉集合だと確定した.
    $C(\R^n)$のコンパクト集合は,各点有界同程度連続閉集合である.
\end{tcolorbox}

\begin{theorem}
    $X$をコンパクト集合,$C(X;\C)$を一様ノルムによるBanach空間とする.
    部分集合$\Phi\subset C(X)$が次の2条件を満たすならば,全有界である:
    \begin{enumerate}
        \item 各点有界:$\forall_{x\in X}\;\sup_{f\in\Phi}\abs{f(x)}<\infty$.すなわち,任意の$x\in X$について$\{f(x)\}_{f\in\Phi}\subset\R$は相対コンパクト.
        \item 同程度連続:$\forall_{\ep>0}\;\forall_{x\in X}\;\exists_{V\in\O(x)}\;\forall_{y\in V}\;\forall_{f\in\Phi}\abs{f(y)-f(x)}<\ep$.
    \end{enumerate}
\end{theorem}

\begin{corollary}
    $\Phi$が上の条件を満たすならば,特に相対コンパクトであるから,任意の$\Phi$の列は一様収束する部分列を持つ.
\end{corollary}

\subsection{同程度連続性}

\begin{tcolorbox}[colframe=ForestGreen, colback=ForestGreen!10!white,breakable,colbacktitle=ForestGreen!40!white,coltitle=black,fonttitle=\bfseries\sffamily,
title=]
    同程度連続な関数族の集積点は$C(X)$に入る.
\end{tcolorbox}

\begin{definition}
    族$\Phi\subset\Map(X,\R)$について,
    \begin{enumerate}
        \item \textbf{同程度連続}であるとは,$\forall_{\ep>0}\;\forall_{x\in X}\;\exists_{V\in\O(x)}\;\forall_{y\in V}\;\forall_{f\in\Phi}\abs{f(y)-f(x)}<\ep$を満たすことをいう.
        \item \textbf{一様に同程度連続}であるとは,$\forall_{\ep>0}\;\exists_{V\in\O(x)}\;\forall_{x,y\in V}\;\forall_{f\in\Phi}\abs{f(y)-f(x)}<\ep$を満たすことをいう.
        \item 連続であるとは,$\forall_{\ep>0}\;\forall_{x\in X}\;\forall_{f\in\Phi}\;\exists_{V\in\O(x)}\;\forall_{y\in V}\;\abs{f(y)-f(x)}<\ep$.
    \end{enumerate}
    $X$がコンパクトであるとき,同程度連続性と一様同程度連続性とは一致する.
\end{definition}
\begin{remarks}
    $\Phi$が一点集合であえるとき,同程度連続性と連続性は一致する.
\end{remarks}

\begin{theorem}[Cauchy sum theorem]
    $\{f_n\}\subset\Map(\R,\R)$は同程度連続であるとする.
    $X$の稠密部分集合上で$(f_n)$が各点収束するならば,$X$上で収束し,極限関数は連続である.
\end{theorem}
\begin{history}
    Cauchy, A. (1821). \textit{Cours d'Analyse}は,連続関数の各点収束極限は連続なる間違った(表現をした)命題を載せていた.
\end{history}

\begin{proposition}
    $X$をコンパクト空間とする.
    この上の同程度連続な関数列が各点収束するならば,一様収束する.
\end{proposition}

\subsection{Dini}

\begin{lemma}
    コンパクト空間$X$上の連続関数$C(X)$のネット$(f_\lambda)$は,単調であるとする:$\lambda\le\mu\Rightarrow f_\lambda\le f_\mu$.
    このネットがある連続関数に各点収束するならば,一様収束する.
\end{lemma}

\section{$C$のコンパクト開位相}

\begin{tcolorbox}[colframe=ForestGreen, colback=ForestGreen!10!white,breakable,colbacktitle=ForestGreen!40!white,coltitle=black,fonttitle=\bfseries\sffamily,
title=]
    $X$がHausdorffでなくとも,局所コンパクトならば,$C(X,Y)$はTopの指数対象となる.
    $X$が更にHausdorffならば,位相同型$\Map(X,\Map(Y,Z))\simeq\Map(X\times Y,Z)$が得られる.
\end{tcolorbox}

\subsection{定義}

\begin{definition}[compact-open topology]
    $(M_{A,U})_{A\in\cC(X),U\in\O(Y)}$を準基として生成される$C(X;Y)$の位相$\O_{Y^X}$を\textbf{コンパクト開位相}という:
    \[M_{A,U}:=\Brace{f\in C(X;Y)\mid f(A)\subset U}\quad(A\in\cC(X),U\in\O_Y)\]
\end{definition}

\begin{definition}[exponential opject]
    有限積を持つ圏$C$において,\textbf{指数対象}とは,組$(X^Y,\ev:X^Y\times Y\to X)$であって,次の普遍性を満たすものをいう:任意の$Z\in C$と$e:Z\times Y\to X$について,唯一の射$u:Z\to X^Y$が存在して,$e=\ev\circ (u\times\id_Y)$.
\end{definition}

\subsection{値写像の連続性}

\begin{proposition}
    $X$を局所コンパクト,$Y$を一般の位相空間とする.
    $\ev:X\times C(X;Y)\to Y$は連続.
\end{proposition}

\subsection{随伴が連続}

\begin{proposition}
    $X,Y,Z$を位相空間とする.(1)$\Rightarrow$(2)が成り立つ.$Y$が局所コンパクトならば(2)$\Rightarrow$(1)も成り立つ.
    \begin{enumerate}
        \item $f:X\times Y\to Z$は連続である.
        \item $\al_f:X\to Z^Y$は連続である.
    \end{enumerate}
\end{proposition}

\subsection{距離空間では広義一様収束位相と一致}

\begin{tcolorbox}[colframe=ForestGreen, colback=ForestGreen!10!white,breakable,colbacktitle=ForestGreen!40!white,coltitle=black,fonttitle=\bfseries\sffamily,
title=]
    Wiener空間で見られた減少である.
\end{tcolorbox}

\begin{proposition}
    $Y$を距離空間とすると,局所コンパクトハウスドルフである.
    このとき,$\Map(X,Y)$上のコンパクト開位相とは,広義一様収束の位相に等しい.
\end{proposition}

\subsection{コンパクトになるとき}

\begin{proposition}
    $X,Y$は第2可算な$T_3$-空間で,$X$が局所コンパクトならば,$Y^X$は第2可算である.
\end{proposition}

\subsection{有界連続関数の空間}

\begin{tcolorbox}[colframe=ForestGreen, colback=ForestGreen!10!white,breakable,colbacktitle=ForestGreen!40!white,coltitle=black,fonttitle=\bfseries\sffamily,
title=]
    $C_b(X)$に限ると,一様収束位相にも一致する.
\end{tcolorbox}

\begin{proposition}
    $X$を位相空間,$Y$を距離空間とする.
    \begin{enumerate}
        \item 写像$D:\Map(X,Y)\to\Map(X\times Y;\R)$を,$D(f):=D_f(x,y)=d(f(x),y)$で定める.するとこれは単射であり,$\norm{D_f-D_g}_\infty=\sup_{x\in X}d(f(x),g(x))$は$l^\infty(X;Y)$に距離を定める.$f\in C_b(X;Y)$と$D_f$が連続であることとは同値.
        \item $C_b(X;Y)$は$l^\infty(X;Y)$の閉部分空間となる.
        \item $e:C_b(X;Y)\times X\to Y$は$C_b$の一様位相と積位相について連続になる.
    \end{enumerate}
\end{proposition}

\subsection{一様収束の性質}

\begin{proposition}
    位相空間$X$上の連続関数列$\{g_n\}\subset C(X)$は一様収束極限$g$を持つならば,$g\in C(X)$である.
\end{proposition}

\subsection{一様収束の判定法}

\begin{proposition}[一様収束の判定法]
    $(f_n)$を位相空間$X$上の関数の列で,各点収束極限
    $f$を持つとする.
    \begin{enumerate}
        \item $(f_n)$は一様収束する.
        \item (Cauchy criterion) $\forall_{\ep>0}\;\exists_{n_0\in\N}\;\forall_{m,n\ge n_0}\;\forall_{x\in E}\;\abs{f_n(x)-f_m(x)}<\ep$.
        \item $\norm{f_n-f}_\infty\to0$.
    \end{enumerate}
\end{proposition}

\begin{proposition}[Weierstrass $M$-test]
    位相空間$X$上の連続関数列$\{f_n\}\subset C(X)$は収束する優級数を定める列$\{M_n\}\subset\R$を持つとする:$\forall_{n\in\N}\;\norm{f_n}_\infty\le M_n,\sum_{n\in\N}M_n\in\R$.
    このとき,級数列$\sum{i=1}^nf_i$は一様収束する.
\end{proposition}

\section{連続線型写像の空間}

\subsection{Banach-Steinhausの定理}

\begin{proposition}[同程度連続性の特徴付け]
    $X,Y$を位相線形空間,
    $\Gamma\subset\Hom_\R(X,Y)$を線形写像の集合とする.
    \begin{enumerate}
        \item $\Gamma$は同程度連続である.
        \item $\forall_{W\in\O_Y(0)}\;\exists_{V\in\O_X(0)}\;\forall_{\Lambda\in\Gamma}\;\Lambda(V)\subset W$.
    \end{enumerate}
\end{proposition}

\begin{theorem}
    $\Gamma\subset\Hom_\R(X,Y)$を同程度連続,$E\subset X$を有界集合とする.
    このとき,ある有界集合$F\subset Y$が存在して,$\forall_{\Lambda\in\Gamma}\;\Lambda(E)\subset F$.
\end{theorem}

\begin{theorem}[Banach-Steinhaus]
    $\Gamma\subset B(X,Y)$を集合,$B:=\Brace{x\in X\mid\Gamma(x):=\Brace{\Lambda x\mid\Lambda\in\Gamma}\subset Y\text{は有界}}$とする.
    $B$は$X$上第2類ならば,$B=X$かつ$\Gamma$は同程度連続である.
\end{theorem}

\begin{proposition}
    $E\subset B(X,Y)$について,次の2条件は同値.
    \begin{enumerate}
        \item $E$は同程度連続である.
        \item $\exists_{M<\infty}\;\forall_{\Lambda\in E}\;\norm{\Lambda}\le M$.
    \end{enumerate}
\end{proposition}

\chapter{参考文献}

\begin{thebibliography}{99}
    \bibitem{斎藤}
    斎藤毅『集合と位相』(東京大学出版会.2016)
    \bibitem{彌永}
    彌永昌吉・彌永健一『集合と位相II』(岩波講座 基礎数学9,1977).
    \bibitem{森田}
    森田紀一『位相空間論』(
    \bibitem{Munkers}
    Munkres, James. (1999). \textit{Topology}.
    \bibitem{Johnstone}
        Peter Johnstone - The point of pointless topology (83)
    \bibitem{Johnstone2}
        Peter Johnstone, Stone Spaces, Cambridge Studies in Advanced Mathematics 3, Cambridge University Press 1982. xxi+370 pp. MR85f:54002, reprinted 1986.
    \bibitem{nLab}
        \url{https://ncatlab.org/nlab/show/Introduction+to+Topology}
    \bibitem{Vickers}
        Steven Vickers, Topology via Logic, Cambridge University Press (1989)
    \bibitem{Hausdorff}
    F. Hausdorff, Grundzüge der Mengenlehre, Veit \& Co., Leipzig, 1914.
    \bibitem{Bourbaki}
    Nicolas Bourbaki 位相I
    \bibitem{Analysis Now}
    Analysis Now
\end{thebibliography}

\end{document}