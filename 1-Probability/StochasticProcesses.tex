\documentclass[uplatex,dvipdfmx]{jsreport}
\title{確率過程とその逆問題}
\author{司馬博文}
\date{\today}
\pagestyle{headings} \setcounter{secnumdepth}{4}
\usepackage{mathtools}
%\mathtoolsset{showonlyrefs=true} %labelを附した数式にのみ附番される設定.
%\usepackage{amsmath} %mathtoolsの内部で呼ばれるので要らない.
\usepackage{amsfonts} %mathfrak, mathcal, mathbbなど.
\usepackage{amsthm} %定理環境.
\usepackage{amssymb} %AMSFontsを使うためのパッケージ.
\usepackage{ascmac} %screen, itembox, shadebox環境.全てLATEX2εの標準機能の範囲で作られたもの.
\usepackage{comment} %comment環境を用いて,複数行をcomment outできるようにするpackage
\usepackage{wrapfig} %図の周りに文字をwrapさせることができる.詳細な制御ができる.
\usepackage[usenames, dvipsnames]{xcolor} %xcolorはcolorの拡張.optionの意味はdvipsnamesはLoad a set of predefined colors. forestgreenなどの色が追加されている.usenamesはobsoleteとだけ書いてあった.
\setcounter{tocdepth}{2} %目次に表示される深さ.2はsubsectionまで
\usepackage{multicol} %\begin{multicols}{2}環境で途中からmulticolumnに出来る.

\usepackage{url}
\usepackage[dvipdfmx,colorlinks,linkcolor=blue,urlcolor=blue]{hyperref} %生成されるPDFファイルにおいて、\tableofcontentsによって書き出された目次をクリックすると該当する見出しへジャンプしたり、さらには、\label{ラベル名}を番号で参照する\ref{ラベル名}やthebibliography環境において\bibitem{ラベル名}を文献番号で参照する\cite{ラベル名}においても番号をクリックすると該当箇所にジャンプする.囲み枠はダサいので,colorlinksで囲み廃止し,リンク自体に色を付けることにした.
\usepackage{pxjahyper} %pxrubrica同様,八登崇之さん.hyperrefは日本語pLaTeXに最適化されていないから,hyperrefとセットで,(u)pLaTeX+hyperref+dvipdfmxの組み合わせで日本語を含む「しおり」をもつPDF文書を作成する場合に必要となる機能を提供する
\definecolor{花緑青}{cmyk}{0.52,0.03,0,0.27}
\definecolor{サーモンピンク}{cmyk}{0,0.65,0.65,0.05}
\definecolor{暗中模索}{rgb}{0.2,0.2,0.2}

\usepackage{tikz}
\usetikzlibrary{positioning,automata} %automaton描画のため
\usepackage{tikz-cd}
\usepackage[all]{xy}
\def\objectstyle{\displaystyle} %デフォルトではxymatrix中の数式が文中数式モードになるので,それを直す.\labelstyleも同様にxy packageの中で定義されており,文中数式モードになっている.

\usepackage[version=4]{mhchem} %化学式をTikZで簡単に書くためのパッケージ.
\usepackage{chemfig} %化学構造式をTikZで描くためのパッケージ.
\usepackage{siunitx} %IS単位を書くためのパッケージ

\usepackage{ulem} %取り消し線を引くためのパッケージ
\usepackage{pxrubrica} %日本語にルビをふる.八登崇之(やとうたかゆき)氏による.

\usepackage{graphicx} %rotatebox, scalebox, reflectbox, resizeboxなどのコマンドや,図表の読み込み\includegraphicsを司る.graphics というパッケージもありますが,graphicx はこれを高機能にしたものと考えて結構です(ただし graphicx は内部で graphics を読み込みます)

\usepackage[breakable]{tcolorbox} %加藤晃史さんがフル活用していたtcolorboxを,途中改ページ可能で.
\tcbuselibrary{theorems} %https://qiita.com/t_kemmochi/items/483b8fcdb5db8d1f5d5e
\usepackage{enumerate} %enumerate環境を凝らせる.
\usepackage[top=15truemm,bottom=15truemm,left=10truemm,right=10truemm]{geometry} %足助さんからもらったオプション

%%%%%%%%%%%%%%% 環境マクロ %%%%%%%%%%%%%%%

\usepackage{listings} %ソースコードを表示できる環境.多分もっといい方法ある.
\usepackage{jvlisting} %日本語のコメントアウトをする場合jlistingが必要
\lstset{ %ここからソースコードの表示に関する設定.lstlisting環境では,[caption=hoge,label=fuga]などのoptionを付けられる.
%[escapechar=!]とすると,LaTeXコマンドを使える.
  basicstyle={\ttfamily},
  identifierstyle={\small},
  commentstyle={\smallitshape},
  keywordstyle={\small\bfseries},
  ndkeywordstyle={\small},
  stringstyle={\small\ttfamily},
  frame={tb},
  breaklines=true,
  columns=[l]{fullflexible},
  numbers=left,
  xrightmargin=0zw,
  xleftmargin=3zw,
  numberstyle={\scriptsize},
  stepnumber=1,
  numbersep=1zw,
  lineskip=-0.5ex
}
%\makeatletter %caption番号を「[chapter番号].[section番号].[subsection番号]-[そのsubsection内においてn番目]」に変更
%    \AtBeginDocument{
%    \renewcommand*{\thelstlisting}{\arabic{chapter}.\arabic{section}.\arabic{lstlisting}}
%    \@addtoreset{lstlisting}{section}
%    }
%\makeatother
\renewcommand{\lstlistingname}{算譜} %caption名を"program"に変更

\newtcolorbox{tbox}[3][]{%
colframe=#2,colback=#2!10,coltitle=#2!20!black,title={#3},#1}

%%%%%%%%%%%%%%% フォント %%%%%%%%%%%%%%%

\usepackage{textcomp, mathcomp} %Text Companionとは,T1 encodingに入らなかった文字群.これを使うためのパッケージ.\textsectionでブルバキに!
\usepackage[T1]{fontenc} %8bitエンコーディングにする.comp系拡張数学文字の動作が安定する.

%%%%%%%%%%%%%%% 数学記号のマクロ %%%%%%%%%%%%%%%

\newcommand{\abs}[1]{\lvert#1\rvert} %mathtoolsはこうやって使うのか!
\newcommand{\Abs}[1]{\left|#1\right|}
\newcommand{\norm}[1]{\|#1\|}
\newcommand{\Norm}[1]{\left\|#1\right\|}
%\newcommand{\brace}[1]{\{#1\}}
\newcommand{\Brace}[1]{\left\{#1\right\}}
\newcommand{\paren}[1]{\left(#1\right)}
\newcommand{\bracket}[1]{\langle#1\rangle}
\newcommand{\brac}[1]{\langle#1\rangle}
\newcommand{\Bracket}[1]{\left\langle#1\right\rangle}
\newcommand{\Brac}[1]{\left\langle#1\right\rangle}
\newcommand{\Square}[1]{\left[#1\right]}
\renewcommand{\o}[1]{\overline{#1}}
\renewcommand{\u}[1]{\underline{#1}}
\renewcommand{\iff}{\;\mathrm{iff}\;} %nLabリスペクト
\newcommand{\pp}[2]{\frac{\partial #1}{\partial #2}}
\newcommand{\ppp}[3]{\frac{\partial #1}{\partial #2\partial #3}}
\newcommand{\dd}[2]{\frac{d #1}{d #2}}
\newcommand{\floor}[1]{\lfloor#1\rfloor}
\newcommand{\Floor}[1]{\left\lfloor#1\right\rfloor}
\newcommand{\ceil}[1]{\lceil#1\rceil}

\newcommand{\iso}{\xrightarrow{\,\smash{\raisebox{-0.45ex}{\ensuremath{\scriptstyle\sim}}}\,}}
\newcommand{\wt}[1]{\widetilde{#1}}
\newcommand{\wh}[1]{\widehat{#1}}

\newcommand{\Lrarrow}{\;\;\Leftrightarrow\;\;}

%ノルム位相についての閉包 https://newbedev.com/how-to-make-double-overline-with-less-vertical-displacement
\makeatletter
\newcommand{\dbloverline}[1]{\overline{\dbl@overline{#1}}}
\newcommand{\dbl@overline}[1]{\mathpalette\dbl@@overline{#1}}
\newcommand{\dbl@@overline}[2]{%
  \begingroup
  \sbox\z@{$\m@th#1\overline{#2}$}%
  \ht\z@=\dimexpr\ht\z@-2\dbl@adjust{#1}\relax
  \box\z@
  \ifx#1\scriptstyle\kern-\scriptspace\else
  \ifx#1\scriptscriptstyle\kern-\scriptspace\fi\fi
  \endgroup
}
\newcommand{\dbl@adjust}[1]{%
  \fontdimen8
  \ifx#1\displaystyle\textfont\else
  \ifx#1\textstyle\textfont\else
  \ifx#1\scriptstyle\scriptfont\else
  \scriptscriptfont\fi\fi\fi 3
}
\makeatother
\newcommand{\oo}[1]{\dbloverline{#1}}

\DeclareMathOperator{\grad}{\mathrm{grad}}
\DeclareMathOperator{\rot}{\mathrm{rot}}
\DeclareMathOperator{\divergence}{\mathrm{div}}
\newcommand{\False}{\mathrm{False}}
\newcommand{\True}{\mathrm{True}}
\DeclareMathOperator{\tr}{\mathrm{tr}}
\newcommand{\M}{\mathcal{M}}
\newcommand{\cF}{\mathcal{F}}
\newcommand{\cD}{\mathcal{D}}
\newcommand{\fX}{\mathfrak{X}}
\newcommand{\fY}{\mathfrak{Y}}
\newcommand{\fZ}{\mathfrak{Z}}
\renewcommand{\H}{\mathcal{H}}
\newcommand{\fH}{\mathfrak{H}}
\newcommand{\bH}{\mathbb{H}}
\newcommand{\id}{\mathrm{id}}
\newcommand{\A}{\mathcal{A}}
% \renewcommand\coprod{\rotatebox[origin=c]{180}{$\prod$}} すでにどこかにある.
\newcommand{\pr}{\mathrm{pr}}
\newcommand{\U}{\mathfrak{U}}
\newcommand{\Map}{\mathrm{Map}}
\newcommand{\dom}{\mathrm{Dom}\;}
\newcommand{\cod}{\mathrm{Cod}\;}
\newcommand{\supp}{\mathrm{supp}\;}
\newcommand{\otherwise}{\mathrm{otherwise}}
\newcommand{\st}{\;\mathrm{s.t.}\;}
\newcommand{\lmd}{\lambda}
\newcommand{\Lmd}{\Lambda}
%%% 線型代数学
\newcommand{\Ker}{\mathrm{Ker}\;}
\newcommand{\Coker}{\mathrm{Coker}\;}
\newcommand{\Coim}{\mathrm{Coim}\;}
\newcommand{\rank}{\mathrm{rank}}
\newcommand{\lcm}{\mathrm{lcm}}
\newcommand{\sgn}{\mathrm{sgn}}
\newcommand{\GL}{\mathrm{GL}}
\newcommand{\SL}{\mathrm{SL}}
\newcommand{\alt}{\mathrm{alt}}
%%% 複素解析学
\renewcommand{\Re}{\mathrm{Re}\;}
\renewcommand{\Im}{\mathrm{Im}\;}
\newcommand{\Gal}{\mathrm{Gal}}
\newcommand{\PGL}{\mathrm{PGL}}
\newcommand{\PSL}{\mathrm{PSL}}
\newcommand{\Log}{\mathrm{Log}\,}
\newcommand{\Res}{\mathrm{Res}\,}
\newcommand{\on}{\mathrm{on}\;}
\newcommand{\hatC}{\hat{\C}}
\newcommand{\hatR}{\hat{\R}}
\newcommand{\PV}{\mathrm{P.V.}}
\newcommand{\diam}{\mathrm{diam}}
\newcommand{\Area}{\mathrm{Area}}
\newcommand{\Lap}{\Laplace}
\newcommand{\f}{\mathbf{f}}
\newcommand{\cR}{\mathcal{R}}
\newcommand{\const}{\mathrm{const.}}
\newcommand{\Om}{\Omega}
\newcommand{\Cinf}{C^\infty}
\newcommand{\ep}{\epsilon}
\newcommand{\dist}{\mathrm{dist}}
\newcommand{\opart}{\o{\partial}}
%%% 解析力学
\newcommand{\x}{\mathbf{x}}
%%% 集合と位相
\renewcommand{\O}{\mathcal{O}}
\renewcommand{\S}{\mathcal{S}}
\renewcommand{\U}{\mathcal{U}}
\newcommand{\V}{\mathcal{V}}
\renewcommand{\P}{\mathcal{P}}
\newcommand{\R}{\mathbb{R}}
\newcommand{\N}{\mathbb{N}}
\newcommand{\C}{\mathbb{C}}
\newcommand{\Z}{\mathbb{Z}}
\newcommand{\Q}{\mathbb{Q}}
\newcommand{\TV}{\mathrm{TV}}
\newcommand{\ORD}{\mathrm{ORD}}
\newcommand{\Tr}{\mathrm{Tr}\;}
\newcommand{\Card}{\mathrm{Card}\;}
\newcommand{\Top}{\mathrm{Top}}
\newcommand{\Disc}{\mathrm{Disc}}
\newcommand{\Codisc}{\mathrm{Codisc}}
\newcommand{\CoDisc}{\mathrm{CoDisc}}
\newcommand{\Ult}{\mathrm{Ult}}
\newcommand{\ord}{\mathrm{ord}}
\newcommand{\maj}{\mathrm{maj}}
%%% 形式言語理論
\newcommand{\REGEX}{\mathrm{REGEX}}
\newcommand{\RE}{\mathbf{RE}}

%%% Fourier解析
\newcommand*{\Laplace}{\mathop{}\!\mathbin\bigtriangleup}
\newcommand*{\DAlambert}{\mathop{}\!\mathbin\Box}
%%% Graph Theory
\newcommand{\SimpGph}{\mathrm{SimpGph}}
\newcommand{\Gph}{\mathrm{Gph}}
\newcommand{\mult}{\mathrm{mult}}
\newcommand{\inv}{\mathrm{inv}}
%%% 多様体
\newcommand{\Der}{\mathrm{Der}}
\newcommand{\osub}{\overset{\mathrm{open}}{\subset}}
\newcommand{\osup}{\overset{\mathrm{open}}{\supset}}
\newcommand{\al}{\alpha}
\newcommand{\K}{\mathbb{K}}
\newcommand{\Sp}{\mathrm{Sp}}
\newcommand{\g}{\mathfrak{g}}
\newcommand{\h}{\mathfrak{h}}
\newcommand{\Exp}{\mathrm{Exp}\;}
\newcommand{\Imm}{\mathrm{Imm}}
\newcommand{\Imb}{\mathrm{Imb}}
\newcommand{\codim}{\mathrm{codim}\;}
\newcommand{\Gr}{\mathrm{Gr}}
%%% 代数
\newcommand{\Ad}{\mathrm{Ad}}
\newcommand{\finsupp}{\mathrm{fin\;supp}}
\newcommand{\SO}{\mathrm{SO}}
\newcommand{\SU}{\mathrm{SU}}
\newcommand{\acts}{\curvearrowright}
\newcommand{\mono}{\hookrightarrow}
\newcommand{\epi}{\twoheadrightarrow}
\newcommand{\Stab}{\mathrm{Stab}}
\newcommand{\nor}{\mathrm{nor}}
\newcommand{\T}{\mathbb{T}}
\newcommand{\Aff}{\mathrm{Aff}}
\newcommand{\rsub}{\triangleleft}
\newcommand{\rsup}{\triangleright}
\newcommand{\subgrp}{\overset{\mathrm{subgrp}}{\subset}}
\newcommand{\Ext}{\mathrm{Ext}}
\newcommand{\sbs}{\subset}\newcommand{\sps}{\supset}
\newcommand{\In}{\mathrm{In}}
\newcommand{\Tor}{\mathrm{Tor}}
\newcommand{\p}{\mathfrak{p}}
\newcommand{\q}{\mathfrak{q}}
\newcommand{\m}{\mathfrak{m}}
\newcommand{\cS}{\mathcal{S}}
\newcommand{\Frac}{\mathrm{Frac}\,}
\newcommand{\Spec}{\mathrm{Spec}\,}
\newcommand{\bA}{\mathbb{A}}
\newcommand{\Sym}{\mathrm{Sym}}
\newcommand{\Ann}{\mathrm{Ann}}
%%% 代数的位相幾何学
\newcommand{\Ho}{\mathrm{Ho}}
\newcommand{\CW}{\mathrm{CW}}
\newcommand{\lc}{\mathrm{lc}}
\newcommand{\cg}{\mathrm{cg}}
\newcommand{\Fib}{\mathrm{Fib}}
\newcommand{\Cyl}{\mathrm{Cyl}}
\newcommand{\Ch}{\mathrm{Ch}}
%%% 数値解析
\newcommand{\round}{\mathrm{round}}
\newcommand{\cond}{\mathrm{cond}}
\newcommand{\diag}{\mathrm{diag}}
%%% 確率論
\newcommand{\calF}{\mathcal{F}}
\newcommand{\X}{\mathcal{X}}
\newcommand{\Meas}{\mathrm{Meas}}
\newcommand{\as}{\;\mathrm{a.s.}} %almost surely
\newcommand{\io}{\;\mathrm{i.o.}} %infinitely often
\newcommand{\fe}{\;\mathrm{f.e.}} %with a finite number of exceptions
\newcommand{\F}{\mathcal{F}}
\newcommand{\bF}{\mathbb{F}}
\newcommand{\W}{\mathcal{W}}
\newcommand{\Pois}{\mathrm{Pois}}
\newcommand{\iid}{\mathrm{i.i.d.}}
\newcommand{\wconv}{\rightsquigarrow}
\newcommand{\Var}{\mathrm{Var}}
\newcommand{\xrightarrown}{\xrightarrow{n\to\infty}}
\newcommand{\au}{\mathrm{au}}
\newcommand{\cT}{\mathcal{T}}
%%% 情報理論
\newcommand{\bit}{\mathrm{bit}}
%%% 積分論
\newcommand{\calA}{\mathcal{A}}
\newcommand{\calB}{\mathcal{B}}
\newcommand{\D}{\mathcal{D}}
\newcommand{\Y}{\mathcal{Y}}
\newcommand{\calC}{\mathcal{C}}
\renewcommand{\ae}{\mathrm{a.e.}\;}
\newcommand{\cZ}{\mathcal{Z}}
\newcommand{\fF}{\mathfrak{F}}
\newcommand{\fI}{\mathfrak{I}}
\newcommand{\E}{\mathcal{E}}
\newcommand{\sMap}{\sigma\textrm{-}\mathrm{Map}}
\DeclareMathOperator*{\argmax}{arg\,max}
\DeclareMathOperator*{\argmin}{arg\,min}
\newcommand{\cC}{\mathcal{C}}
\newcommand{\comp}{\complement}
\newcommand{\J}{\mathcal{J}}
\newcommand{\sumN}[1]{\sum_{#1\in\N}}
\newcommand{\cupN}[1]{\cup_{#1\in\N}}
\newcommand{\capN}[1]{\cap_{#1\in\N}}
\newcommand{\Sum}[1]{\sum_{#1=1}^\infty}
\newcommand{\sumn}{\sum_{n=1}^\infty}
\newcommand{\summ}{\sum_{m=1}^\infty}
\newcommand{\sumk}{\sum_{k=1}^\infty}
\newcommand{\sumi}{\sum_{i=1}^\infty}
\newcommand{\sumj}{\sum_{j=1}^\infty}
\newcommand{\cupn}{\cup_{n=1}^\infty}
\newcommand{\capn}{\cap_{n=1}^\infty}
\newcommand{\cupk}{\cup_{k=1}^\infty}
\newcommand{\cupi}{\cup_{i=1}^\infty}
\newcommand{\cupj}{\cup_{j=1}^\infty}
\newcommand{\limn}{\lim_{n\to\infty}}
\renewcommand{\l}{\mathcal{l}}
\renewcommand{\L}{\mathcal{L}}
\newcommand{\Cl}{\mathrm{Cl}}
\newcommand{\cN}{\mathcal{N}}
\newcommand{\Ae}{\textrm{-a.e.}\;}
\newcommand{\csub}{\overset{\textrm{closed}}{\subset}}
\newcommand{\csup}{\overset{\textrm{closed}}{\supset}}
\newcommand{\wB}{\wt{B}}
\newcommand{\cG}{\mathcal{G}}
\newcommand{\Lip}{\mathrm{Lip}}
\newcommand{\Dom}{\mathrm{Dom}}
%%% 数理ファイナンス
\newcommand{\pre}{\mathrm{pre}}
\newcommand{\om}{\omega}

%%% 統計的因果推論
\newcommand{\Do}{\mathrm{Do}}
%%% 数理統計
\newcommand{\bP}{\mathbb{P}}
\newcommand{\compsub}{\overset{\textrm{cpt}}{\subset}}
\newcommand{\lip}{\textrm{lip}}
\newcommand{\BL}{\mathrm{BL}}
\newcommand{\G}{\mathbb{G}}
\newcommand{\NB}{\mathrm{NB}}
\newcommand{\oR}{\o{\R}}
\newcommand{\liminfn}{\liminf_{n\to\infty}}
\newcommand{\limsupn}{\limsup_{n\to\infty}}
%\newcommand{\limn}{\lim_{n\to\infty}}
\newcommand{\esssup}{\mathrm{ess.sup}}
\newcommand{\asto}{\xrightarrow{\as}}
\newcommand{\Cov}{\mathrm{Cov}}
\newcommand{\cQ}{\mathcal{Q}}
\newcommand{\VC}{\mathrm{VC}}
\newcommand{\mb}{\mathrm{mb}}
\newcommand{\Avar}{\mathrm{Avar}}
\newcommand{\bB}{\mathbb{B}}
\newcommand{\bW}{\mathbb{W}}
\newcommand{\sd}{\mathrm{sd}}
\newcommand{\w}[1]{\widehat{#1}}
\newcommand{\bZ}{\mathbb{Z}}
\newcommand{\Bernoulli}{\mathrm{Bernoulli}}
\newcommand{\Mult}{\mathrm{Mult}}
\newcommand{\BPois}{\mathrm{BPois}}
\newcommand{\fraks}{\mathfrak{s}}
\newcommand{\frakk}{\mathfrak{k}}
\newcommand{\IF}{\mathrm{IF}}
\newcommand{\bX}{\mathbf{X}}
\newcommand{\bx}{\mathbf{x}}
\newcommand{\indep}{\raisebox{0.05em}{\rotatebox[origin=c]{90}{$\models$}}}
\newcommand{\IG}{\mathrm{IG}}
\newcommand{\Levy}{\mathrm{Levy}}
\newcommand{\MP}{\mathrm{MP}}
\newcommand{\Hermite}{\mathrm{Hermite}}
\newcommand{\Skellam}{\mathrm{Skellam}}
\newcommand{\Dirichlet}{\mathrm{Dirichlet}}
\newcommand{\Beta}{\mathrm{Beta}}
\newcommand{\bE}{\mathbb{E}}
\newcommand{\bG}{\mathbb{G}}
\newcommand{\MISE}{\mathrm{MISE}}
\newcommand{\logit}{\mathtt{logit}}
\newcommand{\expit}{\mathtt{expit}}
\newcommand{\cK}{\mathcal{K}}
\newcommand{\dl}{\dot{l}}
\newcommand{\dotp}{\dot{p}}
\newcommand{\wl}{\wt{l}}
%%% 函数解析
\renewcommand{\c}{\mathbf{c}}
\newcommand{\loc}{\mathrm{loc}}
\newcommand{\Lh}{\mathrm{L.h.}}
\newcommand{\Epi}{\mathrm{Epi}\;}
\newcommand{\slim}{\mathrm{slim}}
\newcommand{\Ban}{\mathrm{Ban}}
\newcommand{\Hilb}{\mathrm{Hilb}}
\newcommand{\Ex}{\mathrm{Ex}}
\newcommand{\Co}{\mathrm{Co}}
\newcommand{\sa}{\mathrm{sa}}
\newcommand{\nnorm}[1]{{\left\vert\kern-0.25ex\left\vert\kern-0.25ex\left\vert #1 \right\vert\kern-0.25ex\right\vert\kern-0.25ex\right\vert}}
\newcommand{\dvol}{\mathrm{dvol}}
\newcommand{\Sconv}{\mathrm{Sconv}}
\newcommand{\I}{\mathcal{I}}
\newcommand{\nonunital}{\mathrm{nu}}
\newcommand{\cpt}{\mathrm{cpt}}
\newcommand{\lcpt}{\mathrm{lcpt}}
\newcommand{\com}{\mathrm{com}}
\newcommand{\Haus}{\mathrm{Haus}}
\newcommand{\proper}{\mathrm{proper}}
\newcommand{\infinity}{\mathrm{inf}}
\newcommand{\TVS}{\mathrm{TVS}}
\newcommand{\ess}{\mathrm{ess}}
\newcommand{\ext}{\mathrm{ext}}
\newcommand{\Index}{\mathrm{Index}}
\newcommand{\SSR}{\mathrm{SSR}}
\newcommand{\vs}{\mathrm{vs.}}
\newcommand{\fM}{\mathfrak{M}}
\newcommand{\EDM}{\mathrm{EDM}}
\newcommand{\Tw}{\mathrm{Tw}}
\newcommand{\fC}{\mathfrak{C}}
\newcommand{\bn}{\mathbf{n}}
\newcommand{\br}{\mathbf{r}}
\newcommand{\Lam}{\Lambda}
\newcommand{\lam}{\lambda}
\newcommand{\one}{\mathbf{1}}
\newcommand{\dae}{\text{-a.e.}}
\newcommand{\td}{\text{-}}
\newcommand{\RM}{\mathrm{RM}}
%%% 最適化
\newcommand{\Minimize}{\text{Minimize}}
\newcommand{\subjectto}{\text{subject to}}
\newcommand{\Ri}{\mathrm{Ri}}
%\newcommand{\Cl}{\mathrm{Cl}}
\newcommand{\Cone}{\mathrm{Cone}}
\newcommand{\Int}{\mathrm{Int}}
%%% 圏
\newcommand{\varlim}{\varprojlim}
\newcommand{\Hom}{\mathrm{Hom}}
\newcommand{\Iso}{\mathrm{Iso}}
\newcommand{\Mor}{\mathrm{Mor}}
\newcommand{\Isom}{\mathrm{Isom}}
\newcommand{\Aut}{\mathrm{Aut}}
\newcommand{\End}{\mathrm{End}}
\newcommand{\op}{\mathrm{op}}
\newcommand{\ev}{\mathrm{ev}}
\newcommand{\Ob}{\mathrm{Ob}}
\newcommand{\Ar}{\mathrm{Ar}}
\newcommand{\Arr}{\mathrm{Arr}}
\newcommand{\Set}{\mathrm{Set}}
\newcommand{\Grp}{\mathrm{Grp}}
\newcommand{\Cat}{\mathrm{Cat}}
\newcommand{\Mon}{\mathrm{Mon}}
\newcommand{\CMon}{\mathrm{CMon}} %Comutative Monoid 可換単系とモノイドの射
\newcommand{\Ring}{\mathrm{Ring}}
\newcommand{\CRing}{\mathrm{CRing}}
\newcommand{\Ab}{\mathrm{Ab}}
\newcommand{\Pos}{\mathrm{Pos}}
\newcommand{\Vect}{\mathrm{Vect}}
\newcommand{\FinVect}{\mathrm{FinVect}}
\newcommand{\FinSet}{\mathrm{FinSet}}
\newcommand{\OmegaAlg}{\Omega$-$\mathrm{Alg}}
\newcommand{\OmegaEAlg}{(\Omega,E)$-$\mathrm{Alg}}
\newcommand{\Alg}{\mathrm{Alg}} %代数の圏
\newcommand{\CAlg}{\mathrm{CAlg}} %可換代数の圏
\newcommand{\CPO}{\mathrm{CPO}} %Complete Partial Order & continuous mappings
\newcommand{\Fun}{\mathrm{Fun}}
\newcommand{\Func}{\mathrm{Func}}
\newcommand{\Met}{\mathrm{Met}} %Metric space & Contraction maps
\newcommand{\Pfn}{\mathrm{Pfn}} %Sets & Partial function
\newcommand{\Rel}{\mathrm{Rel}} %Sets & relation
\newcommand{\Bool}{\mathrm{Bool}}
\newcommand{\CABool}{\mathrm{CABool}}
\newcommand{\CompBoolAlg}{\mathrm{CompBoolAlg}}
\newcommand{\BoolAlg}{\mathrm{BoolAlg}}
\newcommand{\BoolRng}{\mathrm{BoolRng}}
\newcommand{\HeytAlg}{\mathrm{HeytAlg}}
\newcommand{\CompHeytAlg}{\mathrm{CompHeytAlg}}
\newcommand{\Lat}{\mathrm{Lat}}
\newcommand{\CompLat}{\mathrm{CompLat}}
\newcommand{\SemiLat}{\mathrm{SemiLat}}
\newcommand{\Stone}{\mathrm{Stone}}
\newcommand{\Sob}{\mathrm{Sob}} %Sober space & continuous map
\newcommand{\Op}{\mathrm{Op}} %Category of open subsets
\newcommand{\Sh}{\mathrm{Sh}} %Category of sheave
\newcommand{\PSh}{\mathrm{PSh}} %Category of presheave, PSh(C)=[C^op,set]のこと
\newcommand{\Conv}{\mathrm{Conv}} %Convergence spaceの圏
\newcommand{\Unif}{\mathrm{Unif}} %一様空間と一様連続写像の圏
\newcommand{\Frm}{\mathrm{Frm}} %フレームとフレームの射
\newcommand{\Locale}{\mathrm{Locale}} %その反対圏
\newcommand{\Diff}{\mathrm{Diff}} %滑らかな多様体の圏
\newcommand{\Mfd}{\mathrm{Mfd}}
\newcommand{\LieAlg}{\mathrm{LieAlg}}
\newcommand{\Quiv}{\mathrm{Quiv}} %Quiverの圏
\newcommand{\B}{\mathcal{B}}
\newcommand{\Span}{\mathrm{Span}}
\newcommand{\Corr}{\mathrm{Corr}}
\newcommand{\Decat}{\mathrm{Decat}}
\newcommand{\Rep}{\mathrm{Rep}}
\newcommand{\Grpd}{\mathrm{Grpd}}
\newcommand{\sSet}{\mathrm{sSet}}
\newcommand{\Mod}{\mathrm{Mod}}
\newcommand{\SmoothMnf}{\mathrm{SmoothMnf}}
\newcommand{\coker}{\mathrm{coker}}

\newcommand{\Ord}{\mathrm{Ord}}
\newcommand{\eq}{\mathrm{eq}}
\newcommand{\coeq}{\mathrm{coeq}}
\newcommand{\act}{\mathrm{act}}

%%%%%%%%%%%%%%% 定理環境(足助先生ありがとうございます) %%%%%%%%%%%%%%%

\everymath{\displaystyle}
\renewcommand{\proofname}{\bf [証明]}
\renewcommand{\thefootnote}{\dag\arabic{footnote}} %足助さんからもらった.どうなるんだ?
\renewcommand{\qedsymbol}{$\blacksquare$}

\renewcommand{\labelenumi}{(\arabic{enumi})} %(1),(2),...がデフォルトであって欲しい
\renewcommand{\labelenumii}{(\alph{enumii})}
\renewcommand{\labelenumiii}{(\roman{enumiii})}

\newtheoremstyle{StatementsWithStar}% ?name?
{3pt}% ?Space above? 1
{3pt}% ?Space below? 1
{}% ?Body font?
{}% ?Indent amount? 2
{\bfseries}% ?Theorem head font?
{\textbf{.}}% ?Punctuation after theorem head?
{.5em}% ?Space after theorem head? 3
{\textbf{\textup{#1~\thetheorem{}}}{}\,$^{\ast}$\thmnote{(#3)}}% ?Theorem head spec (can be left empty, meaning ‘normal’)?
%
\newtheoremstyle{StatementsWithStar2}% ?name?
{3pt}% ?Space above? 1
{3pt}% ?Space below? 1
{}% ?Body font?
{}% ?Indent amount? 2
{\bfseries}% ?Theorem head font?
{\textbf{.}}% ?Punctuation after theorem head?
{.5em}% ?Space after theorem head? 3
{\textbf{\textup{#1~\thetheorem{}}}{}\,$^{\ast\ast}$\thmnote{(#3)}}% ?Theorem head spec (can be left empty, meaning ‘normal’)?
%
\newtheoremstyle{StatementsWithStar3}% ?name?
{3pt}% ?Space above? 1
{3pt}% ?Space below? 1
{}% ?Body font?
{}% ?Indent amount? 2
{\bfseries}% ?Theorem head font?
{\textbf{.}}% ?Punctuation after theorem head?
{.5em}% ?Space after theorem head? 3
{\textbf{\textup{#1~\thetheorem{}}}{}\,$^{\ast\ast\ast}$\thmnote{(#3)}}% ?Theorem head spec (can be left empty, meaning ‘normal’)?
%
\newtheoremstyle{StatementsWithCCirc}% ?name?
{6pt}% ?Space above? 1
{6pt}% ?Space below? 1
{}% ?Body font?
{}% ?Indent amount? 2
{\bfseries}% ?Theorem head font?
{\textbf{.}}% ?Punctuation after theorem head?
{.5em}% ?Space after theorem head? 3
{\textbf{\textup{#1~\thetheorem{}}}{}\,$^{\circledcirc}$\thmnote{(#3)}}% ?Theorem head spec (can be left empty, meaning ‘normal’)?
%
\theoremstyle{definition}
 \newtheorem{theorem}{定理}[section]
 \newtheorem{axiom}[theorem]{公理}
 \newtheorem{corollary}[theorem]{系}
 \newtheorem{proposition}[theorem]{命題}
 \newtheorem*{proposition*}{命題}
 \newtheorem{lemma}[theorem]{補題}
 \newtheorem*{lemma*}{補題}
 \newtheorem*{theorem*}{定理}
 \newtheorem{definition}[theorem]{定義}
 \newtheorem{example}[theorem]{例}
 \newtheorem{notation}[theorem]{記法}
 \newtheorem*{notation*}{記法}
 \newtheorem{assumption}[theorem]{仮定}
 \newtheorem{question}[theorem]{問}
 \newtheorem{counterexample}[theorem]{反例}
 \newtheorem{reidai}[theorem]{例題}
 \newtheorem{ruidai}[theorem]{類題}
 \newtheorem{problem}[theorem]{問題}
 \newtheorem{algorithm}[theorem]{算譜}
 \newtheorem*{solution*}{\bf{[解]}}
 \newtheorem{discussion}[theorem]{議論}
 \newtheorem{remark}[theorem]{注}
 \newtheorem{remarks}[theorem]{要諦}
 \newtheorem{image}[theorem]{描像}
 \newtheorem{observation}[theorem]{観察}
 \newtheorem{universality}[theorem]{普遍性} %非自明な例外がない.
 \newtheorem{universal tendency}[theorem]{普遍傾向} %例外が有意に少ない.
 \newtheorem{hypothesis}[theorem]{仮説} %実験で説明されていない理論.
 \newtheorem{theory}[theorem]{理論} %実験事実とその(さしあたり)整合的な説明.
 \newtheorem{fact}[theorem]{実験事実}
 \newtheorem{model}[theorem]{模型}
 \newtheorem{explanation}[theorem]{説明} %理論による実験事実の説明
 \newtheorem{anomaly}[theorem]{理論の限界}
 \newtheorem{application}[theorem]{応用例}
 \newtheorem{method}[theorem]{手法} %実験手法など,技術的問題.
 \newtheorem{history}[theorem]{歴史}
 \newtheorem{usage}[theorem]{用語法}
 \newtheorem{research}[theorem]{研究}
 \newtheorem{shishin}[theorem]{指針}
 \newtheorem{yodan}[theorem]{余談}
 \newtheorem{construction}[theorem]{構成}
% \newtheorem*{remarknonum}{注}
 \newtheorem*{definition*}{定義}
 \newtheorem*{remark*}{注}
 \newtheorem*{question*}{問}
 \newtheorem*{problem*}{問題}
 \newtheorem*{axiom*}{公理}
 \newtheorem*{example*}{例}
 \newtheorem*{corollary*}{系}
 \newtheorem*{shishin*}{指針}
 \newtheorem*{yodan*}{余談}
 \newtheorem*{kadai*}{課題}
%
\theoremstyle{StatementsWithStar}
 \newtheorem{definition_*}[theorem]{定義}
 \newtheorem{question_*}[theorem]{問}
 \newtheorem{example_*}[theorem]{例}
 \newtheorem{theorem_*}[theorem]{定理}
 \newtheorem{remark_*}[theorem]{注}
%
\theoremstyle{StatementsWithStar2}
 \newtheorem{definition_**}[theorem]{定義}
 \newtheorem{theorem_**}[theorem]{定理}
 \newtheorem{question_**}[theorem]{問}
 \newtheorem{remark_**}[theorem]{注}
%
\theoremstyle{StatementsWithStar3}
 \newtheorem{remark_***}[theorem]{注}
 \newtheorem{question_***}[theorem]{問}
%
\theoremstyle{StatementsWithCCirc}
 \newtheorem{definition_O}[theorem]{定義}
 \newtheorem{question_O}[theorem]{問}
 \newtheorem{example_O}[theorem]{例}
 \newtheorem{remark_O}[theorem]{注}
%
\theoremstyle{definition}
%
\raggedbottom
\allowdisplaybreaks
%\usepackage{mathtools}
%\mathtoolsset{showonlyrefs=true} %labelを附した数式にのみ附番される設定.
%\usepackage{amsmath} %mathtoolsの内部で呼ばれるので要らない.
\usepackage{amsfonts} %mathfrak, mathcal, mathbbなど.
\usepackage{amsthm} %定理環境.
\usepackage{amssymb} %AMSFontsを使うためのパッケージ.
\usepackage{ascmac} %screen, itembox, shadebox環境.全てLATEX2εの標準機能の範囲で作られたもの.
\usepackage{comment} %comment環境を用いて,複数行をcomment outできるようにするpackage
\usepackage{wrapfig} %図の周りに文字をwrapさせることができる.詳細な制御ができる.
\usepackage[usenames, dvipsnames]{xcolor} %xcolorはcolorの拡張.optionの意味はdvipsnamesはLoad a set of predefined colors. forestgreenなどの色が追加されている.usenamesはobsoleteとだけ書いてあった.
\setcounter{tocdepth}{2} %目次に表示される深さ.2はsubsectionまで
\usepackage{multicol} %\begin{multicols}{2}環境で途中からmulticolumnに出来る.

\usepackage{url}
\usepackage[dvipdfmx,colorlinks,linkcolor=blue,urlcolor=blue]{hyperref} %生成されるPDFファイルにおいて、\tableofcontentsによって書き出された目次をクリックすると該当する見出しへジャンプしたり、さらには、\label{ラベル名}を番号で参照する\ref{ラベル名}やthebibliography環境において\bibitem{ラベル名}を文献番号で参照する\cite{ラベル名}においても番号をクリックすると該当箇所にジャンプする.囲み枠はダサいので,colorlinksで囲み廃止し,リンク自体に色を付けることにした.
\usepackage{pxjahyper} %pxrubrica同様,八登崇之さん.hyperrefは日本語pLaTeXに最適化されていないから,hyperrefとセットで,(u)pLaTeX+hyperref+dvipdfmxの組み合わせで日本語を含む「しおり」をもつPDF文書を作成する場合に必要となる機能を提供する
\definecolor{花緑青}{cmyk}{0.52,0.03,0,0.27}
\definecolor{サーモンピンク}{cmyk}{0,0.65,0.65,0.05}
\definecolor{暗中模索}{rgb}{0.2,0.2,0.2}

\usepackage{tikz}
\usetikzlibrary{positioning,automata} %automaton描画のため
\usepackage{tikz-cd}
\usepackage[all]{xy}
\def\objectstyle{\displaystyle} %デフォルトではxymatrix中の数式が文中数式モードになるので,それを直す.\labelstyleも同様にxy packageの中で定義されており,文中数式モードになっている.

\usepackage[version=4]{mhchem} %化学式をTikZで簡単に書くためのパッケージ.
\usepackage{chemfig} %化学構造式をTikZで描くためのパッケージ.
\usepackage{siunitx} %IS単位を書くためのパッケージ

\usepackage{ulem} %取り消し線を引くためのパッケージ
\usepackage{pxrubrica} %日本語にルビをふる.八登崇之(やとうたかゆき)氏による.

\usepackage{graphicx} %rotatebox, scalebox, reflectbox, resizeboxなどのコマンドや,図表の読み込み\includegraphicsを司る.graphics というパッケージもありますが,graphicx はこれを高機能にしたものと考えて結構です(ただし graphicx は内部で graphics を読み込みます)

\usepackage[breakable]{tcolorbox} %加藤晃史さんがフル活用していたtcolorboxを,途中改ページ可能で.
\tcbuselibrary{theorems} %https://qiita.com/t_kemmochi/items/483b8fcdb5db8d1f5d5e
\usepackage{enumerate} %enumerate環境を凝らせる.
\usepackage[top=15truemm,bottom=15truemm,left=10truemm,right=10truemm]{geometry} %足助さんからもらったオプション

%%%%%%%%%%%%%%% 環境マクロ %%%%%%%%%%%%%%%

\usepackage{listings} %ソースコードを表示できる環境.多分もっといい方法ある.
\usepackage{jvlisting} %日本語のコメントアウトをする場合jlistingが必要
\lstset{ %ここからソースコードの表示に関する設定.lstlisting環境では,[caption=hoge,label=fuga]などのoptionを付けられる.
%[escapechar=!]とすると,LaTeXコマンドを使える.
  basicstyle={\ttfamily},
  identifierstyle={\small},
  commentstyle={\smallitshape},
  keywordstyle={\small\bfseries},
  ndkeywordstyle={\small},
  stringstyle={\small\ttfamily},
  frame={tb},
  breaklines=true,
  columns=[l]{fullflexible},
  numbers=left,
  xrightmargin=0zw,
  xleftmargin=3zw,
  numberstyle={\scriptsize},
  stepnumber=1,
  numbersep=1zw,
  lineskip=-0.5ex
}
%\makeatletter %caption番号を「[chapter番号].[section番号].[subsection番号]-[そのsubsection内においてn番目]」に変更
%    \AtBeginDocument{
%    \renewcommand*{\thelstlisting}{\arabic{chapter}.\arabic{section}.\arabic{lstlisting}}
%    \@addtoreset{lstlisting}{section}
%    }
%\makeatother
\renewcommand{\lstlistingname}{算譜} %caption名を"program"に変更

\newtcolorbox{tbox}[3][]{%
colframe=#2,colback=#2!10,coltitle=#2!20!black,title={#3},#1}

%%%%%%%%%%%%%%% フォント %%%%%%%%%%%%%%%

\usepackage{textcomp, mathcomp} %Text Companionとは,T1 encodingに入らなかった文字群.これを使うためのパッケージ.\textsectionでブルバキに!
\usepackage[T1]{fontenc} %8bitエンコーディングにする.comp系拡張数学文字の動作が安定する.

%%%%%%%%%%%%%%% 数学記号のマクロ %%%%%%%%%%%%%%%

\newcommand{\abs}[1]{\lvert#1\rvert} %mathtoolsはこうやって使うのか!
\newcommand{\Abs}[1]{\left|#1\right|}
\newcommand{\norm}[1]{\|#1\|}
\newcommand{\Norm}[1]{\left\|#1\right\|}
%\newcommand{\brace}[1]{\{#1\}}
\newcommand{\Brace}[1]{\left\{#1\right\}}
\newcommand{\paren}[1]{\left(#1\right)}
\newcommand{\bracket}[1]{\langle#1\rangle}
\newcommand{\brac}[1]{\langle#1\rangle}
\newcommand{\Bracket}[1]{\left\langle#1\right\rangle}
\newcommand{\Brac}[1]{\left\langle#1\right\rangle}
\newcommand{\Square}[1]{\left[#1\right]}
\renewcommand{\o}[1]{\overline{#1}}
\renewcommand{\u}[1]{\underline{#1}}
\renewcommand{\iff}{\;\mathrm{iff}\;} %nLabリスペクト
\newcommand{\pp}[2]{\frac{\partial #1}{\partial #2}}
\newcommand{\ppp}[3]{\frac{\partial #1}{\partial #2\partial #3}}
\newcommand{\dd}[2]{\frac{d #1}{d #2}}
\newcommand{\floor}[1]{\lfloor#1\rfloor}
\newcommand{\Floor}[1]{\left\lfloor#1\right\rfloor}
\newcommand{\ceil}[1]{\lceil#1\rceil}

\newcommand{\iso}{\xrightarrow{\,\smash{\raisebox{-0.45ex}{\ensuremath{\scriptstyle\sim}}}\,}}
\newcommand{\wt}[1]{\widetilde{#1}}
\newcommand{\wh}[1]{\widehat{#1}}

\newcommand{\Lrarrow}{\;\;\Leftrightarrow\;\;}

%ノルム位相についての閉包 https://newbedev.com/how-to-make-double-overline-with-less-vertical-displacement
\makeatletter
\newcommand{\dbloverline}[1]{\overline{\dbl@overline{#1}}}
\newcommand{\dbl@overline}[1]{\mathpalette\dbl@@overline{#1}}
\newcommand{\dbl@@overline}[2]{%
  \begingroup
  \sbox\z@{$\m@th#1\overline{#2}$}%
  \ht\z@=\dimexpr\ht\z@-2\dbl@adjust{#1}\relax
  \box\z@
  \ifx#1\scriptstyle\kern-\scriptspace\else
  \ifx#1\scriptscriptstyle\kern-\scriptspace\fi\fi
  \endgroup
}
\newcommand{\dbl@adjust}[1]{%
  \fontdimen8
  \ifx#1\displaystyle\textfont\else
  \ifx#1\textstyle\textfont\else
  \ifx#1\scriptstyle\scriptfont\else
  \scriptscriptfont\fi\fi\fi 3
}
\makeatother
\newcommand{\oo}[1]{\dbloverline{#1}}

\DeclareMathOperator{\grad}{\mathrm{grad}}
\DeclareMathOperator{\rot}{\mathrm{rot}}
\DeclareMathOperator{\divergence}{\mathrm{div}}
\newcommand{\False}{\mathrm{False}}
\newcommand{\True}{\mathrm{True}}
\DeclareMathOperator{\tr}{\mathrm{tr}}
\newcommand{\M}{\mathcal{M}}
\newcommand{\cF}{\mathcal{F}}
\newcommand{\cD}{\mathcal{D}}
\newcommand{\fX}{\mathfrak{X}}
\newcommand{\fY}{\mathfrak{Y}}
\newcommand{\fZ}{\mathfrak{Z}}
\renewcommand{\H}{\mathcal{H}}
\newcommand{\fH}{\mathfrak{H}}
\newcommand{\bH}{\mathbb{H}}
\newcommand{\id}{\mathrm{id}}
\newcommand{\A}{\mathcal{A}}
% \renewcommand\coprod{\rotatebox[origin=c]{180}{$\prod$}} すでにどこかにある.
\newcommand{\pr}{\mathrm{pr}}
\newcommand{\U}{\mathfrak{U}}
\newcommand{\Map}{\mathrm{Map}}
\newcommand{\dom}{\mathrm{Dom}\;}
\newcommand{\cod}{\mathrm{Cod}\;}
\newcommand{\supp}{\mathrm{supp}\;}
\newcommand{\otherwise}{\mathrm{otherwise}}
\newcommand{\st}{\;\mathrm{s.t.}\;}
\newcommand{\lmd}{\lambda}
\newcommand{\Lmd}{\Lambda}
%%% 線型代数学
\newcommand{\Ker}{\mathrm{Ker}\;}
\newcommand{\Coker}{\mathrm{Coker}\;}
\newcommand{\Coim}{\mathrm{Coim}\;}
\newcommand{\rank}{\mathrm{rank}}
\newcommand{\lcm}{\mathrm{lcm}}
\newcommand{\sgn}{\mathrm{sgn}}
\newcommand{\GL}{\mathrm{GL}}
\newcommand{\SL}{\mathrm{SL}}
\newcommand{\alt}{\mathrm{alt}}
%%% 複素解析学
\renewcommand{\Re}{\mathrm{Re}\;}
\renewcommand{\Im}{\mathrm{Im}\;}
\newcommand{\Gal}{\mathrm{Gal}}
\newcommand{\PGL}{\mathrm{PGL}}
\newcommand{\PSL}{\mathrm{PSL}}
\newcommand{\Log}{\mathrm{Log}\,}
\newcommand{\Res}{\mathrm{Res}\,}
\newcommand{\on}{\mathrm{on}\;}
\newcommand{\hatC}{\hat{\C}}
\newcommand{\hatR}{\hat{\R}}
\newcommand{\PV}{\mathrm{P.V.}}
\newcommand{\diam}{\mathrm{diam}}
\newcommand{\Area}{\mathrm{Area}}
\newcommand{\Lap}{\Laplace}
\newcommand{\f}{\mathbf{f}}
\newcommand{\cR}{\mathcal{R}}
\newcommand{\const}{\mathrm{const.}}
\newcommand{\Om}{\Omega}
\newcommand{\Cinf}{C^\infty}
\newcommand{\ep}{\epsilon}
\newcommand{\dist}{\mathrm{dist}}
\newcommand{\opart}{\o{\partial}}
%%% 解析力学
\newcommand{\x}{\mathbf{x}}
%%% 集合と位相
\renewcommand{\O}{\mathcal{O}}
\renewcommand{\S}{\mathcal{S}}
\renewcommand{\U}{\mathcal{U}}
\newcommand{\V}{\mathcal{V}}
\renewcommand{\P}{\mathcal{P}}
\newcommand{\R}{\mathbb{R}}
\newcommand{\N}{\mathbb{N}}
\newcommand{\C}{\mathbb{C}}
\newcommand{\Z}{\mathbb{Z}}
\newcommand{\Q}{\mathbb{Q}}
\newcommand{\TV}{\mathrm{TV}}
\newcommand{\ORD}{\mathrm{ORD}}
\newcommand{\Tr}{\mathrm{Tr}\;}
\newcommand{\Card}{\mathrm{Card}\;}
\newcommand{\Top}{\mathrm{Top}}
\newcommand{\Disc}{\mathrm{Disc}}
\newcommand{\Codisc}{\mathrm{Codisc}}
\newcommand{\CoDisc}{\mathrm{CoDisc}}
\newcommand{\Ult}{\mathrm{Ult}}
\newcommand{\ord}{\mathrm{ord}}
\newcommand{\maj}{\mathrm{maj}}
%%% 形式言語理論
\newcommand{\REGEX}{\mathrm{REGEX}}
\newcommand{\RE}{\mathbf{RE}}

%%% Fourier解析
\newcommand*{\Laplace}{\mathop{}\!\mathbin\bigtriangleup}
\newcommand*{\DAlambert}{\mathop{}\!\mathbin\Box}
%%% Graph Theory
\newcommand{\SimpGph}{\mathrm{SimpGph}}
\newcommand{\Gph}{\mathrm{Gph}}
\newcommand{\mult}{\mathrm{mult}}
\newcommand{\inv}{\mathrm{inv}}
%%% 多様体
\newcommand{\Der}{\mathrm{Der}}
\newcommand{\osub}{\overset{\mathrm{open}}{\subset}}
\newcommand{\osup}{\overset{\mathrm{open}}{\supset}}
\newcommand{\al}{\alpha}
\newcommand{\K}{\mathbb{K}}
\newcommand{\Sp}{\mathrm{Sp}}
\newcommand{\g}{\mathfrak{g}}
\newcommand{\h}{\mathfrak{h}}
\newcommand{\Exp}{\mathrm{Exp}\;}
\newcommand{\Imm}{\mathrm{Imm}}
\newcommand{\Imb}{\mathrm{Imb}}
\newcommand{\codim}{\mathrm{codim}\;}
\newcommand{\Gr}{\mathrm{Gr}}
%%% 代数
\newcommand{\Ad}{\mathrm{Ad}}
\newcommand{\finsupp}{\mathrm{fin\;supp}}
\newcommand{\SO}{\mathrm{SO}}
\newcommand{\SU}{\mathrm{SU}}
\newcommand{\acts}{\curvearrowright}
\newcommand{\mono}{\hookrightarrow}
\newcommand{\epi}{\twoheadrightarrow}
\newcommand{\Stab}{\mathrm{Stab}}
\newcommand{\nor}{\mathrm{nor}}
\newcommand{\T}{\mathbb{T}}
\newcommand{\Aff}{\mathrm{Aff}}
\newcommand{\rsub}{\triangleleft}
\newcommand{\rsup}{\triangleright}
\newcommand{\subgrp}{\overset{\mathrm{subgrp}}{\subset}}
\newcommand{\Ext}{\mathrm{Ext}}
\newcommand{\sbs}{\subset}\newcommand{\sps}{\supset}
\newcommand{\In}{\mathrm{In}}
\newcommand{\Tor}{\mathrm{Tor}}
\newcommand{\p}{\mathfrak{p}}
\newcommand{\q}{\mathfrak{q}}
\newcommand{\m}{\mathfrak{m}}
\newcommand{\cS}{\mathcal{S}}
\newcommand{\Frac}{\mathrm{Frac}\,}
\newcommand{\Spec}{\mathrm{Spec}\,}
\newcommand{\bA}{\mathbb{A}}
\newcommand{\Sym}{\mathrm{Sym}}
\newcommand{\Ann}{\mathrm{Ann}}
%%% 代数的位相幾何学
\newcommand{\Ho}{\mathrm{Ho}}
\newcommand{\CW}{\mathrm{CW}}
\newcommand{\lc}{\mathrm{lc}}
\newcommand{\cg}{\mathrm{cg}}
\newcommand{\Fib}{\mathrm{Fib}}
\newcommand{\Cyl}{\mathrm{Cyl}}
\newcommand{\Ch}{\mathrm{Ch}}
%%% 数値解析
\newcommand{\round}{\mathrm{round}}
\newcommand{\cond}{\mathrm{cond}}
\newcommand{\diag}{\mathrm{diag}}
%%% 確率論
\newcommand{\calF}{\mathcal{F}}
\newcommand{\X}{\mathcal{X}}
\newcommand{\Meas}{\mathrm{Meas}}
\newcommand{\as}{\;\mathrm{a.s.}} %almost surely
\newcommand{\io}{\;\mathrm{i.o.}} %infinitely often
\newcommand{\fe}{\;\mathrm{f.e.}} %with a finite number of exceptions
\newcommand{\F}{\mathcal{F}}
\newcommand{\bF}{\mathbb{F}}
\newcommand{\W}{\mathcal{W}}
\newcommand{\Pois}{\mathrm{Pois}}
\newcommand{\iid}{\mathrm{i.i.d.}}
\newcommand{\wconv}{\rightsquigarrow}
\newcommand{\Var}{\mathrm{Var}}
\newcommand{\xrightarrown}{\xrightarrow{n\to\infty}}
\newcommand{\au}{\mathrm{au}}
\newcommand{\cT}{\mathcal{T}}
%%% 情報理論
\newcommand{\bit}{\mathrm{bit}}
%%% 積分論
\newcommand{\calA}{\mathcal{A}}
\newcommand{\calB}{\mathcal{B}}
\newcommand{\D}{\mathcal{D}}
\newcommand{\Y}{\mathcal{Y}}
\newcommand{\calC}{\mathcal{C}}
\renewcommand{\ae}{\mathrm{a.e.}\;}
\newcommand{\cZ}{\mathcal{Z}}
\newcommand{\fF}{\mathfrak{F}}
\newcommand{\fI}{\mathfrak{I}}
\newcommand{\E}{\mathcal{E}}
\newcommand{\sMap}{\sigma\textrm{-}\mathrm{Map}}
\DeclareMathOperator*{\argmax}{arg\,max}
\DeclareMathOperator*{\argmin}{arg\,min}
\newcommand{\cC}{\mathcal{C}}
\newcommand{\comp}{\complement}
\newcommand{\J}{\mathcal{J}}
\newcommand{\sumN}[1]{\sum_{#1\in\N}}
\newcommand{\cupN}[1]{\cup_{#1\in\N}}
\newcommand{\capN}[1]{\cap_{#1\in\N}}
\newcommand{\Sum}[1]{\sum_{#1=1}^\infty}
\newcommand{\sumn}{\sum_{n=1}^\infty}
\newcommand{\summ}{\sum_{m=1}^\infty}
\newcommand{\sumk}{\sum_{k=1}^\infty}
\newcommand{\sumi}{\sum_{i=1}^\infty}
\newcommand{\sumj}{\sum_{j=1}^\infty}
\newcommand{\cupn}{\cup_{n=1}^\infty}
\newcommand{\capn}{\cap_{n=1}^\infty}
\newcommand{\cupk}{\cup_{k=1}^\infty}
\newcommand{\cupi}{\cup_{i=1}^\infty}
\newcommand{\cupj}{\cup_{j=1}^\infty}
\newcommand{\limn}{\lim_{n\to\infty}}
\renewcommand{\l}{\mathcal{l}}
\renewcommand{\L}{\mathcal{L}}
\newcommand{\Cl}{\mathrm{Cl}}
\newcommand{\cN}{\mathcal{N}}
\newcommand{\Ae}{\textrm{-a.e.}\;}
\newcommand{\csub}{\overset{\textrm{closed}}{\subset}}
\newcommand{\csup}{\overset{\textrm{closed}}{\supset}}
\newcommand{\wB}{\wt{B}}
\newcommand{\cG}{\mathcal{G}}
\newcommand{\Lip}{\mathrm{Lip}}
\newcommand{\Dom}{\mathrm{Dom}}
%%% 数理ファイナンス
\newcommand{\pre}{\mathrm{pre}}
\newcommand{\om}{\omega}

%%% 統計的因果推論
\newcommand{\Do}{\mathrm{Do}}
%%% 数理統計
\newcommand{\bP}{\mathbb{P}}
\newcommand{\compsub}{\overset{\textrm{cpt}}{\subset}}
\newcommand{\lip}{\textrm{lip}}
\newcommand{\BL}{\mathrm{BL}}
\newcommand{\G}{\mathbb{G}}
\newcommand{\NB}{\mathrm{NB}}
\newcommand{\oR}{\o{\R}}
\newcommand{\liminfn}{\liminf_{n\to\infty}}
\newcommand{\limsupn}{\limsup_{n\to\infty}}
%\newcommand{\limn}{\lim_{n\to\infty}}
\newcommand{\esssup}{\mathrm{ess.sup}}
\newcommand{\asto}{\xrightarrow{\as}}
\newcommand{\Cov}{\mathrm{Cov}}
\newcommand{\cQ}{\mathcal{Q}}
\newcommand{\VC}{\mathrm{VC}}
\newcommand{\mb}{\mathrm{mb}}
\newcommand{\Avar}{\mathrm{Avar}}
\newcommand{\bB}{\mathbb{B}}
\newcommand{\bW}{\mathbb{W}}
\newcommand{\sd}{\mathrm{sd}}
\newcommand{\w}[1]{\widehat{#1}}
\newcommand{\bZ}{\mathbb{Z}}
\newcommand{\Bernoulli}{\mathrm{Bernoulli}}
\newcommand{\Mult}{\mathrm{Mult}}
\newcommand{\BPois}{\mathrm{BPois}}
\newcommand{\fraks}{\mathfrak{s}}
\newcommand{\frakk}{\mathfrak{k}}
\newcommand{\IF}{\mathrm{IF}}
\newcommand{\bX}{\mathbf{X}}
\newcommand{\bx}{\mathbf{x}}
\newcommand{\indep}{\raisebox{0.05em}{\rotatebox[origin=c]{90}{$\models$}}}
\newcommand{\IG}{\mathrm{IG}}
\newcommand{\Levy}{\mathrm{Levy}}
\newcommand{\MP}{\mathrm{MP}}
\newcommand{\Hermite}{\mathrm{Hermite}}
\newcommand{\Skellam}{\mathrm{Skellam}}
\newcommand{\Dirichlet}{\mathrm{Dirichlet}}
\newcommand{\Beta}{\mathrm{Beta}}
\newcommand{\bE}{\mathbb{E}}
\newcommand{\bG}{\mathbb{G}}
\newcommand{\MISE}{\mathrm{MISE}}
\newcommand{\logit}{\mathtt{logit}}
\newcommand{\expit}{\mathtt{expit}}
\newcommand{\cK}{\mathcal{K}}
\newcommand{\dl}{\dot{l}}
\newcommand{\dotp}{\dot{p}}
\newcommand{\wl}{\wt{l}}
%%% 函数解析
\renewcommand{\c}{\mathbf{c}}
\newcommand{\loc}{\mathrm{loc}}
\newcommand{\Lh}{\mathrm{L.h.}}
\newcommand{\Epi}{\mathrm{Epi}\;}
\newcommand{\slim}{\mathrm{slim}}
\newcommand{\Ban}{\mathrm{Ban}}
\newcommand{\Hilb}{\mathrm{Hilb}}
\newcommand{\Ex}{\mathrm{Ex}}
\newcommand{\Co}{\mathrm{Co}}
\newcommand{\sa}{\mathrm{sa}}
\newcommand{\nnorm}[1]{{\left\vert\kern-0.25ex\left\vert\kern-0.25ex\left\vert #1 \right\vert\kern-0.25ex\right\vert\kern-0.25ex\right\vert}}
\newcommand{\dvol}{\mathrm{dvol}}
\newcommand{\Sconv}{\mathrm{Sconv}}
\newcommand{\I}{\mathcal{I}}
\newcommand{\nonunital}{\mathrm{nu}}
\newcommand{\cpt}{\mathrm{cpt}}
\newcommand{\lcpt}{\mathrm{lcpt}}
\newcommand{\com}{\mathrm{com}}
\newcommand{\Haus}{\mathrm{Haus}}
\newcommand{\proper}{\mathrm{proper}}
\newcommand{\infinity}{\mathrm{inf}}
\newcommand{\TVS}{\mathrm{TVS}}
\newcommand{\ess}{\mathrm{ess}}
\newcommand{\ext}{\mathrm{ext}}
\newcommand{\Index}{\mathrm{Index}}
\newcommand{\SSR}{\mathrm{SSR}}
\newcommand{\vs}{\mathrm{vs.}}
\newcommand{\fM}{\mathfrak{M}}
\newcommand{\EDM}{\mathrm{EDM}}
\newcommand{\Tw}{\mathrm{Tw}}
\newcommand{\fC}{\mathfrak{C}}
\newcommand{\bn}{\mathbf{n}}
\newcommand{\br}{\mathbf{r}}
\newcommand{\Lam}{\Lambda}
\newcommand{\lam}{\lambda}
\newcommand{\one}{\mathbf{1}}
\newcommand{\dae}{\text{-a.e.}}
\newcommand{\td}{\text{-}}
\newcommand{\RM}{\mathrm{RM}}
%%% 最適化
\newcommand{\Minimize}{\text{Minimize}}
\newcommand{\subjectto}{\text{subject to}}
\newcommand{\Ri}{\mathrm{Ri}}
%\newcommand{\Cl}{\mathrm{Cl}}
\newcommand{\Cone}{\mathrm{Cone}}
\newcommand{\Int}{\mathrm{Int}}
%%% 圏
\newcommand{\varlim}{\varprojlim}
\newcommand{\Hom}{\mathrm{Hom}}
\newcommand{\Iso}{\mathrm{Iso}}
\newcommand{\Mor}{\mathrm{Mor}}
\newcommand{\Isom}{\mathrm{Isom}}
\newcommand{\Aut}{\mathrm{Aut}}
\newcommand{\End}{\mathrm{End}}
\newcommand{\op}{\mathrm{op}}
\newcommand{\ev}{\mathrm{ev}}
\newcommand{\Ob}{\mathrm{Ob}}
\newcommand{\Ar}{\mathrm{Ar}}
\newcommand{\Arr}{\mathrm{Arr}}
\newcommand{\Set}{\mathrm{Set}}
\newcommand{\Grp}{\mathrm{Grp}}
\newcommand{\Cat}{\mathrm{Cat}}
\newcommand{\Mon}{\mathrm{Mon}}
\newcommand{\CMon}{\mathrm{CMon}} %Comutative Monoid 可換単系とモノイドの射
\newcommand{\Ring}{\mathrm{Ring}}
\newcommand{\CRing}{\mathrm{CRing}}
\newcommand{\Ab}{\mathrm{Ab}}
\newcommand{\Pos}{\mathrm{Pos}}
\newcommand{\Vect}{\mathrm{Vect}}
\newcommand{\FinVect}{\mathrm{FinVect}}
\newcommand{\FinSet}{\mathrm{FinSet}}
\newcommand{\OmegaAlg}{\Omega$-$\mathrm{Alg}}
\newcommand{\OmegaEAlg}{(\Omega,E)$-$\mathrm{Alg}}
\newcommand{\Alg}{\mathrm{Alg}} %代数の圏
\newcommand{\CAlg}{\mathrm{CAlg}} %可換代数の圏
\newcommand{\CPO}{\mathrm{CPO}} %Complete Partial Order & continuous mappings
\newcommand{\Fun}{\mathrm{Fun}}
\newcommand{\Func}{\mathrm{Func}}
\newcommand{\Met}{\mathrm{Met}} %Metric space & Contraction maps
\newcommand{\Pfn}{\mathrm{Pfn}} %Sets & Partial function
\newcommand{\Rel}{\mathrm{Rel}} %Sets & relation
\newcommand{\Bool}{\mathrm{Bool}}
\newcommand{\CABool}{\mathrm{CABool}}
\newcommand{\CompBoolAlg}{\mathrm{CompBoolAlg}}
\newcommand{\BoolAlg}{\mathrm{BoolAlg}}
\newcommand{\BoolRng}{\mathrm{BoolRng}}
\newcommand{\HeytAlg}{\mathrm{HeytAlg}}
\newcommand{\CompHeytAlg}{\mathrm{CompHeytAlg}}
\newcommand{\Lat}{\mathrm{Lat}}
\newcommand{\CompLat}{\mathrm{CompLat}}
\newcommand{\SemiLat}{\mathrm{SemiLat}}
\newcommand{\Stone}{\mathrm{Stone}}
\newcommand{\Sob}{\mathrm{Sob}} %Sober space & continuous map
\newcommand{\Op}{\mathrm{Op}} %Category of open subsets
\newcommand{\Sh}{\mathrm{Sh}} %Category of sheave
\newcommand{\PSh}{\mathrm{PSh}} %Category of presheave, PSh(C)=[C^op,set]のこと
\newcommand{\Conv}{\mathrm{Conv}} %Convergence spaceの圏
\newcommand{\Unif}{\mathrm{Unif}} %一様空間と一様連続写像の圏
\newcommand{\Frm}{\mathrm{Frm}} %フレームとフレームの射
\newcommand{\Locale}{\mathrm{Locale}} %その反対圏
\newcommand{\Diff}{\mathrm{Diff}} %滑らかな多様体の圏
\newcommand{\Mfd}{\mathrm{Mfd}}
\newcommand{\LieAlg}{\mathrm{LieAlg}}
\newcommand{\Quiv}{\mathrm{Quiv}} %Quiverの圏
\newcommand{\B}{\mathcal{B}}
\newcommand{\Span}{\mathrm{Span}}
\newcommand{\Corr}{\mathrm{Corr}}
\newcommand{\Decat}{\mathrm{Decat}}
\newcommand{\Rep}{\mathrm{Rep}}
\newcommand{\Grpd}{\mathrm{Grpd}}
\newcommand{\sSet}{\mathrm{sSet}}
\newcommand{\Mod}{\mathrm{Mod}}
\newcommand{\SmoothMnf}{\mathrm{SmoothMnf}}
\newcommand{\coker}{\mathrm{coker}}

\newcommand{\Ord}{\mathrm{Ord}}
\newcommand{\eq}{\mathrm{eq}}
\newcommand{\coeq}{\mathrm{coeq}}
\newcommand{\act}{\mathrm{act}}

%%%%%%%%%%%%%%% 定理環境(足助先生ありがとうございます) %%%%%%%%%%%%%%%

\everymath{\displaystyle}
\renewcommand{\proofname}{\bf [証明]}
\renewcommand{\thefootnote}{\dag\arabic{footnote}} %足助さんからもらった.どうなるんだ?
\renewcommand{\qedsymbol}{$\blacksquare$}

\renewcommand{\labelenumi}{(\arabic{enumi})} %(1),(2),...がデフォルトであって欲しい
\renewcommand{\labelenumii}{(\alph{enumii})}
\renewcommand{\labelenumiii}{(\roman{enumiii})}

\newtheoremstyle{StatementsWithStar}% ?name?
{3pt}% ?Space above? 1
{3pt}% ?Space below? 1
{}% ?Body font?
{}% ?Indent amount? 2
{\bfseries}% ?Theorem head font?
{\textbf{.}}% ?Punctuation after theorem head?
{.5em}% ?Space after theorem head? 3
{\textbf{\textup{#1~\thetheorem{}}}{}\,$^{\ast}$\thmnote{(#3)}}% ?Theorem head spec (can be left empty, meaning ‘normal’)?
%
\newtheoremstyle{StatementsWithStar2}% ?name?
{3pt}% ?Space above? 1
{3pt}% ?Space below? 1
{}% ?Body font?
{}% ?Indent amount? 2
{\bfseries}% ?Theorem head font?
{\textbf{.}}% ?Punctuation after theorem head?
{.5em}% ?Space after theorem head? 3
{\textbf{\textup{#1~\thetheorem{}}}{}\,$^{\ast\ast}$\thmnote{(#3)}}% ?Theorem head spec (can be left empty, meaning ‘normal’)?
%
\newtheoremstyle{StatementsWithStar3}% ?name?
{3pt}% ?Space above? 1
{3pt}% ?Space below? 1
{}% ?Body font?
{}% ?Indent amount? 2
{\bfseries}% ?Theorem head font?
{\textbf{.}}% ?Punctuation after theorem head?
{.5em}% ?Space after theorem head? 3
{\textbf{\textup{#1~\thetheorem{}}}{}\,$^{\ast\ast\ast}$\thmnote{(#3)}}% ?Theorem head spec (can be left empty, meaning ‘normal’)?
%
\newtheoremstyle{StatementsWithCCirc}% ?name?
{6pt}% ?Space above? 1
{6pt}% ?Space below? 1
{}% ?Body font?
{}% ?Indent amount? 2
{\bfseries}% ?Theorem head font?
{\textbf{.}}% ?Punctuation after theorem head?
{.5em}% ?Space after theorem head? 3
{\textbf{\textup{#1~\thetheorem{}}}{}\,$^{\circledcirc}$\thmnote{(#3)}}% ?Theorem head spec (can be left empty, meaning ‘normal’)?
%
\theoremstyle{definition}
 \newtheorem{theorem}{定理}[section]
 \newtheorem{axiom}[theorem]{公理}
 \newtheorem{corollary}[theorem]{系}
 \newtheorem{proposition}[theorem]{命題}
 \newtheorem*{proposition*}{命題}
 \newtheorem{lemma}[theorem]{補題}
 \newtheorem*{lemma*}{補題}
 \newtheorem*{theorem*}{定理}
 \newtheorem{definition}[theorem]{定義}
 \newtheorem{example}[theorem]{例}
 \newtheorem{notation}[theorem]{記法}
 \newtheorem*{notation*}{記法}
 \newtheorem{assumption}[theorem]{仮定}
 \newtheorem{question}[theorem]{問}
 \newtheorem{counterexample}[theorem]{反例}
 \newtheorem{reidai}[theorem]{例題}
 \newtheorem{ruidai}[theorem]{類題}
 \newtheorem{problem}[theorem]{問題}
 \newtheorem{algorithm}[theorem]{算譜}
 \newtheorem*{solution*}{\bf{[解]}}
 \newtheorem{discussion}[theorem]{議論}
 \newtheorem{remark}[theorem]{注}
 \newtheorem{remarks}[theorem]{要諦}
 \newtheorem{image}[theorem]{描像}
 \newtheorem{observation}[theorem]{観察}
 \newtheorem{universality}[theorem]{普遍性} %非自明な例外がない.
 \newtheorem{universal tendency}[theorem]{普遍傾向} %例外が有意に少ない.
 \newtheorem{hypothesis}[theorem]{仮説} %実験で説明されていない理論.
 \newtheorem{theory}[theorem]{理論} %実験事実とその(さしあたり)整合的な説明.
 \newtheorem{fact}[theorem]{実験事実}
 \newtheorem{model}[theorem]{模型}
 \newtheorem{explanation}[theorem]{説明} %理論による実験事実の説明
 \newtheorem{anomaly}[theorem]{理論の限界}
 \newtheorem{application}[theorem]{応用例}
 \newtheorem{method}[theorem]{手法} %実験手法など,技術的問題.
 \newtheorem{history}[theorem]{歴史}
 \newtheorem{usage}[theorem]{用語法}
 \newtheorem{research}[theorem]{研究}
 \newtheorem{shishin}[theorem]{指針}
 \newtheorem{yodan}[theorem]{余談}
 \newtheorem{construction}[theorem]{構成}
% \newtheorem*{remarknonum}{注}
 \newtheorem*{definition*}{定義}
 \newtheorem*{remark*}{注}
 \newtheorem*{question*}{問}
 \newtheorem*{problem*}{問題}
 \newtheorem*{axiom*}{公理}
 \newtheorem*{example*}{例}
 \newtheorem*{corollary*}{系}
 \newtheorem*{shishin*}{指針}
 \newtheorem*{yodan*}{余談}
 \newtheorem*{kadai*}{課題}
%
\theoremstyle{StatementsWithStar}
 \newtheorem{definition_*}[theorem]{定義}
 \newtheorem{question_*}[theorem]{問}
 \newtheorem{example_*}[theorem]{例}
 \newtheorem{theorem_*}[theorem]{定理}
 \newtheorem{remark_*}[theorem]{注}
%
\theoremstyle{StatementsWithStar2}
 \newtheorem{definition_**}[theorem]{定義}
 \newtheorem{theorem_**}[theorem]{定理}
 \newtheorem{question_**}[theorem]{問}
 \newtheorem{remark_**}[theorem]{注}
%
\theoremstyle{StatementsWithStar3}
 \newtheorem{remark_***}[theorem]{注}
 \newtheorem{question_***}[theorem]{問}
%
\theoremstyle{StatementsWithCCirc}
 \newtheorem{definition_O}[theorem]{定義}
 \newtheorem{question_O}[theorem]{問}
 \newtheorem{example_O}[theorem]{例}
 \newtheorem{remark_O}[theorem]{注}
%
\theoremstyle{definition}
%
\raggedbottom
\allowdisplaybreaks
\usepackage[math]{anttor}
\begin{document}
\tableofcontents

\chapter{確率過程と独立性}

\begin{quotation}
    確率的な方法を使って数学的対象を調べることも,現実的対象を調べることも出来る.
    統計推測への応用も,調和解析への応用も考えたい.

    値の空間が等しい確率変数の族を確率過程といい,このときの値域である位相空間を状態空間という.\footnote{最も一般的にはBanach空間を取ることが流行らしい.}
    確率変数族には独立性の概念が拡張できたが,これは応用上自然ではない.
    遥かに緩いクラスとして,マルチンゲールを定義する.
    1930年代に,独立確率変数の和の理論を整備する過程で豊かに育ったKolmogorovのアイデアを一般化する試みの中で,Levyがマルチンゲールの概念を発明し,Doobが理論を立てた.
    Brown運動も確率積分もマルチンゲールになる.

    解析学に可測関数,連続関数,解析関数というようなクラスがあるように,確率論にもマルチンゲール,加法過程,Markov過程,定常過程などのクラスがある.
    解析学に指数関数,Bessel関数などの特殊関数があるように,確率論にもWeiner過程,Poisson過程というような特殊過程がある.
    ただし,分類の指導方針が全く違う.確率論の指導原理は独立性であってきた.
\end{quotation}

\section{有限の場合の独立性}

\begin{tcolorbox}[colframe=ForestGreen, colback=ForestGreen!10!white,breakable,colbacktitle=ForestGreen!40!white,coltitle=black,fonttitle=\bfseries\sffamily,
title=]
    Kolmogorovの本のように,試行の列を考えると筋が良い.
    意味論的な中心は試行=分割$\fA=(A_i)$であるが,数学的な主役はこれが生成する$\sigma$-代数$\sigma[\fA]$である.
    確率変数$X$は定義域上に自明な同値関係を定めるが,これが定める類別が$X$を観測するという試行となる.
    すると,条件付き確率の背後にも試行,すなわち,$\sigma$-代数があることが明瞭に理解できる.
    ちょうど漸近理論において統計的実験の列を考えると筋が良いのに似ている.

    独立性は,分割の直交性で捉えられそうであるが,正確に一致させるためには公理を強める必要がある.
\end{tcolorbox}

\begin{notation}\mbox{}
    \begin{enumerate}
        \item 集合の積を$AB$で,無縁和を$A+B$で表す.
        \item 試行$\fA^1,\fA^2$の積試行を$\fA^1\fA^2$で表す.
    \end{enumerate}
\end{notation}

\begin{definition}[independent, conditional probability, conditional expectation]
    $(\Om,\F,P)$を確率空間とする.
    \begin{enumerate}
        \item 試行とは,$\Om$の直和分割をいう.
        \item 試行の列$\mathfrak{A}^1,\cdots,\mathfrak{A}^n=(A^n_i)_{i\in[r_n]}$が\textbf{(互いに)独立}であるとは,次が成り立つことをいう:\footnote{事象$A$が独立とは,その事象が定める試行$\fA=A+A^\comp$が独立であることをいう.}
        \[\forall_{n\in\N}\;\forall_{k_1\in[r_1],\cdots,k_n\in[r_n]}\;P[A^1_{k_1}\cdots A^n_{k_n}]=P[A^1_{k_1}]\cdots P[A^n_{k_n}]\]
        \item 試行$\fA=(A_i)_{i\in[m]}\;\forall_{i\in[m]}\;P(A_i)>0$のあとの事象$B$の\textbf{条件付き確率}とは,次のように定まる$\sigma[\fA]$-可測でもある確率変数$P[B|\fA]$をいう:
        \[P[B|\fA](\om)=\sum^m_{i=1}P[B|A_i]1_{A_i}(\om).\]
        \item 試行$\fA=(A_i)_{i\in[m]}\;\forall_{i\in[m]}\;P(A_i)>0$のあとの確率変数$X$の\textbf{条件付き期待値}とは,次のように定まる$\sigma[\fA]$-可測でもある確率変数$E[X|\fA]$をいう:
        \[E[X|\fA](\om)=\sum_{i=1}^mE[X|A_i]1_{A_i}(\om)=\sum^m_{i=1}\frac{E[X1_{A_i}]}{P[A_i]}1_{A_i}(\om)\]
    \end{enumerate}
\end{definition}

\begin{lemma}[独立性の条件付き期待値による特徴付け]
    試行$\fA^1,\cdots,\fA^n$について,
    \begin{enumerate}
        \item 互いに独立である.
        \item $\forall_{k\in[n]}\;\forall_{i\in[r_k]}\;P[A_i^k|\fA^1\fA^2\cdots\fA^{(k-1)}]=P[A_i^k]$.
    \end{enumerate}
\end{lemma}

\section{Markov連鎖}

\begin{definition}
    確率変数列$(X_n)$が定める試行の列$(\fA^n)$について,
    \begin{enumerate}
        \item $\forall_{k\in[n]}\;\forall_{i\in[r_k]}\;P[A_i^k|\A^1\cdots\A^{k-1}]=P[A^k_i|\fA^{k-1}]$が成り立つとき,これを\textbf{Markov連鎖}という.
        \item $\forall_{k\in[n]}\;\forall_{i\in[r_k]}\;E[X_{n+1}|\A^1\cdots\A^n]=X_n\;\as$が成り立つとき,これを\textbf{martingale}という.
        \item 
    \end{enumerate}
\end{definition}

\section{無限の場合の独立性}

\begin{tcolorbox}[colframe=ForestGreen, colback=ForestGreen!10!white,breakable,colbacktitle=ForestGreen!40!white,coltitle=black,fonttitle=\bfseries\sffamily,
title=]
    $L^2(\Om)$上に制限して見ると,任意の$X\in L^2(\Om)$に対して,$\cG<\F$-可測関数のなす部分空間$L_\cG^2(\Om)\subset L_\cG^2(\Om)$への直交射影の値(のバージョン)として得られる$L_\cG^1(\Om)$の元を,条件付き期待値という.
    これは最小二乗の意味での最適推定値であるとも言える.
\end{tcolorbox}

\begin{definition}[conditional expectation, conditional probability, regular]
    $(\Om,\F,P)$を確率空間とし,
    $\cG$を$\F$の部分$\sigma$-代数とする.可積分確率変数$X\in L^1(\Om)$について,
    \begin{enumerate}
        \item 次の2条件を満たす,$P$-零集合を除いて一意な確率変数を\textbf{条件付き期待値}といい,$E[X|\cG]$で表す.
        \begin{enumerate}[(a)]
            \item $\cG$-可測でもある$P$-可積分確率変数である.
            \item 任意の$\cG$-可測集合$B\in\cG$上では$X$と期待値が同じ確率変数になる:
            $\forall_{B\in\cG}\;E[X1_B]=E[E[X|\cG]1_B]$.\footnote{これは2段階に分けて積分していると見れる.}
        \end{enumerate}
        \item $P[A|\cG]:=E[1_A|\cG]\;(A\in\F)$を\textbf{条件付き確率}というが,確率測度を定めるとは限らない.これが確率測度を定めるとき,\textbf{正則}条件付き確率という.\footnote{完備で可分な距離空間上のBorel確率空間上では存在と一意性が成り立つ.}
    \end{enumerate}
\end{definition}
\begin{remark}[正則条件付き確率]
    任意の互いに素な可測集合列$\{F_n\}\subset\F$について,条件付き期待値の線形性と単調収束定理より,
    \[P\Square{\sum F_n\middle|\cG}=E\Square{\sum 1_{F_n}\middle|\cG}=\sum E[1_{F_n}|\cG]=\sum P[F_n|\cG]\;\as\]
    が成り立つが,このときの零集合
    \[\cN:=\Brace{\om\in\Om\;\middle|\;P\Square{\sum F_n\middle|\cG}\ne \sum P[F_n|\cG]}\]
    が,任意の(おそらく非可算無限個ある)互いに素な可測集合列$\{F_n\}\subset\F$について,一様に零集合を取れるとは限らないが,
    「標準確率空間」については気にしなくてよい.
\end{remark}

\section{条件付き期待値の性質}

\begin{tcolorbox}[colframe=ForestGreen, colback=ForestGreen!10!white,breakable,colbacktitle=ForestGreen!40!white,coltitle=black,fonttitle=\bfseries\sffamily,
title=]
    $\cG$の元$B\in\cG$に対して,その立場の上での$X$の期待値$E[X1_B]$を返す符号付き測度を$Q:\cG\to\R$と表そう.
    するとその$P|_\cG$に関する密度関数が$E[X|\cG]$である.
    $E[-|\cG]:L^1(\Om)\to\R$は正な線型汎関数となっている.
\end{tcolorbox}

\begin{corollary}
    任意の可積分確率変数$X\in L^1(\Om)$に対して,条件付き期待値$E[X|\cG]$は存在し,零集合での差を除いて一意である.
\end{corollary}
\begin{proof}
    条件付き期待値は,$\cG$の元に対して,その立場の上での$X$の期待値を返す測度$Q:\cG\to\R$の,$(\Om,\cG)$上の確率密度関数であると見れば,
    Radon-Nikodymの定理の簡単な系である.
    任意の事象$B\in\cG$に対して,そのときの$X$の条件付き期待値を返す対応
    $Q(B):=E[1_BX]\;(B\in\cG)$は$(\Om,\cG)$上の測度である.
    これが$P|_{\cG}$に対して絶対連続であることに注意すれば良い:$P[B]=0\Rightarrow Q[B]=0$.
\end{proof}

\begin{proposition}
    $(\Om,\F,P)$を確率空間,$X\in L^1(\Om)$を可積分確率変数,$\cG,\H$を$\F$の$\sigma$-部分代数とする.
    \begin{enumerate}
        \item $Y$も条件付き期待値の定義を満たすとする.このとき,$E[Y]=E[X]$.
        \item $X$が$\cG$-可測であったならば,$E[X|\cG]=X\;\as$
        \item (線型) $E[-|\cG]$は$L^1(\Om)$上の線型汎関数である:$E[a_1X_1+a_2X_2|\cG]=a_1E[X_1|\cG]+a_2E[X_2|\cG]\;\as$
        \item (正) $X\ge0\Rightarrow E[X|\cG]\ge0$.
        \item (単調収束定理) $0\le X_n\nearrow X\Rightarrow E[X_n|\cG]\nearrow E[X|\cG];\as$
        \item (Fatouの補題) $X_n\ge0\Rightarrow E[\liminf X_n|\cG]\le\liminf E[X_n|\cG]\;\as$
        \item (優収束定理) $\forall_{n\in\N}\;\abs{X_n}\in L^1(\Om)$かつ$X_n\xrightarrow{\as}X$ならば,$E[X_n|\cG]\xrightarrow{\as}E[X|\cG]$.
        \item (Jensen) 凸関数$c:\R\to\R$に対して,$c(E[X|\cG])\le E[c(X)|\cG]\;\as$ 特に,$\norm{-}_p\;(p\ge1)$は凸関数であるから$\norm{E[X|\cG]}_p\le\norm{X}_p$.
        \item (Tower property) $\H<\cG\Rightarrow E[E[X|\cG]|\H]=E[X|\H]\;\as$
        \item (可測関数) $Z\in L^\infty(\Om,\cG)$のとき,$E[ZX|\cG]=ZE[X|\cG]\;\as$
        \item (独立性) $\H\perp\sigma[\sigma[X],\cG]\Rightarrow E[X|\sigma[\cG,\H]]=E[X|\cG]\;\as$ 特に,$X\perp\H\Rightarrow E[X|\H]=E[X]\;\as$
    \end{enumerate}
\end{proposition}
\begin{proof}\mbox{}
    \begin{enumerate}
        \item 条件付き期待値の一意性より,$Y=E[X|\cG]\;\as$.任意の$G\in\cG$について,条件付き期待値$E[X|\cG]$は$G$上では$X$と平均が等しいから,
        $E[Y1_G]=E[E[X|\cG]1_G]=E[X1_G]$が成り立つ.$G=\Om$と取れば良い.
        \item $X$は自身の条件付き期待値としての要件を満たすから,一意性より.
        \item 右辺の$a_1E[X_1|\cG]+a_2E[X_2|\cG]$も,$a_1X_1+a_2X_2$の$B\in\cG$上での期待値を与える測度$Q$の確率密度関数となっている.
    \end{enumerate}
\end{proof}

\section{関数空間CとD}

\subsection{ポーランド空間の位相の議論}

\begin{lemma}
    ポーランド空間(完備可分距離空間)の可算積はポーランド空間である.
\end{lemma}

\begin{theorem}
    $S$上のポーランド位相$\tau$とHausdorff位相$\tau_1$を考える.$\tau_1\subset\tau$ならば,これらが定めるBorel $\sigma$-代数は一致する:$\B_\tau(S)=\B_{\tau_1}(S)$.
\end{theorem}

\subsection{$D$空間とSkorokhod位相}

\begin{tcolorbox}[colframe=ForestGreen, colback=ForestGreen!10!white,breakable,colbacktitle=ForestGreen!40!white,coltitle=black,fonttitle=\bfseries\sffamily,
title=]
    関数解析では考察の対称となったことはないようであるが,確率論では$C$と同様に重要である.
    が,明らかに大きすぎる.
\end{tcolorbox}

\begin{notation}
    $T\subset\R$を開区間とする.
\end{notation}

\begin{theorem}
    $C(T)$は,
    \begin{enumerate}
        \item $T$がコンパクトであるとき,一様位相について,可分なBanach空間(特にポーランド空間)となる.
        \item $T$がコンパクトでないとき,広義一様収束位相について,可分なFrechet空間(特にポーランド空間)となる.
    \end{enumerate}
\end{theorem}

\begin{definition}
    $f:T\to\R$が\textbf{第1種不連続}または\textbf{cadlag}または右連続であるとは,$I$上の各点で,右連続かつ有限な左極限が存在することを言う.
    これらの全体を$D(T)$で表す.
\end{definition}

\begin{lemma}
    $C(T)\subset D(T)$であり,(広義)一様収束距離を用いて,同様の位相を入れることが出来る.
    このとき,
    この位相について,$D(T)$は完備であるが,可分でない.
\end{lemma}
\begin{proof}\mbox{}
    \begin{description}
        \item[完備] 一様収束によって,cadlag性は保たれる.
        \item[非可分性] $T:=[0,1]$とコンパクト集合をとっても,定義関数の集合$(1_{[0,\al)})_{\al\in[0,1]}$は$[0,1]$と同じ濃度を持つ非可算集合であるが,集積点を持たない:$\forall_{\al\ne\beta\in[0,1]}\;\norm{1_{[0,\al)}-1_{[0,\beta)}}_\infty=1$.
    \end{description}
\end{proof}

\begin{definition}[Skorohod topology]
    簡単のため$T$を有界閉区間とする.
    $T$の順序を保つ位相同型の全体の群を$\Phi(T)$とすると,$\Phi(T)\subset C(T)\subset D(T)$である.
    \begin{enumerate}
        \item 一様距離$\rho$について,$\forall_{f,g\in D(T)}\;\forall_{\varphi\in\Phi(T)}\;\rho(f,g)=\rho(f\circ\varphi,g\circ\varphi)$が成り立つ.
        \item 次のように$\rho_S$を定めると,これは距離になる:
        \[\rho_S(f,g):=\inf_{\varphi\in\Phi(T)}(\rho(f\circ\varphi,g)+\rho(\varphi,i))\]
        \item この距離$\rho_S$は完備ではないが,一様位相よりも弱い完備かつ可分な位相を定める.これを\textbf{Skorohod位相}という.
        \item $\Phi$のうち,$T$上のLipschitzノルム$\lambda:\Phi\to[0,\infty]$を有限にするもののなす部分集合$\Psi:=\Brace{\varphi\in\Phi\mid\lambda(\varphi)<\infty}$は部分群となる:
        \[\lambda(\varphi):=\sup_{s\ne t\in T}\Abs{\log\frac{\varphi(t)-\varphi(s)}{t-s}}\]
        なお,$\lambda$は対数関数によって,小さいほど$\varphi$の傾きが一様に$1$に近いことを意味するように構成してある.
        \item 次の距離$\rho_B$を\textbf{Billingsleyの距離}といい,Skorohod位相を定める完備な距離である:
        \[\rho_B(f,g):=\inf_{\psi\in\Psi(T)}(\rho(f\circ\psi,g)+\lambda(\psi)).\]
    \end{enumerate}
    $\forall_{f,g\in D(T)}\;\forall_{\varphi\in\Phi(T)}\;\rho(f,g)=\rho(f\circ\varphi,g\circ\varphi)$が成り立つことに注意して,
    \[\rho_S(f,g):=\inf_{\varphi\in\Phi(T)}(\rho(f\circ\varphi,g)+\rho(\varphi,i))\]
    は距離を定める.
\end{definition}

\begin{proposition}
    右連続な階段関数全体の集合は,$D(T)$内で一様収束位相について稠密である.
\end{proposition}

\subsection{Kolmogorov $\sigma$-代数}

\begin{tcolorbox}[colframe=ForestGreen, colback=ForestGreen!10!white,breakable,colbacktitle=ForestGreen!40!white,coltitle=black,fonttitle=\bfseries\sffamily,
title=]
    $\sigma$-代数の全体は完備束をなす.また,full setとnull setは$\delta$-環をなすが,それらの合併は$\sigma$-代数であり,2と表す\cite{伊藤清確率論}.
\end{tcolorbox}

\begin{definition}
    集合$T$上の関数の空間$\F\subset\Map(T,\R)$における$\sigma$-代数を考える.
    任意の$t\in T$に対して,射影$\ev_t:\F\to\R$が可測となるような$\F$上の$\sigma$加法族$\B$の中で最小のものを$\B_K(\F)$で表し,$\F$上の\textbf{Kolmogorov $\sigma$-加法族}という.
\end{definition}

\begin{lemma}
    $\B_K(\F)=\bigvee_{t\in T}\pi^{-1}_t(\B^1)$である.すなわち,Kolmogorov $\sigma$-代数は,$\{\pi^{-1}_t(\B^1)\}_{t\in T}$が束$P(\F)$の中でなす下限となる.
\end{lemma}

\begin{theorem}[一様位相,Skorohod位相のKolmogorov $\sigma$-代数との一致]\mbox{}
    \begin{enumerate}
        \item $\B(C(T))=\B_K(C(T))$.
        \item $\B(D(T))=\B_K(D(T))$.またこれは$D\subset L^1(T)$としての$L^1$-ノルムの位相が生成するBorel $\sigma$-代数とも一致する.
    \end{enumerate}
\end{theorem}

\section{確率過程に関する一般事項}

\subsection{3つの見方}

\begin{tcolorbox}[colframe=ForestGreen, colback=ForestGreen!10!white,breakable,colbacktitle=ForestGreen!40!white,coltitle=black,fonttitle=\bfseries\sffamily,
title=]
    2つの見方を,$\{X_t,t\in T\}$,または,$X.(\om),(X_t(\om),t\in T)$として表現している.
    いずれの見方も,$C$,$D$-過程については同値になる.
    一方で,$\Om\times T$上可測になることは特殊な性質で,これを満たす過程を\textbf{可測過程}という.Brown運動は可測過程である.
\end{tcolorbox}

\begin{lemma}
    可測関数の全体$\L(\Om)$について.
    \begin{enumerate}
        \item 距離$\rho_0(X,Y):=E[\abs{X-Y}\land 1]$について,完備可分な距離空間となる.
        \item \[\forall_{\ep\in(0,1)}\;\ep P[\abs{X-Y}>\ep]\le\rho_0(X,Y)\le\ep+P[\abs{X-Y}>\ep].\]
        \item この距離が定める位相は,確率収束を定める.
    \end{enumerate}
\end{lemma}

\begin{definition}[確率過程の連続性]
    確率過程$(X_t)_{t\in T}:T\to\L(\Om)$について,
    \begin{enumerate}
        \item $(\L(\Om),\rho_0)$について連続であるとき,\textbf{確率連続}であるという:$\forall_{s\in T}\;\forall_{\ep>0}\;\lim_{t\to s}P[\abs{X_t-X_s}>\ep]=0.$
        \item $(\L(\Om),\rho_0)$について一様連続であるとき,\textbf{一様確率連続}であるという.
    \end{enumerate}
    $T$がコンパクトであるとき,2つは同値.
\end{definition}

\begin{theorem}[2つのcurryingの等価性]
    次の2条件は同値.
    \begin{enumerate}
        \item 確率変数の集合$\{X_t\}_{t\in T}$は$C$-過程である:$\forall_{\om\in\Om}\;X.(\om)\in C(T)$.
        \item 見本過程に値を取る写像$\Om\to C(T)$として可測である.
    \end{enumerate}
    $C$を$D$に置き換えても成り立つ.
\end{theorem}

\begin{theorem}[可測過程]\mbox{}
    \begin{enumerate}
        \item 可測写像$\Om\times T\to\R$は,確率過程$T\to\L(\Om)$を定める.
        \item 過程$T\to\L(\Om)$は可測過程であるとする.このとき,見本道の確率変数$\Om\to\L(T)$が定まる.
        \item $C$過程と$D$過程は可測過程である.
    \end{enumerate}
\end{theorem}

\subsection{確率過程の同値性}

\begin{definition}[equivalence / version, finite dimensional marginal distributions]\mbox{}
    \begin{enumerate}
        \item 同じ状態空間$(E,\E)$を持つ$(\Om,\F,P),(\Om',\F',P')$上の2つの過程$X,X'$が\textbf{同値である}または一方が他方の\textbf{バージョン}であるまたは法則同等\cite{伊藤清確率論}であるとは,
        任意の有限部分集合$\{t_1,\cdots,t_n\}\subset\R_+$と任意の可測集合$A_1,\cdots,A_n\in\E$について,
        \[P[X_{t_1}\in A_1,\cdots,X_{t_n}\in A_n]=P'[X'_{t_1}\in A_1,\cdots,X'_{t_n}\in A_n]\]
        \item 測度$P$の$(X_{t_1},\cdots,X_{t_n}):\Om\to E^n$による押し出しを$P_{t_1,\cdots,t_n}:=P^{(X_{t_1},\cdots,X_{t_n})}$で表す.
        任意の有限集合$\{t_1,\cdots,t_n\}\subset\R_+$に関する押し出し全体の集合$\M_X$を\textbf{有限次元分布}(f.d.d)と呼ぶ.
    \end{enumerate}
\end{definition}
\begin{lemma}[過程の同値性の特徴付け]
    $X,Y$について,次の3条件は同値.
    \begin{enumerate}
        \item $X,Y$は同値である.
        \item $\M_X=\M_Y$.
    \end{enumerate}
\end{lemma}

\begin{definition}[modification, indistinguishable]
    定義された確率空間も状態空間も等しい2つの過程$X,Y$について,
    \begin{enumerate}
        \item 2つは\textbf{修正}または\textbf{変形}または同等\cite{伊藤清確率論}であるとは,$\forall_{t\in\R_+}\;X_t=Y_t\;\as$を満たすことをいう.\footnote{\cite{Nualart}ではこの概念をequivalenceまたはvarsionと呼んでいる.}
        \item 2つは\textbf{識別不可能}または強同等\cite{伊藤清確率論}であるとは,殆ど至る所の$\om\in\Om$について,$\forall_{t\in\R_+}\;X_t(\om)=Y_t(\om)$が成り立つことをいう.
    \end{enumerate}
\end{definition}

\begin{lemma}\mbox{}
    \begin{enumerate}
        \item $X,Y$が互いの修正であるならば,同値である.
        \item $X,Y$が互いの修正であり,見本道が殆ど確実に右連続ならば,識別不可能である.
    \end{enumerate}
\end{lemma}

\begin{theorem}
    2つの$C$過程または$D$過程が同値であるならば,見本道の空間$C(T),D(T)$に押し出す確率測度は等しい.
\end{theorem}

\section{情報と情報増大系}

\begin{tcolorbox}[colframe=ForestGreen, colback=ForestGreen!10!white,breakable,colbacktitle=ForestGreen!40!white,coltitle=black,fonttitle=\bfseries\sffamily,
title=]
    情報は$\sigma$-部分代数で,データは確率変数で表すとしたら,2つの構造が何らかの意味で整合して居る必要がある.これを適合的という.
\end{tcolorbox}

\subsection{閉$\sigma$-代数}

\begin{tcolorbox}[colframe=ForestGreen, colback=ForestGreen!10!white,breakable,colbacktitle=ForestGreen!40!white,coltitle=black,fonttitle=\bfseries\sffamily,
title=]
    確率変数$X$に対して,これが$\Om$上に定める分割が生成する最小の閉$\sigma$-代数を$\F[X]<\D(P)$で表すこととしよう.
\end{tcolorbox}

\begin{notation}
    $(\Om,\D(P),P)$上の,$\D(P)$の部分$\sigma$-代数であって,すべての$P$-零集合を含むものを\textbf{閉$\sigma$-代数}または\textbf{情報}といい,$\Phi=\Phi(\Om,P)=\Brace{\B\lor2<\D(P)\mid\B<\D(P)}$でらわす.
\end{notation}

\begin{definition}\mbox{}
    \begin{enumerate}
        \item 可測関数$X:\Om\to S$について,$\F[X]:=X^{-1}(\D(P^X))\lor 2$を,$X$で生成される閉$\sigma$-代数という.
        これは,$X$を可測にする閉$\sigma$-代数の中で最小のものである.
        \item 確率変数の族$\{X_\lambda\}_{\lambda\in\Lambda}$については,$\F[X_\lambda,\lambda\in\Lambda]:=\bigvee_{\lambda\in\Lambda}\F[X_\lambda]$と表す.
    \end{enumerate}
\end{definition}

\begin{theorem}
    $X,Y\in\L(\Om)$について,$X\prec Y\;\as:\Leftrightarrow [\exists_{\varphi\in\Map(\Om,\Om)}\;X=\varphi\circ Y\;\as]$と表すと,これは同値類$\sim$とその上の順序を定め,
    \begin{enumerate}
        \item $Y\prec X\;\as\Leftrightarrow\F[Y]\subset\F[X]$.
        \item $Y\sim X\;\as\Leftrightarrow\F[Y]=\F[X]$.
    \end{enumerate}
\end{theorem}

\subsection{情報増大系}

\begin{tcolorbox}[colframe=ForestGreen, colback=ForestGreen!10!white,breakable,colbacktitle=ForestGreen!40!white,coltitle=black,fonttitle=\bfseries\sffamily,
title=]
    情報系$(\F[X_t])_{t\in T}$について,過去の記憶を取り$(\F[X_s;s\le t])_{t\in T}$とすれば単調増大になり,さらに$\paren{\F_t:=\bigcap_{s>t}\F[X_u;u\le s]}_{t\in T}$とすれば右連続にもなるから,特に意識せず情報系(filtration)と呼ぶこととする.
    また,任意の情報系は,ある実過程が生成することに注意.
\end{tcolorbox}

\begin{definition}[filtration]\mbox{}
    \begin{enumerate}
        \item $T$に関する\textbf{情報増大系}$\{F_t\}_{t\in T}\subset\Phi(\Om,P)$とは,
        \begin{enumerate}[(a)]
            \item 広義単調増大性:$\forall_{s,t\in T}\;s<t\Rightarrow\F_s\subset\F_t$
            \item 右連続性:$\F_t=\F_{t+}:=\bigvee_{s>t}\F_s$
        \end{enumerate}
        を満たす閉$\sigma$-代数の族をいう.
        \item 任意の広義単調増大性を満たす系$(\F_t)$に対して,$(\F_{t+})_{t\in T}$は情報増大系である.これを\textbf{右連続化}という.
        \item 確率過程$(X_t)$に対して,$\paren{\F_t:=\bigcap_{s>t}\F[X_u;u\le s]}$を\textbf{$X$が生成する情報増大系}といい,$F[X]:=(\F_t[X])_{t\in T}$で表す.
        \item 確率過程$(X_t)$が\textbf{右連続}であるとは,$D$-過程であることをいう.
    \end{enumerate}
\end{definition}

\begin{definition}[adapted, predictable]
    $F=(F_t)_{t\in T}$を情報系,$X=(X_t)_{t\in T}$を確率過程とする.
    \begin{enumerate}
        \item $\forall_{t\in T}\;X_t\in\L(\Om,\F_t)$のとき,$X$は$F$に\textbf{適合}するという.これは$\forall_{t\in T}\;\F[X_t]\subset\F_t$に同値.
        \item $\forall_{t\in T}\;X_t\in\L(\Om,\F_{t-1})$のとき,$X$は$F$で\textbf{可予測}であるという.$X_t=E[X|\F_{t-1}]$より,右辺から計算可能になる.
    \end{enumerate}
\end{definition}

\begin{example}[canonical filtration]
    $\F_t:=\cap_{\ep>0}\sigma[X_s\mid s\le t+\ep]$と定めると,右連続で適合的な$\sigma$-部分代数となる.
    これを\textbf{自然な情報系}という.
\end{example}

\begin{proposition}
    $D$-過程$(X_t)_{t\in T}$について,これが過程として生成する情報系と,$D$-値確率変数として生成する情報系とは等しい:
    $\F[X_t,t\in T]=\F[X.]$.
\end{proposition}

\section{停止時}

\begin{tcolorbox}[colframe=ForestGreen, colback=ForestGreen!10!white,breakable,colbacktitle=ForestGreen!40!white,coltitle=black,fonttitle=\bfseries\sffamily,
title=]
    $T$に値を取る確率変数のうち,試行列$\{\tau\le t\}$(いうならば確率過程$(1_{\tau\le t})_{t\in T}$)が$(\F_t)$-適合的でなければ,これはモデルとして認められない場合が多い(大損してからやっぱなかったことにしてほしいとは言えない).
\end{tcolorbox}

\subsection{定義と例}

\begin{definition}[Markov time / stopping time]\mbox{}
    \begin{description}
        \item[離散] 確率変数$\tau:\Om\to\o{\N}$\footnote{$\B(\o{\N})=P(\o{\N})$に注意}が$(\F_n)$-\textbf{Markov時刻}または\textbf{停止時刻}であるとは,
        \[\forall_{n\in\N}\quad\{\tau\le n\}:=\Brace{\om\in\Om\mid\tau(\om)\le n}\in\F_n\]
        を満たすことをいう.
        \item[連続] 確率変数$\tau:\Om\to\o{\R_+}$が$(\F_t)$-\textbf{Markov時刻}または\textbf{停止時刻}であるとは,
        \[\forall_{t\ge0}\;\{\tau\le t\}:=\Brace{\om\in\Om\mid\tau(\om)\le t}\in\F_t\]
        を満たすことを言う.
    \end{description}
\end{definition}

\subsubsection{離散の場合}

\begin{lemma}[離散の場合の特徴付け]
    $\tau:\Om\to\o{\N}$について,次の2条件は同値.
    \begin{enumerate}
        \item $\tau$はMarkov時刻である.
        \item $\forall_{n\in\N}\;\{\tau=n\}\in\F_n$である.
    \end{enumerate}
\end{lemma}

\begin{lemma}[離散停止時の構成]
    停止時$\tau,\tau_1,\tau_2:\Om\to\o{\N}$について,
    \begin{enumerate}
        \item $\tau+m\;(m\in\o{\N})$も停止時だが,一般に$\tau-m$は停止時とは限らない.
        \item $\tau_1\lor\tau_2,\tau_1\land\tau_2$も停止時である.
        \item $\tau_1+\tau_2$も停止時である.
    \end{enumerate}
\end{lemma}


\begin{example}[離散の例]\label{exp-discrete-Markov-time}\mbox{}
    \begin{enumerate}
        \item 定値関数はMarkov時刻である.
        \item \textbf{到達時刻}(first hitting time)とは,$(\F_n)$-適合確率過程$(X_n)$対して,任意の事象$A\in\B(\R)$に対し,
        \[\tau_A(\om):=\min\Brace{n\in\o{\N}\mid X_n(\om)\in A}\]
        で定まる時刻である.ただし,$\min\emptyset=\infty$とする.
        $\{\tau_A\le n\}=\bigcup_{i\in[n]}\Brace{X_i\in A}$より,Markov時刻である.
        \item ある一定額$\al$以上を賭けたら即座に賭けを中止すると決めているとき,この時刻はMarkov時刻である.
        \item \textbf{最終脱出時刻}(last exit time)とは,$(\F_n)$-適合確率過程$(X_n)$対して,任意の事象$A\in\B(\R)$に対し,
        \[\sigma_A(\om):=\max\Brace{n\in\o{\N}\mid X_n(\om)\in A}+1\]
        とすると,これは確率変数ではあるが,Markov時刻にはならない.「これが最後か?」を判定するには,さらに先の情報が必要だからである.
    \end{enumerate}
\end{example}

\subsubsection{連続の場合}

\begin{tcolorbox}[colframe=ForestGreen, colback=ForestGreen!10!white,breakable,colbacktitle=ForestGreen!40!white,coltitle=black,fonttitle=\bfseries\sffamily,
title=]
    離散の場合と異なる点は,$\forall_{t\ge0}\;\Brace{\tau=t}\in\F_t$だけは同値にならない(非可算和を取らないと得られない情報)ことである.
\end{tcolorbox}

\begin{lemma}[連続の場合の特徴付け]
    $\tau:\Om\to[0,\infty]$について,次の2条件は同値.
    \begin{enumerate}
        \item $\tau$はMarkov時刻である.
        \item $\forall_{t\ge0}\;\Brace{\tau<t}\in\F_t$.
        \item $\forall_{t\ge0}\;\Brace{\tau>t}\in\F_t$.
        \item $\forall_{t\ge0}\;\Brace{\tau\ge t}\in\F_t$.
    \end{enumerate}
\end{lemma}

\begin{example}
    $(X_t)$の集合$A\subset\R^d$への到達時刻は
    \[\tau_A(\om):=\inf\Brace{t>0\mid X_t(\om)\in A}\]
    で定める.$A$が開または閉であるとき,$\tau_A$はMarkov時刻になる.
\end{example}

\begin{lemma}
    $\tau:\Om\to\R_+$を確率変数,$(F_t)$を情報系とする.
    \begin{enumerate}
        \item $\tau$は停止時である.
        \item 確率過程$(X_t:=1_{\Brace{t\le\tau}})$は$(F_t)$-適合的である.
    \end{enumerate}
\end{lemma}

\subsection{情報量}

\begin{tcolorbox}[colframe=ForestGreen, colback=ForestGreen!10!white,breakable,colbacktitle=ForestGreen!40!white,coltitle=black,fonttitle=\bfseries\sffamily,
title=]
    過程を事前に決めた規則でランダムに止める過程も,再びたしかに確率過程となる.
    この過程が定める情報を,$\tau$までの情報量という.
    すなわち,止めた前か後かが判別可能であることをいう.
\end{tcolorbox}

\begin{definition}[stopped process, information]
    $\tau$を$(\F_t)$-停止時,$(X_t)_{t\in T}$を$(\F_t)$-適合過程とする.
    \begin{enumerate}
        \item $(X_{t\land\tau(\om)}(\om))_{t\in T}$はを,\textbf{時点$\tau$で止めた過程}という.これはたしかに確率過程になる.
        \item \[\F_\tau:=\Brace{A\in\F\mid\forall_{t\in T}\;A\cap\Brace{\tau\le t}\in\F_t}=\F[X_{t\land\tau(\om)}(\om);t\in T]\]
        を\textbf{時点$\tau$までの情報量}という.
    \end{enumerate}
\end{definition}
\begin{lemma}[well-definedness]
    $T=\o{\N}$と事象$A\subset\F$について,
    次の3条件は同値.
    \begin{enumerate}
        \item $A\in\F_t$.
        \item $A\in\F_\infty$かつ$\forall_{n\in\N}\;A\cap\{\tau=n\}\in\F_n$.
        \item $\forall_{n\in\o{\N}}\;A\cap\{\tau=n\}\in\F_n$.
    \end{enumerate}
\end{lemma}
\begin{example}
    $\tau$が定数$m$のとき,$\F_\tau=\F_m$となる.
\end{example}

\begin{theorem}
    $\tau,\sigma$をMarkov時刻とする.次が成り立つ.
    \begin{enumerate}
        \item $\F_\tau$は$\sigma$-代数である.
        \item $\tau$は$\F_\tau$-可測である.
        \item $\tau\le\sigma$ならば$\F_\tau\subset\F_\sigma$.
    \end{enumerate}
\end{theorem}

\begin{definition}[predictable / announcable, accessible, totally inaccessible]
    停止時$\tau$について,
    \begin{enumerate}
        \item \textbf{可予測}または\textbf{事前通告可能}であるとは,ある$\forall_{\tau>0}\;\tau_n<\tau$を満たす停止時の増大列$(\tau_n)$の極限であることをいう.
        \item \textbf{到達可能}であるとは,ある停止時の列$(\tau_n)$が存在して,殆ど確実に$\exists_{n\in\N}\;\tau_n=\tau$が成り立つことをいう.
        \item \textbf{到達不可能}であるとは,任意の可予測な時刻$\sigma$に対して,$P[\tau=\sigma<\infty]=0$が成り立つことをいう.
    \end{enumerate}
\end{definition}
\begin{example}\mbox{}
    \begin{enumerate}
        \item 適合的な過程の到達時刻(hitting time)は可予測である.
        \item Poisson過程のジャンプ時刻は到達不可能である.
    \end{enumerate}
\end{example}

\subsection{停止時の分解}

\begin{theorem}[停止時の分解]
    
\end{theorem}

\subsection{停止時による局所化}

\begin{tcolorbox}[colframe=ForestGreen, colback=ForestGreen!10!white,breakable,colbacktitle=ForestGreen!40!white,coltitle=black,fonttitle=\bfseries\sffamily,
title=]
    $\tau$で止める確率過程を$X^\tau_t:=X_{\min(t,\tau)}$と表すと,これを用いて種々の性質を局所化出来る.
\end{tcolorbox}

\begin{definition}[locally martingale, locally integrable]\mbox{}
    \begin{enumerate}
        \item $D$-過程が\textbf{局所martingale}であるとは,$\infty$に収束する停止時の増大列$(\tau_n)$が存在して,任意の$n\in\N$について$1_{\tau_n>0}X^{\tau_n}$がmartingaleになることをいう.
        \item 非負な増大過程が\textbf{局所可積分}であるとは,$\infty$に収束する停止時の増大列$(\tau_n)$が存在して,$\forall_{n\in\N}\;1_{\tau_n>0}X^{\tau_n}\in L^1(\Om)$を満たすことをいう.
    \end{enumerate}
\end{definition}

\chapter{マルチンゲール}

\begin{quotation}
    連続確率過程のマルチンゲールは,離散化したあとに適当な連続極限を取ることで,離散の場合の議論に帰着させることが出来る.
\end{quotation}

\section{離散時変数のマルチンゲール}

\subsection{定義}

\begin{definition}[martingale, submartingale]
    確率過程$(X_n)$が\textbf{情報系$(\F_n)$についてマルチンゲール}であるとは,次の3条件が成り立つことをいう:
    \begin{enumerate}
        \item $(\F_n)$-適合的である:$\forall_{n\in\N}\;X_n:\F_n$ measurable.
        \item 可積分列である:$\forall_{n\in\N}\;E[\abs{X_n}]<\infty$.
        \item martingale性:$\forall_{n\in\N}\;E[X_{n+1}|\F_n]=X_n\;\as$
    \end{enumerate}
    (3)の代わりに$\forall_{n\in\N}\;E[X_{n+1}|\F_n]\ge X_n\;\as$が成り立つとき,\textbf{劣マルチンゲール}であるといい,
    $\forall_{n\in\N}\;E[X_{n+1}|\F_n]\le X_n\;\as$が成り立つとき,\textbf{優マルチンゲール}であるという.
\end{definition}
\begin{remarks}
    (3)は$\forall_{A\in\F_n}\;E[X_{n+1}1_A]=E[X_n1_A]$と同値.
    これが成り立つならば,繰り返し期待値の法則より,$m>n\ge1\Rightarrow E[X_m|\F_n]=X_n\;\as$であり,特に,$E[X_n]$は$n$に依らず一定である.
    また,$(X_n)$が劣マルチンゲールであることと,$(-X_n)$が優マルチンゲールであることとは同値.
\end{remarks}

\begin{example}\mbox{}
    \begin{enumerate}
        \item $\{Z_n\}\subset\L^1(\Om)$を期待値$0$かつ独立な確率変数列とし,$\F_n$として自然な情報系を取る.和$X_n:=\sum^n_{k=1}Z_k$はマルチンゲールである.
        $E[X_{n+1}|\F_n]=E[X_n+Z_{n+1}|\F_n]=E[X_n|\F_n]+E[Z_{n+1}|\F_n]$であるが,第1項は$X_n$は$\F_n$可測であるから,$E[X_n|\F_n]=X_n$.また,$\sigma(Z_{n+1})$と$\F_n$は独立だから\ref{lemma-independentness-variable-and-algebra},$E[Z_{n+1}|\F_n]=E[Z_{n+1}]=0\;\as$.
        これはKolmogorovの不等式\ref{thm-Kolmogorov-inequality}ですでにあった消息である.
        \item $(\F_n)$を情報系とし,$X\in\L^1(\Om)$を可積分確率変数とする.$X_n:=E[X|\F_n]$とおけば,$(X_n)$はマルチンゲールである.
        実際,$E[X_{n+1}|\F_n]=E[E[X|\F_{n+1}]|\F_n]\overset{\as}{=}E[X|\F_n]=X_n$.
    \end{enumerate}
    (1)の状況は公平な賭けなどの意味論を持つ.コイントスをして,表なら$+x$円,裏なら$-x$円の賭けで,所持金を$X_n$とすると,これはマルチンゲールである.
\end{example}

\begin{lemma}[劣マルチンゲール性の保存]\mbox{}
    \begin{enumerate}
        \item $\psi:\R\to\R$は下に凸,$(X_n)$をマルチンゲールとする.このとき,$\forall_{n\in\N}\;E[\abs{\psi(X_n)}]<\infty$ならば,$(\psi(X_n))$は劣マルチンゲールである.
        特に,ある$p\ge1$に関して$E[\abs{X_n}^p]<\infty$ならば,$(\abs{X_n}^p)$は劣マルチンゲールである.
        \item 下に凸な関数$\psi:\R\to\R$はさらに広義単調増加であるならば,$(X_n)$が劣マルチンゲールの場合でも,$(\psi(X_n))$は劣マルチンゲールになる.
    \end{enumerate}
\end{lemma}

\subsection{Doob分解}

\begin{theorem}[Doob-Meyer decomposition theorem]
    任意の$(\F_n)$-劣マルチンゲール$(X_n)$は,$\F$-マルチンゲールな$M=(M_n)$と可予測な広義増加過程$A=(A_n)$,すなわち,$0=A_0\le A_1\le\cdots,A_n\in L^1(\F_{n-1})\;(n=1,2,\cdots)$\footnote{マルチンゲールに対して$A_n\in L^1(\F_{n-1})$とは,$\F_{n-1}$-可測の意味しかない.}を満たす列$(A_n)$とに一意的に分解される:$X_n=M_n+A_n\;\as$.
\end{theorem}

\subsection{Doobの任意抽出定理}

\begin{tcolorbox}[colframe=ForestGreen, colback=ForestGreen!10!white,breakable,colbacktitle=ForestGreen!40!white,coltitle=black,fonttitle=\bfseries\sffamily,
title=]
    2つのランダム関数$X,\tau$の交錯を考える.
\end{tcolorbox}

\begin{definition}
    $(\F_n)$-適合な確率過程$(X_n)$と,$\N$-値Markov時刻$\tau$に対して,$X_\tau:\Om\to\R$を$X_\tau(\om):=X_{\tau(\om)}(\om)$で定める.
\end{definition}

\begin{lemma}
    $X_\tau:\Om\to\R$は
    \begin{enumerate}
        \item $\F$-可測である.
        \item $\F_\tau$-可測である.
    \end{enumerate}
\end{lemma}
\begin{proof}\mbox{}
    \begin{enumerate}
        \item $X_\tau$は可測関数の合成$X\circ (\tau,\id):\Om\to\N\times\Om\to\R$であるため.
    \end{enumerate}
\end{proof}

\begin{theorem}[有界停止時刻によるマルチンゲール性の保存]
    $(X_n)$を劣マルチンゲール,$\tau,\sigma$を$\tau\le\sigma$を満たす有界なMarkov時刻とする.このとき,$X_\tau,X_\sigma$は共に可積分で,$E[X_\sigma|\F_\tau]\ge X_\tau\;\as$.
    $(X_n)$がマルチンゲールであるとき,等号成立.
\end{theorem}

\begin{corollary}[optional sampling theorem]
    $(X_n)$を$(\F_n)$-劣マルチンゲール,$(\tau_k)$を有界な$(\F_n)$-マルコフ時刻の広義単調増加列とする.
    このとき,$Y_k:=X_{\tau_k}$は$(\F_{\tau_k})$-劣マルチンゲールである.
\end{corollary}

\subsection{Doobの不等式}

\begin{tcolorbox}[colframe=ForestGreen, colback=ForestGreen!10!white,breakable,colbacktitle=ForestGreen!40!white,coltitle=black,fonttitle=\bfseries\sffamily,
title=]
    劣マルチンゲールに対しては,$\max_{1\le k\le n}X_K$に関する評価を,$X_n$のみを用いて与えられる.
    一般の確率過程では決して成り立たない.
    この背後にはKolmogorovの不等式\ref{thm-Kolmogorov-inequality}がある.
\end{tcolorbox}

\begin{theorem}[Doob inequality]\label{thm-Doob-inequality}
    $(X_n)$を劣マルチンゲールとする.このとき,$X_n^+:=X\lor0$とすると,任意の$a>0$について,次が成り立つ.
    \begin{enumerate}
        \item \[P\paren{\max_{1\le k\le n}X_k\ge a}\le\frac{1}{a}E\Square{X_n1_{\Brace{\max_{1\le k\le n}X_k\ge a}}}\le\frac{1}{a}E[X_n^+]\]
        \item \[P\paren{\min_{1\le k\le n}X_k\le -a}\le\frac{1}{a}E[X_n-X_1]-\frac{1}{a}E\Square{X_n1_{\Brace{\min_{1\le k\le n}X_k\le -a}}}\le\frac{1}{a}E[X_n^+]-\frac{1}{a}E[X_1].\]
    \end{enumerate}
\end{theorem}

\begin{corollary}
    $\{M_n\}\subset\L^p(\Om)$を$p\ge1$乗可積分なマルチンゲールとする.
    このとき,任意の$a>0$に対して,
    \[P\paren{\max_{1\le k\le n}\abs{M_k}\ge a}\le\frac{1}{a^p}E[\abs{M_n}^p].\]
\end{corollary}

\subsection{劣マルチンゲールの収束定理}

\begin{tcolorbox}[colframe=ForestGreen, colback=ForestGreen!10!white,breakable,colbacktitle=ForestGreen!40!white,coltitle=black,fonttitle=\bfseries\sffamily,
title=]
    劣マルチンゲールの正部分の期待値が「有界」ならば,$X_n$は概収束極限を持つ.
    その証明では,劣マルチンゲールの上向き横断回数の評価が肝要になる.
\end{tcolorbox}

\subsubsection{martingale変換}

\begin{definition}[martingale transformation]
    可予測な過程$(H_n)$と$(\F_n)$-適合的な過程$(X_n)$に対して,新たな確率過程$(X'_n):=((H\cdot X)_n)$を次のように定める
    \[X'_n=(H\cdot X)_n:=\begin{cases}
        \sum^n_{k=2}H_k(X_k-X_{k-1}),&n\ge 2,\\
        0,&n=1.
    \end{cases}\]
    $(H\cdot X)_n$を$X_n$の\textbf{マルチンゲール変換}という.
    連続時間の場合は,確率積分$\int^t_0HdX$となる.
\end{definition}
\begin{remarks}
    $(H_n)$は戦略を表し,$(X_n)$は$\Z$上のランダムウォークとすれば,これによる変換$(H\cdot X)_n$は$n$時に所持している利益分の金額となる.
\end{remarks}

\begin{example}
    倍賭けの戦略は,次のように表せる.
    \[H_n:=\begin{cases}
        2H_{n-1},&Z_{n-1}=-1,\\
        1,&Z_{n-1}=1.
    \end{cases}\]
\end{example}

\begin{theorem}
    可予測な確率過程$(H_n)$は有界な列とする:$\forall_{n\in\N}\;\sup_{\om\in\Om}\abs{H_n(\om)}<\infty$.
    このとき,次が成り立つ.
    \begin{enumerate}
        \item $(X_n)$がマルチンゲールならば,$(X'_n)=((H\cdot X)_n)$もマルチンゲールである.
        \item $(X_n)$が劣マルチンゲールで,$(H_n)$が非負ならば,$(X'_n)=((H\cdot X)_n)$も劣マルチンゲールである.
    \end{enumerate}
\end{theorem}

\subsubsection{上渡回数定理}

\begin{tcolorbox}[colframe=ForestGreen, colback=ForestGreen!10!white,breakable,colbacktitle=ForestGreen!40!white,coltitle=black,fonttitle=\bfseries\sffamily,
title=]
    マルチンゲールは,(少なくとも期待値については)単調に増加する傾向があり,いつまでも区間$[a,b]$付近にとどまってはいないか,概収束をする.
\end{tcolorbox}

\begin{definition}[upcrossing number]\mbox{}
    \begin{enumerate}
        \item 実数$a<b$について,列$(\sigma_j)$を次のように定めると,Markov時刻の狭義単調増加列となる:
        \begin{align*}
            \sigma_1&:=\min\Brace{n\ge 1\mid X_n\le a},&\sigma_2&:=\min\Brace{n>\sigma_1\mid X_n\ge b},\\
            \sigma_{2k+1}&:=\min\Brace{n>\sigma_{2k}\mid X_n\le a},&\sigma_{2k+2}&:=\min\Brace{n>\sigma_{2k+1}\mid X_n\ge b}.
        \end{align*}
        \item Markov時刻の狭義単調増加列に対して,$U_n:=\max_{k\in\N\mid\sigma_{2k}\le n}$と定めると,$\N$-値確率変数の列となる.
        成分$U_n$を,\textbf{時刻$n$までの$a\nearrow b$間の上向き横断回数}という.
    \end{enumerate}
\end{definition}

\begin{theorem}
    $(X_n)$が劣マルチンゲールならば,
    \[E[U_n]\le\frac{1}{b-a}E[(X_n-a)^+].\]
\end{theorem}

\subsubsection{劣マルチンゲールの収束定理}

\begin{tcolorbox}[colframe=ForestGreen, colback=ForestGreen!10!white,breakable,colbacktitle=ForestGreen!40!white,coltitle=black,fonttitle=\bfseries\sffamily,
title=]
    劣マルチンゲールに対しても,有界列は収束することに対応する結果が成り立つ.
\end{tcolorbox}

\begin{theorem}
    劣マルチンゲール$(X_n)$は$\sup_{n\in\N}E[X_n^+]<\infty$を満たすとする.このとき,ある確率変数$X$が存在して$X=\lim_{n\to\infty}X_n\;\as$かつ可積分$X\in\L^1(\Om)$である.
\end{theorem}
\begin{remarks}
    劣マルチンゲールに対して,有界性条件$\sup_{n\in\N}E[X_n^+]<\infty$は,平均の一様有界性$\sup_{n\in\N}E[\abs{X_n}]<\infty$に同値.
\end{remarks}

\subsection{積率不等式}

\begin{tcolorbox}[colframe=ForestGreen, colback=ForestGreen!10!white,breakable,colbacktitle=ForestGreen!40!white,coltitle=black,fonttitle=\bfseries\sffamily,
title=]
    Doobの不等式\ref{thm-Doob-inequality}を,マルチンゲールの$p$次のモーメントに関する評価式に書き直せる.
\end{tcolorbox}

\subsubsection{Doobの不等式の一般化}

\begin{theorem}
    $p>1$について,$\{M_n\}\subset\L^p(\Om)$を$p$乗可積分なマルチンゲールとする.このとき,
    \[E\Square{\max_{1\le k\le n}\abs{M_k}^p}\le\paren{\frac{p}{p-1}}^pE[\abs{M_n}^p].\]
\end{theorem}

\subsubsection{Burkholderの不等式}

\begin{notation}
    $\{M_n\}\subset\L^2(\Om)$を,$M_0=0$を初項とする2乗可積分なマルチンゲールとする.
    このとき,$\forall_{n\in\N}\;E[M_n]=0$である.
\end{notation}

\begin{definition}
    $p\ge 1$について,
    マルチンゲール$(M_n)$の\textbf{$p$次変分}または\textbf{$p$次変動}とは,次で定まる実数列$([M]_n)$をいう:
    \[[M]_n:=\sum^n_{k=1}\abs{M_k-M_{k-1}}^p.\]
    特に$p=1$のとき,\textbf{変分}あるいは\textbf{全変動}という.
\end{definition}

\begin{proposition}
    2次変分$[M]_n$は,次の2条件をみたす:
    \begin{enumerate}
        \item $(M_n^2-[M]_n)$はマルチンゲールである.
        \item $([M]_n)$は増加過程である:$0=[M]_0\le[M]_1\le\cdots$.
    \end{enumerate}
\end{proposition}

\begin{remark}
    $(M^2_n)$は劣マルチンゲールだから,Doob分解$M_n^2=N_n+A_n$を持つ.このとき,$(M_n^2-A_n)$はマルチンゲールであるが,$(A_n)$も命題の2条件を満たす.
    $(A_n)$も$(M_n)$の2次変分と呼び,$(\brac{M}_n)$で表す.
    $(\brac{M}_n)$は可予測でもあるが,一般に$([M]_n)$はそうではない.明確な区別が必要である.
    一方で,連続マルチンゲールにおいては,2つの概念は1つに退化する.
\end{remark}

\begin{theorem}[Burkholder-Davis-Gundy]
    $(M_n)$を$M_0=0$を満たす$p$乗可積分なマルチンゲールとする.
    このとき,次が成り立つ:
    \[\forall_{p\ge 1}\;\exists_{c_p,C_p>0}\quad c_pE\Square{[M]^{p/2}_n}\le E\Square{\max_{1\le k\le n}\abs{M_k}^p}\le C_pE\Square{[M]^{p/2}_n}.\]
\end{theorem}
\begin{remarks}
    右辺は,Doobの不等式の$E[\abs{M_n}^p]$を$E[[M]_n^{p/2}]$で置き換えたものになっている.$p=2$のとき両者は一致するが,応用上は2次変分の方が計算しやすいことが多い.
\end{remarks}

\section{連続時変数のマルチンゲール}

\subsection{定義}

\begin{tcolorbox}[colframe=ForestGreen, colback=ForestGreen!10!white,breakable,colbacktitle=ForestGreen!40!white,coltitle=black,fonttitle=\bfseries\sffamily,
title=]
    離散の場合のマルチンゲールは可積分性を仮定していたが,その全貌は右連続性である.
\end{tcolorbox}

\begin{definition}
    $D$-過程$(X_t)$が$(\F_t)$-マルチンゲールであるとは,次の3条件を満たすことをいう.
    \begin{enumerate}
        \item $(\F_t)$-適合である:$\forall_{t\ge0}\;X_t$は$\F_t$-可測.
        \item 可積分である:$\forall_{t\ge0}\;X_t\in\L^1(\Om,\F)$.
        \item $\forall_{0\le s\le t}\;E[X_t|\F_s]=X_s\;\as$
    \end{enumerate}
    条件(3)の代わりに$\forall_{0\le s\le t}\;E[X_t|\F_s]\ge X_s\;\as$をみたすとき,$(\F_t)$-劣マルチンゲールという.
\end{definition}

\subsection{Doobの不等式}

\subsection{Doobの任意抽出定理}

\subsection{劣マルチンゲールの収束定理}

\subsection{Doob-Meyer分解}

\subsection{Burkholderの不等式}

\section{Gauss系}

\chapter{半マルチンゲールと統計解析}

\section{統計推測への応用}

回帰モデル$X_i=f(X_{i-1},\cdots,X_{i-p})+\ep_i$において,
サンプリングが均等でないときなど,$\ep_i$は何か連続的な確率過程を積分して定まる,と考えると
数理モデルとして非常に自然である.連続関数$f$について,
\[Y_t=Y_0+\int^t_0f(Y_s)ds+W_t\]
とし,$W_{t_i}-W_{t_{i-1}}\sim\N(0.t_i-t_{i-1})$を標準Weiner過程とする.

このようなモデルのうち,特に株価の対数を$Y_t$とおいたときに使われるパラメトリックモデルに,
\textbf{Vasicek過程}
\[Y_t=Y_0-\int^t_0\al_1(Y_s-\al_2)ds+\beta W_t\]
などがあり,離散的観測$\{Y{t_0},\cdots,Y_{t_n}\}$に基づいて未知パラメータ$\al_1,\al_2,\beta$の推定を考える.
このときにマルチンゲール理論が使える.

その理由は,martingaleというクラスの形式的定義が,自然に統計モデルの「ノイズの直交性」の拡張となっていると考えられるためである.
これは独立性の仮定による代数規則$E[\ep_i\ep_j]=0$の抽出となっているのである.

大きな応用分野として生存解析におけるcensored data\footnote{消息不明になる瞬間があること.癌の再発データにおいて,他の原因による死亡など.}の解析がある.
このとき,$N_t$を死亡数,$Y_t$をcensorされずに残っている観測対象数,癌の再発時刻の分布関数を$F$,密度関数を$f$とすると,
\[N_t-\int^t_0\al(s)Y_sds\qquad\al(t)=\frac{f(t)}{1-F(t)}\]
はマルチンゲールになる.$\al$はハザード関数といい,患者が時刻$t$で生存しているという条件の下,その時間に死亡する条件付き確率となる.
このマルチンゲールの期待値は常に$0$だから,$N_t$の不偏推定量が見つかったことになる.
なお,
\[\int^t_0\frac{1}{Y_s}(dM_s-\al(s)Y_sds)\]
もマルチンゲールとなることがわかる.

\chapter{Markov過程}

\begin{quote}
    互いに独立な試行の列(確率変数の列)の,マルチンゲールとは別の方向への一般化を考える.
    独立性は一切の過去の履歴に依らないが,Markov性は,現在の状態のみに依存する性質を指す.

    Brown運動は状態空間,時間パラメータのいずれも連続な場合であり,Poisson過程は状態空間は離散的である例である.
    状態空間が離散的な場合,Markov連鎖ともいう.
    またここで偏微分方程式との関係から,確率過程の一般化も自然に出現する.
    添字集合を多様体$M$とした確率過程$\Om\times M\to\R^n$を\textbf{確率場}という.
    このとき,時間概念が空間に置き換わっている.

    $x_{n+1}$の$x_n$への依存の仕方は経時変化しないという,時間的一様性の仮定をおいて議論する.
    すると,Markov過程は推移作用素を定めることで分布が決まる.
    これは大数の法則を一般化する.
    また,推移作用素になり得る作用素は放物型偏微分方程式によって特徴付けられる.
\end{quote}

\section{Kolmogorovの拡張定理}

\begin{tcolorbox}[colframe=ForestGreen, colback=ForestGreen!10!white,breakable,colbacktitle=ForestGreen!40!white,coltitle=black,fonttitle=\bfseries\sffamily,
title=]
    Hopfの拡張定理の一般化である.
\end{tcolorbox}

\begin{theorem}\label{thm-Kolmogorov-extension-theorem}
    確率空間列$(\R^n,\B(\R^n))$上の確率測度列$(\mu_n)$が次の一貫性条件をみたすとき,$(\R^\N,\B(\R^\N))$上の確率測度$\mu$であって$\forall_{A\in\B(\R^n)}\;\mu(A\times\R^\N)=\mu_n(\A)$を満たすものが一意的に存在する.ただし,
    $A\times\R^\N=\Brace{(\om_n)\in\R^\N\mid (\om_1,\cdots,\om_n)\in A}$とした.
    なお,$\R^\N$には直積位相を考える.
    \begin{quotation}
        (consistency) $\forall_{n\in\N}\;\forall_{A\in\B(\R^n)}\;\mu_{n+1}(A\times\R)=\mu_n(A)$.
    \end{quotation}
    特に,この一貫性条件は$\R^\N$上の測度に延長できるための必要十分条件である.
    これは$\R$を一般の完備可分空間としても成り立つ.
\end{theorem}

\section{離散時間のMarkov連鎖}

\begin{notation}[state space]
    $I$を可算集合とし,これを\textbf{状態空間}とする.
    見本過程は列$\N\to I$となり,経時的に$I$上を動き回りことになる.
\end{notation}

\subsection{確率行列}

\begin{history}
    MarkovがMarkov連鎖と確率行列を発明した.言語分析やカードシャッフルの問題に用いるつもりであったが,たちまち他の分野でも有用だと解った.
    確率行列の概念はKolmogorovに引き継がれることとなる.
    実際,量子状態を表す演算子も,行列表示を持つこととなる.
\end{history}

\begin{notation}\mbox{}
    \begin{enumerate}
        \item $\one$はすべての成分が$1$であるような縦ベクトルも表す.
        \item $\delta_i$で,$i$成分のみが$1$でそれ以外が$0$であるようなベクトルを表す.
    \end{enumerate}
\end{notation}

\begin{definition}[stochastic matrix]\mbox{}
    \begin{enumerate}
        \item $I$上の\textbf{確率ベクトル}$(\nu_i)_{i\in I}$とは,$I$上の確率質量関数$I\to[0,1]$をいう.
        \item (可算個の成分を持ち得る)行列$\bP=(p_{ij})_{i,j\in I}$が\textbf{(右)確率的}であるとは,各$i$行ベクトル$(p_{ij})_{j\in I}$がそれぞれ$I$上の確率ベクトルを定めることをいう.
        意味論として,成分$p_{ij}$は,現状態$i$から次の時刻$j$に遷移する確率を定める.
    \end{enumerate}
\end{definition}

\begin{lemma}[確率行列の特徴付け]
    行列$\bP$について,次の3条件は同値.
    \begin{enumerate}
        \item $\bP$は確率的である.
        \item 各行の和が$1$で非負な作用素(積分なので):$f\ge0\Rightarrow\bP f\ge0$かつ$\bP\one=\one$.
        \item 横ベクトル$\nu$が確率ベクトルならば,$\nu\bP$も確率ベクトルである.
    \end{enumerate}
\end{lemma}

\begin{lemma}[確率行列は群をなす?]
    確率行列$\bP=(p_{ij}),\bP'=(p'_{ij})$の積は確率行列である.
\end{lemma}

\subsection{Markov連鎖の定義と構成}

\begin{definition}[transition matrix, Markov chain]
    $I$-値確率変数列$\{X_n\}\subset\Meas(\Om,I)$が,\textbf{初期分布$\nu$,遷移行列$\bP$を持つ空間$I$上のMarkov連鎖}であるとは,次が成り立つことをいう:
    \[\forall_{n\in\N}\;\forall_{i_0,\cdots,i_n\in I}\;P(X_0=i_0,\cdots,X_n=i_n)=\nu_{i_0}p_{i_0i_1}\cdots p_{i_{n-1}i_n}.\]
\end{definition}

\begin{proposition}
    適当な確率空間の上に,初期分布$\nu$と遷移行列$\bP$をもち,殆ど至る所$I$値なMarkov連鎖が存在する.
\end{proposition}
\begin{proof}
    $I$は可算だから単射$I\mono\N$が存在する.以降,$I\mono\N\mono\R$として,$\R$の部分集合と同一視する.
    \begin{description}
        \item[構成] 各$n\in\N$に対して,$(\R^{n+1},\B(\R^{n+1}))$上の測度$P_{n+1}$を
        \[P_{n+1}(A):=\sum_{(i_0,\cdots,i_n)\in I^{n+1}\cap A}\nu_{i_0}p_{i_0i_1}\cdots p_{i_{n-1}i_n}\quad(A\in\B(\R^{n+1}))\]
        とすると,これはたしかに確率測度である.
        \item[一貫性] 任意の$A\in\B(\R^{n+1})$について,
        \[P_{n+2}(A\times\R)=\sum_{(i_0,\cdots,i_n)\in I^{n+1}\cap A}\nu_{i_0}p_{i_0i_1}\cdots p_{i_{n-1}i_n}\paren{\sum_{i_{n+1}\in I}p_{i_ni_{n+1}}}=P_{n+1}(A).\]
        \item[検証] Kolmogorovの拡張定理\ref{thm-Kolmogorov-extension-theorem}より,$(\R^\N,\B(\R^\N))$上の確率測度$P$であって,$P(A\times\R^\N)=P_{n+1}(A)$を満たすものがただ一つ存在する.
        この空間上の実数値確率変数列$(X_n)$を,$X_n(\om)=\om_n\;(\om=(\om_0,\cdots)\in\R^\N)$と定めれば,これは殆ど至る所$I$-値の,求めるMarkov過程である.
    \end{description}
\end{proof}

\begin{example}[i.i.d.はMarkov過程]
    $\bP=(p_{ij})$の行ベクトルが$i$に依らずすべて同じであるとき,$(X_n)$は独立同試行に従う確率変数列となる.
\end{example}

\subsection{Markov性}

\begin{definition}[Markov property]
    過去の軌跡を
    $A_{i_0,\cdots,i_n}:=\Brace{\om\in\Om\mid X_0(\om)=i_0,\cdots,X_n(\om)=i_n}$とし,$P(A_{i_0,\cdots,i_n})>0$ならば,
    $\forall_{i_{n+1}\in I}\;P[X_{n+1}=i_{n+1}|A_{i_0,\cdots,i_n}]=p_{i_ni_{n+1}}$が成り立つことは定義からすぐにわかる.
    これを\textbf{Markov性}という.
\end{definition}

\begin{notation}
    $\F_n:=\sigma[X_0,\cdots,X_n]$を,Markov連鎖の定める自然な増加情報系とする.
\end{notation}

\begin{theorem}
    $(X_n)$は初期分布$\nu$,遷移行列$\bP$のMarkov連鎖とする.
    $m\in\N,i\in I$は$P[X_m=i]>0$を満たすとする.このとき,条件付き確率測度$P[-|X_m=i]$の下で,次の2条件が成り立つ.
    \begin{enumerate}
        \item $(X_{m+n})_{n\in\N}$は初期分布$\delta_i$,遷移行列$\bP$のMarkov連鎖である.
        \item $(X_{m+n})_{n\in\N}$は$\F_m$と独立である.
    \end{enumerate}
\end{theorem}

\begin{lemma}
    確率行列の$n$乗の成分を$\bP^n=:(p_{ij}^{(n)})$と表すこととする.このとき,
    \[\forall_{m\in\N}\;\forall_{i\in I}\;P(X_m=i)>0\Rightarrow P[X_{m+n}=j|X_m=i]=p^{(n)}_{ij}.\]
    この成分を\textbf{$n$ステップ遷移確率}という.
\end{lemma}

\subsection{強Markov性}

\begin{tcolorbox}[colframe=ForestGreen, colback=ForestGreen!10!white,breakable,colbacktitle=ForestGreen!40!white,coltitle=black,fonttitle=\bfseries\sffamily,
title=]
    いかなる時点においても,現在の状況にしか依存しない性質をMarkov性と言うのであった.
    さらに,時間をランダムに定めても,現状にしか依存しないはずである.これを強Markov性という.
\end{tcolorbox}

\begin{theorem}
    $(X_n)$を初期分布$\nu$,遷移確率$\bP$のMarkov連鎖とし,$i\in I$は$P[\tau<\infty,X_\tau=i]>0$とする.
    このとき,条件付き確率$P[-|\tau<\infty,X_\tau=i]$の下で,次の2条件が成り立つ.
    \begin{enumerate}
        \item $(X_{\tau+n})_{n\in\N}$は初期分布$\delta_i$,遷移行列$\bP$を持つMarkov過程である.
        \item $(X_{\tau+n})_{n\in\N}$は$\F_\tau$と独立である.
    \end{enumerate}
\end{theorem}

\section{到達確率と差分作用素}

\begin{tcolorbox}[colframe=ForestGreen, colback=ForestGreen!10!white,breakable,colbacktitle=ForestGreen!40!white,coltitle=black,fonttitle=\bfseries\sffamily,
title=]
    差分は前進$\Delta f(x):=f(x+1)-f(x)$と後退$\nabla f(x):=f(x)-f(x-1)$の2つが考えられる.
    これが連続になると確率微分方程式となるのだ.
\end{tcolorbox}

\subsection{到達確率と特徴付け}

\begin{definition}[hitting / absorption probability]
    $(X_n)$をMarkov過程とする.
    \begin{enumerate}
        \item 集合$A\subset I$に対して,$A$への到達時刻とは,$\tau_A:=\min\Brace{n\in\N\mid X_n\in A}$として定まる可測関数$\Om\to\o{\N}$であった\ref{exp-discrete-Markov-time}.
        \item 初期分布$\delta_i$を持つMarkov過程に関する確率を$P_i$で表す.
        $a_i:=P_i[\tau_A<\infty]$を\textbf{到達確率}または\textbf{吸収確率}という.
        \item $I$上の確率測度$Q_i$を,$i\in I$からの遷移確率$(p_{ij})_{j\in I}$が定めるものとして,任意の$i\in I$に対して$Q_i$-可積分な関数$f:I\to\R$に対する作用素
        $\L:\cap_{i\in I}L^1(I,Q_i)\to\Map(I,\R)$を,$\L f(i):=\sum_{j\in I}p_{ij}f(j)-f(i)$と定め,\textbf{差分作用素}という.
    \end{enumerate}
\end{definition}

\begin{theorem}[到達確率の特徴付け]
    到達確率$(a_i)_{i\in I}$は,方程式系
    \[\forall_{i\in I\setminus A}\;\L a(i)=0,\quad\forall_{i\in A}\;a_i=0\]
    の最小の非負解である.後者は前者の\textbf{境界条件}という.
\end{theorem}
\begin{remarks}
    差分方程式を書き直すと,$i\in I\setminus A$に関して$a_i=\sum_{j\in I}p_{ij}a_j$となり,$i$からの遷移確率に関する,到達確率の平均になる.
\end{remarks}

\subsection{Markov過程の定めるmartingale}

\begin{tcolorbox}[colframe=ForestGreen, colback=ForestGreen!10!white,breakable,colbacktitle=ForestGreen!40!white,coltitle=black,fonttitle=\bfseries\sffamily,
title=]
    一般のMarkov過程について,これをmartingale理論の問題に還元することが可能である.
    これは全く確率微分方程式特有の構造である.
\end{tcolorbox}

\begin{theorem}
    $(X_n)$をMarkov過程,$f\in l^\infty(\R)$を有界関数とする.
    \[Y_n:=f(X_n)-f(X_0)-\sum_{k=0}^{n-1}\L f(X_k)\]
    によって定まる過程$(Y_n)$は$(\F_n)$-マルチンゲールである.
\end{theorem}

\section{有限状態空間上のMarkov連鎖}

\begin{tcolorbox}[colframe=ForestGreen, colback=ForestGreen!10!white,breakable,colbacktitle=ForestGreen!40!white,coltitle=black,fonttitle=\bfseries\sffamily,
title=]
    $\abs{I}<\infty$の場合について,理論の広がりを見る.
\end{tcolorbox}

\subsection{不変分布とエルゴード性}

\begin{tcolorbox}[colframe=ForestGreen, colback=ForestGreen!10!white,breakable,colbacktitle=ForestGreen!40!white,coltitle=black,fonttitle=\bfseries\sffamily,
title=]
    確率行列$\bP$の,確率分布の空間$P(I)$への作用を考えると,不動点が存在する.
\end{tcolorbox}

\begin{definition}[ergodic, irreduciable, aperiodic]
    Markov連鎖$((X_n),I,\bP)$について,
    \begin{enumerate}
        \item $\bP$が\textbf{エルゴード的}であるとは,$\exists_{n_0\in\N}\;\bP^{n_0}>0$が成り立つことをいう.
        \item $\bP$が\textbf{既約}であるとは,$\forall_{i,j\in I}\;\exists_{n_1\in\N}\;p_{ij}^{(n_1)}>0$を満たすことをいう.
        \item 状態$i\in I$が\textbf{非周期的}であるとは,$\exists_{n_2\in\N}\;\forall_{n\ge n_2}\;p_{ii}^{(n)}>0$を満たすことをいう.
    \end{enumerate}
\end{definition}

\begin{lemma}[エルゴード性の特徴付け]
    Markov連鎖$((X_n),I,\bP)$が$\abs{I}<\infty$を満たすとき,次の3条件は同値.
    \begin{enumerate}
        \item $\bP$はエルゴード的である.
        \item $\bP$は既約で,すべての状態$i\in I$は非周期的である.
        \item $\bP$は既約で,ある状態$i\in I$は非周期的である.
    \end{enumerate}
\end{lemma}

\begin{example}\mbox{}
    \begin{enumerate}
        \item 円周の$N$等分点上のランダムウォークは,$N$が奇数ならばエルゴード的であるが,偶数ならば既約であっても非周期的にはならない.
        \item $p_{ii}=1$を満たす$i\in I$をtrapという.これがあるMarkov過程はエルゴード的でない.
    \end{enumerate}
\end{example}

\begin{theorem}[有限状態Markov過程のエルゴード定理]
    Markov連鎖$((X_n),I,\bP)$が$\abs{I}<\infty$を満たし,$\bP$はエルゴード的であるとする.
    このとき,(1)を満たす$I$上の確率分布$\pi$が一意的に存在する.この$\pi$は(2),(3)も満たす.
    \begin{enumerate}
        \item 定常性:$\pi\bP=\pi$.
        \item 極限分布:$\forall_{i,j\in I}\;\lim_{n\to\infty}p_{ij}^{(n)}=\pi_j$.
        \item 混合性:(2)の収束は指数関数的である:$\exists_{C>0}\;\exists_{0<\lambda<1}\;\forall_{n\in\N}\;\forall_{i,j\in I}\;\abs{p_{ij}^{(n)}-\pi_j}\le C\lambda^n$.
    \end{enumerate}
    この分布$\pi$を\textbf{不変分布}または\textbf{定常分布}という.
\end{theorem}

\subsection{大数の法則}

\begin{tcolorbox}[colframe=ForestGreen, colback=ForestGreen!10!white,breakable,colbacktitle=ForestGreen!40!white,coltitle=black,fonttitle=\bfseries\sffamily,
title=]
    Markov連鎖がエルゴード的ならば,独立性の代わりになり,大数の法則が成り立つ.
\end{tcolorbox}

\begin{theorem}[大数の弱法則]\label{thm-law-of-large-number-of-Markov-chain}
    Markov連鎖$((X_n),I,\bP)$が$\abs{I}<\infty$を満たし,$\bP$はエルゴード的であるとする.$\pi$を不変分布とすると,
    関数$f:I\to\R$について,
    \[\frac{1}{n}\sum_{k=1}^nf(X_k)\xrightarrow{p}E^\pi[f].\]
\end{theorem}

\begin{definition}[number of visit]
    $i\in I$に関して,$f(j):=1_{\Brace{j=i}}$と定めると,$\sum^n_{k=1}f(X_k)$とは時刻$n$までの$i$への訪問回数$\tau^{(n)}_i$を表す.
    滞在時間ともいう.
\end{definition}

\begin{corollary}
    $\frac{\tau_i^{(n)}}{n}\xrightarrow{p}\pi_i$.
\end{corollary}

\begin{definition}[stationarity]\mbox{}
    \begin{enumerate}
        \item Markov連鎖$(X_n)_{n\in\N}$が\textbf{定常的}であるとは,$(X_n)_{n\in\N}$と$(X_{n+1})_{n\in\N}$との分布が等しいことをいう.
        \item 初期分布を$\pi$とするエルゴード的なMarkov連鎖を\textbf{定常Markov連鎖}という.
    \end{enumerate}
\end{definition}

\begin{theorem}[高次元化]
    Markov連鎖$((X_n),I,\bP)$が$\abs{I}<\infty$を満たし,$\bP$はエルゴード的であるとする.$\pi$を不変分布とすると,
    関数$f:I^l\to\R\;(l\ge1)$について,
    \[\frac{1}{n}\sum_{k=1}^nf(X_k,\cdots,X_{k+l-1})\xrightarrow{p}E^\pi[f]\]
    ただし,$E^\pi$は定常Markov連鎖$(\o{X}_n)_{n\in[l]}$に関する期待値である.
\end{theorem}

\section{正方格子上のランダムウォーク}

\begin{tcolorbox}[colframe=ForestGreen, colback=ForestGreen!10!white,breakable,colbacktitle=ForestGreen!40!white,coltitle=black,fonttitle=\bfseries\sffamily,
title=]
    次に非有限な状態空間を持つMarkov過程の例を見る.
    代表的なものが,$\Z^d$上のランダムウォークである.
\end{tcolorbox}

\begin{definition}[random walk]
    Markov過程$((X_n),\Z^d,\bP)$の遷移行列$\bP$が,時間一様性に加えて空間一様性$\forall_{x,y,z\in\Z^d}\;p_{xy}=p_{x+z,y+z}$を満たすとき,これを\textbf{酔歩}という.
\end{definition}

\begin{discussion}[加法過程としての構成]\label{discussion-random-walk}
    Markov過程としての一貫性に訴えずとも,空間的一様性に注目すれば,
    初期分布$\nu$を持つ$Z_0$と,分布$p$を持つ$Z_1,Z_2,\cdots$とが独立であるとき,$X_n:=\sum^n_{k=0}Z_k$とすればこれは酔歩である.
\end{discussion}

\subsection{再帰性と非再帰性}

\begin{tcolorbox}[colframe=ForestGreen, colback=ForestGreen!10!white,breakable,colbacktitle=ForestGreen!40!white,coltitle=black,fonttitle=\bfseries\sffamily,
title=]
    平均的な一歩$E[Z_1]$が零ベクトルでない場合,酔歩は非再帰的である.
    また,再帰的であることと無限回$0$を踏むことは同値である.
\end{tcolorbox}

\begin{notation}
    $\nu=\delta_0$,あるいは,$Z_0=0$から始まる酔歩を考える.
    \begin{align*}
        A_n&:=\Brace{X_n=0,\forall_{1\le k\le n-1}\;X_k\ne0}\\
        q&:=\sum^\infty_{n=1}P(A_n)=P[\Brace{\exists_{n\in\N}\;X_n=0}]
    \end{align*}
    とする.
\end{notation}

\begin{definition}[recurrent]
    酔歩が$q=1$を満たすとき\textbf{再帰的}であるという.
\end{definition}

\begin{lemma}
    酔歩について,次の2条件は同値.
    \begin{enumerate}
        \item 再帰的である.
        \item $\sum^\infty_{n=1}P[X_n=0]=\infty$.
    \end{enumerate}
\end{lemma}

\begin{theorem}[非再帰性の十分条件]
    $R:=\max\Brace{\abs{z}\in\R\mid z\in\Z^d,p_{0z}>0}<\infty$と仮定し,
    $m:=\sum_{z\in\Z^d}p_{0z}z=E[Z_1]\in\R^d$とおく.
    $m\ne0$のとき,酔歩は非再帰的である.
\end{theorem}

\subsection{単純ランダムウォークの再帰性と非再帰性}

\begin{tcolorbox}[colframe=ForestGreen, colback=ForestGreen!10!white,breakable,colbacktitle=ForestGreen!40!white,coltitle=black,fonttitle=\bfseries\sffamily,
title=]
    単純酔歩では,2d個の隣点にのみ,そして等確率に移動可能とする.
    $E[Z_1]=0$なので,これだけで再帰性は判定できない.
\end{tcolorbox}

\begin{definition}[simple random walk]
    遷移確率が
    \[p_z:=\begin{cases}
        \frac{1}{2d},&\abs{z}=1,\\
        0,&\abs{z}\ne1
    \end{cases}\]
    となる酔歩を\textbf{単純酔歩}という.
\end{definition}

\begin{theorem}[Polya]
    $d$次元の単純酔歩は,$d=1,2$のとき再帰的であり,$d\ge3$のとき再帰的でない.
\end{theorem}

\section{連続時間Markov過程}

\subsection{Chapman-Kolmogorov方程式}

\begin{tcolorbox}[colframe=ForestGreen, colback=ForestGreen!10!white,breakable,colbacktitle=ForestGreen!40!white,coltitle=black,fonttitle=\bfseries\sffamily,
title=熱核の半群性]
    Markov過程の発展は,確率行列の積で表された.この連続化は,ある発展条件を満たすことである.
    この放物型偏微分方程式をChapman-Kolmogorov方程式という.
    これを解いて推移確率とし,Kolmogorovの拡張定理に基づけば拡散過程が構成できる.
    Kolmogorovは初期から物理学への応用を見据えて,多様体の言葉で論じていた.

    この方法はHormanderが取ったように,偏微分方程式への迂回でもある.
    直接的に確率微分方程式に基づいてBrown運動を「変形する」という確率論的手法を立てたのが伊藤清である.
\end{tcolorbox}

\begin{discussion}
    離散集合$I$上の遷移行列$\bP$が満たす規則は次のようにかける.
    \begin{enumerate}
        \item $\forall_{i\in I}\;\sum_{j\in I}p_{ij}^{(n)}=1$.
        \item $\forall_{i,j\in I}\;\forall_{n,m\in\N}\;\sum_{k\in I}p_{ik}^{(n)}p_{kj}^{(m)}=p_{ij}^{(n+m)}$.
    \end{enumerate}
    $I$を一般のポーランド空間,$\N$を$\R_+$へ,遷移行列を遷移作用素へ一般化したい.
    \begin{enumerate}
        \item $\forall_{s\in\R_+}\;\forall_{x\in S}\;p(s,x,-)\in P(S)$.時刻$0$に$x$から出発するMarkov過程の,時刻$s$での位置の分布.
        \item $\forall_{s,t\in\R_+}\;\forall_{x,z\in S}\;\int_Sp(s,x,dy)p(t,y,dz)=p(s+t,x,dz)$.
        または,$\forall_{A\in\B(S)}\;\int_Sp(s,x,dy)p(t,y,A)=p(s+t,x,A)$.
    \end{enumerate}
    こうして,行列積は積分に一般化される.(2)を時間一様なChapman-Kolmogorovの等式という.
    これは,時刻$0$に$x$から初めて,$s+t$に$A$に至るまでの時刻$s$での経由地$y\in S$について積分しても等しくなる,という意味を持つ.
\end{discussion}

\chapter{加法過程}

\section{加法過程}

\begin{tcolorbox}[colframe=ForestGreen, colback=ForestGreen!10!white,breakable,colbacktitle=ForestGreen!40!white,coltitle=black,fonttitle=\bfseries\sffamily,
title=]
    見本道が殆ど至る所cadlagである過程を$D$-過程という.
    見本過程$\R_+\to\Meas(\Om,\R);s\mapsto X_s$が確率連続な加法過程で$D$-過程でもあるものを,Levy過程という.
    任意の確率連続な加法過程はLevy過程に同等であり,Levy過程の構造は解明済みである.

    大雑把にいえば,連続な加法過程はGauss型,すなわちブラウン運動に限り,非連続的な加法過程はPoisson型に限る.
    任意のLevy過程は,ドリフト付きのBrown運動とLevyのジャンプ過程との和に分解できる.
\end{tcolorbox}

\subsection{定義と例}

\begin{example}
    $\Z^d$-酔歩は次の性質を満たす:任意の長さ$k\ge2$の部分列$(n_j)_{j\in[k]}$について,\textbf{増分}の過程$(X_{n_j}-X_{n_{j-1}})_{j\in[k]}$は独立.
\end{example}

\begin{definition}[increment, additive process]
    次の性質を満たす,初期分布$\delta_0$な$\R^d$-値確率過程$(X_s)_{s\in\R_+}$,すなわち,空間的一様かつ時間一様なMarkov過程を\textbf{加法過程}という.
    \begin{enumerate}\setcounter{enumi}{2}
        \item 任意の長さ$k\ge2$の狭義単調増加非負実数列$(s_j)_{j\in[k]},s_0=0$が定める増分列$(X_{s_j}-X_{s_{j-1}})_{j\in[k]}$は独立.
    \end{enumerate}
    このとき,$S$の原点を$0$とするとその転移確率は一様に$p(s,-):=p(s,0,-)$と表わせ,Chapman-Kolmogorovの等式も
    \[\forall_{s,t\in\R_+}\;\forall_{x,z\in S}\;\int_Sp(s,dy-x)p(t,dz-y)=p(s+t,dz-x)\]
    と表せる.
\end{definition}

\begin{theorem}[Gauss型とPoisson型Levy過程]
    $X$をLevy過程とする.
    \begin{enumerate}
        \item $X$がさらに概連続過程であれば,増分$X_b-X_a\;(b>a)$はGauss分布に従う.
        \item $X$がさらに殆ど至る所飛躍$1$で増加する階段関数を見本過程に持つならば,増分$X_b-X_a\;(b>a)$はPoisson分布に従う.
    \end{enumerate}
\end{theorem}

\subsection{Levy-Ito分解}

\begin{theorem}[Levy過程の分解定理 (Levy-Ito decomposition)]
    $X$をLevy過程とする.
    $\Gamma=\R_+\times(\R\setminus\{0\})$上の固有なPoisson配置$N$と,これと独立なGauss型Levy過程$G$が存在して,
    \[X(t,\om)=G(t,\om)+\lim_{n\to\infty}\Square{\int_{s\le t}\int_{\abs{u}>1/n}(uN(dsdu,\om)-\phi(u)n(dsdu))}\]
    と表せる.
\end{theorem}

\begin{definition}[Levy-Khintchine triplet]
    Levy過程$X$の連続部分$G$を用いて,$m(t):=E[G(t)],v(t):=\Var[G(t)]$は有限確定する.
    また,$m$は連続関数,$v$は連続な単調増加関数,$m(0)=v(0)=0$を満たす.
    また,$n$をPoisson配置$N=N_X$の平均測度,すなわち$\Gamma$上の測度で,次を満たす:
    \[\forall_{t\in\R_+}\quad n(\{t\}\times(\R\setminus\{0\}))=0,\quad\int_{s\le t}\int_{\abs{u}>0}(u^2\land 1)n(dsdu)<\infty.\]
    組$(n,m,v)$を,Levy過程$X$の\textbf{特性量}または\textbf{Levy-Khintchine組}という.
\end{definition}

\begin{theorem}
    上記の条件を満たす$(n,m,v)$に対して,これを特性量として持つLevy過程が存在して,法則同等を除いて一意的である.
\end{theorem}

\begin{definition}
    $X_t-X_s\;(t>s)$の確率法則が$t-s$の値のみに関係するようなLevy過程を,\textbf{時間的に一様なLevy過程}という.
    このとき,特性値は次のように表せる.
    \[m_X(t)=mt,\quad v_X(t)=vt,\quad n_X(dtdu)=dt\cdot n(du).\]
\end{definition}

\section{Brown運動}

\begin{tcolorbox}[colframe=ForestGreen, colback=ForestGreen!10!white,breakable,colbacktitle=ForestGreen!40!white,coltitle=black,fonttitle=\bfseries\sffamily,
title=]
    連続な加法過程は,必然的にGauss型である.これをBrown運動という.
    ドリフトはないものをまずは見る.
\end{tcolorbox}

\subsection{定義}

\begin{tcolorbox}[colframe=ForestGreen, colback=ForestGreen!10!white,breakable,colbacktitle=ForestGreen!40!white,coltitle=black,fonttitle=\bfseries\sffamily,
title=]
    熱方程式$u_t(x,t)=u_{xx}(x,t),u(x,0)=0$の基本解
    \[H(x,t)=\frac{1}{\sqrt{4\pi t}}\exp\paren{-\frac{x^2}{4t}}\]
    は熱核または熱方程式の初期値問題のGreen関数と呼ばれ,初期条件$u(x,0)=f(x)$に関する解は
    \[u(x,t)=H*f=\int_\R H(x-y,t)f(y)dy\]
    と表される.
    熱の拡散と確率の拡散,エントロピーの概念は深いどこかでつながっているのであろうか.
\end{tcolorbox}

\begin{definition}
    確率空間$(\Om,\F,P)$上の実数値確率過程$(B_t)_{t\in\R_+}$が\textbf{Brown運動}であるとは,次の3条件をみたすことをいう:
    \begin{enumerate}
        \item $B_0=0\;\as$
        \item 任意の見本道$B_t(\om):\R_+\to\R$は連続:$B_t\in W$.
        \item 任意の長さ$n\in\N_{>0}$の$\R_+$の狭義増加列$(t_j)_{j\in[n]},t_j=0$が定める増分の組$(B_{t_j}-B_{t_{j-1}})_{j\in[n]}$は互いに独立にGauss分布$N(0,t_{j}-t_{j-1})$に従う.
    \end{enumerate}
\end{definition}
\begin{remarks}
    実は(3)のうち増分の正規性はなくても従うことは,Levy過程の理論による.
\end{remarks}

\begin{theorem}[Brown運動の存在]\label{thm-existence-of-Brownian-motion}
    ある確率空間$(\Om,\F,P)$が存在して,その上にBrown運動が存在する.
\end{theorem}

\subsection{Wiener測度}

\begin{tcolorbox}[colframe=ForestGreen, colback=ForestGreen!10!white,breakable,colbacktitle=ForestGreen!40!white,coltitle=black,fonttitle=\bfseries\sffamily,
title=]
    古典的Wiener空間$W_0$は$0$から始まる連続な見本過程$R_+\to\R$全体の空間で,Banach空間となる.
    Brown運動はここに確率測度を押し出し(見本道のばらつき),Brown運動は,関数解析的には$W_0$上の確率測度の1つと同一視出来る.
\end{tcolorbox}

\begin{notation}[classical Wiener space]
    次の言葉を使えば,Brown運動とはWiener空間に値を持つ確率変数$\Om\to W_0$であって,カリー化$\R_+\to\Meas(\Om,\R)$は任意の有限成分について独立なGauss分布を定めるものをいう.
    \begin{enumerate}
        \item $W=W^1:=C(\R_+)$を連続な見本道の空間とする.
        \item $W_0:=\Brace{w\in W\mid w_0=0}$とする.これを\textbf{古典的Wiener空間}という.
    \end{enumerate}
    それぞれの空間には一様ノルムは入れられないので,広義一様収束位相を考え,Borel集合族によって可測空間とみなす.
    $\R_+$なのでBanach空間ではない.
\end{notation}

\begin{definition}[Wiener measure (23)]
    $(W_0,\B(W_0))$上の射影の族$(B_t)_{t\in\R_+},B_t(\om):=\pr_t(\om)=\om_t$がBrown運動になるような確率測度$P$を\textbf{Wiener測度}という.
\end{definition}

\begin{lemma}
    Wiener測度は一意的に存在する.
\end{lemma}
\begin{proof}\mbox{}
    \begin{description}
        \item[存在] Brown運動の存在\ref{thm-existence-of-Brownian-motion}による.
        ある空間$(\Om,\F,P)$上のBrown運動$B:\Om\to W$を取る.これによる像測度$P^B$は$P^B(W_0)=1$を満たすから,$W_0$への制限を取れば,これがWiener測度である.
        \item[一意性] $W_0$の柱状集合全体$\cC$\ref{remark-cylinder-sets}上では一意的である.$\cC$は$\pi$-系・情報族であり,$\B(W_0)=\sigma(\cC)$を満たすため,一意に延長される.
    \end{description}
\end{proof}

\subsection{特性値}

\begin{lemma}[積率]
    $B_t$の奇数次の積率は消えてきて,偶数次の積率は
    \[E[B_t^2]=t,\quad E[B_t^4]=3t^2,\quad E[B^6_t]=15t^3,\quad,E[B^{2n+1}_t]=(2n+1)(2n-1)\cdots 3\cdot 1t^{2n+1}.\]
\end{lemma}

\begin{lemma}[共分散]
    $\forall_{t,s\in\R_+}\;E[B_tB_s]=t\land s$.
\end{lemma}

\subsection{独立増分性}

\begin{tcolorbox}[colframe=ForestGreen, colback=ForestGreen!10!white,breakable,colbacktitle=ForestGreen!40!white,coltitle=black,fonttitle=\bfseries\sffamily,
title=]
    加法過程としての独立増分性は,martingale問題に繋がる.
\end{tcolorbox}

\begin{notation}
    ブラウン運動$(B_t)_{t\in\R_+}$の自然な情報系を$\F^B_t:=\sigma[B_s;s\le t]$と表す.
\end{notation}

\begin{proposition}
    $0\le s<t$に関して,$B_t-B_s$は$\F_s^B$と独立.
\end{proposition}

\begin{corollary}
    Brown運動$(B_t)_{t\in\R_+}$は$(\F^B_t)$に関してmartingaleである.
    その2次変分は$\brac{B}_t=t$で与えられる.
\end{corollary}

\subsection{可微分性}

\begin{theorem}[Paley-Wiener-Zygmond]
    $B_t(\om)$は$\om\dae$に対して,$t$について至る所微分不可能である.
\end{theorem}

\begin{theorem}[modulus of continuity]
    $1/2$-Holder連続性よりやや悪い連続度を持つ.
    \[\limsup_{t_2-t_1=\ep\searrow0,0\le t_1<t_2\le1}\frac{\abs{B_{t_2}-B_{t_1}}}{\sqrt{2\ep\log(1/\ep)}}=1\;\as\]
\end{theorem}

\begin{theorem}[重複対数の法則]
    \[\limsup_{t\searrow0}\frac{B_t}{\sqrt{2t\log\log(1/t)}}=1\;\as\]
\end{theorem}

\section{Poisson過程}

\begin{tcolorbox}[colframe=ForestGreen, colback=ForestGreen!10!white,breakable,colbacktitle=ForestGreen!40!white,coltitle=black,fonttitle=\bfseries\sffamily,
title=]
    $D$-過程であるが連続過程ではなく,見本過程が至る所ジャンプしているとき,これはPoisson型Levy過程である.
    ここまでいかずとも,少しドリフト$0$分散$0$のBrown運動を混ぜて,ジャンプを持つ加法過程で特に基本的なPoisson過程を見る.
\end{tcolorbox}

\subsection{定義}

\begin{tcolorbox}[colframe=ForestGreen, colback=ForestGreen!10!white,breakable,colbacktitle=ForestGreen!40!white,coltitle=black,fonttitle=\bfseries\sffamily,
title=]
    $\R_+$上に強さ$\lambda$のPoisson点過程を考える.その総数を整数で切ったものをPoisson過程と呼ぼう.
\end{tcolorbox}

\begin{definition}[Poisson process]
    $(\Om,\F,P)$上の$\Z_+$-値確率過程$(N_t)_{t\in\R_+}$が\textbf{パラメータ$\lambda>0$を持つPoisson過程}であるとは,次の条件を満たすことをいう:
    \begin{enumerate}
        \item $N_0=0\;\as$
        \item 任意の見本道$N_t(\om)$は右連続かつ単調増加である.
        \item 任意の長さ$n\in\N_{>0}$の$\R_+$の狭義増加列$(t_j)_{j\in[n]},t_0=0$が定める増分の組$(N_{t_j}-N_{t_{j-1}})_{j\in[n]}$は独立で,それぞれパラメータ$\lambda(t_j-t_{j-1})$を持つPoisson分布に従う.
    \end{enumerate}
\end{definition}

\begin{discussion}[Poisson過程の構成]
    Kolmogorovの拡張定理により存在は保証されるが,次のように構成できる.
    独立にパラメータ$\lambda$の指数分布\ref{def-exponential-distribution}に従う$\R_+$-値確率変数列$(S_j)$を取る:$P[S_j\ge t]=e^{-\lambda t}$.
    なお,パラメータ$\lambda>0$の指数分布密度関数は
    \[p(x)=\lambda e^{-\lambda x}1_{[0,\infty)}(x)\]
    と表せる.
    このとき
    \[Z_0=0,\quad Z_k=\sum^k_{j=1}S_j\]
    とするとこれは強さ$\lambda$のPoisson点過程であり,
    $t\in\R_+$を超えた$Z_k$の数の過程
    \[N_t:=\max\Brace{k\in\N\mid Z_k\le t}\in\o{\Z}_+\]
    はPoisson過程となり,$P[N_t\in\Z_+]=1$.
\end{discussion}

\subsection{独立増分性}

\begin{proposition}
    $0\le s<t$について,$N_t-N_s$は$\F^N_s=\sigma[N_u;u\le s]$と独立である.
\end{proposition}

\begin{corollary}
    $(N_t-\lambda t)_{t\ge0}$は$(\F_t^N)$に関してマルチンゲールである.
\end{corollary}

\section{無限分解可能分布}

\subsection{定義と特徴付け}

\begin{tcolorbox}[colframe=ForestGreen, colback=ForestGreen!10!white,breakable,colbacktitle=ForestGreen!40!white,coltitle=black,fonttitle=\bfseries\sffamily,
title=]
    確率変数$X$が,任意の$n\in\N$に対して,ある分布$\mu_n$が存在してそれに従う独立同分布変数$n$個の和で表せるとき,
    これを無限可解であるという.
    これは半群$P(\R)$\ref{lemma-semigroup-of-distributions}の言葉を用いて定義できる.
\end{tcolorbox}

\begin{definition}
    1次元の分布$\mu$が\textbf{無限可解}であるとは,分布族$\{\mu_t\}_{t\in\R_+}$が存在して,$\mu$を乗法単位元として$\R_+$と同型な連続半群となることをいう:
    \begin{enumerate}
        \item $\R_+\to P(\R);t\mapsto\mu_t$は弱位相に関して連続.
        \item $\forall_{t,s\in\R_+}\;\mu_t*\mu_s=\mu_{t+s}$.
        \item $\mu_0=\delta_0$.
        \item $\mu_1=\mu$.
    \end{enumerate}
\end{definition}

\begin{lemma}
    次の4条件は同値.
    \begin{enumerate}
        \item $\mu$は無限可解である.
        \item $\forall_{n\in\N}\;\mu=\mu_n*\mu_n*\cdots *\mu_n$と表せる.
        \item 任意の$\ep>0$に対し,分解$\mu=\mu_1*\cdots*\mu_n$であって,Levy距離に関して$d_L(\mu_i,\delta)<\ep$を満たすものが存在する.
        \item $\mu_n=\mu_{n1}*\cdots*\mu_{nm(n)},\max_{k\in[m(n)]}d_L(\mu_{nk},\delta)\to0$を満たす列$(\mu_n)$が存在して,$\mu_n\to\mu$.
    \end{enumerate}
\end{lemma}

\begin{example}
    畳み込みについて閉じている分布族\ref{exp-reproducing-families}はみな無限可解な分布の例である.
\end{example}

\begin{example}
    時間的に一様なLevy過程$X$は,任意の$t\in\R_+$について$X_t$は無限可解である.
    また逆に,連続半群$(\mu_t)$に対して,これが定める一様なLevy過程が存在する.
\end{example}

\subsection{Levy分解}

\begin{tcolorbox}[colframe=ForestGreen, colback=ForestGreen!10!white,breakable,colbacktitle=ForestGreen!40!white,coltitle=black,fonttitle=\bfseries\sffamily,
title=]
    無限可解分布,連続半群$\{\mu_t\}_{t\in\R_+}$,時間的に一様なLevy過程(の法則同値類)$X$の間に,次の全単射対応がある:
    \[\mu\xrightarrow{\mu=\mu_1}\{\mu_t\}\xrightarrow{\mu_t=P^{X(t)}}X\]
\end{tcolorbox}

\begin{theorem}[無限可解分布のLevy分解定理]
    任意の無限可解分布$\mu$は,特性関数$\F\mu(z)$が$\F\mu(z)=e^{\psi(z)}$なる形になる.ただし,
    \[\psi(z)=imz-\frac{v}{2}z^2+\int_{\abs{u}>0}(e^{izu}-1-i\phi(u)z)n(du),\quad m\in\R,v\ge0,\int_{\abs{u}>0}(u^2\land 1)n(du)<\infty.\]
\end{theorem}

\begin{definition}
    Poisson分布,Cauchy分布もLevy分解を定め,これらに対応する一様Levy過程をPoisson過程,Cauchy過程という.
\end{definition}

\subsection{複合Poisson過程}

\begin{definition}
        $m=0,v=0$のときの$\psi(z)=\int_{\abs{u>0}}(e^{izu}-1)n(du)$と表せる$\mu$のクラスを,\textbf{複合Poisson分布}という.これに対応する過程を\textbf{複合Poisson過程}という.
\end{definition}

\begin{lemma}
    複合Poisson分布$\mu$は,$v^{*n}$をPoisson分布$p_\lambda$によって加重平均を取ったものである:
    \[\mu=\sum^\infty_{n=1}e^{-\lambda}\frac{\lambda^n}{n!}\nu^{*n}.\]
\end{lemma}

\chapter{拡散過程}

\section{1次元拡散過程}

\begin{tcolorbox}[colframe=ForestGreen, colback=ForestGreen!10!white,breakable,colbacktitle=ForestGreen!40!white,coltitle=black,fonttitle=\bfseries\sffamily,
title=]
    与えられた領域$U$をほとんど確実に出ていく強Markov過程を拡散過程という.
\end{tcolorbox}

\begin{notation}
    $W:=C(\R_+)$上に確率測度の族$(P_x)_{x\in\R}$を考える.
    射影を$X_t:=\pr_t:W\to\R;\om\mapsto\om(t)\;(t\in\R_+)$と表すと,この見本過程$\R_+\to\Map(W,\R)$は任意の$x\in\R$について$(W,P_x)$上連続.
    $\B_t:=\sigma[X_s;s\le t]$とする.
\end{notation}

\begin{definition}[diffusion process]
    確率過程$(X_t)_{t\in\R_+}$が$P_x[X_0=x]=1$と次の条件を満たすとき,$\M:=(\M_x)_{x\in\R},\M_x:=\Brace{X_t(\om)\in\R\mid t\in\R_+,\om\in(W,P_x)}$を,$x$から始まる\textbf{一様な連続強Markov過程}または\textbf{拡散過程}という.
    \begin{quote}
        (強Markov性) $x\in\R$と有限な$(B_t)$-Markov時刻$\tau:W\to\R_+$,$E\in\B^1(\R^1)$について,
        $P_x[X_{\tau+t}\in E\mid\B_\tau]=P_x[X_t\in E]|_{x=X_\tau}$.
    \end{quote}
\end{definition}

\begin{definition}[regular point, regular diffusion process]
    拡散過程$(\M_x)_{x\in\R}$について,
    \begin{enumerate}
        \item $x$は$\M$の\textbf{正則点}であるとは,$P_x[\exists_{t\in\R_+}\;X_t>x]>0,P_x[\exists_{t\in\R_+}\;X_t<x]>0$が成り立つことをいう.
        \item 任意の$x\in\R$が正則点であるとき,$\M$は\textbf{正則}であるという.
    \end{enumerate}
\end{definition}

\chapter{定常過程と時系列解析}

\begin{quotation}
    加法過程とは,増分が定常な過程である.
\end{quotation}

\section{定常過程}

\begin{tcolorbox}[colframe=ForestGreen, colback=ForestGreen!10!white,breakable,colbacktitle=ForestGreen!40!white,coltitle=black,fonttitle=\bfseries\sffamily,
title=]
    定常過程については,スペクトル分解とエルゴード定理が証明できる.
\end{tcolorbox}

\begin{definition}[weak / strong stationary stochastic process]
    確率過程$(X_t)_{t\in\R}$について,
    \begin{enumerate}
        \item $\forall_{t,s,h\in\R}\;m(t+h)=m(t),\Gamma(t+h,s+h)=\Gamma(t,s)$が成り立つとき,すなわち$m$が定数で$\Gamma(s,t)$は$\abs{t-s}$の関数であるとき,\textbf{弱定常過程}という.
        \item 任意の$n\in\N,\{t_i\}_{i\in[n]}\subset\R$について,有限次元分布が任意の平行移動$h\in\R$について等しい:$\Phi_{t_1+h,\cdots,t_n+h}=\Phi_{t_1,\cdots,t_n}$がとき,\textbf{強定常過程}という.
    \end{enumerate}
\end{definition}

\begin{lemma}
    過程$(X_t)_{t\in\R}$について,
    \begin{enumerate}
        \item 強定常かつ$X_0\in L^2(\Om)$のとき,弱定常である.
        \item $(X_t)_{t\in\R}$がGaussであるとき,弱定常性と強定常性とは同値.
    \end{enumerate}
\end{lemma}

\chapter{参考文献}

\begin{thebibliography}{99}
    \bibitem{Williams}
    Williams - Probability with Martingales
    \bibitem{伊藤清確率論}
    伊藤清『確率論』
\end{thebibliography}

\end{document}