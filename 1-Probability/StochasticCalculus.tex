\documentclass[uplatex,dvipdfmx]{jsreport}
\title{確率解析}
\author{司馬博文}
\date{\today}
\pagestyle{headings} \setcounter{secnumdepth}{4}
\usepackage{mathtools}
%\mathtoolsset{showonlyrefs=true} %labelを附した数式にのみ附番される設定.
%\usepackage{amsmath} %mathtoolsの内部で呼ばれるので要らない.
\usepackage{amsfonts} %mathfrak, mathcal, mathbbなど.
\usepackage{amsthm} %定理環境.
\usepackage{amssymb} %AMSFontsを使うためのパッケージ.
\usepackage{ascmac} %screen, itembox, shadebox環境.全てLATEX2εの標準機能の範囲で作られたもの.
\usepackage{comment} %comment環境を用いて,複数行をcomment outできるようにするpackage
\usepackage{wrapfig} %図の周りに文字をwrapさせることができる.詳細な制御ができる.
\usepackage[usenames, dvipsnames]{xcolor} %xcolorはcolorの拡張.optionの意味はdvipsnamesはLoad a set of predefined colors. forestgreenなどの色が追加されている.usenamesはobsoleteとだけ書いてあった.
\setcounter{tocdepth}{2} %目次に表示される深さ.2はsubsectionまで
\usepackage{multicol} %\begin{multicols}{2}環境で途中からmulticolumnに出来る.

\usepackage{url}
\usepackage[dvipdfmx,colorlinks,linkcolor=blue,urlcolor=blue]{hyperref} %生成されるPDFファイルにおいて、\tableofcontentsによって書き出された目次をクリックすると該当する見出しへジャンプしたり、さらには、\label{ラベル名}を番号で参照する\ref{ラベル名}やthebibliography環境において\bibitem{ラベル名}を文献番号で参照する\cite{ラベル名}においても番号をクリックすると該当箇所にジャンプする.囲み枠はダサいので,colorlinksで囲み廃止し,リンク自体に色を付けることにした.
\usepackage{pxjahyper} %pxrubrica同様,八登崇之さん.hyperrefは日本語pLaTeXに最適化されていないから,hyperrefとセットで,(u)pLaTeX+hyperref+dvipdfmxの組み合わせで日本語を含む「しおり」をもつPDF文書を作成する場合に必要となる機能を提供する
\definecolor{花緑青}{cmyk}{0.52,0.03,0,0.27}
\definecolor{サーモンピンク}{cmyk}{0,0.65,0.65,0.05}
\definecolor{暗中模索}{rgb}{0.2,0.2,0.2}

\usepackage{tikz}
\usetikzlibrary{positioning,automata} %automaton描画のため
\usepackage{tikz-cd}
\usepackage[all]{xy}
\def\objectstyle{\displaystyle} %デフォルトではxymatrix中の数式が文中数式モードになるので,それを直す.\labelstyleも同様にxy packageの中で定義されており,文中数式モードになっている.

\usepackage[version=4]{mhchem} %化学式をTikZで簡単に書くためのパッケージ.
\usepackage{chemfig} %化学構造式をTikZで描くためのパッケージ.
\usepackage{siunitx} %IS単位を書くためのパッケージ

\usepackage{ulem} %取り消し線を引くためのパッケージ
\usepackage{pxrubrica} %日本語にルビをふる.八登崇之(やとうたかゆき)氏による.

\usepackage{graphicx} %rotatebox, scalebox, reflectbox, resizeboxなどのコマンドや,図表の読み込み\includegraphicsを司る.graphics というパッケージもありますが,graphicx はこれを高機能にしたものと考えて結構です(ただし graphicx は内部で graphics を読み込みます)

\usepackage[breakable]{tcolorbox} %加藤晃史さんがフル活用していたtcolorboxを,途中改ページ可能で.
\tcbuselibrary{theorems} %https://qiita.com/t_kemmochi/items/483b8fcdb5db8d1f5d5e
\usepackage{enumerate} %enumerate環境を凝らせる.
\usepackage[top=15truemm,bottom=15truemm,left=10truemm,right=10truemm]{geometry} %足助さんからもらったオプション

%%%%%%%%%%%%%%% 環境マクロ %%%%%%%%%%%%%%%

\usepackage{listings} %ソースコードを表示できる環境.多分もっといい方法ある.
\usepackage{jvlisting} %日本語のコメントアウトをする場合jlistingが必要
\lstset{ %ここからソースコードの表示に関する設定.lstlisting環境では,[caption=hoge,label=fuga]などのoptionを付けられる.
%[escapechar=!]とすると,LaTeXコマンドを使える.
  basicstyle={\ttfamily},
  identifierstyle={\small},
  commentstyle={\smallitshape},
  keywordstyle={\small\bfseries},
  ndkeywordstyle={\small},
  stringstyle={\small\ttfamily},
  frame={tb},
  breaklines=true,
  columns=[l]{fullflexible},
  numbers=left,
  xrightmargin=0zw,
  xleftmargin=3zw,
  numberstyle={\scriptsize},
  stepnumber=1,
  numbersep=1zw,
  lineskip=-0.5ex
}
%\makeatletter %caption番号を「[chapter番号].[section番号].[subsection番号]-[そのsubsection内においてn番目]」に変更
%    \AtBeginDocument{
%    \renewcommand*{\thelstlisting}{\arabic{chapter}.\arabic{section}.\arabic{lstlisting}}
%    \@addtoreset{lstlisting}{section}
%    }
%\makeatother
\renewcommand{\lstlistingname}{算譜} %caption名を"program"に変更

\newtcolorbox{tbox}[3][]{%
colframe=#2,colback=#2!10,coltitle=#2!20!black,title={#3},#1}

%%%%%%%%%%%%%%% フォント %%%%%%%%%%%%%%%

\usepackage{textcomp, mathcomp} %Text Companionとは,T1 encodingに入らなかった文字群.これを使うためのパッケージ.\textsectionでブルバキに!
\usepackage[T1]{fontenc} %8bitエンコーディングにする.comp系拡張数学文字の動作が安定する.

%%%%%%%%%%%%%%% 数学記号のマクロ %%%%%%%%%%%%%%%

\newcommand{\abs}[1]{\lvert#1\rvert} %mathtoolsはこうやって使うのか!
\newcommand{\Abs}[1]{\left|#1\right|}
\newcommand{\norm}[1]{\|#1\|}
\newcommand{\Norm}[1]{\left\|#1\right\|}
%\newcommand{\brace}[1]{\{#1\}}
\newcommand{\Brace}[1]{\left\{#1\right\}}
\newcommand{\paren}[1]{\left(#1\right)}
\newcommand{\bracket}[1]{\langle#1\rangle}
\newcommand{\brac}[1]{\langle#1\rangle}
\newcommand{\Bracket}[1]{\left\langle#1\right\rangle}
\newcommand{\Brac}[1]{\left\langle#1\right\rangle}
\newcommand{\Square}[1]{\left[#1\right]}
\renewcommand{\o}[1]{\overline{#1}}
\renewcommand{\u}[1]{\underline{#1}}
\renewcommand{\iff}{\;\mathrm{iff}\;} %nLabリスペクト
\newcommand{\pp}[2]{\frac{\partial #1}{\partial #2}}
\newcommand{\ppp}[3]{\frac{\partial #1}{\partial #2\partial #3}}
\newcommand{\dd}[2]{\frac{d #1}{d #2}}
\newcommand{\floor}[1]{\lfloor#1\rfloor}
\newcommand{\Floor}[1]{\left\lfloor#1\right\rfloor}
\newcommand{\ceil}[1]{\lceil#1\rceil}

\newcommand{\iso}{\xrightarrow{\,\smash{\raisebox{-0.45ex}{\ensuremath{\scriptstyle\sim}}}\,}}
\newcommand{\wt}[1]{\widetilde{#1}}
\newcommand{\wh}[1]{\widehat{#1}}

\newcommand{\Lrarrow}{\;\;\Leftrightarrow\;\;}

%ノルム位相についての閉包 https://newbedev.com/how-to-make-double-overline-with-less-vertical-displacement
\makeatletter
\newcommand{\dbloverline}[1]{\overline{\dbl@overline{#1}}}
\newcommand{\dbl@overline}[1]{\mathpalette\dbl@@overline{#1}}
\newcommand{\dbl@@overline}[2]{%
  \begingroup
  \sbox\z@{$\m@th#1\overline{#2}$}%
  \ht\z@=\dimexpr\ht\z@-2\dbl@adjust{#1}\relax
  \box\z@
  \ifx#1\scriptstyle\kern-\scriptspace\else
  \ifx#1\scriptscriptstyle\kern-\scriptspace\fi\fi
  \endgroup
}
\newcommand{\dbl@adjust}[1]{%
  \fontdimen8
  \ifx#1\displaystyle\textfont\else
  \ifx#1\textstyle\textfont\else
  \ifx#1\scriptstyle\scriptfont\else
  \scriptscriptfont\fi\fi\fi 3
}
\makeatother
\newcommand{\oo}[1]{\dbloverline{#1}}

\DeclareMathOperator{\grad}{\mathrm{grad}}
\DeclareMathOperator{\rot}{\mathrm{rot}}
\DeclareMathOperator{\divergence}{\mathrm{div}}
\newcommand{\False}{\mathrm{False}}
\newcommand{\True}{\mathrm{True}}
\DeclareMathOperator{\tr}{\mathrm{tr}}
\newcommand{\M}{\mathcal{M}}
\newcommand{\cF}{\mathcal{F}}
\newcommand{\cD}{\mathcal{D}}
\newcommand{\fX}{\mathfrak{X}}
\newcommand{\fY}{\mathfrak{Y}}
\newcommand{\fZ}{\mathfrak{Z}}
\renewcommand{\H}{\mathcal{H}}
\newcommand{\fH}{\mathfrak{H}}
\newcommand{\bH}{\mathbb{H}}
\newcommand{\id}{\mathrm{id}}
\newcommand{\A}{\mathcal{A}}
% \renewcommand\coprod{\rotatebox[origin=c]{180}{$\prod$}} すでにどこかにある.
\newcommand{\pr}{\mathrm{pr}}
\newcommand{\U}{\mathfrak{U}}
\newcommand{\Map}{\mathrm{Map}}
\newcommand{\dom}{\mathrm{Dom}\;}
\newcommand{\cod}{\mathrm{Cod}\;}
\newcommand{\supp}{\mathrm{supp}\;}
\newcommand{\otherwise}{\mathrm{otherwise}}
\newcommand{\st}{\;\mathrm{s.t.}\;}
\newcommand{\lmd}{\lambda}
\newcommand{\Lmd}{\Lambda}
%%% 線型代数学
\newcommand{\Ker}{\mathrm{Ker}\;}
\newcommand{\Coker}{\mathrm{Coker}\;}
\newcommand{\Coim}{\mathrm{Coim}\;}
\newcommand{\rank}{\mathrm{rank}}
\newcommand{\lcm}{\mathrm{lcm}}
\newcommand{\sgn}{\mathrm{sgn}}
\newcommand{\GL}{\mathrm{GL}}
\newcommand{\SL}{\mathrm{SL}}
\newcommand{\alt}{\mathrm{alt}}
%%% 複素解析学
\renewcommand{\Re}{\mathrm{Re}\;}
\renewcommand{\Im}{\mathrm{Im}\;}
\newcommand{\Gal}{\mathrm{Gal}}
\newcommand{\PGL}{\mathrm{PGL}}
\newcommand{\PSL}{\mathrm{PSL}}
\newcommand{\Log}{\mathrm{Log}\,}
\newcommand{\Res}{\mathrm{Res}\,}
\newcommand{\on}{\mathrm{on}\;}
\newcommand{\hatC}{\hat{\C}}
\newcommand{\hatR}{\hat{\R}}
\newcommand{\PV}{\mathrm{P.V.}}
\newcommand{\diam}{\mathrm{diam}}
\newcommand{\Area}{\mathrm{Area}}
\newcommand{\Lap}{\Laplace}
\newcommand{\f}{\mathbf{f}}
\newcommand{\cR}{\mathcal{R}}
\newcommand{\const}{\mathrm{const.}}
\newcommand{\Om}{\Omega}
\newcommand{\Cinf}{C^\infty}
\newcommand{\ep}{\epsilon}
\newcommand{\dist}{\mathrm{dist}}
\newcommand{\opart}{\o{\partial}}
%%% 解析力学
\newcommand{\x}{\mathbf{x}}
%%% 集合と位相
\renewcommand{\O}{\mathcal{O}}
\renewcommand{\S}{\mathcal{S}}
\renewcommand{\U}{\mathcal{U}}
\newcommand{\V}{\mathcal{V}}
\renewcommand{\P}{\mathcal{P}}
\newcommand{\R}{\mathbb{R}}
\newcommand{\N}{\mathbb{N}}
\newcommand{\C}{\mathbb{C}}
\newcommand{\Z}{\mathbb{Z}}
\newcommand{\Q}{\mathbb{Q}}
\newcommand{\TV}{\mathrm{TV}}
\newcommand{\ORD}{\mathrm{ORD}}
\newcommand{\Tr}{\mathrm{Tr}\;}
\newcommand{\Card}{\mathrm{Card}\;}
\newcommand{\Top}{\mathrm{Top}}
\newcommand{\Disc}{\mathrm{Disc}}
\newcommand{\Codisc}{\mathrm{Codisc}}
\newcommand{\CoDisc}{\mathrm{CoDisc}}
\newcommand{\Ult}{\mathrm{Ult}}
\newcommand{\ord}{\mathrm{ord}}
\newcommand{\maj}{\mathrm{maj}}
%%% 形式言語理論
\newcommand{\REGEX}{\mathrm{REGEX}}
\newcommand{\RE}{\mathbf{RE}}

%%% Fourier解析
\newcommand*{\Laplace}{\mathop{}\!\mathbin\bigtriangleup}
\newcommand*{\DAlambert}{\mathop{}\!\mathbin\Box}
%%% Graph Theory
\newcommand{\SimpGph}{\mathrm{SimpGph}}
\newcommand{\Gph}{\mathrm{Gph}}
\newcommand{\mult}{\mathrm{mult}}
\newcommand{\inv}{\mathrm{inv}}
%%% 多様体
\newcommand{\Der}{\mathrm{Der}}
\newcommand{\osub}{\overset{\mathrm{open}}{\subset}}
\newcommand{\osup}{\overset{\mathrm{open}}{\supset}}
\newcommand{\al}{\alpha}
\newcommand{\K}{\mathbb{K}}
\newcommand{\Sp}{\mathrm{Sp}}
\newcommand{\g}{\mathfrak{g}}
\newcommand{\h}{\mathfrak{h}}
\newcommand{\Exp}{\mathrm{Exp}\;}
\newcommand{\Imm}{\mathrm{Imm}}
\newcommand{\Imb}{\mathrm{Imb}}
\newcommand{\codim}{\mathrm{codim}\;}
\newcommand{\Gr}{\mathrm{Gr}}
%%% 代数
\newcommand{\Ad}{\mathrm{Ad}}
\newcommand{\finsupp}{\mathrm{fin\;supp}}
\newcommand{\SO}{\mathrm{SO}}
\newcommand{\SU}{\mathrm{SU}}
\newcommand{\acts}{\curvearrowright}
\newcommand{\mono}{\hookrightarrow}
\newcommand{\epi}{\twoheadrightarrow}
\newcommand{\Stab}{\mathrm{Stab}}
\newcommand{\nor}{\mathrm{nor}}
\newcommand{\T}{\mathbb{T}}
\newcommand{\Aff}{\mathrm{Aff}}
\newcommand{\rsub}{\triangleleft}
\newcommand{\rsup}{\triangleright}
\newcommand{\subgrp}{\overset{\mathrm{subgrp}}{\subset}}
\newcommand{\Ext}{\mathrm{Ext}}
\newcommand{\sbs}{\subset}\newcommand{\sps}{\supset}
\newcommand{\In}{\mathrm{In}}
\newcommand{\Tor}{\mathrm{Tor}}
\newcommand{\p}{\mathfrak{p}}
\newcommand{\q}{\mathfrak{q}}
\newcommand{\m}{\mathfrak{m}}
\newcommand{\cS}{\mathcal{S}}
\newcommand{\Frac}{\mathrm{Frac}\,}
\newcommand{\Spec}{\mathrm{Spec}\,}
\newcommand{\bA}{\mathbb{A}}
\newcommand{\Sym}{\mathrm{Sym}}
\newcommand{\Ann}{\mathrm{Ann}}
%%% 代数的位相幾何学
\newcommand{\Ho}{\mathrm{Ho}}
\newcommand{\CW}{\mathrm{CW}}
\newcommand{\lc}{\mathrm{lc}}
\newcommand{\cg}{\mathrm{cg}}
\newcommand{\Fib}{\mathrm{Fib}}
\newcommand{\Cyl}{\mathrm{Cyl}}
\newcommand{\Ch}{\mathrm{Ch}}
%%% 数値解析
\newcommand{\round}{\mathrm{round}}
\newcommand{\cond}{\mathrm{cond}}
\newcommand{\diag}{\mathrm{diag}}
%%% 確率論
\newcommand{\calF}{\mathcal{F}}
\newcommand{\X}{\mathcal{X}}
\newcommand{\Meas}{\mathrm{Meas}}
\newcommand{\as}{\;\mathrm{a.s.}} %almost surely
\newcommand{\io}{\;\mathrm{i.o.}} %infinitely often
\newcommand{\fe}{\;\mathrm{f.e.}} %with a finite number of exceptions
\newcommand{\F}{\mathcal{F}}
\newcommand{\bF}{\mathbb{F}}
\newcommand{\W}{\mathcal{W}}
\newcommand{\Pois}{\mathrm{Pois}}
\newcommand{\iid}{\mathrm{i.i.d.}}
\newcommand{\wconv}{\rightsquigarrow}
\newcommand{\Var}{\mathrm{Var}}
\newcommand{\xrightarrown}{\xrightarrow{n\to\infty}}
\newcommand{\au}{\mathrm{au}}
\newcommand{\cT}{\mathcal{T}}
%%% 情報理論
\newcommand{\bit}{\mathrm{bit}}
%%% 積分論
\newcommand{\calA}{\mathcal{A}}
\newcommand{\calB}{\mathcal{B}}
\newcommand{\D}{\mathcal{D}}
\newcommand{\Y}{\mathcal{Y}}
\newcommand{\calC}{\mathcal{C}}
\renewcommand{\ae}{\mathrm{a.e.}\;}
\newcommand{\cZ}{\mathcal{Z}}
\newcommand{\fF}{\mathfrak{F}}
\newcommand{\fI}{\mathfrak{I}}
\newcommand{\E}{\mathcal{E}}
\newcommand{\sMap}{\sigma\textrm{-}\mathrm{Map}}
\DeclareMathOperator*{\argmax}{arg\,max}
\DeclareMathOperator*{\argmin}{arg\,min}
\newcommand{\cC}{\mathcal{C}}
\newcommand{\comp}{\complement}
\newcommand{\J}{\mathcal{J}}
\newcommand{\sumN}[1]{\sum_{#1\in\N}}
\newcommand{\cupN}[1]{\cup_{#1\in\N}}
\newcommand{\capN}[1]{\cap_{#1\in\N}}
\newcommand{\Sum}[1]{\sum_{#1=1}^\infty}
\newcommand{\sumn}{\sum_{n=1}^\infty}
\newcommand{\summ}{\sum_{m=1}^\infty}
\newcommand{\sumk}{\sum_{k=1}^\infty}
\newcommand{\sumi}{\sum_{i=1}^\infty}
\newcommand{\sumj}{\sum_{j=1}^\infty}
\newcommand{\cupn}{\cup_{n=1}^\infty}
\newcommand{\capn}{\cap_{n=1}^\infty}
\newcommand{\cupk}{\cup_{k=1}^\infty}
\newcommand{\cupi}{\cup_{i=1}^\infty}
\newcommand{\cupj}{\cup_{j=1}^\infty}
\newcommand{\limn}{\lim_{n\to\infty}}
\renewcommand{\l}{\mathcal{l}}
\renewcommand{\L}{\mathcal{L}}
\newcommand{\Cl}{\mathrm{Cl}}
\newcommand{\cN}{\mathcal{N}}
\newcommand{\Ae}{\textrm{-a.e.}\;}
\newcommand{\csub}{\overset{\textrm{closed}}{\subset}}
\newcommand{\csup}{\overset{\textrm{closed}}{\supset}}
\newcommand{\wB}{\wt{B}}
\newcommand{\cG}{\mathcal{G}}
\newcommand{\Lip}{\mathrm{Lip}}
\newcommand{\Dom}{\mathrm{Dom}}
%%% 数理ファイナンス
\newcommand{\pre}{\mathrm{pre}}
\newcommand{\om}{\omega}

%%% 統計的因果推論
\newcommand{\Do}{\mathrm{Do}}
%%% 数理統計
\newcommand{\bP}{\mathbb{P}}
\newcommand{\compsub}{\overset{\textrm{cpt}}{\subset}}
\newcommand{\lip}{\textrm{lip}}
\newcommand{\BL}{\mathrm{BL}}
\newcommand{\G}{\mathbb{G}}
\newcommand{\NB}{\mathrm{NB}}
\newcommand{\oR}{\o{\R}}
\newcommand{\liminfn}{\liminf_{n\to\infty}}
\newcommand{\limsupn}{\limsup_{n\to\infty}}
%\newcommand{\limn}{\lim_{n\to\infty}}
\newcommand{\esssup}{\mathrm{ess.sup}}
\newcommand{\asto}{\xrightarrow{\as}}
\newcommand{\Cov}{\mathrm{Cov}}
\newcommand{\cQ}{\mathcal{Q}}
\newcommand{\VC}{\mathrm{VC}}
\newcommand{\mb}{\mathrm{mb}}
\newcommand{\Avar}{\mathrm{Avar}}
\newcommand{\bB}{\mathbb{B}}
\newcommand{\bW}{\mathbb{W}}
\newcommand{\sd}{\mathrm{sd}}
\newcommand{\w}[1]{\widehat{#1}}
\newcommand{\bZ}{\mathbb{Z}}
\newcommand{\Bernoulli}{\mathrm{Bernoulli}}
\newcommand{\Mult}{\mathrm{Mult}}
\newcommand{\BPois}{\mathrm{BPois}}
\newcommand{\fraks}{\mathfrak{s}}
\newcommand{\frakk}{\mathfrak{k}}
\newcommand{\IF}{\mathrm{IF}}
\newcommand{\bX}{\mathbf{X}}
\newcommand{\bx}{\mathbf{x}}
\newcommand{\indep}{\raisebox{0.05em}{\rotatebox[origin=c]{90}{$\models$}}}
\newcommand{\IG}{\mathrm{IG}}
\newcommand{\Levy}{\mathrm{Levy}}
\newcommand{\MP}{\mathrm{MP}}
\newcommand{\Hermite}{\mathrm{Hermite}}
\newcommand{\Skellam}{\mathrm{Skellam}}
\newcommand{\Dirichlet}{\mathrm{Dirichlet}}
\newcommand{\Beta}{\mathrm{Beta}}
\newcommand{\bE}{\mathbb{E}}
\newcommand{\bG}{\mathbb{G}}
\newcommand{\MISE}{\mathrm{MISE}}
\newcommand{\logit}{\mathtt{logit}}
\newcommand{\expit}{\mathtt{expit}}
\newcommand{\cK}{\mathcal{K}}
\newcommand{\dl}{\dot{l}}
\newcommand{\dotp}{\dot{p}}
\newcommand{\wl}{\wt{l}}
%%% 函数解析
\renewcommand{\c}{\mathbf{c}}
\newcommand{\loc}{\mathrm{loc}}
\newcommand{\Lh}{\mathrm{L.h.}}
\newcommand{\Epi}{\mathrm{Epi}\;}
\newcommand{\slim}{\mathrm{slim}}
\newcommand{\Ban}{\mathrm{Ban}}
\newcommand{\Hilb}{\mathrm{Hilb}}
\newcommand{\Ex}{\mathrm{Ex}}
\newcommand{\Co}{\mathrm{Co}}
\newcommand{\sa}{\mathrm{sa}}
\newcommand{\nnorm}[1]{{\left\vert\kern-0.25ex\left\vert\kern-0.25ex\left\vert #1 \right\vert\kern-0.25ex\right\vert\kern-0.25ex\right\vert}}
\newcommand{\dvol}{\mathrm{dvol}}
\newcommand{\Sconv}{\mathrm{Sconv}}
\newcommand{\I}{\mathcal{I}}
\newcommand{\nonunital}{\mathrm{nu}}
\newcommand{\cpt}{\mathrm{cpt}}
\newcommand{\lcpt}{\mathrm{lcpt}}
\newcommand{\com}{\mathrm{com}}
\newcommand{\Haus}{\mathrm{Haus}}
\newcommand{\proper}{\mathrm{proper}}
\newcommand{\infinity}{\mathrm{inf}}
\newcommand{\TVS}{\mathrm{TVS}}
\newcommand{\ess}{\mathrm{ess}}
\newcommand{\ext}{\mathrm{ext}}
\newcommand{\Index}{\mathrm{Index}}
\newcommand{\SSR}{\mathrm{SSR}}
\newcommand{\vs}{\mathrm{vs.}}
\newcommand{\fM}{\mathfrak{M}}
\newcommand{\EDM}{\mathrm{EDM}}
\newcommand{\Tw}{\mathrm{Tw}}
\newcommand{\fC}{\mathfrak{C}}
\newcommand{\bn}{\mathbf{n}}
\newcommand{\br}{\mathbf{r}}
\newcommand{\Lam}{\Lambda}
\newcommand{\lam}{\lambda}
\newcommand{\one}{\mathbf{1}}
\newcommand{\dae}{\text{-a.e.}}
\newcommand{\td}{\text{-}}
\newcommand{\RM}{\mathrm{RM}}
%%% 最適化
\newcommand{\Minimize}{\text{Minimize}}
\newcommand{\subjectto}{\text{subject to}}
\newcommand{\Ri}{\mathrm{Ri}}
%\newcommand{\Cl}{\mathrm{Cl}}
\newcommand{\Cone}{\mathrm{Cone}}
\newcommand{\Int}{\mathrm{Int}}
%%% 圏
\newcommand{\varlim}{\varprojlim}
\newcommand{\Hom}{\mathrm{Hom}}
\newcommand{\Iso}{\mathrm{Iso}}
\newcommand{\Mor}{\mathrm{Mor}}
\newcommand{\Isom}{\mathrm{Isom}}
\newcommand{\Aut}{\mathrm{Aut}}
\newcommand{\End}{\mathrm{End}}
\newcommand{\op}{\mathrm{op}}
\newcommand{\ev}{\mathrm{ev}}
\newcommand{\Ob}{\mathrm{Ob}}
\newcommand{\Ar}{\mathrm{Ar}}
\newcommand{\Arr}{\mathrm{Arr}}
\newcommand{\Set}{\mathrm{Set}}
\newcommand{\Grp}{\mathrm{Grp}}
\newcommand{\Cat}{\mathrm{Cat}}
\newcommand{\Mon}{\mathrm{Mon}}
\newcommand{\CMon}{\mathrm{CMon}} %Comutative Monoid 可換単系とモノイドの射
\newcommand{\Ring}{\mathrm{Ring}}
\newcommand{\CRing}{\mathrm{CRing}}
\newcommand{\Ab}{\mathrm{Ab}}
\newcommand{\Pos}{\mathrm{Pos}}
\newcommand{\Vect}{\mathrm{Vect}}
\newcommand{\FinVect}{\mathrm{FinVect}}
\newcommand{\FinSet}{\mathrm{FinSet}}
\newcommand{\OmegaAlg}{\Omega$-$\mathrm{Alg}}
\newcommand{\OmegaEAlg}{(\Omega,E)$-$\mathrm{Alg}}
\newcommand{\Alg}{\mathrm{Alg}} %代数の圏
\newcommand{\CAlg}{\mathrm{CAlg}} %可換代数の圏
\newcommand{\CPO}{\mathrm{CPO}} %Complete Partial Order & continuous mappings
\newcommand{\Fun}{\mathrm{Fun}}
\newcommand{\Func}{\mathrm{Func}}
\newcommand{\Met}{\mathrm{Met}} %Metric space & Contraction maps
\newcommand{\Pfn}{\mathrm{Pfn}} %Sets & Partial function
\newcommand{\Rel}{\mathrm{Rel}} %Sets & relation
\newcommand{\Bool}{\mathrm{Bool}}
\newcommand{\CABool}{\mathrm{CABool}}
\newcommand{\CompBoolAlg}{\mathrm{CompBoolAlg}}
\newcommand{\BoolAlg}{\mathrm{BoolAlg}}
\newcommand{\BoolRng}{\mathrm{BoolRng}}
\newcommand{\HeytAlg}{\mathrm{HeytAlg}}
\newcommand{\CompHeytAlg}{\mathrm{CompHeytAlg}}
\newcommand{\Lat}{\mathrm{Lat}}
\newcommand{\CompLat}{\mathrm{CompLat}}
\newcommand{\SemiLat}{\mathrm{SemiLat}}
\newcommand{\Stone}{\mathrm{Stone}}
\newcommand{\Sob}{\mathrm{Sob}} %Sober space & continuous map
\newcommand{\Op}{\mathrm{Op}} %Category of open subsets
\newcommand{\Sh}{\mathrm{Sh}} %Category of sheave
\newcommand{\PSh}{\mathrm{PSh}} %Category of presheave, PSh(C)=[C^op,set]のこと
\newcommand{\Conv}{\mathrm{Conv}} %Convergence spaceの圏
\newcommand{\Unif}{\mathrm{Unif}} %一様空間と一様連続写像の圏
\newcommand{\Frm}{\mathrm{Frm}} %フレームとフレームの射
\newcommand{\Locale}{\mathrm{Locale}} %その反対圏
\newcommand{\Diff}{\mathrm{Diff}} %滑らかな多様体の圏
\newcommand{\Mfd}{\mathrm{Mfd}}
\newcommand{\LieAlg}{\mathrm{LieAlg}}
\newcommand{\Quiv}{\mathrm{Quiv}} %Quiverの圏
\newcommand{\B}{\mathcal{B}}
\newcommand{\Span}{\mathrm{Span}}
\newcommand{\Corr}{\mathrm{Corr}}
\newcommand{\Decat}{\mathrm{Decat}}
\newcommand{\Rep}{\mathrm{Rep}}
\newcommand{\Grpd}{\mathrm{Grpd}}
\newcommand{\sSet}{\mathrm{sSet}}
\newcommand{\Mod}{\mathrm{Mod}}
\newcommand{\SmoothMnf}{\mathrm{SmoothMnf}}
\newcommand{\coker}{\mathrm{coker}}

\newcommand{\Ord}{\mathrm{Ord}}
\newcommand{\eq}{\mathrm{eq}}
\newcommand{\coeq}{\mathrm{coeq}}
\newcommand{\act}{\mathrm{act}}

%%%%%%%%%%%%%%% 定理環境(足助先生ありがとうございます) %%%%%%%%%%%%%%%

\everymath{\displaystyle}
\renewcommand{\proofname}{\bf [証明]}
\renewcommand{\thefootnote}{\dag\arabic{footnote}} %足助さんからもらった.どうなるんだ?
\renewcommand{\qedsymbol}{$\blacksquare$}

\renewcommand{\labelenumi}{(\arabic{enumi})} %(1),(2),...がデフォルトであって欲しい
\renewcommand{\labelenumii}{(\alph{enumii})}
\renewcommand{\labelenumiii}{(\roman{enumiii})}

\newtheoremstyle{StatementsWithStar}% ?name?
{3pt}% ?Space above? 1
{3pt}% ?Space below? 1
{}% ?Body font?
{}% ?Indent amount? 2
{\bfseries}% ?Theorem head font?
{\textbf{.}}% ?Punctuation after theorem head?
{.5em}% ?Space after theorem head? 3
{\textbf{\textup{#1~\thetheorem{}}}{}\,$^{\ast}$\thmnote{(#3)}}% ?Theorem head spec (can be left empty, meaning ‘normal’)?
%
\newtheoremstyle{StatementsWithStar2}% ?name?
{3pt}% ?Space above? 1
{3pt}% ?Space below? 1
{}% ?Body font?
{}% ?Indent amount? 2
{\bfseries}% ?Theorem head font?
{\textbf{.}}% ?Punctuation after theorem head?
{.5em}% ?Space after theorem head? 3
{\textbf{\textup{#1~\thetheorem{}}}{}\,$^{\ast\ast}$\thmnote{(#3)}}% ?Theorem head spec (can be left empty, meaning ‘normal’)?
%
\newtheoremstyle{StatementsWithStar3}% ?name?
{3pt}% ?Space above? 1
{3pt}% ?Space below? 1
{}% ?Body font?
{}% ?Indent amount? 2
{\bfseries}% ?Theorem head font?
{\textbf{.}}% ?Punctuation after theorem head?
{.5em}% ?Space after theorem head? 3
{\textbf{\textup{#1~\thetheorem{}}}{}\,$^{\ast\ast\ast}$\thmnote{(#3)}}% ?Theorem head spec (can be left empty, meaning ‘normal’)?
%
\newtheoremstyle{StatementsWithCCirc}% ?name?
{6pt}% ?Space above? 1
{6pt}% ?Space below? 1
{}% ?Body font?
{}% ?Indent amount? 2
{\bfseries}% ?Theorem head font?
{\textbf{.}}% ?Punctuation after theorem head?
{.5em}% ?Space after theorem head? 3
{\textbf{\textup{#1~\thetheorem{}}}{}\,$^{\circledcirc}$\thmnote{(#3)}}% ?Theorem head spec (can be left empty, meaning ‘normal’)?
%
\theoremstyle{definition}
 \newtheorem{theorem}{定理}[section]
 \newtheorem{axiom}[theorem]{公理}
 \newtheorem{corollary}[theorem]{系}
 \newtheorem{proposition}[theorem]{命題}
 \newtheorem*{proposition*}{命題}
 \newtheorem{lemma}[theorem]{補題}
 \newtheorem*{lemma*}{補題}
 \newtheorem*{theorem*}{定理}
 \newtheorem{definition}[theorem]{定義}
 \newtheorem{example}[theorem]{例}
 \newtheorem{notation}[theorem]{記法}
 \newtheorem*{notation*}{記法}
 \newtheorem{assumption}[theorem]{仮定}
 \newtheorem{question}[theorem]{問}
 \newtheorem{counterexample}[theorem]{反例}
 \newtheorem{reidai}[theorem]{例題}
 \newtheorem{ruidai}[theorem]{類題}
 \newtheorem{problem}[theorem]{問題}
 \newtheorem{algorithm}[theorem]{算譜}
 \newtheorem*{solution*}{\bf{[解]}}
 \newtheorem{discussion}[theorem]{議論}
 \newtheorem{remark}[theorem]{注}
 \newtheorem{remarks}[theorem]{要諦}
 \newtheorem{image}[theorem]{描像}
 \newtheorem{observation}[theorem]{観察}
 \newtheorem{universality}[theorem]{普遍性} %非自明な例外がない.
 \newtheorem{universal tendency}[theorem]{普遍傾向} %例外が有意に少ない.
 \newtheorem{hypothesis}[theorem]{仮説} %実験で説明されていない理論.
 \newtheorem{theory}[theorem]{理論} %実験事実とその(さしあたり)整合的な説明.
 \newtheorem{fact}[theorem]{実験事実}
 \newtheorem{model}[theorem]{模型}
 \newtheorem{explanation}[theorem]{説明} %理論による実験事実の説明
 \newtheorem{anomaly}[theorem]{理論の限界}
 \newtheorem{application}[theorem]{応用例}
 \newtheorem{method}[theorem]{手法} %実験手法など,技術的問題.
 \newtheorem{history}[theorem]{歴史}
 \newtheorem{usage}[theorem]{用語法}
 \newtheorem{research}[theorem]{研究}
 \newtheorem{shishin}[theorem]{指針}
 \newtheorem{yodan}[theorem]{余談}
 \newtheorem{construction}[theorem]{構成}
% \newtheorem*{remarknonum}{注}
 \newtheorem*{definition*}{定義}
 \newtheorem*{remark*}{注}
 \newtheorem*{question*}{問}
 \newtheorem*{problem*}{問題}
 \newtheorem*{axiom*}{公理}
 \newtheorem*{example*}{例}
 \newtheorem*{corollary*}{系}
 \newtheorem*{shishin*}{指針}
 \newtheorem*{yodan*}{余談}
 \newtheorem*{kadai*}{課題}
%
\theoremstyle{StatementsWithStar}
 \newtheorem{definition_*}[theorem]{定義}
 \newtheorem{question_*}[theorem]{問}
 \newtheorem{example_*}[theorem]{例}
 \newtheorem{theorem_*}[theorem]{定理}
 \newtheorem{remark_*}[theorem]{注}
%
\theoremstyle{StatementsWithStar2}
 \newtheorem{definition_**}[theorem]{定義}
 \newtheorem{theorem_**}[theorem]{定理}
 \newtheorem{question_**}[theorem]{問}
 \newtheorem{remark_**}[theorem]{注}
%
\theoremstyle{StatementsWithStar3}
 \newtheorem{remark_***}[theorem]{注}
 \newtheorem{question_***}[theorem]{問}
%
\theoremstyle{StatementsWithCCirc}
 \newtheorem{definition_O}[theorem]{定義}
 \newtheorem{question_O}[theorem]{問}
 \newtheorem{example_O}[theorem]{例}
 \newtheorem{remark_O}[theorem]{注}
%
\theoremstyle{definition}
%
\raggedbottom
\allowdisplaybreaks
%\usepackage{mathtools}
%\mathtoolsset{showonlyrefs=true} %labelを附した数式にのみ附番される設定.
%\usepackage{amsmath} %mathtoolsの内部で呼ばれるので要らない.
\usepackage{amsfonts} %mathfrak, mathcal, mathbbなど.
\usepackage{amsthm} %定理環境.
\usepackage{amssymb} %AMSFontsを使うためのパッケージ.
\usepackage{ascmac} %screen, itembox, shadebox環境.全てLATEX2εの標準機能の範囲で作られたもの.
\usepackage{comment} %comment環境を用いて,複数行をcomment outできるようにするpackage
\usepackage{wrapfig} %図の周りに文字をwrapさせることができる.詳細な制御ができる.
\usepackage[usenames, dvipsnames]{xcolor} %xcolorはcolorの拡張.optionの意味はdvipsnamesはLoad a set of predefined colors. forestgreenなどの色が追加されている.usenamesはobsoleteとだけ書いてあった.
\setcounter{tocdepth}{2} %目次に表示される深さ.2はsubsectionまで
\usepackage{multicol} %\begin{multicols}{2}環境で途中からmulticolumnに出来る.

\usepackage{url}
\usepackage[dvipdfmx,colorlinks,linkcolor=blue,urlcolor=blue]{hyperref} %生成されるPDFファイルにおいて、\tableofcontentsによって書き出された目次をクリックすると該当する見出しへジャンプしたり、さらには、\label{ラベル名}を番号で参照する\ref{ラベル名}やthebibliography環境において\bibitem{ラベル名}を文献番号で参照する\cite{ラベル名}においても番号をクリックすると該当箇所にジャンプする.囲み枠はダサいので,colorlinksで囲み廃止し,リンク自体に色を付けることにした.
\usepackage{pxjahyper} %pxrubrica同様,八登崇之さん.hyperrefは日本語pLaTeXに最適化されていないから,hyperrefとセットで,(u)pLaTeX+hyperref+dvipdfmxの組み合わせで日本語を含む「しおり」をもつPDF文書を作成する場合に必要となる機能を提供する
\definecolor{花緑青}{cmyk}{0.52,0.03,0,0.27}
\definecolor{サーモンピンク}{cmyk}{0,0.65,0.65,0.05}
\definecolor{暗中模索}{rgb}{0.2,0.2,0.2}

\usepackage{tikz}
\usetikzlibrary{positioning,automata} %automaton描画のため
\usepackage{tikz-cd}
\usepackage[all]{xy}
\def\objectstyle{\displaystyle} %デフォルトではxymatrix中の数式が文中数式モードになるので,それを直す.\labelstyleも同様にxy packageの中で定義されており,文中数式モードになっている.

\usepackage[version=4]{mhchem} %化学式をTikZで簡単に書くためのパッケージ.
\usepackage{chemfig} %化学構造式をTikZで描くためのパッケージ.
\usepackage{siunitx} %IS単位を書くためのパッケージ

\usepackage{ulem} %取り消し線を引くためのパッケージ
\usepackage{pxrubrica} %日本語にルビをふる.八登崇之(やとうたかゆき)氏による.

\usepackage{graphicx} %rotatebox, scalebox, reflectbox, resizeboxなどのコマンドや,図表の読み込み\includegraphicsを司る.graphics というパッケージもありますが,graphicx はこれを高機能にしたものと考えて結構です(ただし graphicx は内部で graphics を読み込みます)

\usepackage[breakable]{tcolorbox} %加藤晃史さんがフル活用していたtcolorboxを,途中改ページ可能で.
\tcbuselibrary{theorems} %https://qiita.com/t_kemmochi/items/483b8fcdb5db8d1f5d5e
\usepackage{enumerate} %enumerate環境を凝らせる.
\usepackage[top=15truemm,bottom=15truemm,left=10truemm,right=10truemm]{geometry} %足助さんからもらったオプション

%%%%%%%%%%%%%%% 環境マクロ %%%%%%%%%%%%%%%

\usepackage{listings} %ソースコードを表示できる環境.多分もっといい方法ある.
\usepackage{jvlisting} %日本語のコメントアウトをする場合jlistingが必要
\lstset{ %ここからソースコードの表示に関する設定.lstlisting環境では,[caption=hoge,label=fuga]などのoptionを付けられる.
%[escapechar=!]とすると,LaTeXコマンドを使える.
  basicstyle={\ttfamily},
  identifierstyle={\small},
  commentstyle={\smallitshape},
  keywordstyle={\small\bfseries},
  ndkeywordstyle={\small},
  stringstyle={\small\ttfamily},
  frame={tb},
  breaklines=true,
  columns=[l]{fullflexible},
  numbers=left,
  xrightmargin=0zw,
  xleftmargin=3zw,
  numberstyle={\scriptsize},
  stepnumber=1,
  numbersep=1zw,
  lineskip=-0.5ex
}
%\makeatletter %caption番号を「[chapter番号].[section番号].[subsection番号]-[そのsubsection内においてn番目]」に変更
%    \AtBeginDocument{
%    \renewcommand*{\thelstlisting}{\arabic{chapter}.\arabic{section}.\arabic{lstlisting}}
%    \@addtoreset{lstlisting}{section}
%    }
%\makeatother
\renewcommand{\lstlistingname}{算譜} %caption名を"program"に変更

\newtcolorbox{tbox}[3][]{%
colframe=#2,colback=#2!10,coltitle=#2!20!black,title={#3},#1}

%%%%%%%%%%%%%%% フォント %%%%%%%%%%%%%%%

\usepackage{textcomp, mathcomp} %Text Companionとは,T1 encodingに入らなかった文字群.これを使うためのパッケージ.\textsectionでブルバキに!
\usepackage[T1]{fontenc} %8bitエンコーディングにする.comp系拡張数学文字の動作が安定する.

%%%%%%%%%%%%%%% 数学記号のマクロ %%%%%%%%%%%%%%%

\newcommand{\abs}[1]{\lvert#1\rvert} %mathtoolsはこうやって使うのか!
\newcommand{\Abs}[1]{\left|#1\right|}
\newcommand{\norm}[1]{\|#1\|}
\newcommand{\Norm}[1]{\left\|#1\right\|}
%\newcommand{\brace}[1]{\{#1\}}
\newcommand{\Brace}[1]{\left\{#1\right\}}
\newcommand{\paren}[1]{\left(#1\right)}
\newcommand{\bracket}[1]{\langle#1\rangle}
\newcommand{\brac}[1]{\langle#1\rangle}
\newcommand{\Bracket}[1]{\left\langle#1\right\rangle}
\newcommand{\Brac}[1]{\left\langle#1\right\rangle}
\newcommand{\Square}[1]{\left[#1\right]}
\renewcommand{\o}[1]{\overline{#1}}
\renewcommand{\u}[1]{\underline{#1}}
\renewcommand{\iff}{\;\mathrm{iff}\;} %nLabリスペクト
\newcommand{\pp}[2]{\frac{\partial #1}{\partial #2}}
\newcommand{\ppp}[3]{\frac{\partial #1}{\partial #2\partial #3}}
\newcommand{\dd}[2]{\frac{d #1}{d #2}}
\newcommand{\floor}[1]{\lfloor#1\rfloor}
\newcommand{\Floor}[1]{\left\lfloor#1\right\rfloor}
\newcommand{\ceil}[1]{\lceil#1\rceil}

\newcommand{\iso}{\xrightarrow{\,\smash{\raisebox{-0.45ex}{\ensuremath{\scriptstyle\sim}}}\,}}
\newcommand{\wt}[1]{\widetilde{#1}}
\newcommand{\wh}[1]{\widehat{#1}}

\newcommand{\Lrarrow}{\;\;\Leftrightarrow\;\;}

%ノルム位相についての閉包 https://newbedev.com/how-to-make-double-overline-with-less-vertical-displacement
\makeatletter
\newcommand{\dbloverline}[1]{\overline{\dbl@overline{#1}}}
\newcommand{\dbl@overline}[1]{\mathpalette\dbl@@overline{#1}}
\newcommand{\dbl@@overline}[2]{%
  \begingroup
  \sbox\z@{$\m@th#1\overline{#2}$}%
  \ht\z@=\dimexpr\ht\z@-2\dbl@adjust{#1}\relax
  \box\z@
  \ifx#1\scriptstyle\kern-\scriptspace\else
  \ifx#1\scriptscriptstyle\kern-\scriptspace\fi\fi
  \endgroup
}
\newcommand{\dbl@adjust}[1]{%
  \fontdimen8
  \ifx#1\displaystyle\textfont\else
  \ifx#1\textstyle\textfont\else
  \ifx#1\scriptstyle\scriptfont\else
  \scriptscriptfont\fi\fi\fi 3
}
\makeatother
\newcommand{\oo}[1]{\dbloverline{#1}}

\DeclareMathOperator{\grad}{\mathrm{grad}}
\DeclareMathOperator{\rot}{\mathrm{rot}}
\DeclareMathOperator{\divergence}{\mathrm{div}}
\newcommand{\False}{\mathrm{False}}
\newcommand{\True}{\mathrm{True}}
\DeclareMathOperator{\tr}{\mathrm{tr}}
\newcommand{\M}{\mathcal{M}}
\newcommand{\cF}{\mathcal{F}}
\newcommand{\cD}{\mathcal{D}}
\newcommand{\fX}{\mathfrak{X}}
\newcommand{\fY}{\mathfrak{Y}}
\newcommand{\fZ}{\mathfrak{Z}}
\renewcommand{\H}{\mathcal{H}}
\newcommand{\fH}{\mathfrak{H}}
\newcommand{\bH}{\mathbb{H}}
\newcommand{\id}{\mathrm{id}}
\newcommand{\A}{\mathcal{A}}
% \renewcommand\coprod{\rotatebox[origin=c]{180}{$\prod$}} すでにどこかにある.
\newcommand{\pr}{\mathrm{pr}}
\newcommand{\U}{\mathfrak{U}}
\newcommand{\Map}{\mathrm{Map}}
\newcommand{\dom}{\mathrm{Dom}\;}
\newcommand{\cod}{\mathrm{Cod}\;}
\newcommand{\supp}{\mathrm{supp}\;}
\newcommand{\otherwise}{\mathrm{otherwise}}
\newcommand{\st}{\;\mathrm{s.t.}\;}
\newcommand{\lmd}{\lambda}
\newcommand{\Lmd}{\Lambda}
%%% 線型代数学
\newcommand{\Ker}{\mathrm{Ker}\;}
\newcommand{\Coker}{\mathrm{Coker}\;}
\newcommand{\Coim}{\mathrm{Coim}\;}
\newcommand{\rank}{\mathrm{rank}}
\newcommand{\lcm}{\mathrm{lcm}}
\newcommand{\sgn}{\mathrm{sgn}}
\newcommand{\GL}{\mathrm{GL}}
\newcommand{\SL}{\mathrm{SL}}
\newcommand{\alt}{\mathrm{alt}}
%%% 複素解析学
\renewcommand{\Re}{\mathrm{Re}\;}
\renewcommand{\Im}{\mathrm{Im}\;}
\newcommand{\Gal}{\mathrm{Gal}}
\newcommand{\PGL}{\mathrm{PGL}}
\newcommand{\PSL}{\mathrm{PSL}}
\newcommand{\Log}{\mathrm{Log}\,}
\newcommand{\Res}{\mathrm{Res}\,}
\newcommand{\on}{\mathrm{on}\;}
\newcommand{\hatC}{\hat{\C}}
\newcommand{\hatR}{\hat{\R}}
\newcommand{\PV}{\mathrm{P.V.}}
\newcommand{\diam}{\mathrm{diam}}
\newcommand{\Area}{\mathrm{Area}}
\newcommand{\Lap}{\Laplace}
\newcommand{\f}{\mathbf{f}}
\newcommand{\cR}{\mathcal{R}}
\newcommand{\const}{\mathrm{const.}}
\newcommand{\Om}{\Omega}
\newcommand{\Cinf}{C^\infty}
\newcommand{\ep}{\epsilon}
\newcommand{\dist}{\mathrm{dist}}
\newcommand{\opart}{\o{\partial}}
%%% 解析力学
\newcommand{\x}{\mathbf{x}}
%%% 集合と位相
\renewcommand{\O}{\mathcal{O}}
\renewcommand{\S}{\mathcal{S}}
\renewcommand{\U}{\mathcal{U}}
\newcommand{\V}{\mathcal{V}}
\renewcommand{\P}{\mathcal{P}}
\newcommand{\R}{\mathbb{R}}
\newcommand{\N}{\mathbb{N}}
\newcommand{\C}{\mathbb{C}}
\newcommand{\Z}{\mathbb{Z}}
\newcommand{\Q}{\mathbb{Q}}
\newcommand{\TV}{\mathrm{TV}}
\newcommand{\ORD}{\mathrm{ORD}}
\newcommand{\Tr}{\mathrm{Tr}\;}
\newcommand{\Card}{\mathrm{Card}\;}
\newcommand{\Top}{\mathrm{Top}}
\newcommand{\Disc}{\mathrm{Disc}}
\newcommand{\Codisc}{\mathrm{Codisc}}
\newcommand{\CoDisc}{\mathrm{CoDisc}}
\newcommand{\Ult}{\mathrm{Ult}}
\newcommand{\ord}{\mathrm{ord}}
\newcommand{\maj}{\mathrm{maj}}
%%% 形式言語理論
\newcommand{\REGEX}{\mathrm{REGEX}}
\newcommand{\RE}{\mathbf{RE}}

%%% Fourier解析
\newcommand*{\Laplace}{\mathop{}\!\mathbin\bigtriangleup}
\newcommand*{\DAlambert}{\mathop{}\!\mathbin\Box}
%%% Graph Theory
\newcommand{\SimpGph}{\mathrm{SimpGph}}
\newcommand{\Gph}{\mathrm{Gph}}
\newcommand{\mult}{\mathrm{mult}}
\newcommand{\inv}{\mathrm{inv}}
%%% 多様体
\newcommand{\Der}{\mathrm{Der}}
\newcommand{\osub}{\overset{\mathrm{open}}{\subset}}
\newcommand{\osup}{\overset{\mathrm{open}}{\supset}}
\newcommand{\al}{\alpha}
\newcommand{\K}{\mathbb{K}}
\newcommand{\Sp}{\mathrm{Sp}}
\newcommand{\g}{\mathfrak{g}}
\newcommand{\h}{\mathfrak{h}}
\newcommand{\Exp}{\mathrm{Exp}\;}
\newcommand{\Imm}{\mathrm{Imm}}
\newcommand{\Imb}{\mathrm{Imb}}
\newcommand{\codim}{\mathrm{codim}\;}
\newcommand{\Gr}{\mathrm{Gr}}
%%% 代数
\newcommand{\Ad}{\mathrm{Ad}}
\newcommand{\finsupp}{\mathrm{fin\;supp}}
\newcommand{\SO}{\mathrm{SO}}
\newcommand{\SU}{\mathrm{SU}}
\newcommand{\acts}{\curvearrowright}
\newcommand{\mono}{\hookrightarrow}
\newcommand{\epi}{\twoheadrightarrow}
\newcommand{\Stab}{\mathrm{Stab}}
\newcommand{\nor}{\mathrm{nor}}
\newcommand{\T}{\mathbb{T}}
\newcommand{\Aff}{\mathrm{Aff}}
\newcommand{\rsub}{\triangleleft}
\newcommand{\rsup}{\triangleright}
\newcommand{\subgrp}{\overset{\mathrm{subgrp}}{\subset}}
\newcommand{\Ext}{\mathrm{Ext}}
\newcommand{\sbs}{\subset}\newcommand{\sps}{\supset}
\newcommand{\In}{\mathrm{In}}
\newcommand{\Tor}{\mathrm{Tor}}
\newcommand{\p}{\mathfrak{p}}
\newcommand{\q}{\mathfrak{q}}
\newcommand{\m}{\mathfrak{m}}
\newcommand{\cS}{\mathcal{S}}
\newcommand{\Frac}{\mathrm{Frac}\,}
\newcommand{\Spec}{\mathrm{Spec}\,}
\newcommand{\bA}{\mathbb{A}}
\newcommand{\Sym}{\mathrm{Sym}}
\newcommand{\Ann}{\mathrm{Ann}}
%%% 代数的位相幾何学
\newcommand{\Ho}{\mathrm{Ho}}
\newcommand{\CW}{\mathrm{CW}}
\newcommand{\lc}{\mathrm{lc}}
\newcommand{\cg}{\mathrm{cg}}
\newcommand{\Fib}{\mathrm{Fib}}
\newcommand{\Cyl}{\mathrm{Cyl}}
\newcommand{\Ch}{\mathrm{Ch}}
%%% 数値解析
\newcommand{\round}{\mathrm{round}}
\newcommand{\cond}{\mathrm{cond}}
\newcommand{\diag}{\mathrm{diag}}
%%% 確率論
\newcommand{\calF}{\mathcal{F}}
\newcommand{\X}{\mathcal{X}}
\newcommand{\Meas}{\mathrm{Meas}}
\newcommand{\as}{\;\mathrm{a.s.}} %almost surely
\newcommand{\io}{\;\mathrm{i.o.}} %infinitely often
\newcommand{\fe}{\;\mathrm{f.e.}} %with a finite number of exceptions
\newcommand{\F}{\mathcal{F}}
\newcommand{\bF}{\mathbb{F}}
\newcommand{\W}{\mathcal{W}}
\newcommand{\Pois}{\mathrm{Pois}}
\newcommand{\iid}{\mathrm{i.i.d.}}
\newcommand{\wconv}{\rightsquigarrow}
\newcommand{\Var}{\mathrm{Var}}
\newcommand{\xrightarrown}{\xrightarrow{n\to\infty}}
\newcommand{\au}{\mathrm{au}}
\newcommand{\cT}{\mathcal{T}}
%%% 情報理論
\newcommand{\bit}{\mathrm{bit}}
%%% 積分論
\newcommand{\calA}{\mathcal{A}}
\newcommand{\calB}{\mathcal{B}}
\newcommand{\D}{\mathcal{D}}
\newcommand{\Y}{\mathcal{Y}}
\newcommand{\calC}{\mathcal{C}}
\renewcommand{\ae}{\mathrm{a.e.}\;}
\newcommand{\cZ}{\mathcal{Z}}
\newcommand{\fF}{\mathfrak{F}}
\newcommand{\fI}{\mathfrak{I}}
\newcommand{\E}{\mathcal{E}}
\newcommand{\sMap}{\sigma\textrm{-}\mathrm{Map}}
\DeclareMathOperator*{\argmax}{arg\,max}
\DeclareMathOperator*{\argmin}{arg\,min}
\newcommand{\cC}{\mathcal{C}}
\newcommand{\comp}{\complement}
\newcommand{\J}{\mathcal{J}}
\newcommand{\sumN}[1]{\sum_{#1\in\N}}
\newcommand{\cupN}[1]{\cup_{#1\in\N}}
\newcommand{\capN}[1]{\cap_{#1\in\N}}
\newcommand{\Sum}[1]{\sum_{#1=1}^\infty}
\newcommand{\sumn}{\sum_{n=1}^\infty}
\newcommand{\summ}{\sum_{m=1}^\infty}
\newcommand{\sumk}{\sum_{k=1}^\infty}
\newcommand{\sumi}{\sum_{i=1}^\infty}
\newcommand{\sumj}{\sum_{j=1}^\infty}
\newcommand{\cupn}{\cup_{n=1}^\infty}
\newcommand{\capn}{\cap_{n=1}^\infty}
\newcommand{\cupk}{\cup_{k=1}^\infty}
\newcommand{\cupi}{\cup_{i=1}^\infty}
\newcommand{\cupj}{\cup_{j=1}^\infty}
\newcommand{\limn}{\lim_{n\to\infty}}
\renewcommand{\l}{\mathcal{l}}
\renewcommand{\L}{\mathcal{L}}
\newcommand{\Cl}{\mathrm{Cl}}
\newcommand{\cN}{\mathcal{N}}
\newcommand{\Ae}{\textrm{-a.e.}\;}
\newcommand{\csub}{\overset{\textrm{closed}}{\subset}}
\newcommand{\csup}{\overset{\textrm{closed}}{\supset}}
\newcommand{\wB}{\wt{B}}
\newcommand{\cG}{\mathcal{G}}
\newcommand{\Lip}{\mathrm{Lip}}
\newcommand{\Dom}{\mathrm{Dom}}
%%% 数理ファイナンス
\newcommand{\pre}{\mathrm{pre}}
\newcommand{\om}{\omega}

%%% 統計的因果推論
\newcommand{\Do}{\mathrm{Do}}
%%% 数理統計
\newcommand{\bP}{\mathbb{P}}
\newcommand{\compsub}{\overset{\textrm{cpt}}{\subset}}
\newcommand{\lip}{\textrm{lip}}
\newcommand{\BL}{\mathrm{BL}}
\newcommand{\G}{\mathbb{G}}
\newcommand{\NB}{\mathrm{NB}}
\newcommand{\oR}{\o{\R}}
\newcommand{\liminfn}{\liminf_{n\to\infty}}
\newcommand{\limsupn}{\limsup_{n\to\infty}}
%\newcommand{\limn}{\lim_{n\to\infty}}
\newcommand{\esssup}{\mathrm{ess.sup}}
\newcommand{\asto}{\xrightarrow{\as}}
\newcommand{\Cov}{\mathrm{Cov}}
\newcommand{\cQ}{\mathcal{Q}}
\newcommand{\VC}{\mathrm{VC}}
\newcommand{\mb}{\mathrm{mb}}
\newcommand{\Avar}{\mathrm{Avar}}
\newcommand{\bB}{\mathbb{B}}
\newcommand{\bW}{\mathbb{W}}
\newcommand{\sd}{\mathrm{sd}}
\newcommand{\w}[1]{\widehat{#1}}
\newcommand{\bZ}{\mathbb{Z}}
\newcommand{\Bernoulli}{\mathrm{Bernoulli}}
\newcommand{\Mult}{\mathrm{Mult}}
\newcommand{\BPois}{\mathrm{BPois}}
\newcommand{\fraks}{\mathfrak{s}}
\newcommand{\frakk}{\mathfrak{k}}
\newcommand{\IF}{\mathrm{IF}}
\newcommand{\bX}{\mathbf{X}}
\newcommand{\bx}{\mathbf{x}}
\newcommand{\indep}{\raisebox{0.05em}{\rotatebox[origin=c]{90}{$\models$}}}
\newcommand{\IG}{\mathrm{IG}}
\newcommand{\Levy}{\mathrm{Levy}}
\newcommand{\MP}{\mathrm{MP}}
\newcommand{\Hermite}{\mathrm{Hermite}}
\newcommand{\Skellam}{\mathrm{Skellam}}
\newcommand{\Dirichlet}{\mathrm{Dirichlet}}
\newcommand{\Beta}{\mathrm{Beta}}
\newcommand{\bE}{\mathbb{E}}
\newcommand{\bG}{\mathbb{G}}
\newcommand{\MISE}{\mathrm{MISE}}
\newcommand{\logit}{\mathtt{logit}}
\newcommand{\expit}{\mathtt{expit}}
\newcommand{\cK}{\mathcal{K}}
\newcommand{\dl}{\dot{l}}
\newcommand{\dotp}{\dot{p}}
\newcommand{\wl}{\wt{l}}
%%% 函数解析
\renewcommand{\c}{\mathbf{c}}
\newcommand{\loc}{\mathrm{loc}}
\newcommand{\Lh}{\mathrm{L.h.}}
\newcommand{\Epi}{\mathrm{Epi}\;}
\newcommand{\slim}{\mathrm{slim}}
\newcommand{\Ban}{\mathrm{Ban}}
\newcommand{\Hilb}{\mathrm{Hilb}}
\newcommand{\Ex}{\mathrm{Ex}}
\newcommand{\Co}{\mathrm{Co}}
\newcommand{\sa}{\mathrm{sa}}
\newcommand{\nnorm}[1]{{\left\vert\kern-0.25ex\left\vert\kern-0.25ex\left\vert #1 \right\vert\kern-0.25ex\right\vert\kern-0.25ex\right\vert}}
\newcommand{\dvol}{\mathrm{dvol}}
\newcommand{\Sconv}{\mathrm{Sconv}}
\newcommand{\I}{\mathcal{I}}
\newcommand{\nonunital}{\mathrm{nu}}
\newcommand{\cpt}{\mathrm{cpt}}
\newcommand{\lcpt}{\mathrm{lcpt}}
\newcommand{\com}{\mathrm{com}}
\newcommand{\Haus}{\mathrm{Haus}}
\newcommand{\proper}{\mathrm{proper}}
\newcommand{\infinity}{\mathrm{inf}}
\newcommand{\TVS}{\mathrm{TVS}}
\newcommand{\ess}{\mathrm{ess}}
\newcommand{\ext}{\mathrm{ext}}
\newcommand{\Index}{\mathrm{Index}}
\newcommand{\SSR}{\mathrm{SSR}}
\newcommand{\vs}{\mathrm{vs.}}
\newcommand{\fM}{\mathfrak{M}}
\newcommand{\EDM}{\mathrm{EDM}}
\newcommand{\Tw}{\mathrm{Tw}}
\newcommand{\fC}{\mathfrak{C}}
\newcommand{\bn}{\mathbf{n}}
\newcommand{\br}{\mathbf{r}}
\newcommand{\Lam}{\Lambda}
\newcommand{\lam}{\lambda}
\newcommand{\one}{\mathbf{1}}
\newcommand{\dae}{\text{-a.e.}}
\newcommand{\td}{\text{-}}
\newcommand{\RM}{\mathrm{RM}}
%%% 最適化
\newcommand{\Minimize}{\text{Minimize}}
\newcommand{\subjectto}{\text{subject to}}
\newcommand{\Ri}{\mathrm{Ri}}
%\newcommand{\Cl}{\mathrm{Cl}}
\newcommand{\Cone}{\mathrm{Cone}}
\newcommand{\Int}{\mathrm{Int}}
%%% 圏
\newcommand{\varlim}{\varprojlim}
\newcommand{\Hom}{\mathrm{Hom}}
\newcommand{\Iso}{\mathrm{Iso}}
\newcommand{\Mor}{\mathrm{Mor}}
\newcommand{\Isom}{\mathrm{Isom}}
\newcommand{\Aut}{\mathrm{Aut}}
\newcommand{\End}{\mathrm{End}}
\newcommand{\op}{\mathrm{op}}
\newcommand{\ev}{\mathrm{ev}}
\newcommand{\Ob}{\mathrm{Ob}}
\newcommand{\Ar}{\mathrm{Ar}}
\newcommand{\Arr}{\mathrm{Arr}}
\newcommand{\Set}{\mathrm{Set}}
\newcommand{\Grp}{\mathrm{Grp}}
\newcommand{\Cat}{\mathrm{Cat}}
\newcommand{\Mon}{\mathrm{Mon}}
\newcommand{\CMon}{\mathrm{CMon}} %Comutative Monoid 可換単系とモノイドの射
\newcommand{\Ring}{\mathrm{Ring}}
\newcommand{\CRing}{\mathrm{CRing}}
\newcommand{\Ab}{\mathrm{Ab}}
\newcommand{\Pos}{\mathrm{Pos}}
\newcommand{\Vect}{\mathrm{Vect}}
\newcommand{\FinVect}{\mathrm{FinVect}}
\newcommand{\FinSet}{\mathrm{FinSet}}
\newcommand{\OmegaAlg}{\Omega$-$\mathrm{Alg}}
\newcommand{\OmegaEAlg}{(\Omega,E)$-$\mathrm{Alg}}
\newcommand{\Alg}{\mathrm{Alg}} %代数の圏
\newcommand{\CAlg}{\mathrm{CAlg}} %可換代数の圏
\newcommand{\CPO}{\mathrm{CPO}} %Complete Partial Order & continuous mappings
\newcommand{\Fun}{\mathrm{Fun}}
\newcommand{\Func}{\mathrm{Func}}
\newcommand{\Met}{\mathrm{Met}} %Metric space & Contraction maps
\newcommand{\Pfn}{\mathrm{Pfn}} %Sets & Partial function
\newcommand{\Rel}{\mathrm{Rel}} %Sets & relation
\newcommand{\Bool}{\mathrm{Bool}}
\newcommand{\CABool}{\mathrm{CABool}}
\newcommand{\CompBoolAlg}{\mathrm{CompBoolAlg}}
\newcommand{\BoolAlg}{\mathrm{BoolAlg}}
\newcommand{\BoolRng}{\mathrm{BoolRng}}
\newcommand{\HeytAlg}{\mathrm{HeytAlg}}
\newcommand{\CompHeytAlg}{\mathrm{CompHeytAlg}}
\newcommand{\Lat}{\mathrm{Lat}}
\newcommand{\CompLat}{\mathrm{CompLat}}
\newcommand{\SemiLat}{\mathrm{SemiLat}}
\newcommand{\Stone}{\mathrm{Stone}}
\newcommand{\Sob}{\mathrm{Sob}} %Sober space & continuous map
\newcommand{\Op}{\mathrm{Op}} %Category of open subsets
\newcommand{\Sh}{\mathrm{Sh}} %Category of sheave
\newcommand{\PSh}{\mathrm{PSh}} %Category of presheave, PSh(C)=[C^op,set]のこと
\newcommand{\Conv}{\mathrm{Conv}} %Convergence spaceの圏
\newcommand{\Unif}{\mathrm{Unif}} %一様空間と一様連続写像の圏
\newcommand{\Frm}{\mathrm{Frm}} %フレームとフレームの射
\newcommand{\Locale}{\mathrm{Locale}} %その反対圏
\newcommand{\Diff}{\mathrm{Diff}} %滑らかな多様体の圏
\newcommand{\Mfd}{\mathrm{Mfd}}
\newcommand{\LieAlg}{\mathrm{LieAlg}}
\newcommand{\Quiv}{\mathrm{Quiv}} %Quiverの圏
\newcommand{\B}{\mathcal{B}}
\newcommand{\Span}{\mathrm{Span}}
\newcommand{\Corr}{\mathrm{Corr}}
\newcommand{\Decat}{\mathrm{Decat}}
\newcommand{\Rep}{\mathrm{Rep}}
\newcommand{\Grpd}{\mathrm{Grpd}}
\newcommand{\sSet}{\mathrm{sSet}}
\newcommand{\Mod}{\mathrm{Mod}}
\newcommand{\SmoothMnf}{\mathrm{SmoothMnf}}
\newcommand{\coker}{\mathrm{coker}}

\newcommand{\Ord}{\mathrm{Ord}}
\newcommand{\eq}{\mathrm{eq}}
\newcommand{\coeq}{\mathrm{coeq}}
\newcommand{\act}{\mathrm{act}}

%%%%%%%%%%%%%%% 定理環境(足助先生ありがとうございます) %%%%%%%%%%%%%%%

\everymath{\displaystyle}
\renewcommand{\proofname}{\bf [証明]}
\renewcommand{\thefootnote}{\dag\arabic{footnote}} %足助さんからもらった.どうなるんだ?
\renewcommand{\qedsymbol}{$\blacksquare$}

\renewcommand{\labelenumi}{(\arabic{enumi})} %(1),(2),...がデフォルトであって欲しい
\renewcommand{\labelenumii}{(\alph{enumii})}
\renewcommand{\labelenumiii}{(\roman{enumiii})}

\newtheoremstyle{StatementsWithStar}% ?name?
{3pt}% ?Space above? 1
{3pt}% ?Space below? 1
{}% ?Body font?
{}% ?Indent amount? 2
{\bfseries}% ?Theorem head font?
{\textbf{.}}% ?Punctuation after theorem head?
{.5em}% ?Space after theorem head? 3
{\textbf{\textup{#1~\thetheorem{}}}{}\,$^{\ast}$\thmnote{(#3)}}% ?Theorem head spec (can be left empty, meaning ‘normal’)?
%
\newtheoremstyle{StatementsWithStar2}% ?name?
{3pt}% ?Space above? 1
{3pt}% ?Space below? 1
{}% ?Body font?
{}% ?Indent amount? 2
{\bfseries}% ?Theorem head font?
{\textbf{.}}% ?Punctuation after theorem head?
{.5em}% ?Space after theorem head? 3
{\textbf{\textup{#1~\thetheorem{}}}{}\,$^{\ast\ast}$\thmnote{(#3)}}% ?Theorem head spec (can be left empty, meaning ‘normal’)?
%
\newtheoremstyle{StatementsWithStar3}% ?name?
{3pt}% ?Space above? 1
{3pt}% ?Space below? 1
{}% ?Body font?
{}% ?Indent amount? 2
{\bfseries}% ?Theorem head font?
{\textbf{.}}% ?Punctuation after theorem head?
{.5em}% ?Space after theorem head? 3
{\textbf{\textup{#1~\thetheorem{}}}{}\,$^{\ast\ast\ast}$\thmnote{(#3)}}% ?Theorem head spec (can be left empty, meaning ‘normal’)?
%
\newtheoremstyle{StatementsWithCCirc}% ?name?
{6pt}% ?Space above? 1
{6pt}% ?Space below? 1
{}% ?Body font?
{}% ?Indent amount? 2
{\bfseries}% ?Theorem head font?
{\textbf{.}}% ?Punctuation after theorem head?
{.5em}% ?Space after theorem head? 3
{\textbf{\textup{#1~\thetheorem{}}}{}\,$^{\circledcirc}$\thmnote{(#3)}}% ?Theorem head spec (can be left empty, meaning ‘normal’)?
%
\theoremstyle{definition}
 \newtheorem{theorem}{定理}[section]
 \newtheorem{axiom}[theorem]{公理}
 \newtheorem{corollary}[theorem]{系}
 \newtheorem{proposition}[theorem]{命題}
 \newtheorem*{proposition*}{命題}
 \newtheorem{lemma}[theorem]{補題}
 \newtheorem*{lemma*}{補題}
 \newtheorem*{theorem*}{定理}
 \newtheorem{definition}[theorem]{定義}
 \newtheorem{example}[theorem]{例}
 \newtheorem{notation}[theorem]{記法}
 \newtheorem*{notation*}{記法}
 \newtheorem{assumption}[theorem]{仮定}
 \newtheorem{question}[theorem]{問}
 \newtheorem{counterexample}[theorem]{反例}
 \newtheorem{reidai}[theorem]{例題}
 \newtheorem{ruidai}[theorem]{類題}
 \newtheorem{problem}[theorem]{問題}
 \newtheorem{algorithm}[theorem]{算譜}
 \newtheorem*{solution*}{\bf{[解]}}
 \newtheorem{discussion}[theorem]{議論}
 \newtheorem{remark}[theorem]{注}
 \newtheorem{remarks}[theorem]{要諦}
 \newtheorem{image}[theorem]{描像}
 \newtheorem{observation}[theorem]{観察}
 \newtheorem{universality}[theorem]{普遍性} %非自明な例外がない.
 \newtheorem{universal tendency}[theorem]{普遍傾向} %例外が有意に少ない.
 \newtheorem{hypothesis}[theorem]{仮説} %実験で説明されていない理論.
 \newtheorem{theory}[theorem]{理論} %実験事実とその(さしあたり)整合的な説明.
 \newtheorem{fact}[theorem]{実験事実}
 \newtheorem{model}[theorem]{模型}
 \newtheorem{explanation}[theorem]{説明} %理論による実験事実の説明
 \newtheorem{anomaly}[theorem]{理論の限界}
 \newtheorem{application}[theorem]{応用例}
 \newtheorem{method}[theorem]{手法} %実験手法など,技術的問題.
 \newtheorem{history}[theorem]{歴史}
 \newtheorem{usage}[theorem]{用語法}
 \newtheorem{research}[theorem]{研究}
 \newtheorem{shishin}[theorem]{指針}
 \newtheorem{yodan}[theorem]{余談}
 \newtheorem{construction}[theorem]{構成}
% \newtheorem*{remarknonum}{注}
 \newtheorem*{definition*}{定義}
 \newtheorem*{remark*}{注}
 \newtheorem*{question*}{問}
 \newtheorem*{problem*}{問題}
 \newtheorem*{axiom*}{公理}
 \newtheorem*{example*}{例}
 \newtheorem*{corollary*}{系}
 \newtheorem*{shishin*}{指針}
 \newtheorem*{yodan*}{余談}
 \newtheorem*{kadai*}{課題}
%
\theoremstyle{StatementsWithStar}
 \newtheorem{definition_*}[theorem]{定義}
 \newtheorem{question_*}[theorem]{問}
 \newtheorem{example_*}[theorem]{例}
 \newtheorem{theorem_*}[theorem]{定理}
 \newtheorem{remark_*}[theorem]{注}
%
\theoremstyle{StatementsWithStar2}
 \newtheorem{definition_**}[theorem]{定義}
 \newtheorem{theorem_**}[theorem]{定理}
 \newtheorem{question_**}[theorem]{問}
 \newtheorem{remark_**}[theorem]{注}
%
\theoremstyle{StatementsWithStar3}
 \newtheorem{remark_***}[theorem]{注}
 \newtheorem{question_***}[theorem]{問}
%
\theoremstyle{StatementsWithCCirc}
 \newtheorem{definition_O}[theorem]{定義}
 \newtheorem{question_O}[theorem]{問}
 \newtheorem{example_O}[theorem]{例}
 \newtheorem{remark_O}[theorem]{注}
%
\theoremstyle{definition}
%
\raggedbottom
\allowdisplaybreaks
\usepackage[math]{anttor}
\begin{document}
\tableofcontents

\chapter{Brown運動}



\section{記法}

\begin{notation}
    正整数$k,n\ge1$について,
    \begin{enumerate}
        \item $C_b^k(\R^n)$で,$C^k$-級で,任意の$k$階以下の偏導関数は有界である関数の空間とする.
        \item $C_0^k(\R^n):=C_c(\R^n)\cap C_b^k(\R^n)$を,コンパクト台を持つもののなす部分空間とする.
        \item $C_p^\infty(\R^n)$は滑らかな関数であって,その関数自身とその任意の偏導関数は高々多項式の速度で増加する関数の空間とする.\footnote{緩増加関数.\url{https://ncatlab.org/nlab/show/tempered+distribution}}
        \item $C_b^\infty(\R^n):=C_p^\infty(\R^n)\cap\bigcap_{k\in\N}C_b^k(\R^n)$を任意の偏導関数が有界なもののなす部分空間とする.
        \item $C_0^\infty(\R^n):=C^\infty(\R^n)\cap C_c(\R^n)$をコンパクト台を持つ滑らかな関数の空間とする.
        \item $\L(\Om,\F)$上で$(\Om,\F)$上の可測関数の空間,$\L_\cG(\Om,\F)$で部分$\sigma$-代数$\cG$について$\cG$-可測関数のなす部分空間,$L_b(\Om,\F)$で有界$\F$-可測関数のなす部分空間を表すとする.
    \end{enumerate}
    また,$(\Om,\F,P)$を確率空間とし,$L^p(\Om)$によってその上の$L^p$空間を表す.
\end{notation}

\section{Gauss分布}

\begin{tcolorbox}[colframe=ForestGreen, colback=ForestGreen!10!white,breakable,colbacktitle=ForestGreen!40!white,coltitle=black,fonttitle=\bfseries\sffamily,
title=]
    これから(可分な)無限次元Hilbert空間上のGauss測度に関する積分論を展開することになるから,
    Gauss測度も無限次元に通用する言葉で定義する.
    $\R^n$上の正規分布の一般化であると同時に,確率過程の平均と共分散との一般化でもある.
    \cite{Prato}による.
    実は,これで一般のBanach空間上のGauss測度を定義したこととなる\ref{thm-Structure-theorem-for-Gaussian-measures}.
\end{tcolorbox}

\begin{notation}
    $H$を可分Hilbert空間,$(H,\B(H))$をBorel可測空間,$P(H)$をその上のBorel確率測度全体の集合とする.
\end{notation}

\subsection{平均と分散}

\begin{tcolorbox}[colframe=ForestGreen, colback=ForestGreen!10!white,breakable,colbacktitle=ForestGreen!40!white,coltitle=black,fonttitle=\bfseries\sffamily,
title=]
    今まで確率空間$(\Om,\F)$に言及して平均と分散を定義することはなかったが,
    可分Hilbert空間$H$を確率空間として,その分布の扱い方を定める.
    $H$上の有限次元分布は有界線型汎関数を通じて定まるが,それはすべて内積で表現できることから,内積を用いた定義になるが,これは成分$X_i$の拡張に過ぎない.
\end{tcolorbox}

\begin{definition}[mean, covariance operator]
    $\mu\in P(H)$について,
    \begin{enumerate}
        \item $E[\abs{x}]<\infty$のとき,任意の$x\in H$に対して,その内積の平均$E[\brac{x|Y}]\;(Y\sim\mu)$を対応させる
        線型形式$F:H\to\R;x\mapsto\int_H\brac{x|y}\mu(dy)$は有界である(Cauchy-Schwartzによる).
        
        このRiesz表現$\exists!_{m\in H}\;F(x)=\brac{x|m}$を定める元$m\in H$を$\mu$の\textbf{平均}という.
        \item $E[\abs{x}]\le E[\abs{x}^2]<\infty$のとき,双線型形式$G:H\times H\to\R;(x,y)\mapsto\int_H\brac{x|z-m}\brac{y|z-m}\mu(dz)$は有界かつ対称(自己共役)である.
        
        このRiesz表現を定める自己有界作用素$\exists!_{Q\in B(H)}\;\brac{Qx|y}=\brac{x|Qy}=G(x,y)$を\textbf{共分散}という.
    \end{enumerate}
\end{definition}
\begin{remarks}\mbox{}
    \begin{enumerate}
        \item 平均は,$E[\brac{x|Y}]=\brac{x|m}$という可換性を満たす元$m$,
        \item 共分散には2つの見方があり,
        対称な(半)双線型形式$G(x,y)=E[\brac{x|Z-m}\brac{y|Z-m}]=E[(\brac{x|Z}-E[\brac{x|Z}])(\brac{y|Z}-E[\brac{y|Z}])]$とも($\brac{x|Z},\brac{y|Z}$を改めて$X,Y:H\to\R$と定めれば見慣れた共分散の定義$\Cov[X,Y]$となる),
        それを表現する有界線型作用素$Q\in B(H)$とも理解できる.
        \item 内積はベクトル$(X_1,\cdots,X_n)$からの成分の抽出の一般化と見ると,$\brac{x|Z}$とは成分$Z_x$と思える.つまり,
        $x,y\in H$は添字集合と見る.
    \end{enumerate}
\end{remarks}

\begin{lemma}[共分散作用素の性質]
    $E[\abs{x}^2]<\infty$とする.
    \begin{enumerate}
        \item $Q$は正作用素であり,かつ自己共役である.
        \item $H$の任意の正規直交基底$(e_k)$について,$\Tr Q:=\sum_{k\in\N}\brac{Qe_k,e_k}=\int_H\abs{x-m}^2\mu(dx)<\infty$.
        \item $Q$はコンパクト作用素である.(2)と併せると特に,$Q$は跡類作用素である:$Q\in B^1(H)$.
        \item 分散共分散公式:$E[\abs{x}^2]=\Tr Q+\abs{m}^2$.
    \end{enumerate}
\end{lemma}
\begin{proof}\mbox{}
    \begin{enumerate}
        \item $Q$を定める双線型形式$G$は対称であったから,$Q$も対称である:$G(x,y)=G(y,x)\Rightarrow\brac{Qx|y}=\brac{Q^*x|y}$.
        また,$\forall_{x,x\ge0}\;\brac{Qx|x}=G(x,x)\ge0$より正である.
        \item 単調収束定理とParseval恒等式より,
        \begin{align*}
            \Tr Q=\sum_{k\in\N}\brac{Qe_k|e_k}&=\sum_{k\in\N}\int_H\abs{\brac{x-m|e_k}}^2\mu(dx)\\
            &=\int_H\sum_{k\in\N}\abs{\brac{x-m|e_k}}^2\mu(dx)=\int_H\abs{x-m}^2\mu(dx)<\infty.
        \end{align*}
        \item 正作用素$T$が$\exists_{p>0}\;\Tr(\abs{T}^p)<\infty$を満たすならば,コンパクト作用素である.
    \end{enumerate}
\end{proof}

\begin{notation}
    正作用素の空間を$B^+(H)$とし,有限な跡を持つ作用素の空間を$B^+_1(H)\subset B^+(H)$で表す.
\end{notation}

\subsection{Gauss測度の定義と構成}

\begin{tcolorbox}[colframe=ForestGreen, colback=ForestGreen!10!white,breakable,colbacktitle=ForestGreen!40!white,coltitle=black,fonttitle=\bfseries\sffamily,
title=]
    $H$上のGauss分布を特性関数の言葉を用いて定義する.
    このように定義することで,Wiener測度などの無限次元空間上のGauss測度を扱える.
    そこでの積分が確率解析の主な戦場である.
\end{tcolorbox}

\begin{definition}[Gaussian measure]
    $\mu\in P(H)$が\textbf{Gauss測度$N(m,Q)$}であるとは,ある$m\in H,Q\in L^+_1(H)$を用いて,
    \[\wh{\mu}(h)=e^{i(m|h)}e^{-\frac{1}{2}(Qh|h)}\]
    と表せるものをいう.$Q$が単射$\Ker Q=0$であるとき,$\mu$は非退化であるという.

    なお,自己共役作用素$Q=Q^*$が単射であることは,$\Im Q$がノルム閉ならば$Q=Q^*$が可逆であることに同値であるが,実は$\Im Q$がノルム閉であることは必ずしも成り立たない.
    $\mu=N_{m,Q}$と表す.
\end{definition}

\subsection{Gauss測度の構成}

\begin{tcolorbox}[colframe=ForestGreen, colback=ForestGreen!10!white,breakable,colbacktitle=ForestGreen!40!white,coltitle=black,fonttitle=\bfseries\sffamily,
title=]
    Banach空間上のGauss測度には特徴があり,ペアリングによって位相が移り合うように,測度も移り合う.
    双対空間上のGauss測度をノイズと呼ぶ.
    なお,ペアリングによって微分演算を移すのが超関数微分,積分演算を移すのがPettis積分であった.

    こうして,正規分布に従う独立同分布列(族)の存在定理が,ホワイトノイズ関数という一般的な形で得られる.
    これが$H$を可分にしていた理由である.
    するとこのホワイトノイズを標準的な対象として,任意の測度空間$(A,\A,\mu)$にGauss測度を$(L^2(A))^*$の元という形で構成することも出来る.
\end{tcolorbox}

または次のように定義することも出来る\cite{Revus and Yor}

\begin{proposition}[一般化された独立同分布列の存在定理:ホワイトノイズ]\label{prop-existence-of-noises}
    $H$を可分実Hilbert空間とする.
    ある確率空間$(\Om,\F,P)$とその上の確率変数の族$X=(X_h)_{h\in H}$が存在して,次の2条件を満たす:
    \begin{enumerate}
        \item 写像$X:H\to\Meas(\Om,\R);h\mapsto X_h$は線型である.
        \item 任意の$h\in H$について,確率変数$X_h$は中心化されたGauss変数で,線型写像$X:H\to L^2(\Om)$は等長である:$E[X_h^2]=\norm{h}^2_H$.
        \item 埋め込み$\{X_h\}_{h\in H}\mono L^2(\Om)$は$L^2(\Om)$のGauss部分空間である.
    \end{enumerate}
\end{proposition}
\begin{remark}
    この同一視により,独立性$X_h\indep X_{h'}$は添字空間$H$における直交性として理解できる.
\end{remark}

\begin{definition}[Gaussian measure / white noise]
    $(A,\A,\mu)$を可分な$\sigma$-有限測度空間とする.$H:=L^2(A,\A,\mu)$として,中心化されたGauss変数の族$X=(X_h)_{h\in H}$を取る.
    この写像$X:L^2(A,\mu)\to\L(\Om)$を$(A,\A)$上の\textbf{強度$\mu$のGauss測度/白色雑音}という.
    測度確定な可測集合$F\in\A,\mu(F)<\infty$のGauss測度$X(F)$は$X(1_F)$とも表す.
\end{definition}



\section{Gauss確率変数}

\subsection{Hilbert空間上のGauss変数}

\begin{tcolorbox}[colframe=ForestGreen, colback=ForestGreen!10!white,breakable,colbacktitle=ForestGreen!40!white,coltitle=black,fonttitle=\bfseries\sffamily,
title=]
    特にGauss測度の押し出しによって構成されるGauss変数を考える.
    一般の有界線型汎関数,特に回転変換は
    Gauss測度を保つ.
\end{tcolorbox}

\begin{proposition}[有界線型汎関数はGauss測度を保つ]
    $(H,\mu)$を可分な正規Hilbert空間とする:$\mu\sim N(a,Q)$.
    このとき,任意の可分Hilbert空間$K$へのaffine変換$T:H\to K;x\mapsto Ax+b\;(A\in B(H,K),b\in K)$とすると,$T_*\mu\sim N_{Aa+b,AQA^*}$.
\end{proposition}
\begin{proof}
    特性関数を計算すると,任意の$k\in K$について,
    \begin{align*}
        \wh{T_*\mu}(k)&=\int_K e^{i(k|y)}(T_*\mu)(dy)\\
        &=\int_He^{i(k|T(x))}\mu(dx)\\
        &=\int_He^{i(k|Ax+b)}\mu(dx)\\
        &=e^{i(k|b)}\int_He^{i(A^*k|x)}\mu(dx)=e^{i(k|Aa+b)}e^{-\frac{1}{2}(AQA^*k,k)}.
    \end{align*}
\end{proof}

\begin{corollary}
    $(H,\mu)$を可分な正規Hilbert空間とする:$\mu\sim N(0,Q)$.
    任意の点$z_1,\cdots,z_n\in H$について,$A:H\to\R^n$を$Ax:=((x|z_1),\cdots,(x|z_n))\;(x\in H)$と定めると,
    $Q_A:=(Qz_i,z_j)_{i,j\in[n]}$について$A\sim N(0,Q_A)$.
\end{corollary}

\begin{corollary}[Gauss測度の回転不変性]
    線型写像$R:H\times H\to H\times H$を
    \[R(x,y):=\begin{pmatrix}\cos\theta&-\sin\theta\\\sin\theta&\cos\theta\end{pmatrix}\begin{pmatrix}x\\y\end{pmatrix}\]
    と定めると,$(H\times H,\mu\times\mu)\;(\mu\sim N(0,Q))$上の2つの測度$\mu\times\mu$と$R_*\mu\times\mu$とは等しい:
    \[\forall_{F\in\L_b(H\times H;H\times H)}\;\int_{H\times H}F(x,y)\mu(dx)\mu(dy)=\int_{H\times H}F(R(x,y))\mu(dx)\mu(dy).\]
\end{corollary}

\subsection{Gauss過程の特徴付け}

\begin{tcolorbox}[colframe=ForestGreen, colback=ForestGreen!10!white,breakable,colbacktitle=ForestGreen!40!white,coltitle=black,fonttitle=\bfseries\sffamily,
title=]
    $C(T,\R)$上のGauss変数を特にGauss過程ともいう.
\end{tcolorbox}

\begin{definition}[Gaussian process, centered]
    $(\Om,\F,P)$上の実確率過程$X=(X_t)_{t\in\R_+}$について,
    \begin{enumerate}
        \item \textbf{Gauss過程}であるとは,任意の有限部分集合$\{t_1,\cdots,t_n\}\subset \R_+$について,確率ベクトル$(X_{t_1},\cdots,X_{t_n}):\Om\to\R^n$は$n$次元正規分布に従うことをいう.
        \item Gauss過程$B$の\textbf{共分散}とは,関数$\Gamma:\R_+\times\R_+\to\R;(s,t)\mapsto\Cov[X_s,X_t]=E[(X_s-E[X_s])(X_t-E[X_t])]$をいう.
        \item \textbf{中心化されている}とは,$\forall_{t\in\R_+}\;E[X_t]=0$を満たすことをいう.
    \end{enumerate}
\end{definition}

\begin{definition}[semi-definite positive function]
    関数$\Gamma:T\times T\to\R$が\textbf{半正定値}であるとは,
    $T$の任意の有限部分集合$\{t_1,\cdots,t_d\}\subset T\;(d\in\N)$に対して,行列$(\Gamma(t_i,t_j))_{i,j\in[d]}$は半正定値であることをいう.
\end{definition}

\begin{proposition}[Gauss過程の共分散の特徴付け]\label{prop-existence-of-Gaussian-process}
    $T$を任意の集合とする.関数$\Gamma:T\times T\to\R$と任意の関数$m:T\to\R$について,次の2条件は同値.
    \begin{enumerate}
        \item ある平均$m$のGauss過程が存在して,その共分散である.
        \item 対称な半正定値関数である.
    \end{enumerate}
\end{proposition}
\begin{proof}
    Kolmogorovの拡張定理\ref{thm-Kolmogorov-extension-theorem}による.
    ここでは$a=0$として証明する.
    \begin{description}
        \item[(1)$\Rightarrow$(2)] $T$の対称性は積の可換性より明らか.
        任意の有限部分集合$\{t_i\}_{i\in[n]}\subset T$と任意のベクトル$a=(a_i)_{i\in[n]}\in\C^n$を取る.
        これについて,2次形式$a^*(\Sigma(i,j))a$が非負であることを示せば良い.
        \begin{align*}
            \sum_{i,j\in[n]}&=\Gamma(t_i,t_j)a_i\o{a_j}\\
            &=E\Square{\sum_{i\in[n]}a_i(X_{t_i}-E[X_{t_i}])\sum_{j\in[n]}\o{a_j}(X_{t_j}-E[X_{t_j}])}\\
            &=E\Square{\Abs{\sum_{i\in[n]}a_i(X_{t_i}-E[X_{t_i}])}^2}\ge0.
        \end{align*}
        \item[(2)$\Rightarrow$(1)] 
        確率分布族$(P_{t_1,\cdots,t_n})_{n\in\N_{\ge1},\{t_i\}\subset T}$を,$\R^n$上の平均$0$,分散共分散行列を$\Sigma_{t_1,\cdots,t_n}:=(\Gamma(t_i,t_j))_{i,j\in[n]}$とする正規分布とする.
        $\Gamma$は対称は半正定値関数としたので,$\Sigma_{t_1,\cdots,t_n}$も対称な半正定値行列であり,これに対応する正規分布はたしかに存在する.
        また定義より,一貫性条件を満たす.
        よって,$(P_{t_1,\cdots,t_n})_{n\in\N_{\ge1},\{t_i\}\subset T}$を有限次元周辺分布とする
        確率過程$X=(X_t)_{t\in\R_+}$が存在するが,これは平均$0$で分散$\Gamma$のGauss過程である.
    \end{description}
\end{proof}

\begin{lemma}[共分散公式]
    $X,Y$をGauss確率変数とする.
    $X_t:=tX+Y\;(t\in\R_+)$によって定まる確率過程$X=(X_t)_{t\in\R_+}$はGauss過程であって,
    平均$m_X(t)=tE[X]+E[Y]$と分散
    \[\Gamma_X(s,t)=st\Var[X]+(s+t)\Cov[X,Y]+\Var[Y]\]
    を持つ.
\end{lemma}
\begin{proof}
    平均は期待値の線形性より明らか.
    分散については,次のように計算が進む.
    ただし,見やすくするため$\mu_X:=E[X],\mu_Y:=E[Y]\in\R$とする.
    \begin{align*}
        \Gamma(s,t)&=E[(X_s-E[X_s])(X_t-E[X_t])]\\
        &=E[X_sX_t]-(s\mu_X+\mu_Y)E[X_t]-(t\mu_X+\mu_Y)E[X_s]+(s\mu_X+\mu_Y)(t\mu_X+\mu_Y)\\
        &=stE[X^2]+(s+t)E[XY]+E[Y^2]+(s\mu_X+\mu_Y)(t\mu_X+\mu_Y)=st\Var[X]+(s+t)\Cov[X,Y]+\Var[Y].
    \end{align*}
\end{proof}

\subsection{Gauss変数の独立性}

\begin{tcolorbox}[colframe=ForestGreen, colback=ForestGreen!10!white,breakable,colbacktitle=ForestGreen!40!white,coltitle=black,fonttitle=\bfseries\sffamily,
title=]
    あるGauss確率変数の成分(一般化して,有界線型汎関数との合成)の間の独立性は,直交性で捉えられる.
\end{tcolorbox}

\begin{proposition}[一般の確率変数の独立性の特徴付け]
    $X_1,\cdots,X_n:\Om\to H$を確率変数,$X:=(X_1,\cdots,X_n)$を積写像とする.
    \begin{enumerate}
        \item $X_1,\cdots,X_n$は独立である.
        \item 特性関数が分解する:$\forall_{h=(h_1,\cdots,h_n)\in H^n}\;\wh{X}(h)=\prod_{i=1}^n\wh{X_i}(h_j)$.
        \item 任意の期待値が分解する:$\forall_{\varphi_1,\cdots,\varphi_n\in\L_b(\Om;\R)}\;E[\varphi_1(X_1)\cdots\varphi_n(X_n)]=E[\varphi_1(X_1)]\cdots E[\varphi_n(X_n)]$.
    \end{enumerate}
\end{proposition}

\begin{notation}[正規Hilbert空間上の線型な実確率変数]
    $H$上の実確率変数のうち,特に有界線型汎関数によって定まるもの:$F_v(x)=(x,v)\;(v\in H)$を考える.これはGauss確率変数である:$F_v\sim N_{\brac{Qv,v}}$.
\end{notation}

\begin{proposition}[Gauss変数の独立性の特徴付け]
    $v_1,\cdots,v_n\in H$とする.
    \begin{enumerate}
        \item 線型確率変数$F_{v_1},\cdots,F_{v_n}$が独立である.
        \item $(Q_{v_1,\cdots,v_n})$は対角行列である.
    \end{enumerate}
    (2)は,各$F_{v_i},F_{v_j}\;(i\ne j)$が$L^2(H)$上で直交することに同値.
\end{proposition}

\subsection{Gauss空間}

\begin{tcolorbox}[colframe=ForestGreen, colback=ForestGreen!10!white,breakable,colbacktitle=ForestGreen!40!white,coltitle=black,fonttitle=\bfseries\sffamily,
    title=]
    $X$が$d$次元Gauss変数であるとは,任意の線型汎関数$\al:\R^d\to\R$について$\al(X)$がGauss確率変数であることと同値.
    この他にも,Gauss変数の独立性はGauss空間で幾何学的に捉えられる.
\end{tcolorbox}

\begin{notation}
    $X\sim N(0,0)$として,定数関数はGauss確率変数とする.
\end{notation}

\begin{proposition}[Gauss確率変数全体の空間は閉部分空間をなす]
    $(X_n)$を$X\in\Meas(\Om,\R)$に確率収束するGauss確率変数の列とする.このとき,$X$もGaussで,族$\{\abs{X_n}^p\}$は一様可積分で,$X_n\to X$は任意の$p\ge1$について$L^p$-収束もする.
    また,$X$の平均と共分散はそれぞれの各点収束極限である.
\end{proposition}

\begin{definition}[Gaussian subspace]
    Hilbert空間$L^2(\Om,\F,P)$の閉部分空間であって,中心化されたGauss確率変数のみからなるものを\textbf{Gauss空間}という.
\end{definition}

\begin{proposition}[独立性の特徴付け]
    $(G_i)_{i\in I}$をあるGauss空間の閉部分空間の族とする.次の2条件は同値:
    \begin{enumerate}
        \item $\sigma$-代数$\sigma(G_i)$の族は独立である.
        \item 各$G_i$は組ごとに直交する:$\forall_{i,j\in I}\;G_i\perp G_j$.
    \end{enumerate}
\end{proposition}

\subsection{ホワイトノイズ関数}

\begin{tcolorbox}[colframe=ForestGreen, colback=ForestGreen!10!white,breakable,colbacktitle=ForestGreen!40!white,coltitle=black,fonttitle=\bfseries\sffamily,
title=]
    $Q$がコンパクト作用素であること,正作用素であること,すべてが交錯している.
    white noiseとは中心化された有限な分散を持つ独立確率変数の組として得られる確率ベクトルをいう.
\end{tcolorbox}

\begin{notation}
    $\mu=N_Q$を中心化された非退化なGauss分布とする.二乗根作用素$Q^{1/2}$の像$\Im Q^{1/2}$をCameron-Martin部分空間という.
    これは$H$の真の部分空間であるが,実は$H$上稠密である.
    この部分空間上に定まる線型作用素$W_-(-):\Im Q^{1/2}\times H\to\R$を
    \[W_f(x):=\Brac{Q^{-1/2}f,x}\quad(f\in Q^{1/2}(H),x\in H)\]
    とおく.

    なお,$Q$が単射(=像の上で可逆)であることより$Q^{1/2}:H\to\Im Q^{1/2}$も単射で,よって$\Im Q^{1/2}$上で可逆だから,$Q^{-1/2}:\Im Q^{1/2}\to H$はひとまず定まっている.
    次に$W_-$は有界線型であるから,
    一意な延長$\o{W}:H\mono L^2(H,\mu)$が存在する.この$L^2(H,\mu)$の部分空間は有界線型汎関数のみからなり,特にGauss空間である.
    すると,独立性の特徴付けより,$W_{f_1},\cdots,W_{f_n}$が独立であることと,$f_1,\cdots,f_n$が直交であることは同値.
\end{notation}

\begin{definition}
    \[Q^{1/2}x:=\sum^\infty_{k=1}\sqrt{\lambda_k}(x|e_k)e_k\quad(x\in H)\]
    を二乗根作用素という.この像を\textbf{Gauss測度$N(0,Q)$に関するCameron-Martin部分空間}という.
\end{definition}

\begin{lemma}\mbox{}
    \begin{enumerate}
        \item $Q\in B^1(H)$は可逆でない.特に,$Q^{-1}$は非有界作用素である.
        \item $Q(H)\subsetneq H$である.
        \item $Q(H)$は$H$上稠密である.
        \item $Q^{1/2}(H)$も$H$上稠密である.
        \item $W:Q^{1/2}(H)\to L^2(H,\mu)$は$H$上に一意に延長し,Hilbert空間の埋め込みを定める.
        \item $\Im W=\oo{H^*}$.
    \end{enumerate}
\end{lemma}

\begin{definition}[white noise]\label{def-white-noise-1}
    Hilbert空間の埋め込み$W:H\mono L^2(H,\mu);f\mapsto \brac{Q^{-1/2}x,f}=\sum_{h=1}^\infty\lambda_h^{-1/2}\brac{x,e_h}\brac{f,e_h}$
    を\textbf{ホワイトノイズ}という.
\end{definition}

\begin{proposition}[white noiseはGauss系である]\mbox{}\label{prop-property-of-white-noise}
    \begin{enumerate}
        \item $z\in H$に対して,値$W_z$は$N(0,\abs{z}^2)$に従う実Gauss変数である.
        \item $z_1,\cdots,z_n\in H\;(n\in\N)$に対して,$(W_{z_1},\cdots,W_{z_n})$は$N_n(0,(\brac{z_h,z_k})_{h,k\in[n]})$に従うGauss変数である.
        \item $W_{z_1},\cdots,W_{z_n}$が独立であることと$z_1,\cdots,z_n$が直交することは同値.
    \end{enumerate}
\end{proposition}

\begin{proposition}[Cameron-Martin部分空間は零集合]
    $\mu(Q^{1/2}(H))=0$.
\end{proposition}

\begin{tbox}{red}{}
    この記法を使うと,$D$を勾配として,$M\varphi:=Q^{1/2}D\varphi$がMalliavin微分となる.
    $M$は可閉作用素になり,その閉包はMalliavin-Sobolev空間$D^{1,2}(H,\mu)$上の作用素となる.
    随伴$M^*$はSkorohod積分またはGauss発散作用素という.
\end{tbox}

\section{確率過程の扱い}

\subsection{確率過程の同値性}

\begin{tcolorbox}[colframe=ForestGreen, colback=ForestGreen!10!white,breakable,colbacktitle=ForestGreen!40!white,coltitle=black,fonttitle=\bfseries\sffamily,
title=]
    確率過程の全体への同値類の入れ方には選択の余地がある.
    \cite{Revus and Yor}では有限次元周辺分布が一致することを,\cite{Nualart},\cite{Prato}では修正であることを「同値」と言っている.
    同値類の中に存在する別の元を「バージョン」ということは変わらない.
\end{tcolorbox}

\begin{definition}[equivalence / version, finite dimensional marginal distributions]\mbox{}
    \begin{enumerate}
        \item 同じ状態空間$(E,\E)$を持つ$(\Om,\F,P),(\Om',\F',P')$上の2つの過程$X,X'$が\textbf{同値である}または一方が他方の\textbf{バージョン}であるとは,
        任意の有限部分集合$\{t_1,\cdots,t_n\}\subset\R_+$と任意の可測集合$A_1,\cdots,A_n\in\E$について,
        \[P[X_{t_1}\in A_1,\cdots,X_{t_n}\in A_n]=P'[X'_{t_1}\in A_1,\cdots,X'_{t_n}\in A_n]\]
        \item 測度$P$の$(X_{t_1},\cdots,X_{t_n}):\Om\to E^n$による押し出しを$P_{t_1,\cdots,t_n}:=P^{(X_{t_1},\cdots,X_{t_n})}$で表す.
        任意の有限集合$\{t_1,\cdots,t_n\}\subset\R_+$に関する押し出し全体の集合$\M_X$を\textbf{有限次元分布}(f.d.d)と呼ぶ.
    \end{enumerate}
\end{definition}
\begin{lemma}[過程の同値性の特徴付け]
    $X,Y$について,次の3条件は同値.
    \begin{enumerate}
        \item $X,Y$は同値である.
        \item $\M_X=\M_Y$.
    \end{enumerate}
\end{lemma}

\begin{definition}[modification / equivalence / version, indistinguishable]
    定義された確率空間も状態空間も等しい2つの過程$X,Y$について,
    \begin{enumerate}
        \item 2つは\textbf{修正}または\textbf{変形}であるとは,$\forall_{t\in\R_+}\;X_t=Y_t\;\as$を満たすことをいう.\footnote{\cite{Nualart}ではこの概念をequivalenceまたはvarsionと呼んでいる.}
        \item 2つは\textbf{識別不可能}であるとは,殆ど至る所の$\om\in\Om$について,$\forall_{t\in\R_+}\;X_t(\om)=Y_t(\om)$が成り立つことをいう.
    \end{enumerate}
\end{definition}

\begin{lemma}\mbox{}
    \begin{enumerate}
        \item $X,Y$が互いの修正であるならば,同値である.
        \item $X,Y$が互いの修正であり,見本道が殆ど確実に右連続ならば,識別不可能である.
    \end{enumerate}
\end{lemma}

\subsection{Kolmogorovの拡張定理}

\begin{tcolorbox}[colframe=ForestGreen, colback=ForestGreen!10!white,breakable,colbacktitle=ForestGreen!40!white,coltitle=black,fonttitle=\bfseries\sffamily,
    title=]
    Kolmogorovの拡張定理は,Hopfの拡張定理の一般化である.
    そこから確率過程論についても,種々の対象が構成できる.
    $\R_+$上の確率過程も,有限次元周辺分布を指定することで構成できるが,$\R^\infty$の場合と違って一意性は担保されない.
\end{tcolorbox}

\begin{theorem}
    確率空間列$(\R^n,\B(\R^n))$上の確率測度列$(\mu_n)$が次の一貫性条件をみたすとき,$(\R^\N,\B(\R^\N))$上の確率測度$\mu$であって$\forall_{A\in\B(\R^n)}\;\mu(A\times\R^\N)=\mu_n(\A)$を満たすものが一意的に存在する.ただし,
    $A\times\R^\N=\Brace{(\om_n)\in\R^\N\mid (\om_1,\cdots,\om_n)\in A}$とした.
    なお,$\R^\N$には直積位相を考える.
    \begin{quotation}
        (consistency) $\forall_{n\in\N}\;\forall_{A\in\B(\R^n)}\;\mu_{n+1}(A\times\R)=\mu_n(A)$.
    \end{quotation}
    特に,この一貫性条件は$\R^\N$上の測度に延長できるための必要十分条件である.
    これは$\R$を一般の完備可分空間としても成り立つ.
\end{theorem}

\begin{corollary}
    $\R$上の確率測度$\mu$に対して,ある確率空間が存在して,これを法則とする独立な確率変数の列$(X_n)_{n\in\N}$が存在する.
\end{corollary}

\begin{corollary}[Hilbert空間により添字付けられた確率過程]
    $H$を可分実Hilbert空間とする.
    ある確率空間$(\Om,\F,P)$とその上の確率変数の族$X=(X_h)_{h\in H}$が存在して,次の2条件を満たす:
    \begin{enumerate}
        \item 写像$X:H\to\Meas(\Om,\R);h\mapsto X_h$は線型である.
        \item 任意の$h\in H$について,確率変数$X_h$は中心化されたGauss変数で,線型写像$X:H\to L^2(\Om)$は等長である:$E[X_h^2]=\norm{h}^2_H$.
        \item $\Im X\simeq_\Hilb H$は$L^2(\Om)$のGauss部分空間である.
    \end{enumerate}
\end{corollary}
\begin{remark}
    この同一視により,独立性$X_h\indep X_{h'}$は添字空間$H$における直交性として理解できる.
\end{remark}

\begin{theorem}[確率過程版]\label{thm-Kolmogorov-extension-theorem}
    確率分布族$\{P_{t_1,\cdots,t_n}\}_{n\in\N_{\ge1},t_i\in\R_+}$が次の条件を満たすとき,ある確率空間$(\Om,\F,P)$が存在して,これらを有限次元周辺分布とする確率過程$(X_t)_{t\in\R_+}$が存在する.
    \begin{enumerate}[({C}1)]
        \item $P_{t_1,\cdots,t_n}\in P(\R^n)$である.
        \item 任意の部分集合$\{t_{k_1}<\cdots<t_{k_m}\}\subset\{t_1<\cdots<t_n\}$について,$P_{t_{k_1}},\cdots,P_{t_{k_m}}$は$P_{t_1,\cdots,t_n}$の対応する周辺分布である.
    \end{enumerate}
\end{theorem}

\section{Brown運動の定義と特徴付け}

\begin{tcolorbox}[colframe=ForestGreen, colback=ForestGreen!10!white,breakable,colbacktitle=ForestGreen!40!white,coltitle=black,fonttitle=\bfseries\sffamily,
title=]
    Brown運動は,平均$0$で分散が$\Gamma=\min$であるようなGauss過程であって,殆ど確実に連続であるような過程である.
\end{tcolorbox}

\begin{definition}[(standard) Brownian motion / Wiener process]
    $(\Om,\F,P)$上の実確率過程$(B_t)_{t\in\R_+}$が\textbf{標準Brown運動}であるとは,次の4条件を満たすことをいう:
    \begin{enumerate}[({B}1)]
        \item $B_0=0\;\as$
        \item $\forall_{n=2,3,\cdots}\;\forall_{0\le t_1<\cdots<t_n}\;B_{t_n}-B_{t_{n-1}},\cdots,B_{t_2}-B_{t_1}$は独立.
        \item 増分について,$B_t-B_s\sim N(0,t-s)\;(0\le s<t)$.
        \item $\Om\to\Meas(\R_+,\R);\om\mapsto(t\mapsto B_t(\om))$について,殆ど確実に$t\mapsto B_t(\om)$は連続.
    \end{enumerate}
    $d$個の独立なBrown運動$B^1,\cdots,B^d$の積$B=(B^1_t,\cdots,B^d_t)_{t\in\R_+}$を\textbf{$d$次元Brown運動}という.
\end{definition}
\begin{example}[Brown運動ではない例]
    Brown運動$(B_t)_{t\in\R_+}$に対して,独立でランダムな時刻$U:\Om\to[0,1]$を考える.$U$は$[0,1]$上の一様分布に従うとする.
    \[\wt{B}_t:=\begin{cases}
        B_t,&t\ne U,\\
        0,&t=U.
    \end{cases}\]
    とすると,$P[s\ne U\land t\ne U]=0$に注意して,
    \[E[\wt{B}_t\wt{B}_s]=E[B_tB_s|s\ne U\land t\ne U]+0=E[B_tB_s1_{\Brace{s\ne U,t\ne U}}]P[s\ne U\land t\ne U]=s\land t.\]
    だから,これは平均$0$で共分散$s\land t$のGauss過程である.
    特に,Brown運動のバージョンである.
    一方で,$U=0$である場合を除いて見本道は連続でない,すなわち殆ど確実に見本道は連続でない.
\end{example}

\begin{lemma}[高次元正規分布の特徴付け]
    正規分布に従う独立な実確率変数$B_{t_1},\cdots,B_{t_n}$について,
    \begin{enumerate}
        \item $B_{t_1},\cdots,B_{t_n}$の線型結合は正規分布に従う確率変数を定める.
        \item 積写像$(B_{t_1},\cdots,B_{t_n}):\Om\to\R^n$は$n$次元正規分布に従う.
    \end{enumerate}
    実は,独立性の仮定は必要ない.
    一般に,$n$次元確率ベクトル$X=(X_1,\cdots,X_n):\Om\to\R^n$が$n$次元正規分布に従うことと,任意の線型汎関数$f\in(\R^n)^*$に関して$f(X):\Om\to\R$が正規確率変数であることとは同値.
\end{lemma}
\begin{proof}\mbox{}
    \begin{enumerate}
        \item $B_{t_1}\sim N(\mu_1,\sigma_1),B_{t_2}\sim N(\mu_2,\sigma_2)$として,$B_{t_1}+B_{t_2}$が正規分布に従うことを示せば十分である.
        一般に,独立な2つの確率変数の和の特性関数は,期待値が積に対して分解することより,元の確率変数の特性関数の積となる:\[\varphi_{X+Y}(u)=E[e^{iuX+Y}]=E[e^{iuX}e^{iuY}]=E[e^{iuX}]E[e^{iuY}]=\varphi_X(u)\varphi_Y(u).\]
        よって,
        \begin{align*}
            \varphi_{B_{t_1}+B_{t_2}}(u)&=\exp\paren{i\mu_1u-\frac{1}{2}\sigma_1^2u^2}\exp\paren{i\mu_2u-\frac{1}{2}\sigma^2_2u^2}\\
            &=\exp\paren{i(\mu_1+\mu_2)u-\frac{1}{2}(\sigma^2_1+\sigma^2_2)u^2}
        \end{align*}
        であるが,これは$B_{t_1}+B_{t_2}\sim N(\mu_1+\mu_2,\sigma^2_1+\sigma^2_2)$を意味する.
        \item $(B_{t_1},\cdots,B_{t_n}):\Om\to\R^n$の特性関数は,それぞれの特性関数の積となるが,正規分布の特性関数の積は正規分布の特性関数となることは(1)で見た:$\varphi_{(B_{t_1},\cdots,B_{t_n})}(u)=\exp\paren{i(\mu_1u_1+\cdots+\mu_nu_n)-\frac{1}{2}(\sigma_1^2u_1^2+\cdots\sigma_n^2u_n^2)}$.
        \item 任意の確率変数$X=(X_1,\cdots,X_n)$と線型汎関数$f(x)=a^\top x\;(a\in\R^n)$について,
        \[\varphi_{f(X)}(v)=E[e^{ivf(X)}]=E[e^{iva^\top X}]=\varphi_X(va)\]
        が成り立つことに注意する.
        \begin{description}
            \item[$\Rightarrow$] $X=(X_1,\cdots,X_n)\sim N_d(\mu,\Sigma)$のとき,右辺によって左辺$\varphi_{f(X)}$が正規分布の特性関数であることが分かる.
            \item[$\Leftarrow$] 任意の$f(X)$が正規であるとき,$v=1$と取ることによって,任意の$a\in\R^n$に対して$\varphi_X(a)$の値が左辺によって分かる.
            このとき,$\varphi_X$の関数形は,多変量正規分布の特性関数に他ならないと分かる.
        \end{description}
    \end{enumerate}
\end{proof}
\begin{remarks}[Cramer-Wold]
    (3)の結果は$\R^n$上の一般の確率分布について成り立ち,Cramer-Woldの定理と呼ばれる.
    原理は単純明快で,一般の$X\in\R^n$について,$\forall_{a\in\R^n}\;\varphi_{a^\top X}(v)=\varphi_X(va)$が成り立つため,$f\in(\R^n)^*$を通じた挙動が特性関数を一意に定める.
\end{remarks}

\begin{lemma}[多次元正規確率変数の成分間の独立性の特徴付け]
    $Y_1:=(X_1,\cdots,X_{k_1})^\top,\cdots,Y_l=(X_{k_{l-1}},\cdots,X_d)^\top\;(l\ge 2)$のように,$X$を$l$個の確率変数$Y_1,\cdots,Y_l$に分ける.
    この分割に対して,$\Sigma$のブロック$\Sigma_{a,b}:=\Cov[Y_a,Y_b]\;(a,b\in[l])$を考える.
    \begin{enumerate}
        \item $Y_1,\cdots,Y_l$は独立.
        \item $\forall_{a,b\in[l]}\;a\ne b\Rightarrow\Sigma_{a,b}=O$.
    \end{enumerate}
\end{lemma}

\begin{proposition}
    実確率過程$B=(B_t)_{t\in\R_+}$について,次の2条件は同値.
    \begin{enumerate}
        \item $B$は(B1),(B2),(B3)を満たす.
        \item $B$は平均$0$共分散$\Gamma(s,t):=\min(s,t)$のGauss過程である.
    \end{enumerate}
\end{proposition}
\begin{proof}\mbox{}
    \begin{description}
        \item[(1)$\Rightarrow$(2)] Gauss過程であることを示すには,任意の$0<t_1<\cdots<t_n\in\R_+$を取り,$B:=(B_{t_1},\cdots,B_{t_n}):\Om\to\R^n$が$n$次元の正規分布に従うことを示せば良い.
        まず,仮定(B1),(B2)より,$B_{t_1},B_{t_2}-B_{t_1},\cdots,B_{t_n}-B_{t_{n-1}}$は独立であり,それぞれが正規分布に従う.
        したがって,積写像$A:=(B_{t_1},B_{t_2}-B_{t_1},\cdots,B_{t_n}-B_{t_{n-1}}):\Om\to\R^n$は$n$次元正規分布に従う(補題(2)).
        行列
        \[J:=\begin{bmatrix}1&0&\cdots&\cdots&0\\1&1&0&\cdots&0\\\vdots&\ddots&\ddots&\ddots&\vdots\\1&\cdots&\cdots&1&0\\1&\cdots&\cdots&1&1\end{bmatrix}\]
        が定める線形変換を$f:\R^n\to\R^n$とおくと,これは明らかに可逆で,$B=f(A)$が成り立つ.
        ここで,任意の線型汎関数$q\in(\R^n)^*$に関して,$q(B)=q(f(A))$は,$q\circ A:\R^n\to\R$が線型であることから,正規分布に従う.
        よって,$B$も正規分布に従う.

        また,仮定(B3)より平均は$m(t)=E[B_t]=0$で,共分散は,$s\ge t$のとき,増分の独立性(B2)に注意して
        \[\Gamma(s,t)=\Cov[B_s,B_t]=E[B_sB_t]=E[B_s(B_t-B_s+B_s)]=E[B_s(B_t-B_s)]+E[B_s^2]=s.\]
        $\Gamma$の対称性より,これは$\Gamma(s,t)=\min(s,t)$を意味する.
        \item[(2)$\Rightarrow$(1)]
        \begin{enumerate}[({B}1)]
            \item 平均と分散を考えると,$m(0)=E[B_0]=0$かつ$\Gamma(0,0)=E[B_0^2]=0$.よって,$E[\abs{B_0}]=0$より,$B_0=0\;\as$
            \item Gauss過程であることより,組$A:=(B_{t_n}-B_{t_{n-1}},\cdots,B_{t_2}-B_{t_1})$は$n$次元正規分布に従う.これらが独立であることを示すには,補題より$A$の分散共分散行列$\Sigma_A$の非対角成分がすべて$0$であることを示せば良い.
            平均が$0$で$\Gamma(s,t)=\min(s,t)$であることより,
            任意の$1<i< j\in[n]$について,共分散の双線型性に注意して,
            \begin{align*}
                \Cov[B_{t_i}-B_{t_{i-1}},B_{t_j}-B_{t_{j-1}}]&=E[(B_{t_i}-B_{t_{i-1}})(B_{t_j}-B_{t_{j-1}})]\\
                &=E[B_{t_i}B_{t_j}]-E[B_{t_{i-1}}B_j]-E[B_{t_i}B_{t_{j-1}}]+E[B_{t_{i-1}}B_{t_{j-1}}]=i-(i-1)-i+(i-1)=0.
            \end{align*}
            \item 平均が$0$で$\Gamma(s,t)=\min(s,t)$であることより,
            \[E[(B_t-B_s)^2]=E[B_t^2]+E[B_s^2]-2E[B_tB_s]=t+s-2s=t-s.\]
        \end{enumerate}
    \end{description}
\end{proof}

\section{Brown運動の構成}

\begin{tcolorbox}[colframe=ForestGreen, colback=ForestGreen!10!white,breakable,colbacktitle=ForestGreen!40!white,coltitle=black,fonttitle=\bfseries\sffamily,
    title=]
    構成のアイデアは複数ある.
    \begin{enumerate}
        \item Gauss過程としての構成.
        \item Fourier級数としての構成.
        \item 対称な酔歩の分布極限としての構成.
    \end{enumerate}
\end{tcolorbox}

\begin{shishin}
    $T>0$として,区間$[0,T]$上の過程$B=(B_t)_{t\in[0,T]}$を条件(B1)から(B4)を満たすように構成すれば,
    過程の列$(Y^{(i)})_{i\in\N}$を「つなげた」過程を
    \[W_t:=\paren{\sum^{\floor{t/T}}_{i=1}Y_T^{(i)}}+Y^{[t/T]+1}_{t-[t/T]T}\]
    と定めると,これが目標の$\R_+$上のBrown運動となる.
    よって以降の構成の議論では,有界区間$[0,T]$上に構成することを目指す.
\end{shishin}

\subsection{Gauss過程としての構成}

\begin{theorem}[Kolmogorov's continuity criterion / 正規化定理 / 連続変形定理]\label{thm-Kolmogorov-continuity-criterion}
    実過程$X=(X_t)_{t\in[0,T]}$が
    \[\exists_{\al,\beta,K>0}\;\forall_{s,t\in[0,T]}\;E[\abs{X_{t}-X_s}^\beta]\le K\abs{t-s}^{1+\al}\]
    を満たすならば,$X$のある連続な修正$\wt{X}$が識別不可能な違いを除いて一意的に存在して,
    任意の$0\le\gamma<\frac{\al}{\beta}$について次を満たす:
    \[\exists_{G_\gamma\in\L(\Om';\R)}\;\forall_{s,t\in[0,T]}\qquad\abs{\wt{X}_t-\wt{X}_s}\le G_\gamma\abs{t-s}^\gamma.\]
    特に,$\wt{X}$の見本道は$\gamma$-次Holder連続である.
\end{theorem}
\begin{proof}
    \cite{Revus and Yor}I.2 Thm(2.1)は,一般の添字集合$t\in[0,1)^d$上のBanach空間値過程$(X_t)$について示している.
\end{proof}
\begin{remarks}
    Gauss過程は積率の計算により,この十分条件を満たす.
\end{remarks}

\begin{construction}\label{construction-Kolmogorov}
    $\Gamma(s,t)=\min(s,t)$は明らかに対称である.
    さらに半正定値であることを示せば,これを共分散とする平均$0$のGauss過程が存在する\ref{prop-existence-of-Gaussian-process}.
    まず,$\min(s,t)=\int_{\R_+}1_{[0,s]}(r)1_{[s,t]}(r)dr$であることに注意すると,
    任意の$n\in\N_{\ge1}$と$a_1,\cdots,a_n\in\R$について,
    \begin{align*}
        \sum_{i,j\in[n]}a_ia_j\min(t_i,t_j)&=\sum_{i,j\in[n]}\int^\infty_01_{[0,t_i]}(r)1_{[0,t_j]}(r)dr\\
        &=\int^\infty_0\paren{\sum_{i=1}^na_i1_{[0,t_i]}(r)}^2dr\ge0.
    \end{align*}
    最後に,この過程が条件(B4)を満たすことを見れば良い.
    これはKolmogorovの正規化定理\ref{thm-Kolmogorov-continuity-criterion}による.
    任意の$0\le s\le  t$について,$B_t-B_s\sim N(0,t-s)$であり,正規分布の奇数次の中心積率は$0$で偶数次の中心積率は
    \[\forall_{k\in\N}\quad E[\abs{B_t-B_s}^{2k}]=\frac{(2k)!}{2^kk!}(t-s)^k\]
    と表せるから,ある$B$の修正が存在して,任意の閉区間$[0,T]$上で$\gamma<\frac{k-1}{2k}$について$\gamma$-次Holder連続である.
    特に,任意の$\gamma<\frac{1}{2}$について,$B$の見本道$t\mapsto B_t(\om)$は殆ど至る所$\gamma$-Holder連続である.
\end{construction}

\subsection{ランダムな係数を持ったFourier級数としての構成}

\begin{tcolorbox}[colframe=ForestGreen, colback=ForestGreen!10!white,breakable,colbacktitle=ForestGreen!40!white,coltitle=black,fonttitle=\bfseries\sffamily,
title=]
    $t\in[0,\pi]$を固定した関数$f(s)=s\land t\in\cS^*([0,\pi])$のFourier級数を計算することで,
    \[s\land t=\frac{st}{\pi}+\frac{2}{\pi}\sum^\infty_{k=1}\frac{\sin ks\sin kt}{k^2}\]
    を得る.すると,$Z_n\sim N(0,1)$について,
    \[W_t:=\frac{t}{\sqrt{\pi}}Z_0+\sqrt{\frac{2}{\pi}}\sum^\infty_{k=1}Z_k\frac{\sin kt}{k}\]
    と定めれば,$(W_t)_{t\in[0,\pi]}$は平均$0$で共分散$E[W_sW_t]=s\land t$のGauss過程であることが期待できる.
    あとは,上記のFourier級数の収束の問題が残るのみである.
    これは以下の議論で$e_n(r)=\frac{1}{\sqrt{\pi}}\cos nt$と取った場合であり,このときのBrown運動$W_t$の表示を\textbf{Paley-Wiener表現}という.
    Wienerが1923年に初めてBrown運動を構成したのはこの方法による.
\end{tcolorbox}

\subsubsection{一般理論}

\begin{notation}
    $T>0$とし,$L^2([0,T])$の正規直交系$(e_n)_{n\ge0}$を取る.
    $\{Z_n\}_{n\ge0}\subset L^2(\Om)$を$N(0,1)$に従う独立同分布確率変数列とする.
\end{notation}

\begin{theorem}[Ito-Nishio]
    上述の設定において,
    \begin{enumerate}
        \item $\sum^\infty_{n=0}Z_n\int^t_0e_n(r)dr$は$L^2(\Om)$内で収束する.
        \item この収束は殆ど至る所一様収束である:$\sup_{t\in[0,T]}\Abs{\sum^N_{n=0}Z_n\int^t_0e_n(r)dr-B_t}\xrightarrow{\as}0$.
        \item 極限$B\in L^2(\Om)$は中心化された共分散$\Gamma=\min$を持つGauss過程である.
    \end{enumerate}
\end{theorem}

\subsubsection{Wienerの構成}

\begin{tcolorbox}[colframe=ForestGreen, colback=ForestGreen!10!white,breakable,colbacktitle=ForestGreen!40!white,coltitle=black,fonttitle=\bfseries\sffamily,
title=]
    Wienerが1923年に初めてBrown運動を構成したのは,基底として三角級数を取った場合である\cite{Bass}(Thm 6.1).
\end{tcolorbox}

\subsubsection{Levyの構成}

\begin{tcolorbox}[colframe=ForestGreen, colback=ForestGreen!10!white,breakable,colbacktitle=ForestGreen!40!white,coltitle=black,fonttitle=\bfseries\sffamily,
title=]
    LevyとCiesielskiは基底としてウェーブレットの一例であるHaar関数系を取った.
    まずは,明示的にHaar基底を使わずに構成し,線形補間としての意味を重視する.
    極めて具体的で,「こうすればBrown運動を構成できる」というのがよく分かる.
    $D_n:=\Brace{\frac{k}{2^n}\in[0,1]\;\middle|\;0\le k\le 2^n}$上に点をうち,その線型補間を行い,$C([0,1])$上での極限を取る.
    この場合も,見本道の連続性は自然に従い,また$\R_+\times\Om\to\R$の積空間上での可測性もわかりやすい.
\end{tcolorbox}

\begin{lemma}\mbox{}
    \begin{enumerate}
        \item $X_1,X_2\sim N(0,\sigma^2)$を独立同分布確率変数とする.
        このとき,$X_1+X_2$と$X_1-X_2$も独立で,いずれも$N(0,2\sigma^2)$に従う.
        \item $X\sim N(0,1)$とする.このとき,
        \[\forall_{x>0}\quad\frac{x}{x^2+1}\frac{1}{\sqrt{2\pi}}e^{-\frac{x^2}{2}}\le P[X>x]\le\frac{1}{x}\frac{1}{\sqrt{2\pi}}e^{-\frac{x^2}{2}}.\]
        \item $(X_n)$を正規確率ベクトルの列で,$\lim_{n\to\infty}X_n=X\;\as$を満たすとする.$b:=\lim_{n\to\infty}E[X_n],C:=\lim_{n\to\infty}\Cov[X_n]$が存在するならば,$X$は$b,C$が定める正規確率変数である.
    \end{enumerate}
\end{lemma}
\begin{proof}\mbox{}
    \begin{enumerate}
        \item $\frac{1}{\sqrt{2}\sigma}(X_1+X_2,X_1-X_2)^\top$は,標準正規分布$\frac{1}{\sigma}(X_1,X_2)^\top$の直交行列$\frac{1}{\sqrt{2}}(1,1;1,-1)$による変換であるから,再び標準正規分布である.
        \item 右辺の不等式については,
        \begin{align*}
            P[X>x]&=\frac{1}{\sqrt{2\pi}}\int^\infty_xe^{-\frac{u^2}{2}}du\\
            &\le\frac{1}{\sqrt{2\pi}}\frac{u}{x}\int^\infty_xe^{-\frac{u^2}{2}}du
            =\frac{1}{\sqrt{2\pi}}\frac{1}{x}e^{-\frac{x^2}{2}}.
        \end{align*}
    \end{enumerate}
\end{proof}

\begin{construction}
    $[0,1]$内の小数第$n$位までの2進有理数の集合を
    \[\D_n:=\Brace{\frac{k}{2^n}\in[0,1]\;\middle|\;0\le k\le 2^n}\]
    で表し,$\D:=\cup_{n\in\N}\D_n$とする.
    各$\D_n$上に点を打ち,その線型補完として$C([0,1])$の列を得て,その一様収束極限が(存在して)Gauss過程であることを導けば良い.
    \begin{description}
        \item[列の構成] $N(0,1)$に従う独立同分布確率変数列が,ある確率空間$(\Om,\A,P)$上に存在する:$\{Z_n\}\subset L^2(\Om)$.これを用いて,
        $\D_n$上の確率過程で,
        \begin{enumerate}
            \item $B_0=0,B_1=Z_1$,
            \item $\forall_{r<s<t\in\D_n}\;B_s-B_r\perp B_t-B_s\sim N(0,t-s)$,
            \item $(B_d)_{d\in\D_n}\perp(Z_t)_{t\in\D\setminus\D_n}$,
        \end{enumerate}
        の3条件を満たすものが任意の$n\in\N$について存在することを示す.

        $n=0$のとき,(1)で定まる$B_0,B_1$は,$B_1-B_0=Z_1\sim N(0,1)$かつ$(0,Z_1)$は$(Z_t)_{t\in\D\setminus\{0,1\}}$と独立である.
        $n>0$のとき,$d\in\D_n\setminus\D_{n-1}$について
        \[B_d:=\frac{B_{d-1/2^n}+B_{d+1/2^n}}{2}+\frac{Z_d}{2^{(n+1)/2}}\]
        とおけば良い.実際,このとき$\D_{n-1}\ni B_{d-1/2^n},B_{d+1/2^n}\perp (Z_t)_{t\in\D\setminus\D_{n-1}}$より,帰納法の仮定から$B_d\perp(Z_t)_{t\in\D\setminus\D_n}\subset(Z_t)_{t\in\D\setminus\D_{n-1}}$が成り立つ.
        よって後は条件(2)の成立を示せば良い.
        帰納法の仮定から$\frac{B_{d-1/2^n}-B_{d+1/2^n}}{2},\frac{Z_d}{2^{(n+1)/2}}$は独立に$N(0,1/2^{n+1})$に従う.
        よって,和と差$B_d-B_{d+1/2^n},B_d-B_{d-1/2^n}$も独立に$N(0,1/2^n)$に従う.このとき,$(B_d-B_{d-1/2^n})_{d\in\D_n\setminus\{0\}}$が独立であることを示せば十分である.
        これらは多次元正規分布を定めるから,対独立性を示せば十分である.
        
        これら新たな点が定める新たな$1/2^n$増分は,$\D_{n-1}$の点の凸結合としたから,$\D_{n-1}$の言葉で表せて,独立性は帰納法の仮定から従う.
        \item[一様収束極限の存在]
        $(B_d)_{d\in\D_n\setminus\D_{n-1}}$の線型補間を$F_n:[0,1]\to\R$と表す($D_{-1}=\emptyset$とする)と,
        \[\forall_{n\in\N}\;\forall_{d\in\D_n}\quad B_d=\sum^n_{i=0}F_i(d)=\sum^\infty_{i=0}F_i(d)\]
        が成り立つ.このとき,$\sum^\infty_{i=0}F_i(t)$は$C([0,1])$の一様ノルムについて収束すること,従って$\norm{F_n}_\infty$が$0$に収束することを示せば良い.

        任意の$c>1$と($\frac{1}{c\sqrt{n}}\le\sqrt{\frac{\pi}{2}}$を満たすくらい)十分大きな$n$について,
        \[P[\abs{Z_d}\ge c\sqrt{n}]\le2\frac{1}{c\sqrt{n}}\frac{1}{\sqrt{2\pi}}e^{-\frac{c^2n}{2}}\le e^{-\frac{c^2n}{2}}.\]
        よって,ある$N\in\N$について,
        \begin{align*}
            \sum^\infty_{n=N}P[\exists_{d\in\D_n}\;\abs{Z_d}\ge c\sqrt{n}]&<\sum^\infty_{n=N}\sum_{d\in\D_n}P[\abs{Z_d}\ge c\sqrt{n}]\\
            &\le\sum^\infty_{n=N}(2^n+1)\exp\paren{-\frac{c^2n}{2}}
        \end{align*}
        が成り立つから,$c>\sqrt{2\log 2}$を満たす$c$を取れば,級数$\sum^\infty_{n=0}P[\exists_{d\in\D_n}\;\abs{Z_d}\ge c\sqrt{n}]<\infty$が成り立つ.
        これを満たす$c$を一つ任意に取ると,Borel-Cantelliの補題から,
        \[P[\limsup_{n\to\infty}\Brace{\exists_{d\in\D_n}\;\abs{Z_d}\ge c\sqrt{n}}]\le\lim_{N\to\infty}\sum_{n=N}^\infty P[\exists_{d\in\D_n}\;\abs{Z_d}\ge c\sqrt{n}]=0.\]
        すなわち,$\exists_{N\in\N}\;\forall_{n\ge N}\;\forall_{d\in D_n}\;\abs{Z_d}<c\sqrt{n}$.
        これは,$\forall_{n\ge N}\;\norm{F_n}_\infty<c\sqrt{n}2^{-n/2}$を意味する.
        よって,$B_t:=\sum^\infty_{i=0}F_i(t)$は確率$1$で一様収束する.
        \item[極限過程はBrown運動である]
        任意に$0\le t<s\in[0,1]$を取ると,$\D$の列$(t_n),(s_n)$が存在して$t,s$にそれぞれ収束する.
        極限過程$B$の連続性より,$B_s=\lim_{n\to\infty}B_{s_n}$が成り立つ.
        よって,$B_{t_n}\sim N(0,t_n)$より,
        $E[B_t]=\lim_{n\to\infty}E[B_{t_n}]=0$.
        また,
        \[\Cov[B_s,B_t]=\lim_{n\to\infty}\Cov[B_{s_n},B_{t_n}]=\lim_{n\to\infty}s_n=s.\]
        以上より,$(B_t)_{t\in I}$はBrown運動である.
    \end{description}
\end{construction}

\begin{proposition}[jointly measurable]
    このように構成された$(B_t)_{t\in\R_+}$は,$\Om\times\R_+$上可測である.
    ただし,$(\Om,\A,P)$とは,独立同分布確率変数列$(Z_n)$の定義域となる確率空間とする.
\end{proposition}
\begin{proof}
    $B_t=\sum^\infty_{i=0}F_i(t)$と定めたから,$\forall_{n\in\N}\;F_n(t)$は$\Om\times[0,1]$上可測であることを示せば良い.
    $\R_+$上のBrown運動は,これら$(B_t)_{t\in[0,1]}$の和として表せるため.
    任意の$c\in\R$について,
    \[\Brace{(\om,t)\in\Om\times[0,1]\mid F_n(t)>c}\]
    が可測であることを示せばよいが,この集合は,
    \begin{enumerate}
        \item $c\ge0$のとき$\bigcup_{k=1}^{2^{n-1}}\Brace{Z_{\frac{2k-1}{2^n}}\ge c}\times\paren{\frac{2k-1}{2^n}-\frac{Z_{\frac{2k-1}{2^n}}-c}{Z_{\frac{2k-1}{2^n}}}\frac{1}{2^n},\frac{2k-1}{2^n}+\frac{Z_{\frac{2k-1}{2^n}}-c}{Z_{\frac{2k-1}{2^n}}}\frac{1}{2^n}}$.
        \item $c<0$のとき\[\bigcup_{k=1}^{2^{n-1}}\Brace{Z_{\frac{2k-1}{2^n}}>c}\times\Square{\frac{2k-2}{2^n},\frac{2k}{2^n}}\cup\Brace{Z_{\frac{2k-1}{2^n}}\le c}\times\paren{\left[\frac{2k-2}{2^n},\frac{2k-1}{2^n}-\frac{Z_{\frac{2k-1}{2^n}}-c}{Z_{\frac{2k-1}{2^n}}}\frac{1}{2}\right)\cup\left(\frac{2k-1}{2^n}+\frac{Z_{\frac{2k-1}{2^n}}-c}{Z_{\frac{2k-1}{2^n}}}\frac{1}{2^n},\frac{2k}{2^n}\right]}.\]
    \end{enumerate}
    と等しいから,たしかに$\Om\times[0,1]$の積$\sigma$-加法族の元である.
\end{proof}

\subsection{酔歩としての構成}

\begin{tcolorbox}[colframe=ForestGreen, colback=ForestGreen!10!white,breakable,colbacktitle=ForestGreen!40!white,coltitle=black,fonttitle=\bfseries\sffamily,
title=]
    酔歩のスケールとして得られるこの方法は,Brown運動が基本的な対象であることの直感的な説明となる(これを「不変性」という表現で捉えている).
    また,$C(\R_+)$における極限であるから,連続性は自然に従う.
    酔歩と同様にBrown運動は1,2次元においては再帰的であるが,3次元以上では過渡的である.
    もはや酔歩とは異なる点に,スケール不変性が挙げられる.

    さらにここでDonskerの定理が登場する.
    中心極限定理を連続化したものがDonskerの定理(functional central limit theorem)であるが,i.i.d.の誤差の分布は,$-\infty,\infty$では$y=0$になるBrown運動を固定端で走らせれば良い.
\end{tcolorbox}

\begin{notation}[酔歩]
    任意の$T>0$に対して,これを$n\in\N$等分する考え方で,独立同分布に従う$n$個の確率変数$\xi_1,\cdots,\xi_n\sim(0,T/n)$,または$X_i:=\sqrt{n}\xi_i\sim(0,T)$を考える.
    この和$R_k:=\sum_{i=1}^k\xi_i=\frac{1}{\sqrt{n}}\sum^k_{i=1}X_i\;(k\in[n])$を考えると,過程$(R_k)_{k\in[n]}$は$n$ステップの酔歩で,分散が$T/n$である.
    $n$をscale parameterという.
    この過程について,$n\to\infty$の極限を取ることを考える.すると,ステップ数は増え,ステップの幅は小さくなる.
    このとき,中心極限定理より,列$(R_k)_{k\in\N}$は正規分布$N(0,T)$に分布収束する.
\end{notation}

\begin{discussion}[酔歩の線形補間]
    $R_n(t):=\frac{1}{\sqrt{n}}\sum_{i=1}^{\floor{tn/T}}X_i$はcadlagではあるが連続ではないため,より簡単な対象にしたい.そこで,
    $\forall_{k=0,\cdots,n}\;S_n\paren{\frac{kT}{n}}=R_k$を満たす$S_n:[0,T]\to\R$を,線型補間による連続延長とする.
    すると過程$(S_n)_{n\in\N}:\N\times\Om'\to C([0,T])$が定まる.
    2つの過程$(R_n),(S_n)$について,差は$\ep_n(T):=\max(\xi_1,\cdots,\xi_{\floor{nT}+1})$
    なる過程を超えない.
    %$\R^d$上の下側区間の定義関数全体からなる集合$\F$は任意の分布$P$に関してDonskerクラスになるというのがDonskerによる古典的な結果である.
\end{discussion}

\begin{proposition}
    $E[X_1^2]<\infty$ならば,任意の$T>0$に対して次が成り立つ:$\forall_{\delta>0}\;P\Square{\sup_{t\in[0,T]}\abs{R_n(t)-S_n(t)}>\delta}=P(\ep_n(T)>\delta)\to0\;(n\to\infty)$.
\end{proposition}

\begin{theorem}[Donsker's invariance principle/ functional central limit theorem]
    $(X_m)$を平均$0$分散$T$の独立同分布とする.
    これが定める線形補間$(S_n)_{n\in\N}:\N\times\Om'\to C([0,T])$は,Brown運動$(B_t:=\sqrt{T}W_t)_{t\in\R_+}:\Om\to C([0,T])$に分布収束する.
\end{theorem}
\begin{remark}[Donsker's theorem]
    Donskerの元々の論文に載っている定理は,この形であった.
    その後,セミパラの研究者が,外積分に基づいて,独自の理論を作った.
    なお,「不変性」とは,増分過程$(\xi_i)$または$(X_i)$の分布に依らないことを指す.
\end{remark}
\begin{remarks}
    この証明は,一般の酔歩(部分和過程)を,停止時の列におけるBrown運動の値として表現できるというSkorokhodの結果を用いて,Skorokhod埋め込み表現から証明する.
\end{remarks}

\subsection{連続時間Markov過程としての構成}

\begin{tcolorbox}[colframe=ForestGreen, colback=ForestGreen!10!white,breakable,colbacktitle=ForestGreen!40!white,coltitle=black,fonttitle=\bfseries\sffamily,
title=]
    熱核を用いて実際に確率測度を$\R^D$上に構成する.$D\subset[0,1]$は二進有理数の全体とし,Kolmogorovの拡張定理を用いる極めて具体的な構成である\cite{舟木}.
\end{tcolorbox}

\subsection{martingale法による構成}

\begin{tcolorbox}[colframe=ForestGreen, colback=ForestGreen!10!white,breakable,colbacktitle=ForestGreen!40!white,coltitle=black,fonttitle=\bfseries\sffamily,
title=]
    Gauss過程を得たあと,残るは条件(B4)の連続性であるが,これを示すのにKolmogorovの正規化定理に依らず,martingale性を利用する方法がある\cite{Bass}(Thm 6.2).
\end{tcolorbox}


\begin{theorem}[martingaleによる正則化]
    平均$0$,共分散$\min$のGauss過程$(W_t)_{t\in[0,1]}$にはあるバージョンが存在し,Brown運動である(見本道は殆ど確実に連続である).
\end{theorem}

\subsection{見本道の空間上への構成}

\begin{tcolorbox}[colframe=ForestGreen, colback=ForestGreen!10!white,breakable,colbacktitle=ForestGreen!40!white,coltitle=black,fonttitle=\bfseries\sffamily,
title=]
    $C(\R_+)$上の測度を直接構成して,$\id:C(\R_+)\to C(\R_+)$としてBrown運動を得ることも出来る\ref{prop-construction-of-Wiener-measure}.
\end{tcolorbox}

\subsection{白色雑音からの構成}

\begin{tcolorbox}[colframe=ForestGreen, colback=ForestGreen!10!white,breakable,colbacktitle=ForestGreen!40!white,coltitle=black,fonttitle=\bfseries\sffamily,
title=]
    \ref{sec-Wiener-integral}節へ.
\end{tcolorbox}

\section{Brown運動の変種}

\subsection{固定端Brown運動}

\begin{definition}[Brownian bridge]
    共分散を$\Gamma(s,t)=\min(s,t)-\frac{st}{T}$とする中心化されたGauss過程$(B_t)_{t\in[0,T]}\;(T>0)$を\textbf{Brown橋}という.
    共分散の形から,両端$\partial[0,T]$で固定され,真ん中$T/2$で最も不確実性が大きくなる過程であると分かる.
    これはBrown運動を条件づけた過程$B_t:=[W_t|W_T=0]\;(t\in[0,T])$と理解できる.
\end{definition}
\begin{remarks}
    これは次の微分方程式の解としても実現される:
    \[dX_t=dB_t+\frac{-X_t}{T-t}dt,\;t\in[0,T),\quad X_0=0.\]
\end{remarks}

\begin{construction}
    1次元Brown運動$(W_t)$に対して
    $\paren{W_t-\frac{t}{T}W_T}_{t\le T}$とすれば,これはBrown橋である.
\end{construction}

\begin{proposition}
    Brown橋は共分散$\Cov[X_s,X_t]=s\lor t-st$が定める連続な中心化されたBrown過程である.
\end{proposition}

\subsection{多様体上のBrown運動}

\begin{definition}
    Riemann多様体$(M,g)$上のLaplace作用素$\Laplace_M/2$が生成する拡散過程$(X_t,P_x)_{x\in M}$を$M$上のBrown運動と呼ぶ.ただし,$P_x$は熱方程式の基本解を用いて表示できる.
\end{definition}

\begin{proposition}
    任意の$x\in\R$に対して,$B(0)=x$を満たす\textbf{両側Brown運動}$(B_t)_{t\in\R}$が存在する.
\end{proposition}

\begin{proposition}
    任意の多様体上のBrown運動に対して,Euclid空間へ多様体を埋め込むこと,または直交枠束を用いることで,あるベクトル場が定める確率微分方程式の解として実現される.
\end{proposition}

\subsection{幾何Brown運動}

\begin{example}[geometric Brownian motion]
    他の確率過程として,
    \[\exp\paren{\mu t+\sigma B_t-\paren{\frac{\al^2t}{2}}}\]
    を幾何ブラウン運動という.
\end{example}

\subsection{ドリフトと拡散係数を持ったBrown運動}

\begin{example}[Brownian motion with drift and diffusion coefficient]
    標準Brown運動$(B_t)$に対して,$X_t:=x+\mu t+\sigma B_t$によって定まる過程を,$x\in\R$から開始し,$\mu\in\R$のドリフトを持ち,拡散係数$\sigma^2\in\R$のBrown運動という.
\end{example}

\subsection{Ornstein-Uhlenbeck拡散}

\begin{tcolorbox}[colframe=ForestGreen, colback=ForestGreen!10!white,breakable,colbacktitle=ForestGreen!40!white,coltitle=black,fonttitle=\bfseries\sffamily,
title=]
    摩擦の存在下での質量の大きいBrown粒子の速度のモデルとして最小の応用が見つかった,定常過程かつMarkov過程でもあるGauss過程である.
    なお,この3条件を満たす(時空間変数の線形変換の別を除いて)唯一の非自明な過程である.
    時間的に平均へドリフトする,平均回帰性(mean-reverting)を持つために,拡散過程でもある.
\end{tcolorbox}

\begin{definition}[Ornstein-Uhlenbeck diffusion]
    Brown運動に,時間変数に$\exp(2-)$を合成し,時間に応じて$e^{-t}$にスケールした過程
    $(X_t:=e^{-t}B_{e^{2t}})_{t\in\R}$を\textbf{Ornstein-Uhlenbeck過程}または\textbf{平均回帰過程}という.
\end{definition}
\begin{remarks}\mbox{}
    \begin{enumerate}
        \item 一次元周辺分布は$\forall_{t\in\R}\;X_t\sim N(0,1)$である.
        \item 離散時間の過程に$AR(1)$というものがあり,この連続化とみなせる.
        \item Brown運動と違って,時間反転可能であり,$(X_t)_{t\in\R_+},(X_{-t})_{t\in\R_+}$は分布が等しい.
        \item $-\infty$近傍の振る舞いはBrown運動の$0$近傍の振る舞いに関係がある.
        \item これが満たす確率微分方程式をLengevin方程式という:
        \[\dd{U(t)}{t}=-\lambda U(t)+\dot{B}(t)\]
        ただし$\dot{B}$は白色雑音である.
    \end{enumerate}
\end{remarks}



\subsection{Brownian sheet}

\begin{tcolorbox}[colframe=ForestGreen, colback=ForestGreen!10!white,breakable,colbacktitle=ForestGreen!40!white,coltitle=black,fonttitle=\bfseries\sffamily,
title=]
    $N=2$のBrown運動は,Wiener空間上の確率解析において基本的な役割を果たす.
\end{tcolorbox}

\begin{definition}
    $\R_+^N$をパラメータにもつ中心化されたGauss確率変数の族$(X_a)_{a\in\R_+^N}$が次の2条件を満たすとき,$N$次元パラメータのBrown運動という:
    \begin{enumerate}
        \item $E[X_aX_b]=a\land b:=\prod_{i=1}^N(a_i\land b_i)$.
        \item $P[X_0=0]=1$.
    \end{enumerate}
\end{definition}
\begin{proposition}
    $N=2$のとき,独立な標準Brown運動の列$(B_t^{(n)})_{n\in\N}$を用いて,
    \[B_{s.t}:=sB_t^{(0)}+\sum_{n=1}^\infty B_t^{(n)}\frac{\sqrt{2}}{n\pi}\sin(n\pi s)\quad(s,t)\in[0,1]\times\R_+\]
    で構成される.
\end{proposition}

\section{Brown運動の性質}

\subsection{フラクタル曲線としての見本道の対称性}

\begin{tcolorbox}[colframe=ForestGreen, colback=ForestGreen!10!white,breakable,colbacktitle=ForestGreen!40!white,coltitle=black,fonttitle=\bfseries\sffamily,
title=]
    スケール不変であり,時間反転が可能である=関数$1/t$と$t$によって時間と空間のパラメータを変換すると再びBrown運動である.
    大数の法則によるとBrown運動が$y=\pm x$が囲む空間から無限回出ることはありえないが,$y=\pm\sqrt{n}$が囲む空間からは殆ど確実に無限回出ていく.
\end{tcolorbox}

\begin{proposition}\mbox{}\label{prop-character-of-Brownian-motion-1}
    \begin{enumerate}
        \item 自己相似性・スケール不変性:$\forall_{a>0}\;(X_t:=a^{-1/2}B_{at})_{t\ge0}$はBrown運動である.
        \item (原点)対称性:$(-B)_{t\in\R_+}$はBrown運動である.
        \item 時間一様性:$\forall_{h>0}\;(B_{t+h}-B_h)_{t\in\R_+}$はBrown運動である.
        \item 時間の巻き戻し:$(X_t:=B_1-B_{1-t})_{t\in[0,1]}$と$(B_t)_{t\in[0,1]}$とは分布が等しい.
        \item 時間反転不変性:次もBrown運動である:
        \[X_t=\begin{cases}
            tB_{1/t}&t>0,\\0&t=0
        \end{cases}\]
        \item 大数の法則:$\lim_{t\to\infty}\frac{B_t}{t}=0\;\as$\footnote{時間反転により,$\infty$での性質を$0$に引き戻して考えることが出来る,という議論の順番がきれいだと考えて入れ替えた.}        
        \item $1/2$-大数の法則\footnote{大数の法則と併せてみると,長い目で見て,Brown運動は線型関数よりは遅く増加するが,$\sqrt{t}$よりはlimsupが速く増加する.}:
        \[\limsup_{n\to\infty}\frac{B_n}{\sqrt{n}}=+\infty\;\as,\qquad\liminf_{n\to\infty}\frac{B_n}{\sqrt{n}}=-\infty\;\as\]
    \end{enumerate}
\end{proposition}
\begin{proof}\mbox{}
    \begin{enumerate}
        \item 
        \begin{enumerate}[({B}1)]
            \item $X_0=a^{-1/2}B_0=0\;\as$
            \item $a^{-1/2}(B_{at_n}-B_{at_{n-1}}),\cdots,a^{-1/2}(B_{2}-B_{1})$は,独立な確率変数の可測関数$\R\to\R;x\mapsto a^{-1/2}x$との合成であるから,再び独立である.
            \item $B_{at}-B_{as}\sim N(0,a(t-s))$だから,$X_t-X_s=a^{-1/2}(B_{at}-B_{as})\sim N(0,a(t-s))$.
            \item $t\mapsto B_t(\om)$が連続になる$\om\in\Om$について,$t\mapsto a^{-1/2}B_{at}$は連続関数との合成からなるため連続.
        \end{enumerate}
        \item (1)と同様の議論による.
        \item (1)と同様の議論による.
        \item (1)と同様の議論とWiener測度の一意性による?
        \item $(X_t)$が平均$0$共分散$\min$のGauss過程であることを示してから,(B4)を示す.
        \begin{description}
            \item[Gauss過程であることの証明] 
            有限次元周辺分布$(X_{t_1},\cdots,X_{t_n})$は,正規確率ベクトル$(B_{1/t_1},\cdots,B_{1/t_n})$の,行列$\diag(t_1,\cdots,t_n)$が定める変換による値であるから,再び多変量正規分布に従う.
            よって,$(X_t)$はGauss過程である.
            平均は$E[X_t]=E[tB_{1/t}]=0$.共分散は
            \[\Cov[X_s,X_t]=E[X_sX_t]=stE[B_{1/s}B_{1/t}]=st\min\paren{\frac{1}{s},\frac{1}{t}}=\min(s,t).\]
            \item[見本道の連続性の証明]
            $t\mapsto X(t)$は$t>0$については殆ど確実に連続であるから,$t=0$での連続性を示せば良い.
            \begin{enumerate}[(a)]
                \item まず,$(X_t)_{t\in\R_+\cap\Q}$の分布は,標準Brown運動の制限$(B_t)_{t\in\R_+\cap\Q}$が$\R^\infty$上に誘導するものと等しい(Kolmogorovの拡張定理の一意性より).よって,$\lim_{t\searrow0,t\in\Q}X_t=\lim_{t\searrow0,t\in\Q}B_t=0\;\as$
                \item $(X_t)$は$\R_{>0}$上で殆ど確実に連続であることと,$\R_{>0}\cap\Q$のその上での稠密性より,極限は一致する:$\lim_{t\searrow0}X_t=\lim_{t\searrow,t\in\Q}X_t=0\;\as$
                
                実際,あるfull set $F\in\F$が存在して$\forall_{\om\in F}\;X_t(\om)$は$t\in\R_{>0}$上連続であるが,
                このとき\[\Abs{\lim_{t\searrow0}X_t(\om)-\lim_{t\searrow 0,t\in\Q}X_t(\om)}=2\ep>0\]とすると,
                \[\exists_{\delta_1>0}\;0<s<\delta_1\Rightarrow\Abs{X_s(\om)-\lim_{t\searrow0}X_t(\om)}<\ep\]
                \[\exists_{\delta_2>0}\;0<s<\delta_2\Rightarrow\Abs{X_s(\om)-\lim_{t\searrow 0,t\in\Q}X_t(\om)}<\ep\]
                が成り立つが,このとき$(0,\min(\delta_1,\delta))\cap\Q\ne\emptyset$上での$X_t$の値について矛盾が生じている.
            \end{enumerate}
        \end{description}
        \item (5)より,
        \[\lim_{t\to\infty}\frac{B(t)}{t}=\lim_{t\to\infty}X(1/t)=X(0)=0\;\as\]
        \item 任意の$c\in\Z_+$について,Brown運動のスケール不変性(1)より,$\{B_n>c\sqrt{n}\}\Leftrightarrow\{n^{-1/2}B_n>c\}\Leftrightarrow\{B_1>c\}$である.
        まず,集合に関するFatouの補題から,$P[\Brace{B_n>c\sqrt{n}\;\io}]\ge\limsup_{n\to\infty}P[\Brace{B_n>c\sqrt{n}}]=P[\Brace{B_1>c}]>0$.
        次に,$X_n:=B_n-B_{n-1}$とすると,$\Brace{B_n>c\sqrt{n}\;\io}=\Brace{\sum^n_{j=1}X_j>c\sqrt{n}\;\io}$は$(X_n)$について可換な事象であるから(手前の有限個の$X_1,\cdots,X_m$に対する置換に対して不変)であるから,Hewitt-Savageの定理より,$P[\Brace{B_n>c\sqrt{n}\;\io}]=1$.
        最後に,任意の$c\in\Z_+$について合併事象を取ると,$P\Square{\limsup_{n\to\infty}\frac{B_n}{\sqrt{n}}=\infty}=1$を得る.
    \end{enumerate}
\end{proof}

\begin{proposition}[ユニタリ変換不変性 / conformal invariance property]
    多次元正規分布の性質が一般化される.
    $B$を$d$次元Brown運動,$U\in\GL_d(\R)$を直交行列とする.このとき,$(X_t:=UB_t)_{t\in\R_+}$はBrown運動である.
\end{proposition}

\subsection{連続性概念の補足}

\begin{tcolorbox}[colframe=ForestGreen, colback=ForestGreen!10!white,breakable,colbacktitle=ForestGreen!40!white,coltitle=black,fonttitle=\bfseries\sffamily,
title=]
    一様連続性の強さを定量化する手法が連続度である.
    $\ep$-$\delta$論法において,$\delta$から$\ep$を与える関数関係を分類することで,連続性を分類する.
\end{tcolorbox}

\begin{definition}[modulus of continuity]
    $(X,\norm{-}_X),(Y,\norm{-}_Y)$をノルム空間とし,$f:X\to Y$を関数とする.
    $X,Y$のそれ以外の位相的性質は不問とした方が良い.
    \begin{enumerate}
        \item $\om(\delta;f):=\sup_{\abs{x-y}\le\delta}\abs{f(x)-f(y)}$とおくと,$\om:[0,\infty]\to[0,\infty]$が定まる.
        このとき,$\forall_{f\in\Map(X,Y)}\;\om(0;f)=0$に注意.
    \end{enumerate}
\end{definition}

\subsubsection{大域的性質}

\begin{lemma}[uniform continuity]
    次の2条件は同値.
    \begin{enumerate}
        \item $\om(-;f)$は$\delta=0$にて連続:$\lim_{\delta\to0}\om(\delta;f)=0$.
        \item $f$は一様連続.
    \end{enumerate}
\end{lemma}

\begin{definition}[Lipschitz continuity, Holder continuity]
    $f:X\to Y$を関数とする.
    \begin{enumerate}
        \item $\om(\delta;f)=O(\delta)$を満たすとき,\footnote{$O(\delta)\;(\delta\to0)$ではなく,$\delta$の全域で$\delta$の定数倍で抑えられることをいう.}$f$はLipschitz定数$\sup_{\delta\in(0,\infty]}\frac{\om(\delta;f)}{\delta}$のLipschitz連続関数という.
        \item $\al\in[0,1]$について$\om(\delta;f)=O(\delta^\al)$を満たすとき,$f$は$\al$-次Holder連続であるという.なお,$\al$は大きいほど$\delta^\al\;(\delta\in(0,1))$は小さいから,条件は強く,$\al=0$のときは有界性に同値.
        \item $\abs{f}_{C^{0,\al}}:=\sup_{x\ne y}\frac{\abs{f(x)-f(y)}}{\abs{x-y}}$をHolder係数といい,これはHolder空間$C^{0,\al}$に半ノルムを定める.
    \end{enumerate}
    これらの条件は,$\om(\delta;f)$の$\delta\to0$のときの$0$への収束の速さによる分類であるから,これらを満たす関数は特に一様連続であることは明らか.
\end{definition}
\begin{remarks}
    Lipschitz連続性は,$X$上で任意の2点$x\ne y\in X$を取っても,その間の$f$の増分の傾き$\abs{f(x)-f(y)}/\abs{x-y}$はある定数を超えないことを意味する.
    特に有界な1次導関数を持つ場合,その上限$\sup_{x\in X}f'(x)$はLipschitz定数以下になるが,一致するとは限らない.
\end{remarks}

\subsubsection{局所的性質}

\begin{lemma}[continuous, locally uniform continuous]
    $f:X\to Y$と$x_0\in X$について,
    次の2条件は同値.
    \begin{enumerate}
        \item $\lim_{\delta\to0}\om(\delta;f|_{U_\delta(x_0)})=0$.
        \item $f$は$x_0$において連続.
    \end{enumerate}
    また次の2条件も同値.
    \begin{enumerate}
        \item ある近傍$x_0\in U\osub X$が存在して,$f|_U$の連続度$\om$は$\delta=0$にて連続.
        \item $f$は$x_0$において局所一様連続.
    \end{enumerate}
\end{lemma}
\begin{remark}
    この2つは$X$が局所コンパクトならば同値になる.したがって,たとえば無限次元Banach空間では2つの概念は区別する必要がある.
\end{remark}
\begin{proof}\mbox{}
    \begin{description}
        \item[(1)$\Rightarrow$(2)] 
        任意の$\ep>0$を取ると,仮定よりある$\delta>0$が存在して,$\sup_{\abs{x-y}\le\delta,x,y\in U_{\delta}(x_0)}\abs{f(x)-f(y)}<\ep$.
        よって特に$\abs{x_0-y}<\delta$ならば,$\abs{f(x_0)-f(y)}\le \sup_{\abs{x-y}\le\delta,x,y\in U_{\delta}(x_0)}\abs{f(x)-f(y)}<\ep$.
        \item[(2)$\Rightarrow$(1)]
        任意の$\ep>0$を取ると,ある$\delta>0$について$\forall_{y\in X}\;\abs{x_0-y}<\delta\Rightarrow\abs{f(x_0)-f(y)}<\ep/2$が成り立つ.
        このとき,
        \[\sup_{\abs{x-y}\le\delta,x,y\in U_\delta(x_0)}\abs{f(x)-f(y)}\le\sup_{\abs{x-y}\le\delta,x,y\in U_\delta(x_0)}\paren{\abs{f(x)-f(x_0)}+\abs{f(x_0)-f(y)}}<\ep.\]
    \end{description}
\end{proof}

\begin{definition}
    Holder係数$\abs{f}_{C^{0,\al}}:=\sup_{x\ne y}\frac{\abs{f(x)-f(y)}}{\abs{x-y}}\in[0,\infty]$が$X$のあるコンパクト集合上で有界ならば,局所Holder連続であるという.
\end{definition}

\subsubsection{関数の分類}

\begin{tcolorbox}[colframe=ForestGreen, colback=ForestGreen!10!white,breakable,colbacktitle=ForestGreen!40!white,coltitle=black,fonttitle=\bfseries\sffamily,
title=]
    ある$\om\in\Om$に対して,これを連続度として持つ関数として,連続性に基づいた類別を作れる.
    似た概念に一様可積分性がある.
\end{tcolorbox}

\begin{definition}
    $x=0$において連続で消える関数全体を
    \[\Om:=\Brace{\om\in\Map([0,\infty],[0,\infty])\mid\lim_{\delta\to0}\om(\delta)=\om(0)=0}\]
    と表す.
    $f:X\to Y,x_0\in X$について,
    \begin{enumerate}
        \item $\om_{x_0}(\delta;f):=\om(\delta;f|_{U_{\delta}(x_0)})$を,$x_0$における局所連続度としよう.
        \item 局所連続度の上方集合を$\Om_{x_0}(f):=\Brace{\om_{x_0}\in\Om\mid\forall_{y\in X}\;\abs{f(x_0)-f(y)}\le\om_{x_0}(\abs{x_0-y})}$と表すと,大域連続度の上方集合は$\Om(f):=\bigcap_{x\in X}\Om_x(f)$と表せる.
    \end{enumerate}
\end{definition}

\begin{definition}[equicontinuity, uniform equicontinuity]
    関数族$\{f_\lambda\}\subset\Map(X,Y)$について,
    \begin{enumerate}
        \item 同程度連続であるとは,任意の$x\in X$において,ある$\om_x(\delta;f)\in\Om_x(f)$が存在して,任意の$f_\lambda$の局所連続度(の上界)となっていることをいう.
        \item 一様に同程度連続であるとは,ある$\om(\delta;f)\in\Om(f)$が存在して,任意の$f_\lambda$の大域連続度(の上界)となっていることをいう.
    \end{enumerate}
    これらの概念も,$X$がコンパクトな場合は一致する.
\end{definition}

\begin{lemma}
    同程度連続な関数列$(f_n)$がある関数$f$に各点収束するとき,$f$は連続である.
\end{lemma}

\begin{theorem}
    有界閉区間$I:=[a,b]$上の関数列$\{f_n\}_{n\in\N}\subset l^\infty(I)$について,次の2条件を満たすならば一様ノルムについて相対コンパクトである,すなわち,一様収束する部分列を持つ.
    \begin{enumerate}
        \item 一様有界である:$\exists_{C\in\R}\;\forall_{n\in\N}\;\norm{f_n}_\infty\le M$.
        \item 一様に同程度連続である.
    \end{enumerate}
\end{theorem}

\subsubsection{確率化}

\begin{definition}[modulus of continuity]
    Brown運動のコンパクト集合への制限$B:[0,1]\times\Om\to\R$の\textbf{連続度}$\om_B$とは,
    \[\sup_{s\ne t\in[0,1]}\frac{\abs{B_t-B_s}}{\om_B(\abs{s-t})}\le 1\]
    かつ$\lim_{\delta\searrow0}\om_B(\delta)=0$を満たす関数$\om_B(\delta):[0,1]\times\Om\to\R_+$をいう.
\end{definition}
\begin{remark}
    Brown運動の見本道は連続だから,
    \[\limsup_{h\searrow 0}\sup_{t\in[0,1-h]}\frac{\abs{B_{t+h}-B_t}}{\varphi(h)}\le 1\]
    を考えても問題ない.
    これは,特に偏差$h$が小さい場合に関する評価になる.
    通常の意味での連続度$\om_B(f)$を得るためには,$[0,1]$上で$h$以上離れている2点についての$\sup_{h\le\abs{s-t}\le\delta}\abs{B_s-B_t}$と$\varphi(\delta)$との大きい方を$\om_B(\delta)$として取れば良い.

    一方で,supを外して,計測関数$\varphi:\R_+\to\R_+$であって,$\limsup_{t\in\R_+}\frac{B(t)}{\varphi(t)}\in\R_+$を満たすものを探すことは,連続度と対になる.
\end{remark}

\subsection{見本道の連続性}

\begin{tcolorbox}[colframe=ForestGreen, colback=ForestGreen!10!white,breakable,colbacktitle=ForestGreen!40!white,coltitle=black,fonttitle=\bfseries\sffamily,
    title=]
    見本道はコンパクト集合上で一様連続である.それがどのくらい強いかを測る尺度がHolder連続性であるが,
    見本道は任意の$0\le\gamma<\frac{1}{2}$について,$\gamma$-Holder連続であることは,Kolmogorovの連続変形定理\ref{thm-Kolmogorov-continuity-criterion}から従う.

    連続度の全体は最小元をもつが,ことに$\delta$が十分$0$に近いとき,$\om_B(\delta)=\sqrt{2\delta\log(\delta)}$が最小となる.これは,$\delta^{1/2}$より少し悪いことを意味する.
\end{tcolorbox}

\begin{proposition}\mbox{}
    \begin{enumerate}
        \item $1/2$-Holder連続ではない:$P\Square{\sup_{s\ne t\in[0,1]}\frac{\abs{B_t-B_s}}{\sqrt{\abs{t-s}}}=+\infty}=1$.\footnote{実は,局所$1/2$-Holder連続な点も存在するが,高々零集合である.また,$\al>1/2$については,ほとんど確実に,任意の点で$\al$-Holder連続でない.これは微分可能性についてのPaleyの結果よりも強い主張である.}
        \item 連続度の例:ある定数$C>0$について,十分小さい任意の$h>0$と任意の$h\in[0,1-h]$について,\[\abs{B_{t+h}-B_t}\le C\sqrt{h\log(1/h)}\;\as\]
        \item $\forall c<\sqrt{2}$について,$\forall_{\ep>0}\;\exists_{h\in(0,\ep)}\;\exists_{t\in[0,1-h]}\;\abs{B_{t+h}-B_t}\ge c\sqrt{h\log(1/h)}$.
        \item 最適評価(Levy 1937):
        $C=\sqrt{2}$のとき最適な連続度となる:\footnote{$\limsup$を$\lim$に置き換えても成り立つ.}
        \[\limsup_{\delta\searrow 0}\sup_{s,t\in[0,1],\abs{t-s}<\delta}\frac{\abs{B_t-B_s}}{\sqrt{2\abs{t-s}\log\abs{t-s}}}=1\;\as\]
        \item 任意の$\al<1/2$について,殆ど確実に見本道は任意の点で局所$\al$-Holder連続である.
        \item 任意の$\al>1/2$について,殆ど確実に見本道は任意の点で局所$\al$-Holder連続でない.これはPaley-Wiener-Zygmund 1933よりも強いことに注意.
        \item ほとんど確実に,見本道のどこかには局所$1/2$-Holder連続であるような点が存在する.これをslow pointという.
    \end{enumerate}
\end{proposition}
\begin{proof}\mbox{}
    \begin{enumerate}
        \item 時間反転により$1/2$-大数の法則\ref{prop-character-of-Brownian-motion-1}(7)に帰着する.
        $\sup_{s\ne t\in[0,1]}\frac{\abs{B_t-B_s}}{\sqrt{\abs{t-s}}}>\sup_{t\in(0,1]}\frac{\abs{B_t-B_0}}{\sqrt{\abs{t-0}}}$であるから,
        \[P\Square{\sup_{t\in(0,1]}\frac{\abs{B_t}}{\sqrt{t}}=\infty}=1\]
        を示せば良いが,実は
        \[P\Square{\limsup_{t\searrow0}\frac{\abs{B_t}}{\sqrt{t}}=\infty}=1\]
        が成り立つ.実際,$X_t$を(5)の時間反転とすると,
        \[\limsup_{t\searrow0}\frac{\abs{B_t}}{\sqrt{t}}=\limsup_{t\searrow0}\sqrt{t}\abs{X_{1/t}}=\limsup_{s\to\infty}\frac{\abs{X_s}}{\sqrt{s}}=\infty\;\as\]
        \item 保留.
        \item a
        \item (7)と(8)の関係同様,時間反転による.
        \item e
        \item $X$を,$B$の時間反転とし(5),この$t=0$における微分可能性を考える.
        このとき,(7)より,
        \begin{align*}
            D^*X(0)&=\limsup_{h\searrow0}\frac{X_h-X_0}{h}\\
            &\ge\limsup_{n\to\infty}\frac{X_{1/n}-X_0}{1/n}\\
            &\ge\limsup_{n\to\infty}\sqrt{n}X_{1/n}=\limsup_{n\to\infty}\frac{B_n}{\sqrt{n}}=\infty.
        \end{align*}
        すなわち,$D^*X(0)=\infty$で,$D_*X(0)=-\infty$も同様にして得る.特に,$X_t$は$t=0$にて微分可能でない.
        
        ここで,任意の$t>0$に対して,$Y_s:=B_{t+s}-B_t$とすると,$Y_s$も標準Brown運動で(3),$Y_0$にて微分可能でない.
        よって,$B$も$B_t$において微分可能でない.
    \end{enumerate}
\end{proof}

\begin{proposition}\mbox{}
    \begin{enumerate}
        \item 最小包絡関数:$\limsup_{t\to\infty}\frac{B_t}{\sqrt{2t\log\log(t)}}=1\;\as$\footnote{$1/2$-Holder連続性が$1/2$-大数の法則の系であったように,繰り返し対数の法則はこの系になる.}
        \item 一点における連続性の結果・繰り返し対数の法則(Khinchin 1933):\[\forall_{s\in\R_+}\;\limsup_{t\searrow s}\frac{\abs{B_t-B_s}}{\sqrt{2\abs{t-s}\log\log\abs{t-s}}}=1\;\as\]
    \end{enumerate}
\end{proposition}

\subsection{見本道の可微分性}


\begin{proposition}
    関数$f$の右・左微分係数を$D^*f(t):=\limsup_{h\searrow0}\frac{f(t+h)-f(t)}{h},D_*f(t)$で表すとする
    \begin{enumerate}
        \item 任意の$t\in\R_+$について,殆ど確実に$B_t$は$t$において微分可能でない上に,殆ど確実に$D^*B(t)=+\infty\land D_*B(t)=-\infty$.
        \item (Paley-Wiener-Zygmund 1933) $B$の見本道は殆ど確実に至る所微分不可能性である.\footnote{Paley et al. 1933, Dvoretzky et al. 1961.}
        さらに,ほとんど確実に$\forall_{t\in\R_+}\;D^*B(t)=+\infty\lor D_*B(t)=-\infty$.
        \item ほとんど確実に$\forall_{t\in[0,1)}\;D^*B(t)\in\{\pm\infty\}$という主張は正しくない.
    \end{enumerate}
\end{proposition}

\subsection{変分}

\begin{tcolorbox}[colframe=ForestGreen, colback=ForestGreen!10!white,breakable,colbacktitle=ForestGreen!40!white,coltitle=black,fonttitle=\bfseries\sffamily,
title=]
    確率過程の2次変分過程は,高頻度取引の分野においてrealized volatilityといい,重要な統計量となっている.
    その漸近分散を求めるのに,混合型中心極限定理が必要となる.
    この定理は非エルゴード的な確率システムにおける「中心極限定理」となる.
\end{tcolorbox}

\begin{proposition}[erratic]\mbox{}
    \begin{enumerate}
        \item 殆ど確実に,任意の$0<a<b<\infty$に対して,Brown運動の見本道は$[a,b]$上単調でない.
        \item ほとんど確実に,局所的な増加点$t\in(0,\infty)\;\st\;\exists_{t\in(a,b)\osub\R_+}\;[\forall_{s\in(a,t)}\;f(s)\le f(t)]\land[\forall_{s\in(t,b)}\;f(t)\le f(s)]$を持たない.
    \end{enumerate}
\end{proposition}

\subsubsection{2次変分}

\begin{notation}
    $\Sigma(0,T)$で,$[0,T]$上の分割$\sigma$全体の集合とする.すると,$\sigma_1\le\sigma_2:\Leftrightarrow\abs{\sigma_1}\le\abs{\sigma_2}$によって分割の集合には順序が定まる.
    \[J_\sigma:=\sum_{k=1}^n\abs{B_{t_k}-B_{t_{k-1}}}^2\]
    によって,写像$J:\Sigma(0,T)\to\R_+$が定まる.
\end{notation}

\begin{proposition}[2次変動の$L^2$-収束]
    $[0,t]$の部分分割$\pi:=(0=t_0<t_1<\cdots<t_n=t)$の系列$(\pi_n),\pi_n\subset\pi_{n+1}$について,$\abs{\pi}:=\max_{j\in n}(t_{j+1}-t_j)$とする.
    このとき,次の$L^2(\Om)$-収束が成り立つ:
    \[\lim_{\abs{\pi}\to0}J_\pi=\lim_{\abs{\pi}\to0}\sum^{n-1}_{j=0}(B_{t_{j+1}}-B_{t_j})^2=t\quad\in L^2(\Om).\]
\end{proposition}
\begin{proof}
    確率変数列を$\xi_j:=(B_{t_{j+1}}-B_{t_j})^2-(t_{j+1}-t_j)\;(j\in n)$とおくと,これらは中心化された独立な確率変数列になる.
    実際,
    \[E[\xi_j]=E[(B_{t_{j+1}}-B_{t_j})^2]-E[t_{j+1}-t_j]=t_{j+1}-t_j-(t_{j+1}-t_j)=0.\]
    また,各$\xi_j$は独立な確率変数列$(B_{t_{j+1}}-B_{t_j})$の可測関数による像であるから,やはり独立である.
    
    正規分布の偶数次の中心積率は$\mu_{2r}=\frac{(2r)!}{2^rr!}\sigma^{2r}$と表せるため,$E[(B_{t_{j+1}}-B_{t_j})^4]=3(t_{j+1}-t_j)^2$であることに注意すると,
    \begin{align*}
        E\Square{\paren{\sum^{n-1}_{j=0}(B_{t_{j+1}}-B_{t_j})^2-t}^2}&=E\Square{\paren{\sum^{n-1}_{j=0}\xi_j}^2}=\sum^{n-1}_{j=0}E[\xi^2_j]\\
        &=\sum^{n-1}_{j=0}\paren{3(t_{j+1}-t_j)^2-2(t_{j+1}-t_j)^2+(t_{j+1}-t_j)^2}\\
        &=2\sum^{n-1}_{j=0}(t_{j+1}-t_j)^2\le 2t\abs{\pi}\xrightarrow{\abs{\pi}\to0}0.
    \end{align*}
\end{proof}
\begin{remark}
    より強く,概収束も示せる.なお,$(\pi_i)$が部分分割の系列でない場合,反例が殆ど確実に存在する.
    というのも,殆ど確実に分割の列で$\abs{\pi_n}\to 0$を満たすものが存在して,
    \[\limsup_{n\to\infty}\sum^{k(n)}_{j=1}(B(t_j^{(n)})-B(t^{(n)}_{j-1}))^2=\infty\]
    が成り立つ.
    他にこういうものを排除する十分条件として,$\sum_{n\in\N}\abs{\pi_n}<\infty$がある.
\end{remark}

\begin{proposition}[2次変動の概収束]
    $[0,t]$の分割の列$(\pi^n)_{n\ge 1},\pi^n:=\Brace{0=t^n_0<\cdots<t^n_{k_n}=t}$は$\sum_{n\in\N}\abs{\pi^n}<\infty$を満たすとする.
    このとき,
    \[\sum^{k_n-1}_{j=0}\paren{B_{t^n_{j+1}}-B_{t^n_j}}^2\xrightarrow{\as}t.\]
\end{proposition}

\subsubsection{1次変分}

\begin{definition}[bounded variation]
    右連続関数$f:[0,t]\to\R$が\textbf{有界変動}であるとは,
    \[V_f^{(1)}(t):=\sup\sum^k_{j=1}\abs{f(t_j)-f(t_{j-1})}<\infty\]
    を満たすことをいう.有界変動関数は2つの単調増加関数の差で表されるクラスと一致する.
\end{definition}


\begin{corollary}[全変動の発散]
    区間$[0,t]$における全変動
    \[V:=\sup_{\pi}\sum^{n-1}_{j=0}\abs{B_{t_{j+1}}-B_{t_j}}\]
    は殆ど確実に$\infty$である.
\end{corollary}
\begin{proof}
    Brown運動の見本道は殆ど確実に連続であるから,$\sup_{j\in n}\abs{B_{t_{j+1}}-B_{t_j}}\xrightarrow{\abs{\pi}\to0}0$である.
    よって,もし$V<\infty$ならば,2乗和
    \begin{align*}
        V_2:=\sum^{n-1}_{j=1}(B_{t_{j+1}}-B_{t_j})^2&\le\sup_{j\in n}\abs{B_{t_{j+1}}-B_{t_j}}\paren{\sum^{n-1}_{j=0}\abs{B_{t_{j+1}}-B_{t_j}}}\\
        &\le V\sup_{j\in n}\abs{B_{t_{j+1}}-B_{t_j}}\xrightarrow{\abs{\pi}\to0}0.
    \end{align*}
    よって,2次変動の$L^2$-極限は,
    \[\lim_{\abs{\pi}\to0}\sum^{n-1}_{j=0}(B_{t_{j+1}}-B_{t_j})^2=P[V=\infty]\paren{\lim_{\abs{\pi}\to0}\sum^{n-1}_{j=0}(B_{t_{j+1}}-B_{t_j})^2}=t\]
    と表せる.
    $P[V<\infty]=0$でない限り,2次変動が$t$に$L^2$-収束することに矛盾する.
\end{proof}



\section{Wiener積分と白色雑音}\label{sec-Wiener-integral}

\begin{tcolorbox}[colframe=ForestGreen, colback=ForestGreen!10!white,breakable,colbacktitle=ForestGreen!40!white,coltitle=black,fonttitle=\bfseries\sffamily,
    title=]
    Wiener測度は,Wiener過程によって誘導されるGauss測度の例である.
    Wiener測度が定めるLebesgue積分を考えると,
    Stieltjes積分の状況がうまく模倣出来,これは伊藤積分の制限になる.
    このとき,Brown運動が有界変動関数(分布関数)に当たるが,
    ホワイトノイズが積分にあたる.
\end{tcolorbox}

\subsection{定義と像}

\begin{tcolorbox}[colframe=ForestGreen, colback=ForestGreen!10!white,breakable,colbacktitle=ForestGreen!40!white,coltitle=black,fonttitle=\bfseries\sffamily,
title=$\Om$に関する抽象化を施して無限次元積分を定義する]
    有界変動な部分は零集合であるから,Stieltjes積分は零集合上でしか定まらない.
    そこで,Riemann和=単関数に関するLebesgue積分を取る作用素$\E_0\to L^2(\Om)$の一意な延長として得られる有界線型作用素$L^2(\R_+)\to L^2(\Om)$をWiener積分とする.
    すなわち,Stieltjes和は任意の$\om\in\Om$については定まらないから,$L^2(\Om,\F,P)$上の元とみてそこでの収束先を積分と定めよう.
\end{tcolorbox}

\begin{notation}
    \[\E_0:=\Brace{\varphi_t=\sum^{n-1}_{j=0}a_j1_{(t_j,t_{j+1}]}(t)\in L^2(\R_+)\;\middle|\;n\ge 1,a_0,\cdots,a_{j-1}\in\R,0=t_0<\cdots<t_n}\]
    は$L^2(\R_+)$の稠密部分空間である.
\end{notation}

\begin{definition}[Paley-Wiener integral]\label{def-Wiener-integral}
    単関数$\varphi\in\E_0$上の有界線型作用素$\E_0\to L^2(\Om)$
    \[\int_{\R_+}\varphi_tdB_t:=\sum^n_{j=0}a_j(B_{t_{j+1}}-B_{t_j})\]
    は等長写像であるが,この$L^2(\R_+)$への連続延長$B:L^2(\R_+)\mono L^2(\Om)$を\textbf{Wiener積分}という.
    像は$B(\varphi)=\int^\infty_0\varphi_tdB_t\sim N(0,\norm{\varphi}^2_{L^2(\R_+)})$を満たし,$\Im B<L^2(\Om)$は$L^2(\R_+)$と等長同型な,Brown運動が生成するGauss部分空間となる.
\end{definition}

\begin{lemma}
    任意の$\varphi\in L^2(\R_+)$のWiener測度によるLebesgue積分は,正規分布に従う確率変数
    $B(\varphi)\sim N(0,\norm{\varphi}^2_{L^2(\R_+)})$となる.
\end{lemma}

\subsection{白色雑音}

\begin{tcolorbox}[colframe=ForestGreen, colback=ForestGreen!10!white,breakable,colbacktitle=ForestGreen!40!white,coltitle=black,fonttitle=\bfseries\sffamily,
title=]
    ホワイトノイズ\ref{def-white-noise-1}とWiener積分\ref{def-Wiener-integral}は同一視出来る.
    ここではそれを利用して,Wiener積分の言葉を用いてホワイトノイズを定義する.
\end{tcolorbox}

\begin{definition}[white noise]
    $D\subset\R^m$をBorel集合とする.$l$を$\R^m$上のLebesgue測度とする.
    \begin{enumerate}
        \item $D$内の測度確定な集合の全体を$\M(D):=\Brace{A\in\B(\R^m)\cap P(D)\mid l(A)<\infty}$と表す.
        \item $\M(D)$上の平均$0$のGauss過程$(W(A))_{A\in\M(D)}$であって,$\Gamma(A,B)=E[W(A)W(B)]=l(A\cap B)$を満たすものをいう.
    \end{enumerate}
\end{definition}
\begin{remarks}
    この族$W:\M(D)\to L^2(\Om)$は,埋め込み$\M(D)\mono L^2(D);A\mapsto 1_A$によって
    線型な等長写像$L^2(D)\ni\varphi\mapsto\int_D\varphi(x)W(dx)\in\o{W}\subset L^2(\Om)$に延長出来る.
    そこで,有界線型作用素としてもホワイトノイズは定義できる\ref{def-white-noise-1}.
    ただし,$\o{W}$は$W=\{W(A)\}_{A\in\D}\subset L^2(\Om)$が生成するGauss部分空間である.
\end{remarks}

\subsection{Brown運動の超関数微分としての白色雑音}

\begin{tcolorbox}[colframe=ForestGreen, colback=ForestGreen!10!white,breakable,colbacktitle=ForestGreen!40!white,coltitle=black,fonttitle=\bfseries\sffamily,
title=]
    関数を,積分核として線型作用素を定めるという働きの側面に注目して一般化したのが超関数であった.
    そこで,Brown運動を定める線型作用素(の仮想的積分核)として,白色雑音を定める.この関係を「白色雑音は,ブラウン運動の弱微分である」という.
    これは$\S'$上のGauss測度として一意に定まる.
\end{tcolorbox}

\begin{discussion}[白色雑音は,Brown運動の超関数の意味での時間微分である]
    Wiener測度を緩増加超関数の空間に実現することで,仮想的な時間微分$dB_t$は自然に捉えられる.
    $\S$をSchwartzの急減少関数の空間とし,$\S'$をその双対空間とし,$\brac{-,-}$をこの間のペアリングとする.
    このとき,次の定理が成り立つ.
    \begin{theorem}[白色雑音の存在定理]\label{thm-existence-of-white-noise}
        $(\S',\B(\S'))$上には,標準Gauss測度$N(0,\id_{\S})$がただ一つ存在する.すなわち,
        次を満たす$(\S',\B(\S'))$上の確率測度$\nu$がただ一つ存在する:
        \[\forall_{\varphi\in\S}\quad\int_{\S'}\exp{i\brac{u,\varphi}}\nu(du)=\exp\paren{-\frac{1}{2}\int_\R\varphi^2(t)dt}=\exp\paren{\frac{1}{2}\brac{\id\varphi,\varphi}}\]
    \end{theorem}
    このときの確率空間$(\S',\B(\S'),\nu)$を\textbf{白色雑音}という.
    これを引き起こす過程$\dot{B}$を,$\R$上のBrown運動$(B_t)_{t\in\R}$を用いて構成する.
    \[\dot{B}(\varphi):=-\int_\R B_t\varphi'(t)dt\quad(\varphi\in\S)\]
    とすると,これは$(\S,\B(\S))$上の$\S'$-値確率変数であり,白色雑音$\nu$に従う.
\end{discussion}

\subsection{Brown運動の特徴付け}

\begin{tcolorbox}[colframe=ForestGreen, colback=ForestGreen!10!white,breakable,colbacktitle=ForestGreen!40!white,coltitle=black,fonttitle=\bfseries\sffamily,
title=]
    Brown運動はWiener積分を通じてホワイトノイズを定義する.
    一方でホワイトノイズの累積分布関数としてBrown運動が得られる.
    これはStieljes積分を通じて,ある有界変動関数$d\rho$が有界線型汎関数$\int-d\rho\in (C([a,b]))^*$を定める関係と
    と相似形であるが,用いている測度は今回は$C(\R_+)$上のWiener測度である.
\end{tcolorbox}

\begin{proposition}
    $f\in L^2(\R_+)$のWiener積分はホワイトノイズである:$W_f=\int^\infty_0f(s)dB(s)$.
\end{proposition}

\begin{proposition}
    $H:=L^2([0,T])$とし,$\mu=N(0,Q)$を非退化なGauss測度とする.
    $W:H\to L^2(H,\mu)$を$\R_+$上のホワイトノイズとする.
    \begin{enumerate}
        \item $B_t:=W([0,t])=W_{1_{[0,t]}}$と定めるとこれはBrown運動の修正である.
        \item この過程の同値類は,$B\in C(\R_+;L^{2m}(H,\mu))\;(\forall_{m\in\N})$を満たす連続過程である.
    \end{enumerate}
\end{proposition}
\begin{proof}\mbox{}
    \begin{enumerate}
        \item \begin{enumerate}[({B}1)]
            \item $B_t\sim N(0,t)\;(t\ge0)$より,$B_0=0$.
            \item 任意の$0\le s<t$について,$B_t-B_s=W_{1_{(s,t]}}\sim N(0,t-s)$.
            \item 任意の$0\le t_1<\cdots<t_n$に対して,$,1_{(t_1,t_2]},\cdots,1_{(t_{n-1},t_n]}$は直交系である.よって\ref{prop-property-of-white-noise}より,その像$B_{t_2}-B_{t_1},\cdots,B_{t_n}-B_{t_{n-1}}$も独立である.
            \item Gamma関数の反転公式より,
            \[\int^1_0(1-r)^{\al-1}r^{-\al}dr=B(\al,1-\al)=\frac{\pi}{\sin\pi\al}\]
            から,恒等式
            \[\int^t_s(t-\sigma)^{\al-1}(\sigma-s)^{-\al}d\sigma=\frac{\pi}{\sin\pi\al}\quad(0\le s\le\sigma\le t,\al\in(0,1))\]
            を得る.
            これを用いれば,
            \[1_{[0,t]}(s)=\frac{\sin\pi\al}{\pi}\int^t_0(t-\sigma)^{\al-1}\underbrace{1_{[0,\sigma]}(x)(\sigma-s)^{-\al}}_{=:g_\sigma(s)}d\sigma\qquad(s\ge0)\]
            を得る.このとき,$g_\sigma\in L^2([0,T])$かつ$\norm{g_\sigma}^2=\int^T_0g_\sigma^2(s)ds=\frac{\sigma^{1-\al}}{1-\al}$より,$\al\in\paren{0,\frac{1}{2}}$とリスケールすると,$\norm{g_\sigma}^2=\frac{\sigma^{1-2\al}}{1-2\al}$.
            よって,$W:H\to L^2(H,\mu)$は有界線型であったから特に一様連続で,
            \[B_t=W_{1_{[0,t]}}=\frac{\sin\pi\al}{\pi}\int^t_0(t-\sigma)^{\al-1}W_{g_\sigma}d\sigma\]
            と表示でき,$W_{g_\sigma}\sim N\paren{0,\frac{\sigma^{1-2\al}}{1-2\al}}$.
            補題より,あとは$\sigma\mapsto W_{g_\sigma}(x)\in L^{2m}([0,T])\;\mu\dae x\in H$を示せば良い.
            これは正規分布の$2m$次の積率とFubiniの定理から分かる.
        \end{enumerate}
        \item 任意の$t>s$について,$B_t-B_s\sim N_{t-s}$の$2m$次の積率から分かる.
    \end{enumerate}
\end{proof}

\begin{lemma}
    $m>1,\al\in\paren{\frac{1}{2m},1},T>0$とし,$f\in L^{2m}([0,T];H)$とする.このとき,
    \[F(t):=\int^t_0(t-\sigma)^{\al-1}f(\sigma)d\sigma\quad t\in[0,T]\]
    と定めると,$F\in C([0,T];H)$.
\end{lemma}

\subsection{Path積分}

\begin{tcolorbox}[colframe=ForestGreen, colback=ForestGreen!10!white,breakable,colbacktitle=ForestGreen!40!white,coltitle=black,fonttitle=\bfseries\sffamily,
title=]
    量子力学,量子場理論で生まれた考え方である.
\end{tcolorbox}

\begin{discussion}
    伝播関数(propagator)は,粒子が移動する際の確率振幅を与える.
    これはある積分核$K$が与える積分変換作用素$U$で与えられるが,この積分核が曲者である.
    場の付値が関数$\varphi$で与えられ,関数$\varphi$の空間上の,作用汎関数$S$の積分
    \[K(x,y):=\int\exp(iS(\varphi))D\varphi\]
    で与えられる.しかし,$D\varphi$なる測度が不明瞭どころか,対応する測度が存在しないこともある.
    \textbf{道積分(path integral)}の語源は,多様体$X$上の粒子を記述するシグマ模型においては,$\varphi\in C([0,1],X)$は厳密に「道」になるため.
    ここで(古典的)Wiener空間が登場する.
\end{discussion}

\section{Wiener空間}

\begin{tcolorbox}[colframe=ForestGreen, colback=ForestGreen!10!white,breakable,colbacktitle=ForestGreen!40!white,coltitle=black,fonttitle=\bfseries\sffamily,
title=]
    Wiener(1923)が,初期の無限次元空間とその上の微積分の例となった.
\end{tcolorbox}

\subsection{古典的Wiener空間}

\begin{tcolorbox}[colframe=ForestGreen, colback=ForestGreen!10!white,breakable,colbacktitle=ForestGreen!40!white,coltitle=black,fonttitle=\bfseries\sffamily,
title=]
    Brown運動は,その見本道全体の確率空間$(\Om,\F,P)$上の恒等写像として標準的に実現できる.
    これが4つ目の構成となる.
\end{tcolorbox}

\begin{definition}[Wiener space]\mbox{}
    \begin{enumerate}
        \item $\Om:=\Brace{\om\in C(\R_+;\R)\mid\om(0)=0}$.
        \item $\F$を,$\R_+$上の広義一様収束位相が生成するBorel $\sigma$-代数とする.このコンパクト開位相について,$\Om$はFrechet空間となる.
    \end{enumerate}
\end{definition}

\begin{proposition}[Wiener測度の構成]\label{prop-construction-of-Wiener-measure}
    $(\Om,\F,P)$をWiener空間とする.
    \begin{enumerate}
        \item $\R_+$上の広義一様収束位相が生成するBorel $\sigma$-代数$\F$は次の円筒集合の集合$\cC\subset P(\Om)$によって生成される:$\B(\Om)=\sigma[\cC]$
        \[\cC=\Brace{C=\Brace{\om\in\Om\mid\forall_{i\in[k]}\;\om(t_i)\in A_i}\in P(\Om)\;\middle|\;k\in\N_{\ge1},A_1,\cdots,A_k\in\B(\R),0\le t_1<\cdots<t_k}\]
        \item $p_t(x)=(2\pi t)^{-1/2}e^{-x^2/(2t)}$を正規分布$N(0,t)$の確率密度関数とする.
        \[P(C)=\int_{A_1\times\cdots\times A_k}p_{t_1}(x_1)p_{t_2-t_1}(x_2-x_1)\cdots p_{t_k-t_{k-1}}(x_k-x_{k-1})dx_1\cdots dx_k.\]
        が成り立ち,これの$\F$上への延長はたしかに一意的である.
        \item 任意の$\om\in\Om$に対して,$B_t(\om):=\om(t)$と見本道を定めると,この対応$\Om\to\Om$はBrown運動である.
    \end{enumerate}
\end{proposition}
\begin{proof}\mbox{}
    \begin{enumerate}
        \item \begin{description}
            \item[$\sigma(\cC)\subset\F$] 任意の$k\in\N,0\le t_1<\cdots<t_k,A_1,\cdots,A_k\in\B^1(\R)$に対して,$C(t_1\in A_1,\cdots,t_k\in A_k)=\bigcap_{i\in[k]}\pr_{t_i}^{-1}(A_i)$と表せる.$A_i$が全て開集合であるとき,$C$も開集合となる.$A_i$が一般のとき,写像の逆像の集合演算に対する関手性より,$C$は開集合または閉集合の可算和・積で表せる.
            \item[$\F\subset\sigma(\cC)$] コンパクト開位相は半ノルムの族$(\rho_n(\om):=\sup_{t\in K_n}\abs{\om(t)})_{n\in\N}$が定める始位相と一致する:任意の$n\in\N,A\in\B^1(\R)$に関して,$\rho_n^{-1}(A)=W(K_n,A)\in\sigma[\cC]$である.
        \end{description}
        \item 
        $(\Om,\F,P)$は$\sigma$-有限である:時刻$t\in\R_+$を固定し,$\R$の増大コンパクト集合$(A_n)$に対して$C(t\in A_n)$を考えると,$P(C(t\in A_n))<1$である.
        あとは$\cC$が有限加法的であることと,$P$がその上で完全加法的であることを示せば良い.
        \begin{description}
            \item[$\cC$の有限加法性] 任意の$C(\om_1\in A_1,\cdots,\om_k\in A_k)\cup C(\om'_1\in A'_1,\cdots,\om'_l\in A'_l)\in\cC$,$C(\om_1\in A_1,\cdots,\om_k\in A_k)^\comp=C(\om_1\in A_1^\comp,\cdots,\om_k\in A_k^\comp)\in\cC$.
            \item[$P$の完全加法性] $\{C_n\}\subset\cC$を互いに素な集合列で,$C:=\sum_{n\in\N}C_n\in\cC$を満たすとする.これについて$P(C)=\sum_{n\in\N}P(C_n)$を示せば良い.
            このとき,必要ならば添字を並び替えることにより,ある$k\le l\in\N$を用いて,
            \begin{align*}
                C&=\Brace{\om\in\Om\mid\om(t_1)\in A_1,\cdots,\om(t_k)\in A_k,\om(t_{k+1})\in\R,\cdots,\om(t_l)\in\R},&
                C_n&=\Brace{\om\in\Om\mid\om(t_1)\in A_1^n,\cdots,\om(t_l)\in A_l^n}
            \end{align*}
            と表せる.ただし,
            \[A_1=\sum_{n\in\N}A_1^n,\cdots,A_k=\sum_{n\in\N}A_k^n,\R=\sum_{n\in\N}A_i^n\;(k<i\le l).\]
            このとき,$P(C)=\sum_{n\in\N}P(C_n)$は明らか.
        \end{description}
        \item $(B_t)_{t\in\R_+}$の任意の見本道$B_t(\om)=\om(t)$は連続である.
        Brown運動の存在は認めたから,あとは$P$がBrown運動の法則であることを示せば良い.
        \begin{enumerate}[({B}1)]
            \item $\forall_{\om\in\Om}\;B_0(\om)=\om(0)=0$.
            \item 任意の$0\le t_1<\cdots,t_n$について,$B_{t_n}-B_{t_{n-1}},\cdots,B_{t_2}-B_{t_1}$が独立であることを示すために,確率ベクトル$(B_{t_2}-B_{t_1},\cdots,B_{t_n}-B_{t_{n-1}})^\top$の特性関数を調べる.
            Brown運動$(W_t)$の独立増分性より,
            \begin{align*}
                \int_\Om e^{iu\cdot \begin{pmatrix}B_{t_2}-B_{t_1}\\\vdots\\B_{t_n}-B_{t_{n-1}}\end{pmatrix}}&=\int_{\Om'}e^{iu\cdot \begin{pmatrix}W_{t_2}-W_{t_1}\\\vdots\\W_{t_n}-W_{t_{n-1}}\end{pmatrix}}dP'\\
                &=\int_{\Om'}e^{iu(W_{t_2}-W_{t_1})}dP'\times\cdots\times\int_{\Om'}e^{iu(W_{t_n}-W_{t_{n-1}})}dP'\\
                &=\int_\Om e^{iu(B_{t_2}-B_{t_1})}dP\times\cdots\times\int_\Om e^{iu(B_{t_n}-B_{t_{n-1}})}dP.
            \end{align*}
            \item 任意の$0\le s<t$について,$B_t-B_s\sim N(0,t-s)$を示す.補題の確率変数$W:\Om'\to\Om$を用いて,
            \begin{align*}
                \varphi(u)&=\int_\Om e^{iuB_t(\om)-B_s(\om)}dP(\om)\\
                &=\int_\Om e^{iu(\om(t)-\om(s))}P(d\om)\\
                &=\int_\Om e^{iu W(t,\om')-W(s,\om')}P'(d\om')=\exp\paren{-\frac{1}{2}u^2(t-s)}\quad(u\in\R)
            \end{align*}
        \end{enumerate}
    \end{enumerate}
\end{proof}

\begin{lemma}[コンパクト開位相の性質]
    $K\compsub\R_*,U\osub\R$について,
    \[W(K,U):=\Brace{f\in\Om| f(K)\subset U}\]
    を準基\footnote{一般にコンパクト集合の有限共通部分はコンパクトとは限らないことに注意.}として生成される$\Om$上の位相を\textbf{コンパクト開位相}といい,次が成り立つ.
    \begin{enumerate}
        \item $e:\Om\times\R_+\to\R$は連続.特に$\pr_t:\Om\to\R\;(t\in\R_+)$は連続(逆は言えないことに注意).
        \item $K_n:=[0,n]$として,半ノルムの族$(\rho_n(\om):=\sup_{t\in K_n}\abs{\om(t)})_{n\in\N}$が定める始位相と一致する.
    \end{enumerate}
\end{lemma}

\begin{lemma}[Wiener測度の特徴付け]
    $(B_t)$を$(\Om',\F',P')$上のBrown運動とする.
    このとき,$B:\Om'\to\Om;\om'\mapsto B(-,\om')$とするとこれは可測で,
    命題\ref{prop-construction-of-Wiener-measure}(2)で定まる$\Om$上の測度$P$に対して$P'^B=P\;\on\F$を満たす.
    特に,次の変数変換の公式を得る:
    \[\forall_{\varphi\in\L^1(\Om,\F,P)}\quad\int_\Om\varphi(\om)P(d\om)=\int_{\Om'}\varphi(B(-,\om'))P'(d\om').\]
\end{lemma}
\begin{proof}
    可測性は次の議論より明らか.

    任意の$k\in\N,0\le t_1<\cdots<t_k,A_1,\cdots,A_k\in\B^1(\R)$について,$C(t_1,\cdots,t_k;A_1,\cdots,A_k)$上での値が一致すればよいが,Brown運動の独立増分性と,増分が正規分布に従うことより,
    \begin{align*}
        P'^B(C(t_1,\cdots,t_k;A_1,\cdots,A_k))&=P'[B_{t_1}\in A_1,\cdots,B_{t_k}\in A_k]\\
        &=\int_{A_1\times\cdots\times A_k}p_{t_1}(x_1)p_{t_2-t_1}(x_2-x_1)\cdots p_{t_k-t_{k-1}}(x_k-x_{k-1})dx_1\cdots dx_k\\
        &=P[C(t_1,\cdots,t_k;A_1,\cdots,A_k)].
    \end{align*}
\end{proof}

\subsection{Cameron-Matrin定理}

\begin{tcolorbox}[colframe=ForestGreen, colback=ForestGreen!10!white,breakable,colbacktitle=ForestGreen!40!white,coltitle=black,fonttitle=\bfseries\sffamily,
title=]
    ドリフト$F:\R_+\to\R$付きのBrown運動を調べることは,標準Brown運動を調べることに等しい.
    これはちょうど$F$-Brown橋と標準Brown橋の関係と同じである.
    これは,Banach空間$\Om$上のGauss測度$\mu$の,平行移動に対する挙動を調べていることになる.

    この消息を支える$\Om$の稠密部分空間が,普遍的な役割を果たす.
\end{tcolorbox}

\subsubsection{Cameron-Martin部分空間}

\begin{tcolorbox}[colframe=ForestGreen, colback=ForestGreen!10!white,breakable,colbacktitle=ForestGreen!40!white,coltitle=black,fonttitle=\bfseries\sffamily,
title=]
    $F(0)=0$を満たす
    見本道
    $F\in C([0,1])$のうち,
    ある$F'\in L^2([0,1])$の積分として得られるものの全体をCameron-Martin部分空間という.
\end{tcolorbox}

\begin{notation}[skelton / Cameron-Martin subspace]\mbox{}
    \begin{enumerate}
        \item 標準Brown運動の法則を$L_0$,ドリフト$F$付きのBrown運動の法則を$L_F$で表す.
        $L_F\ll L_0\Leftrightarrow L_0(A)=0\Rightarrow L_F(A)=0$がいかなるときか?
        まず,$F$が連続で,$F(0)=0$が必要なのは明らかである.そうでなければ,$L_F$は連続でなかったり,$0$からスタートしない見本道に正の確率を許してしまうため.
        \item $D[0,1]:=\Brace{F\in C[0,1]\mid\exists_{f\in L^2[0,1]}\;\forall_{t\in[0,1]}\;F(t)=\int^t_0f(s)ds}$をDirichlet空間という.より一般にDirichlet空間は,Dirichlet積分なる半ノルムが有限となるようなHardy空間の部分空間で,正則関数の再生核Hilbert空間ともなる.
        このDirichlet空間には特別な名前がついており,skeltonまたはCameron-Martin部分空間と呼ばれる.
    \end{enumerate}
\end{notation}

\begin{definition}[Dirichlet space]
    $\Om\subset\C$上のDirichlet空間とは,正則関数の再生核Hilbert空間がなす$H^2(\Om)$の部分空間で,
    \[\D(\Om):=\Brace{f\in H^2(\Om)\mid \D(f)<\infty}\quad\D(f):=\frac{1}{\pi}\iint_\Om\abs{f'(z)}^2dA=\frac{1}{4\pi}\iint_\Om\abs{\partial_xf}^2+\abs{\partial_yf}^2dxdy.\]
    この右辺が変分原理であるDirichlet原理を定めるDirichlet積分になっていることから名前がついた.
\end{definition}
\begin{definition}[RKHS: reproductive kernel Hilbert space]
    $X$を集合,$H\subset\Map(X;\R)$をHilbert空間とする.
    評価関数$\ev_x:H\to\R$が任意の$x\in X$について有界線型汎関数である$\forall_{x\in X}\;\ev_x\in B(H)$とき,$H$を\textbf{再生核Hilbert空間}という.
    すなわち,Rieszの表現定理よりある$K_x\in H$が存在して$\ev_x(-)=(-|K_x)$と表せる.この対応$X\to H;x\mapsto K_x$が導く双線型形式$K:X\times X\to\R;(x,y)\mapsto K(x,y):=(K_x|K_y)$を\textbf{再生核}という.
    再生核は対称で半正定値である.
\end{definition}
\begin{theorem}
    関数$K:X\times X\to\R$は対称かつ半正定値であるとする.このとき,ただ一つのHilbert空間$H$が$\Map(X;\R)$内に存在して,$K$を再生核として持つ.
\end{theorem}

\subsubsection{Cameron-Martin定理}

\begin{tcolorbox}[colframe=ForestGreen, colback=ForestGreen!10!white,breakable,colbacktitle=ForestGreen!40!white,coltitle=black,fonttitle=\bfseries\sffamily,
title=]
    $L_F\ll L_0$ならば,標準Brown運動$B$のa.s.性質は$B+F$に引き継がれる.
    これはちょうどCameron-Martin部分空間について$F\in D[0,1]$に同値.
    したがって微分を見てみると良い.
\end{tcolorbox}

\begin{definition}[equivalence measure, absolute continuous, singular]
    可測空間$(\Om,\B)$上の測度を考える.
    \begin{enumerate}
        \item 2つの測度の零集合の全体が一致するとき,すなわち互いに絶対連続であるとき$\mu\ll\nu\land\nu\ll\mu$,これらは同値であるという.
        \item 絶対連続性は,測度の同値類の間に順序を定める.
        \item 2つの測度の「台」が分解出来るとき,すなわち,ある分割$\Om=A\sqcup A^\comp$が存在して$P(A)\cap\B\subset\cN(\mu)$かつ$P(A^\comp)\cap\B\subset\cN(\nu)$が成り立つとき,特異であるといい$\mu\perp\nu$と表す.
        \item Lebesgue分解により,任意の2つの$\sigma$-有限測度$\mu,\nu$について,一方をもう一方に対して絶対連続部分と特異部分との和に分解できる.
    \end{enumerate}
\end{definition}

\begin{theorem}[Cameron-Martin]
    任意の連続関数$F\in C[0,1],F(0)=0$について,次が成り立つ:
    \begin{enumerate}
        \item $F\notin D[0,1]$ならば,$L_F\perp L_0$である.
        \item $F\in D[0,1]$ならば,$L_F$と$L_0$は同値である.
    \end{enumerate}
\end{theorem}
\begin{remarks}
    可微分なドリフト$F$については,Brown運動の法則は同値になる.
    すなわち,ほとんど確実な事象(見本道の連続性や可微分性)が等しい.
\end{remarks}

\subsubsection{証明}

\begin{tcolorbox}[colframe=ForestGreen, colback=ForestGreen!10!white,breakable,colbacktitle=ForestGreen!40!white,coltitle=black,fonttitle=\bfseries\sffamily,
title=]
    証明にはmartingaleを用いる.
\end{tcolorbox}

\begin{notation}
    $F\in C[0,1],n>0$について,2進小数点$\D_n$で分割して考えた,
    $F$の二次変分
    \[Q(F):=\lim_{\abs{\Delta}\to0}\sum_{k=1}^n\paren{F(t_k)-F(t_{k-1})}^2\]
    に至る列を
    \[Q_n(F):=2^n\sum^{2^n}_{j=1}\Square{F\paren{\frac{j}{2^n}}-F\paren{\frac{j-1}{2^n}}}^2\]
    と表す.
\end{notation}

\begin{lemma}
    $F\in C[0,1],F(0)=0$について,
    \begin{enumerate}
        \item $(Q_n(F))_{n\in\N}$は増加列である.
        \item $F\in D[0,1]\Leftrightarrow\sup_{n\in\N}Q_n(F)<\infty$.
        \item $F\in D[0,1]\Rightarrow Q_n(F)\xrightarrow{n\to\infty}\norm{F'}^2_2$.
    \end{enumerate}
\end{lemma}
\begin{proof}\mbox{}
    \begin{enumerate}
        \item $(a+b)^2=a^2+2ab+b^2\le 2a^2+2b^2$より,任意の$j\in[2^n]$について,
        \begin{align*}
            \Square{F\paren{\frac{j}{2^n}}-F\paren{\frac{j-1}{2^n}}}^2&=\Square{\paren{F\paren{\frac{2j-1}{2^{n+1}}}-F\paren{\frac{j-1}{2^n}}}+\paren{F\paren{\frac{j}{2^n}}-F\paren{\frac{2j-1}{2^{n+1}}}}}^2\\
            &\le2\Square{F\paren{\frac{2j-1}{2^{n+1}}}-F\paren{\frac{j-1}{2^n}}}^2+2\Square{F\paren{\frac{j}{2^n}}-F\paren{\frac{2j-1}{2^{n+1}}}}^2
        \end{align*}
        これを$j\in[2^n]$について和を取ると,
        \[\sum^{2^n}_{j=1}\Square{F\paren{\frac{j}{2^n}}-F\paren{\frac{j-1}{2^n}}}^2\le 2\sum^{2^{n+1}}_{j=1}\Square{F\paren{\frac{j}{2^{n+1}}}-F\paren{\frac{j-1}{2^{n+1}}}}^2\]
        より,$(Q_n(F))_{n\in\N}$の単調増加性を得る.
        \item \begin{description}
            \item[$\Rightarrow$] $F\in D[0,1]$のとき,ある$f:=F'\in L^2[0,1]$が存在して,Cauchy-Schwartzより任意の$j\in[2^n]$について
            \[2^n\paren{\int^{j2^{-n}}_{(j-1)2^{-n}}fdt}^2\le 2^n\int^{j2^{-n}}_{(j-1)2^{-n}}1^2dt\int^{j2^{-n}}_{(j-1)2^{-n}}f^2dt=\int^{j2^{-n}}_{(j-1)2^{-n}}f^2dt\]
            が成り立つから,
            \[Q_n(F)=2^n\sum^{2^n}_{j=1}\paren{\int^{j2^{-n}}_{(j-1)2^{-n}}fdt}^2\le \sum^{2^n}_{j=1}\int^{j2^{-n}}_{(j-1)2^{-n}}f^2dt=\norm{f}_2^2<\infty.\]
            \item[$\Leftarrow$]

        \end{description}
    \end{enumerate}
\end{proof}

\begin{lemma}[Paley-Wiener stochastic integral]
    $(B_t)_{t\in\R_+}$を標準Brown運動とする.$F\in D[0,1]$について,
    \[\xi_n:=2^n\sum^{2^n}_{j=1}\Square{F\paren{\frac{j}{2^n}}-F\paren{\frac{j-1}{2}}}\Square{B\paren{\frac{j}{2^n}}-B\paren{\frac{j-1}{2^n}}}\]
    はほとんど確実に$L^2$-収束する.
    この極限を$\int^1_0F'dB$で表す.
\end{lemma}

\subsection{抽象的Wiener空間}

\begin{tcolorbox}[colframe=ForestGreen, colback=ForestGreen!10!white,breakable,colbacktitle=ForestGreen!40!white,coltitle=black,fonttitle=\bfseries\sffamily,
title=]
    Leonard Gross 31- により,一般のGauss測度を調べるために開発され,見本道の空間$\Om$よりもCameron-Martin空間を主役に置く.
    可分Hilbert空間$H$上の経路積分など,うまくGauss測度が定義できないことがあるが,これは$H$を稠密部分空間として含むBanach空間$B$を得て,その上での積分として理解できる.
    この組$(H,B)$を抽象Wiener空間という.
    $B$上にGauss測度が存在するかやその性質など,多くの部分は$H$で決まる.

    Gauss測度の構造定理によると,任意のGauss測度はある抽象Wiener空間上に実現される.
\end{tcolorbox}

\begin{example}[古典的Wiener空間]
    $H$をホワイトノイズ全体の集合,すなわち,$b'(0)=0$を満たす$L^2$-導関数$b'$を持つ実関数$b:[0,T]\to\R$にノルム$\norm{b}:=\int_{[0,T]}b'^2(t)dt$を入れて得るHilbert空間とする:
    \[H:=L^{2,1}_0(T;\R^n):=\Brace{b:[0,T]\to\R\mid 0\text{から始まる見本道}b\text{は絶対連続で微分}b'\text{は}L^2\text{-可積分である}}\]
    このとき,$B:=\Brace{W\in C(T)\mid W(0)=0}$を見本道の空間とすれば,$(H,B)$は古典的Wiener空間である.
    $B$上のGauss測度はBrown運動の法則に一致し,Cameron-Matrin部分空間$H$を零集合に持つ.
    実際,$H$の元であるような見本道(特に微分可能である)は,Brown運動の見本道としては出現しない.
    逆に言えば,$H$上にうまく積分を定義できれば,$B$上に連続延長するかもしれない.
    これは測度論・函数解析と確率論の奇妙な融合である!
\end{example}

\begin{discussion}
    $H$を可分Hilbert空間とする.$H$上の柱状集合を,有限個の有界線型汎函数$\varphi_1,\cdots,\varphi_n\in H^*$を用いて,
    \[\Brace{h\in H\mid\varphi_1(h)\in I_1,\cdots,\varphi_n(h)\in I_n}\;(I_1,\cdots,I_n\in\B^1(\R))\]
    と表せる集合の全体$\cC$とする.これはある$\sigma$-代数$\sigma[\cC]$を生成する.
    $(H,\cC)$上にGaussな有限加法的測度は上述の方法で定まるが(これを\textbf{柱状集合Gauss測度}という.),これは$\sigma[\cC]$上に完全加法的な延長を持たない.

    ここで,可分Banach空間$B$で,連続線型単射$i:H\mono B$で像が$B$上稠密であるようなものが存在するとする.
    $B$の柱状集合$C$を,$i^*B\subset H$が柱状集合であることとして定義する.
    Gross (67)は,$B$の柱状集合体$\cC$上のGaussな有限加法的測度が完全加法的な延長を持つための必要十分条件を導いた.
    これは$B$が$H$に比べて「十分に大きい」ことと理解できる.
    このとき,$H$は零集合になる($H$にGauss測度が存在しないことと整合的).
\end{discussion}

\begin{theorem}[Structure theorem for Gaussian measures (Kallianpur–Sato–Stefan 1969 and Dudley–Feldman–le Cam 1971)]\label{thm-Structure-theorem-for-Gaussian-measures}
    任意の可分Banach空間$B$上の非退化なGauss測度$\mu$とする.このとき,抽象Wiener空間$i:H\mono B$が存在して,$\mu$は$H$上の柱状集合Gauss測度の押し出しによって得られる.
\end{theorem}
\begin{remarks}
    このときの$B$上のGauss測度$\mu$は,任意の汎関数$f\in B^*$に対して,$f_*\mu$は1次元正規分布である.
\end{remarks}

\section{Brownian Filtration}

\begin{tcolorbox}[colframe=ForestGreen, colback=ForestGreen!10!white,breakable,colbacktitle=ForestGreen!40!white,coltitle=black,fonttitle=\bfseries\sffamily,
title=]
    確率変数の族$(B_t)_{t\in[0,T]}$と$B_s-B_T\;(s>T)$は独立である.
    これは,任意の有限部分集合$\{0\le t_1<\cdots<t_n\}\subset[0,T]$に対して,確率ベクトル$(B_{t_1},\cdots,B_{t_n})$と変数$B_S-B_T$とが独立であることをいう.
    これを,$\F_T\indep B_s-B_T$と表現出来る.
    これを示すためには,再び柱状集合をうまく取ることが大事な技法となる.
\end{tcolorbox}

\begin{definition}[natural filtration]
    $B$を$(\Om,\F,P)$上のBrown運動とする.
    \begin{enumerate}
        \item $\F_t\subset\F\;(t\in\R_+)$を,確率変数族$\{B_s\}_{s\in[0,t]}$と零集合$\cN(P)$が生成する完備な$\sigma$-加法族とする:$\F_t:=\sigma[\cN(P),\B_s|s\in[0,t]]$.
        \item $(\Om,\F,P)$が古典的Wiener空間であるとき,$\F_0=\{\emptyset,\Om\}$であり,$\F_t$は時刻$t$までの柱状集合
        \[C(t_1,\cdots,t_n;A_1,\cdots,A_n)\;(t_n\le t)\]
        が生成する$\sigma$-加法族に一致する.
        \item このときの閉$\sigma$-代数の増大系$\{\F_t\}_{t\in\R_+}$を\textbf{自然な情報系}という.
    \end{enumerate}
\end{definition}

\begin{lemma}[自然な情報系の特徴付け]\label{lemma-Blumenthal}
    $(B_t)$を任意のBrown運動で,確率空間$(\Om,\F,P)$上に定義されているとする.
    このBrown運動の自然な情報系$(\F_t)$について,ここでは$\F_t=\sigma[B_s|s\le t]$として必ずしもすべての$P$-零集合を含まないとして扱う.
    必要に応じて完備化すれば良い.
    \begin{enumerate}
        \item (左連続性) $\F_t$は柱状集合$C(t_1,\cdots,t_n;I_1,\cdots,I_n)\;(0\le t_1<\cdots<t_n\le t)$の全体によって生成される.また,$t_n<t$としても変わらない.
        \item ($0$における右連続性) また$\F$はBrown運動の族$(B_{t+h}-B_h)_{t\in\R_+,h>0}$の柱状集合$D(t_1,\cdots,t_n;h;I_1,\cdots,I_n)\;(h>0)$の全体によっても生成される:
        \[D(t_1,\cdots,t_n;h;I_1,\cdots,I_n):=\Brace{\om\in\Om\mid B_{t_1+h}(\om)-B_{h}(\om)\in I_1,\cdots,B_{t_n+h}(\om)-B_h(\om)\in I_n}\]
        \item ($B_s\indep B_t-B_s$の一般化) 任意の$0\le s<t$について,$\F_s$は$B_t-B_s$に対して独立である.したがって特に,任意の$\F_s$-可測関数$X:\Om\to\R$は$B_t-B_s$と独立である.
        \item (Blumenthal one-zero law) $A\in\F_0^+:=\cap_{\ep>0}\F_\ep$ならば,$P[A]\in\{0,1\}$.
    \end{enumerate}
\end{lemma}
\begin{proof}\mbox{}
    \begin{enumerate}
        \item $\F_t$は任意の有限列$I_1,\cdots,I_n\in\B^1(\R)$と任意の有限列$0\le s_1<\cdots<s_n\le t$を用いて,$C(s_1,\cdots,s_n;I_1,\cdots,I_n)$によって生成されるということである.
        また,$t_n<t$を満たす柱状集合の全体を用いても,見本道の連続性より,任意の開区間$I\osub\R$について,$\Brace{B_t\in I}=\limsup_{n\to\infty}\Brace{B_{t-1/n}\in I}$と表せる.
        \item $n\in\N,0\le t_1<\cdots<t_n,h>0,I_i\in\B^1(\R)$に対して,
        \[D(t_1,\cdots,t_n;h;I_1,\cdots,I_n):=\Brace{\om\in\Om\mid B_{t_1+h}(\om)-B_{h}(\om)\in I_1,\cdots,B_{t_n+h}(\om)-B_h(\om)\in I_n}\]
        なる形の集合全体が生成する$\sigma$-代数を$\cG$とすると,実は$\cG=\F$である.
        \begin{enumerate}[(a)]
            \item $\F\subset\cG$は,任意の柱状集合$C(s_1,\cdots,s_n;A_1,\cdots,A_n)$は,$\cap_{i\in[n]}C(s_i;A_i)$と表せるから,
            $C(s;A)\;(s\ge0,A\in\B^1(\R))$が$\cG$に入っていることを示せば良い.また$A$として区間$(\infty,a)\;(a\in\R)$を取ると,
            \[\Brace{B_s<a}\overset{\as}{=}\Brace{\lim_{n\to\infty}(B_{s+1/n}-B_{1/n})<a}=\limsup_{n\to\infty}\Brace{B_{s+1/n}-B_{1/n}<a}\in\cG\]
            \item $\cG\subset\F$は,$(B'_s:=B_{s+h}-B_h)_{s\in\R_+}$は再びBrown運動であることより,$D(t_1;h;I_i)=\Brace{B'_{t_1}\in I_1}\in\sigma[B'_{t_1}]\subset\F$.
        \end{enumerate}
        \item $\F_s$は,$C(s_1,\cdots,s_n;J_1,\cdots,J_n)\;(0\le s_1<\cdots<s_n\le s,J_i\in\B^1(\R))$の形の柱状集合がなす集合体によって生成されるから,Dynkin族定理より,
        \[P[(B_t-B_s\in I),C(s_1,\cdots,s_n;J_1,\cdots,J_n)]=P[B_t-B_s\in I]P[C(s_1,\cdots,s_n;J_1,\cdots,J_n)]\]
        を示せば,残りの等式は単調収束定理より従う.

        これは,%2つの確率ベクトル$(B_{s_1},\cdots,B_{s_n}),B_t-B_s$の独立性から従う.
        %任意の$i\in[n]$に対して,$\Cov[B_{s_i},B_t-B_s]=\Cov[B_{s_i},B_t]-\Cov[B_{s_i},B_s]=0$を示せば良いが,これは明らか.
        %というのも,
        左辺は
        \[\int_{J_1\times\cdots\times J_n\times I}p_{s_1}(x_1)p_{s_2}(x_2-x_1)\cdots p_{s_n}(x_n-x_{n-1})p_{t-s}(x)dx_1\cdots dx_ndx\]
        と表せるが,これはFubiniの定理より右辺に等しい.
        \item 
        $A\in\F^+_0$を任意に取る.
        \begin{description}
            \item[$A\indep\cG$] $A$は$\F^+_0$の元であるから,特に$A\in\F_h$である.
            よって,任意の有限部分集合$\{0\le t_1<\cdots<t_n\}\subset\R_+$について,
            各$B_{t_i+h}-B_{h}$と独立であり,
            したがって$D(t_1,\cdots,t_n;h;I_1,\cdots,I_n)$とも独立である.
            $t_1,\cdots,t_n,h,I_1,\cdots,I_n$は任意に取ったから,$A\indep\cG$(Dynkin族定理と単調収束定理による).
            すなわち,$\forall_{G\in\cG}\;P[A\cap G]=P[A]P[G]$.
            \item[結論] 
            $\F^+_0\subset\F\subset\cG$より,$G=A$と取れる.よって,$P[A]=P[A]P[A]$を得る.
        \end{description}
    \end{enumerate}
\end{proof}

\begin{lemma}[自然な情報系の右連続性]
    $\F_t$が完備であることより,次の2条件が成り立つ.
    \begin{enumerate}
        \item $(\F_t)$-適合的過程の任意のバージョンは適合的である.
        \item $(\F_t)$は右連続である:$\forall_{t\in\R_+}\cap_{s>t}\F_s=\F_t$.
    \end{enumerate}
\end{lemma}
\begin{proof}\mbox{}
    \begin{enumerate}
        \item $(X_t)_{t\in\R_+}$を$(\Om,\F,P)$上の$(\F_t)$-適合的な確率過程とする:$\forall_{t\in\R_+}\;X_t\in\L_{\F_t}(\Om)$.
        すなわち,$\forall_{t\in\R_+}\;\forall_{A\in\B^1(\R)}\;\Brace{X_t\in A}\in\F_t$.
        $Y$を$X$の同値な確率過程とすると,$\forall_{t\in\R_+}\;P^{X_t}=P^{Y_t}$より,
        特に$P[X_t\in A]=P[Y_t\in A]$が成り立つ.
        よって$Y$
        も$(\F_t)$-適合的になる.
        \item 
        任意の$t\in\R_+$を取る.$t=0$のときは補題(2)より成り立つ.

        $t>0$のときは,$(B_{h+t}-B_t)_{h\in\R_+}$は再びBrown運動であるから,このBrown運動についての自然な情報系を$(\cG_h)$とすると,
        $\cG_0$の元はすべてfull setかnull setであるから,$\cG_0\subset\F_t$,特に
        $\sigma[\F_t,\cG_0]=\F_t$であることより,
        \[\F_t=\sigma[\F_t,\cG_0]=\cap_{\ep\ge0}\sigma[\F_t,\cG_\ep]=\cap_{\ep>0}\sigma[\F_t,\cG_\ep].\]
    \end{enumerate}
\end{proof}

\section{Markov性}

\subsection{Markov性}

\begin{theorem}[Markov property]
    Brown運動$(B_t)_{t\in\R_+}$は自然な情報系$(\F_t)$についてMarkov過程となる.
    すなわち,
    \[(P_tf)(x)=\int_\R f(y)p_t(x-y)dy\;(t>0)\]
    で,時刻$s$に$x$に居た場合の$f(B_{s+t})$の値の平均を返す関数$P_tf:\R\to\R$への対応$P_t:L_b(\R)\to\L(\R)$
    のなす1-パラメータ連続半群$(P_t)_{t\in\R_+}$を用いて,
    \[\forall_{f\in\L(\R)\cap l^\infty(\R)}\;\forall_{s\ge0,t>0}\;E[f(B_{s+t})|\F_s]=(P_tf)(B_s).\]
    また,$f\in\L(\R)_+$としてもよい.
\end{theorem}
\begin{proof}
    補題\ref{lemma-Blumenthal}(1)より
    $\F_s\indep B_{s+t}-B_s$だから,$\forall_{x\in\R}\;f(B_{s+t}-B_s+x)\indep\F_s$だから,
    \begin{align*}
        E[f(B_{s+t})|\F_s]&=E[f(B_{s+t}-B_s+B_s)|\F_s]\\
        &=E[f(B_{s+t}-B_s+x)|\F_s]|_{x=B_s}\\
        &=E[f(B_{s+t}-B_s+x)]|_{x=B_s}\\
        &=\int_\R f(y+x)\frac{1}{\sqrt{2\pi t}}e^{-\frac{y^2}{2t}}dy|_{x=B_s}\\
        &=\int_\R f(y+B_s)\frac{1}{\sqrt{2\pi t}}e^{-\frac{y^2}{2t}}dy\\
        &=\int_\R f(y)\frac{1}{\sqrt{2\pi t}}e^{-\frac{(B_s-y)^2}{2t}}dy=(P_tf)(B_s).
    \end{align*}
\end{proof}

\subsection{Markov半群}

\begin{definition}[transition semigroup, Chapman-Kolmogorov equation]
    定理に登場した
    $P_t\varphi(x):=E[\varphi(B_t+x)]\;(\varphi\in \L_b(\R))$
    で定まる$\{P_t\}_{t\in\R_+}\subset\Map(\L_b(\R),\L(\R))$は半群の準同型となる:
    \begin{enumerate}
        \item $P_0=\id$.
        \item $P_t\circ P_s=P_{t+s}$,すなわち,
        \[\forall_{f\in L_b(\R)}\;\iint_{\R\times\R} f(y)p_s(x-y)p_t(z-x)dydx=\int_\R f(y)p_{s+t}(z-y)dy.\]
    \end{enumerate}
    これを\textbf{遷移半群}という.
    
    また,$P_t$は正作用素である:$\varphi\ge0\Rightarrow P_t\varphi\ge0$.
\end{definition}


\begin{proposition}[Brown運動の遷移半群が満たす偏微分方程式]
    $d$-次元Brown運動の遷移確率のなす
    連続半群$(P_t)_{t\in\R_+}$を定める積分核
    \[p_t(x-y)=\frac{1}{(2\pi t)^{d/2}}\exp\paren{-\frac{\abs{x-y}^2}{2t}}\]
    は,熱方程式の初期値問題
    \[\pp{p}{t}=\frac{1}{2}\Laplace p\;(t>0)\qquad p_0(x-y)=\delta_0(x-y)\]
    の解である.
\end{proposition}

\section{Brown運動に付随するmartingale}

\subsection{代表的なマルチンゲール}

\begin{theorem}
    次の過程は$(\F_t)$-マルチンゲールである.
    \begin{enumerate}
        \item Brown運動:$(B_t)_{t\in\R_+}$.
        \item $(B^2_t-t)_{t\in\R_+}$.
        \item 幾何Brown運動:$(\exp(aB_t-a^2t/2))_{t\in\R_+}\;(a\in\R)$.
        \item 複素幾何Brown運動:$(\exp(iaB_t+a^2t/2))_{t\in\R_+}\;(a\in\R)$.
    \end{enumerate}
\end{theorem}
\begin{proof}\mbox{}
    \begin{enumerate}
        \item $\forall_{0\le s<t}\;B_t-B_s\indep\F_s$補題\ref{lemma-Blumenthal}(1)より,$E[B_t-B_s|\F_s]=E[B_t-B_s]=0$.
        \item 任意の$0\le s<t$について,
        \begin{align*}
            E[B^2_t|\F_t]&=E[(B_t-B_s+B_s)^2|\F_s]\\
            &=E[(B_t-B_s)^2|\F_s]+2E[(B_t-B_s)B_s|\F_s]+E[B^2_s|\F_s]\\
            &=E[(B_t-B_s)^2]+2B_sE[B_t-B_s|\F_s]+B^2_s\\
            &=t-s+B_2^s.
        \end{align*}
        \item 任意の$0\le s<t$について,$B_t-B_s\sim N(0,t-s)$なので,
        \begin{align*}
            E\Square{\exp\paren{aB_t-\frac{a^2t}{2}}\middle|\F_s}&=\exp(aB_s)E\Square{\exp\paren{a(B_t-B_s)-\frac{a^2t}{2}}}\\
            &=\exp\paren{aB_s-\frac{a^2t}{2}}\frac{1}{\sqrt{2\pi(t-s)}}\int_\R e^{ax}e^{-\frac{x^2}{2(t-s)}}dx\\
            &=\exp\paren{aB_s-\frac{a^2t}{2}}e^{\frac{(t-s)^2a^2}{2(t-s)}}\frac{1}{\sqrt{2\pi(t-s)}}\int_\R e^{-\frac{(x-(t-s)a)^2}{2(t-s)}}dx\\
            &=\exp\paren{aB_s-\frac{sa^2}{2}}.
        \end{align*}
        \item 任意の$0\le s<t$について,$B_t-B_s\sim N(0,t-s)$の特性関数より,
        \begin{align*}
            E\Square{\exp\paren{iaB_t+\frac{a^2t}{2}}\middle|\F_s}&=\exp\paren{iaB_s+\frac{a^2t}{2}}E\Square{\exp\paren{ia(B_t-B_s)}}\\
            &=\exp\paren{iaB_s+\frac{a^2t}{2}}\exp\paren{-\frac{1}{2}a^2(t-s)}=\exp\paren{iaB_s+\frac{a^2s}{2}}.
        \end{align*}
    \end{enumerate}
\end{proof}
\begin{remarks}
    (3)は正規分布の積率母関数とも,対数正規分布とも見れる.
\end{remarks}

\subsection{到達時刻}

\begin{tcolorbox}[colframe=ForestGreen, colback=ForestGreen!10!white,breakable,colbacktitle=ForestGreen!40!white,coltitle=black,fonttitle=\bfseries\sffamily,
title=Markov過程の到達時刻はLaplace変換の有名な応用先である]
    Brown運動とそれに付随するマルチンゲールを用いて,到達時刻$\tau_a$のLaplace変換を得ることが出来る.
    これは積率母関数を得たことになるから,Laplace逆変換によって$\tau_a$の分布が分かる.
\end{tcolorbox}

\begin{lemma}[到達時刻]
    $a>0$に対して
    $\tau_a:=\inf\Brace{t\in\R_+\mid B_t=a}$とすると,これは$(\F_t)$-停止時である.
    ただし,$\Brace{t\in\R_+\mid B_t=a}=\emptyset$のとき,$\tau_a=\infty$とする.
\end{lemma}
\begin{proof}
    \begin{align*}
        \Brace{\tau_a>t}&=\cap_{s\in[0,t]}\Brace{B_s<a}\\
        &=\cup_{k\in\N}\cap_{s\in[0,t]}\Brace{B_s\le a-\frac{1}{k}}\overset{\as}{=}\cup_{k\in\N}\cap_{s\in[0,t]\cap\Q}\Brace{B_s\le a-\frac{1}{k}}\in\F_t.
    \end{align*}
    最後の等号は,$\subset$は明らかであるが,$\supset$は,殆ど至る所の$\om\in\Om$に対して$s\mapsto B_s(\om)$が連続であるため,$\exists_{k\in\N}\;\forall_{s\in[0,t]\cap\Q}\;B_s\le a-\frac{1}{k}$は$[0,t]$上でも同様の条件が成り立つことを含意する.
    よって,$\Brace{\tau_a\le t}\in\F_t$もわかる.
\end{proof}
\begin{remarks}
    $\Brace{\tau_a<t}$を示しても十分であるが,これは情報系$(\F_t)$が右連続であることに依る.
\end{remarks}

\begin{proposition}[Laplace transformation of Brownian hitting time]\label{prop-hitting-time}
    任意の$a>0$について,
    \[\forall_{\al>0}\;E[e^{-\al\tau_a}]=e^{-\sqrt{2\al}a}.\]
\end{proposition}
\begin{proof}
    幾何Brown運動のmartingale性に注目することで,これだけから出てくる.
    \begin{description}
        \item[martingaleの用意] \mbox{}\\
        任意の$\lambda>0$について,$M_t:=e^{\lambda B_t-\lambda^2t/2}$はmartingaleである.\footnote{$\lambda=0$のときは$M_t=1$となり,この場合は不適.考えない.}
        よって特に$\forall_{t\in\R_+}\;E[M_t]=E[M_0]=1$.
        \item[この幾何Brown運動についての任意抽出] \mbox{}\\
        Doobの任意抽出定理より,任意の$N\ge 1$について,$E[M_{\tau_a\land N}|\F_0]=M_0$.両辺の期待値を取って,$E[M_{\tau_a\land N}]=1$.
        \begin{enumerate}[(a)]
            \item (関数列の優関数)  $\forall_{N\in\N}\;M_{\tau_a\land N}=\exp\paren{\lambda B_{\tau_a\land N}-\lambda^2(\tau_a\land N)/2}\le e^{a\lambda}$より,$N\to\infty$の極限について優収束定理が使える.
            \item (関数列の極限) $e^{t\paren{\lambda\frac{B_t}{t}-\frac{\lambda^2}{2}}}\xrightarrow{t\to\infty}e^{-\infty}=0$を示す.
            いま,大数の法則より\[\lambda\frac{B_t}{t}-\frac{\lambda^2}{2}\xrightarrow{t\to\infty}\frac{-\lambda^2}{2}<0\;(\lambda>0\text{のとき})\]だから,
            十分大きな$T>0$について,$\forall_{t\ge T}\;\lambda\frac{B_t}{t}-\frac{\lambda^2}{2}=:-\ep<0$より,$0\le M_t=e^{t\paren{\lambda\frac{B_t}{t}-\frac{\lambda^2}{2}}}\le e^{-\ep T}$.
            
            $T\to\infty$の極限を考えると,$M_t\xrightarrow{t\to\infty}0$.
            よって,
            \[\lim_{N\to\infty}M_{\tau_a\land N}=\begin{cases}
                M_{\tau_a}&\tau_a<\infty,\\
                0&\tau_a=\infty.
            \end{cases}\]
        \end{enumerate}
        よって優収束定理から,結局$E[M_{\tau_a}]=1$とわかる.特に,
        $E[1_{\Brace{\tau_a<\infty}}M_{t_a}]=1-E[1_{\Brace{\tau_a=\infty}}M_{t_a}]=1$.
        すなわち,
        \[E\Square{1_{\Brace{\tau_a<\infty}}\exp\paren{-\frac{\lambda^2\tau_a}{2}}}=e^{-\lambda a}.\]
        あとは$1_{\Brace{\tau_a<\infty}}$の部分を外せば良い.
        \item[到達時刻はほとんど確実に有限] \mbox{}
        $\lambda>0$は任意とした.$\forall_{\lambda>0}\;1_{\Brace{\tau_a<\infty}}\exp\paren{-\frac{\lambda^2\tau_a}{2}}\le1_{\Brace{\tau_a<\infty}}$より可積分な上界が存在する.
        $\lambda\to0$に極限について,単調収束定理より,$E[1_{\Brace{\tau_a<\infty}}]=P[\tau_a<\infty]=1$.
        よって,
        \[E\Square{\exp\paren{-\frac{\lambda^2\tau_a}{2}}}=e^{-\lambda a}.\]
        $\al=\lambda^2/2$の場合を考えれば良い.
    \end{description}
\end{proof}
\begin{remarks}
    なお,$\tau_a$の確率密度関数を$f:\R_+\to[0,1]$とすると,
    \[E[e^{-\al\tau_a}]=\int_{\R_+}e^{-\al s}f(s)ds=\L[f](\al)\]
    とみれる.
    また$M_t\xrightarrow{t\to\infty}0$はマルチンゲール収束定理からも導ける.
\end{remarks}

\begin{corollary}[到達時刻の分布]\mbox{}
    \begin{enumerate}
        \item Laplace逆変換より,Brown運動の到達時刻の分布関数は,
        \[P[\tau_a\le t]=\int^t_0\frac{ae^{-a^2/(2s)}}{\sqrt{2\pi s^3}}ds.\]
        \item $E[\tau_ae^{-\al\tau_a}]=\frac{ae^{-\sqrt{2\al}a}}{\sqrt{2\al}}$.
        \item $E[\tau_a]=+\infty$.
    \end{enumerate}
\end{corollary}
\begin{proof}\mbox{}
    \begin{enumerate}
        \item Laplace逆変換による.
        \item 両辺の$\al$に関する微分による.$e^{-\al\tau_a}$は$\om$に関して可積分であるのはもちろん,$\al$についても可微分で,またその導関数$-\tau_ae^{-\al\tau_a}$も$\om$に関して可積分だから,微分と積分は交換して良い.
        \item $\al\to0$の極限を考えることによる(単調収束定理).
    \end{enumerate}
\end{proof}
\begin{remarks}
    $\tau_a$の平均は存在しない(発散する)が,殆ど確実に有限になるという不可解な消息がある.
    また,(1)は解析的な方法に依らず,鏡像の原理からも導ける.
\end{remarks}

\subsection{脱出時刻}

\begin{tcolorbox}[colframe=ForestGreen, colback=ForestGreen!10!white,breakable,colbacktitle=ForestGreen!40!white,coltitle=black,fonttitle=\bfseries\sffamily,
title=]
    脱出時刻は到達時刻が定める.
\end{tcolorbox}

\begin{proposition}[どっちの端から脱出するかの確率]
    $a<0<b$のとき,
    \[P[\tau_a<\tau_b]=\frac{b}{b-a}.\]
\end{proposition}
\begin{proof}\mbox{}
    \begin{description}
        \item[任意抽出定理] Doobの任意抽出定理より,$\forall_{t\in\R_+}\;E[B_{t\land\tau_a\land\tau_b}]=E[B_0]=0\;\as$
        $\forall_{t\in\R_+}\;B_{t\land\tau_a\land\tau_b}\in[a,b]$より,優収束定理から$E[B_{\tau_a\land\tau_b}]=0$を得る.

        ここで到達時刻は$\tau_a,\tau_b<\infty\;\as$を満たすから,$\Om=\Brace{\tau_a<\tau_b}\sqcup\Brace{\tau_b<\tau_a}\sqcup\Brace{\tau_b=\tau_a<\infty}\sqcup\Brace{\tau_a\land\tau_b=\infty}$の直和分解のうち最後の2項は零集合だから,
        \[0=P[\tau_a<\tau_b]a+(1-P[\tau_a<\tau_b])b.\]
    \end{description}
\end{proof}

\begin{proposition}[脱出時刻の期待値]\label{prop-mean-of-Brownian-exit-time}
    $a<0<b$のとき,脱出時刻を
    $T:=\inf\Brace{t\in\R_+\mid B_t\notin(a,b)}$とすると$E[T]=-ab$.
\end{proposition}
\begin{proof}
    $B_t^2-t$はmartingaleであるから,任意抽出定理より$\forall_{t\in\R_+}\;E[B^2_{T\land t}]=E[T\land t]$.
    $t\to\infty$の極限を考えて,
    \begin{align*}
        E[T]&=E[B_T^2]=a^2\frac{b}{b-a}+b^2\frac{a}{a-b}\\
        &=\frac{ab(a-b)}{b-a}=-ab.
    \end{align*}
\end{proof}

\section{強Markov性}

\subsection{強Markov性の証明}

\begin{theorem}[Hunt (1956) and Dynkin and Yushkevich (1956)]\mbox{}\\
    $T:\Om\to\R_+$を有限な$(\F_t)$-停止時とする.このとき,$(\wt{B}_t:=B_{T+t}-B_T)_{t\in\R_+}$は再びBrown運動で,$\F_T$と独立である.
\end{theorem}
\begin{proof}\mbox{}
    \begin{description}
        \item[$T$が有界なとき]
        任意の$\lambda\in\R$について,$M_t:=\exp\paren{i\lambda B_t+\frac{\lambda^2t}{2}}$はmartingaleである.
        よってDoobの任意抽出定理より,
        $\forall_{0\le s\le t}\;E[M_{T+t}|\F_{T+s}]=E[M_{T+s}]$から,$\frac{\lambda^2(T+t)}{2},i\lambda B_{T+s}$が$\F_{T+s}$-可測であることに注意すると,
        \begin{align*}
            E\Square{\exp\paren{i\lambda B_{T+t}+\frac{\lambda^2(T+t)}{2}}\middle|\F_{T+s}}&=E\Square{\exp\paren{i\lambda B_{T+s}+\frac{\lambda^2(T+s)}{2}}}\\
            E\Square{\exp\paren{i\lambda(\underbrace{B_{T+t}-B_{T+s}}_{=\wt{B}_t-\wt{B}_s})}\middle|\F_{T+s}}&=E\Square{\exp\paren{-\frac{\lambda^2}{2}(t-s)}}=\exp\paren{-\frac{\lambda^2}{2}(t-s)}.
        \end{align*}
        これは,
        まず任意の$0\le s\le t$について$\wt{B_t}-\wt{B_s}\sim N(0,t-s)$であることを表しており,さらに任意の$0\le t_1\le t_2\le t_3$について,$B_{t_3}-B_{t_2}\indep B_{t_2}-B_{t_1}$も含意している.
        $B_{T+0}-B_T=0$と,$(B_t)_{t\in\R_+}$から受け継いだ連続性とを併せるとBrown運動であることが分かる.
        \item[$T$が一般のとき] \ref{prop-mean-of-Brownian-exit-time}と同様に処理すれば良い.
        
        任意の$N\in\N$に対して,$T\land N$は有界な停止時だから,$\forall_{0\le s\le t}\;T\land N+s\le T\land N+t\le N+t$も有界.
        よって,Doobの任意抽出定理より,
        \[\forall_{0\le s\le t}\quad E\Square{M_{T\land N+t}|\F_{T\land N+s}}=M_{T\land N+s}.\]
        すなわち,
        \[\forall_{N\in\N}\;\forall_{0\le s\le t}\quad E\Square{\exp\paren{i\lambda\paren{B_{T\land N+t}-B_{T\land N+s}}}\;\middle|\;\F_{T\land N+s}}=\exp\paren{-\frac{\lambda^2}{2}(t-s)}.\]

        ここで,
        \[E\Square{\exp\paren{i\lambda(B_{T+t-B_{T+s}})}|\F_{T+s}}=\exp\paren{-\frac{\lambda^2}{2}(t-s)}\]
        を示したい.右辺は明らかに$\F_{T+s}$-可測で可積分であるから,任意の$A\in\F_{T+s}$に対して
        \[E\Square{\exp\paren{-\frac{\lambda^2}{2}(t-s)}1_A}=E\Square{E\Square{\exp\paren{i\lambda(B_{T+t}-B_{T+s})}|\F_{T+s}}1_{A}}\]
        を示せば良い.いま,$A\in\F_{T+s}$より,$A\cap\Brace{T\le N}\in\F_{N+s}$である.特に,$A\cap\Brace{T\le N}\in\F_{T\land N+s}$.
        よって,右辺は
        \begin{align*}
            &E\Square{E\Square{\exp\paren{i\lambda(B_{T+t}-B_{T+s})}|\F_{T+s}}1_{A}}\\
            &=E\Square{E\Square{\exp\paren{i\lambda(B_{T+t}-B_{T+s})}|\F_{T+s}}1_{A\cap\Brace{T\le N}}}+E\Square{E\Square{\exp\paren{i\lambda(B_{T+t}-B_{T+s})}|\F_{T+s}}1_{A\cap\Brace{T>N}}}\\
            &=E\Square{\exp\paren{-\frac{\lambda^2}{2}(t-s)}1_{A\cap\Brace{T\le N}}}+E\Square{E\Square{\exp\paren{i\lambda(B_{T+t}-B_{T+s})}|\F_{T+s}}1_{A\cap\Brace{T>N}}}.
        \end{align*}
        ここで$N\to\infty$の極限を考えると,第1項は単調収束定理,第2項は優収束定理により,総じて右辺は
        \[\exp\paren{-\frac{\lambda^2}{2}(t-s)}\]
        に収束する.
    \end{description}
\end{proof}

\begin{corollary}[時間一様性]
    $(P_t)$をBrown運動に対応する連続半群とすると,
    任意の有界な可測関数$f\in\L(\R)\cap l^\infty(\R)$と有限な停止時$T:\Om\to\R_+$について,
    \[E[f(B_{T+t})|\F_T]=(P_tf)(B_T).\]
\end{corollary}

\subsection{鏡像の原理}

\begin{tcolorbox}[colframe=ForestGreen, colback=ForestGreen!10!white,breakable,colbacktitle=ForestGreen!40!white,coltitle=black,fonttitle=\bfseries\sffamily,
title=]
    確率過程が定めるsupの過程の尾部確率の評価は,
    Kolmogorovの不等式から始まる不等式原理である.
    Brown運動については,元のBrown運動のことばで完全に特徴付けられる.

    この結果は,到達時刻がもたらす強Markov性から従う.
\end{tcolorbox}

\begin{theorem}[Levy's reflection princple (1939)]
    $M_t:=\sup_{s\in[0,t]}B_s$とする.
    \[\forall_{a>0}\;P[M_t\ge a]=2P[B_t>a].\]
\end{theorem}
\begin{proof}
    時刻$\tau_a<\infty$で折り返した確率過程を
    \[\wh{B}_t:=B_t1_{\Brace{t\le\tau_a}}+(2a-B_t)1_{\Brace{t>\tau_a}}\]
    と定める.
    \begin{description}
        \item[$\wh{B}$は再びBrown運動である] いま,到達時刻$\tau_a$は有限な停止時だから,$(B_{t+\tau_a}-a)_{t\in\R_+},(-B_{t+\tau_a}+a)_{t\in\R_+}$はいずれもBrown運動であり,$B_{\tau_a}$と独立.
        
        したがって,$(B_t)_{t\in[0,\tau_a]}$の右端に接続した過程は,いずれも同じ過程を定める.1つ目はBrown運動$(B_t)$で,2つ目が$(\wt{B}_t)$である.
        したがって,$(\wh{B}_t)$もBrown運動である.
        \item[結論]
        $\{M_t\ge a\}=\Brace{B_t>a}\sqcup\Brace{M_t\ge a,B_t\le a}$と場合分けすると,2つ目の場合は$\Brace{\wt{B}_t\ge a}$と同値.
        よって,
        \[P[M_t\ge a]=P[B_t>a]+P[\wt{B}_t\ge a]=2P[B_t>a].\]
    \end{description}
\end{proof}

\begin{corollary}\label{cor-density-of-sup-process-of-Brownian-motion}
    任意の$a>0$に対して,$M_a:=\sup_{t\in[0,a]}B_t$は次の密度関数を持つ:
    \[p(x)=\frac{2}{\sqrt{2\pi a}}e^{-x^2/(2a)}1_{[0,\infty)}(x).\]
\end{corollary}

\subsection{Brown運動の最大値}

\begin{lemma}
    殆ど確実に,Brown運動$(B_t)_{t\in[0,1]}$ただ一つの点にて最大値を取る.
\end{lemma}
\begin{proof}\mbox{}
    \begin{description}
        \item[方針] $[0,1]$の分割
        \[\I_n:=\Brace{\Square{\frac{j-1}{2^n},\frac{j}{2^n}}\in P([0,1])\;\middle|\;1\le j\le 2^n}\]
        について,
        \[G:=\Brace{\om\in\Om\mid\exists_{t_1\ne t_2\in[0,1]}\;\sup_{t\in[0,1]}B_t=B_{t_1}=B_{t_2}}\subset\cup_{n\ge 1}\cup_{I_1,I_2\in\I_n,I_1\cap I_2=\emptyset}\Brace{\sup_{t\in I_1}B_t=\sup_{t\in I_2}B_t}\]
        が成り立つので,$\forall_{n\in\N}\;\forall_{I_1,I_2\in\I_n}\;I_1\cap I_2=\emptyset\Rightarrow P\Square{\sup_{t\in I_1}B_t=\sup_{t\in I_2}B_t}=0$を示せば良い.
        \item[証明] 任意の閉区間$[a,b]\subset[0,1]$について,確率変数$\sup_{t\in[a,b]}B_t:\Om\to\R$の確率分布$P[\sup_{t\in[a,b]}B_t|\F_a]:\Om\to[0,1]$は絶対連続であることを示せば十分である.
        
        $\sup_{t\in[a,b]}B_t=\sup_{t\in[a,b]}(B_t-B_a)+B_a$とみると,
        $\sup_{t\in[a,b]}(B_t-B_a)$を$\F_a$で条件づけたものは,$\sup_{t\in[0,b-a]}B_t$と同じ確率分布を持つ.
        系\ref{cor-density-of-sup-process-of-Brownian-motion}より,これは絶対連続である.
    \end{description}
\end{proof}


\section{Laplace作用素}

\begin{tcolorbox}[colframe=ForestGreen, colback=ForestGreen!10!white,breakable,colbacktitle=ForestGreen!40!white,coltitle=black,fonttitle=\bfseries\sffamily,
title=]
    $P_t[f(x)]=E[f(x+B_t)]$をおくことにより,$\{P_t\}_{t\in\R_+}$は$C_0(\R)$上でHille-吉田の意味での強連続半群となり,
    その生成作用素$\lim_{t\to0}\frac{P_t-I}{t}=\frac{\nabla}{2}$がLaplace作用素である.$I$は恒等作用素とした.
\end{tcolorbox}



\chapter{確率積分}

\begin{quotation}
    確率論ははじめから偏微分方程式論と密接に関係していたことを思うと,確率論自体が測度論で基礎付けられることは自然であった.
    さらに,確率積分なる概念も自然に定義できるはずである.

    Brown運動$B_t$は微分可能ではないし,有界変動にもならないので,Stieltjes積分としては定義できる道はない.
    連鎖律は伊藤の公式と呼び,微分は出来ないから積分の言葉で定式化される.余分な右辺第3項が特徴である.
    この手法は,一般の$L^2$-有界なマルチンゲールに関して使える.これはDoobが初めに指摘し,渡辺信三,国田寛 (67,\cite{KunitaWatanabe})が理論を作り,変数変換公式を得た.
    さらにMeyer (67,\cite{Meyer})が精緻な理論を組み立てる.
    余分な右辺第3項を修正するのがStratonovich積分/対称確率積分であるが,$X$に対してより強い条件を必要とすることとなる.

    伊藤清はKolmogov (1931,\cite{Kolmogorov31})から発想した.一般の連続Markov過程を,連続関数$a,b$を用いて
    \[E[X_{t+\Delta}-X|X_t=x]=a(t,x)\Delta+o(\Delta),\quad \Var[X_{t+\Delta}-X_t|X_t=x]=b(t,x)\Delta+o(\Delta)\]
    で定める,という出発点が前文に書いてあった.これはBrown運動の場合から次の変換
    \[X_{t+\Delta}-X=a(t,X_t)\Delta+\sqrt{b(t,X_t)}(B_{t+\Delta}-B_t)+o(\Delta)\]
    を経て得られる,ということでもあるが,この式は意味を持たない.基礎付けることができれば,Brown運動の見本路に関する知識を,Markov過程の見本路を調べることに応用できそうである.

    確率微分方程式を解くには,例えば同値な確率積分方程式をPicardの逐次近似法によって解くことが考えられる.
\end{quotation}

\section{確率積分}

\begin{tcolorbox}[colframe=ForestGreen, colback=ForestGreen!10!white,breakable,colbacktitle=ForestGreen!40!white,coltitle=black,fonttitle=\bfseries\sffamily,
title=]
    博士論文\cite{Ito42}でLévy過程のLévy-伊藤分解定理を定め,\cite{Ito44}で確率積分を定義した.
\end{tcolorbox}

\subsection{発展的可測}

\begin{tcolorbox}[colframe=ForestGreen, colback=ForestGreen!10!white,breakable,colbacktitle=ForestGreen!40!white,coltitle=black,fonttitle=\bfseries\sffamily,
title=]
    過程が可測であることより弱い条件として,発展的可測性を定義する.
    発展的可測な確率変数に関しては,見本道毎の積分ではなくて,より大局的な視点で見れる.
    そしてこのクラスについては,$\int^t_0u_sds$が$\F_t$-可測となる.
\end{tcolorbox}

\begin{definition}[progressive measurable]
    過程$(u_t)_{t\in\R_+}$が\textbf{発展的可測}であるとは,各時点$t$までの過程$u|_{\Om\times[0,t]}$が積写像として可測であることをいう:
    \[\forall_{t\in\R_+}\;u|_{\Om\times[0,t]}\in\L(\Om\times[0,t],\F_t\times\B([0,t])).\]
\end{definition}

\begin{lemma}[発展的可測性の十分条件]
    適合的な可測過程$u:\Om\times\R_+\to\R$には,発展的可測なバージョンが存在する(Meyer 1984, Th'm 4.6).
\end{lemma}

\subsection{単過程上の定義}

\begin{tcolorbox}[colframe=ForestGreen, colback=ForestGreen!10!white,breakable,colbacktitle=ForestGreen!40!white,coltitle=black,fonttitle=\bfseries\sffamily,
title=]
    「発展的可測な2乗可積分過程のなすHilbert空間」を$(L^2(\F\times\B^1(\R))\subset )L^2(\P)$で表す,これは過程を積空間上の関数と見ている.
    単過程とは,時間軸を有限に分解して,それぞれの区間上で特定の2乗可積分確率変数とみなせる挙動をする過程をいう.
\end{tcolorbox}

\begin{notation}\mbox{}
    \begin{enumerate}
        \item $\P$を積空間$\Om\times\R_+$上の$\sigma$-代数で,$1_A\;(A\in\P)$が発展的可測になるものとする:
        \[\P:=\Brace{A\in P(\Om\times\R_+)\mid A\cap(\Om\times[0,t])\in\F_t\times\B([0,t])}\]
        \item これが定める2乗可積分関数のHilbert空間を$L^2(\P):=L^2(\Om\times\R_+,\P,P\times l)$と表す,ただし$l$はLebesgue測度とした.
        また,$L^2_T(\P):=L^2(\Om\times[0,T],\P|_{\Om\times[0,T]},P\times l)$なる略記も用いる.
        \item ノルムを
        \[\norm{u}^2_{L^2(\P)}:=E\Square{\int^\infty_{0}u_s^2ds}=\int^\infty_0E[u^2_s]ds\]
        で表す($u^2_s$が非負値可測関数であることに注意すると,最後の等式はFubiniの定理による).
    \end{enumerate}
\end{notation}

\begin{definition}[simple process, stochastic integral]\mbox{}
    \begin{enumerate}
        \item 
        $u=(u_t)_{t\in\R_+}\in L^2(\P)$が\textbf{単過程}であるとは,
        \[u_t=\sum_{j=0}^{n-1}\phi_j1_{(t_j,t_{j+1}]}(t),\qquad 0\le t_0<t_1<\cdots<t_n,\phi_j\in\L^2_{\F_{t_j}}(\Om)\]
        を満たすことをいう.単関数全体の集合を$\E\subset L^2(\P)$で表す.
        \item 積分$I:\E\to\L^2(\Om)$を
        \[I(u)=\int^\infty_0 u_tdB_t:=\sum^{n-1}_{j=0}\phi_j(B_{t_{j+1}}-B_{t_j})\]
        と定める.
    \end{enumerate}
\end{definition}
\begin{remarks}
    すでにこの時点で,$\phi_j$を$1_{(t_j,t_{j+1}]}$に対して$\F_{t_j}$-可測としているため,$\phi_j\indep B_{t_{j+1}}-B_{t_j}$を含意することがトリックになる.
\end{remarks}

\begin{lemma}[単関数上の積分の性質]
    積分$I:\E\to\L^2(\Om)$は,$u,v\in\E,a,b\in\R$について次を満たす.
    \begin{enumerate}
        \item 線形性:$\int^\infty_0(au_t+bv_t)dB_t=a\int^\infty_0u_tdB_t+b\int^\infty_0v_tdB_t$.
        \item 中心化:$E[I(u)]=E\Square{\int^\infty_0u_tdB_t}=0$.
        \item 等長性:$\norm{I(u)}^2_{\L^2(\Om)}=E\Square{\paren{\int^\infty_0u_tdB_t}^2}=E\Square{\int^\infty_0u^2_tdt}=\norm{u}^2_{\L^2(\P)}$.
    \end{enumerate}
\end{lemma}
\begin{proof}\mbox{}
    \begin{enumerate}
        \item $u_t,v_t$が定める$\R_+$の分割の細分を取って考えると良い.
        \item $\forall_{j\in n}\;\phi_j\indep B_{t_{j+1}}-B_{t_j}$より,
        \[E\Square{\int^\infty_0u_tdB_t}=\sum^{n-1}_{j=0}E[\phi_j(B_{t_{j+1}}-B_{t_j})]=\sum^{n-1}_{j=0}E[\phi_j]E[B_{t_{j+1}}-B_{t_j}]=0.\]
        \item 
    \end{enumerate}
\end{proof}

\subsection{延長}

\begin{proposition}
    $\E\subset L^2(\P)$は稠密.
\end{proposition}
\begin{proof}
    $\E\subset L^2(\P)$の間に,$C$-過程のクラス
    \[C:=\Brace{u\in L^2(\P)\mid u:\R_+\to L^2(\Om)\text{は連続}}\]
    を考える.$C$-過程は可測過程であることに注意すると,たしかに$L^2(\P)$より小さい:$\E\subset C\subset L^2(\P)$.
    \begin{description}
        \item[$C$-過程による近似] 任意の$u\in L^2(\P)$を取る.
        \[u^{(n)}_t:=n\int^t_{\paren{t-\frac{1}{n}}\lor0}u_sds\]
        と定めると,$u^{(n)}:\R_+\to L^2(\Om)$は連続であり,したがって$\F\times\B^1(\R)$-可測.特に,$u^{(n)}\in L^2(\P)$.
        このとき,$(u_t^{(n)})_{n\in\N}\to u_t$すなわち$\lim_{n\to\infty}E\Square{\int^\infty_0\abs{u_t-u_t^{(n)}}^2dt}=0$.

    \end{description}
\end{proof}

\begin{corollary}
    確率積分$I:\E\to L^2(\Om)$は,線型な等長同型$I:L^2(\P)\to L^2(\Om)$に延長できる.
\end{corollary}
\begin{proof}
    任意の$u\in L^2(\P)$に対して,$\E$上の$L^2(\P)$-収束列$(u^{(n)})$が取れ,また$I:\E\to L^2(\Om)$の等長性より,極限
    \[I(u):=\lim_{n\to\infty}\int^\infty_0u_t^{(n)}dB_t\]
    が存在する.あとは,収束列$(u^{(n)})$の取り方に依らないことを示せば良い.
\end{proof}

\begin{proposition}
    任意の$u,v\in\L^2(\P)$について,
    \begin{enumerate}
        \item $E[I(u)]=0$.
        \item $E[I(u)I(v)]=E\Square{\int^\infty_0u_sv_sds}=E[I(uv)]$.
    \end{enumerate}
\end{proposition}

\subsection{不定積分}

\begin{definition}
    任意の$[0,T]\;(T>0)$上の積分を
    \[\int^T_0u_sdB_s:=\int^\infty_0u_s1_{[0,T]}(s)dB_s\]
    と定める.これは任意の$u\in L^2_T(\P)$に対しても定義できる:
    \[L^2_T(\P):=L^2(\Om\times[0,T],\P|_{\Om\times[0,T]},P\times l).\]
\end{definition}

\begin{example}
    Brown運動自体は発展的可測である.
    \[\int^T_0B_tdB_t=\frac{1}{2}B^2_T-\frac{1}{2}T\]
    が成り立つ.
\end{example}

\section{不定確率積分の過程}

\begin{notation}
    発展的可測な2乗可積分関数のうち,任意の不定積分も有限確定するクラス
    \[L^2_\infty(\P):=\Brace{u\in L^2(\P)\;\middle|\;\forall_{t>0}\;E\Square{\int^t_0u_s^2ds}<\infty}\]
    については,不定積分の過程$\paren{\int^t_0 u_sdB_s}_{t\in\R_+}:\R_+\to\L^2(\Om)$が確定する.
\end{notation}

\begin{proposition}
    不定積分の過程$\paren{M_t:=\int^t_0 u_sdB_s}_{t\in\R_+}\;(u\in L^2_\infty(\P))$について,
    \begin{enumerate}
        \item 加法性:$a\le b\le c\in\R_+$について,$\int^b_au_sdB_s+\int^c_bu_sdB_s=\int^c_au_sdB_s$.
        \item 可換性:$a<b\in\R_+$と$\F_a$-可測有界関数$F:\Om\to\R$について,$\int^b_aFu_sdB_s=F\int^b_au_sdB_s$.
        \item マルチンゲール性:$\{M_t\}_{t\in\R_+}\subset\subset L^2(\Om)$は$(\F_t)_{t\in\R_+}$-マルチンゲールであり,連続なバージョンが存在する.
        \item 最大不等式:$\forall_{T,\lambda>0}\;P\Square{\sup_{t\in[0,T]}\abs{M_t}>\lambda}\le\frac{1}{\lambda^2}E\Square{\int^T_0u^2_tdt}$かつ$E\Square{\sup_{t\in[0,T]}\abs{M_t}^2}\le 4E\Square{\int^T_0u^2_tdt}$.これは$u\in L^2(\P)$に過ぎずとも,$T=\infty$の場合成立する.
        \item 2次変分:$[0,t]$の分割$\pi:=\{0=t_0<t_1<\cdots<t_n=t\}$に関して,$S^2_\pi(u):=\sum^{n-1}_{j=0}\paren{\int^{t_{j+1}}_{t_j}u_sdB_s}^2\xrightarrow{L^2(\Om)}\int^t_0u^2_sds$.
        \item 停止時までの積分:$u\in L^2(\P),\tau:\Om\to\R_+$を停止時とする.このとき,$u1_{[0,\tau]}\in L^2(\P)$で,
        \[\int^\infty_0u_t1_{[0,\tau]}(t)dB_t=\int^\tau_0u_tdB_t.\]
    \end{enumerate}
\end{proposition}

\begin{corollary}
    \begin{description}
        \item[(5)] $\forall_{p>0,T>0}$
        \[c_pE\Square{\Abs{\int^T_0u^2_sds}^{p/2}}\le E\Square{\sup_{t\in[0,T]}\Abs{\int^t_0u_sdB_s}^p}\le C_pE\Square{\Abs{\int^T_0u^2_sds}^{p/2}}.\]
    \end{description}
\end{corollary}

\section{一般の過程の積分}

\begin{tcolorbox}[colframe=ForestGreen, colback=ForestGreen!10!white,breakable,colbacktitle=ForestGreen!40!white,coltitle=black,fonttitle=\bfseries\sffamily,
title=]
    $L^2_\infty(\P)$より広く,一般の局所自乗可積分な見本過程を持つ過程を積分できる.
\end{tcolorbox}

\begin{notation}
    \[L^2_\loc(\P):=\Brace{u\in L(\P)\mid}\]
\end{notation}

\section{伊藤の公式}

\section{田中の公式}

\section{伊藤の公式の多次元化}

\section{Stratonovich積分}

\section{後退確率積分}

\section{積分表現定理}

\begin{tcolorbox}[colframe=ForestGreen, colback=ForestGreen!10!white,breakable,colbacktitle=ForestGreen!40!white,coltitle=black,fonttitle=\bfseries\sffamily,
title=]
    $L^2(\Om)$の元は積分で表現出来る.
\end{tcolorbox}

\section{Girsanovの定理}

\begin{tcolorbox}[colframe=ForestGreen, colback=ForestGreen!10!white,breakable,colbacktitle=ForestGreen!40!white,coltitle=black,fonttitle=\bfseries\sffamily,
title=]
    ドリフトを持ったBrown運動が通常のBrown運動に戻るような測度変換を与える.
\end{tcolorbox}

\chapter{微分作用素と発散作用素}

\begin{quotation}
    Malliavin (1976) はWiener空間またはGauss空間上での無限次元解析を創始した.
\end{quotation}

\section{有限次元での議論}

\section{Malliavin微分}

\subsection{正規Hilbert空間上での定義}

\begin{notation}
    正規な可分Hilbert空間$(H,\mu:=N_Q)$を考える.可算な正規直交基底$(e_n)$が取れる.
    \begin{enumerate}
        \item $\brac{e_1,\cdots,e_n}$が張る部分空間への直交射影を$P_n$で表す.
        \item ${}^uC_b(H)$で$H$上の一様連続な有界汎関数全体の集合を表す.これは,一様ノルムについてBanach空間をなす.
        \item ${}^uC_b^k(H)$で$k$階一様連続にFrechet微分可能な汎関数全体のなす空間に,各階の導関数に関する1-ノルムを入れて得るBanach空間とする.
        \item 成分を$x_k:=\brac{x,e_k}\;(x\in H)$,$D_k\varphi:=\brac{D\varphi,e_k}\;(\varphi\in {}^uC_b^1(H))$.このとき,$D\varphi\in H^*$に注意.
        \item \textbf{指数関数の空間}を,部分空間
        \[\E(H):=\Brac{\Re(\ep_h)\in L^2(H,\mu)\mid \ep_h(x):=e^{i\brac{x,h}},x,h\in H}\]
        とする.
    \end{enumerate}
\end{notation}

\begin{definition}
    $M:=Q^{1/2}D:\E(H)\to L^2(H,\mu;H)$とする.これは,
    \begin{enumerate}
        \item $\varphi\in\E(H)$はFrechet導関数$D\varphi:H\to H^*$を持つ.これに$Q^{1/2}:H\to\Im Q^{1/2}<H$を合成する.
        \item $DW_f$は,$W_f$が$f\in Q^{1/2}(H)$に関してしか定義されていない通り,稠密部分空間$\Im Q^{1/2}$上にしか定まっていない.が,$Q^{1/2}DW_f$は$f\in H$上で定まる.
    \end{enumerate}
\end{definition}

\subsection{指数関数による近似}

\begin{proposition}
    任意の$\varphi\in{}^uC_b^1(H)$について,ある二重列$\{\varphi_{n,k}\}\subset\E(H)$が存在して,
    \begin{enumerate}
        \item $\forall_{x\in H}\;\lim_{n\to\infty}\lim_{k\to\infty}\varphi_{n,k}(x)=\varphi(x)$.
        \item $\forall_{x\in H}\;\lim_{n\to\infty}\lim_{k\to\infty}D\varphi_{n,k}(x)=D\varphi(x)$.
        \item $\forall_{n,k\in\N}\;\norm{\varphi_{n,k}}_0+\norm{D\varphi_{n,k}}_0\le\norm{\varphi}_0+\norm{D\varphi}_0$.
    \end{enumerate}
\end{proposition}

\begin{corollary}\mbox{}
    \begin{enumerate}
        \item ${}^uC_b(H)$は$L^2(H,\mu)$上稠密である.
        \item $\E(H)$は$L^2(H,\mu)$上稠密である.
    \end{enumerate}
\end{corollary}

\subsection{連続延長}

\begin{proposition}
    $M:\E(H)\to L^2(H,\mu;H)$は可閉である.
    この定義域を\textbf{Malliavin-Sobolev空間}といい,$D^{1,2}(H,\mu)$で表す.
    1階導関数との1-ノルムについて,これはHilbert空間をなす.
\end{proposition}

\begin{proposition}[$C^1$-級緩増加関数はMalliavin微分可能である]
    $C_1^p(H)\subset D^{1,2}(H,\mu)$.
\end{proposition}

\begin{proposition}[Lipschitz関数はMalliavin微分可能である]
    $\Lip^1(H)\subset D^{1,2}(H,\mu)$.
\end{proposition}

\begin{proposition}[微分積分学の基本定理?]
    任意の$f\in H$について,$W_f\in D^{1,2}(H,\mu)$で,$MW_f=f$.
\end{proposition}

\section{Skorokhod積分}

\begin{tcolorbox}[colframe=ForestGreen, colback=ForestGreen!10!white,breakable,colbacktitle=ForestGreen!40!white,coltitle=black,fonttitle=\bfseries\sffamily,
title=]
    随伴$M^*$をSkorohod積分といい,$\delta$とも表す.
\end{tcolorbox}

\section{Sobolev空間}

\section{確率積分としての発散}

\section{IsonormalなGauss過程}

\chapter{確率微分方程式}

\begin{quotation}
    元々Markov過程論から純粋に数学的に生じた問題意識の解決のために確率微分方程式が
    開発された.
    しかし,確率論的な現象は自然界にありふれている.
    常微分方程式の定める流れに沿って輸送された物理量は,移流方程式と呼ばれる1階の偏微分方程式を満たす.
    Brown運動に沿って輸送された物理量(熱など)は,熱伝導方程式・拡散方程式と呼ばれる2階の偏微分方程式を満たす.
    この対応関係は確率微分方程式を導入することでさらに一般化され,2階の放物型・楕円形の偏微分方程式の解を
    確率的に表示することが出来るようになる.
    こうして,確率微分方程式は,ポテンシャル論・偏微分方程式論や微分幾何学との架け橋になる.

    Brown運動は,空間的に一様な確率場での積分曲線だと思えば,さらに一般に空間的な一様性の仮定を取った場合が確率微分方程式であり,
    これはBrown運動の変形として得られるというのが伊藤清のアイデアである.
\end{quotation}

\section{概観}

\subsection{最適輸送理論}

\begin{definition}
    
\end{definition}

\subsection{移流方程式}

\begin{tcolorbox}[colframe=ForestGreen, colback=ForestGreen!10!white,breakable,colbacktitle=ForestGreen!40!white,coltitle=black,fonttitle=\bfseries\sffamily,
title=]
    Brown運動は,位置を忘れて粒子の視点から見た,確率ベクトル場から受ける「流れ」だと理解できる.
    したがってその背景には確率ベクトル場がある.
\end{tcolorbox}

\begin{definition}[advection, 時間的に一様なベクトル場による輸送方程式]\mbox{}
    \begin{enumerate}
        \item 物理量のスカラー場や物質がベクトル場によって輸送されること・経時変化することを,\textbf{移流}という.
        \item 定ベクトル$b=(b^1,b^2)\in\R^2$が定める空間一様な移流に関する初期値問題$u=u(t,x)$は,
        \begin{align*}
            \pp{u}{t}+b\cdot\nabla u=0,\quad(t>0,x=(x^1,x^2)\in\R^2),\\
            u(0,x)=f(x)\in C^1(\R^2)
        \end{align*}
        と表される.このとき,$X_t(x):=x-bt$は,時刻$t$に$x$に居る粒子が,時刻$0$にどこにいたかを表す.よって,初期状態として与えられた物理量$f:\R^2\to\R$を用いて,時刻$t$に位置$x\in\R^2$で観測される物理量は$f(X_t(x))$で表される.
        これは大数の法則に物理的な意味論を与える\ref{thm-law-of-large-number-of-Markov-chain}.
    \end{enumerate}
\end{definition}

\begin{discussion}[変数係数の移流問題]
    ベクトル場$b=b(t,x):\R\times\R^2\to\R^2$に対して,方程式
    \[\pp{u}{t}+b(t,x)\cdot\nabla u=0\quad(t>0,x\in\R^2)\]
    を考える.$y\in\R^2$から出発した粒子の位置を表す変数$Y_t\in\R^2$を固定して,そこでの時間変化を表す
    常微分方程式に関する初期値問題$\dot{Y}_t=b(t,Y_t)\;(t\in\R_+);Y_0=y\in\R^2$を考える.
    すると,解$u(t,x)$は次を満たす必要がある:
    \begin{align*}
        \pp{}{u}\paren{u(t,Y_t(y))}&=\pp{u}{t}(t,Y_t(y))+\dot{Y}_t(y)\cdot\nabla u(t,Y_t(y))\\
        &=\paren{\pp{u}{t}+b\cdot\nabla u}(t,Y_t(y))=0.
    \end{align*}
    よって,$u(t,Y_t(y))=u(0,Y_0(y))=f(y)$なる第一積分が見つかったことになる.

    こうして元の偏微分方程式は,$y\in\R^2$に居た粒子が受けることになる流れ$Y_t$に沿った輸送を記述する方程式であったと理解できる.
    または,ベクトル場による移流方程式とは,ある移流$Y_t$の第一積分が満たすべき方程式とも見れる.
    \footnote{これがKolmogorovが発見したものだった?}
\end{discussion}
\begin{remark}[2つの関連]
    いま,$Y_t:\R^2\times\R_+\to\R^2$は,各$y\in\R^2$に対して,これが受けることになる力の時系列$Y_t(y)$を与えている.
    これが可逆であるとする:$X_t:=Y_t^{-1}:\R^2\to\R^2$.すると,出発点$y$が$t$時刻にいる位置$x$について,$y=X_t(x)$と辿る確率過程になる.
    これは解について$u(t,x)=f(X_t(x))$なる表示を与える.

    ベクトル場が時間に依存しないとき,$Y_t$は可逆であり,逆$X_t$は常微分方程式$\dot{X}_t=-b(X_t),X_0=x$を満たす.
    定数係数の場合は,これを解いたものと見れるから,たしかに一般化となっている.
\end{remark}

\subsection{Brown運動が定める確率ベクトル場}

\begin{tcolorbox}[colframe=ForestGreen, colback=ForestGreen!10!white,breakable,colbacktitle=ForestGreen!40!white,coltitle=black,fonttitle=\bfseries\sffamily,
title=]
    移流方程式は,時間変化するベクトル場$\R_+\to\Map(\R^2,\R^2)$による物理量の輸送を考えた.
    もし,ベクトル場が確率的であったら?前節の例はひとつの見本道に過ぎないとしたら?
    すなわち,前節ではスタート地点$y\in\R^2$を根源事象とみて,時間変化するベクトル場を確率過程と見たが,ここに新たにランダム性の発生源を加えるのである.
\end{tcolorbox}

\subsection{確率微分方程式}

\begin{example}[空間的に一様な場合]
    $b\in\R^2,\al=(\al^i_k)_{i,k\in[2]}$が定める確率過程$X_t(x):=x-\al B_t-bt\;(t\in\R_+,x\in\R^2)$を考える.
    $X_t$は,定数係数ベクトル場による輸送と,Brown運動とを合成した運動に他ならない.
    実際,仮に$B_t$が$t$について微分可能であるならば,$\dot{X}_t=-\al\dot{B}_t-b$となるから,時間発展するベクトル場$-\al\dot{B}_t-b$による移流であると考えられる.
    $\al$がBrown運動の強さと歪みの情報を含んでいる.
    ただし,Brown運動の時刻$t$における$y\in\R^2$に関する分布は
    \[P[B_t\in dy]=p(t,y)dy:=\frac{1}{2\pi t}e^{-\abs{y}^2/2t}dy\]
    と表せる.

    すると,$X_t$による輸送の平均値を
    \[u(t,x):=E[f(X_t(x))]\]
    とおけば,これは
    \[q(t,x,z):=\frac{1}{2\pi t\abs{\det\al}}\exp\paren{-\frac{\abs{\al^{-1}(z-x+bt)}^2}{2t}}\]
    とおくことで
    \begin{align*}
        u(t,x)&=\int_{\R^2}f(x-al-bt)p(t,y)\\
        &=\int_{\R^2}f(z)q(t,x,z)dz.
    \end{align*}
    と整理出来る.これは熱伝導方程式と呼ばれる放物型方程式の一般解であり,$q(t,x,z)$が一般解と呼ばれるものである:
    \[\pp{u}{t}=\frac{1}{2}\sum^2_{i,j=1}a^{ij}\pp{^2u}{x^i\partial x^j}-b\cdot\nabla u,\quad(t>0,x\in\R^2).\]
\end{example}

\begin{example}[変数係数の場合]
    $\al,b$が共に$x\in\R^2$に依存する場合,任意の$t>0$について,常微分方程式
    \[\dot{X}_t=\al(X_t)\dot{B}_t+b(X_t)\quad(X_0=x\in\R^2)\]
    を考えたい.しかし$B_t$は微分可能でないから,積分形で代わりに表現することを考える.
    すなわち,$\dot{B}_sds=\dd{B_s}{s}ds=dB_s$とし,この右辺を定義することを考える.
    \[X_t=x+\int_0t\al(X_s)dB_s+\int^t_0b(X_s)ds.\]
    すると次の問題は,右辺第2項がStieltjesの意味での積分として定義することは出来ない.
    $B_t$は有界変動でないので,線積分の発想はお門違いである.
    これは確率論的な考察で乗り越えることが出来る.

    この確率微分方程式の解$X_t$を求めれば,平均値$u(t,x)$は放物型偏微分方程式
    \[\pp{u}{t}=\frac{1}{2}\sum^2_{i,j=1}a^{ij}(x)\pp{^2u}{x^i\partial x^j}+b(x)\cdot\nabla u\quad(t>0,x\in\R^2,a(x):=\al(x){}^t\!\al(x))\]
    を満たすことになる.
    これはいわば,確率過程$X_t$から得られる平均処置効果のような統計量の1つが,偏微分方程式で記述される物理法則を満たすということに過ぎない.
    $X_t$は非常に豊かな情報を湛えていて,偏微分方程式はその1断面に過ぎないと言えるだろう.
\end{example}


\chapter{無限次元確率微分方程式}

\begin{quotation}
    確率微分方程式はランダムなゆらぎを持つ常微分方程式であるから,同様にランダムなゆらぎを持つ偏微分方程式に当たる概念も自然に現れるはずである.
\end{quotation}

\chapter{汎関数解析}

\begin{quotation}
    厚地\cite{厚地}が近い研究をしている.
    \begin{quote}
        種々の多様体とその上の特徴的な関数族を訪ね歩きたいのだが,このように調べたい関数にその土地固有の確率過程を放り込んで展開すれば,何かしら見えてくると思われる.
    \end{quote}
\end{quotation}

\begin{thebibliography}{99}
    \bibitem{Nualart}
    David Nualart and Eulalia Nualart "Introduction to Malliavin Calculus"
    \bibitem{Revus and Yor}
    Daniel Revuz, and Marc Yor "Continuous Martingales and Brownian Motion"
    \bibitem{Prato}
    Giuseppe Prato - Introduction to Stochastic Analysis and Malliavin Calculus
    \bibitem{Vershynin}
    Vershynin -  High-Dimensional Probability
    \bibitem{Bass}
    Richard Bass - Stochastic Processes
    \bibitem{舟木}
    舟木直久『確率微分方程式』
    \bibitem{厚地}
    厚地淳『確率論と関数論』
    \bibitem{Morters and Peres}
    Morters and Peres - Brownian Motion
    \bibitem{Kolmogorov31}
    Kolmogorov, A. N. (1933). Analytical methods in probability theory. 「私はKolmogorovのこの論文(「解析的方法」)の序文にあるアイデアからヒントを得て,マルコフ過程の軌道を表す確率微分方程式を導入したが,これが私のその後の研究の方向を決めることになった.」
    \bibitem{Ito42}
    Kiyosi Ito (1942). "Differential equations determining a Markoff process" (PDF). Zenkoku Sizyo Sugaku Danwakai-si (J. Pan-Japan Math. Coll.) (1077): 1352–1400.
    \bibitem{Ito44}
    Kiyosi Itô (1944). "Stochastic integral". Proceedings of the Imperial Academy. 20 (8): 519–524.
    \bibitem{KunitaWatanabe}
    Kunita, H., and Watanabe, S. (1967). On Square Integrable Martingales. \textit{Nagoya Mathematics Journal}. 30: 209-245.
    \bibitem{Meyer}
    Meyer, P. A. (1967). Intégrales Stochastiques. \textit{Séminaire de Probabilités I}. Lecture Notes in Math., 39: 72-162. 劣martingaleのDoob-Meyer分解を用いて,確率積分が一般の半マルチンゲールについて定義された.
    こうして確率解析の復権が起こった.
\end{thebibliography}

\end{document}