\documentclass[uplatex,dvipdfmx]{jsreport}
\title{確率論}
\author{司馬博文}
\date{\today}
\pagestyle{headings} \setcounter{secnumdepth}{4}
%\usepackage{mathtools}
%\mathtoolsset{showonlyrefs=true} %labelを附した数式にのみ附番される設定.
%\usepackage{amsmath} %mathtoolsの内部で呼ばれるので要らない.
\usepackage{amsfonts} %mathfrak, mathcal, mathbbなど.
\usepackage{amsthm} %定理環境.
\usepackage{amssymb} %AMSFontsを使うためのパッケージ.
\usepackage{ascmac} %screen, itembox, shadebox環境.全てLATEX2εの標準機能の範囲で作られたもの.
\usepackage{comment} %comment環境を用いて,複数行をcomment outできるようにするpackage
\usepackage{wrapfig} %図の周りに文字をwrapさせることができる.詳細な制御ができる.
\usepackage[usenames, dvipsnames]{xcolor} %xcolorはcolorの拡張.optionの意味はdvipsnamesはLoad a set of predefined colors. forestgreenなどの色が追加されている.usenamesはobsoleteとだけ書いてあった.
\setcounter{tocdepth}{2} %目次に表示される深さ.2はsubsectionまで
\usepackage{multicol} %\begin{multicols}{2}環境で途中からmulticolumnに出来る.

\usepackage{url}
\usepackage[dvipdfmx,colorlinks,linkcolor=blue,urlcolor=blue]{hyperref} %生成されるPDFファイルにおいて、\tableofcontentsによって書き出された目次をクリックすると該当する見出しへジャンプしたり、さらには、\label{ラベル名}を番号で参照する\ref{ラベル名}やthebibliography環境において\bibitem{ラベル名}を文献番号で参照する\cite{ラベル名}においても番号をクリックすると該当箇所にジャンプする.囲み枠はダサいので,colorlinksで囲み廃止し,リンク自体に色を付けることにした.
\usepackage{pxjahyper} %pxrubrica同様,八登崇之さん.hyperrefは日本語pLaTeXに最適化されていないから,hyperrefとセットで,(u)pLaTeX+hyperref+dvipdfmxの組み合わせで日本語を含む「しおり」をもつPDF文書を作成する場合に必要となる機能を提供する
\definecolor{花緑青}{cmyk}{0.52,0.03,0,0.27}
\definecolor{サーモンピンク}{cmyk}{0,0.65,0.65,0.05}
\definecolor{暗中模索}{rgb}{0.2,0.2,0.2}

\usepackage{tikz}
\usetikzlibrary{positioning,automata} %automaton描画のため
\usepackage{tikz-cd}
\usepackage[all]{xy}
\def\objectstyle{\displaystyle} %デフォルトではxymatrix中の数式が文中数式モードになるので,それを直す.\labelstyleも同様にxy packageの中で定義されており,文中数式モードになっている.

\usepackage[version=4]{mhchem} %化学式をTikZで簡単に書くためのパッケージ.
\usepackage{chemfig} %化学構造式をTikZで描くためのパッケージ.
\usepackage{siunitx} %IS単位を書くためのパッケージ

\usepackage{ulem} %取り消し線を引くためのパッケージ
\usepackage{pxrubrica} %日本語にルビをふる.八登崇之(やとうたかゆき)氏による.

\usepackage{graphicx} %rotatebox, scalebox, reflectbox, resizeboxなどのコマンドや,図表の読み込み\includegraphicsを司る.graphics というパッケージもありますが,graphicx はこれを高機能にしたものと考えて結構です(ただし graphicx は内部で graphics を読み込みます)

\usepackage[breakable]{tcolorbox} %加藤晃史さんがフル活用していたtcolorboxを,途中改ページ可能で.
\tcbuselibrary{theorems} %https://qiita.com/t_kemmochi/items/483b8fcdb5db8d1f5d5e
\usepackage{enumerate} %enumerate環境を凝らせる.
\usepackage[top=15truemm,bottom=15truemm,left=10truemm,right=10truemm]{geometry} %足助さんからもらったオプション

%%%%%%%%%%%%%%% 環境マクロ %%%%%%%%%%%%%%%

\usepackage{listings} %ソースコードを表示できる環境.多分もっといい方法ある.
\usepackage{jvlisting} %日本語のコメントアウトをする場合jlistingが必要
\lstset{ %ここからソースコードの表示に関する設定.lstlisting環境では,[caption=hoge,label=fuga]などのoptionを付けられる.
%[escapechar=!]とすると,LaTeXコマンドを使える.
  basicstyle={\ttfamily},
  identifierstyle={\small},
  commentstyle={\smallitshape},
  keywordstyle={\small\bfseries},
  ndkeywordstyle={\small},
  stringstyle={\small\ttfamily},
  frame={tb},
  breaklines=true,
  columns=[l]{fullflexible},
  numbers=left,
  xrightmargin=0zw,
  xleftmargin=3zw,
  numberstyle={\scriptsize},
  stepnumber=1,
  numbersep=1zw,
  lineskip=-0.5ex
}
%\makeatletter %caption番号を「[chapter番号].[section番号].[subsection番号]-[そのsubsection内においてn番目]」に変更
%    \AtBeginDocument{
%    \renewcommand*{\thelstlisting}{\arabic{chapter}.\arabic{section}.\arabic{lstlisting}}
%    \@addtoreset{lstlisting}{section}
%    }
%\makeatother
\renewcommand{\lstlistingname}{算譜} %caption名を"program"に変更

\newtcolorbox{tbox}[3][]{%
colframe=#2,colback=#2!10,coltitle=#2!20!black,title={#3},#1}

%%%%%%%%%%%%%%% フォント %%%%%%%%%%%%%%%

\usepackage{textcomp, mathcomp} %Text Companionとは,T1 encodingに入らなかった文字群.これを使うためのパッケージ.\textsectionでブルバキに!
\usepackage[T1]{fontenc} %8bitエンコーディングにする.comp系拡張数学文字の動作が安定する.

%%%%%%%%%%%%%%% 数学記号のマクロ %%%%%%%%%%%%%%%

\newcommand{\abs}[1]{\lvert#1\rvert} %mathtoolsはこうやって使うのか!
\newcommand{\Abs}[1]{\left|#1\right|}
\newcommand{\norm}[1]{\|#1\|}
\newcommand{\Norm}[1]{\left\|#1\right\|}
%\newcommand{\brace}[1]{\{#1\}}
\newcommand{\Brace}[1]{\left\{#1\right\}}
\newcommand{\paren}[1]{\left(#1\right)}
\newcommand{\bracket}[1]{\langle#1\rangle}
\newcommand{\brac}[1]{\langle#1\rangle}
\newcommand{\Bracket}[1]{\left\langle#1\right\rangle}
\newcommand{\Brac}[1]{\left\langle#1\right\rangle}
\newcommand{\Square}[1]{\left[#1\right]}
\renewcommand{\o}[1]{\overline{#1}}
\renewcommand{\u}[1]{\underline{#1}}
\renewcommand{\iff}{\;\mathrm{iff}\;} %nLabリスペクト
\newcommand{\pp}[2]{\frac{\partial #1}{\partial #2}}
\newcommand{\ppp}[3]{\frac{\partial #1}{\partial #2\partial #3}}
\newcommand{\dd}[2]{\frac{d #1}{d #2}}
\newcommand{\floor}[1]{\lfloor#1\rfloor}
\newcommand{\Floor}[1]{\left\lfloor#1\right\rfloor}
\newcommand{\ceil}[1]{\lceil#1\rceil}

\newcommand{\iso}{\xrightarrow{\,\smash{\raisebox{-0.45ex}{\ensuremath{\scriptstyle\sim}}}\,}}
\newcommand{\wt}[1]{\widetilde{#1}}
\newcommand{\wh}[1]{\widehat{#1}}

\newcommand{\Lrarrow}{\;\;\Leftrightarrow\;\;}

%ノルム位相についての閉包 https://newbedev.com/how-to-make-double-overline-with-less-vertical-displacement
\makeatletter
\newcommand{\dbloverline}[1]{\overline{\dbl@overline{#1}}}
\newcommand{\dbl@overline}[1]{\mathpalette\dbl@@overline{#1}}
\newcommand{\dbl@@overline}[2]{%
  \begingroup
  \sbox\z@{$\m@th#1\overline{#2}$}%
  \ht\z@=\dimexpr\ht\z@-2\dbl@adjust{#1}\relax
  \box\z@
  \ifx#1\scriptstyle\kern-\scriptspace\else
  \ifx#1\scriptscriptstyle\kern-\scriptspace\fi\fi
  \endgroup
}
\newcommand{\dbl@adjust}[1]{%
  \fontdimen8
  \ifx#1\displaystyle\textfont\else
  \ifx#1\textstyle\textfont\else
  \ifx#1\scriptstyle\scriptfont\else
  \scriptscriptfont\fi\fi\fi 3
}
\makeatother
\newcommand{\oo}[1]{\dbloverline{#1}}

\DeclareMathOperator{\grad}{\mathrm{grad}}
\DeclareMathOperator{\rot}{\mathrm{rot}}
\DeclareMathOperator{\divergence}{\mathrm{div}}
\newcommand{\False}{\mathrm{False}}
\newcommand{\True}{\mathrm{True}}
\DeclareMathOperator{\tr}{\mathrm{tr}}
\newcommand{\M}{\mathcal{M}}
\newcommand{\cF}{\mathcal{F}}
\newcommand{\cD}{\mathcal{D}}
\newcommand{\fX}{\mathfrak{X}}
\newcommand{\fY}{\mathfrak{Y}}
\newcommand{\fZ}{\mathfrak{Z}}
\renewcommand{\H}{\mathcal{H}}
\newcommand{\fH}{\mathfrak{H}}
\newcommand{\bH}{\mathbb{H}}
\newcommand{\id}{\mathrm{id}}
\newcommand{\A}{\mathcal{A}}
% \renewcommand\coprod{\rotatebox[origin=c]{180}{$\prod$}} すでにどこかにある.
\newcommand{\pr}{\mathrm{pr}}
\newcommand{\U}{\mathfrak{U}}
\newcommand{\Map}{\mathrm{Map}}
\newcommand{\dom}{\mathrm{Dom}\;}
\newcommand{\cod}{\mathrm{Cod}\;}
\newcommand{\supp}{\mathrm{supp}\;}
\newcommand{\otherwise}{\mathrm{otherwise}}
\newcommand{\st}{\;\mathrm{s.t.}\;}
\newcommand{\lmd}{\lambda}
\newcommand{\Lmd}{\Lambda}
%%% 線型代数学
\newcommand{\Ker}{\mathrm{Ker}\;}
\newcommand{\Coker}{\mathrm{Coker}\;}
\newcommand{\Coim}{\mathrm{Coim}\;}
\newcommand{\rank}{\mathrm{rank}}
\newcommand{\lcm}{\mathrm{lcm}}
\newcommand{\sgn}{\mathrm{sgn}}
\newcommand{\GL}{\mathrm{GL}}
\newcommand{\SL}{\mathrm{SL}}
\newcommand{\alt}{\mathrm{alt}}
%%% 複素解析学
\renewcommand{\Re}{\mathrm{Re}\;}
\renewcommand{\Im}{\mathrm{Im}\;}
\newcommand{\Gal}{\mathrm{Gal}}
\newcommand{\PGL}{\mathrm{PGL}}
\newcommand{\PSL}{\mathrm{PSL}}
\newcommand{\Log}{\mathrm{Log}\,}
\newcommand{\Res}{\mathrm{Res}\,}
\newcommand{\on}{\mathrm{on}\;}
\newcommand{\hatC}{\hat{\C}}
\newcommand{\hatR}{\hat{\R}}
\newcommand{\PV}{\mathrm{P.V.}}
\newcommand{\diam}{\mathrm{diam}}
\newcommand{\Area}{\mathrm{Area}}
\newcommand{\Lap}{\Laplace}
\newcommand{\f}{\mathbf{f}}
\newcommand{\cR}{\mathcal{R}}
\newcommand{\const}{\mathrm{const.}}
\newcommand{\Om}{\Omega}
\newcommand{\Cinf}{C^\infty}
\newcommand{\ep}{\epsilon}
\newcommand{\dist}{\mathrm{dist}}
\newcommand{\opart}{\o{\partial}}
%%% 解析力学
\newcommand{\x}{\mathbf{x}}
%%% 集合と位相
\renewcommand{\O}{\mathcal{O}}
\renewcommand{\S}{\mathcal{S}}
\renewcommand{\U}{\mathcal{U}}
\newcommand{\V}{\mathcal{V}}
\renewcommand{\P}{\mathcal{P}}
\newcommand{\R}{\mathbb{R}}
\newcommand{\N}{\mathbb{N}}
\newcommand{\C}{\mathbb{C}}
\newcommand{\Z}{\mathbb{Z}}
\newcommand{\Q}{\mathbb{Q}}
\newcommand{\TV}{\mathrm{TV}}
\newcommand{\ORD}{\mathrm{ORD}}
\newcommand{\Tr}{\mathrm{Tr}\;}
\newcommand{\Card}{\mathrm{Card}\;}
\newcommand{\Top}{\mathrm{Top}}
\newcommand{\Disc}{\mathrm{Disc}}
\newcommand{\Codisc}{\mathrm{Codisc}}
\newcommand{\CoDisc}{\mathrm{CoDisc}}
\newcommand{\Ult}{\mathrm{Ult}}
\newcommand{\ord}{\mathrm{ord}}
\newcommand{\maj}{\mathrm{maj}}
%%% 形式言語理論
\newcommand{\REGEX}{\mathrm{REGEX}}
\newcommand{\RE}{\mathbf{RE}}

%%% Fourier解析
\newcommand*{\Laplace}{\mathop{}\!\mathbin\bigtriangleup}
\newcommand*{\DAlambert}{\mathop{}\!\mathbin\Box}
%%% Graph Theory
\newcommand{\SimpGph}{\mathrm{SimpGph}}
\newcommand{\Gph}{\mathrm{Gph}}
\newcommand{\mult}{\mathrm{mult}}
\newcommand{\inv}{\mathrm{inv}}
%%% 多様体
\newcommand{\Der}{\mathrm{Der}}
\newcommand{\osub}{\overset{\mathrm{open}}{\subset}}
\newcommand{\osup}{\overset{\mathrm{open}}{\supset}}
\newcommand{\al}{\alpha}
\newcommand{\K}{\mathbb{K}}
\newcommand{\Sp}{\mathrm{Sp}}
\newcommand{\g}{\mathfrak{g}}
\newcommand{\h}{\mathfrak{h}}
\newcommand{\Exp}{\mathrm{Exp}\;}
\newcommand{\Imm}{\mathrm{Imm}}
\newcommand{\Imb}{\mathrm{Imb}}
\newcommand{\codim}{\mathrm{codim}\;}
\newcommand{\Gr}{\mathrm{Gr}}
%%% 代数
\newcommand{\Ad}{\mathrm{Ad}}
\newcommand{\finsupp}{\mathrm{fin\;supp}}
\newcommand{\SO}{\mathrm{SO}}
\newcommand{\SU}{\mathrm{SU}}
\newcommand{\acts}{\curvearrowright}
\newcommand{\mono}{\hookrightarrow}
\newcommand{\epi}{\twoheadrightarrow}
\newcommand{\Stab}{\mathrm{Stab}}
\newcommand{\nor}{\mathrm{nor}}
\newcommand{\T}{\mathbb{T}}
\newcommand{\Aff}{\mathrm{Aff}}
\newcommand{\rsub}{\triangleleft}
\newcommand{\rsup}{\triangleright}
\newcommand{\subgrp}{\overset{\mathrm{subgrp}}{\subset}}
\newcommand{\Ext}{\mathrm{Ext}}
\newcommand{\sbs}{\subset}\newcommand{\sps}{\supset}
\newcommand{\In}{\mathrm{In}}
\newcommand{\Tor}{\mathrm{Tor}}
\newcommand{\p}{\mathfrak{p}}
\newcommand{\q}{\mathfrak{q}}
\newcommand{\m}{\mathfrak{m}}
\newcommand{\cS}{\mathcal{S}}
\newcommand{\Frac}{\mathrm{Frac}\,}
\newcommand{\Spec}{\mathrm{Spec}\,}
\newcommand{\bA}{\mathbb{A}}
\newcommand{\Sym}{\mathrm{Sym}}
\newcommand{\Ann}{\mathrm{Ann}}
%%% 代数的位相幾何学
\newcommand{\Ho}{\mathrm{Ho}}
\newcommand{\CW}{\mathrm{CW}}
\newcommand{\lc}{\mathrm{lc}}
\newcommand{\cg}{\mathrm{cg}}
\newcommand{\Fib}{\mathrm{Fib}}
\newcommand{\Cyl}{\mathrm{Cyl}}
\newcommand{\Ch}{\mathrm{Ch}}
%%% 数値解析
\newcommand{\round}{\mathrm{round}}
\newcommand{\cond}{\mathrm{cond}}
\newcommand{\diag}{\mathrm{diag}}
%%% 確率論
\newcommand{\calF}{\mathcal{F}}
\newcommand{\X}{\mathcal{X}}
\newcommand{\Meas}{\mathrm{Meas}}
\newcommand{\as}{\;\mathrm{a.s.}} %almost surely
\newcommand{\io}{\;\mathrm{i.o.}} %infinitely often
\newcommand{\fe}{\;\mathrm{f.e.}} %with a finite number of exceptions
\newcommand{\F}{\mathcal{F}}
\newcommand{\bF}{\mathbb{F}}
\newcommand{\W}{\mathcal{W}}
\newcommand{\Pois}{\mathrm{Pois}}
\newcommand{\iid}{\mathrm{i.i.d.}}
\newcommand{\wconv}{\rightsquigarrow}
\newcommand{\Var}{\mathrm{Var}}
\newcommand{\xrightarrown}{\xrightarrow{n\to\infty}}
\newcommand{\au}{\mathrm{au}}
\newcommand{\cT}{\mathcal{T}}
%%% 情報理論
\newcommand{\bit}{\mathrm{bit}}
%%% 積分論
\newcommand{\calA}{\mathcal{A}}
\newcommand{\calB}{\mathcal{B}}
\newcommand{\D}{\mathcal{D}}
\newcommand{\Y}{\mathcal{Y}}
\newcommand{\calC}{\mathcal{C}}
\renewcommand{\ae}{\mathrm{a.e.}\;}
\newcommand{\cZ}{\mathcal{Z}}
\newcommand{\fF}{\mathfrak{F}}
\newcommand{\fI}{\mathfrak{I}}
\newcommand{\E}{\mathcal{E}}
\newcommand{\sMap}{\sigma\textrm{-}\mathrm{Map}}
\DeclareMathOperator*{\argmax}{arg\,max}
\DeclareMathOperator*{\argmin}{arg\,min}
\newcommand{\cC}{\mathcal{C}}
\newcommand{\comp}{\complement}
\newcommand{\J}{\mathcal{J}}
\newcommand{\sumN}[1]{\sum_{#1\in\N}}
\newcommand{\cupN}[1]{\cup_{#1\in\N}}
\newcommand{\capN}[1]{\cap_{#1\in\N}}
\newcommand{\Sum}[1]{\sum_{#1=1}^\infty}
\newcommand{\sumn}{\sum_{n=1}^\infty}
\newcommand{\summ}{\sum_{m=1}^\infty}
\newcommand{\sumk}{\sum_{k=1}^\infty}
\newcommand{\sumi}{\sum_{i=1}^\infty}
\newcommand{\sumj}{\sum_{j=1}^\infty}
\newcommand{\cupn}{\cup_{n=1}^\infty}
\newcommand{\capn}{\cap_{n=1}^\infty}
\newcommand{\cupk}{\cup_{k=1}^\infty}
\newcommand{\cupi}{\cup_{i=1}^\infty}
\newcommand{\cupj}{\cup_{j=1}^\infty}
\newcommand{\limn}{\lim_{n\to\infty}}
\renewcommand{\l}{\mathcal{l}}
\renewcommand{\L}{\mathcal{L}}
\newcommand{\Cl}{\mathrm{Cl}}
\newcommand{\cN}{\mathcal{N}}
\newcommand{\Ae}{\textrm{-a.e.}\;}
\newcommand{\csub}{\overset{\textrm{closed}}{\subset}}
\newcommand{\csup}{\overset{\textrm{closed}}{\supset}}
\newcommand{\wB}{\wt{B}}
\newcommand{\cG}{\mathcal{G}}
\newcommand{\Lip}{\mathrm{Lip}}
\newcommand{\Dom}{\mathrm{Dom}}
%%% 数理ファイナンス
\newcommand{\pre}{\mathrm{pre}}
\newcommand{\om}{\omega}

%%% 統計的因果推論
\newcommand{\Do}{\mathrm{Do}}
%%% 数理統計
\newcommand{\bP}{\mathbb{P}}
\newcommand{\compsub}{\overset{\textrm{cpt}}{\subset}}
\newcommand{\lip}{\textrm{lip}}
\newcommand{\BL}{\mathrm{BL}}
\newcommand{\G}{\mathbb{G}}
\newcommand{\NB}{\mathrm{NB}}
\newcommand{\oR}{\o{\R}}
\newcommand{\liminfn}{\liminf_{n\to\infty}}
\newcommand{\limsupn}{\limsup_{n\to\infty}}
%\newcommand{\limn}{\lim_{n\to\infty}}
\newcommand{\esssup}{\mathrm{ess.sup}}
\newcommand{\asto}{\xrightarrow{\as}}
\newcommand{\Cov}{\mathrm{Cov}}
\newcommand{\cQ}{\mathcal{Q}}
\newcommand{\VC}{\mathrm{VC}}
\newcommand{\mb}{\mathrm{mb}}
\newcommand{\Avar}{\mathrm{Avar}}
\newcommand{\bB}{\mathbb{B}}
\newcommand{\bW}{\mathbb{W}}
\newcommand{\sd}{\mathrm{sd}}
\newcommand{\w}[1]{\widehat{#1}}
\newcommand{\bZ}{\mathbb{Z}}
\newcommand{\Bernoulli}{\mathrm{Bernoulli}}
\newcommand{\Mult}{\mathrm{Mult}}
\newcommand{\BPois}{\mathrm{BPois}}
\newcommand{\fraks}{\mathfrak{s}}
\newcommand{\frakk}{\mathfrak{k}}
\newcommand{\IF}{\mathrm{IF}}
\newcommand{\bX}{\mathbf{X}}
\newcommand{\bx}{\mathbf{x}}
\newcommand{\indep}{\raisebox{0.05em}{\rotatebox[origin=c]{90}{$\models$}}}
\newcommand{\IG}{\mathrm{IG}}
\newcommand{\Levy}{\mathrm{Levy}}
\newcommand{\MP}{\mathrm{MP}}
\newcommand{\Hermite}{\mathrm{Hermite}}
\newcommand{\Skellam}{\mathrm{Skellam}}
\newcommand{\Dirichlet}{\mathrm{Dirichlet}}
\newcommand{\Beta}{\mathrm{Beta}}
\newcommand{\bE}{\mathbb{E}}
\newcommand{\bG}{\mathbb{G}}
\newcommand{\MISE}{\mathrm{MISE}}
\newcommand{\logit}{\mathtt{logit}}
\newcommand{\expit}{\mathtt{expit}}
\newcommand{\cK}{\mathcal{K}}
\newcommand{\dl}{\dot{l}}
\newcommand{\dotp}{\dot{p}}
\newcommand{\wl}{\wt{l}}
%%% 函数解析
\renewcommand{\c}{\mathbf{c}}
\newcommand{\loc}{\mathrm{loc}}
\newcommand{\Lh}{\mathrm{L.h.}}
\newcommand{\Epi}{\mathrm{Epi}\;}
\newcommand{\slim}{\mathrm{slim}}
\newcommand{\Ban}{\mathrm{Ban}}
\newcommand{\Hilb}{\mathrm{Hilb}}
\newcommand{\Ex}{\mathrm{Ex}}
\newcommand{\Co}{\mathrm{Co}}
\newcommand{\sa}{\mathrm{sa}}
\newcommand{\nnorm}[1]{{\left\vert\kern-0.25ex\left\vert\kern-0.25ex\left\vert #1 \right\vert\kern-0.25ex\right\vert\kern-0.25ex\right\vert}}
\newcommand{\dvol}{\mathrm{dvol}}
\newcommand{\Sconv}{\mathrm{Sconv}}
\newcommand{\I}{\mathcal{I}}
\newcommand{\nonunital}{\mathrm{nu}}
\newcommand{\cpt}{\mathrm{cpt}}
\newcommand{\lcpt}{\mathrm{lcpt}}
\newcommand{\com}{\mathrm{com}}
\newcommand{\Haus}{\mathrm{Haus}}
\newcommand{\proper}{\mathrm{proper}}
\newcommand{\infinity}{\mathrm{inf}}
\newcommand{\TVS}{\mathrm{TVS}}
\newcommand{\ess}{\mathrm{ess}}
\newcommand{\ext}{\mathrm{ext}}
\newcommand{\Index}{\mathrm{Index}}
\newcommand{\SSR}{\mathrm{SSR}}
\newcommand{\vs}{\mathrm{vs.}}
\newcommand{\fM}{\mathfrak{M}}
\newcommand{\EDM}{\mathrm{EDM}}
\newcommand{\Tw}{\mathrm{Tw}}
\newcommand{\fC}{\mathfrak{C}}
\newcommand{\bn}{\mathbf{n}}
\newcommand{\br}{\mathbf{r}}
\newcommand{\Lam}{\Lambda}
\newcommand{\lam}{\lambda}
\newcommand{\one}{\mathbf{1}}
\newcommand{\dae}{\text{-a.e.}}
\newcommand{\td}{\text{-}}
\newcommand{\RM}{\mathrm{RM}}
%%% 最適化
\newcommand{\Minimize}{\text{Minimize}}
\newcommand{\subjectto}{\text{subject to}}
\newcommand{\Ri}{\mathrm{Ri}}
%\newcommand{\Cl}{\mathrm{Cl}}
\newcommand{\Cone}{\mathrm{Cone}}
\newcommand{\Int}{\mathrm{Int}}
%%% 圏
\newcommand{\varlim}{\varprojlim}
\newcommand{\Hom}{\mathrm{Hom}}
\newcommand{\Iso}{\mathrm{Iso}}
\newcommand{\Mor}{\mathrm{Mor}}
\newcommand{\Isom}{\mathrm{Isom}}
\newcommand{\Aut}{\mathrm{Aut}}
\newcommand{\End}{\mathrm{End}}
\newcommand{\op}{\mathrm{op}}
\newcommand{\ev}{\mathrm{ev}}
\newcommand{\Ob}{\mathrm{Ob}}
\newcommand{\Ar}{\mathrm{Ar}}
\newcommand{\Arr}{\mathrm{Arr}}
\newcommand{\Set}{\mathrm{Set}}
\newcommand{\Grp}{\mathrm{Grp}}
\newcommand{\Cat}{\mathrm{Cat}}
\newcommand{\Mon}{\mathrm{Mon}}
\newcommand{\CMon}{\mathrm{CMon}} %Comutative Monoid 可換単系とモノイドの射
\newcommand{\Ring}{\mathrm{Ring}}
\newcommand{\CRing}{\mathrm{CRing}}
\newcommand{\Ab}{\mathrm{Ab}}
\newcommand{\Pos}{\mathrm{Pos}}
\newcommand{\Vect}{\mathrm{Vect}}
\newcommand{\FinVect}{\mathrm{FinVect}}
\newcommand{\FinSet}{\mathrm{FinSet}}
\newcommand{\OmegaAlg}{\Omega$-$\mathrm{Alg}}
\newcommand{\OmegaEAlg}{(\Omega,E)$-$\mathrm{Alg}}
\newcommand{\Alg}{\mathrm{Alg}} %代数の圏
\newcommand{\CAlg}{\mathrm{CAlg}} %可換代数の圏
\newcommand{\CPO}{\mathrm{CPO}} %Complete Partial Order & continuous mappings
\newcommand{\Fun}{\mathrm{Fun}}
\newcommand{\Func}{\mathrm{Func}}
\newcommand{\Met}{\mathrm{Met}} %Metric space & Contraction maps
\newcommand{\Pfn}{\mathrm{Pfn}} %Sets & Partial function
\newcommand{\Rel}{\mathrm{Rel}} %Sets & relation
\newcommand{\Bool}{\mathrm{Bool}}
\newcommand{\CABool}{\mathrm{CABool}}
\newcommand{\CompBoolAlg}{\mathrm{CompBoolAlg}}
\newcommand{\BoolAlg}{\mathrm{BoolAlg}}
\newcommand{\BoolRng}{\mathrm{BoolRng}}
\newcommand{\HeytAlg}{\mathrm{HeytAlg}}
\newcommand{\CompHeytAlg}{\mathrm{CompHeytAlg}}
\newcommand{\Lat}{\mathrm{Lat}}
\newcommand{\CompLat}{\mathrm{CompLat}}
\newcommand{\SemiLat}{\mathrm{SemiLat}}
\newcommand{\Stone}{\mathrm{Stone}}
\newcommand{\Sob}{\mathrm{Sob}} %Sober space & continuous map
\newcommand{\Op}{\mathrm{Op}} %Category of open subsets
\newcommand{\Sh}{\mathrm{Sh}} %Category of sheave
\newcommand{\PSh}{\mathrm{PSh}} %Category of presheave, PSh(C)=[C^op,set]のこと
\newcommand{\Conv}{\mathrm{Conv}} %Convergence spaceの圏
\newcommand{\Unif}{\mathrm{Unif}} %一様空間と一様連続写像の圏
\newcommand{\Frm}{\mathrm{Frm}} %フレームとフレームの射
\newcommand{\Locale}{\mathrm{Locale}} %その反対圏
\newcommand{\Diff}{\mathrm{Diff}} %滑らかな多様体の圏
\newcommand{\Mfd}{\mathrm{Mfd}}
\newcommand{\LieAlg}{\mathrm{LieAlg}}
\newcommand{\Quiv}{\mathrm{Quiv}} %Quiverの圏
\newcommand{\B}{\mathcal{B}}
\newcommand{\Span}{\mathrm{Span}}
\newcommand{\Corr}{\mathrm{Corr}}
\newcommand{\Decat}{\mathrm{Decat}}
\newcommand{\Rep}{\mathrm{Rep}}
\newcommand{\Grpd}{\mathrm{Grpd}}
\newcommand{\sSet}{\mathrm{sSet}}
\newcommand{\Mod}{\mathrm{Mod}}
\newcommand{\SmoothMnf}{\mathrm{SmoothMnf}}
\newcommand{\coker}{\mathrm{coker}}

\newcommand{\Ord}{\mathrm{Ord}}
\newcommand{\eq}{\mathrm{eq}}
\newcommand{\coeq}{\mathrm{coeq}}
\newcommand{\act}{\mathrm{act}}

%%%%%%%%%%%%%%% 定理環境(足助先生ありがとうございます) %%%%%%%%%%%%%%%

\everymath{\displaystyle}
\renewcommand{\proofname}{\bf [証明]}
\renewcommand{\thefootnote}{\dag\arabic{footnote}} %足助さんからもらった.どうなるんだ?
\renewcommand{\qedsymbol}{$\blacksquare$}

\renewcommand{\labelenumi}{(\arabic{enumi})} %(1),(2),...がデフォルトであって欲しい
\renewcommand{\labelenumii}{(\alph{enumii})}
\renewcommand{\labelenumiii}{(\roman{enumiii})}

\newtheoremstyle{StatementsWithStar}% ?name?
{3pt}% ?Space above? 1
{3pt}% ?Space below? 1
{}% ?Body font?
{}% ?Indent amount? 2
{\bfseries}% ?Theorem head font?
{\textbf{.}}% ?Punctuation after theorem head?
{.5em}% ?Space after theorem head? 3
{\textbf{\textup{#1~\thetheorem{}}}{}\,$^{\ast}$\thmnote{(#3)}}% ?Theorem head spec (can be left empty, meaning ‘normal’)?
%
\newtheoremstyle{StatementsWithStar2}% ?name?
{3pt}% ?Space above? 1
{3pt}% ?Space below? 1
{}% ?Body font?
{}% ?Indent amount? 2
{\bfseries}% ?Theorem head font?
{\textbf{.}}% ?Punctuation after theorem head?
{.5em}% ?Space after theorem head? 3
{\textbf{\textup{#1~\thetheorem{}}}{}\,$^{\ast\ast}$\thmnote{(#3)}}% ?Theorem head spec (can be left empty, meaning ‘normal’)?
%
\newtheoremstyle{StatementsWithStar3}% ?name?
{3pt}% ?Space above? 1
{3pt}% ?Space below? 1
{}% ?Body font?
{}% ?Indent amount? 2
{\bfseries}% ?Theorem head font?
{\textbf{.}}% ?Punctuation after theorem head?
{.5em}% ?Space after theorem head? 3
{\textbf{\textup{#1~\thetheorem{}}}{}\,$^{\ast\ast\ast}$\thmnote{(#3)}}% ?Theorem head spec (can be left empty, meaning ‘normal’)?
%
\newtheoremstyle{StatementsWithCCirc}% ?name?
{6pt}% ?Space above? 1
{6pt}% ?Space below? 1
{}% ?Body font?
{}% ?Indent amount? 2
{\bfseries}% ?Theorem head font?
{\textbf{.}}% ?Punctuation after theorem head?
{.5em}% ?Space after theorem head? 3
{\textbf{\textup{#1~\thetheorem{}}}{}\,$^{\circledcirc}$\thmnote{(#3)}}% ?Theorem head spec (can be left empty, meaning ‘normal’)?
%
\theoremstyle{definition}
 \newtheorem{theorem}{定理}[section]
 \newtheorem{axiom}[theorem]{公理}
 \newtheorem{corollary}[theorem]{系}
 \newtheorem{proposition}[theorem]{命題}
 \newtheorem*{proposition*}{命題}
 \newtheorem{lemma}[theorem]{補題}
 \newtheorem*{lemma*}{補題}
 \newtheorem*{theorem*}{定理}
 \newtheorem{definition}[theorem]{定義}
 \newtheorem{example}[theorem]{例}
 \newtheorem{notation}[theorem]{記法}
 \newtheorem*{notation*}{記法}
 \newtheorem{assumption}[theorem]{仮定}
 \newtheorem{question}[theorem]{問}
 \newtheorem{counterexample}[theorem]{反例}
 \newtheorem{reidai}[theorem]{例題}
 \newtheorem{ruidai}[theorem]{類題}
 \newtheorem{problem}[theorem]{問題}
 \newtheorem{algorithm}[theorem]{算譜}
 \newtheorem*{solution*}{\bf{[解]}}
 \newtheorem{discussion}[theorem]{議論}
 \newtheorem{remark}[theorem]{注}
 \newtheorem{remarks}[theorem]{要諦}
 \newtheorem{image}[theorem]{描像}
 \newtheorem{observation}[theorem]{観察}
 \newtheorem{universality}[theorem]{普遍性} %非自明な例外がない.
 \newtheorem{universal tendency}[theorem]{普遍傾向} %例外が有意に少ない.
 \newtheorem{hypothesis}[theorem]{仮説} %実験で説明されていない理論.
 \newtheorem{theory}[theorem]{理論} %実験事実とその(さしあたり)整合的な説明.
 \newtheorem{fact}[theorem]{実験事実}
 \newtheorem{model}[theorem]{模型}
 \newtheorem{explanation}[theorem]{説明} %理論による実験事実の説明
 \newtheorem{anomaly}[theorem]{理論の限界}
 \newtheorem{application}[theorem]{応用例}
 \newtheorem{method}[theorem]{手法} %実験手法など,技術的問題.
 \newtheorem{history}[theorem]{歴史}
 \newtheorem{usage}[theorem]{用語法}
 \newtheorem{research}[theorem]{研究}
 \newtheorem{shishin}[theorem]{指針}
 \newtheorem{yodan}[theorem]{余談}
 \newtheorem{construction}[theorem]{構成}
% \newtheorem*{remarknonum}{注}
 \newtheorem*{definition*}{定義}
 \newtheorem*{remark*}{注}
 \newtheorem*{question*}{問}
 \newtheorem*{problem*}{問題}
 \newtheorem*{axiom*}{公理}
 \newtheorem*{example*}{例}
 \newtheorem*{corollary*}{系}
 \newtheorem*{shishin*}{指針}
 \newtheorem*{yodan*}{余談}
 \newtheorem*{kadai*}{課題}
%
\theoremstyle{StatementsWithStar}
 \newtheorem{definition_*}[theorem]{定義}
 \newtheorem{question_*}[theorem]{問}
 \newtheorem{example_*}[theorem]{例}
 \newtheorem{theorem_*}[theorem]{定理}
 \newtheorem{remark_*}[theorem]{注}
%
\theoremstyle{StatementsWithStar2}
 \newtheorem{definition_**}[theorem]{定義}
 \newtheorem{theorem_**}[theorem]{定理}
 \newtheorem{question_**}[theorem]{問}
 \newtheorem{remark_**}[theorem]{注}
%
\theoremstyle{StatementsWithStar3}
 \newtheorem{remark_***}[theorem]{注}
 \newtheorem{question_***}[theorem]{問}
%
\theoremstyle{StatementsWithCCirc}
 \newtheorem{definition_O}[theorem]{定義}
 \newtheorem{question_O}[theorem]{問}
 \newtheorem{example_O}[theorem]{例}
 \newtheorem{remark_O}[theorem]{注}
%
\theoremstyle{definition}
%
\raggedbottom
\allowdisplaybreaks
\usepackage{mathtools}
%\mathtoolsset{showonlyrefs=true} %labelを附した数式にのみ附番される設定.
%\usepackage{amsmath} %mathtoolsの内部で呼ばれるので要らない.
\usepackage{amsfonts} %mathfrak, mathcal, mathbbなど.
\usepackage{amsthm} %定理環境.
\usepackage{amssymb} %AMSFontsを使うためのパッケージ.
\usepackage{ascmac} %screen, itembox, shadebox環境.全てLATEX2εの標準機能の範囲で作られたもの.
\usepackage{comment} %comment環境を用いて,複数行をcomment outできるようにするpackage
\usepackage{wrapfig} %図の周りに文字をwrapさせることができる.詳細な制御ができる.
\usepackage[usenames, dvipsnames]{xcolor} %xcolorはcolorの拡張.optionの意味はdvipsnamesはLoad a set of predefined colors. forestgreenなどの色が追加されている.usenamesはobsoleteとだけ書いてあった.
\setcounter{tocdepth}{2} %目次に表示される深さ.2はsubsectionまで
\usepackage{multicol} %\begin{multicols}{2}環境で途中からmulticolumnに出来る.

\usepackage{url}
\usepackage[dvipdfmx,colorlinks,linkcolor=blue,urlcolor=blue]{hyperref} %生成されるPDFファイルにおいて、\tableofcontentsによって書き出された目次をクリックすると該当する見出しへジャンプしたり、さらには、\label{ラベル名}を番号で参照する\ref{ラベル名}やthebibliography環境において\bibitem{ラベル名}を文献番号で参照する\cite{ラベル名}においても番号をクリックすると該当箇所にジャンプする.囲み枠はダサいので,colorlinksで囲み廃止し,リンク自体に色を付けることにした.
\usepackage{pxjahyper} %pxrubrica同様,八登崇之さん.hyperrefは日本語pLaTeXに最適化されていないから,hyperrefとセットで,(u)pLaTeX+hyperref+dvipdfmxの組み合わせで日本語を含む「しおり」をもつPDF文書を作成する場合に必要となる機能を提供する
\definecolor{花緑青}{cmyk}{0.52,0.03,0,0.27}
\definecolor{サーモンピンク}{cmyk}{0,0.65,0.65,0.05}
\definecolor{暗中模索}{rgb}{0.2,0.2,0.2}

\usepackage{tikz}
\usetikzlibrary{positioning,automata} %automaton描画のため
\usepackage{tikz-cd}
\usepackage[all]{xy}
\def\objectstyle{\displaystyle} %デフォルトではxymatrix中の数式が文中数式モードになるので,それを直す.\labelstyleも同様にxy packageの中で定義されており,文中数式モードになっている.

\usepackage[version=4]{mhchem} %化学式をTikZで簡単に書くためのパッケージ.
\usepackage{chemfig} %化学構造式をTikZで描くためのパッケージ.
\usepackage{siunitx} %IS単位を書くためのパッケージ

\usepackage{ulem} %取り消し線を引くためのパッケージ
\usepackage{pxrubrica} %日本語にルビをふる.八登崇之(やとうたかゆき)氏による.

\usepackage{graphicx} %rotatebox, scalebox, reflectbox, resizeboxなどのコマンドや,図表の読み込み\includegraphicsを司る.graphics というパッケージもありますが,graphicx はこれを高機能にしたものと考えて結構です(ただし graphicx は内部で graphics を読み込みます)

\usepackage[breakable]{tcolorbox} %加藤晃史さんがフル活用していたtcolorboxを,途中改ページ可能で.
\tcbuselibrary{theorems} %https://qiita.com/t_kemmochi/items/483b8fcdb5db8d1f5d5e
\usepackage{enumerate} %enumerate環境を凝らせる.
\usepackage[top=15truemm,bottom=15truemm,left=10truemm,right=10truemm]{geometry} %足助さんからもらったオプション

%%%%%%%%%%%%%%% 環境マクロ %%%%%%%%%%%%%%%

\usepackage{listings} %ソースコードを表示できる環境.多分もっといい方法ある.
\usepackage{jvlisting} %日本語のコメントアウトをする場合jlistingが必要
\lstset{ %ここからソースコードの表示に関する設定.lstlisting環境では,[caption=hoge,label=fuga]などのoptionを付けられる.
%[escapechar=!]とすると,LaTeXコマンドを使える.
  basicstyle={\ttfamily},
  identifierstyle={\small},
  commentstyle={\smallitshape},
  keywordstyle={\small\bfseries},
  ndkeywordstyle={\small},
  stringstyle={\small\ttfamily},
  frame={tb},
  breaklines=true,
  columns=[l]{fullflexible},
  numbers=left,
  xrightmargin=0zw,
  xleftmargin=3zw,
  numberstyle={\scriptsize},
  stepnumber=1,
  numbersep=1zw,
  lineskip=-0.5ex
}
%\makeatletter %caption番号を「[chapter番号].[section番号].[subsection番号]-[そのsubsection内においてn番目]」に変更
%    \AtBeginDocument{
%    \renewcommand*{\thelstlisting}{\arabic{chapter}.\arabic{section}.\arabic{lstlisting}}
%    \@addtoreset{lstlisting}{section}
%    }
%\makeatother
\renewcommand{\lstlistingname}{算譜} %caption名を"program"に変更

\newtcolorbox{tbox}[3][]{%
colframe=#2,colback=#2!10,coltitle=#2!20!black,title={#3},#1}

%%%%%%%%%%%%%%% フォント %%%%%%%%%%%%%%%

\usepackage{textcomp, mathcomp} %Text Companionとは,T1 encodingに入らなかった文字群.これを使うためのパッケージ.\textsectionでブルバキに!
\usepackage[T1]{fontenc} %8bitエンコーディングにする.comp系拡張数学文字の動作が安定する.

%%%%%%%%%%%%%%% 数学記号のマクロ %%%%%%%%%%%%%%%

\newcommand{\abs}[1]{\lvert#1\rvert} %mathtoolsはこうやって使うのか!
\newcommand{\Abs}[1]{\left|#1\right|}
\newcommand{\norm}[1]{\|#1\|}
\newcommand{\Norm}[1]{\left\|#1\right\|}
%\newcommand{\brace}[1]{\{#1\}}
\newcommand{\Brace}[1]{\left\{#1\right\}}
\newcommand{\paren}[1]{\left(#1\right)}
\newcommand{\bracket}[1]{\langle#1\rangle}
\newcommand{\brac}[1]{\langle#1\rangle}
\newcommand{\Bracket}[1]{\left\langle#1\right\rangle}
\newcommand{\Brac}[1]{\left\langle#1\right\rangle}
\newcommand{\Square}[1]{\left[#1\right]}
\renewcommand{\o}[1]{\overline{#1}}
\renewcommand{\u}[1]{\underline{#1}}
\renewcommand{\iff}{\;\mathrm{iff}\;} %nLabリスペクト
\newcommand{\pp}[2]{\frac{\partial #1}{\partial #2}}
\newcommand{\ppp}[3]{\frac{\partial #1}{\partial #2\partial #3}}
\newcommand{\dd}[2]{\frac{d #1}{d #2}}
\newcommand{\floor}[1]{\lfloor#1\rfloor}
\newcommand{\Floor}[1]{\left\lfloor#1\right\rfloor}
\newcommand{\ceil}[1]{\lceil#1\rceil}

\newcommand{\iso}{\xrightarrow{\,\smash{\raisebox{-0.45ex}{\ensuremath{\scriptstyle\sim}}}\,}}
\newcommand{\wt}[1]{\widetilde{#1}}
\newcommand{\wh}[1]{\widehat{#1}}

\newcommand{\Lrarrow}{\;\;\Leftrightarrow\;\;}

%ノルム位相についての閉包 https://newbedev.com/how-to-make-double-overline-with-less-vertical-displacement
\makeatletter
\newcommand{\dbloverline}[1]{\overline{\dbl@overline{#1}}}
\newcommand{\dbl@overline}[1]{\mathpalette\dbl@@overline{#1}}
\newcommand{\dbl@@overline}[2]{%
  \begingroup
  \sbox\z@{$\m@th#1\overline{#2}$}%
  \ht\z@=\dimexpr\ht\z@-2\dbl@adjust{#1}\relax
  \box\z@
  \ifx#1\scriptstyle\kern-\scriptspace\else
  \ifx#1\scriptscriptstyle\kern-\scriptspace\fi\fi
  \endgroup
}
\newcommand{\dbl@adjust}[1]{%
  \fontdimen8
  \ifx#1\displaystyle\textfont\else
  \ifx#1\textstyle\textfont\else
  \ifx#1\scriptstyle\scriptfont\else
  \scriptscriptfont\fi\fi\fi 3
}
\makeatother
\newcommand{\oo}[1]{\dbloverline{#1}}

\DeclareMathOperator{\grad}{\mathrm{grad}}
\DeclareMathOperator{\rot}{\mathrm{rot}}
\DeclareMathOperator{\divergence}{\mathrm{div}}
\newcommand{\False}{\mathrm{False}}
\newcommand{\True}{\mathrm{True}}
\DeclareMathOperator{\tr}{\mathrm{tr}}
\newcommand{\M}{\mathcal{M}}
\newcommand{\cF}{\mathcal{F}}
\newcommand{\cD}{\mathcal{D}}
\newcommand{\fX}{\mathfrak{X}}
\newcommand{\fY}{\mathfrak{Y}}
\newcommand{\fZ}{\mathfrak{Z}}
\renewcommand{\H}{\mathcal{H}}
\newcommand{\fH}{\mathfrak{H}}
\newcommand{\bH}{\mathbb{H}}
\newcommand{\id}{\mathrm{id}}
\newcommand{\A}{\mathcal{A}}
% \renewcommand\coprod{\rotatebox[origin=c]{180}{$\prod$}} すでにどこかにある.
\newcommand{\pr}{\mathrm{pr}}
\newcommand{\U}{\mathfrak{U}}
\newcommand{\Map}{\mathrm{Map}}
\newcommand{\dom}{\mathrm{Dom}\;}
\newcommand{\cod}{\mathrm{Cod}\;}
\newcommand{\supp}{\mathrm{supp}\;}
\newcommand{\otherwise}{\mathrm{otherwise}}
\newcommand{\st}{\;\mathrm{s.t.}\;}
\newcommand{\lmd}{\lambda}
\newcommand{\Lmd}{\Lambda}
%%% 線型代数学
\newcommand{\Ker}{\mathrm{Ker}\;}
\newcommand{\Coker}{\mathrm{Coker}\;}
\newcommand{\Coim}{\mathrm{Coim}\;}
\newcommand{\rank}{\mathrm{rank}}
\newcommand{\lcm}{\mathrm{lcm}}
\newcommand{\sgn}{\mathrm{sgn}}
\newcommand{\GL}{\mathrm{GL}}
\newcommand{\SL}{\mathrm{SL}}
\newcommand{\alt}{\mathrm{alt}}
%%% 複素解析学
\renewcommand{\Re}{\mathrm{Re}\;}
\renewcommand{\Im}{\mathrm{Im}\;}
\newcommand{\Gal}{\mathrm{Gal}}
\newcommand{\PGL}{\mathrm{PGL}}
\newcommand{\PSL}{\mathrm{PSL}}
\newcommand{\Log}{\mathrm{Log}\,}
\newcommand{\Res}{\mathrm{Res}\,}
\newcommand{\on}{\mathrm{on}\;}
\newcommand{\hatC}{\hat{\C}}
\newcommand{\hatR}{\hat{\R}}
\newcommand{\PV}{\mathrm{P.V.}}
\newcommand{\diam}{\mathrm{diam}}
\newcommand{\Area}{\mathrm{Area}}
\newcommand{\Lap}{\Laplace}
\newcommand{\f}{\mathbf{f}}
\newcommand{\cR}{\mathcal{R}}
\newcommand{\const}{\mathrm{const.}}
\newcommand{\Om}{\Omega}
\newcommand{\Cinf}{C^\infty}
\newcommand{\ep}{\epsilon}
\newcommand{\dist}{\mathrm{dist}}
\newcommand{\opart}{\o{\partial}}
%%% 解析力学
\newcommand{\x}{\mathbf{x}}
%%% 集合と位相
\renewcommand{\O}{\mathcal{O}}
\renewcommand{\S}{\mathcal{S}}
\renewcommand{\U}{\mathcal{U}}
\newcommand{\V}{\mathcal{V}}
\renewcommand{\P}{\mathcal{P}}
\newcommand{\R}{\mathbb{R}}
\newcommand{\N}{\mathbb{N}}
\newcommand{\C}{\mathbb{C}}
\newcommand{\Z}{\mathbb{Z}}
\newcommand{\Q}{\mathbb{Q}}
\newcommand{\TV}{\mathrm{TV}}
\newcommand{\ORD}{\mathrm{ORD}}
\newcommand{\Tr}{\mathrm{Tr}\;}
\newcommand{\Card}{\mathrm{Card}\;}
\newcommand{\Top}{\mathrm{Top}}
\newcommand{\Disc}{\mathrm{Disc}}
\newcommand{\Codisc}{\mathrm{Codisc}}
\newcommand{\CoDisc}{\mathrm{CoDisc}}
\newcommand{\Ult}{\mathrm{Ult}}
\newcommand{\ord}{\mathrm{ord}}
\newcommand{\maj}{\mathrm{maj}}
%%% 形式言語理論
\newcommand{\REGEX}{\mathrm{REGEX}}
\newcommand{\RE}{\mathbf{RE}}

%%% Fourier解析
\newcommand*{\Laplace}{\mathop{}\!\mathbin\bigtriangleup}
\newcommand*{\DAlambert}{\mathop{}\!\mathbin\Box}
%%% Graph Theory
\newcommand{\SimpGph}{\mathrm{SimpGph}}
\newcommand{\Gph}{\mathrm{Gph}}
\newcommand{\mult}{\mathrm{mult}}
\newcommand{\inv}{\mathrm{inv}}
%%% 多様体
\newcommand{\Der}{\mathrm{Der}}
\newcommand{\osub}{\overset{\mathrm{open}}{\subset}}
\newcommand{\osup}{\overset{\mathrm{open}}{\supset}}
\newcommand{\al}{\alpha}
\newcommand{\K}{\mathbb{K}}
\newcommand{\Sp}{\mathrm{Sp}}
\newcommand{\g}{\mathfrak{g}}
\newcommand{\h}{\mathfrak{h}}
\newcommand{\Exp}{\mathrm{Exp}\;}
\newcommand{\Imm}{\mathrm{Imm}}
\newcommand{\Imb}{\mathrm{Imb}}
\newcommand{\codim}{\mathrm{codim}\;}
\newcommand{\Gr}{\mathrm{Gr}}
%%% 代数
\newcommand{\Ad}{\mathrm{Ad}}
\newcommand{\finsupp}{\mathrm{fin\;supp}}
\newcommand{\SO}{\mathrm{SO}}
\newcommand{\SU}{\mathrm{SU}}
\newcommand{\acts}{\curvearrowright}
\newcommand{\mono}{\hookrightarrow}
\newcommand{\epi}{\twoheadrightarrow}
\newcommand{\Stab}{\mathrm{Stab}}
\newcommand{\nor}{\mathrm{nor}}
\newcommand{\T}{\mathbb{T}}
\newcommand{\Aff}{\mathrm{Aff}}
\newcommand{\rsub}{\triangleleft}
\newcommand{\rsup}{\triangleright}
\newcommand{\subgrp}{\overset{\mathrm{subgrp}}{\subset}}
\newcommand{\Ext}{\mathrm{Ext}}
\newcommand{\sbs}{\subset}\newcommand{\sps}{\supset}
\newcommand{\In}{\mathrm{In}}
\newcommand{\Tor}{\mathrm{Tor}}
\newcommand{\p}{\mathfrak{p}}
\newcommand{\q}{\mathfrak{q}}
\newcommand{\m}{\mathfrak{m}}
\newcommand{\cS}{\mathcal{S}}
\newcommand{\Frac}{\mathrm{Frac}\,}
\newcommand{\Spec}{\mathrm{Spec}\,}
\newcommand{\bA}{\mathbb{A}}
\newcommand{\Sym}{\mathrm{Sym}}
\newcommand{\Ann}{\mathrm{Ann}}
%%% 代数的位相幾何学
\newcommand{\Ho}{\mathrm{Ho}}
\newcommand{\CW}{\mathrm{CW}}
\newcommand{\lc}{\mathrm{lc}}
\newcommand{\cg}{\mathrm{cg}}
\newcommand{\Fib}{\mathrm{Fib}}
\newcommand{\Cyl}{\mathrm{Cyl}}
\newcommand{\Ch}{\mathrm{Ch}}
%%% 数値解析
\newcommand{\round}{\mathrm{round}}
\newcommand{\cond}{\mathrm{cond}}
\newcommand{\diag}{\mathrm{diag}}
%%% 確率論
\newcommand{\calF}{\mathcal{F}}
\newcommand{\X}{\mathcal{X}}
\newcommand{\Meas}{\mathrm{Meas}}
\newcommand{\as}{\;\mathrm{a.s.}} %almost surely
\newcommand{\io}{\;\mathrm{i.o.}} %infinitely often
\newcommand{\fe}{\;\mathrm{f.e.}} %with a finite number of exceptions
\newcommand{\F}{\mathcal{F}}
\newcommand{\bF}{\mathbb{F}}
\newcommand{\W}{\mathcal{W}}
\newcommand{\Pois}{\mathrm{Pois}}
\newcommand{\iid}{\mathrm{i.i.d.}}
\newcommand{\wconv}{\rightsquigarrow}
\newcommand{\Var}{\mathrm{Var}}
\newcommand{\xrightarrown}{\xrightarrow{n\to\infty}}
\newcommand{\au}{\mathrm{au}}
\newcommand{\cT}{\mathcal{T}}
%%% 情報理論
\newcommand{\bit}{\mathrm{bit}}
%%% 積分論
\newcommand{\calA}{\mathcal{A}}
\newcommand{\calB}{\mathcal{B}}
\newcommand{\D}{\mathcal{D}}
\newcommand{\Y}{\mathcal{Y}}
\newcommand{\calC}{\mathcal{C}}
\renewcommand{\ae}{\mathrm{a.e.}\;}
\newcommand{\cZ}{\mathcal{Z}}
\newcommand{\fF}{\mathfrak{F}}
\newcommand{\fI}{\mathfrak{I}}
\newcommand{\E}{\mathcal{E}}
\newcommand{\sMap}{\sigma\textrm{-}\mathrm{Map}}
\DeclareMathOperator*{\argmax}{arg\,max}
\DeclareMathOperator*{\argmin}{arg\,min}
\newcommand{\cC}{\mathcal{C}}
\newcommand{\comp}{\complement}
\newcommand{\J}{\mathcal{J}}
\newcommand{\sumN}[1]{\sum_{#1\in\N}}
\newcommand{\cupN}[1]{\cup_{#1\in\N}}
\newcommand{\capN}[1]{\cap_{#1\in\N}}
\newcommand{\Sum}[1]{\sum_{#1=1}^\infty}
\newcommand{\sumn}{\sum_{n=1}^\infty}
\newcommand{\summ}{\sum_{m=1}^\infty}
\newcommand{\sumk}{\sum_{k=1}^\infty}
\newcommand{\sumi}{\sum_{i=1}^\infty}
\newcommand{\sumj}{\sum_{j=1}^\infty}
\newcommand{\cupn}{\cup_{n=1}^\infty}
\newcommand{\capn}{\cap_{n=1}^\infty}
\newcommand{\cupk}{\cup_{k=1}^\infty}
\newcommand{\cupi}{\cup_{i=1}^\infty}
\newcommand{\cupj}{\cup_{j=1}^\infty}
\newcommand{\limn}{\lim_{n\to\infty}}
\renewcommand{\l}{\mathcal{l}}
\renewcommand{\L}{\mathcal{L}}
\newcommand{\Cl}{\mathrm{Cl}}
\newcommand{\cN}{\mathcal{N}}
\newcommand{\Ae}{\textrm{-a.e.}\;}
\newcommand{\csub}{\overset{\textrm{closed}}{\subset}}
\newcommand{\csup}{\overset{\textrm{closed}}{\supset}}
\newcommand{\wB}{\wt{B}}
\newcommand{\cG}{\mathcal{G}}
\newcommand{\Lip}{\mathrm{Lip}}
\newcommand{\Dom}{\mathrm{Dom}}
%%% 数理ファイナンス
\newcommand{\pre}{\mathrm{pre}}
\newcommand{\om}{\omega}

%%% 統計的因果推論
\newcommand{\Do}{\mathrm{Do}}
%%% 数理統計
\newcommand{\bP}{\mathbb{P}}
\newcommand{\compsub}{\overset{\textrm{cpt}}{\subset}}
\newcommand{\lip}{\textrm{lip}}
\newcommand{\BL}{\mathrm{BL}}
\newcommand{\G}{\mathbb{G}}
\newcommand{\NB}{\mathrm{NB}}
\newcommand{\oR}{\o{\R}}
\newcommand{\liminfn}{\liminf_{n\to\infty}}
\newcommand{\limsupn}{\limsup_{n\to\infty}}
%\newcommand{\limn}{\lim_{n\to\infty}}
\newcommand{\esssup}{\mathrm{ess.sup}}
\newcommand{\asto}{\xrightarrow{\as}}
\newcommand{\Cov}{\mathrm{Cov}}
\newcommand{\cQ}{\mathcal{Q}}
\newcommand{\VC}{\mathrm{VC}}
\newcommand{\mb}{\mathrm{mb}}
\newcommand{\Avar}{\mathrm{Avar}}
\newcommand{\bB}{\mathbb{B}}
\newcommand{\bW}{\mathbb{W}}
\newcommand{\sd}{\mathrm{sd}}
\newcommand{\w}[1]{\widehat{#1}}
\newcommand{\bZ}{\mathbb{Z}}
\newcommand{\Bernoulli}{\mathrm{Bernoulli}}
\newcommand{\Mult}{\mathrm{Mult}}
\newcommand{\BPois}{\mathrm{BPois}}
\newcommand{\fraks}{\mathfrak{s}}
\newcommand{\frakk}{\mathfrak{k}}
\newcommand{\IF}{\mathrm{IF}}
\newcommand{\bX}{\mathbf{X}}
\newcommand{\bx}{\mathbf{x}}
\newcommand{\indep}{\raisebox{0.05em}{\rotatebox[origin=c]{90}{$\models$}}}
\newcommand{\IG}{\mathrm{IG}}
\newcommand{\Levy}{\mathrm{Levy}}
\newcommand{\MP}{\mathrm{MP}}
\newcommand{\Hermite}{\mathrm{Hermite}}
\newcommand{\Skellam}{\mathrm{Skellam}}
\newcommand{\Dirichlet}{\mathrm{Dirichlet}}
\newcommand{\Beta}{\mathrm{Beta}}
\newcommand{\bE}{\mathbb{E}}
\newcommand{\bG}{\mathbb{G}}
\newcommand{\MISE}{\mathrm{MISE}}
\newcommand{\logit}{\mathtt{logit}}
\newcommand{\expit}{\mathtt{expit}}
\newcommand{\cK}{\mathcal{K}}
\newcommand{\dl}{\dot{l}}
\newcommand{\dotp}{\dot{p}}
\newcommand{\wl}{\wt{l}}
%%% 函数解析
\renewcommand{\c}{\mathbf{c}}
\newcommand{\loc}{\mathrm{loc}}
\newcommand{\Lh}{\mathrm{L.h.}}
\newcommand{\Epi}{\mathrm{Epi}\;}
\newcommand{\slim}{\mathrm{slim}}
\newcommand{\Ban}{\mathrm{Ban}}
\newcommand{\Hilb}{\mathrm{Hilb}}
\newcommand{\Ex}{\mathrm{Ex}}
\newcommand{\Co}{\mathrm{Co}}
\newcommand{\sa}{\mathrm{sa}}
\newcommand{\nnorm}[1]{{\left\vert\kern-0.25ex\left\vert\kern-0.25ex\left\vert #1 \right\vert\kern-0.25ex\right\vert\kern-0.25ex\right\vert}}
\newcommand{\dvol}{\mathrm{dvol}}
\newcommand{\Sconv}{\mathrm{Sconv}}
\newcommand{\I}{\mathcal{I}}
\newcommand{\nonunital}{\mathrm{nu}}
\newcommand{\cpt}{\mathrm{cpt}}
\newcommand{\lcpt}{\mathrm{lcpt}}
\newcommand{\com}{\mathrm{com}}
\newcommand{\Haus}{\mathrm{Haus}}
\newcommand{\proper}{\mathrm{proper}}
\newcommand{\infinity}{\mathrm{inf}}
\newcommand{\TVS}{\mathrm{TVS}}
\newcommand{\ess}{\mathrm{ess}}
\newcommand{\ext}{\mathrm{ext}}
\newcommand{\Index}{\mathrm{Index}}
\newcommand{\SSR}{\mathrm{SSR}}
\newcommand{\vs}{\mathrm{vs.}}
\newcommand{\fM}{\mathfrak{M}}
\newcommand{\EDM}{\mathrm{EDM}}
\newcommand{\Tw}{\mathrm{Tw}}
\newcommand{\fC}{\mathfrak{C}}
\newcommand{\bn}{\mathbf{n}}
\newcommand{\br}{\mathbf{r}}
\newcommand{\Lam}{\Lambda}
\newcommand{\lam}{\lambda}
\newcommand{\one}{\mathbf{1}}
\newcommand{\dae}{\text{-a.e.}}
\newcommand{\td}{\text{-}}
\newcommand{\RM}{\mathrm{RM}}
%%% 最適化
\newcommand{\Minimize}{\text{Minimize}}
\newcommand{\subjectto}{\text{subject to}}
\newcommand{\Ri}{\mathrm{Ri}}
%\newcommand{\Cl}{\mathrm{Cl}}
\newcommand{\Cone}{\mathrm{Cone}}
\newcommand{\Int}{\mathrm{Int}}
%%% 圏
\newcommand{\varlim}{\varprojlim}
\newcommand{\Hom}{\mathrm{Hom}}
\newcommand{\Iso}{\mathrm{Iso}}
\newcommand{\Mor}{\mathrm{Mor}}
\newcommand{\Isom}{\mathrm{Isom}}
\newcommand{\Aut}{\mathrm{Aut}}
\newcommand{\End}{\mathrm{End}}
\newcommand{\op}{\mathrm{op}}
\newcommand{\ev}{\mathrm{ev}}
\newcommand{\Ob}{\mathrm{Ob}}
\newcommand{\Ar}{\mathrm{Ar}}
\newcommand{\Arr}{\mathrm{Arr}}
\newcommand{\Set}{\mathrm{Set}}
\newcommand{\Grp}{\mathrm{Grp}}
\newcommand{\Cat}{\mathrm{Cat}}
\newcommand{\Mon}{\mathrm{Mon}}
\newcommand{\CMon}{\mathrm{CMon}} %Comutative Monoid 可換単系とモノイドの射
\newcommand{\Ring}{\mathrm{Ring}}
\newcommand{\CRing}{\mathrm{CRing}}
\newcommand{\Ab}{\mathrm{Ab}}
\newcommand{\Pos}{\mathrm{Pos}}
\newcommand{\Vect}{\mathrm{Vect}}
\newcommand{\FinVect}{\mathrm{FinVect}}
\newcommand{\FinSet}{\mathrm{FinSet}}
\newcommand{\OmegaAlg}{\Omega$-$\mathrm{Alg}}
\newcommand{\OmegaEAlg}{(\Omega,E)$-$\mathrm{Alg}}
\newcommand{\Alg}{\mathrm{Alg}} %代数の圏
\newcommand{\CAlg}{\mathrm{CAlg}} %可換代数の圏
\newcommand{\CPO}{\mathrm{CPO}} %Complete Partial Order & continuous mappings
\newcommand{\Fun}{\mathrm{Fun}}
\newcommand{\Func}{\mathrm{Func}}
\newcommand{\Met}{\mathrm{Met}} %Metric space & Contraction maps
\newcommand{\Pfn}{\mathrm{Pfn}} %Sets & Partial function
\newcommand{\Rel}{\mathrm{Rel}} %Sets & relation
\newcommand{\Bool}{\mathrm{Bool}}
\newcommand{\CABool}{\mathrm{CABool}}
\newcommand{\CompBoolAlg}{\mathrm{CompBoolAlg}}
\newcommand{\BoolAlg}{\mathrm{BoolAlg}}
\newcommand{\BoolRng}{\mathrm{BoolRng}}
\newcommand{\HeytAlg}{\mathrm{HeytAlg}}
\newcommand{\CompHeytAlg}{\mathrm{CompHeytAlg}}
\newcommand{\Lat}{\mathrm{Lat}}
\newcommand{\CompLat}{\mathrm{CompLat}}
\newcommand{\SemiLat}{\mathrm{SemiLat}}
\newcommand{\Stone}{\mathrm{Stone}}
\newcommand{\Sob}{\mathrm{Sob}} %Sober space & continuous map
\newcommand{\Op}{\mathrm{Op}} %Category of open subsets
\newcommand{\Sh}{\mathrm{Sh}} %Category of sheave
\newcommand{\PSh}{\mathrm{PSh}} %Category of presheave, PSh(C)=[C^op,set]のこと
\newcommand{\Conv}{\mathrm{Conv}} %Convergence spaceの圏
\newcommand{\Unif}{\mathrm{Unif}} %一様空間と一様連続写像の圏
\newcommand{\Frm}{\mathrm{Frm}} %フレームとフレームの射
\newcommand{\Locale}{\mathrm{Locale}} %その反対圏
\newcommand{\Diff}{\mathrm{Diff}} %滑らかな多様体の圏
\newcommand{\Mfd}{\mathrm{Mfd}}
\newcommand{\LieAlg}{\mathrm{LieAlg}}
\newcommand{\Quiv}{\mathrm{Quiv}} %Quiverの圏
\newcommand{\B}{\mathcal{B}}
\newcommand{\Span}{\mathrm{Span}}
\newcommand{\Corr}{\mathrm{Corr}}
\newcommand{\Decat}{\mathrm{Decat}}
\newcommand{\Rep}{\mathrm{Rep}}
\newcommand{\Grpd}{\mathrm{Grpd}}
\newcommand{\sSet}{\mathrm{sSet}}
\newcommand{\Mod}{\mathrm{Mod}}
\newcommand{\SmoothMnf}{\mathrm{SmoothMnf}}
\newcommand{\coker}{\mathrm{coker}}

\newcommand{\Ord}{\mathrm{Ord}}
\newcommand{\eq}{\mathrm{eq}}
\newcommand{\coeq}{\mathrm{coeq}}
\newcommand{\act}{\mathrm{act}}

%%%%%%%%%%%%%%% 定理環境(足助先生ありがとうございます) %%%%%%%%%%%%%%%

\everymath{\displaystyle}
\renewcommand{\proofname}{\bf [証明]}
\renewcommand{\thefootnote}{\dag\arabic{footnote}} %足助さんからもらった.どうなるんだ?
\renewcommand{\qedsymbol}{$\blacksquare$}

\renewcommand{\labelenumi}{(\arabic{enumi})} %(1),(2),...がデフォルトであって欲しい
\renewcommand{\labelenumii}{(\alph{enumii})}
\renewcommand{\labelenumiii}{(\roman{enumiii})}

\newtheoremstyle{StatementsWithStar}% ?name?
{3pt}% ?Space above? 1
{3pt}% ?Space below? 1
{}% ?Body font?
{}% ?Indent amount? 2
{\bfseries}% ?Theorem head font?
{\textbf{.}}% ?Punctuation after theorem head?
{.5em}% ?Space after theorem head? 3
{\textbf{\textup{#1~\thetheorem{}}}{}\,$^{\ast}$\thmnote{(#3)}}% ?Theorem head spec (can be left empty, meaning ‘normal’)?
%
\newtheoremstyle{StatementsWithStar2}% ?name?
{3pt}% ?Space above? 1
{3pt}% ?Space below? 1
{}% ?Body font?
{}% ?Indent amount? 2
{\bfseries}% ?Theorem head font?
{\textbf{.}}% ?Punctuation after theorem head?
{.5em}% ?Space after theorem head? 3
{\textbf{\textup{#1~\thetheorem{}}}{}\,$^{\ast\ast}$\thmnote{(#3)}}% ?Theorem head spec (can be left empty, meaning ‘normal’)?
%
\newtheoremstyle{StatementsWithStar3}% ?name?
{3pt}% ?Space above? 1
{3pt}% ?Space below? 1
{}% ?Body font?
{}% ?Indent amount? 2
{\bfseries}% ?Theorem head font?
{\textbf{.}}% ?Punctuation after theorem head?
{.5em}% ?Space after theorem head? 3
{\textbf{\textup{#1~\thetheorem{}}}{}\,$^{\ast\ast\ast}$\thmnote{(#3)}}% ?Theorem head spec (can be left empty, meaning ‘normal’)?
%
\newtheoremstyle{StatementsWithCCirc}% ?name?
{6pt}% ?Space above? 1
{6pt}% ?Space below? 1
{}% ?Body font?
{}% ?Indent amount? 2
{\bfseries}% ?Theorem head font?
{\textbf{.}}% ?Punctuation after theorem head?
{.5em}% ?Space after theorem head? 3
{\textbf{\textup{#1~\thetheorem{}}}{}\,$^{\circledcirc}$\thmnote{(#3)}}% ?Theorem head spec (can be left empty, meaning ‘normal’)?
%
\theoremstyle{definition}
 \newtheorem{theorem}{定理}[section]
 \newtheorem{axiom}[theorem]{公理}
 \newtheorem{corollary}[theorem]{系}
 \newtheorem{proposition}[theorem]{命題}
 \newtheorem*{proposition*}{命題}
 \newtheorem{lemma}[theorem]{補題}
 \newtheorem*{lemma*}{補題}
 \newtheorem*{theorem*}{定理}
 \newtheorem{definition}[theorem]{定義}
 \newtheorem{example}[theorem]{例}
 \newtheorem{notation}[theorem]{記法}
 \newtheorem*{notation*}{記法}
 \newtheorem{assumption}[theorem]{仮定}
 \newtheorem{question}[theorem]{問}
 \newtheorem{counterexample}[theorem]{反例}
 \newtheorem{reidai}[theorem]{例題}
 \newtheorem{ruidai}[theorem]{類題}
 \newtheorem{problem}[theorem]{問題}
 \newtheorem{algorithm}[theorem]{算譜}
 \newtheorem*{solution*}{\bf{[解]}}
 \newtheorem{discussion}[theorem]{議論}
 \newtheorem{remark}[theorem]{注}
 \newtheorem{remarks}[theorem]{要諦}
 \newtheorem{image}[theorem]{描像}
 \newtheorem{observation}[theorem]{観察}
 \newtheorem{universality}[theorem]{普遍性} %非自明な例外がない.
 \newtheorem{universal tendency}[theorem]{普遍傾向} %例外が有意に少ない.
 \newtheorem{hypothesis}[theorem]{仮説} %実験で説明されていない理論.
 \newtheorem{theory}[theorem]{理論} %実験事実とその(さしあたり)整合的な説明.
 \newtheorem{fact}[theorem]{実験事実}
 \newtheorem{model}[theorem]{模型}
 \newtheorem{explanation}[theorem]{説明} %理論による実験事実の説明
 \newtheorem{anomaly}[theorem]{理論の限界}
 \newtheorem{application}[theorem]{応用例}
 \newtheorem{method}[theorem]{手法} %実験手法など,技術的問題.
 \newtheorem{history}[theorem]{歴史}
 \newtheorem{usage}[theorem]{用語法}
 \newtheorem{research}[theorem]{研究}
 \newtheorem{shishin}[theorem]{指針}
 \newtheorem{yodan}[theorem]{余談}
 \newtheorem{construction}[theorem]{構成}
% \newtheorem*{remarknonum}{注}
 \newtheorem*{definition*}{定義}
 \newtheorem*{remark*}{注}
 \newtheorem*{question*}{問}
 \newtheorem*{problem*}{問題}
 \newtheorem*{axiom*}{公理}
 \newtheorem*{example*}{例}
 \newtheorem*{corollary*}{系}
 \newtheorem*{shishin*}{指針}
 \newtheorem*{yodan*}{余談}
 \newtheorem*{kadai*}{課題}
%
\theoremstyle{StatementsWithStar}
 \newtheorem{definition_*}[theorem]{定義}
 \newtheorem{question_*}[theorem]{問}
 \newtheorem{example_*}[theorem]{例}
 \newtheorem{theorem_*}[theorem]{定理}
 \newtheorem{remark_*}[theorem]{注}
%
\theoremstyle{StatementsWithStar2}
 \newtheorem{definition_**}[theorem]{定義}
 \newtheorem{theorem_**}[theorem]{定理}
 \newtheorem{question_**}[theorem]{問}
 \newtheorem{remark_**}[theorem]{注}
%
\theoremstyle{StatementsWithStar3}
 \newtheorem{remark_***}[theorem]{注}
 \newtheorem{question_***}[theorem]{問}
%
\theoremstyle{StatementsWithCCirc}
 \newtheorem{definition_O}[theorem]{定義}
 \newtheorem{question_O}[theorem]{問}
 \newtheorem{example_O}[theorem]{例}
 \newtheorem{remark_O}[theorem]{注}
%
\theoremstyle{definition}
%
\raggedbottom
\allowdisplaybreaks
\usepackage[math]{anttor}
\begin{document}
\tableofcontents

\chapter{モデルの枠組み}

\begin{quotation}
    Probability theory is concerned with mathematical models of phenomena that exhibit randomness, or more generally phenomena about which one has incomplete information.
    確率論の概念を数学的な公理に落とし込む段階に於ける
    形式科学的な議論をまとめる.\footnote{Notice that in this respect probability theory has a similar status as (other(?!)) theories of physics: there is a mathematical model (measure theory here as the model for probability theory, or for instance symplectic geometry as a model for classical mechanics) which can be studied all in itself, and then there is in addition a more or less concrete idea of how from that model one may deduce statements about the observable world (the average outcome of a dice role using probability theory, or the observability of the next solar eclipse using Hamiltonian mechanics).\url{https://ncatlab.org/nlab/show/probability+theory}}
    ランダム性・不確定性に関係する概念のうち,例えば,独立性以前のエントロピーなどの概念は殆ど数学的に確定する.
    \begin{description}
        \item[試行] とは,類等式の双対のような\footnote{類等式は群作用による終域の分解だが,試行は可測関数による始域の分解},標本空間の分割で,
        事象とはそのうちの1つである.この分割上の割合の分配なる観念を人類は確率と呼ぶが,これは正規化された測度に他ならない.\footnote{もっと無限な概念が扱えたなら,測度の語ももっと細かく分類すべきかもしれない.}
        \item[確率変数] とは標本空間上の実線型空間に入る射で,標本空間上に同値類による分割を定める.
        $\R^n$-値点の考え方を導入することで,数学理論を観測値の概念に持ち込むことができる.
        その背景には射があり,線型代数・微分積分学の関手がそこまで還元しているのである.
        こうして,$X$は$\R^n$の元であるとも見るし,射$X:\Om\to\R^n$とも見る.
        \item[確率] とは標本空間上の非負で正規化された加法的集合関数で,\textbf{確率分布}とは確率変数によるこの測度の押し出しである.この対応を分布から辿る時「確率変数が分布に従う」という(当該の分布を押し出すような確率変数を1つ取る,という条件と同値).
        \item[エントロピー] とは,系で考えうる全ての試行の期待驚愕度の上限である.
    \end{description}
    これらの類比はFréchetの研究において初めて完全に示された.

    「有限試行によって確率論の考え方を理解したならば,解析学の手法を用いて,
    現代確率論の対象である無限試行に進むのは容易である.」注意力の質の方が重要である.
    \begin{quote}
        統計モデルとしての多様な確率分布族と,それらに対する種々の統計推測法について解説する.多くの例を通じ,受講者が確率統計の基本事項に習熟することを目標とする.確率的な構造の表現からはじめ,確率の性質,確率変数と確率分布,独立性等の用語を準備し,離散確率分布とその例と計算法,連続分布とその例,確率変数の期待値,変数変換の公式,混合分布,指数型分布族,多次元分布の基礎について解説する.前半の確率の基礎概念の導入の後,確率モデルの推定について紹介する.不偏推定が統計推測の数理的構造を理解するための例となる.十分性,因子分解定理,完備性,ラオ•ブラックウェルの定理,レーマン•シェフェの定理,統計的決定理論の枠組み,ベイズ推定について説明する予定である.
    \end{quote}
\end{quotation}

\section{確率の数理的構造:Kolmogorovの公理}

\begin{tcolorbox}[colframe=ForestGreen, colback=ForestGreen!10!white,breakable,colbacktitle=ForestGreen!40!white,coltitle=black,fonttitle=\bfseries\sffamily,
title=測度論的確率論という枠組み]
    確率空間は測度空間の一種と見る.
    確率変数は関数空間の元と見て,平均は作用素と見て構成する.\footnote{こうして,量子論と確率論が関数解析を舞台として合流する.}
\end{tcolorbox}

\subsection{Kolmogorovの公理}

\begin{notation}[mutually exclusive]\mbox{}
    \begin{enumerate}
        \item 集合$A,B$が排反であるとは,互いに素であることをいう.この時の和事象を$A+B,A\coprod B$で表し直和という.\cite{伊藤清}では直和も$\sum_{i=1}^n A_i$と表す.$P(A+B)=P(A)+P(B),P(A\cup B)=P(A)+P(B)-P(A\cap B)$と書き分けられる.
        \item 族$(A_i)_{i\in I}$が排反であるとは,$\forall_{i,j\in I}\;i\ne j\Rightarrow A_i\cap A_j=\emptyset$であることをいう.
        \item $A\subset B$のときの差事象$A\setminus B$を固有差といい,$B-A$と書く.
    \end{enumerate}
\end{notation}

\begin{definition}[sample space, sample, event, total event]\mbox{}
    \begin{enumerate}
        \item \textbf{標本空間}$\Omega$の元$\omega\in\Omega$を標本(点)という.\footnote{見本空間,見本点ともいう.}
        \item $\sigma$-加法族$\calF\subset P(\Omega)$の元を\textbf{事象}と呼ぶ.\footnote{部分集合を条件や関係だけでなく事象と捉えるのは,集合論の初歩でもやるようだ.}事象としての$\Omega\in\calF$を全事象という.
    \end{enumerate}
\end{definition}

\begin{definition}[probability]
    $\sigma$-代数$\mathcal{F}$に対して,次を満たす測度$P:\mathcal{F}\to[0,1]$を\textbf{確率測度}または\textbf{試行$T$の確率法則}という.
    \begin{enumerate}
        \item $P(A)\ge 0\;(\forall_{A\in\calF})$.
        \item $P(\Omega)=1$.
        \item ($\sigma$-加法性) $P(\cup_{i\in\N}A_i)=\sum_{i\in\N}P(A_i)$,ただし族$(A_i):\N\to\calF$は$\forall_{i\ne j}\;A_i\cap A_j=\emptyset$を満たす.
    \end{enumerate}
\end{definition}

\begin{definition}[random variable, probability distribution / law]
    確率空間を$(\Omega,\calF)$,$(\X,\B)$を可測空間とする.
    \begin{enumerate}
        \item \textbf{$\X$-値確率変数}とは,可測関数$X:\Omega\to\X$をいう.
        \item これに沿って引き起こされる像測度$X_*P=P^X:P(\X)\supset\B\to[0,1]$を$P^X(B):=P(X^{-1}(B))$で定める:\footnote{逆像写像$X^*$により,なんとか反変に見えるから許したが,nLabでは$f_*\mu$で表されている.}
        \[\xymatrix{
            &{[0,1]}\\
            P(\Omega)\ar[ur]^-{P}&&P(\X)\ar[ul]_-{P^X}\ar[ll]_-{X^*}\\
            \Omega\ar[u]^-{P}\ar[rr]_-X&&\X\ar[u]_-P
        }\]
        これを\textbf{$X$の確率分布}または\textbf{確率法則}という.
        \item 特に,確率$P$は,恒等写像$\id_\Omega:\Omega\to\Omega$についての確率分布$P^{\id_\Omega}$である.
        \item 像$\Im X$も同様の記法$\Om^X$で表し,これを\textbf{$X$の標本空間}という.
        \item $\Om^X$は\textbf{混合試行}$T_X$の標本空間と考える.$X$の定める同一視による商空間を考えているために「混合」という.
    \end{enumerate}
\end{definition}
\begin{remarks}[Kolmogorov 1933]
    すなわち,確率空間から出る射が確率変数で,それにより押し出される測度が確率分布である.
\end{remarks}
\begin{notation}
    $a\in\X$として,確率$P(X^{-1}(a))\in[0,1]$を$P(X=a)$と表す.
    $B\subset\X$として,確率$P(X^{-1}(B))\in[0,1]$を$P(X\in B)$と表す.
    これは$E[1_B(X)]$にも等しい.
\end{notation}

\begin{definition}[joint distribution, marginal distribution]\mbox{}
    \begin{enumerate}
        \item 確率変数の列$X_1,\cdots,X_n$に対して,積写像$(X_1,\cdots,X_n)$を\textbf{同時分布}または\textbf{結合分布}という.
        \item 同時分布から一部の確率変数を消去してえる分布を\textbf{周辺分布}という.\footnote{表の欄外(margin)に行や列の和を記載することから周辺(marginal)と呼ばれるようになった.}
    \end{enumerate}
\end{definition}

\begin{definition}[可積分]\mbox{}
    \begin{enumerate}
        \item 実確率変数$X:\Omega\to\R$が可積分であるとは,$E[\abs{X}]=\int_{\Omega}\abs{X(\omega)}dP(\omega)<\infty$すなわち,$X\in L^1$であることをいう.
        \item $L^1(\Om,\mu):=\Brace{f\in\Map(\Om,\o{\R})\mid\int_\Om\abs{f(\om)}\mu(d\om)<\infty.}$と表す.
    \end{enumerate}
\end{definition}

\begin{definition}[stochastic process]
    確率変数の族$(X_t)_{t\in T}$を,特に$T$が全順序集合のとき,\textbf{確率過程}と言う.
\end{definition}

\begin{lemma}
    関数の族$\{X_n\}\subset\Map(\Om,\R)$について,次の2条件は同値.
    \begin{enumerate}
        \item $(X_n)$は確率過程である.
        \item $X:\N\times\Om\to\R$は$\B(\N)\times\F/\B(\R)$-可測である.
    \end{enumerate}
\end{lemma}

\subsection{$\sigma$-加法性}

\begin{tcolorbox}[colframe=ForestGreen, colback=ForestGreen!10!white,breakable,colbacktitle=ForestGreen!40!white,coltitle=black,fonttitle=\bfseries\sffamily,
title=]
    事象とその確率なる概念が数学では$\sigma$-加法性に当たることを考察するのにもっとも直感的な例として,
    コイントスを繰り返す行為を一つの空間内で表現したい場合,部分集合の族は自然$\sigma$-代数をなす.
\end{tcolorbox}

\begin{example}[確率空間を取り直すのではなく,$\sigma$-加法族を発展させる]
    素朴に,$n$回試行を行ったときの標本空間は$\Om_n:=\Map([n],2)$であり,事象のなす$\sigma$-加法族は$P(\Om_n)$である.

    一方で,$\Om:=\Map(\N,2)$を全体集合とすると,$n$回試行を行ったときの事象のなす部分集合族は,
    $\om^n\in\Om_n$に対して,$A(\om^n):=\Brace{\om\in\Om\mid\forall_{i\in[n]}\;\om_i=\om_i^n}$とすると,
    対応$A_n:P(\Om_n)\to P(\Om);B\mapsto A(B):=\cup_{\om^n\in B}A(\om^n)$を用いて,$\cF_n:=\Im A_n\subset P(\Om)$は,$\sigma$-加法族の構造を持つ.
    逆に,それ以外の構造は見出し難い.
\end{example}

\begin{definition}[cylinder set]
    この枠組みを一般化する.可測空間$(S,\cS)$に値を取る試行を繰り返す試行の標本空間を$\Om:=S^\N$とする.
    その部分空間$C\subset\Om$であって,ある$n\in\N$を用いて
    \[C=C(t_1,\cdots,t_n;A_1,\cdots,A_n):=\Brace{\om\in\Om\mid\forall_{i\in[n]}\;\om_{t_i}\in A_i}\qquad\text{ただし,}t_i\in\N,A_i\in\cS\]
    と表せるとき,これを\textbf{柱状集合}という.
    これは第$t_i$回の試行の結果を指定して得られる事象である.
    $\Om$上の$\sigma$-代数としては,柱状集合をすべて含むものを取る.
\end{definition}

\section{有界測度論の確率論的解釈}\label{sec-event}

\begin{tcolorbox}[colframe=ForestGreen, colback=ForestGreen!10!white,breakable,colbacktitle=ForestGreen!40!white,coltitle=black,fonttitle=\bfseries\sffamily,
title=有界測度論]
    測度とは,部分集合=事象に対して「全体に占める割合」を定める.
    類等式ではないが,全事象をどのように分解するかが特徴量となる(エントロピー).
    そこで,有界測度論の枠組みを確率論的に解釈する方法を以下に示す.
\end{tcolorbox}

\subsection{極限の定義}

\begin{tcolorbox}[colframe=ForestGreen, colback=ForestGreen!10!white,breakable,colbacktitle=ForestGreen!40!white,coltitle=black,fonttitle=\bfseries\sffamily,
title=事象列には極限なる演算を定義したい]
    極限とは「十分遠くでは変わらない」ことである.距離空間では点列の極限が距離で定義できたが,
    $\F$には擬距離しか定まらない.
    しかし,$\F$には特別に,列の極限の定義が,論理によって定められる.
    そしてこれは,$(\F,\triangle)$が距離を定めるときの「点列」の定義に一致する.
    「完備」は,ここに於て通じ合っているが,いずれにしろ距離概念が暗黙にあるということを捉える試みがいまだに続いている.
\end{tcolorbox}

\begin{definition}[上極限事象,下極限事象,極限事象]
    事象列$(A_n):\N\to\calF$に対して,次の事象が定義できる.
    \begin{enumerate}
        \item $\limsup_{n\to\infty}A_n:=\cap^\infty_{m=1}\cup_{m\ge n}A_m$を上極限事象という.「事象列のうち無限個が起きる」という条件i.o.を表す.
        \item $\liminf_{n\to\infty}A_n:=\cup^\infty_{m=1}\cap_{m\ge n}A_m$を下極限事象という.「事象列のうちある番号から先の事象が全て起きる」という条件f.e.を表す.
        \item 2つの集合が一致するとき,集合列$(A_n)$は\textbf{収束する}といい,$\lim_{n\to\infty}A_n$を\textbf{極限集合}という.
    \end{enumerate}
\end{definition}

\begin{lemma}[事象としての意味論:infinitely often, with finite exceptions]
    $\{A_n\}_{n\in\N}\subset P(X)$を集合列とする.
    \begin{enumerate}
        \item $\limsup_{n\to\infty}A_n=\Brace{x\in X\mid \#\{n\in\N\mid x\in A_n\}=\infty}$.
        \item $\liminf_{n\to\infty}A_n=\Brace{x\in X\mid \#\{n\in\N\mid x\notin A_n\}<\infty}$.
    \end{enumerate}
\end{lemma}
\begin{proof}\mbox{}
    \begin{enumerate}
        \item \begin{align*}
            xが\#\{n\in\N\mid x\in A_n\}=\infty を満たす&\Leftrightarrow \{n\in\N\mid x\in A_n\}は非有界である\\
            &\Leftrightarrow \forall_{n\in\N}\;\exists_{m\ge n}\;x\in A_m\Leftrightarrow x\in\cap_{n\in\N}\cup_{m\ge n}A_m.
        \end{align*}
        \item \begin{align*}
            xが\#\{n\in\N\mid x\notin A_n\}<\infty を満たす&\Leftrightarrow \{n\in\N\mid x\notin A_n\}は有界である\\
            &\Leftrightarrow\exists_{n\in\N}\;\forall_{m\ge n}\;x\in A_m\Leftrightarrow x\in\cup_{n\in\N}\cap_{m\ge n}A_m.
        \end{align*}
    \end{enumerate}
\end{proof}

\begin{lemma}[$\sigma$-代数の構造を求めて]
    有限な測度を備えた空間$(X,\F,\mu)$について,
    \begin{enumerate}
        \item 対称差について,$d(A,B):=A\triangle B\;(A,B\in\F)$と定めると,これは擬距離関数となる.
        \item $\F$が$\sigma$-代数として完備であるとき,$(\F,\triangle)$の距離等化によって得る距離空間は完備である.
    \end{enumerate}
\end{lemma}
\begin{proof}\mbox{}
    \begin{enumerate}
        \item $\mu$は有限としたから,$\Im d\subset[0,\infty)$である.
        対称性と$d(A,A)=0$は明らか.三角不等式については,$A,B,C\in\F$について,
        \[A\triangle C=(A\setminus C)\setminus B\sqcup (A\setminus C)\cap B\sqcup (C\setminus A)\setminus B\sqcup(C\setminus A)\cap B\]と直和分解できるが,
        それぞれについて,
        \[(A\setminus C)\setminus B\subset A\setminus B,\quad(A\setminus C)\cap B\subset B\setminus C,\quad (C\setminus A)\setminus B\subset C\setminus B,\quad(C\setminus A)\cap B\subset B\setminus A\]
        より,$\mu(A\triangle C)\le\mu(A\triangle B)+\mu(B\triangle C)$.
        \item 略.
    \end{enumerate}
\end{proof}
\begin{remarks}
    はじめは技術的で煩瑣な手続きに思えた測度空間の完備化であるが,これにより$\mu:\F\to[0,\infty)$が完備距離空間の射に,これが定める積分が有界線型作用素$L^p(X)\to\R$となる.
\end{remarks}

\subsection{連続写像としての測度}

\begin{tcolorbox}[colframe=ForestGreen, colback=ForestGreen!10!white,breakable,colbacktitle=ForestGreen!40!white,coltitle=black,fonttitle=\bfseries\sffamily,
title=]
    確率の素朴な加法性に対して,その$\sigma$-加法性も考えたいとは,数学的には連続性を考えたいということである.
    $\F$内に定義した極限の構造に対して,これを保存することをいう.

    測度$\mu:\F\to[0,1]$は,$\F$上の擬距離$d(A,B):=\mu(A\triangle B)$について連続であることは示せる.
    しかし,前節で定義した「集合列の極限」が定める位相は,この擬距離が誘導する位相とは異なる.
    注\ref{remark-topology-of-sigma-algebra}のように,$\infty$と$0$は区別できないフィルターとなっている.
\end{tcolorbox}

\begin{proposition}[測度の連続性と劣加法性]\label{prop-character-of-measurable-sets}
    可測空間$(\Omega,\calF)$の可測集合の列$(A_i):\N\to\calF$について,次が成り立つ.
    \begin{enumerate}
        \item 次の4条件は同値.
        \begin{enumerate}[(a)]
            \item $\mu:\F\to[0,1]$は測度である.
            \item 単調増加列$(A_n)$に対して,$\lim_{n\to\infty}\mu(A_n)=\mu(\cup_{n=1}^\infty A_n)$.
            \item 単調減少列$(A_n)$に対して,$\lim_{n\to\infty}\mu(A_n)=\mu(\cap_{n=1}^\infty A_n)$.
            \item $\emptyset$に収束する単調減少列$(A_n)$に対して,$\lim_{n\to\infty}P(A_n)=0$.
        \end{enumerate}
        \item (subadditivity) 測度$\mu$に対して,$\mu(\cup^n_{i=1}A_i)\le\sum^n_{i=1}\mu(A_i)\;(n\in\N)$.
    \end{enumerate}
\end{proposition}
\begin{proof}\mbox{}
    \begin{enumerate}
        \item $B_n:=A_n-A_{n-1}$とおくと(ただし$A_0=\emptyset$とする),$(B_n)_{n\in\N}$は排反であり,$\cup_{n\in\N}A_n=\cup_{n\in\N}B_n$.
        よって,\begin{align*}
            P\paren{\cup_{n=1}^\infty A_n}&=\sum^\infty_{n=1}P(B_n)\\
            &=\lim_{n\to\infty}\sum^n_{k=1}P(B_n)=\lim_{n\to\infty}P(A_n)
        \end{align*}
        \item 族$(A_n^\complement)_{n\in\N}$が単調増加列であることに気をつけて,
        \begin{align*}
            P\paren{\cap^\infty_{n=1}A_n}&=1-P\paren{\cup^\infty_{n=1}A_n^\complement}\\
            &=1-\lim_{n\to\infty}P(A_n^\complement)=1-\paren{1-\lim_{n\to\infty}P(A_n)}=\lim_{n\to\infty}P(A_n).
        \end{align*}
        \item $B_n:=\cup_{k=1}^nA_k$とおくと,これは単調列である.極限は不等式を保存するから,
        \[P(\cup^\infty_{n=1}A_n)=\lim_{n\to\infty}P(B_n)\le\lim_{n\to\infty}\sum^n_{k=1}P(A_k).\]
    \end{enumerate}
\end{proof}
\begin{remarks}[連続写像としての測度]
    確率測度$\mu:\F\to[0,1]$は,の射を定める.
    一方で,一般の測度$\mu:\F\to[0,\infty)$については,条件$\mu(A_1)<\infty$が必要となる.
\end{remarks}
\begin{remark}\label{remark-topology-of-sigma-algebra}
    (2)で$\mu(A_1)=\infty$を許すならば,$A_i:=2^n\N$とすることで,$\lim_{n\to\infty}\mu(A_n)=\infty,\mu(\cap_{n\in\N}A_n)=0$を満たす列が作れてしまう.
\end{remark}

\begin{tbox}{red}{}
    測度の連続性が実際にどういう位相について連続なのかを考え途中.
    特に,(1)と(2)で,$\mu(A_1)<\infty$の条件が必要になる非対称性がどこから来るのか探していて,おそらく$[0,\infty]$に入れる位相から.
    これは収束空間の知識が必要か?2つの「完備」の概念は,フィルターの意味で一緒なのではないか?
\end{tbox}

\subsection{Borel-Cantelli}

\begin{tcolorbox}[colframe=ForestGreen, colback=ForestGreen!10!white,breakable,colbacktitle=ForestGreen!40!white,coltitle=black,fonttitle=\bfseries\sffamily,
title=有界測度論の初等的な結論に,確率論的な解釈を与えることが出来る.]
    $\limsup:\F^\N\to\F$の性質を考える.
\end{tcolorbox}

\begin{proposition}[集合に関するFatouの補題]
    $\limsup:\F^\N\to\F$は次を満たす.
    \begin{enumerate}
        \item $\F$の列$(A_n)$について,$\mu\paren{\liminf_{n\to\infty}A_n}\le\liminf_{n\to\infty}\mu(A_n)$.
        \item $\F$の列$(A_n)$について,$\mu(\cup_{n\in\N}A_n)<\infty$ならば,$\limsup_{n\to\infty}\mu(A_n)\le\mu\paren{\limsup_{n\to\infty}A_n}$.
    \end{enumerate}
\end{proposition}

\begin{theorem}[Borel-Cantelli lemma]\label{lemma-Borel-Cantellii}\mbox{}
    \begin{enumerate}
        \item 列$(A_i):\N\to\calF$が独立であろうと無かろうと,一般の測度$\mu$について,\[\sum^\infty_{n=1}\mu(A_n)<\infty\quad\Rightarrow\quad \mu(\limsup_{n\to\infty}A_n)=0,\quad \mu(\liminf_{n\to\infty}A_n^\complement)=\mu(X)\footnote{確率測度ならば$\mu(X)=1$である}.\]
        \item 列$(A_i):\N\to\calF$が独立であるとき,$\sum^\infty_{n=1}P(A_n)=\infty\quad\Rightarrow\quad P(\limsup_{n\to\infty}A_n)=1,\quad P(\liminf_{n\to\infty}A_n^\complement)=0$.
    \end{enumerate}
\end{theorem}
\begin{proof}\mbox{}
    \begin{enumerate}
        \item $B_m:=\cup^\infty_{n=m}A_n$と定めると,これは単調減少列であるから$\mu(\cap^\infty_{m=1}B_m)=\lim_{m\to\infty}\mu(B_m)$.
        和が収束する列は$0$に収束する($\lim_{m\to\infty}\sum^\infty_{n=m}\mu(A_n)=0$)ことに注意して,
        \begin{align*}
            \mu(\limsup_{n\to\infty}A_n)&=\mu(\cap^\infty_{m=1}B_m)
            =\lim_{m\to\infty}\mu(B_m)&単調列の積\ref{prop-character-of-measurable-sets}(3)\\
            &\le \lim_{m\to\infty}\sum^\infty_{n=m}\mu(A_n)=0.&劣加法性\ref{prop-character-of-measurable-sets}(4)
        \end{align*}
        また,de Morganの法則より,
        \begin{align*}
            P(\liminf_{n\to\infty}A_n^\complement)&=P(\cup_{n\to\infty}\cap^\infty_{m=n}A_m^\complement)\\
            &=1-P(\cap_{n\to\infty}\cup_{m=n}^\infty A_n)=1-P(\limsup_{n\to\infty}A_n)=1.
        \end{align*}
        \item (1)と同様にして,$P(\limsup_{n\to\infty}A_n)=P(\cap^\infty_{n=1}\cup_{k\ge n}A_k)=\lim_{n\to\infty}P(\cup_{k\ge n}A_k)$である.最右辺を評価すると,
        \begin{align*}
            1-P(\cup_{k\ge n}A_k)&=P(\cap_{k\ge n}A_k^\complement)\\
            &\le P(\cap_{k\ge n}^pA_k^\complement)&\cap_{k\ge n}A_k^\complement\subset\cap^p_{k\ge n}A_k^\complement\\
            &=\prod_{k=n}^pP(A_k^\complement)&独立性\\
            &=\prod_{k=n}^p(1-P(A_k))\le\exp\paren{-\sum^p_{k=n}P(A_n)}\xrightarrow{p\to\infty}e^{-\infty}=0.&1-x\le e^{-x}
        \end{align*}
        もう一つの結論もde Morganの定理から従う.
    \end{enumerate}
\end{proof}
\begin{remarks}[どうやら洗練された証明はこの一通りである]
    (2)が極めて非自明であるが,余事象を自在に使いこなして解析関数を持ち出して不等式評価へ.複利の式だね.
\end{remarks}
\begin{example}
    無限の猿定理はこの補題(2)の特別な場合である。
\end{example}

\begin{corollary}
    事象列$(A_n)$が次の(1)と,(2)(a),(2)(b)のいずれかの2条件を満たすならば,$P\paren{\limsup_{n\to\infty}A_n}=0,\lim_{n\to\infty}P(A_n)=0$が成り立つ.
    \begin{enumerate}
        \item $\liminf_{n\to\infty}P(A_n)=0$.
        \item \begin{enumerate}[(a)]
            \item $\sum_{n\in\N}P(A^\comp_n\cap A_{n+1})<\infty$.
            \item $\sum_{n\in\N}P(A_n\cap A^\comp_{n+1})<\infty$.
        \end{enumerate}
    \end{enumerate}
\end{corollary}

\subsection{Hewitt-Savage}

\begin{definition}
    $X_1,X_2,\cdots\in\Meas(\Om,\R)$を確率変数列とする.
    事象$\{\om\in\Om\mid\forall_{i\in\N}\;X_i(\om)\in A\}\in\F$が\textbf{交換可能}であるとは,有限個の入れ替えについて
    \[\forall_{n\in\N}\;\forall_{\sigma\in S_n}\;\Brace{X_1,X_2,\cdots\in A}\subset\Brace{X_{\sigma_1},X_{\sigma_2}\cdots\in A}\]
    が成り立つことをいう.
\end{definition}

\begin{lemma}
    $E\in\F$がi.i.d.列$(X_n)$に関する交換可能な事象であるとき,$P[E]\in\{0,1\}$である.
\end{lemma}

\subsection{Bonferroni}

\begin{tcolorbox}[colframe=ForestGreen, colback=ForestGreen!10!white,breakable,colbacktitle=ForestGreen!40!white,coltitle=black,fonttitle=\bfseries\sffamily,
title=有界測度論としては初等的でも,確率論的には非自明で名前がほしい結果は多い]
    
\end{tcolorbox}

\begin{proposition}[Bonferroniの不等式]
    \[P\paren{\cap_{i=1}^nA_i}\ge\sum^n_{i=1}P(A_i)-(n-1)\]
    等号成立条件は,$A^\comp_1,\cdots,A^\comp_n$が背反のとき.
\end{proposition}
\begin{proof}
    $A^\comp_1,\cdots,A^\comp_n$について,劣加法性より.
\end{proof}

\section{確率変数論}

\begin{tcolorbox}[colframe=ForestGreen, colback=ForestGreen!10!white,breakable,colbacktitle=ForestGreen!40!white,coltitle=black,fonttitle=\bfseries\sffamily,
title=]
    有界測度空間への米田埋め込みが,確率論の世界への跳躍である.
    確率変数とは,$(\R,\B_1(\R))$への表現に他ならない.
    この観点から,合成$g\circ X_i$を$g(X_i)$などと書く上に,始域$\Om$は自由に取って考察することとなる.

    また,測度を無理やり可測関数の押し出しと見る場面は少ないかもしれないが,確率論においては,たしかに
    確率変数は確率分布を一般化する.
\end{tcolorbox}

\begin{proposition}[確率変数の構成]
    確率変数$X_1,\cdots,X_n$と可測関数$g:\R^n\to\R$に対して,$Y:=g(X_1,\cdots,X_n)$は再び可測関数である.
\end{proposition}

\section{分布関数論}

\begin{tcolorbox}[colframe=ForestGreen, colback=ForestGreen!10!white,breakable,colbacktitle=ForestGreen!40!white,coltitle=black,fonttitle=\bfseries\sffamily,
title=]
    $\R$上のRadon積分には,Lebesgue-Stieltjes積分という名のついた古典的構成法があるのであった.
    これにより,分布関数とこれが定める積分(すなわち測度)とは一対一対応する.
\end{tcolorbox}

\begin{definition}[distribution function]
    実確率変数$X$の定める
    $(\R,\B_1(\R))$上の確率測度$\mu$について,関数$F:\R\to[0,1];x\mapsto\mu((-\infty,x])$を,$X$または$\mu$の\textbf{分布関数}という.
\end{definition}
\begin{remark}
    $F(x):=\mu((-\infty,x))$と定めると,左半連続なバージョンで同様の議論が展開可能である.
\end{remark}

\begin{lemma}[分布関数の特徴付け]
    $(\R,\B_1(\R))$上の確率測度$\mu$の分布関数を$F$とする.
    \begin{enumerate}
        \item $F$は広義単調増加である.
        \item $\lim_{x\to\infty}F(x)=1,\lim_{x\to\infty}F(x)=0$をみたす有界関数である.
        \item 右半連続である.
        \item $F(x-):=\lim_{y\nearrow x}F(y)$と表すと,$\mu(\{x\})=F(x)-F(x-)$を満たす.
    \end{enumerate}
\end{lemma}
\begin{remarks}
    右半連続関数$F:\R\to[0,1]$に対して,Stieltjes積分$\int(-)dF:C_c(\R)\to\R$,またその延長としてLebesgue-Stieltjes積分$\int(-)dF:\L^1(\R)\to\R$が定まる.
    すなわち,確率分布$\mu$が一意に定まる.
    逆に,$\R$上の任意のRadon積分に対して,これに等しい分布関数とそれが定めるStieltjes積分が存在するから,確率分布と分布関数は一対一対応する.
\end{remarks}

\begin{definition}[discrete, absolute continuous, density, singular]
    分布関数$F:\R\to\R$について,
    \begin{enumerate}
        \item $F$が\textbf{不連続}であるとは,Lebesgue-Stieltjes測度$dF$がデルタ測度の可算和も許した凸結合で表せるときをいう.
        \item $F$が\textbf{絶対連続}であるとは,Lebesgue-Stieltjes測度$dF$が,Lebesgue測度$dx$に対して絶対連続であることをいう:$dF\ll dx$.このとき,$F$は密度関数$p$を定める:$F(x)=\int_{-\infty}^xp(y)dy$.
        \item $F$が\textbf{特異}であるとは,$F$は関数として連続であるが,Lebesgue-Stieltjes測度$dF$がLebesgue測度$dx$と互いに特異であるときをいう:$\exists_{A\in\B_1(\R)}\;dF(A)=1\land m(A)=0$.
    \end{enumerate}
\end{definition}
\begin{remark}
    $F$が連続であることと,これが定めるRadon積分が連続であること$\forall_{x\in X}\;\int[\{x\}]=0$は一般に同値になる.
\end{remark}

\begin{example}
    Cantor関数などは,特異型分布関数の例である.
\end{example}

\begin{theorem}[Lebesgue decomposition]
    任意の分布関数$F:\R\to\R$は,不連続分布関数$F_1$,絶対連続分布関数$F_2$,特異分布関数$F_3$が一意的に存在して,これらの線型結合で表せる.
\end{theorem}

\section{確率測度のFourier逆変換}

\begin{tcolorbox}[colframe=ForestGreen, colback=ForestGreen!10!white,breakable,colbacktitle=ForestGreen!40!white,coltitle=black,fonttitle=\bfseries\sffamily,
title=]
    特性関数の収束は確率分布の弱収束に対応する.これを用いて中心極限定理が証明される.
\end{tcolorbox}

\begin{definition}
    確率測度$\mu\in P(\R,\B(\R))$の\textbf{特性関数}$\varphi:\R\to\R$とは,
    \[\varphi(\xi)=\varphi_\mu(\xi):=\int_\R e^{i\xi x}\mu(dx)\]
    をいう.
    確率変数の特性関数は,これが定める分布の特性関数とする:$\varphi(\xi)=\varphi_X(\xi):=E[e^{i\xi X}]$.
\end{definition}

\begin{theorem}[一意性定理]
    特性関数の全体と確率測度の全体とに標準的な全単射が存在する.すなわち,任意の$\mu_1,\mu_2\in\P^d$について,次は同値.
    \begin{enumerate}
        \item $\mu_1=\mu_2$.
        \item $\varphi_{\mu_1}=\varphi_{\mu_2}$.
    \end{enumerate}
\end{theorem}
\begin{proof}
    反転公式\ref{lemma-反転公式}
    により,任意の開区間の測度は$\varphi$が一意に定める.
\end{proof}

\section{確率空間の生息圏:可測空間の圏}

\begin{tcolorbox}[colframe=ForestGreen, colback=ForestGreen!10!white,breakable,colbacktitle=ForestGreen!40!white,coltitle=black,fonttitle=\bfseries\sffamily,
title=]
    $\Meas$は$\Set$上位相的である.また,完備かつ余完備である.
    最先端はmonadとvaluationの理論だと見受けられる.\footnote{There is at least some similarity of the concept of random variables to usage of the function monad (“reader monad”) in the context of monads in computer science.\url{https://ncatlab.org/nlab/show/random+variable}}
\end{tcolorbox}

\subsection{Measの特徴}

\begin{definition}[topological category]
    代数的構造を集合演算などから抽出し,空間的概念を得る構成(数Ord,位相空間Top,可測空間Meas,多様体Diff)を形式化する.\footnote{全て$D=\Set$の例だが,$D=\Grp$として,位相群などを考えても良い.}
    束$U:C\to D$を考える.$C$の対象を空間とよび,射を射と呼ぶ.$D$の対象を代数とよび,射を準同型と呼ぶ.

    $C$が\textbf{$D$上の位相的な圏}であるとは,任意の代数$X\in D$と任意の$D$-射の族$f_i:X\to U(S_i)\in D$に対して,initial lift $(T,m_i:T\to S_i)\in C$が存在することをいう.liftとは,次の条件を満たす$C$の対象と射の組である:
    \[\forall_{T'\in C}\;\exists_{g':U(T')\to X}\;\exists_{m'_i:T'\to S_i}\;g'\circ f_i=U(m'_i)\Rightarrow\exists!_{n:T'\to T}\;U(n)=g'\land n\circ m_i=m'_i.\]
    $C$の対象は$D$の対象と$U:C\to D$が定めるinitial structure / weak structureの組として表せる.
\end{definition}

\begin{example}[Meas]
    $D=\Alg_\sigma\subset\Set$とすると,$C=\Meas$である.
\end{example}

\begin{definition}[Borel可測空間]
    関手$\B:\Top\to\Meas$に対して,$\B(S)$をBorel $\sigma$-加法族という.
\end{definition}

\subsection{Measでの極限構成}

\begin{tcolorbox}[colframe=ForestGreen, colback=ForestGreen!10!white,breakable,colbacktitle=ForestGreen!40!white,coltitle=black,fonttitle=\bfseries\sffamily,
title=]
    Measはcartesian categoryではないのは周知の事実である(直積測度の選択はエントロピー最大の公理を必要とする恣意的なものであり,Fubiniの定理で議論される).
    そこでも出来る直積構成がある.Kolmogorov productは一般のsymmetric semicartesian monoidal category\footnote{CMCとは直積によって圏のモノイド構造=テンソル積が与えられる圏であるが,一般にテンソル積の単位が終対象によって与えられる時,もうすでに十分「直積っぽい」ということでSMCという.}で定義され,無限次元のテンソル積を構成する方法である.cartesianである場合は,直積の概念と一致する.
\end{tcolorbox}

\begin{proposition}[完備かつ余完備]
    $\Meas$は完備かつ余完備である.すなわち,全ての図式は極限と余極限をもつ.
\end{proposition}

\begin{definition}[filtered category]
    filtered categoryとは,任意の有限な図式が余錐を持つような圏をいう.
    有向集合の圏化された概念である.
\end{definition}

\begin{definition}[lattice of projections, lattice of finite projections, Kolmogorov product]
    $(C,\otimes,1)$をsymmetric semicartesian monoidal categoryとする.\footnote{すなわち,$C$は終対象$1$をもち,これを単位とするモノイド構造$\otimes:C\times C\to C$を備える.}
    \begin{enumerate}
        \item $C$の対象の有限列$(X_n)_{n\in N}$について,射影のなす有限Boole代数(を細い圏とみなした圏)$B$から$C$への関手$B\to C$が存在する.これを\textbf{射影の束}という.
        \item $C$の対象の族$(X_i)_{i\in I}$について,任意の有限部分集合$F,S\subset I,\;S\subset F$に対して$\oplus_{i\in F}X_i\to\oplus_{j\in S}X_j$の形をした射影のなす束からの関手$B\to C$が存在する.これを\textbf{有限な射影の束}という.
        \item 有限な射影の束$B\to C$はcofiltered diagramである.このcofiltered limitが存在するとき,これを\textbf{コルモゴロフ積}という.
    \end{enumerate}
\end{definition}

\section{線型汎関数}

\begin{tcolorbox}[colframe=ForestGreen, colback=ForestGreen!10!white,breakable,colbacktitle=ForestGreen!40!white,coltitle=black,fonttitle=\bfseries\sffamily,
title=統計的問題では線型汎関数の推定が主眼となる所以である]
    モデル上の汎関数$\P\to\R$を母数という.
    標本空間$\X^n$上の関数$\X^n\to\R$をその推定量という.
\end{tcolorbox}

\subsection{1次元の場合}

\begin{tcolorbox}[colframe=ForestGreen, colback=ForestGreen!10!white,breakable,colbacktitle=ForestGreen!40!white,coltitle=black,fonttitle=\bfseries\sffamily,
title=]
    確率変数なる枠組みを採用すると,確率測度は$\Meas(X,\R)$上の線型作用素を定め,これを$E$で表す.
    ここから派生した作用素が重要な特徴量を定める.
\end{tcolorbox}

\begin{definition}[(population) mean / expectation, (population) variance]
    線型作用素$\Meas(X,\R)\to\R$を定める.
    \begin{enumerate}
        \item $E[X]:=\int_\Om X(\om)P(d\om)$が実数であるとき,特に離散の場合は$\sum_{\om\in\Om}X(\om)P\{\om\}$を\textbf{$X$の(母)平均}という.\footnote{標本平均と区別していう.平均は位置(location)母数の例で,分散は尺度(scale)母数の例である.}
            $E(X,A):=E(X1_A)$と表す.
        \item $\Var[X]=V[X]:=E[(X-EX)^2]$が有限であるとき,これを\textbf{分散}という.
        \item $\Cov(X,Y)=V(X,Y):=E[(X-EX)(Y-EY)]=E[XY]-E[X]E[Y]$を\textbf{共分散}$\Meas(X,\R)\times\Meas(X,\R)\to\R$という.
        \item $\sigma(X)=\sqrt{V(X)}$を\textbf{標準偏差}という.
        \item $R(X,Y):=\frac{V(X,Y)}{\sigma(X)\sigma(Y)}$を\textbf{相関係数}という.
    \end{enumerate}
\end{definition}

$E:L^1(\Om)\to\R$は
\begin{enumerate}
    \item 正な線型汎関数で,
    \item 独立確率変数の積に関して分解する$E[X_1\cdots X_n]=E[X_1]\cdots E[X_n]$.
\end{enumerate}

$\Var:L^2(\Om)\to\R_+$は
\begin{enumerate}
    \item 位置変数について不変$\forall_{a\in\R}\;\Var[X+a]=\Var[X]$で,
    \item 2次の斉次性$\forall_{a\in\R}\;\Var[aX]=a^2\Var[X]$を持ち,
    \item 独立確率変数の和に対して分解する$\Var[X_1+\cdots+X_n]=\Var[X_1]+\cdots+\Var[X_n]$.
\end{enumerate}

$\Cov:L^2(\Om)\times L^2(\Om)\to\R$は後述するが,退化した性質は次の通り\ref{prop-1d-covariance}
\begin{enumerate}
    \item 対称性:$\Cov[X,Y]=\Cov[Y,Z]$.
    \item 双線型性:$\Cov[aX+bY,Z]=a\Cov[X,Z]+b\Cov[Y,Z]$.
    \item 定数で消える:$\Cov[X,1]=0$.特に,$\Cov[aX+b,Y]=a\Cov[X,Y]$.
    \item 共分散公式:$\Cov[X,Y]=E[XY]-E[X]E[Y]$.
\end{enumerate}

\subsection{平均の一般化}

\begin{tcolorbox}[colframe=ForestGreen, colback=ForestGreen!10!white,breakable,colbacktitle=ForestGreen!40!white,coltitle=black,fonttitle=\bfseries\sffamily,
title=]
    平均とはその確率測度に関する積分である.確率変数が可分な実Hilbert空間値になったとき,これはPettis積分に対応する.
    $\R^n$の分布は任意の線型汎関数による$\R$への一次元投影によって特徴付けられる(Cramer-Wold)ことは,
    弱可測性とPettis積分の結果と一致する.
\end{tcolorbox}

\begin{proposition}[Pettis integral]
    $H$を可分Hilbert空間,$X:\Om\to H$を弱可測関数とする.このとき,$\norm{X}:\Om\to\R$が可積分ならば,ただ一つの元$E[X]\in H$が存在して,
    \[\forall_{x\in H}\;(E[X]|x)=\int_H (X(\om)|x)\mu(d\om).\]
\end{proposition}


\begin{definition}[平均]
    $H$を可分Hilbert空間とし,$\mu\in P(H)$をその上のBorel確率測度とする.
    $\mu$の平均$m\in H$とは,恒等関数$X=\id_H$のPettis積分とする:
    \[\forall_{x\in H}\quad\brac{m,x}=\int_H\brac{x,y}\mu(dy).\]
\end{definition}
\begin{remarks}
    すなわち,平均の各$x$-成分は,$x$-成分の平均に等しい.
    平均の定める線型汎関数$F(x)=\int_H(x|y)\mu(dy)=(x|m)$は,ランダムな$y\in H$と内積を取ったときの期待値は$m$と内積を取ることと等しい,という可換性を表していると思える.
\end{remarks}

\subsection{分散の一般化}

\begin{tcolorbox}[colframe=ForestGreen, colback=ForestGreen!10!white,breakable,colbacktitle=ForestGreen!40!white,coltitle=black,fonttitle=\bfseries\sffamily,
title=]
    $\R^n$の分散は$n\times n$行列になるが,要は$B(H)$の元である.
    分散公式は$\Tr Q=E[\abs{x}^2]-\abs{m}^2$という形になる.
    共分散は内積とのアナロジーで考えると極めてわかりやすい.
\end{tcolorbox}

$\Cov:L^2(\Om;\R^r)\times L^2(\Om;\R^c)\to M_{r,c}(\R)$は$\Cov[X,Y]:=E[(X-E[X])(Y-E[Y])^\top]$で定まり,まるで「中心化された,内積の逆」のようなものである\ref{prop-multidimensional-covariacne}.
\begin{enumerate}
    \item エルミート対称性:$\Cov[X,Y]=\Cov[Y,X]^\top$.
    \item 双線型性:$\Cov[aX+bY,Z]=a\Cov[X,Z]+b\Cov[Y,Z]$.
    \item 2つ合わせる:$\Cov[AX,Y]=A\Cov[X,Y]$.よって$\Cov[X,BY]=\Cov[X,Y]B^\top$.
    \item 定数で消える:$\Cov[X,I]=0$.
    \item 共分散公式:$\Cov[X,Y]=E[XY^\top]-E[X]E[Y]^\top$.
    \item 共分散公式を一般化すると,一般の確率ベクトルの2次形式の平均についての表示を得る:
    \[E[X^\top AY]=\Tr(\Cov[X,Y]^\top A)+E[X]^\top AE[Y].\]
\end{enumerate}

$\Var[X]:=\Cov[X,X]:L^2(\Om;\R^r)\to M_r(\R)$を分散共分散行列という.
半正定値で(非負の固有値をもち),直交系となる固有ベクトル系を持つ.
固有値は固有ベクトルのRayleigh商として表せる:$\lambda_i=\frac{\norm{Av_i}^2}{\norm{v_i}^2}\ge0$.

\begin{definition}
    双線型写像$G:H\times H\to\R$を表現する有界作用素$Q\in B(H)$を分散という:
    \[G(x,y)=\int_H(x|z-m)(y|z-m)\mu(dz)=(Qx|y).\]
\end{definition}
\begin{proposition}
    $E[\abs{x}^2]<\infty$とする.
    \begin{enumerate}
        \item 共分散$Q$は正作用素(半正定値)であり,かつ,対称である:$(Qx|y)=(x|Qy)$.
        \item 跡が2次の中心化モーメントに等しく,$\Tr Q<\infty$である(普段見る分散とはこれである).
        \item また2次のモーメントは$E[\abs{x}^2]=\Tr Q+\abs{m}^2$と表わせ,これを共分散公式という.
        \item $Q$はコンパクト作用素である.
    \end{enumerate}
\end{proposition}

\begin{tbox}{red}{}
    平均はPettis積分に直感的な理解を与え,分散は跡に直感的な理解を与える.
\end{tbox}

\subsection{他の測度に関する線型作用素}

\begin{definition}[entropy]
    Shanonnのエントロピーとは自己情報量$I(p):=-\log_2p\;(p\in\X)$が定める積分作用素$H:\Meas(\X,\R)\to[0,\log\al]$である.
\end{definition}

\subsection{統計的推論}

\begin{tcolorbox}[colframe=ForestGreen, colback=ForestGreen!10!white,breakable,colbacktitle=ForestGreen!40!white,coltitle=black,fonttitle=\bfseries\sffamily,
title=]
    統計的推論において,確率論の枠組みをどのように用いるかを比較対象する.
\end{tcolorbox}

\begin{definition}[frequentist probability]
    現実の事象の確率とは,仮想的に試行を繰り返したときの,相対頻度の極限として得られる極限分布として想定し,
    我々が眼前にする現象はこれの実現値であるという仮定をおいて行われる統計的推論である.

    想定した真の確率分布の特性量の計算や,真の確率分布に対する標本からの推定などが含まれる.
\end{definition}
\begin{example}[頻度主義的推論]
    偏差値は,統計的な分布が正規分布で近似できることを暗黙裡に認めて算出している.
    この仮定が数理的に妥当であることは,確率論の結果による.
\end{example}

\subsection{積分作用素に関する確率不等式}

\begin{tcolorbox}[colframe=ForestGreen, colback=ForestGreen!10!white,breakable,colbacktitle=ForestGreen!40!white,coltitle=black,fonttitle=\bfseries\sffamily,
title=]
    積分作用素に関する測度論的結果に,確率論的な解釈を与える.
\end{tcolorbox}

\begin{theorem}[Chebyshevの不等式]\label{thm-Chebyshev-inequality}\mbox{}
    $\psi:\R\to\R_{\ge0}$を非負なBorel可測な関数とする$\psi\ge0$.
    \[\forall_{A\in\B(\R)}\quad P[X\in A]\le\frac{E[\psi(X)]}{\inf_{x\in A}\psi(x)}.\]
\end{theorem}
\begin{proof}
    \[\forall_{x\in\R}\quad\paren{\inf_{x\in A}\psi(x)}1_{\Brace{x\in A}}\le\psi(x)\]
    が成り立つ.両辺の期待値を取れば良い.
\end{proof}

\begin{corollary}\mbox{}
    \begin{enumerate}
        \item 特に$\psi(x)=\abs{x}^p\;(p>0),A:=\Brace{\abs{x}\ge\ep}$とすると,$P[\abs{X}\ge\ep]\le\frac{1}{\ep^p}E[\abs{X}^p]$.
        \item $\forall_{a>0}\;P\{\abs{X(\om)-EX}>a\sigma(X)\}\le\frac{1}{a^2}$.
        \item $f:\R\to\R_{\ge 0}$を非負値単調増加関数とし,$X$を$E[\abs{X}]<\infty,E[\abs{f(X)}]<\infty$を満たす確率変数とする.$\forall_{a\in\R}\;f(a)>0\Rightarrow P(X\ge a)\le\frac{E[f(X)]}{f(a)}$.
    \end{enumerate}
\end{corollary}
\begin{proof}\mbox{}
    \begin{enumerate}
        \item 
    \begin{description}
        \item[$\sigma(X)=0$のとき] $\sigma(X)=0\Lrarrow X(\om)=EX\;\as$であるから,上の不等式は当然成り立つ.
        \item[$\sigma(X)\ne0$のとき] 求める事象を$A:=\Brace{\om\in\Om\mid\abs{X(\om)-EX}>a\sigma(X)}$とおくと,
        \[\sigma(X)^2=E(X-EX)^2\ge E((X-EX)^2,A)\ge a^2\sigma(X)^2P(A)\]
        と評価できる.
    \end{description}
        \item $f$が単調減少であることと,$f$が非負値であるから$\int_{\{X<a\}}f(X)dP\ge 0$であることより,
        \begin{align*}
            E[f(X)]&=\int_{\{X\ge a\}}f(X)dP+\int_{\{X< a\}}f(X)dP\\
            \ge f(a)P(X\ge a).
        \end{align*}
    \end{enumerate}
\end{proof}
\begin{remarks}[平均値から標準偏差の$a$倍以上離れる確率は$\frac{1}{a^2}$以下である.]
    まさかそんなに当然な評価の変形だったのか.
\end{remarks}

\begin{lemma}
    凸関数$\psi:\R\to\R$について,$X,\psi(X)\in\L^1(X)$のとき,
    \[\psi(E[X])\le E[\psi(X)].\]
\end{lemma}

\begin{lemma}
    共役指数$p,q\in[1,\infty]$について,$\norm{fg}_1\le\norm{f}_p\norm{g}_q$.
    特に$p=q=2$のとき$E[XY]\le\sqrt{E[X^2]E[Y^2]}$で,さらに$Y=1$のとき$(E[X])^2\le E[X^2]$.
\end{lemma}

\subsection{Lebesgue空間の包含関係}

\begin{theorem}
    確率空間において,$p<q\in[1,\infty]$のとき,$\L^q(X)\subset\L^p(X)$かつ$\norm{-}_p\le\norm{-}_q$である.
\end{theorem}

\section{$\sigma$-加法族が定める構造(要再考)}

\begin{tcolorbox}[colframe=ForestGreen, colback=ForestGreen!10!white,breakable,colbacktitle=ForestGreen!40!white,coltitle=black,fonttitle=\bfseries\sffamily,
title=]
実は,確率論で直積測度を考えるときは,暗黙にエントロピー最大の原理を仮定している.
確率論の枠組みではエントロピーを厳密に定義でき,これは系の状態を決定するために必要な情報量=乱雑さの小ささを計っている.
Entropy is a measure of disorder, given by the amount of information necessary to precisely specify the state of a system.\footnote{\url{https://ncatlab.org/nlab/show/entropy}}
\end{tcolorbox}

\begin{remark}[情報としての$\sigma$-加法族]
    $\sigma$-代数の細かさによって,観察の粒度を表現する.
    「情報」が追加されるたびに確率測度$P$を取り替えるのではなく,$P$は一つ
    の固定されたものと考えて「情報」の変化は$\sigma$-加法族の細分化として表現する.
    新たに追加されていくのである.
    例えば,確率変数$X$の値がわかれば$X^2$の値もわかるが,逆は成り立たない.すなわち,$\sigma$-加法族の引き戻しについて,$(X^2)^*(\B)\subsetneq X^*(\B)=(X^3)^*(\B)$.\footnote{この考え方を確率変数が定めるfiltration $\sigma[X_0,\cdots,X_n]$という.}
    したがって,外から観測できる範囲は$(\Omega,\F,P)$の$\sigma$-部分代数を自然になすので,これは情報を表すと考えると自然である.\footnote{例えば事象$B\in\F$の与える情報は$\sigma$-加法族$\{\emptyset,\Omega,B,\Omega\setminus B\}$で表されると考えられる.}
    こうして,確率変数$X$が$\F_t$可測である,というと,時刻$t$時点の情報で値が(確率1で)定まっていることになる.\footnote{サイコロを2回振る事象は,2回降った時点で値が$0,1$の2値に収束するはずである.情報$\F_t$がどこから来たかは不問とする.この間に$\sigma$-代数の構造があるため,他の確率分布$\X\to\R$からの引き戻しや,他の事象についての知識からも持って来れるはず.}
    そこで,問題は平均$E[X_n|\F_m]\;(m<n)$を求めることとなる.したがって,このモデルでは条件付き確率だらけである.
\end{remark}

\begin{definition}[関数が生成する$\sigma$-加法族]
    関数$X_1,\cdots,X_n:\Om\to\X$に対して,これら全てを可測とするような$\Om$上の最小の$\sigma$-加法族を$\sigma(X_1,\cdots,X_n)$で表す.
\end{definition}

\begin{definition}[principle of maximum entropy]\mbox{}
    \begin{enumerate}
        \item $\Om_1\times\Om_2$が考えている試行$T_1\times T_2$を試行の\textbf{直結合}という.
        \item 次の図式を可換にする確率測度$\wt{P}$は一般に複数存在するが,$\wt{P}\{(\om_1,\om_2)\}=P_1\{\om_1\}P_2\{\om_2\}$を満たすものはHahn-Kolmogorovの定理よりただ一つである.
        \item これを直積測度の公理としても良いし,エントロピー最大の原理からも従う.確率測度$P$のエントロピーとは,$\ep(P):=\sum^m_{i=1}P\{a_i\}\log\frac{1}{P\{a_i\}}\;(\Om=\{a_i\}_{i\in[m]})$と定まる.
    \end{enumerate}
\end{definition}

\begin{definition}[surprisal / self-information / information content, expected surprisal, almost partiton, entropy]
    $(X,\M,\mu)$を確率空間とする.
    \begin{enumerate}
        \item 可測集合$A\in\M$の新鮮さとは,$\sigma_\mu(A):=-\log\mu(A)=\log\frac{1}{\mu(A)}(\ge0)$をいう.ただし,$\mu(A)=0$の時は$\sigma_\mu(A)=\infty$とする.これは可測関数$\sigma_\mu:\M\to[0,\infty]$を定める.「もしその事象$A\in\M$が起こったら観測者がどれほど驚くか」をモデリングしていると考えられる.$\mu(A)=1$のとき,$\sigma_\mu(A)=0$であり,$\mu(A)=0$のとき,$\sigma_\mu(A)=\infty$である.
        \item $h_\mu(A):=\sigma_\mu(A)\mu(A)$を期待驚愕度という.このとき$\mu(A)=0\Rightarrow h_\mu(A)=0$である.これは$\mu(A)=e^{-1}$のときに最大値$e^{-1}\log e$を取り,$h_\mu(\emptyset)=h_\mu(X)=0$を満たす上に凸な関数である.
        \item 合併がfull set,任意の2つの共通部分が零集合となるような$X$の族を\textbf{概分割}という.
        \item \textbf{$\sigma$-加法族$\M$のエントロピー}とは,\[H_\mu(\M):=\sup\Brace{\sum_{A\in\F}h_\mu(A)\in\R_{\ge0}\;\middle|\;\F\subset\M,\abs{\F}<\om,X=\biguplus\F}\]
        のことをいう.$X$の可測集合による有限な直和分割の期待驚愕度の和として定められる.\footnote{こういうものは本当に極限構成になっている.$X$の可算な分割はdirectedで$h_\mu$はconcaveであるため.}
        なお,実際$\F$は$X$の概分割で十分である.
    \end{enumerate}
\end{definition}
\begin{remarks}
    $\log$は本質的ではない.$1\mapsto 0,0\mapsto\infty$を満たす関数で,$\frac{1}{x}$より急激でないものならば良かったのではないか.
    また,$\log$の底は分野によってさまざまである.Shanonnのentropyでは$2$,
\end{remarks}

\section{分割と情報}

\begin{tcolorbox}[colframe=ForestGreen, colback=ForestGreen!10!white,breakable,colbacktitle=ForestGreen!40!white,coltitle=black,fonttitle=\bfseries\sffamily,
title=]
    ぼくが数学を始めるきっかけとなった写像の標準分解:
    写像が定める同値類と写像との関係と全くパラレルである.
    $X$が定める同値関係が$Y$より細かいこと(正確には$\exists_{N\in\N(\Om)}\;\forall_{\ep_1,\ep_2\in\Om\setminus N}\;X(\om_1)=X(\om_2)\Rightarrow Y(\om_1)=Y(\om_2)$)と,$\F[Y]\subset\F[X]$は同値である.
    $X$はものの位置でもなんでも,あらゆる物理的観測を表し得る.
    観測は$\Om$の分割を定め,分割は$\sigma$-代数を生成する.
\end{tcolorbox}

\begin{notation}
    確率変数$X$に対して,これが$\Om$上に定める分割が生成する最小の閉$\sigma$-代数を$\F[X]<\D(P)$で表すこととしよう.
    すなわち,
    $(\Om,\D(P),P)$上の,$\D(P)$の部分$\sigma$-代数であって,すべての$P$-零集合を含むものを\textbf{閉$\sigma$-代数}といい,$\Phi=\Phi(\Om,P)=\Brace{\B\lor2<\D(P)\mid\B<\D(P)}$でらわす.
\end{notation}

\begin{definition}\mbox{}
    \begin{enumerate}
        \item 可測関数$X:\Om\to S$について,$\F[X]:=X^{-1}(\D(P^X))\lor 2$を,$X$で生成される閉$\sigma$-代数という.
        これは,$X$を可測にする閉$\sigma$-代数の中で最小のものである.
        \item 確率変数の族$\{X_\lambda\}_{\lambda\in\Lambda}$については,$\F[X_\lambda,\lambda\in\Lambda]:=\bigvee_{\lambda\in\Lambda}\F[X_\lambda]$と表す.
    \end{enumerate}
\end{definition}

\begin{theorem}[すべての閉$\sigma$-代数はある観測の結果とみなせる]
    任意の閉$\sigma$-加法族$\F$に対して,ある実確率変数$X\in\L(\Om)$が存在して,$\F=\F[X]$が成り立つ.
\end{theorem}

\begin{theorem}
    $X,Y\in\L(\Om)$について,$X\prec Y\;\as:\Leftrightarrow [\exists_{\varphi\in\Map(\Om,\Om)}\;X=\varphi\circ Y\;\as]$と表すと,これは同値類$\sim$とその上の順序を定め,
    \begin{enumerate}
        \item $Y\prec X\;\as\Leftrightarrow\F[Y]\subset\F[X]$.
        \item $Y\sim X\;\as\Leftrightarrow\F[Y]=\F[X]$.
    \end{enumerate}
\end{theorem}

\chapter{有界測度論}

\begin{quotation}
    確率分布を測度として定式化したが,まずは,有界な測度(を正規化したもの)としての純粋な性質を調べる.
    有界な測度は直ちに局所有限でかつ内部正則であるので,Radon測度である.
    そしてRadon測度が有限であるとき,(外部)正則なBorel測度である.
    \begin{enumerate}
        \item 測度の収束には種々の位相が考えられ,修羅の相を呈している.ここで重要になる有界測度特有の概念に緊密性があり,有界測度の列が緊密であるとき,3つの関数のクラス$C_c(X),C_0(X),C_b(X)$が定める収束概念は一致する.
        その上,緊密な確率測度列の収束先は再び確率測度となる.
        \item 確率変数に収束の概念が多様に定義できる.
        その方法は確率変数の可測性に依らないので,なるべく一般的な形で述べる.
        \item 次に,確率変数の標準的な構成法を測度論の形で与える.実際に確率測度なる対象が存在することの保証も数学的には肝要である.
    \end{enumerate}
\end{quotation}

\section{確率測度の収束}

\begin{tcolorbox}[colframe=ForestGreen, colback=ForestGreen!10!white,breakable,colbacktitle=ForestGreen!40!white,coltitle=black,fonttitle=\bfseries\sffamily,
    title=Banach空間論の発展を後押しした消息]
    確率変数は,確率測度を押し出す.この構造を用いて確率変数の収束「法則収束」を定義するから,まずは測度の収束を論じる.

    測度の弱収束の理論は,A. D. AlexandorffとProhorovによる.
    Cramer-Woldのように,線型汎関数が定める1次元周辺分布が確率測度を決定する消息がある.
    そこで,測度の弱収束は,その上の有界線型汎関数を通じて定義することは自然である.

    凸解析と関数解析が肝要となる.Alexandroff自身も,凸集合の幾何の研究が,自身の抽象測度論の研究の源泉になったと言っている.
    経験過程は極点の凸結合であると捉えられる.極点の凸結合の極限で真の分布を探そうとする営みが経験過程論であるか!?
\end{tcolorbox}

\subsection{測度と積分の双対性}

\begin{tcolorbox}[colframe=ForestGreen, colback=ForestGreen!10!white,breakable,colbacktitle=ForestGreen!40!white,coltitle=black,fonttitle=\bfseries\sffamily,
title=測度の双対空間と前双対空間]
    測度の空間の位相は,双対空間から入れるから,測度の空間の双対空間とpredualとを考察しておく必要がある.

    この消息ははじめRadon 1913\footnote{Radon, J. Theorie und Anwendungen der absolut additiven Mengenfunktionen. Sitz. Akad.
    Wiss. Wien, Math.-naturwiss. Kl. IIa. 1913. B. 122. S. 1295–1438. [245, 246]}によって研究され,その位相を「弱収束」と呼んだ.
    1907年にRieszの表現定理が独立に証明され,Hilbert空間の研究が進んでいた頃からの用語で,
    Banachが「弱収束」と「$*$-弱収束」を一般のノルム空間上に定義した1929年よりも前の用語である.
\end{tcolorbox}

\begin{notation}
    $X$をHausdorff空間,$X$のBorel集合体を$\A$,Borel $\sigma$-集合体を$\B$で表し,
    $\A$上の有限加法的な正則Borel符号付測度の空間を$M_f(X)$,$\B$上の$\sigma$-加法的な正則Borel符号付測度の空間を$M_\sigma(X)$で表す.
    これらに全変動ノルムを入れると,有限な加法的集合関数は有界変動だから,そのままノルム空間となる.
    明らかに,$M_f(X)\subset C_b(X)^*$である.
\end{notation}

\begin{definition}[Radon measure, Radon charge]\mbox{}
    \begin{enumerate}
        \item $(X,\B)$上のBorel測度$\mu:\B\to[0,\infty]$が\textbf{Radon測度}であるとは,次の2条件を満たすことをいう:
        \begin{enumerate}[(a)]
            \item locally finite:$\forall_{x\in X}\;\exists_{x\in C\subset X}\;\mu(C)<\infty$.
            \item inner regularity / tightness:$\forall_{B\osub X}\;\mu(B)=\sup_{K\compsub B}\mu(B\K)$.
        \end{enumerate}
        実は,Radon測度は測度確定な可測集合$B\in\M^1$について,外正則でもある.
        \item $(X,\B)$上の\textbf{Radon電荷}とは,複素測度$\mu:\B\to\C$であって,全変動測度$\abs{\mu}:\B\to[0,\infty)$がRadon測度であるものをいう.
        ただし,複素測度の全変動とは,$\abs{\mu}(E)=\sup\sum_{i=1}^\infty\abs{\mu(E_i)}$として定まる写像をいう.
        実は複素測度は必ず有界変動を持ち,全変動$\abs{\mu}(X)$をノルムとしてBanach空間をなす.これを$\RM(X)$で表す.
    \end{enumerate}
\end{definition}

\begin{theorem}[Radon測度とRadon積分の等価性]
    $X$が局所コンパクトでもあるとき,
    $(X,\B)$上のBorel測度$\mu:\B\to[0,\infty]$について,次の2条件は同値:
    \begin{enumerate}
        \item $\mu$はRadon測度である.
        \item $\mu$が定める写像$C_c(X;\R)\ni f\mapsto\int fd\mu$は正な線型汎関数である. 
    \end{enumerate}
\end{theorem}

\begin{proposition}[Alexandrov (1940):有限加法的集合関数の表現]\label{prop-representation-theorem-C_b}
    $X$を任意の正規空間とする.任意の有界汎関数$\Lambda\in C_b(X)^*$に対して,ただ一つの$\mu\in M_f(X)$が存在して,$\forall_{f\in C_b(X)}\;\Lambda(f)=\int_Xfd\mu$を満たし,$\norm{\Lambda}=\norm{\mu}$を満たす.
    すなわち,Banach空間として等長同型である$C_b(X)^*\simeq_\Ban M_f(X)$.
\end{proposition}

\begin{proposition}[Riesz-Markov-Kakutani representation theorem]\label{prop-representation-theorem-C_0}
    局所コンパクトハウスドルフ空間$X$について,
    \begin{enumerate}
        \item 正な有界線型汎関数のなす正錐と,Radon測度のなす正錐$\o{M(X)_+}$とは,位相同型である:$(C_c(X,\C)_+)^*\simeq_\Top\o{\RM(X)_+}$.
        \item 等長同型が存在する:$C_0(X,\C)^*\simeq_\Ban M(X)$.
    \end{enumerate}
\end{proposition}

\subsection{測度の収束の定義4種}

\begin{tbox}{red}{$w^*$-位相と呼んでも確率測度についてはwell-definedであるが,点列の収束としてはやはり別物}
    局所コンパクトハウスドルフ空間$X$上の
    \[\int fd\mu_n\to\int fd\mu\]
    が定める測度の収束概念$\mu_n\to\mu$は,確率測度に限れば,$f\in C_b(X)$としても$f\in C_c(X)$としても同じ位相を定める.
    これらはまとめて\textbf{$w^*$-位相}または\textbf{弱位相}と呼ばれるが,
    唯一の違いは$P(X)\subset M(X)$自体のコンパクト性である.
    一般に$P(X)$は$\sigma(P(X),C_c(X))$-コンパクトとは限らないため,収束先が$P(X)$に収まらない可能性だけが問題となり,これは一様緊密性の仮定をおくことで解決される(Prohorov).
    そこで収束としては「弱位相」と「漠収束」と呼び分けることとなる.

    $P(X)$に限らなければ,Alexandrovの結果より$M(X)=(C_0(X))^*\subsetneq M_f(X)=(C_b(X))^*$であるから,$M(X)$に$\sigma(M(X),C_b(X))$-位相を入れると$\sigma(M(X),C_c(X))$-位相より強くなるはずである.
\end{tbox}

\begin{tcolorbox}[colframe=ForestGreen, colback=ForestGreen!10!white,breakable,colbacktitle=ForestGreen!40!white,coltitle=black,fonttitle=\bfseries\sffamily,
title=測度の空間はBanach空間であるが,それ以上に具体的対象である.]
    一般に測度の空間はBanach空間となり,これのBanach空間としての弱収束と,「測度の弱収束」は\textbf{相違する}.
    そこで,Banach空間としての弱収束と同じ文脈で議論したい場合,有界測度論ではBourbakiの言葉を借りて「\textbf{狭収束}」と呼ばれる.
    しかし$X$が局所コンパクト空間であるとき,測度の空間に対して前節で議論したような表現定理が成り立つ.
    すると,測度という概念の性質上,双対空間の部分空間のクラス$C_c(X)\subset C_0(X)\subset C_b(X)\subset M(X)^*$に応じて,特別な位相が入るのである.
    $C_b(X)$が定める強い位相を\textbf{弱収束}といい,$C_c(X),C_0(X)$が定める弱い位相を混同して\textbf{漠収束}という.
    3つとも言うなればBanach空間の$w^*$-位相であり,この表現では区別が出来ない.
    それぞれ$\sigma(M(X),C_b(X)),\sigma(M(X),C_c(X)),\sigma(M(X),C_0(X))$-位相と言うよりほかはない.
    ここでは,$\sigma(M(X),C_c(X))$-位相を漠位相ということとする.
    %古典的な弱収束の概念は,Borel確率空間に値を取るBorel確率変数について定義された.
    %しかし,可分でない空間に値を取る場合は,Borel完全加法族は非常に大きいため,Borel可測性は条件として強すぎる.
    %たとえそのような関数の極限がBorel可測であっても,である(この重要な例が経験過程である).
    %これは非常に不自然で,関数解析的な視点からは地を這っているように無明である.
    %Skorokhodは可分となるような距離を見つけたが,この方向の対処よりも抜本的な解決がほしい.
    %Dudleyが,一様ノルムについてのBanach空間と見たまま,弱収束の定義を拡張する方向を開拓し,Hoffmann-Jorgensenが外積分の基盤を作った.
\end{tcolorbox}

\begin{definition}[strong, weak, narrow, wide / vague / weak star]
    可測空間$(X,\B)$上の有界変動を持つ符号付きRadon測度の空間$M(X)
    $\footnote{符号付き測度は,有限ならば有界変動を持つ.}は,全変動$\abs{\mu}$が定めるノルムについてBanach空間をなす.
    ノルム位相と異なる位相で代表的なものは以下の通りである.\footnote{\url{https://encyclopediaofmath.org/wiki/Convergence_of_measures}}
    \begin{enumerate}
        \item まず各可測集合上での収束(setwise convergence)が考えられる.
        \item ノルム収束は,変動収束(convergence in variation)または強収束とも言う.
        \item 任意の有界線型汎関数$F\in M_b(X)^*$について$F(\mu_n)\to F(\mu)$を満たすとき,\textbf{弱収束}という.
        \item $X$が位相空間で,$\B$がBorel $\sigma$-代数であるとき,任意の有界連続関数$f\in C_b(X)$について$\mu_nf\to\mu f$を満たすとき,\textbf{狭収束}(narrow topology),また確率測度に限っては\textbf{弱収束}という.\footnote{Bourbakiによる造語で,いまでは稀.元々は"convergence étroite".}
        $X$が局所コンパクトハウスドルフ空間であるとき,$\mu_n$が$\mu$に狭収束するならば,$\forall_{A\in\B(X)}\;\abs{\mu}(\partial A)=0$が必要であり,これが狭収束を特徴付ける\ref{thm-Portmanteau}
        \item $X$が局所コンパクト空間で,$\B$がBorel $\sigma$-代数であるとき,更に別の位相が定まる.
        任意のコンパクト台を持つ連続関数$f\in C_c(X)$について$\mu_nf\to\mu f$を満たすとき,\textbf{広収束}(wide topology)または\textbf{$w^*$-収束}という.
        一般に狭位相より弱いが,$X$がHausdorffのとき,確率測度の空間$P(X)$には同じ位相を定める.
        なお,$P(X)$は広位相についてコンパクトとは限らない,すなわち収束先は$\mu(X)=1$を満たさないことがある.\footnote{例は極点の列$(\delta_n)$が与える.$X$がコンパクトであるとき,$\mu(X)=1$を必ず満たす.}
    \end{enumerate}
\end{definition}
%\begin{remark}
    %局所コンパクトハウスドルフ空間$X$について,$C_c(X)$の一様ノルムについての閉包が$C_0(X)$であるから,このうちどの2つを試験関数のクラスとして採用しても,同じ位相を定める.\footnote{\url{https://math.stackexchange.com/questions/313986/are-vague-convergence-and-weak-convergence-of-measures-both-weak-convergence}}
%\end{remark}

\begin{example}[退化する例:コンパクトハウスドルフ空間]
    $X$がコンパクトハウスドルフであるとき,Rieszの表現定理は同型$M_b(X)=(C(X))^*$を引き起こす.
    このとき,$C(X)=C_c(X)=C_0(X)=C_b(X)$
    であるから,狭位相,広位相はいずれも(関数解析の意味で)$w^*$-位相と一致する.
    \footnote{なお,一般に$C(X)$は回帰的でなく,双対空間$(M_b(X))^*$は$C(X)$より広いから,$w^*$-位相は弱位相より弱いことは関数解析の結果である.}
\end{example}

\begin{example}[局所コンパクトハウスドルフ空間での例]
    $X$が局所コンパクトハウスドルフ空間であるとき,Riesz-Markov-Kakutaniの定理より$M_b(X)=(C_0(X))^*$を得て\ref{prop-representation-theorem-C_0},同時に$M(X)=(C_c(X))^*_+$を満たす\ref{prop-representation-theorem-C_c}.
    なお,$M_b(X)$は有限Radon測度の空間で,$M(X)$はRadon測度全体の空間とした.
    よって,いずれも(特定のpredualに対する)$w^*$-位相を備えるが,位相としては別物である.
    $\delta_n$は$0$に$C_c$-収束するが,$C_0$では収束しない:$f(x)=\sin(x)/x$を取ると収束しない.

    一方で,$C_b(X)$の双対空間の中には,測度として表せないものも存在する\ref{prop-representation-theorem-C_b}.
    これは$M(X)\subsetneq M_f(X)=(C_b(X))^*$\ref{prop-representation-theorem-C_b}という意味では,$w^*$-位相の相対位相である.
\end{example}

\begin{example}[実数上の測度の収束]
    $X=\R$のとき,$1_{[a,b)}$という形の特整関数が生成する部分空間は,$C_0(\R)$上稠密であるから,漠位相は「ある稠密部分集合$D\subset\R$が存在して,$\forall_{a<b\in D}\;\mu_n((a,b])\to\mu((a,b])$を満たすことと同値になる.
\end{example}

\begin{remarks}[緊密性とは,この消息によって要求される確率論的概念である]\label{remarks-tightness}
    こうして,$(\delta_n)$の例などを省くためには,確率測度の列がある種のコンパクト性を満たす必要がある.
    緊密性を$\forall_{\ep>0}\;\exists_{K_\ep\cpt X}\;\forall_{n\in\N}\;\mu_n(K_\ep^\comp)<\ep$と定めると,意味は
    the sequence of measures are all almost supported in a compact set, so there is no possibility of mass "escaping at infinity" as it was the case with $(\delta_n)$.\footnote{\url{https://math.stackexchange.com/questions/313986/are-vague-convergence-and-weak-convergence-of-measures-both-weak-convergence}}
    特に,Radon測度$(\mu_n)$が緊密ならば$\mu_n$は有限である.

    こうして,標準的な設定が出来上がる.
    有界測度の列が緊密であるとき,3つの関数のクラス$C_c(X),C_0(X),C_b(X)$が定める収束概念は一致する.
\end{remarks}

\subsection{確率測度のノルム収束}

\begin{tcolorbox}[colframe=ForestGreen, colback=ForestGreen!10!white,breakable,colbacktitle=ForestGreen!40!white,coltitle=black,fonttitle=\bfseries\sffamily,
    title=]
    Radon電荷のBanach空間$M(\Om)$上でのノルム収束を,全変動収束という.
    その双対空間は$C_0(\Om,\C)$であるが(Riesz-Markov),この元に対する収束が弱収束である.
    確率測度の空間$P(X)$はもちろんノルム閉ではないが,凸である.
\end{tcolorbox}

\begin{lemma}[確率測度の全変動距離の特徴付け]
    $\mu,\nu\in P(\Om,\F,P)$について,
    \[\norm{\mu-\nu}=2\sup_{A\in\F}\abs{\mu(A)-\nu(A)}.\]
\end{lemma}

\begin{corollary}[Scheffe]
    $\{X_n\},X\ll\mu$とする.$p_n\to p\;\mu\dae$が成り立つならば,$X_n$は$X$に全変動ノルムについて収束する:$\norm{X_n-X}\to0$.
\end{corollary}

\subsection{弱収束の定義と特徴付け}

\begin{definition}[weak convergence of measure, vague convergence]
    一般の局所コンパクトハウスドルフ空間$X$について,Rieszの表現定理より$P(X)\subset M(X)\simeq C_0(X)^*$と同一視出来る.
    このBorel可測空間$(X,\B(X))$上の確率測度の空間$P(X)\subset C_0(X)^*$の列$(\mu_n)$について,
    \begin{enumerate}
        \item $\forall_{f\in C_b(X)}\;\int_Xfd\mu_n\to\int_Xfd\mu$,平均で表すと$\forall_{f\in C_b(X)}\;E[f(X_n)]\to E[f(X)]$を,\textbf{測度の弱収束}と呼ぶ.これを$\mu_n\Rightarrow\mu$と表す.
        \item さらに条件を弱めて$\forall_{f\in C_0(X)}\;\int_Xfd\mu_n\to\int_Xfd\mu$は$w^*$-位相に対応し,\textbf{漠収束}と呼ばれる.\footnote{これを弱位相と呼ぶこともあるが,$C_0(X)$はほとんど回帰的である例がないので,不適切である.}これは上述の狭位相と広位相が退化して一つになったものである.
    \end{enumerate}
\end{definition}
\begin{remarks}
    $C_b(X)$を試験関数とした場合は,確率変数が無限遠へ飛ぶ場合のすべてを検知するが,$C_c(X)\subset C_0(X)$を用いた場合は見逃す場合がある.
\end{remarks}

\begin{theorem}[Portmanteau定理:弱収束の特徴付け (Alexandroff)\footnote{I don' tknow who invented such a nonsensical name for Alexandroff’s theorem.\cite{Bogachev}}]\label{thm-Portmanteau}
    $\{\mu_n\}\subset P(\R),\mu\in P(\R)$について,次の5条件は同値.これは一般の距離空間$X$について成り立つ.
    \begin{enumerate}
        \item $\mu_n\Rightarrow\mu$.
        \item 任意の開集合$G\osub\R$について,$\liminf_{n\to\infty}\mu_n(G)\ge\mu(G)$.
        \item 任意の閉集合$C\subset\R$について,$\limsup_{n\to\infty}\mu_n(C)\le\mu(C)$.
        \item $A\in\B(\R)$について,$\mu(\partial A)=0\Rightarrow\lim_{n\to\infty}\mu_n(A)=\mu(A)$.この条件を満たす集合$A$を\textbf{$P$-連続集合}という.\footnote{このような集合は体/代数をなす.}
        \item $\mu_n,\mu$が定める分布関数$F_n,F$について,$\forall_{x\in\R}\;\lim_{y\nearrow x}F(y)=x\Rightarrow\lim_{n\to\infty}F_n(x)=F(x)$.
    \end{enumerate}
    また,次とも同値.
    \begin{enumerate}\setcounter{enumi}{5}
        \item 任意の一様連続な有界関数$f$について,$P_nf\to Pf$.なお,有界性の仮定を外しても同値である.
        \item 任意の非負有界連続関数$f$に関して,$\liminf_{n\to\infty}P_nf\ge Pf$.
    \end{enumerate}
\end{theorem}

\begin{theorem}[その他の集合族による特徴付け]
    距離空間上の測度空間$(X,\B(X))$について,$\A\subset\B(X)$を乗法族とする:$X\in\A,\forall_{A,B\in\A}\;A\cap B\in\A$.
    任意の開集合が$\A$の元の可算和で表せるとき,$\forall_{A\in\A}\;P_n(A)\to P(A)$は$P_n\Rightarrow P$を含意する.
\end{theorem}

\begin{theorem}[3つ目の特徴付け]
    列$(P_n)$が$P$に弱収束するための必要十分条件は,これが相対コンパクトであること,すなわち任意の部分列$(P_{n_i})$が,$P$に弱収束する部分列を持つことである.
\end{theorem}

\subsection{漠収束の特徴付け}

\begin{proposition}[漠収束の特徴付け]\label{prop-characterization-of-value-convergence}
    $\{\mu_n\}\subset P(\R),\mu\in M(\R)$について,次の2条件は同値.
    \begin{enumerate}
        \item $(\mu_n)$は$\mu$に漠収束し,$\mu\in P(\R)$を満たす.
        \item $(\mu_n)$は$\mu$に弱収束する.
    \end{enumerate}
\end{proposition}

\subsection{一様収束のための十分条件}

\begin{tcolorbox}[colframe=ForestGreen, colback=ForestGreen!10!white,breakable,colbacktitle=ForestGreen!40!white,coltitle=black,fonttitle=\bfseries\sffamily,
title=]
    必要になったらで.
\end{tcolorbox}

\begin{theorem}[Ranga Rao (62)]
    $X$を可分な距離空間,$(\mu_n)$をその上の確率測度の列とする.
    このとき,次の2条件は同値.
    \begin{enumerate}
        \item $\mu_n\Rightarrow\mu$.
        \item 任意の関数族$\F\subset C_b(X)$であって,一様に有界で,同程度連続であるものについて,$\lim_{n\to\infty}\sup_{f\in\F}\abs{\mu_nf-\mu f}=0$が成り立つ.
    \end{enumerate}
\end{theorem}

\subsection{連続写像定理}

\begin{notation}
    距離空間$(S_i,d_i)\;(i=1,2)$と,確率測度$\nu,\nu_n\in P(S_1)$と可測写像$T:S_1\to S_2$について,
\end{notation}

\begin{proposition}
    $\nu_n\to\nu$かつ$T$が連続ならば,$\nu_n^T\to\nu^T$.
\end{proposition}

\begin{lemma}
    可測写像の列$\{T_n\}\subset\Meas(S_1,S_2)$について,次の条件は同値.
    \begin{description}
        \item[(C)] $x$に収束する$S_1$の任意の点列$(x_n)$について,$T_n(x_n)\to T(x)$が成り立つ.
        \item[(C')] $\forall_{\ep>0}\;\exists_{(n_0,\delta)\in\N\times\R_+}\;\forall_{(n,x')\in\N\times S_1}\;n\ge n_0\land d_1(x,x')<\delta\Rightarrow d_2(T(x),T_n(x'))<\ep$.
    \end{description}
\end{lemma}

\begin{theorem}
    $\nu_n\to\nu$とする.あるfull set$C\in\B(S_1),\nu(C)=1$が存在して,この上で$x\in C$は条件(C)を満たすとする.このとき,$\nu_n^{T_n}\to\nu^T$.
\end{theorem}

\begin{corollary}
    可測写像$T:S_1\to S_2$は,$\nu(C)=1$なる可測集合$C\in\B(S_1)$上各点連続であるとする.このとき,$\nu_n\to\nu$ならば$\nu^T\to\nu^T$.
\end{corollary}

\section{確率測度の空間}

\begin{tcolorbox}[colframe=ForestGreen, colback=ForestGreen!10!white,breakable,colbacktitle=ForestGreen!40!white,coltitle=black,fonttitle=\bfseries\sffamily,
title=]
    $P(X)$の幾何学的性質が基礎的になってくるので,ここにまとめることとする.
\end{tcolorbox}

\begin{notation}
    $X$を距離空間とし,その上の
    Borel確率測度のなす空間を$P(X)$と表す.
    距離空間$X$上のDirac測度全体の集合を$\Delta\subset P(X)\subset C_b(X)^*$で表す.
\end{notation}

\subsection{可分性と距離付け可能性}

\begin{lemma}
    $X$は,弱位相における$\Delta$と位相同型である.
\end{lemma}

\begin{lemma}
    $\Delta$は$P(X)$内で点列コンパクトである.\footnote{一般に部分集合$A\subset X$の閉包$\o{A}$は点列閉包$B$を含むが,点列閉包と一致するのは一般に$X$が距離空間の場合のみ.}
\end{lemma}

\begin{lemma}
    $X$を全有界な距離空間とする.$X$上の有界な一様連続関数の集合$U(X)\subset C_b(X)$は,一様ノルムの下で可分なBanach空間となる.
\end{lemma}

\begin{theorem}[確率測度の空間の弱位相の距離化可能性]\label{thm-metrizability-of-P(X)-in-weak-topology}
    次の2条件は同値.
    \begin{enumerate}
        \item $X$は可分な距離空間である.
        \item 弱位相を備えた$P(X)$は距離付け可能で可分である.
    \end{enumerate}
\end{theorem}

\begin{example}[Levy-Prokhorov距離]
    $B_\ep(E):=\{x\in X\mid\rho(x,E)<\ep\}$と表す.
    \[\eta(\mu,\nu):=\inf\Brace{\ep>0\mid\forall_{E\in\B(X)}\;\mu(E)\le\nu(B_\ep(E))+\ep\land\nu(E)\le\mu(B_\ep(E))+\ep}\]
    は距離を定め,これが定める位相は$*$-弱位相に一致する.
    対応する分布関数の空間の距離をLevyの距離といい,これが先に提出され,Prokhorovがこの形に一般化した.
\end{example}

\begin{example}[1-Wasserstein distance / Kantorovich-Rubinstein distance / Monge-Kantorovich distance]
    さらに$X$を完備とする(すべて併せてポーランド空間とする).
    このとき,次は$P(X)$の距離を与える:
    \[W_1(\mu,\nu)=\sup\Brace{\int\varphi d\mu-\int\varphi d\nu\in\R\;\middle|\;\varphi\in\Lip(X;\R),L(\varphi)\le 1}.\]
\end{example}

\subsection{稠密部分集合の遺伝}

\begin{theorem}
    $X$を可分距離空間,$E\subset X$は稠密とする.
    $E$に含まれる可測集合上のみに台をもつ確率測度のなす部分空間
    \[\Brace{\mu\in P(X)\mid\supp\mu\subset P(E)}\]
    は$P(X)$内で稠密である.
\end{theorem}

\subsection{コンパクト性}

\begin{theorem}[可算稠密部分集合の構成]
    $X$を可分距離空間,$D$をその可算な稠密部分集合とする.
    このとき,$D$の有限部分集合を台とするような確率測度のなす集合$F(D)$は,$P(X)$において稠密である.
\end{theorem}

\begin{corollary}
    $D$が可分である時,Dirac測度としての標準的な埋め込み$D\mono\M(D)$の像の凸包は稠密である.
\end{corollary}

\begin{theorem}[コンパクト性 (Bogoliubov and Krylov)]
    距離空間$X$について,次の2条件は同値.
    \begin{enumerate}
        \item $X$はコンパクトである.
        \item $P(X)$は弱コンパクトで距離化可能.
    \end{enumerate}
\end{theorem}

\subsection{完備性}

\begin{theorem}[完備性]
    可分な距離空間$X$について,次の2条件は同値.
    \begin{enumerate}
        \item $X$は完備である.
        \item $P(X)$は完備である.
    \end{enumerate}
\end{theorem}

\begin{theorem}
    $D$がポーランドである時,$\M(D)$もポーランドである.
\end{theorem}

\subsection{相対コンパクト集合の特徴付け}

\begin{tcolorbox}[colframe=ForestGreen, colback=ForestGreen!10!white,breakable,colbacktitle=ForestGreen!40!white,coltitle=black,fonttitle=\bfseries\sffamily,
title=]
    ここで緊密性の概念が出現する.\ref{remarks-tightness}も参照.例えばmassが無限遠点に逃げていく列などが省かれる.
\end{tcolorbox}

\begin{definition}[totally bounded / weakly compact]
    確率測度の族$\Gamma\subset P(X)$が\textbf{相対コンパクト}であるとは,任意の列について,ある部分列が存在してこれが弱収束することをいう.
\end{definition}
\begin{remarks}
    可分距離空間$X$上の確率測度の空間$P(X)$の弱位相は距離化可能で可分である\ref{thm-metrizability-of-P(X)-in-weak-topology}.
    距離空間について,可分性と全有界性とは同値である.
    また,距離空間について,コンパクト性と点列コンパクト性と「完備かつ全有界」であることとは同値だから,
    $S$が完備可分距離空間である場合,相対コンパクト性を全有界ともいう(完備距離空間において,コンパクト集合と閉集合とは同値).
\end{remarks}

\begin{definition}[(uniformly) tight / tendue]
    確率測度の族$\Gamma\subset P(X)$が\textbf{(一様に)緊密}\footnote{測度が内部正則であることも「緊密」というので,それから見れば「一様に緊密」と言いたくなる.}であるとは,
    $\forall_{\ep>0}\;\exists_{K\overset{\text{cpt}}{\subset} X}\;\forall_{\mu\in\Gamma}\;\mu(K)\ge 1-\ep$.
\end{definition}

\begin{theorem}[Prokhorov:緊密性とは,相対コンパクト性の特徴付けである]
    $X$を距離空間とする.確率測度の族$\Gamma\subset P(X)$について,(1)$\Rightarrow$(2)が成り立つ.
    $X$が完備かつ可分であるとき,(2)$\Rightarrow$(1)も成り立つ.
    \begin{enumerate}
        \item  $\Gamma$は(一様に)緊密である.
        \item $\Gamma$は弱(狭)位相について相対コンパクトである.
    \end{enumerate}
\end{theorem}

\begin{corollary}[漠収束の特徴付け]
    完備可分距離空間$X$上の確率測度列$(\mu_n)$について,次の2条件は同値.\footnote{\url{https://math.stackexchange.com/questions/313986/are-vague-convergence-and-weak-convergence-of-measures-both-weak-convergence}}
    \begin{enumerate}
        \item $\mu$に弱収束する.
        \item $\mu$に漠収束し,かつ$(\mu_n)$は一様に緊密である.
    \end{enumerate}
\end{corollary}
\begin{proof}
    (1)$\Rightarrow$(2)は明らかだから,(2)$\Rightarrow$(1)を示す.
    漠収束の特徴付け\ref{prop-characterization-of-value-convergence}より,
    極限測度$\mu$が$P(X)$に属することを示せば十分である.
    略.
\end{proof}

\subsection{コンパクト空間上の確率測度}

\begin{tcolorbox}[colframe=ForestGreen, colback=ForestGreen!10!white,breakable,colbacktitle=ForestGreen!40!white,coltitle=black,fonttitle=\bfseries\sffamily,
title=]
    経験過程は重要な意味を持つ.これを,極点の凸結合として幾何学的に説明できないか?
    経験過程の有限性は実用性に通じるが,これは組み合わせ論的な本質も備えているのではないか?

    $X$がコンパクトのとき,その上の弱位相と$w^*$-位相は一致する.
\end{tcolorbox}

\begin{lemma}[ポーランド空間上の確率測度はRadon測度]
    $S$を可分な距離空間とする.$(S,\B(S))$上のBorel確率測度$P$は,
    \begin{enumerate}
        \item 正則である:
        \[P(B)=\sup_{F\subset B;F:closed}P(F)=\inf_{B\subset G\osub S}P(G).\]
        \item $S$が完備ならば,さらに外部正則である:
        \[\forall_{\ep>0}\;\exists_{K\compsub S}\; P(K)>1-\ep.\]
    \end{enumerate}
\end{lemma}

\begin{proposition}
    コンパクトハウスドルフ空間$X$上のBanach代数$C(X)=C_b(X)=C_0(X)=C_c(X)$を考える,但しノルムは一様ノルムとした.
    $C(X)$の双対空間を$M(X)$,$P(X):=\Brace{\mu\in M(X)\mid\norm{\mu}\le 1,\mu(1)=1}$を確率測度のなす部分空間とする.
    \begin{enumerate}
        \item $P(X)$は$M(X)$の凸集合である.
        \item $P(X)$は$w^*$-コンパクトである.
        \item $P(X)$の極点はDirac測度$\delta_x\;(x\in X),\forall_{f\in C(X)}\;\delta_x(f)=f(x)$である.
    \end{enumerate}
\end{proposition}
\begin{proof}\mbox{}
    \begin{enumerate}
        \item $\mu_1,\mu_2\in P(X)$と$\lambda\in(0,1)$を任意に取ると,$\lambda\mu_1+(1-\lambda)\mu_2\in P(X)$がわかる.
        \item 線型汎函数$\ev_1:M(X)\to\bF;\mu\mapsto\mu(1)$は$w^*$-位相について連続である
        $M(X)=(C(X))^*$の閉単位球$B^*$は$w^*$-コンパクト\ref{thm-Alaoglu}である.
    \end{enumerate}
\end{proof}
\begin{remarks}
    コンパクトハウスドルフ空間$X$上の確率測度は,$C(X)$上において,有限な台を持つ測度(=Radon chargeが定める積分の空間)によって各点近似(各点収束の位相で近似)が出来る.
\end{remarks}


\section{積分の拡張}

\begin{tcolorbox}[colframe=ForestGreen, colback=ForestGreen!10!white,breakable,colbacktitle=ForestGreen!40!white,coltitle=black,fonttitle=\bfseries\sffamily,
title=]
    まずは,関数解析の力を借りて,積分を非可測な関数に対しても延長する,Hoffmann-Jorgensenの理論を展開する.
    この議論は一般の局所コンパクトハウスドルフ空間$X$上のRadon積分$\int:C_c(X)\to\R$に関して展開できる.
\end{tcolorbox}

\begin{definition}[outer integral]
    $(\Om,\A,P)$を確率空間,$T:\Om\to\o{\R}$を写像とする.
    \begin{enumerate}
        \item $E^*[T]:=\inf\Brace{E[U]\in\R\mid U\ge T,U\in\Meas(\Om,\o{\R}),E[U]\in\R}$を,押し出された測度$T_*P$に関する$\o{\R}$上のRadon上積分とする.
        \item 写像列$(X_n:\Om_n\to\Om)$がBorel可測関数$X:\Om_\infty\to\Om$に弱収束するとは,$\forall_{f\in C_b(\Om)}\;E^*[f(X_n)]\to E[f(X)]$を満たすことをいう.
    \end{enumerate}
\end{definition}

\section{確率変数列の収束}

\begin{tcolorbox}[colframe=ForestGreen, colback=ForestGreen!10!white,breakable,colbacktitle=ForestGreen!40!white,coltitle=black,fonttitle=\bfseries\sffamily,
title=]
    次に,測度の収束の一般化として,確率変数列に対して収束を定義する.
    測度の弱収束は,作用素の$w^*$-収束と同値で,確率変数列の法則収束と同値.
\end{tcolorbox}

\subsection{定義}

\begin{definition}[almost sure convergence, converge in probability, converge in the mean of order $p$, converge in law / distribution]
    $(X_n)$を確率空間$(\Om,\F,P)$上の実確率変数の列とする.
    \begin{enumerate}
        \item $(X_n)$が$X$に\textbf{概収束}する$X_n\to X\;\as$とは,$P\paren{\lim_{n\to\infty}X_n=X}=1$が成り立つことをいう.
        \item $(X_n)$が$X$に\textbf{確率収束}する$X_n\xrightarrow{p}X$とは,$\forall_{\ep>0}\;\lim_{n\to\infty}P(\abs{X_n-X}>\ep)=0$が成り立つことをいう.
        \item $(X_n)$が$X$に\textbf{$p$次平均収束}するとは,$\lim_{n\to\infty}E[\abs{X_n-X}^p]=0\;(p\ge 1)$を満たすことをいう.
        \item $(X_n)$が$X$に\textbf{法則収束}または\textbf{分布収束}する$X_n\xrightarrow{d}X$とは,$\forall_{f\in C_b(\R)}\;\lim_{n\to\infty}E[f(X_n)]=E[f(X)]$を満たすことをいう.
    \end{enumerate}
\end{definition}
\begin{remark}[法則収束の特異性]
    概収束,確率収束,$p$次平均収束は$X_n$と$X$の間に関係$=$や演算$-$が定義されていることが必要であるから,列$(X_n)$は同一の確率空間上で定義されている必要があるが,法則収束は実はその必要はなく,純粋に実数上に引き起こす測度のみによって決まる概念であるから,始域は抽象化されている.
\end{remark}

\begin{remark}
    $(\X_n,\A_n,P_n)$上の$S$-値確率変数列$(X_n)$と
    ある定点$c\in S$に対して,\[X_n\xrightarrow{p}c:\Leftrightarrow\forall_{\ep>0}\;\lim_{n\to\infty}P_n[d(X_n,c)>\ep]=0\]
    も確率収束と呼ばれるが,この場合は法則収束$P_n^{X_n}\Rightarrow\delta_c$と同値になる.
\end{remark}

\subsection{収束の間の関係}

\begin{tcolorbox}[colframe=ForestGreen, colback=ForestGreen!10!white,breakable,colbacktitle=ForestGreen!40!white,coltitle=black,fonttitle=\bfseries\sffamily,
title=]
    \[\xymatrix@R-2pc{
        \text{概収束}\ar[dr]\\
        &\text{確率収束}\ar[r]&\text{法則収束}\\
        p\text{次確率収束}\ar[ur]
    }\]
\end{tcolorbox}

\begin{lemma}[相互関係1]
    $(X_n)$が$X$に確率収束するならば,法則収束する.
\end{lemma}

\begin{theorem}[相互関係2]
    $(X_n),X$について,
    \begin{enumerate}
        \item 概収束するならば,確率収束する.
        \item $p$次平均収束するならば,確率収束する.
    \end{enumerate}
\end{theorem}

\begin{proposition}[確率収束の特徴付け]
    可分距離空間$S$-値確率変数$X_n,X$について,次の3条件は同値.
    \begin{enumerate}
        \item $(X_n)$は$X$に確率収束する.
        \item $E[d(X_n,X)\land 1]\to 0\;(n\to\infty)$.
        \item $\{X_n\}$は概収束に関して相対コンパクト.
    \end{enumerate}
\end{proposition}

\begin{corollary}[弱めた逆が成り立つ]
    $X_n$は$X$に確率収束,または$p$次平均収束するとする.
    このときある部分列が存在して,概収束する.
\end{corollary}

\subsection{概収束の性質}

\begin{tcolorbox}[colframe=ForestGreen, colback=ForestGreen!10!white,breakable,colbacktitle=ForestGreen!40!white,coltitle=black,fonttitle=\bfseries\sffamily,
title=]
    概収束を定める位相は距離付け不可能で,確率収束を定める位相は距離付け可能である.
    $p$次平均収束とは,$L^p$-ノルムが定める位相についての収束と同値.

    法則収束は,分布の空間$P(\Om)$の上で距離付け可能である.
\end{tcolorbox}

\begin{proposition}
    概収束は距離付け不可能である.
\end{proposition}

\subsection{確率収束の性質}

\begin{proposition}[確率収束に関する連続写像定理]
    $(S_i,d_i)\;(i=1,2)$を可分距離空間とする.
    $X,X_n$を$S_1$-値確率変数,$h:S_1\to S_2$を可測写像とする.
    $P[X\in C]=1$を満たす可測集合$C\in\B(S_1)$上で$h$は連続であるとする.
    このとき,$X_n\xrightarrow{p}X\Rightarrow h(X_n)\xrightarrow{p}h(X)$.
\end{proposition}

\begin{proposition}[確率収束の遺伝]
    $X_n,X,Y_n,Y$を確率変数とする.
    \begin{enumerate}
        \item 定数$c\in\R$については,$X_n\xrightarrow{p}c$と$X_n\xrightarrow{d}c$とは同値.
        \item $X_n\xrightarrow{d}X$かつ$d(X_n,Y_n)\xrightarrow{p}0$ならば,$Y_n\xrightarrow X$.
        \item $X_n\xrightarrow{d}X$かつ$Y_n\xrightarrow{p}c$ならば,$(X_n,Y_n)\xrightarrow{d}(X,c)$.
        \item $X_n\xrightarrow{p}X$かつ$Y_n\xrightarrow{p}Y$ならば,$(X_n,Y_n)\xrightarrow{p}(X,Y)$.
    \end{enumerate}
\end{proposition}
\begin{remark}
    一方で,結合分布の弱収束は,各成分の弱収束より強い\ref{thm-Slutsky}.
    これも,各成分の分布だけでは結合分布が定まらないことに起因する.
\end{remark}

\begin{proposition}[確率収束を定める距離1]
    $X,Y\in\Meas(\Om,\R)$について,$d(X,Y):=E\Square{\frac{\abs{X-Y}}{1+\abs{X-Y}}}$と定める.
    \begin{enumerate}
        \item $d$は距離を定める.
        \item $d$について収束することと,確率収束することは同値.ただし,$X=Y\;\as$を同一視する.
    \end{enumerate}
\end{proposition}

\begin{proposition}[確率収束を定める距離2]
        可測関数の全体$\L(\Om)$について.
        \begin{enumerate}
            \item 距離$\rho_0(X,Y):=E[\abs{X-Y}\land 1]$について,完備可分な距離空間となる.
            \item \[\forall_{\ep\in(0,1)}\;\ep P[\abs{X-Y}>\ep]\le\rho_0(X,Y)\le\ep+P[\abs{X-Y}>\ep].\]
            \item この距離が定める位相は,確率収束を定める.
        \end{enumerate}
\end{proposition}

\subsection{連続写像定理とデルタ法}

\begin{tcolorbox}[colframe=ForestGreen, colback=ForestGreen!10!white,breakable,colbacktitle=ForestGreen!40!white,coltitle=black,fonttitle=\bfseries\sffamily,
title=]
    分布収束は,有界測度論の世界が流入した風景が広がっている.
    分布収束は連続写像で保たれ,平均の周りでの分散についてはについては,デルタ法が成り立つ.
\end{tcolorbox}

\subsubsection{連続写像定理}

\begin{proposition}[連続写像は弱収束を保つ]
    $\forall_{T\in C(S_1;S_2)}\;\nu_n\to\nu\Rightarrow \nu_n^T\to\nu^T$.
\end{proposition}

\begin{theorem}[連続写像定理]
    $T,T_n:S_1\to S_2$を可測写像とし,$P[X\in C]=1$を満たすある可測集合$C\in\B(S_1)$が存在して,条件
    \begin{quote}
        (C) 任意の$x\in C$に収束する任意の列$\{x_n\}\subset S_1$に対して,$T_n(x_n)\xrightarrow{n\to\infty}T(x)$.
    \end{quote}
    が満たされるとする.
    このとき,$X_n\xrightarrow{d}X$ならば,$T_n(X_n)\xrightarrow{d}T(X)$.
    特に,$\forall_{n\in\N}\;T_n=T$の場合を考えれば,$T$が$C$上連続ならば,$T(X_n)\xrightarrow{d}T(X)$が成り立つ.
\end{theorem}

\subsubsection{デルタ法}

\begin{tcolorbox}[colframe=ForestGreen, colback=ForestGreen!10!white,breakable,colbacktitle=ForestGreen!40!white,coltitle=black,fonttitle=\bfseries\sffamily,
title=]
    パラメータ$\theta\in\R^k$の推定量$T_n:\X^n\to\R^k$と,関数$\varphi:\R^k\to\R$を考える.
    $T_n\xrightarrow{p}\theta$が成り立つとき,$\varphi$が$\theta$で連続ならば$\varphi(T_n)\xrightarrow{p}\varphi(\theta)$も成り立つ.
    では,$\sqrt{n}(T_n-\theta)\xrightarrow{d}\varphi'(\theta)\sqrt{n}(T_n-\theta)$も成り立つのであろうか?
\end{tcolorbox}

\begin{theorem}[デルタ法]\label{thm-delta-method}
    関数$f:\R^{d_1}\supset A\to\R^{d_2}$は$c\in A$で微分可能であるとする.
    $\forall_{n\in\N}\;P[X_n\in A]=1$を満たす$d_1$次元確率変数列$\{X_n\}\subset\R^{d_1}$に対して,
    $\infty$に発散する係数列$\{b_n\}\subset\R$が定める線型変換は$b_n(X_n-c)\xrightarrow{d}Z\in\R^d$を満たすとする.このとき,
    \[b_n[f(X_n)-f(c)]-\partial_xf(x)b_n(X_n-c)\xrightarrow{p}0\quad(n\to\infty).\]
    特に,$b_n[f(X_n)-f(c)]\xrightarrow{d}\partial_xf(c)Z\;(n\to\infty)$である.
    $\partial_xf(c)$は$d_2\times d_1$-行列であることに注意.
\end{theorem}
\begin{remarks}
    列$(b_n[f(X_n)-f(c)])$は$(b_n\partial_xf(c)(X_n-c))$に漸近同等であることがわかった.
    これは1次のTaylor展開に拠ると思えることが名前の由来である.
    これをBanach空間上で行う技法を関数デルタ法という.
\end{remarks}

\subsection{分布収束の性質}

\begin{tcolorbox}[colframe=ForestGreen, colback=ForestGreen!10!white,breakable,colbacktitle=ForestGreen!40!white,coltitle=black,fonttitle=\bfseries\sffamily,
title=]
    連続写像定理を用いて,分布収束の遺伝法則が求まる.
    さらに,同じ分布に収束するためには,差$X_n-Y_n$が$0$に確率収束することが必要である.
    これを漸近同等といい,さらに一様緊密性と併せて
    $O_p(1),o_p(1)$の概念が先に定まる.
    $O_p(1)$の概念はタイトともいうが,$\R$に限れば,確率収束の意味で有界であるための条件に読める.
\end{tcolorbox}

\subsubsection{分布収束の遺伝}

\begin{theorem}[Slutsky]\label{thm-Slutsky}
    $X,X_n,Y_n$を確率ベクトルまたは確率行列とする.
    $c$を定数ベクトルまたは定数行列とする.$X_n\xrightarrow{d}X,Y_n\xrightarrow{p}c$のとき,次が成り立つ.
    \begin{enumerate}
        \item $(X_n,Y_n)\xrightarrow{d}(X,c)$.
        \item $X_n+Y_n\xrightarrow{d}X+c$.
        \item $Y_nX_n\xrightarrow{d}cX$.
        \item $c$が正則行列のとき,$Y_n^{-1}X_n\xrightarrow{d}c^{-1}X$.
    \end{enumerate}
\end{theorem}
\begin{remark}
    定値関数ではない$Y$に対して$Y_n\xrightarrow{p}Y$であっても,結合ベクトルの収束$(X_n,Y_n)\xrightarrow{d}(X,Y)$は一般には成り立たない.
\end{remark}

\begin{corollary}
    $X,X_n,Y_n$が$d$次元確率変数で,$X_n\xrightarrow{d}X$かつ$X_n-Y_n\xrightarrow{p}0$を満たすならば,$Y_n\xrightarrow{d}X$が成り立つ.
\end{corollary}

\subsubsection{漸近同等性}

\begin{definition}[asymptotically equivalent, bounded in probability / tight]\mbox{}
    \begin{enumerate}
        \item 条件$X_n-Y_n\xrightarrow{p}0$なる条件を,$(X_n),(Y_n)$は\textbf{漸近同等}であるといい,$X_n\equiv^aY_n$と表すこととする.
        \item $X_n=o_p(1):\Leftrightarrow X_n\xrightarrow{p}0$とし,(一様に)緊密または\textbf{確率有界}であることを$X_n=O_p(1):\Leftrightarrow\forall_{\ep>0}\;\exists_{M>0}\;\sup_{n\in\N}P[\abs{X_n}\ge M]<\ep$と表す.
    \end{enumerate}
\end{definition}

\begin{lemma}
    $X_n\xrightarrow{d}X$ならば,$X_n=O_p(1)$である.特に,$X_n=o_p(1)$ならば$X_n=O_p(1)$である.
\end{lemma}
\begin{proof}
    漠収束の特徴付け\ref{prop-characterization-of-value-convergence}より.
\end{proof}

\begin{lemma}
    $(X_n),(Y_n)$を確率変数列とする.
    \begin{enumerate}
        \item $X_n=O_p(1),Y_n=O_p(1)$ならば,$X_nY_n=O_p(1),X_n+Y_n=O_p(1)$.
        \item $X_n=O_p(1),Y_n=o_p(1)$ならば,$X_nY_n=o_p(1),X_n+Y_n=O_p(1)$.
    \end{enumerate}
\end{lemma}

\begin{definition}
    正数列$\{r_n\}\subset\R_{>0}$に対して,$X_n=o_p(r_n):\Leftrightarrow\frac{X_n}{r_n}=o_p(1)$とし,$X_n=O_p(r_n):\Leftrightarrow\frac{X_n}{r_n}=O_p(1)$.
\end{definition}

\subsection{安定収束}

\begin{definition}
    任意の確率収束する確率変数列$Y_n\xrightarrow{p}Y$に対して$(X_n,Y_n)\xrightarrow{d}(X,Y)$が成り立つとき,$X_n\xrightarrow{d}X\;(stably)$と表し,\textbf{安定収束}するという.
\end{definition}
\begin{example}
    独立確率変数列$(X_n)$,マルチンゲール$(X_n)$など,中心極限定理が従う多くの場合,これが法則収束するとき,安定収束する.
\end{example}
\begin{remark}
    安定収束の概念は非エルゴード的統計において最も基本的な役割を演じる.非エルゴード統計とは,条件付きFisher情報量の極限にランダム性が残る場合をいう.
\end{remark}

\subsection{弱収束が定める概収束列}

\begin{theorem}[Skorokhod representation theorem]
    距離空間$(S,\cS)$上の確率測度の列$(P_n)$は,可分な台をもつ$P$に弱収束するとする.
    このとき,確率空間$(\Om,\F,P)$とその上の$S$-値確率要素の列$(X_n)$が存在して,確率測度$(P_n)$をそれぞれ誘導し,$P$に概収束する.
    また,$S=\R$のとき,$\Om=(0,1)$,$X_n(y)=\inf\Brace{z\in\R\mid P_n(-\infty,z]\ge y}$と取れる.
\end{theorem}
\begin{remarks}
    あくまで測度の$*$-弱収束が主軸である,とも捉えられる.
\end{remarks}

\subsection{一様可積分性の定義と特徴付け}

\begin{tcolorbox}[colframe=ForestGreen, colback=ForestGreen!10!white,breakable,colbacktitle=ForestGreen!40!white,coltitle=black,fonttitle=\bfseries\sffamily,
title=]
    平均作用素はノルム連続であるが,各収束概念を保存することとの関係は込み入っている.そこで,$E$が収束を保つための十分条件を調べると同時に,
    概収束と$p$-次平均収束の関係を精査する.
\end{tcolorbox}

\begin{definition}[uniformly integrable]\mbox{}
    \begin{enumerate}
        \item 列$(X_n)$が\textbf{一様可積分}であるとは,
        \[\lim_{\lambda\to\infty}\sup_{n\in\N}E[\abs{X_n}|\abs{X_n}\ge\lambda]=0\]
        が成り立つことをいう.
        \item 族$(X_\lambda)_{\lam\in\Lambda}$が\textbf{一様可積分}であるとは,
        \[\lim_{A\to\infty}\sup_{\lambda\in\Lambda}E_\lambda[\abs{X_\lambda}1_{\abs{X_\lambda}}\ge A]=0\]
        が成り立つことをいう.ただし,確率変数$X_\lambda$の定義域を確率空間$(\Om_\lambda,\F_\lambda,P_\lambda)$とし,これが定める積分を$E_\lambda$とした.
    \end{enumerate}
\end{definition}
\begin{remarks}
    列$(\abs{X_n})$の末尾の積分が,一様に$0$に近づくことをいう.
    また定義から,$\{X_n\}$が一様可積分であることと$\{\abs{X_n}\}$が一様可積分であることとは同値で,このとき$\sup_{\lambda\in\Lambda}E_\lambda[\abs{X_\lam}]<\infty$が成り立つ(これは一部特徴付ける).
    つまり,分布族$\{P^{X_\lambda}_\lambda\}$の性質であって,例えば列$X_n(x):=x1_{[0,n]}(x)$は,$P$が正規分布のとき一様可積分であるが,Cauchy分布であるときはそうではない.
\end{remarks}

\begin{lemma}[一様可積分性の十分条件]
    $Y$を可積分な確率変数とする.
    \begin{enumerate}
        \item $\lim_{\lambda\to\infty}E[\abs{Y}|\abs{Y}\ge\lambda]=0$.
        \item 列$(X_n)$が$\forall_{n\in\N}\;\abs{X_n}\le Y$を満たすならば,$(X_n)$は一様可積分である.
    \end{enumerate}
\end{lemma}

\begin{lemma}[一様可積分性の特徴付け]
    列$(X_n)$について,次の2条件は同値.
    \begin{enumerate}
        \item $(X_n)$は一様可積分である.
        \item あるBorel可測関数$\psi:\R\to\R_+$が存在して,$\lim_{\abs{x}\to\infty}\frac{\abs{x}}{\psi(x)}=0,\sup_{n\in\N}E[\psi(X_n)]<\infty$を満たす.
    \end{enumerate}
    特に,$p>1$が存在して$\sup_{n\in\N}E[\abs{X_n}^p]<\infty$を満たすならば,$(X_n)$は一様可積分である.
\end{lemma}

\begin{corollary}
    $0<p<q$とするとき,$\sup_{\lambda\in\Lam}E_\lam[\abs{X_\lam}^q]<\infty$ならば,$\{\abs{X_\lam}^p\}_{\lam\in\Lam}$は一様可積分である.
\end{corollary}

\begin{lemma}[一様可積分性の特徴付け]
    列$(X_n)$が一様可積分であることは,次の2条件が成り立つことと同値:
    \begin{enumerate}
        \item $\sup_{n\in\N}E[\abs{X_n}]<\infty$.
        \item 測度としての$E$は絶対連続:$\forall_{\ep>0}\;\exists_{\delta>0}\;\forall_{A\in\F}\;P[A]<\delta\Rightarrow\sup_{n\in\N}E[\abs{X_n},A]<\ep$.
    \end{enumerate}
\end{lemma}

\subsection{収束概念が退化するための十分条件}

\begin{tcolorbox}[colframe=ForestGreen, colback=ForestGreen!10!white,breakable,colbacktitle=ForestGreen!40!white,coltitle=black,fonttitle=\bfseries\sffamily,
title=]
    一様可積分性が,次の系のような形で,$p$-次平均収束が他の収束概念と一致するための必要十分条件になる.
\end{tcolorbox}

\begin{theorem}[一般化されたLebesgueの優収束定理]
    $(X_n)$は一様可積分とする.
    \begin{enumerate}
        \item $X_n$が$X$に概収束するならば,$1$次平均収束する:$\lim_{n\to\infty}E[\abs{X_n-X}]=0$.
        特に,極限$X$は可積分で,$\lim_{n\to\infty}E[X_n]=E[X]$.
        \item $X_n$が$X$に確率収束することと,$1$次平均収束することとは同値.
    \end{enumerate}
\end{theorem}

\begin{corollary}[一様可積分性の特徴付け]
    可積分な確率変数の列$(X_n)$は$X$に概収束するとする.このとき次の3条件は同値:
    \begin{enumerate}
        \item $(X_n)$は一様可積分である.
        \item $(X_n)$は$X$に$1$次平均収束する:$\lim_{n\to\infty}E[\abs{X_n-X}]=0$.
        \item $E[\abs{X_n}]\to E[\abs{X}]$かつ$E[\abs{X}]<\infty$.
    \end{enumerate}
\end{corollary}

\begin{corollary}[さらに一般化]
    任意の$q\ge1$について,$E[\abs{X_n}^q]<\infty$かつ$X_n\xrightarrow{p}X$のとき,次の3条件は同値.
    \begin{enumerate}
        \item $\{\abs{X_n}^q\}_{n\in\N}$は一様可積分である.
        \item $E[\abs{X_n-X}^q]\to0$.
        \item $E[\abs{X_n}^q]\to E[\abs{X}^q]<\infty$.
    \end{enumerate}
\end{corollary}

\begin{theorem}[Dunford-Pettis]
    列$(X_n)$について,次の2条件は同値.
    \begin{enumerate}
        \item $(X_n)$は一様可積分である.
        \item $\{X_n\}\subset L^1(\Om)$は弱相対コンパクトである:$\forall_{Y\in L^\infty(\Om)}\;\lim_{n\to\infty}E[X_nY]=E[XY]$.
    \end{enumerate}
\end{theorem}

\section{不等式}

\subsection{確率不等式}

\begin{tcolorbox}[colframe=ForestGreen, colback=ForestGreen!10!white,breakable,colbacktitle=ForestGreen!40!white,coltitle=black,fonttitle=\bfseries\sffamily,
title=]
    $E[X]=\frac{\int XdP}{\int 1dP}$とみなせば,一般の測度$P$について成り立つ.
\end{tcolorbox}

\begin{notation}
    $(\Om,\F,P)$を確率空間とする.
\end{notation}

\begin{proposition}[Markov (1884), Chebyshev / Bienayme (1876)]\label{prop-Markov}
    $(\Om,\F,P)$上の確率変数$X$について,
    \begin{enumerate}
        \item (Markov) $\forall_{a>0}\;P[\abs{X}\ge a]\le\frac{1}{a}E[\abs{X}]$.
        \item (Chebyshev) 単調増加関数$g:\R_+\to\R_+$に対して,
        \[\forall_{a\ge 0}\quad P[\abs{X}\ge a]\le\frac{1}{g(a)}E[g(\abs{X})]\]
    \end{enumerate}
\end{proposition}
\begin{proof}\mbox{}
    \begin{enumerate}
        \item 事象$\{\abs{X}\ge a\}$上で$\frac{\abs{X}}{a}\ge 1$より,
        \[P[\abs{X}\ge a]=E[1_{\Brace{\abs{X}\ge a}}]\le E\Square{\frac{\abs{X}}{a}1_{\Brace{\abs{X}\ge a}}}\le \frac{1}{a}E[\abs{X}].\]
        \item $a\le\abs{X}\Rightarrow g(a)\le g(\abs{X})$より,
        \[P[\abs{X}\ge a]\le P[g(\abs{X})\ge g(a)]\le\frac{1}{g(a)}E[g(\abs{X})].\]
    \end{enumerate}
\end{proof}

\subsection{凸不等式}

\begin{tcolorbox}[colframe=ForestGreen, colback=ForestGreen!10!white,breakable,colbacktitle=ForestGreen!40!white,coltitle=black,fonttitle=\bfseries\sffamily,
title=]
    確率測度に関する期待値は,凸結合の一般化にもなっているため,凸解析の道具も流入する.
\end{tcolorbox}

\begin{proposition}[Jensenの不等式]
    確率変数$X$と凸関数$g:\R\to\R$に対して,$X,g(X)\in\L^1(\Om)$ならば,
    \[g(E[X])\le E[g(X)].\]
\end{proposition}

\subsection{Lebesgue空間のノルム不等式}

\begin{tcolorbox}[colframe=ForestGreen, colback=ForestGreen!10!white,breakable,colbacktitle=ForestGreen!40!white,coltitle=black,fonttitle=\bfseries\sffamily,
title=]
    $L^p(\R^d)\;(p\in[1,\infty])$はノルムに関して完備であるという事実は,Riesz-Fisherの定理と呼ばれている.
\end{tcolorbox}

\begin{proposition}[Holder]
    $p,q,r\in[1,\infty]$は$p^{-1}+q^{-1}=r^{-1}$を満たすとする.このとき,
    $X\in L^p\land Y\in L^q$ならば$XY\in L^r$であって,
    \[\norm{XY}_r\le\norm{p}\norm{q}.\]
\end{proposition}

\begin{corollary}\mbox{}
    \begin{enumerate}
        \item Cauchy-Schwarz $\norm{XY}_1\le\norm{X}_2\norm{Y}_2$.
        \item Lyapunov $0<a<b$かつ$\mu(\Om)=1$のとき,$X\in L^b$ならば$\norm{X}_a\le\norm{X}_b$.
    \end{enumerate}
\end{corollary}

\begin{proposition}[Minkowski]
    $p\in[1,\infty),X,Y\in L^p$について,
    \[\norm{X+Y}_p\le\norm{X}_p+\norm{X}_p\]
\end{proposition}

\subsection{畳み込みと不等式}

\begin{notation}
    一般に,局所コンパクトで単模である群$G$の両側不変Haar測度$\mu$に対して,$G$上の関数の畳み込みが考えられ,同じ結果が成り立つ.
\end{notation}

\begin{definition}
    \[f*g(x):=\int_{\R^d}f(x-y)g(y)dy\]
    によって,$\Meas(\R^d,\R)^2\nrightarrow\Meas(\R^d,\R)$を定める.
    Youngの不等式より,$L^p(\R^d)\times L^q(\R^d)$上に制限すると,積を定める.
\end{definition}

\begin{proposition}[Young]
    $p,q,r\in[1,\infty],p^{-1}+q^{-1}=r^{-1}+1$とする.$f\in L^p(\R^d),g\in L^q(\R^d)$のとき,$f^*g$は殆ど至るところ存在し,$f*g\in L^r(\R^d)$を満たし,
    \[\norm{f*g}_r\le\norm{f}_p\norm{g}_q.\]
\end{proposition}

\section{一般の試行と確率測度}

\begin{definition}[null set, complete]\mbox{}\label{def-complete-measure}
    \begin{enumerate}
        \item $P(N)=0$を満たす$P$-可測集合$N$を,$P$-零集合という.
        \item 任意の$P$-零集合の,任意の部分集合も全て$P$-可測(したがって零)である時,$P$を完備確率測度という.
    \end{enumerate}
\end{definition}

\begin{proposition}[Lebesgue expantion]
    任意の確率測度$P$に対し,完備拡張は必ず存在し,最小のものが一意に定まる.これを$P$のLebesgue拡大という.
\end{proposition}

\section{確率測度の拡張定理}

\begin{definition}[regular probability measure]\label{def-regular-measure}
    位相空間上の確率測度を考える.
    \begin{enumerate}
        \item 定義域がBorel classと一致するものを,\textbf{Borel確率測度}という.
        \item Borel確率測度のLebesgue拡大を\textbf{正則確率測度}という.
    \end{enumerate}
\end{definition}

\section{確率測度の直積}

\begin{tcolorbox}[colframe=ForestGreen, colback=ForestGreen!10!white,breakable,colbacktitle=ForestGreen!40!white,coltitle=black,fonttitle=\bfseries\sffamily,
title=]
    独立性は$P$の集合積に対する関手性だから,数学的には直積と関係が深い.
\end{tcolorbox}

\begin{definition}
    $(\Om_n,\F_n,P_n)$を確率空間の列とする.$\Om:=\prod_{n\in\N}\Om_n$とし,$\pr_n:\Om\to\Om_n$を射影とする.射影が$\Om$上に定める$\sigma$-加法族を,積$\sigma$-加法族という.
    これをKolmogorovの$\sigma$-加法族ともいう.
    この上の確率測度を,直積確率測度という.
\end{definition}
\begin{remark}\label{remark-cylinder-sets}
    射影$q_n:=(\pr_1,\cdots,\pr_n)$による$\sigma$-加法族$\cS_1\times\cdots\times\cS_n$の引き戻し$q_i^*(A_i)\;(A_i\in\cS_i)$
    \[C_A^{(n)}=\Brace{\om\in\Om\mid(\om_1,\cdots,\om_n)\in A}\quad A\in\cS_1\times\cdots\times\cS_n\]
    を\textbf{柱状集合}という.
    柱状集合の全体$\cC$は有限加法族で,これが生成する$\sigma$-加法族がKolmogorov $\sigma$-加法族である.
\end{remark}

\begin{lemma}\mbox{}
    \begin{enumerate}
        \item 直積確率測度は一意的である.
        \item 直積確率測度は存在する.
    \end{enumerate}
\end{lemma}

\begin{theorem}\label{thm-product-sigma-algebra}
    $\Om_n$はいずれもPoland空間で,$\F_n$がBorel集合族であるとき,コルモゴロフの$\sigma$-加法族$\F$は,$\Om$の直積位相が生成するBorel集合族に一致する.
\end{theorem}

\section{標準確率空間}

\begin{tcolorbox}[colframe=ForestGreen, colback=ForestGreen!10!white,breakable,colbacktitle=ForestGreen!40!white,coltitle=black,fonttitle=\bfseries\sffamily,
title=]
    多様体に対するEuclid空間のように,標準的な空間を用意するために,同型の理論を整備し,$(\R,\mu)$の同型類を特徴づける.
\end{tcolorbox}

\begin{definition}[canonical measure]\label{def-canonical-measure}
    $P$を$\Omega$上の完備な確率測度とする.$(\Omega,P)$が,正則確率測度を備えた実数空間$(\R,\mu)$に同型である時,$(\Omega,P)$を\textbf{標準確率空間}と呼ぶ.
\end{definition}

\begin{definition}[pushforward measure]
    $S$上の測度$P$の,可測関数$f:S\to T$による
    像測度を,$Pf^{-1}$または$f_*P$で表す.
\end{definition}

\begin{lemma}
    $P$が完備ならば$Q$も完備である.
\end{lemma}

\chapter{確率論の基礎概念}

\begin{quotation}
    単なる可測関数と測度としての範囲を超える確率変数・確率分布特有の概念と,その極限の取り扱いを議論する.
    大事な概念は独立性と収束である.
    \begin{enumerate}
        \item 古典的な確率論を特徴付けてきたところの数学的な概念は,試行の独立性と確率変数の独立性の概念である.
        確率変数列の独立性は,確率測度が定める関係である.
        Laplace, Poisson, Chebyshev, Markov, von Mises, Bernsteinの古典的研究は,独立な確率変数列についての基礎的な研究であり,Markov, Bernsteinによる現代的な研究は,完全な独立性よりも弱めたMarkov性の条件を課して考察している.
        このようにして,独立性の概念に,確率論固有の問題意識,少なくともその萌芽が見られる\cite{Kolmogorov}.
        なお事象の独立性とは,測度$P:\F\to[0,1]$が,積$\cap$なる演算に対して可換(準同型)であること:$P(A\cap B)=P(A)\cdot P(B)$と理解できる.
        \item 独立性は,事象に対する定義から,$\sigma$-加法族・確率変数族へと一般化され,独立な確率変数列なる対象が構成される.
        \item 
    \end{enumerate}
\end{quotation}

\section{確率測度の変換:条件付き確率とBayesの定理}

\begin{tcolorbox}[colframe=ForestGreen, colback=ForestGreen!10!white,breakable,colbacktitle=ForestGreen!40!white,coltitle=black,fonttitle=\bfseries\sffamily,
title=確率空間の万華鏡による拡大]
    離散の場合を考えると明らかであるが,確率空間とは「事象をどのように分割するか」が肝要になり,情報を得ることとはそれについての知識の更新だと言える.
    その基本的な言葉は,「条件付き確率」にある.
    部分代数を取り出しているようなもので,再帰的な構造がある.分母の$P(A)$は規格化条件で,再び確率測度を与えるため,これは確率測度の変換の一種である.
    ここでは可算な分割のみを考える(一般には条件付き期待値の議論になる).

    さらに進んだものだと,重点サンプリングやGirsanov変換が,ファイナンスやフィルタリング問題で用いられる測度変換の例である.
\end{tcolorbox}

\begin{definition}[conditional probability]\label{def-conditional-probability}\mbox{}
    \begin{enumerate}
        \item $A,B\in\calF,P(A)>0$とする.
        \[P(B\mid A):=\frac{P(A\cap B)}{P(A)}\]
        を,事象$A$の下での$B$の条件付き確率という.$P_A(B)$とも表す.\footnote{$i:A\mono\Om$についての引き戻し測度である?}
        $P(A)=0$の時はこの値は任意に定めることで,2変数測度$P(\cdot\mid\cdot):\calF\times\calF\to[0,1]$が定まる.
        \item $P(\cdot|A)$は再び$\Om$上の確率測度となる.
        \item $P_X(B):=P(B|X=\cdot)$は$X$上の可測関数となる.
    \end{enumerate}
\end{definition}
\begin{remark}[conditional expectation]
    新たな確率空間$(A,\F\cap P(A),P_A)$における積分作用素を,条件付き期待値という.
    これらの定義は,事象$A\in\F$を指定している裏で暗黙に指定されている$\sigma$-代数$\F\cap P(A)$について一般的に定義できる.
\end{remark}

\begin{proposition}[law of total probability:全確率の分解]\label{prop-law-of-total-probability}
    $(H_i)_{i\in I}$を互いに背反で$\Omega=\coprod_{i\in I}H_i$を満たす事象の族とする.
    \[\forall_{A\in\calF}\;P(A)=\sum_{i\in I}P(A\mid H_i)\cdot P(H_i).\]
\end{proposition}
\begin{proof}
    $A=A\cap\paren{\cup_{i\in I}H_i}=\cup_{i\in I}A\cap H_i$より,
    \begin{align*}
        P(A)&=P\paren{\cup_{i\in I}A\cap H_i}\\
        &=\sum_{i\in I}P(A\cap H_i)&加法性\\
        &=\sum_{i\in I}P(A\mid H_i)P(H_i)&定義\ref{def-conditional-probability}
    \end{align*}
\end{proof}
\begin{remarks}
    こうして\ref{sec-event}節で確認したとおり,分割$\sum H_i$の取り方が測度の取り替えについて肝要になる.
\end{remarks}

\begin{theorem}[Bayes]
    $(H_i)_{i\in I}$を互いに背反で$\Omega=\coprod_{i\in I}H_i$を満たす事象の族とする.$P(A)>0$ならば,
    \[P(H_i\mid A)=\frac{P(A\mid H_i)P(H_i)}{\sum_{j\in I}P(A\mid H_j)P(H_j)}.\]
\end{theorem}
\begin{proof}
    \begin{align*}
        P(H_i\mid A)&=\frac{P(H_i\cap A)}{P(A)}&条件付き確率の定義\ref{def-conditional-probability}\\
        &=\frac{P(A\mid H_i)P(H_i)}{\sum_{j\in I}P(A\mid H_j)P(H_j)}&分母は定義\ref{def-conditional-probability},分子は全確率の分解則\ref{prop-law-of-total-probability}
    \end{align*}
\end{proof}
\begin{remarks}
    条件付き確率の第一引数・第二引数の入れ替え法則という意味で,何かの変換則に見える.情報の更新規則.\footnote{In quantum mechanics, the collapse of the wavefunction may be seen as a generalization of Bayes's Rule to quantum probability theory. This is key to the Bayesian interpretation of quantum mechanics.}
    $I=1$の場合は,
    \[P(H\mid E)=\frac{P(E\mid H)P(H)}{P(E)}\quad\Leftrightarrow\quad P(H\mid E)P(E)=P(E\mid H)P(H)=P(E\cap H)\]
    となる.
\end{remarks}

\section{事象の独立性}

\begin{tcolorbox}[colframe=ForestGreen, colback=ForestGreen!10!white,breakable,colbacktitle=ForestGreen!40!white,coltitle=black,fonttitle=\bfseries\sffamily,
title=積への関手性]
    測度は純粋的に加法的な集合関数であって,積との相互関係は一切考えなかった.
    ここで,積への関手性は,確率論的に重要な意味論を持つことをみる.
    独立性は確率測度$\P$の構造のみに依存する,純粋に確率論的な概念である.
\end{tcolorbox}

\subsection{互いに独立な事象}

\begin{definition}[independent]
    事象$A,B\in\calF$について,
    \begin{enumerate}
        \item 可測空間$(\Omega,\calF)$で事象$A,B\in\calF$が背反であるとは,$A\cap B=\emptyset$であることをいう.
        \item 測度空間$(\Omega,\calF,\P)$で事象$A,B\in\calF$が独立であるとは,$\P(A\cap B)=\P(A)\P(B)$であることをいう.$P(A)>0$のとき,$P(B\mid A)=P(B)$に同値.
    \end{enumerate}
\end{definition}
\begin{remark}
    $P(A),P(B)>0$のとき,$A,B$が背反であることと独立であることとは両立し得ない(背反である).
    これは,「同時には起こらない」というのは従属関係であって,無関係たり得ないとも捉えられる.
\end{remark}

\begin{lemma}[補集合演算に関する関手性]
    $A,B\in\F$について,次の4条件は同値.
    \begin{enumerate}
        \item $A,B$は独立.
        \item $A^\comp,B$は独立.
        \item $A,B^\comp$は独立.
        \item $A^\comp,B^\comp$は独立.
    \end{enumerate}
\end{lemma}

\begin{theorem}[独立性の特徴付け]
    2つの事象$X,Y\in\calF$について,次の3条件は同値.
    \begin{enumerate}
        \item $X,Y$は独立である.
        \item $N:=\Brace{x\in\Om^X\mid \exists_{y\in\Om^Y}\;P(Y=y|X=x)\ne P(Y=y)}$について,$P^X(N)=0$.
        \item 殆ど至る所$P_X(Y=y)=P(Y=y)$.
    \end{enumerate}
\end{theorem}
\begin{proof}\mbox{}
    \begin{description}
        \item[(1)$\Rightarrow$(2)] 任意の$x\in N$について,$X,Y$の独立性より$P(X=x)>0$ならば$P_{X=x}\{Y=y\}=P\{Y=y\}$が必要だから,$P\{X=x\}=0$.よって,$P^X(N)=P\{X\in N\}=\sum_{x\in N}P\{X=x\}=0$.
    \end{description}
\end{proof}

\subsection{事象族の独立性}

\begin{definition}[事象族の独立性, pairwisely independent]\mbox{}
    \begin{enumerate}
        \item 列$(A_i)_{i\in n}:n\to\calF$が独立であるとは,$\forall_{2\le k\le n}\;\forall_{1\le i_1<\cdots<i_k\le n}\;P(A_{i_1}\cap\cdots\cap A_{i_n})=P(A_{i_1})\cdots P(A_{i_k})$.
        \item 族$(A_i)_{i\in I}$が独立であるとは,任意の有限部分集合について$\forall_{n\ge 2}\;A_1,\cdots,A_nが独立$であることとする.
        \item 族$(A_i)_{i\in I}$が対独立であるとは,$\forall_{i\ne j\in I}\;A_i\cap A_j=\emptyset$.
    \end{enumerate}
\end{definition}

\begin{example}
    対独立であるが独立ではない例がある.
\end{example}

\begin{lemma}[独立性の遺伝]
    事象$A_1,\cdots,A_n$が独立であるとする.
    \begin{enumerate}
        \item 事象$A_1^\complement,\cdots,A_n^\complement$も独立である.
        \item $A_1\cup A_2, A_3\cap A_4,A_5$は独立である.
    \end{enumerate}
\end{lemma}

\begin{lemma}[独立事象に対する確率の関手性]\label{lemma-functority-on-independence}
    列$(A_i)_{i\in\N}$が独立ならば,
    \[P\paren{\cap^\infty_{n=1}A_n}=\prod_{n=1}^\infty P(A_n).\]
\end{lemma}
\begin{proof}
    $B_n:=\cup^n_{k=1}A_k$と定めると,これは単調列であるから,
    \begin{align*}
        P\paren{\cap^\infty_{n=1}A_n}&=\lim_{n\to\infty}P(B_n)\\
        &=\lim_{n\to\infty}\prod_{k=1}^nP(A_k)=\prod_{n=1}^\infty P(A_n).
    \end{align*}
\end{proof}

\subsection{$\sigma$-代数の独立性}

\begin{definition}[$\sigma$-代数の独立性]\mbox{}
    \begin{enumerate}
        \item $\F$の部分$\sigma$-代数$\F_1,\F_2$が独立であるとは,$\forall_{C_1\in\F_1,C_2\in\F_2}\;P(C_1\cap C_2)=P(C_1)\cap P(C_2)$を満たすことをいう.
        \item 部分代数の有限列$(\F_k)_{k\in[n]}$が独立であるとは,$\forall_{k\in[n]}\;\forall_{C_k\in\F_k}\;P(\cap_{k\in[n]}C_k)=\prod_{k\in[n]}P(C_k)$を満たすことをいう.
        \item 部分代数族が独立であるとは,任意の有限部分集合が独立であることをいう.
        \item 部分代数族が対独立であるとは,任意の2つの部分代数$\F_i,\F_j$が互いに独立であることをいう.
    \end{enumerate}
\end{definition}

\section{確率変数と確率分布}

\begin{tcolorbox}[colframe=ForestGreen, colback=ForestGreen!10!white,breakable,colbacktitle=ForestGreen!40!white,coltitle=black,fonttitle=\bfseries\sffamily,
title=]
    前節で確率測度の変換を初等的な言葉で捉えた.
    ここで確率変数とは,確率測度の変換を引き起こす射であると考えられる.
    人間は,主に実数値のものを考える.$\R$は数理モデルにおいて特別なのだ.
\end{tcolorbox}

顕著な特徴として,統計推測などにおいて,確率変数$X:\Om\to\X$が誘導する測度$P^X$を観測することが問題となり,$(\Om,\F,P)$の特定の構造には執着しない.特に,経験過程論のように,$(\Om,\F,P)$を拡大して考えることがしばしばである.

\section{確率変数の独立性}

\begin{tcolorbox}[colframe=ForestGreen, colback=ForestGreen!10!white,breakable,colbacktitle=ForestGreen!40!white,coltitle=black,fonttitle=\bfseries\sffamily,
title=]
    確率空間とは「分割」の定め方である.
    確率変数が独立であるとは,これらが定める分割($\sigma$-加法族)が独立になることをいう.
    これは明らかに,一般化された概念である.

    独立性は$P$の集合積に対する関手性だから,数学的には直積によって構成する.
\end{tcolorbox}

\begin{definition}[independent]\label{def-independentness-of-random-variables}
    確率変数$X_1,\cdots,X_n:\Omega\to\X_i$が独立であるとは,次を満たすことをいう:
    \[\forall_{B_1\in\B_1,\cdots,B_n\in\B_n}\;\P^X(X_1\in B_1,\cdots,X_n\in B_n)=\P^X(X_1\in B_1)\times\cdots\times\P^X(X_n\in B_n).\]
    ただし,$X:=(X_1,\cdots,X_n)$を同時分布,$P^X$は直積測度$P^{X_1}\times\cdots\times P^{X_n}$とした.とした.事象$(A_i)$の独立性は,$X_i=\chi_{A_i}$の場合に当たる.
\end{definition}

\begin{proposition}[可測関数は独立性を保つ]\label{prop-measurable-functions-perserve-independentness}
    可測空間$(\Y_i,\calC_i)$への射$f_i:\X_i\to\Y_i$について,
    確率変数$X_1,\cdots,X_n$が独立ならば,$f_1(X_1),\cdots,f_n(X_n)$も独立である.
\end{proposition}
\begin{proof}
    合成$f_i\circ X_i$も可測だから,確かに$f_i(X_i)$も確率変数である.
    任意の$C_i\in\calC_i$について,$P^Y(f_i(X_i)\in C_i)=P^X(X_i\in f^{-1}_i(C_i))$であるから,
    \begin{align*}
        P^Y(f_1(X_1)\in C_1,\cdots,f_n(X_n)\in C_n)&=P^X(X_1\in f^{-1}_1(C_1))\times\cdots\times P^X(X_n\in f^{-1}_n(C_n))\\
        &=P^Y(f_1(X_1)\in C_1)\times\cdots\times P^X(f_n(X_n)\in C_n).
    \end{align*}
\end{proof}

\begin{proposition}
    確率変数$A_1,\cdots,A_n:\Omega\to\X$が独立であるとする.
    任意の$I:=[n]\supset J$について,$\{A_j^\complement,A_k\mid j\in J,k\in I\setminus J\}$は独立である.
\end{proposition}
\begin{proof}
    写像$f:\X\to\X$を,$A_j$を$A_j^\complement$に写し,$A_i$を変えない$f(A_i)=A_i$写像として定義できたなら,この像変数も独立であることから従う.
\end{proof}

\section{独立な確率変数}

\begin{tcolorbox}[colframe=ForestGreen, colback=ForestGreen!10!white,breakable,colbacktitle=ForestGreen!40!white,coltitle=black,fonttitle=\bfseries\sffamily,
title=確率変数が独立のとき,作用素に種々の関手性が生じる]
    関数の積分と確率変数の期待値との間にある類似点が明らかになってきた.こうした類推はさらに拡張され,
    独立な確率変数のさまざまな性質は,対応する直交関数の性質と完全に類似しているものとみなされるようになった\cite{Kolmogorov}.
\end{tcolorbox}

\subsection{確率変数の独立性}

\begin{definition}
    $S_k$値確率変数列$(X_k)_{k\in[n]}$が独立とは,
    \[\forall_{A_1\in\cS_1,\cdots,A_n\in\cS_n}\;P[X_k\in A_k,k\in[n]]=\prod^n_{k=1}P[X_k\in A_k]\]
\end{definition}

\begin{lemma}[well-definedness]\label{lemma-independentness-variable-and-algebra}
    次の2条件は同値.
    \begin{enumerate}
        \item $(X_n)$は独立.
        \item $(\sigma(X_n))$は独立.
    \end{enumerate}
\end{lemma}

\subsection{独立な確率変数に対する関手性}

\begin{proposition}[実空間の議論への持ち上げ]\label{prop-積関数の期待値}
    実確率変数の族$X:=(X_1,\cdots,X_n):\Om\to\R^n$と任意の可測関数$f:\R^n\to\R$に対して,
    \begin{enumerate}
        \item $h:=f\circ X:\Om\to\R^n\to\R$は実確率変数である.
        \item $E[h]=\int_{\R^n}f(x_1,\cdots,x_n)P^X(dx_1,\cdots,dx_n)$.
    \end{enumerate}
\end{proposition}
\begin{proof}
    可測関数の合成は可測だから,$h$は当然実確率変数である.
    \begin{description}
        \item[$f$が特性関数の場合] $f=\chi_A\;(A\in\B(\R^n))$とする.
        \begin{align*}
            E[\chi_A\circ X]&=E[\chi_{X^{-1}(A)}]=1\cdot P(X^{-1}(A))\\
            &=P^X(A)=\int_{\R^n}\chi_A(x_1,\cdots,x_n)P^X(dx).
        \end{align*}
        \item[$f$が単関数の場合]
        積分の線形性より成り立つ.
        \item[$f$が一般の可測関数の場合]
        $f=f^+-f^-$について,それぞれの非負値単関数近似から,Lebesgueの優収束定理より.
    \end{description}
\end{proof}

\begin{corollary}[期待値が積を保つ条件]\label{cor-mean-of-product-of-independent-variables}
    実確率変数の族$(X_i)_{i\in[n]}$が独立である時,任意の可積分な可測関数列$(f_i)_{i\in[n]}$に対して,$h:=(f_1\circ X_1)\cdot(f_2\circ X_2)\cdots(f_n\circ X_n)$とすると,$h$も可積分で次が成り立つ:$E[h]=\prod_{i=1}^nE[f_i\circ X_i]$.
\end{corollary}
\begin{proof}
    命題\ref{prop-積関数の期待値}より
    \[E[h]=\int_{\R^n}f_1(x_1)\cdots f_n(x_n)P^X(dx_1,\cdots,dx_n)\]
    であるが,$P^X$は直積測度$P^{X_1}\times\cdots\times P^{X_n}$であり\ref{def-independentness-of-random-variables},
    測度空間$\R^n$は$\sigma$-有限であるから,Fubiniの定理より,$\prod_{i\in[n]}E[f_i(X_i)]$に等しい.
\end{proof}

\begin{corollary}[分散が和を保つ条件]\label{cor-linearity-of-Var-on-independent-variables}
    $(X_i)_{i\in[n]}$を二乗可積分で:$E[\abs{X_i}^2]<\infty\;(i=1,\cdots,n)$,対独立な確率変数の列とする.
    この時,次が成り立つ:$\Var\Square{\sum^n_{i=1}X_i}=\sum^n_{i=1}\Var[X_i]$.
\end{corollary}
\begin{proof}
    2変数の場合,
    \begin{align*}
        V[X+Y]&=E[X+Y-E[X+Y]]=E[((X-EX)+(Y-EY))^2]\\
        &=V[X]+V[Y]+V(X,Y).
    \end{align*}
    であるから,互いに独立な2変数の共分散は$0$であることを示せば良いが,
    \[V(X,Y)=E[(X-EX)(Y-EY)]=E[XY]-E[X]E[Y]=0.\]
\end{proof}
\begin{remarks}
    ものすごく余弦定理っぽく,内積の構造がある.実際Hilbert空間の内積である.
    だから二乗可積分の条件があるのだ.
\end{remarks}

\subsection{独立同分布}

\begin{tcolorbox}[colframe=ForestGreen, colback=ForestGreen!10!white,breakable,colbacktitle=ForestGreen!40!white,coltitle=black,fonttitle=\bfseries\sffamily,
title=]
    さらに理想的なクラスを定義する.
    確率変数は分布を定めるが,分布からそれを定める確率変数が存在するかはある種の逆問題で,必ずしも自明でない.
\end{tcolorbox}

\begin{definition}[independent and identically distributed]
    実確率変数の列$(X_i)_{i\in\N}$が互いに独立であるだけでなく,任意の分布$P^{X_i}\;(i\in\N)$が等しい時,$(X_i)_{i\in\N}$は\textbf{独立同分布}を持つという.
\end{definition}

\begin{theorem}[Kolmogorov extension theorem]\label{thm-Kolmogorov-extension-theorem}
    確率空間の列$(\R^n,\B(\R^n),P_n)$が次の条件を満たすとする:
    \begin{quote}
        (consistency) $\forall_{A\in\B(\R^n)}\;P_n(A)=P_{n+k}(A\times\R^k)$.
    \end{quote}
    この時,$(\R^\infty,\B(\R^\infty))$上の確率測度$P$で,$\forall_{A\in\B(\R^n)}\;P(\pi_n^{-1}(A))=P_n(A)$を満たすものが唯一つ存在する.ただし,$\pi_n:=(\pr_1,\cdots,\pr_n)$と定めた.
\end{theorem}
\begin{proof}\mbox{}
    \begin{description}
        \item[方針] 任意の$A_n\in\B(\R^n)$に対して,$\Lambda:=\pi^{-1}_n(A_n)$での値を$Q(\Lambda):=P_n(A_n)$とする有限加法的な確率測度$Q:\B(\R^\infty)\to[0,1]$を考える.有限加法性の確認は,族$(A_n)_{n\in N}$に対して$m:=\max_{n\in N}\dim(A_n)$として$\R^m$上での$P_n$の有限加法性を考えれば良い.
        なお,この$Q$はwell-definedである:$A_n\in\B(\R^n),A_m\in\B(\R^m)$を用いて$\Lambda=\pi^{-1}_n(A_n)=\pi^{-1}_m(A_m)$と2通りで表せる場合でも,一貫性の条件より$Q(\Lambda)=P_n(A_n)=P_m(A_m)$である.
        $\A:=\Brace{\pi^{-1}_n(A_n)\in\R^\infty\mid A_n\in\B(\R^n),n\in\N}$は集合体であることは同様にして明らか.すると,$Q$が$\A$上で$\sigma$-加法的であることを示せば,$\R^\infty$は$\sigma$-有限であるから,$\sigma(\A)=\B(\R^\infty)$上への一意的な延長が存在し,$Q$の定め方よりこれが条件を満たす(Hopf-Kolmogorovの拡張定理).
        すると,補題より,任意の単調減少列$(A_n)_{n\in\N}$について,$\alpha:=\lim_{n\to\infty}Q(A_n)>0\Rightarrow\cap_{n\in\N}A_n\ne\emptyset$を示せば良い.
        \item[証明]
        単調減少列$(\Lambda_n)$を任意に取ると,$\Lambda_n=\pi^{-1}_{n_i}(A_{n_i})$と表せる.$\max_{n\in\N}n_i$が存在するならば同様にして自明だから,$\{n_i\}_{n\in\N}\subset\N$は非有界とする.すると,部分列をとって添字を打ち直すことより,$\Lambda_n=\pi^{-1}_n(A_n)$として良い.

        Borel集合の位相的正則性より,$C_n\subset A_n,P_n(A_n\setminus C_n)<\frac{\alpha}{2^{n+1}}$を満たすコンパクト集合の列$(C_n\subset\R^n)_{n\in\N}$が取れる.$D_n:=\pi^{-1}_n(C_n)$と定めると,$D_n\subset\Lambda_n,Q(\Lambda_n\setminus D_n)<\frac{\alpha}{2^{n+1}}$を満たす.
        $\o{D}_n:=\cap_{k=1}^nD_k$とおくと,
        \begin{align*}
            Q(\o{D}_n)&=Q(\Lambda_n)-Q(\Lambda_n\setminus\o{D}_n)\\
            &\ge Q(\Lambda_n)-\sum^n_{k=1}Q(\Lambda_k\setminus D_k)\ge\frac{\alpha}{2}
        \end{align*}
        より,$\o{D}_n\ne\emptyset$である.よって,空でない閉集合の単調減少列の極限は空ではなく,$\cap^\infty_{k=1}D_k\ne\emptyset$.

        こうして,$D_k\subset\Lambda_k$であって,$\emptyset\subsetneq\cap^\infty_{k=1}D_k\subset\cap^\infty_{k=1}\Lambda_k$が従う.
    \end{description}
\end{proof}
\begin{remark}
    これはHopfの拡張定理の一般化に当たる.
    一般の完備可分距離空間上の確率空間の列について成り立つ.
    さらに一般的な空間上については,$\sigma$-加法族の表現が変わる\ref{thm-product-sigma-algebra}.
\end{remark}

\begin{lemma}[有限加法的確率空間の完全加法性の単調族による特徴付け]
    $Q$を有限加法的な確率測度,$\A$を集合体とする.この時,$Q$についての次の2条件は同値.
    \begin{enumerate}
        \item $\A$上$\sigma$-加法的である.すなわち,$A\in\A$に収束する互いに素な$\A$-列$\{A_n\}_{n\in\N}\subset\A$について,$Q\paren{A}=\sum_{n\in\N}Q(A_n)$.
        \item 任意の$\A$の単調減少列$(A_n)$に対して,$Q\paren{\lim_{n\to\infty}A_n}=Q\paren{\bigcap_{n\in\N}A_n}=\lim_{n\to\infty}Q(A_n)$.
        \item 任意の$\A$の単調減少列$(A_n)$に対して,
        $\lim_{n\to\infty}Q(A_n)>0\Rightarrow\cap_{n\in\N}A_n\ne\emptyset$である.
    \end{enumerate}
\end{lemma}
\begin{proof}\mbox{}
    \begin{description}
        \item[(1)$\Rightarrow$(2)] $\A$上の有限加法的測度$\mu$が$\A$上$\sigma$-加法的であることは,任意の単調増加列$(A_n)$について$\mu(\lim_{n\to\infty}A_n)=\mu(\cup_{n\in\N}A_n)=\lim_{n\to\infty}A_n$を満たすことと同値.任意の単調減少列$(A_n)$に対して,その補集合の定める単調増加列を考えると,\[Q(\lim_{n\to\infty}A_n)=Q(\cap_{n\in\N}A_n)=1-Q(\cup_{n\in\N}\o{A_n})=1-\lim_{n\to\infty}Q(\o{A_n})=\lim_{n\to\infty}Q(A_n).\]最後の等号は$Q$の有限加法性$\forall_{n\in\N}\;Q(A_n+\o{A_n})=Q(A_n)+Q(\o{A_n})=1$による.
        \item[(2)$\Rightarrow$(3)] 自明.
        \item[(3)$\Rightarrow$(1)] $\A$内に収束する互いに素な列$(A_n)$を任意に取り,$A:=\cup_{n\in\N}A_n$と定める.
        $B_1:=A,B_n:=A\setminus\paren{\cup_{i=1}^{n-1}A_{i}}$と帰納的に定めると,これは$\emptyset$に収束する単調減少列である.よって,
        \begin{align*}
            \lim_{n\to\infty}Q(B_n)=0&\Leftrightarrow \lim_{n\to\infty}Q(A\setminus(\cup_{i=1}^{n-1}A_i))=0\\
            &\Leftrightarrow \lim_{n\to\infty}(Q(A)-Q(\cup_{i=1}^{n-1}A_i))=0&\because Qの有限加法性\\
            &\Leftrightarrow Q(A)=\lim_{n\to\infty}Q(\cup_{i=1}^{n-1}A_i)=\sum_{n=1}^\infty Q(A_n).&\because Qの有限加法性
        \end{align*}
    \end{description}
\end{proof}

\begin{theorem}[独立同分布をもつ確率変数の族の存在]\label{thm-existence-of-random-variables-to-iid}
    $\R$上の確率測度$\mu$を分布にもつ$\R^\infty$上の独立同分布$(X_i)_{i\in\N}$が存在する.
\end{theorem}
\begin{proof}\mbox{}
    \begin{description}
        \item[構成] 確率空間の列$((\R^l,\B(\R^l),P_l))_{l\in\N}$を$P_l:=\otimes^l_{i=1}\mu$と定めると,一貫性条件を満たすからKolmogorovの拡張定理\ref{thm-Kolmogorov-extension-theorem}より,$\R^\infty$上の確率測度$P$で
        $\forall_{A\in\B(\R^n)}\;P(\pr_n^{-1}(A))=P_n(A)$を満たすものが定まる.これに対して,$X_n:=\pr_n:\R^\infty\to\R$と定めれば良い.
        \item[確認] 実際,
        \begin{enumerate}
            \item (同分布) 任意の$i\in\N$と$E_i\in\B(\R)$に対して,$P$の定め方より,
            \begin{align*}
                P^{X_i}(E_i)&=P(\Brace{X_i\in E_i})=P\paren{\cap_{j<i}\Brace{X_i\in\R}\cap\Brace{X_i\in E_i}}\\
                &=P\paren{\pi^{-1}(\R^{i-1}\times E_i)}=P_i(\R^{i-1}\times E_i)&Pの定め方(一貫性条件)\\
                &=1\cdot\mu(E_i).&P_iの定義
            \end{align*}
            よって$\forall_{i\in\N}\;P^{X_i}=\mu$であるから,$(X_i)_{i\in\N}$は同分布.
            \item 
            任意の$i\in\N$に対して,$\wt{X}:=(X_1,\cdots,X_n)$を同時分布とすると,任意の$E=E_1\times\cdots\times E_n\in\B(\R^n)$に対して,
            \begin{align*}
                P^{\wt{X}}(E)&=P\paren{\cap^n_{i=1}\Brace{X_i\in E}}=P(\pi^{-1}_n(E))\\
                &=P_n(E)=\prod^n_{i=1}\mu(E_i)=\prod^n_{i=1}P^{X_i}(E_i)
            \end{align*}
            より,独立性も従う.
        \end{enumerate}
    \end{description}
\end{proof}

\begin{example}[Bernoulli sequence / process]\label{exp-Bernoulli-Process}
    特に確率空間$2$を介する射は,ヤヌス対象や2進法やTVではないが,特殊なクラスの確率変数である.
    $P(\{X_i=0\})=p,P(\{X_i=1\})=1-p\;(0\le p\le 1)$なる分布$P^X:\{0,1\}\to[0,1]$に従う2値確率変数の列$(X_i)_{i\in\N}$を\textbf{Bernoulli過程}という.
    その存在は独立同分布を持つ確率変数の族の存在\ref{thm-existence-of-random-variables-to-iid}により保証される.
    最初の$n$回のうちの成功数は二項分布に従い,$r$回成功するのに必要な回数は負の二項分布に従う.$r=1$を幾何分布という.
\end{example}

\subsection{独立確率変数列}

\begin{theorem}
    $(\mu_n)$を$(\R,\B(\R))$上の確率測度の列とする.
    このとき,ある確率空間$(\Om,\F,P)$とその上の独立な確率変数列$(X_n)$が存在して,$\forall_{n\in\N}\;X_n\sim\mu_n$を満たす.
\end{theorem}

\section{確率変数の和・積・商の分布}

\begin{tcolorbox}[colframe=ForestGreen, colback=ForestGreen!10!white,breakable,colbacktitle=ForestGreen!40!white,coltitle=black,fonttitle=\bfseries\sffamily,
title=]
    確率変数の積$(X_1,\cdots,X_n):\Om_1\times\cdots\times\Om_n\to\X_1\times\cdots\times\X_n$が引き起こす分布を\textbf{同時・結合分布}という.
    これは単に標準的な構成であるが,では,確率変数の演算は,測度の演算にどのように対応するのであろうか?
    独立な確率変数の和が引き起こす分布は,測度の畳み込みである.
\end{tcolorbox}

\begin{definition}[convolution of measures]
    $(\R,\B(\R))$上の測度$\mu,\nu$の\textbf{畳み込み}とは,確率測度
    \[\mu*\nu(E):=\int_\R\nu(E-y)\mu(dy)=\iint_\R\chi_E(x+y)\nu(dx)\mu(dy)\]
    を指す.ただし,$E-y:=\Brace{x-y\in\R\mid x\in E}$を$E$を平行移動した集合とした.
    \footnote{集合$E$を平行移動しながら,元の位置から移動させた時の測度の変化を足し上げていく.}
\end{definition}

\begin{proposition}
    2つの確率変数$X_1,X_2$は像測度$\mu,\nu$を定め,互いに独立であるとする.この時,確率変数$X_1+X_2$の像測度は畳み込み$\mu*\nu$である.
\end{proposition}
\begin{proof}
    同時分布を$X:=(X_1,X_2)$とおくと,$X_1,X_2$は互いに独立であるから,像測度は直積測度に一致する:$P^X=P^{X_1}\otimes P^{X_2}$.
    よって,任意の事象$E\in\B(\R)$について,
    \begin{align*}
        \mu*\nu(E)&=\iint_\R\chi_E(x+y)P^{X_1}(dx)P^{X_2}(dy)\\
        &=\iint_\R\chi_E(x+y)P^X(dxdy)\\
        &=P(X_1+X_2\in E)=P^X(E).
    \end{align*}
\end{proof}

\section{可分完全確率測度}

\begin{tcolorbox}[colframe=ForestGreen, colback=ForestGreen!10!white,breakable,colbacktitle=ForestGreen!40!white,coltitle=black,fonttitle=\bfseries\sffamily,
title=]
    Kolmogorovは晩年完全性を追加した.
    伊藤清の教科書では可分性も追加した.
    完全性は像測度に遺伝する.
    可分性はどこで効いてくるかはわからない.
\end{tcolorbox}

\begin{definition}[perfect (Kolmogorov)]\label{def-perfect-measure}
    完備な確率空間$(S,\mu)$上の任意の$\mu$-可測関数$f:S\to\R$に対し,
    像測度$\mu_*f$が正則になるとき,確率測度$\mu$を\textbf{完全}という.
\end{definition}

\begin{definition}[separating family, separable]\mbox{}
    \begin{enumerate}
        \item $S$上の集合族$\A$が分離族であるとは,$\forall_{s_1\ne s_2\in S}\;\exists_{A\in\A}\;1_A(s_1)\ne 1_A(s_2)$を満たすことをいう.
        \item 完備確率測度$\mu$について,$\dom(\mu)$が可算分離族を含むとき,$\mu$を\textbf{可分}という.
    \end{enumerate}
\end{definition}

\begin{theorem}
    $\mu$を$S$上の確率測度,$f:S\to T$を可測関数とする.
    像測度$\nu:=\mu f^{-1}$は完全である.
\end{theorem}

\section{事象と確率変数}

\begin{notation}[extension]
    条件$\alpha\subset\Omega$について,$\alpha$を成立させるような元からなる集合$\{\alpha\}:=\{\omega\in\Omega\mid \alpha(\omega)\}$を$\alpha$の外延という.
    すると,$\{X(\omega)\le a\}=X^{-1}((-\infty,a])$などと表せる.
\end{notation}

\section{条件付き期待値}

\begin{tcolorbox}[colframe=ForestGreen, colback=ForestGreen!10!white,breakable,colbacktitle=ForestGreen!40!white,coltitle=black,fonttitle=\bfseries\sffamily,
    title=]
    条件付き期待値は素朴には,部分代数$\cG\subset\F$について,各$B\in\cG$上で,$\F$が定める測度を積分する演算である.
    構成論はLebesgue積分論で終わらせて居るため,定義は性質のみによって行い,零関数の差に目を瞑る.
\end{tcolorbox}

\subsection{動機}

\begin{discussion}[事象の条件付き確率が定める条件付き期待値]
    ある事象$B\in\F$が定める条件付き確率$P(-|B)$は,$(\Om,\F)$上の確率測度である.
    これが定める積分を,条件付き期待値と呼べる.
    \[E[X|B]:=\frac{E[X,B]}{P(B)}.\]
    ただし,$E[X,B]=E[1_BX]$とした.
    これは,積分範囲の制限と規格化に他ならない.
    これだけでは,概念の射程が限られる.
\end{discussion}

\begin{discussion}[条件付き確率の一般化]
    そこで,条件付き確率の概念を一般の$\sigma$-代数に一般化することで,条件付き期待値を一般化することを考える.
    まずは,有限な直和分割が生成する$\sigma$-代数を考える.

    $(B_i)$を事象による$\Om$の有限な直和分割でいずれも零集合でないとする.
    $P[A|(B_i)]:=\sum^n_{i=1}P(A|B_i)1_{B_i}$と定め,条件付き期待値は
    \[E[X|(B_i)]:=\sum_{i=1}^nE[X|B_i]1_{B_i}\]
    と定めると,これは先程の定義の凸結合が与える単関数となっている.
    $(B_i)$の生成する$\sigma$-代数を$\cG$で表すと,それぞれを$P[A|\cG],E[A|\cG]$と表す.
\end{discussion}

\begin{discussion}[測度論的議論]
    有限生成とは限らない一般の$\sigma$-代数$\cG$に対する条件付き期待値を定義したい.
    単関数の無限和とは,積分に他ならない.
    実は裏技が存在して,満たすべき性質を指定するのみで,Radon-Nykodymの定理により,平均は一意的に定まる.
\end{discussion}

\subsection{定義}

\begin{tcolorbox}[colframe=ForestGreen, colback=ForestGreen!10!white,breakable,colbacktitle=ForestGreen!40!white,coltitle=black,fonttitle=\bfseries\sffamily,
title=]
    $E[-|\cG]:\Meas_\F(\Om,\R)\to L^1(\Om,\cG)$は,$\F$-可測な確率変数に対して,ある$\cG$-可測な確率変数の同値類を与える.
    しかし,$E[X,B]=E[Y,B]$を満たすため,確率変数としての本質は変わらない.すなわち,
    $\cG$に応じて,解像度を粗くするのである.
\end{tcolorbox}

\begin{definition}[conditional expectation]
    次の条件を満足する確率変数$Y:\Om\to\R$を,$\cG$に関する$X$の\textbf{条件付き期待値}と呼ぶ.
    \begin{enumerate}
        \item $Y$は$\cG$-可測で$P$-可積分.
        \item 任意の$B\in\cG$に対して$E[X,B]=E[Y,B]$すなわち$\int_BX(\om)P(d\om)=\int_BY(\om)P(d\om)$を満たす.
    \end{enumerate}
    この$Y$を$E[X|\cG]$で表す.
    $X$が可測関数の特性関数である場合,$P(A|\cG):=E[1_A|\cG]\;(A\in\F)$を\textbf{条件付き確率}という.
\end{definition}

\begin{example}[自明な例]\mbox{}
    \begin{enumerate}
        \item $\cG=\F$のとき,$E[X|\F]=X\;\as$である.
        \item $\cG=\{\emptyset,\Om\}$のとき,$E[X|\cG]=E[X]$で,定数関数である.
    \end{enumerate}
\end{example}

\begin{lemma}[well-definedness]\mbox{}
    \begin{enumerate}
        \item $E[X|\cG]$は存在する.
        \item $E[X|\cG]$は$P$-零集合を除いて一意である.
    \end{enumerate}
\end{lemma}
\begin{proof}
    $Q(B):=E[X,B]=E{1_BX}\;(B\in\cG)$をおくことで,$Q$は$(\Om,\cG)$上の確率測度を定める.
    いま,$P|_\cG$に関して$Q$は絶対連続:$\forall_{B\in\cG}\;P(B)=0\Rightarrow Q(B)=0$が成り立つから,Radon-Nikodymの定理より,
    ある$\cG$-可測で$P$-可積分な関数$Y:\Om\to\R$が存在して,$\forall_{B\in\cG}\;Q(B)=\int_BY(d\om)P(d\om)$が成り立つ.
    よって,(1),(2)が成り立つ.
\end{proof}

\begin{example}[条件付き確率]
    $(\Om_j)_{j\in\N}$を$\Om$の分割で,$P[\Om_j]>0$とする.
    $\cG:=\sigma[\Om_j|j\in\N]$と定めると,可積分確率変数$X$に関して,
    \[E[X|\cG]=\sum_{j\in\N}\frac{E[X1_{\Om_j}]}{P(\Om_j)}1_{\Om_j}\quad\as\]
    が成り立つ.
\end{example}

\subsection{性質}

\begin{tcolorbox}[colframe=ForestGreen, colback=ForestGreen!10!white,breakable,colbacktitle=ForestGreen!40!white,coltitle=black,fonttitle=\bfseries\sffamily,
title=]
    $E[-|\cG]:\L^1(\Om,\F)\to L^1(\Om,\cG)$は関数空間上の正な線型作用素である.
    1次平均収束を保つという意味で,$\L^1(\Om,\F)$上ノルム連続,すなわち有界である.

    また,Hilbert空間$L^2(\Om,\F)$上で見ると,$E[-|\cG]$は,$\cG$-可測関数がなす閉部分空間への直交射影となる.
\end{tcolorbox}

\begin{lemma}
    $X,Y\in\L^1(\Om,\F)$とする.
    \begin{enumerate}
        \item 線形性:$\forall_{a,b\in\R}\;E[aX+bY|\cG]=aE[X|\cG]+bE[Y|\cG]\;\as$.\footnote{除外集合$N_{a,b}$は任意の$a,b\in\R$について一様に取ることは一般には出来ない.}
        \item 正:$X\ge0\as\Rightarrow E[X|\cG]\ge0\as$.特に,$X\le Y\as\Rightarrow E[X|\cG]\le E[Y|\cG]\as$.
        \item $X\in\Meas_\cG(\Om,\R)$,$XY\in\L^1(\Om,\F)$のとき,$E[XY|\cG]=XE[Y|\cG]\as$.特に,$E[X|\cG]=X\as$.
        \item $\H\subset\cG$を部分$\sigma$-代数とする.$E[E[X|\cG]|\H]=E[X|\H]\as$.特に,$E[E[X|\cG]]=E[X]$.
        \item $\sigma(X)$と$\cG$とが独立ならば,$E[X|\cG]=E[X]\as$.したがって,$f:\R\to\R$をBorel可測関数とすると,$f(X)\in\L^1(\Om,\F)\Rightarrow E[f(X)|\cG]=E[f(X)]\as$.
    \end{enumerate}
\end{lemma}
\begin{proof}\mbox{}
    \begin{enumerate}
        \item 右辺$Z:=aE[X|\cG]+bE[Y|\cG]$は定義より,$\cG$-可測関数で$P$-可積分である.任意の$B\in\cG$について,
        \begin{align*}
            E[aX+bY,B]&=aE[X,B]+bE[Y,B]\\
            &=aE[E[X|\cG],B]+bE[E[Y|\cG],B]\\
            &=E[aE[X|\cG]+bE[Y|\cG],B]=E[Z,B]
        \end{align*}
        が成り立つ.よって,条件付き期待値の一意性より,
        $E[aX+bY|\cG]=Z\as$.
        \item $Z:=E[X|\cG]$は$\forall_{B\in\cG}\;E[Z,B]=E[X,B]\ge0\as$を満たす.これは$Z\ge\as$を含意する.
        \item 右辺は$\cG$-可測で$P$-可積分だから,任意の$B\in\cG$について$E[XY,B]=E[XE[Y|\cG],B]$を示せば,条件付き期待値の一意性から従う.
        単関数の場合から示す.
        \item 両辺とも$\H$-可測で$P$-可積分だから,任意の$B\in\H$について
        \[E[E[E[X|\cG],\H],B]=E[X,B]\]
        を示せば,条件付き期待値の一意性から従う.
        この左辺はまず$E[E[X|\cG],B]$に一致する必要があるが,$B\in\cG$でもあるから,これらはさらに$E[X,B]$なる右辺に一致する必要がある.
        \item 独立性は,$\forall_{B\in\cG}\;E[X,B]=E[1_BX]=E[X]P(B)=E[E[X],B]$を含意する\ref{cor-mean-of-product-of-independent-variables}.これは$E[X|\cG]=E[X]\as$を意味する.
        また,$f(X)$と$\cG$も独立である\ref{prop-measurable-functions-perserve-independentness}.
    \end{enumerate}
\end{proof}

\begin{proposition}[Jensenの不等式]
    $\psi:\R\to\R$を凸関数とする.
    $X,\psi(X)\in\L^1(\Om)$のとき,
    \[\psi(E[X|\cG])\le E[\psi(X)|\cG]\as\]
    特に,$X\in\L^p(\Om)$のとき,$\abs{E[X|\cG]}^p\le E[\abs{X}^p|\cG]\as$.
\end{proposition}

\begin{proposition}[条件付き期待値の連続性]
    $\L^1(\Om)$の列$(X_n)$が$X$に$1$次平均収束するとき,$E[X_n|\cG]$も$E[X|\cG]$に$1$次平均収束する.
\end{proposition}

\begin{theorem}[直交射影としての条件付き期待値]
    $L_\cG^2$を,$L^2(\Om,\F)$内の$\cG$-可測関数がなす閉部分空間とする.
    $Y\in L^2(\Om,\F)$について,
    \[E\Square{(Y-E[Y|\cG])^2}=\min_{Z\in L^2_\cG}E[(Y-Z)^2].\]
\end{theorem}

\subsection{可測写像を与えたもとでの条件付き期待値}

\begin{tcolorbox}[colframe=ForestGreen, colback=ForestGreen!10!white,breakable,colbacktitle=ForestGreen!40!white,coltitle=black,fonttitle=\bfseries\sffamily,
title=]
    部分$\sigma$-代数$i:(\Om,\cG)\mono(\Om,\F)$から,一般の可測写像へさらに一般化する.
\end{tcolorbox}

\begin{notation}
    $(\Om,\F,P)$上の可積分確率変数$X\in\L^1(\Om)$と,可測空間$(\cT,\B)$への可測写像$T:\Om\to\cT$を考える.
\end{notation}

\begin{definition}
    次の2条件を満たす関数$g:\cT\to\R$を,\textbf{$T=t$の下での$X$の条件付き期待値}という:
    \begin{enumerate}
        \item $g$は$\B$-可測かつ$P^T$可積分.
        \item $\forall_{B\in\B}\;\int_{T^{-1}(B)}X(\om)P(d\om)=\int_Bg(t)P^T(dt)$.
    \end{enumerate}
    $g$は$P^T$-零集合を除いて一意であり,これを$E[X|T=t]$で表す.すなわち,
    \[\forall_{B\in\B}\quad\int_{T^{-1}(B)}X(\om)P(d\om)=\int_BE[X|T=t]P^T(dt).\]
\end{definition}

\begin{lemma}[well-definedness]
    \[E[X|\sigma(T)]=E[X|T]\;P\text{-}\as\]
    ただし,$\sigma[T]:=\Brace{T^{-1}(B)\in P(\Om)\mid B\in\B}$で,$E[X|T](\om):=E[X|T=t]|_{t=T(\om)}$とした.
\end{lemma}

\begin{definition}[条件付き確率]\mbox{}
    \begin{enumerate}
        \item 部分$\sigma$-加法族$\cG\subset\F$に関する事象$A\in\F$の条件付き確率を,$P[A|\cG]:=E[1_A|\cG]$で定める.
        \item $T=t$の下での事象$A\in\F$の条件付き確率を,$P[A|T=t]:=E[1_A|T=t]$で定める.
    \end{enumerate}
\end{definition}
\begin{remark}
    これは規格化しておらず,実際に$A$上に確率測度を定めるかどうかは問うていない.
\end{remark}

\subsection{正則条件付き確率}

\begin{definition}[regular conditional probability]
    族$(p(\om,A))_{w\in\Om,A\in\F}$が次の3条件を満たすとき,\textbf{部分$\sigma$-代数$\cG$が与えられたときの正則条件付き確率}であるという:
    \begin{enumerate}
        \item $\forall_{\om\in\Om}\;p(\om,-):\F\to[0,1]$は確率測度を定める.
        \item $\forall_{A\in\F}\;p(-,A):\Om\to[0,1]$は$\cG$-可測.
        \item $\forall_{A\in \F}\;\forall_{B\in\cG}\;P(A\cap B)=\int_Bp(\om,A)P(d\om)$.
    \end{enumerate}
\end{definition}

\begin{definition}
    可測空間$(\Om,\F)$が条件(S)を満たすとは,
    \begin{enumerate}
        \item \cite{吉田}では,ある距離$d$によって完備可分距離空間となり,$\F$はそのBorel $\sigma$-加法族となること.
        \item \cite{伊藤清}では,可分完全確率空間とする.
        \item Ikeda-Watanabeでは,標準確率空間とする.
    \end{enumerate}
    応用上(1)で十分らしいので,これでいこう.
\end{definition}

\begin{theorem}[存在と一意性]
    条件(S)を満たす可測空間$(\Om,\F)$上の,任意の確率測度$P$と任意の部分$\sigma$-代数$\cG$に対して,$\cG$が与えられたときの正則条件付き確率$(p(\om,A))_{\om\in\Om,A\in\F}$は存在し,零集合の差を除いて一意である.
\end{theorem}

\begin{definition}
    $T:(\Om,\F,P)\to(\cT,\B),X:(\Om,\F,P)\to(\X,\A)$を可測とする.
    $(p(t,A))_{t\in\cT,A\in\A}$が,\textbf{$T=t$の下での$X$の条件付き確率分布}であるとは,次の3条件を満たすことをいう:
    \begin{enumerate}
        \item $\forall_{t\in\cT}\;p(t,-):\A\to[0,1]$は$(\X,\A)$上の確率測度である.
        \item $\forall_{A\in\A}\;p(-,A):\cT\to[0,1]$は$\B$-可測関数である.
        \item $\forall_{A\in\A}\;\forall_{B\in\B}\;P[X\in A,T\in B]=\int_Bp(t,A)P^T(dt)$.
    \end{enumerate}
\end{definition}

\begin{theorem}[存在と一意性]
    条件(S)を満たす可測空間$(\X,\A)$上の,$T=t$の下での$X$の条件付き確率分布は存在し,零集合の差を除いて一意である.
\end{theorem}

\begin{proposition}[Fubiniの類似]
    $f:\cT\times\X\to\R$は$\B\times\A$-可測で,$P^{(T,X)}$-可積分であるとする.このとき,
    \[\int_{\cT\times\X}f(t,x)dP^{(T,X)}(t,x)=\int_\cT\Square{\int_\X f(t,x)p(t,dx)}dP^T(t).\]
    特に,可積分実確率変数$X$に対して,
    \[E[X|T=t]=\int_\R xp(t,dx)\;P^T\text{-}\as\]
\end{proposition}

\begin{proposition}[多変量正規分布の条件付き分布は再び多変量正規分布となる]
    $d_1$次元確率変数$X_1$と$d_2$次元確率変数$X_2$の結合分布は$d_1+d_2$変量正規分布で,$E[X_i]=\mu_i,\Cov[X_i,X_j]=\Sigma_{ij}$とおく.
    $\Sigma_{11}\in M_{d_1}(\R)$が正則ならば,$X_1=x_1$の下での$X_2$の正則条件付き分布は,多変量正規分布$N_{d_2}(\mu_2+\Sigma_{21}\Sigma_{11}^{-1}(x_1-\mu_1),\Sigma_{22}-\Sigma_{21}\Sigma^{-1}_{11}\Sigma{12})$である.
\end{proposition}
\begin{remark}
    $\Sigma_{11}$が退化しているときも,$\Sigma^{-1}_{11}$を一般化逆行列とすれば,同様の結果が成り立つ.
\end{remark}

\section{$0$-$1$法則}

\begin{tcolorbox}[colframe=ForestGreen, colback=ForestGreen!10!white,breakable,colbacktitle=ForestGreen!40!white,coltitle=black,fonttitle=\bfseries\sffamily,
title=]
    末尾事象は,$\F$の部分$\sigma$-代数の列$(\B_k)$のうち,無限個によって指定される事象(各有限部分列と独立な事象)をいう.
    例えば,級数$\sum_{n=0}^\infty X_n$が収束するという事象は末尾事象である.

    これにより,「情報」的な概念を完全に$\sigma$-加法族に翻訳しつつある.Borel-Cantelliの補題の一般化であることは明らかである.
\end{tcolorbox}

\begin{definition}[tail field]
    $(\B_k)$を,$\F$の部分$\sigma$-代数の列とする.このとき,
    \[\cG_k:=\sigma\paren{\bigcup_{j=k}^\infty}\B_j,\qquad\cT:=\bigcap_{k=1^\infty}\cG_k\]
    と定まる$\cT$を,\textbf{末尾加法族}という.
\end{definition}

\begin{theorem}[Kolmogorov 0-1]
    $\forall_{A\in\cT}\;P(A)\in\{0,1\}$.
\end{theorem}

\begin{corollary}
    $(X_n)$を独立な確率変数列とする.見本平均の概収束極限$Y:=\lim_{n\to\infty}\frac{1}{n}\sum_{k=1}^nX_k$が存在するとする.このおき,$Y$は殆ど確実に定数である.
\end{corollary}

\chapter{独立確率変数列の和}

\begin{quotation}
    確率過程論への入門として,独立確率変数列の和に関して成り立つ極限定理を調べる.
    実際,大数の法則の一般化・精緻化が喫緊の問題であった1930年代の確率論では最重要分野であった.
    この理論の一般化がmartingaleである.
    \begin{enumerate}
        \item 見本平均は,和を規格化したものである(新たな測度を考えているとも捉えられる).この極限が収束することは末尾事象で,概収束か概発散かが起こる.実は確率$1$で収束する.
        \item 弱法則は,対独立性と分散の一様有界性を必要とする.強法則は独立性に関しては強い条件を要求するが,分散の有界性については弱められる.
        \item 実は応用上もっとも中心的な興味は,大数の法則の収束の速度に関する情報である.実は,速度は$O(1/\sqrt{n})$であり,1次の項の形も標準的に得られるが,収束の強さは法則収束までである.
    \end{enumerate}

    大偏差原理を見ると,確率論は関数解析と結びついて,現代の物理学を生んだような,とてつもない表現力を持ち得る可能性を感じる.
\end{quotation}

\section{確率不等式}

\begin{tcolorbox}[colframe=ForestGreen, colback=ForestGreen!10!white,breakable,colbacktitle=ForestGreen!40!white,coltitle=black,fonttitle=\bfseries\sffamily,
title=]
    Markovの不等式\ref{prop-Markov}より
    \[\forall_{k\in[n]}\;P\Square{\abs{S_k}\ge a}\le\frac{1}{a^2}E[\abs{S_k}^2]=\frac{1}{a^2}\sum_{i=1}^kV_i\le\frac{1}{a^2}\sum^n_{i=1}V_i\]
    同様の事実が,$\max_{k\in[n]}\abs{S_n}$についても同じ評価が出来る.
    そこで,マルチンゲールへの視点変更が起こった.
    劣マルチンゲールであるから,最後の時点$S_n$にだけ注目すれば良いのである.
\end{tcolorbox}

\begin{theorem}[Kolmogorov]\label{thm-Kolmogorov-inequality}
    実確率変数列$\{X_n\}\subset L^2(\Om,\F,P)$は独立で,$E[X_n]=0,V_n:=\Var[X_n]<\infty$を満たすとする.
    このとき,$S_k:=\sum^k_{i=1}X_k$とおくと,
    \[\forall_{a>0}\quad P[\max_{k\in[n]}\abs{S_k}\ge a]\le\frac{1}{a^2}\sum^n_{i=1}V_i.\]
\end{theorem}
\begin{proof}
    \[A^*:=\Brace{\om\in\Om\mid\max_{k\in[n]}\abs{S_k}\ge a}\qquad A^*_k:=\Brace{\om\in\Om\mid\forall_{i\in[k-1]}\;\abs{S_i}<a\land\abs{S_k}\ge a}\]
    とおくと,$A^*=\sum_{k\in[n]}A^*_k$が成り立ち,$A^*_k\in\sigma[X_1,\cdots,X_k]$.
    いま,$(S^2_n)$は劣マルチンゲールで,特に$\forall_{k\in[n-1]}\;E[Z^2_n,A_k^*]\ge E[Z_k^2,A_k^*]$より,
    \begin{align*}
        P[A^*]=\sum_{k\in[n]}P[A^*_k]&\le\sum_{k\in[n]}\frac{1}{a^2}E[S_k^2,A_k^*]&\because A^*_k\text{上では}a^2\le S^2_k\\
        &\le\frac{1}{a^2}\sum_{k\in[n]}E[S_n^2,A^*_k]
        =\frac{1}{a^2}E[S_n^2,A^*]\\
        &\le\frac{1}{a^2}E[S_n^2]
        =\frac{1}{a^2}\sum^n_{i=1}V_i.
    \end{align*}
\end{proof}

\section{独立同分布に関する大数の法則}

\begin{tcolorbox}[colframe=ForestGreen, colback=ForestGreen!10!white,breakable,colbacktitle=ForestGreen!40!white,coltitle=black,fonttitle=\bfseries\sffamily,
title=]
    大数の法則は物理学実験でも扱ったある種の自然現象であるが,
    これが定理として導けるような公理系を,我々は用意できたのである.
    実際は,いずれの大数の法則も,仮定は可積分性$E[\abs{X_1}]<\infty$で十分.
\end{tcolorbox}

\begin{definition}[convergence in probability, almost sure convergence]\mbox{}
    \begin{enumerate}
        \item 確率変数の族$(Y_i)_{i\in\N}$と確率変数$X$について,次が成り立つ時,$(Y_n)$は$X$に\textbf{確率収束}するという:$\forall_{\epsilon>0}\;\lim_{n\to\infty}P(\abs{Y_n-X}\ge\epsilon)=0$.
        \item 確率変数の族$(Y_i)_{i\in\N}$と確率変数$X$について,次が成り立つ時,$(Y_n)$は$X$に\textbf{概収束}するという:$P\paren{\lim_{n\to\infty}Y_n=X}=1$.
    \end{enumerate}
    それぞれを形式化すると,
    \begin{enumerate}
        \item $\forall_{\ep_1>0}\;\forall_{\ep_2>0}\;\exists_{N>0}\;\forall_{n\ge N}\;P(\Abs{Y_n-X}\ge\ep_1)<\ep_2$.
        \item $\forall_{\ep>0}\;\exists_{N>0}\;\forall_{n\ge N}\;P(Y_n=X)=1-\ep$.
    \end{enumerate}
    すると,(2)$\Rightarrow$(1)であるが,(1)$\Rightarrow$(2)は反例が構成できる.\footnote{$Y_n$がかさばりながら$X$に近づくとき,$\ep$範囲には必ず入るが,$\ep/2$範囲には半分しか入らない,というような収束の仕方もあるはずである.}
\end{definition}

独立同分布の場合は期待値について
\[E\Square{\Abs{\sum^n_{i=1}(X_i-\mu)}^{2k}}=\sum^n_{i_1,\cdots,i_{2k}}E[(X_{i_1}-\mu)\cdots(X_{i_{2k}}-\mu)]\le Cn^k\quad\exists_{C\in\R}\]
という評価が使えるので議論が簡単になる.

\subsection{独立同分布での大数の弱法則}

\begin{tcolorbox}[colframe=ForestGreen, colback=ForestGreen!10!white,breakable,colbacktitle=ForestGreen!40!white,coltitle=black,fonttitle=\bfseries\sffamily,
title=]
    まず収束とは何かが問題になる.収束とは基本的に距離空間で定義される概念であるが,測度を用いて一般化することが出来るのであった.
    確率収束は概収束より弱いので,
\end{tcolorbox}

\begin{notation}
    確率空間$(\Om,\F,P)$上の独立同分布$(X_i)_{i\in\N}$に対して,$S_n:=\sum_{i=1}^n$と定める.
\end{notation}

\begin{theorem}[weak law of large numbers]\label{thm-weak-law-of-large-numbers}
    独立同分布を持つ確率変数の族$(X_i)_{i\in\N}$が二乗可積分であるとする:$E[(X_1)^2]<\infty$.
    この時,次が成り立つ:
    \[\forall_{\epsilon>0}\;\lim_{n\to\infty}P\paren{\Abs{\frac{S_n}{n}-E[X_1]}\ge\epsilon}=0.\]
\end{theorem}
\begin{proof}
    Schwarzの不等式より,$E[\abs{X_1}]\le\sqrt{E[(x_1)^2]}<\infty$であるから,特に可積分である.
    任意の$\epsilon>0$と$n\in\N_{>0}$に対して,$X:=\abs{S_n-nE[X_1]},f(a)=a^2$とすると,
    Chebyshevの不等式\ref{thm-Chebyshev-inequality}より,
    \[P(\abs{S_n-nE[X_1]}\ge n\epsilon)\le\frac{E[\abs{S_n-nE[X_1]}^2]}{(n\epsilon)^2}=\frac{nE[\abs{X_1-E[X_1]}^2]}{(n\ep)^2}\]
    より結論を得る.なお,最右辺の変形は,独立な確率変数に対する分散の線型性\ref{cor-linearity-of-Var-on-independent-variables}による.
\end{proof}

\subsection{独立同分布での大数の強法則}

\begin{tcolorbox}[colframe=ForestGreen, colback=ForestGreen!10!white,breakable,colbacktitle=ForestGreen!40!white,coltitle=black,fonttitle=\bfseries\sffamily,
title=]
    確率収束は位相を定め,また距離化可能でもあるが,
\end{tcolorbox}

\begin{theorem}[strong law of large numbers]
    独立同分布を持つ確率変数の族$(X_i)_{i\in\N}$が4乗可積分であるとする:$E[(X_1)^4]<\infty$.
    この時,次が成り立つ:
    \[P\paren{\lim_{n\to\infty}\frac{S_n}{n}=E[X_1]}=1.\]
\end{theorem}
\begin{proof}
    以下のことに注意する.
    \begin{enumerate}
        \item Schwarzの不等式より$E[\abs{X_1}]\le(E[(X_1)^2])^{1/2}\le(E[(X_1)^4])^{1/4}<\infty$であるから特に可積分.
        \item 2次と4次の中心積率は
        \begin{align*}
            E[(X_1-\mu)^4]&\le 8(E[(X_1)^4]+\mu^4)<\infty,&E[(X_1-\mu)^2]&\le 2(E[(X_1)^2]+\mu^2)<\infty,
        \end{align*}
        と評価できる.
    \end{enumerate}
    \begin{description}
        \item[Chebyshevの不等式で期待値へ還元] $A^\ep_n:=\Brace{\om\in\Om\;\middle|\;\Abs{\frac{S_n(\om)}{n}-m}>\ep}$と置くと,$f(a)=a^4$についてのChebyshevの不等式\ref{thm-Chebyshev-inequality}より,
        \[P(A^\ep_n)=P\paren{\Abs{\sum^n_{i=1}(X_i-\mu)}>\ep n}\le\frac{E\Square{\paren{\sum^n_{i=1}(X_i-\mu)}^4}}{(\ep n)^4}\]
        と評価できる.
        \item[期待値を抑える] 4次の中心積率の和は
        \begin{align*}
            E\Square{\paren{\sum^n_{i=1}(X_i-\mu)^4}}&=\sum^n_{i_1,i_2,i_3,i_4=1}E[(X_{i_1}-\mu)(X_{i_2}-\mu)(X_{i_3}-\mu)(X_{i_4}-\mu)]\\
            &=nE[(X_1-\mu)^4]+3n(n-1)\paren{E[(X_1-\mu)^2]}^2&\because 独立な変数の共分散は0\\
            &\le Cn^2\quad\exists_{C\in\R}
        \end{align*}
        と評価できるから,
        \[\sum_{n=1}^\infty P(A^\ep_n)\le\sum^\infty_{n=1}\frac{C}{(\ep n)^4}n^2=\frac{C}{\ep^4}\sum^\infty_{n=1}\frac{1}{n^2}<\infty\]
        である.
        \item[Borel-Cantelliの補題]
        $B_\ep:=\limsup_{n\to\infty}A_n^\ep=\Brace{\om\in\Om\;\middle|\;\limsup_{n\to\infty}\Abs{\frac{S_n(\om)}{n}-\mu}>\ep}$とすると,Borel-Cantelliの補題\ref{lemma-Borel-Cantellii}(1)より,$P(B_\ep)=0$.
        \item[結論]
        以上より,補集合が
        \begin{align*}
            P\paren{\Brace{w\in\Om\;\middle|\;\lim_{n\to\infty}\Abs{\frac{S_n(\om)}{n}=\mu}}^\comp}&=P\paren{\lim_{\ep\to0}\Brace{\om\in\Om\;\middle|\;\limsup_{n\to\infty}\Abs{\frac{S_n(\om)}{n}-\mu}>\ep}}\\
            &\le P(\lim_{n\to\infty}B_\ep)=\lim_{\ep\to 0}P(B_\ep)\because (B_\ep)_{\ep>0}は単調減少族より.\\
            &=0.
        \end{align*}
    \end{description}
\end{proof}

\subsection{証明抽出と評価の精緻化}

\begin{tcolorbox}[colframe=ForestGreen, colback=ForestGreen!10!white,breakable,colbacktitle=ForestGreen!40!white,coltitle=black,fonttitle=\bfseries\sffamily,
title=]
    次の定理はHausdorffがBernoulli列の場合について最初に証明した.
\end{tcolorbox}

\begin{theorem}
    $\forall_{k\in\N}\;E[\abs{X_1}^k]<\infty$のとき,任意の$\ep'>0$に対して,$P\paren{\lim_{n\to\infty}\frac{S_n-n\mu}{n^{1/2+\ep'}}=0}=1$.$\ep'=1/2$のときを大数の強法則という.
\end{theorem}

\begin{theorem}[収束のオーダー:law of iterated logarithm (Khinchin)]
    $E[(X_1)^2]<\infty$のとき,$X_1$の分散を$\sigma^2$とすると,
    \[P\paren{\limsup_{n\to\infty}\frac{\abs{S_n}-n\mu}{\sqrt{2n\log\log n}}=\sigma}=1\]
\end{theorem}

\section{一般の大数の弱法則}

\begin{definition}
    一般の$\L^1(\Om,\F,P)$の列$(X_n)$に対して,$Y_n:=\frac{1}{n}\sum_{k=1}^nX_n,\o{m}_n:=\frac{1}{n}\sum_{k=1}^nm_k$とおく.
    \begin{enumerate}
        \item $\lim_{n\to\infty}P(\abs{Y_n-\o{m}_n}>\ep)=0$が成り立つとき,大数の弱法則を満たすという.
        \item $P\paren{\lim_{n\to\infty}\abs{Y_n-\o{m}_n}=0}=0$が成り立つとき,大数の強法則を満たすという.
    \end{enumerate}
\end{definition}

\begin{theorem}
    $\{X_n\}\subset\L^1(\Om,\F,P)$が次の2条件を満たすとき,大数の弱法則を満たす:
    \begin{enumerate}
        \item $(X_n)$は対独立である.
        \item $\sup_{n\in\N}\Var[X_n]<\infty$.特に,$\{X_n\}\subset\L^2(\Om,\F,P)$である.
    \end{enumerate}
    なお,確率収束するだけでなく,特に$2$次平均収束する.
\end{theorem}
\begin{proof}
    任意の$\ep>0$を取る.
    \begin{align*}
        P[\abs{\wt{Y_n}}>\ep]&\le\frac{E[\wt{Y}^2_n]}{\inf_{\abs{x}>\ep}x^2}&\because\text{Chebyshevの不等式}\ref{thm-Chebyshev-inequality}\\
        &=\frac{E[\wh{Y}^2_n]}{\ep^2}=\frac{1}{\ep^2}E[(Y_n-\o{m_n})^2]\\
        &=\frac{1}{\ep^2}E\Square{\frac{1}{n^2}\paren{\sum^n_{k=1}X_k-\sum^n_{k=1}m_k}^2}\\
        &=\frac{1}{\ep^2n^2}\sum^n_{j,k=1}E[(X_j-m_j)(X_k-m_k)]\\
        &=\frac{1}{\ep^2n^2}\sum^n_{k=1}E[(X_k-m_k)^2]&\because 対独立\\
        &=\frac{1}{\ep^2n^2}\sum^n_{k=1}\Var[X_k]\\
        &\le\frac{1}{\ep^2n}\sup_{n\in\N}\Var[X_n]\xrightarrow{n\to\infty}0.
    \end{align*}
    結局証明では,$E[\abs{Y_n-\o{m_n}}^2]\xrightarrow{n\to\infty}0$を示しているので,2次平均収束する.
\end{proof}

\section{一般の大数の強法則}

\begin{theorem}[Kolmogorov 1]
    $\{X_n\}\subset\L^1(\Om,\F,P)$が次の2条件を満たすとき,大数の強法則を満たす:
    \begin{enumerate}
        \item $(X_n)$は独立である.
        \item $\sum^\infty_{n=1}\frac{1}{n^2}\Var[X_n]<\infty$.
    \end{enumerate}
\end{theorem}

\begin{theorem}[Kolmogorov 2]
    独立同分布に従う確率変数の族$\{X_n\}\subset\L^1(\Om,\F,P)$は,大数の強法則を満たす.
\end{theorem}

\section{数学における大数の法則的現象}

\subsection{Weierstrassの多項式近似}

\begin{tcolorbox}[colframe=ForestGreen, colback=ForestGreen!10!white,breakable,colbacktitle=ForestGreen!40!white,coltitle=black,fonttitle=\bfseries\sffamily,
title=]
    Bernsteinの基底関数$b_{k,b}$をBernolli試行$B(n,x)$の確率$b(k;n,x)$を表していると見ると,Weierstrassの多項式近似の議論は大数の法則の議論と同じ構造をしている.
    すなわち,各サンプル$k/n\in[0,1]$で重み付きに近似していけば,$k/n\to x$に概収束するから,$f(x)$は$P_n(x)$で近似できる.
    すると,全ての関数は確率変数の退化(特殊化)なのかもしれない.となると,物理学理論が確率論化したのは自然で,いずれ全ての理論がそうなるであろうという新たな自然法則に向き合いつつあるのかもしれない.
    2項展開の各項をBernolli過程の確率を表す項$b(k;n,x)$と見る,という見方は,確率論を形式化した恩恵なのかもしれない.
\end{tcolorbox}

\begin{definition}[Bernstein polynomial]
    $b_{k,b}(x):=\begin{pmatrix}n\\k\end{pmatrix}x^k(1-x)^{n-k}\;(k\in n+1)$の形で表される多項式を,$n$次の\textbf{Bernsteinの(基底)関数}という.
    これらは$n$次以下の多項式がなす実線型空間の基底をなし,1の分割をなす:$\sum^n_{k=0}b_{k,n}=1$.
\end{definition}

これに対して,$B_n:C([0,1])\to\R[x]$を$B_n(f):=\sum_{k=0}^nf\paren{\frac{k}{n}}b_{k,n}$とおくと,一様位相において$\lim_{n\to\infty}B_n(f)=f$.

\begin{theorem}[Weierstrassの多項式近似]
    任意の連続関数$f\in C([0,1])$について,多項式の列$(P_n)_{n\in\N}\;(\deg P_n=n)$が存在して,$\lim_{n\to\infty}\max_{x\in[0,1]}\abs{P_n(x)-f(x)}=0$.
\end{theorem}
\begin{proof}\mbox{}
    \begin{description}
        \item[構成] 任意の$x\in[0,1]$について,これを成功確率とするBernoulli試行$B(n,x)$に従うBernoulli列$(X_i^x)_{i\in\N}$を取る(独立同分布に従う確率変数の列の存在定理\ref{thm-existence-of-random-variables-to-iid}).
        これが定める確率変数を$S_n^x:=\sum^n_{i=1}X^x_i$とすると,このBernoulli試行$B(n,x)$の期待値の$f$による押し出しの期待値を$P_n(x):=E\Square{f\paren{\frac{S_n^x}{n}}}$とすると,$k\in n+1$回成功する確率はそれぞれ$P(S_n=k)=\begin{pmatrix}n\\k\end{pmatrix}x^k(1-x)^{n-k}$と表せるため,
        \[P_n(x)=\sum^n_{k=1}f\paren{\frac{k}{n}}\cdot\begin{pmatrix}n\\k\end{pmatrix}x^k(1-x)^{n-k}\]
        とも表せる.
        \item[検証]
        いま,$\delta:\R_{>0}\to\R_{\ge 0}$を$\delta(\ep):=\sup\Brace{\abs{f(x)-f(y)}\in\R_{\ge 0}\mid x,y\in[0,1],\abs{x-y}\le\ep}$と定めると,$[0,1]$上の関数は連続ならば一様連続だから,$\ep\to0$のとき$\delta(\ep)\to0$.
        また,$M:=\max_{x\in[0,1]}\abs{f(x)}$とおくと,任意の$\ep>0$に対して,
        \begin{align*}
            \max_{x\in[0,1]}\abs{f(x)-P_n(x)}&=\max_{x\in[0,1]}\Abs{E\Square{f(x)-f\paren{\frac{S_n}{n}}}}&P_n(x)=E\Square{f\paren{\frac{S_n}{n}}}\\
            &\le\max_{x\in[0,1]}E\Square{\Abs{f(x)-f\paren{\frac{S_n}{n}}}}\\
            &=\max_{x\in[0,1]}\Brace{\int_{\Brace{\om\in\Om\mid\Abs{\frac{S_n(\om)}{n}-\mu}\ge\ep}}\Abs{f(x)-f\paren{\frac{S_n}{n}}}dP\right.\\
            &\hphantom{====}\left.+\int_{\Brace{\om\in\Om\mid\Abs{\frac{S_n(\om)}{n}-\mu}<\ep}}\Abs{f(x)-f\paren{\frac{S_n}{n}}}dP}\\
            &\le\max_{x\in[0,1]}2MP\paren{\Abs{S_n(\om){n}-x}\ge\ep}+\delta(\ep)&第一項はMの2倍で,第二項は\delta(\ep)で抑えられる\\
            &\le\max_{x\in[0,1]}\frac{2Me[\abs{X_1^x-x}^2]}{n\ep^2}+\delta(\ep)&大数の弱法則\ref{thm-weak-law-of-large-numbers}と同様Chebyshevの不等式\\
            &\le\frac{2M}{n\ep^2}+\delta(\ep)&B(1,x)の分散\le x(1-x)\le 1
        \end{align*}
        と評価できる.すると,$\forall_{\ep>0}\;\exists_{N>0}\;\forall_{n\ge N}\;\max_{x\in[0,1]}\abs{f(x)-P_n(x)}<\ep$を得た.
    \end{description}
\end{proof}
\begin{remarks}
    $f\in C([0,1])$を,確率空間$([0,1],\B([0,1]),P)$上の実確率変数だと思うと,少し難しすぎる.
    そこで,離散的な確率空間へと引き戻して考え,これらの離散空間からの実確率変数の列$\lim_{n\to\infty}(n^{-1})^*f$の極限だと考える:
    \[\xymatrix@R-2pc{
        n+1\ar[r]^-{\times\frac{1}{n}}&[0,1]\ar[r]^-{f}&\R\\
        \rotatebox[origin=c]{90}{$\in$}&\rotatebox[origin=c]{90}{$\in$}&\rotatebox[origin=c]{90}{$\in$}\\
        k\ar@{|->}[r]&\frac{k}{n}\ar@{|->}[r]&f\paren{\frac{k}{n}}.
    }\]
    すると,$f(x)$の値は,大数の法則より,確率$x\in[0,1]$で成功するBernoulli試行$B(n,x)$の期待値という確率変数$S_n/n$の$f$による押し出しで,近似できる.
\end{remarks}

\section{物理学における大数の法則的現象}

\subsection{Maxwell分布}

\begin{tcolorbox}[colframe=ForestGreen, colback=ForestGreen!10!white,breakable,colbacktitle=ForestGreen!40!white,coltitle=black,fonttitle=\bfseries\sffamily,
title=]
    $n$-粒子系の速度の分布を考えたい.
    熱力学的平衡状態についていくつかの仮定をおくと,正規分布のクラスとして,Maxwell-Boltzmann分布を得る.
\end{tcolorbox}

\begin{notation}
    半径$\sqrt{n}$の球面$\sqrt{n}S^{n-1}\subset\R^n$上の一様確率測度を$\sigma_n(dx)$とする.
    $k\le n$について,$(x_1,\cdots,x_k)$上の周辺分布を$k$次周辺分布といい,$\sigma^k_n\in\P(\R^k)$で表す.
    $\mu:=\mu_{0,1}\in\P(\R)$を平均$0$,分散$1$の標準正規分布とする.$\mu^k\in\P(\R^k)$を$k$重直積とすると,平均$0\in\R^k$,共分散行列$I_k\in M_k(\R)$を持つ$\R^k$上の正規分布となる.
\end{notation}
\begin{remarks}
    この条件は,$n$-粒子系の速度を$x_i$とし,条件$\sum_{i=1}^nx_i^2=n$を規格化されたエネルギー保存則とする.
    そして,一様確率測度は,等重率の原理なる作業仮説によって置かれる仮定であり,これらの条件の下で起こり得るすべての事象は等しい確率を持つとする.
    なお,このような仮定によって条件付けられた分布を微視的正準Gibbs分布(microcanonical Gibbs distribution)という.
\end{remarks}

\begin{theorem}
    $\forall_{k\in\N}\;\sigma^k_n\Rightarrow\mu^k\;(n\to\infty)$.
\end{theorem}

\subsection{熱力学的極限}

\begin{notation}
    $\Lambda_L:=[-L,L]^d\subset\R^d$に閉じ込められた$N$-粒子系を考える.
    単位体積あたりの粒子数$\frac{N}{(2L)^d}$は一定であるとして,$L,N\to\infty$の極限を考えたい.
    これを熱力学的極限という.
    このときの$\R^d$上の$N$-粒子の分布を,\textbf{強さ$\lambda$のPoisson点過程}という.

    $S:=\Lambda_L,p(A):=\frac{m(A)}{(2L)^d}\;(A\in\B(\R^d)\cap P(\Lambda_L))$が定める確率空間$(\Om:=S^N,P:=\prod_{i=1}^Np)$を考えると,$P$は正準Gibbs分布である.
\end{notation}

\begin{theorem}
    部分空間$D\subset\Lambda_L$に対して,
    その範囲で発見される粒子数を表す確率変数$n(D,-):\Om\to\Z$を
    \[n(D,\om):=\Abs{\Brace{k\in[N]\mid\om_k\in D}}\]
    と定めると,
    \[\forall_{l\in\N}\quad\lim_{L,N\to\infty,\frac{N}{(2L)^d}\to\lambda}P(n(D)=l)=e^{-\lambda m(D)}\frac{(\lambda m(D))^l}{l!}\]
\end{theorem}

\section{中心極限定理}

\begin{tcolorbox}[colframe=ForestGreen, colback=ForestGreen!10!white,breakable,colbacktitle=ForestGreen!40!white,coltitle=black,fonttitle=\bfseries\sffamily,
title=]
    偏差値は,統計的な分布が正規分布で近似できることを暗黙裡に認めて算出している.
    どう考えても「極限分布の標準分解」とか呼ぶべきだと思うが,Pólyaが1920年の論文で「確率論において中心的な役割を果たすであろう」ということから命名した.
    
    標準化された確率変数の$2$次までの積率は確定しており,漸近的に$3$次以上のキュムラントも消えるための十分条件がi.i.d.である.
\end{tcolorbox}

\subsection{標準化された部分和についての結果}

\begin{theorem}[Lindeberg-Levy]
    独立同分布を持つ確率変数列$\{X_n\}\subset\L^1(\Om,\F,P)$は$E[X_n]=\mu\in\R,\Var[X_n]=\sigma^2\in\R_+$を満たすとする.
    このとき,確率変数$Z_n:=\frac{1}{\sqrt{n}}\sum^n_{k=1}(X_k-\mu)$は,正規分布$N(0,\sigma^2)$に法則収束する.特に,
    \[\forall_{a<b\in\R}\quad\lim_{n\to\infty}P(a\le Z_n\le b)=\frac{1}{\sqrt{2\pi v}}\int^b_ae^{-\frac{x^2}{2v}}dx.\]
\end{theorem}

\subsection{Lindeberg-Fellerの一般化}

\begin{tcolorbox}[colframe=ForestGreen, colback=ForestGreen!10!white,breakable,colbacktitle=ForestGreen!40!white,coltitle=black,fonttitle=\bfseries\sffamily,
title=]
    独立同分布に従う確率変数の分散が有限な場合,それらの和の確率分布は正規分布に収束する.
    これを種々の作業仮説をおいて示すことが出来る.
    一方で,確率変数が従う分布の裾が重く($\abs{x}^{-a-1}\;(0<a<2)$の冪乗),分散が発散する場合,正規分布には収束せず,特性指数$a$の安定分布に収束する.
\end{tcolorbox}

\begin{theorem}
    $(k_n)_{n\in\N}:\N\to\N$に対して,$(\xi_{n,j})_{j\in[k_n]}=\xi_{n,1},\cdots,\xi_{n,k_n}\in\R^d$を各$n\in\N$が定める独立な$d$次元確率変数の$k_n$-組とし,
    次の2条件が成り立つとする.
    \begin{enumerate}[({A}1)]
        \item $\xi_{n,j}\in\L^2(\R^d)$は2乗可積分で,期待値ベクトルは零とする:$E[\xi_{n,j}]=0$.
        \item $\Sigma_n:=\sum^{k_n}_{j=1}\Var[\xi_{n,j}]\in M_d(\R)$は極限$\Sigma:=\lim_{n\to\infty}\Sigma_n$を持ち,さらに次が成り立つ:
        $\forall_{\ep>0}\;\sum^{k_n}_{j=1}E[\abs{\xi_{n,j}}^21_{\Brace{\abs{\xi_{n,j}}\ge\ep}}]\xrightarrow{n\to\infty}0$.
    \end{enumerate}
    このとき,
    \[\sum^{k_n}_{j=1}\xi_{n,j}\xrightarrow{d}N_d(0,\Sigma)\quad(n\to\infty)\]
\end{theorem}
\begin{remarks}
    通常は$(k_n)_{n\in\N}:\N\to\N$を恒等写像として,$\xi_{n,j}:=\frac{X_j}{\sqrt{n}}$としてLindeberg条件を満たすものを構成している.
\end{remarks}

\begin{example}
    $X_j=(Y_j,Z_j)$は独立に同一の2変量正規分布$N_2(\mu,\Sigma)$に従うとする.$\rho:=\rho(Y_j,Z_j)$とする.
    標本相関係数$\wh{\rho}_n$の漸近分布を求めたい.
    中心極限定理とデルタ法\ref{thm-delta-method}を組み合わせることで,
    $\sqrt{n}(\wh{\rho}_n-\rho)\xrightarrow{d}N(0,(1-\rho^2)^2)$とわかる.

    さらに良い結果を引き出すために,関数
    \[g(\rho):=\frac{1}{2}\log\paren{\frac{1+\rho}{1-\rho}}\]
    による変換$g(\wh{\rho}_n)$を考える.これを\textbf{Z変換}という.
    これについてもう一度デルタ法を用いると,
    $\sqrt{n}(g(\wh{\rho}_n)-g(\rho))\xrightarrow{d}N(0,1)$となり,漸近分散はパラメータ$\rho$に依存しなくなる.このような変換を\textbf{分散安定化変換}という.
\end{example}

\subsection{ノルム収束についての中心極限定理}

\begin{theorem}
    $(\mu,\sigma^2)$に従う2乗可積分な独立同分布列$\{Y_i\}\subset L^2(\Om,\F,P)$について,
    特性関数がある$\nu\ge1$について$\nu$乗可積分$\int_{\C}\abs{\varphi(t)}^\nu dt<\infty$ならば,和$S_n:=\sum^n_{i=1}Y_i$は$N(n\mu,\sqrt{n}\sigma^2)$にノルム収束する:
    \[\sup_{B\in\B(\R)}\Abs{P[S_n\in B]-\int_B\frac{1}{\sqrt{2\pi n\sigma^2}}\exp\paren{-\frac{1}{2}\frac{(x-n\mu)^2}{n\sigma^2}}dx}\to0.\]
\end{theorem}

\subsection{経験分布に対する拡張}

\begin{theorem}[Glivenko-Cantelli]
    $F_n$を経験分布関数,$F$を真の分布関数とする.経験分布関数は,分布関数の一致推定量である:$P[\lim_{n\to\infty}\sup_{x\in\R}\abs{F_n(x)-F(x)}=0]=1$.
\end{theorem}

\begin{discussion}
    $F_n$の推定量としての誤差分布を考えたい.各点$x\in\R$毎に見ると,$\sqrt{n}(F_n(x)-F(x))$は,$\X^n$上に定まり,$\R$上に値を取る確率変数で,$N(0,F(x)(1-F(x)))$に分布収束する.
    結合分布は多変量正規分布に収束する.では,$\sqrt{n}(F_n-F)$自体はどうか?これは,標本が与えられる毎に分布関数を確定させる$\X^n\to l^\infty(\R)$なる確率変数で,$F$-ブラウニアン橋と呼ばれる関数$\bG_F:\X^\infty\to l^\infty(\R)$が定める分布に弱収束する.
    Banach空間値確率変数の共分散は定義していないが(いわば無限の成分を持つ行列$\R^2\to\R$),各組$(x_k,x_l)\in\R^2$に対して$\Cov[\bG_F(x_k),\bG_F(x_l)]=F(x_k\land x_l)-F(x_k)F(x_l)$が成り立つ.
\end{discussion}

\section{中心極限定理の誤差}

\begin{tcolorbox}[colframe=ForestGreen, colback=ForestGreen!10!white,breakable,colbacktitle=ForestGreen!40!white,coltitle=black,fonttitle=\bfseries\sffamily,
title=]
    極限分布として正規分布を持つ確率変数を,その極限分布によって近似したときの近似誤差は,$1/\sqrt{N}$のオーダーを持つ.
    これはあまりにも遅い,収束が線型よりも遅い.
    そのために,別の方法で目標の分布が「裾が軽い」ことを示す必要が応用上出てくる,これが集中不等式である.
\end{tcolorbox}

\begin{theorem}[Berry-Esseen central limit theorem]\label{thm-Berry-Esseen}
    独立同分布を持つ確率変数列$\{X_n\}\subset\L^1(\Om,\F,P)$は$E[X_n]=\mu\in\R,\Var[X_n]=\sigma^2\in\R_+$を満たすとする.
    $S_N:=X_1+\cdots+X_N$について,
    \[Z_N:=\frac{S_N+E[S_N]}{\sqrt{\Var[S_N]}}=\frac{1}{\sigma\sqrt{N}}\sum^N_{i=1}(X_i-\mu)\xrightarrow{d}N(0,1)\]
    であるが,$\rho:=\frac{E[\abs{X_1-\mu}^3]}{\sigma^3}$と$g\sim N(0,1)$について,
    \[\forall_{N\in\N}\;\forall_{t\in\R}\quad\Abs{P[Z_N\ge t]-P[g\ge t]}\le\frac{\rho}{\sqrt{N}}\]
\end{theorem}

\section{Poissonの少数の法則}

\begin{notation}
    これより,単位時間内に電話がかかってくる回数や事故が起こる回数などは,Poisson分布によって近似するのが筋が良いことがわかる.
\end{notation}

\begin{notation}
    点列$\{p_n\}\subset(0,1)$は$\lim_{n\to\infty}np_n=\lambda>0$を満たすとし,
    これが定める$n$回の独立同試行$(S^n:=2^n,\F,P),(\pr_n)_*P(\{1\})=p_n$の列を考える.
    これは,$n$が増加するにつれて,$1$が出る確率は小さくなっていくが,$n$回が終わってみたときに$1$が出る回数の平均は$\lambda$で変わらないようになっている.
\end{notation}

\begin{theorem}
    $Z_n:S^n\to\Z$を$S^n$上の1-ノルムとする($1$が出る回数).
    このとき,$Z_n$は$\Pois(\lambda)$に法則収束する:$\forall_{l\in\N}\;\lim_{n\to\infty}P(Z_n=l)=e^{-\lambda}\frac{\lambda^l}{l!}$.
\end{theorem}

\section{大偏差原理}

\begin{tcolorbox}[colframe=ForestGreen, colback=ForestGreen!10!white,breakable,colbacktitle=ForestGreen!40!white,coltitle=black,fonttitle=\bfseries\sffamily,
    title=無限次元空間におけるLaplace原理]
    分布を近似するにあたって,偏差が大きい部分の挙動を捉える.
    大数の法則から漏れた部分$a$(=偏差$\abs{a-m}$の大きいもの)の確率は$0$に収束し$P(X_n=a)\xrightarrow{n\to\infty}0$,
    中心極限定理により指数減衰$P(X_n\ge a)=\int_{z\ge a}p_n^{X_n}(z)e^{-nI(z)}dz=e^{-I(a)}$(第2項あやしい)をするのであるが,
    そのときの係数$I(z)$は,
    「大数の法則が指定する集中点に一番近い事象」すなわち「起こりにくい事象の中で最も起こりやすい事象」が支配するという原理である.
    これはLaplaceの原理の無限次元版だとみなすと筋が良い.
\end{tcolorbox}

\begin{history}\mbox{}
    \begin{enumerate}
        \item 1929にKhinchinがBernoulli列について扱う.
        \item 1938にCramerが$\exists_{t>0}\;E[e^{t\abs{X_1}}]<\infty$の条件の下で一般化.
        \item 1966にSchilderが確率過程=汎関数型の大偏差原理を定立.
        \item 1970sにDonsker-Varadhanの理論が生まれる.
    \end{enumerate}
\end{history}

\subsection{定義}

\begin{notation}
    $X$を可分完備距離空間とし,同時にある$\sigma$-代数$\F$により可測空間でもあるとする.
\end{notation}

\begin{definition}[large deviations technique]
    $(X,\F)$上の確率測度の列$(\mu_n)_{n\in\N}$が大偏差原理をみたすとは,ある下半連続関数$I:X\to\R_+$が存在して,任意の可測集合$\Gamma\in\F$に対して
    \[-\inf_{x\in\Gamma^\circ}I(x)\le\liminf_{n\to\infty}\frac{1}{n}\log\mu_n(\Gamma)\le\limsup_{n\to\infty}\frac{1}{n}\log\mu_n(\Gamma)\le-\inf_{x\in\o{\Gamma}}I(x)\]
    が成り立つことをいう.
    このとき,$I$を\textbf{rate関数}という.
\end{definition}

\begin{definition}[good rate function]
    $I$が\textbf{良いレート関数}であるとは,任意の$l\ge0$に対して,等位集合$I^{-1}((-\infty,l])=\Brace{x\in X\mid I(x)\le l}$がコンパクトであることをいう.
\end{definition}

\subsection{Laplaceの原理}

\begin{tcolorbox}[colframe=ForestGreen, colback=ForestGreen!10!white,breakable,colbacktitle=ForestGreen!40!white,coltitle=black,fonttitle=\bfseries\sffamily,
title=]
    「大きなパラメータを持つ指数関数の積分の漸近挙動は,被積分関数の最大値付近からの寄与だけで決まる」という経験則である.
\end{tcolorbox}

\begin{example}
    有界閉区間$[a,b]$上の連続関数$f,g>0$について,
    \[\int^b_ae^{nf(x)}g(x)dx\approx e^{n\max_{x\in[a,b]}f(x)}\quad(n\to\infty)\]
    が成り立つ.ただし,$F(n)\approx G(n):\Leftrightarrow\frac{\log F(n)}{\log G(n)}\xrightarrow{n\to\infty}1$と定めた.
\end{example}
\begin{remark}
    この結果を非有界な区間に一般化しようとすると,種々の技術的問題が生じる.
    が,同様の結果は成り立つことが多い.そこで「原理」と呼ばれている.
\end{remark}

\begin{example}
    (対数の比を考えることで)定数倍を無視した弱い形のStirlingの公式$n!=\int^\infty_0x^ne^{-x}dx\approx n^ne^{-n}\;(n\to\infty)$をLaplaceの原理から説明する.積分変換によって
    \[n^n\int^\infty_0y^ne^{-ny}dy=n^n\int^\infty_0ne^{n\log y-y}dy\]
    と書き直せる.すると,$\max_{y\in\R_+}(\log y-y)=-1$と$\log y-y\xrightarrow{y\to 0}-\infty,\log y-y\xrightarrow{\infty}-\infty$より,最大値$e^{-n}$だけが積分に寄与することが予想される.
\end{example}

\subsection{Cramerの理論}

\begin{tcolorbox}[colframe=ForestGreen, colback=ForestGreen!10!white,breakable,colbacktitle=ForestGreen!40!white,coltitle=black,fonttitle=\bfseries\sffamily,
title=]
    $m:=E[X_1]<\infty$に対して,$A\in\F$が$d(m,A)>0$を満たせば,$\lim_{n\to\infty}P(Y_n\in A)=0$であるが,このときの収束の速さは指数的に減衰する.
\end{tcolorbox}

\begin{theorem}
    独立同分布を持つ確率変数列$(X_i)$の積率母関数の定義域$D_\nu:=\Brace{t\in\R\mid M(t):=E[e^{tX_1}]<\infty}$は,$0$を内点として持つとする.
    このとき,$S_n:=\sum_{i=1}^nX_i$とおく.
    \begin{enumerate}
        \item $\forall_{a>E[X_1]}\;\lim_{n\to\infty}\frac{1}{n}\log P(S_n\ge an)=-I(a)$.ただし,$I(z):=\sup_{t\in\R}(zt-\log M(t))$をキュムラント母関数$\log M$のLegendre変換とした.
        \item $I$は$\R$上下半連続な凸関数であり,$\lim_{z\to\pm\infty}I(z)=\infty,\forall_{z\in\R}\;I(z)\ge 0=I(E[X_1])$を満たす.
    \end{enumerate}
\end{theorem}
\begin{remarks}
    $E[X_1]<a$について
    $A:=[a,\infty)$とおくと,
    \[\lim_{n\to\infty}\frac{1}{n}\log P\paren{\frac{1}{n}S_n\in A}=-\inf_{z\in A}I(z)\]
    と書き換えられる.すなわち,事象$\frac{S_n}{n}\in[a,\infty)$が成り立つとき,$n\to\infty$の目で見ると,殆ど$a$の近くの値を取るような事象によって実現されている」ということである.

\end{remarks}

\subsection{Schilderの理論}

\begin{tcolorbox}[colframe=ForestGreen, colback=ForestGreen!10!white,breakable,colbacktitle=ForestGreen!40!white,coltitle=black,fonttitle=\bfseries\sffamily,
title=確率過程に関する大偏差原理]
    $I(\phi)$を,連続関数$\phi$の持つエネルギーだとすると,「起こりにくい事象の中では最も起こりやすい事象=エネルギーが最小になる事象が起こる」という大偏差原理は,物理現象としても極めて自然な現象であることがわかる.
\end{tcolorbox}

\subsection{Varadhanの理論}

\begin{tcolorbox}[colframe=ForestGreen, colback=ForestGreen!10!white,breakable,colbacktitle=ForestGreen!40!white,coltitle=black,fonttitle=\bfseries\sffamily,
title=]
    大偏差原理は,Laplaceの原理が有界線型汎関数の列に対しても成り立つための十分条件を与えると捉えると,その指数関数への注目と,位相のことばを用いた定義が自然に思える.
\end{tcolorbox}

\begin{lemma}
    $(X,\F)$上の確率測度の列$(\mu_n)$が,$I$を良いレート関数として大偏差原理を満たすとする.
    このとき,$X$上の有界連続関数$f\in C_b(X)$について,
    \[\int e^{nf(x)}\mu_n(dx)\approx\exp\paren{n\sup_{x\in X}\abs{f(x)-I(x)}}\quad(n\to\infty)\]
\end{lemma}

\chapter{漸近理論}

\begin{quotation}
    Kolmogorov\cite{Kolmogorov}
\end{quotation}

\section{歴史}

\begin{tcolorbox}[colframe=ForestGreen, colback=ForestGreen!10!white,breakable,colbacktitle=ForestGreen!40!white,coltitle=black,fonttitle=\bfseries\sffamily,
title=]
    物理学者も解析学者も参入した対象であり,中心極限定理とはやはり夢のテーマであった.
    ロシアが強いことに疑問もない.
    正規分布には,算術平均が最尤推定量になるということと,微小分布へ無限分解可能な分布としての2つの普遍性がある.
\end{tcolorbox}

\begin{history}[Stirlingの公式とde Moivre-Laplaceの定理]
    FermatとPascalの往復書簡(1654)ののちに,HuygensがDe  Ratiociniis in Ludo Aleae (1657)を書き,
    これを
    James Bernoulliが,まとめてArs Conjectandi (1713)が死後に出版された.
    この時点で第四部では推論に応用することを考えたのである.
    復元抽出の標本数$n$をどこまで大きくすれば,確率$p$を一定の誤差以下で評価できるかを考えた.
    これは多項分布の問題であるから,二項分布やBernoulli数の技法が第二部でまとめられている.
    この問題を乗り越えたのがDe Moivre (1733)で,多項分布の確率が正規曲線$e^{-x^2}$の積分で近似できることを発見した.
    この自分の論文を英訳し,The Doctrine of Chances, 2nd Ed. (1738)に収録して,友人に配った.
    さらに$n!$の近似の問題を解いて精緻化したのがStirlingである(De Moivreの友人).
    この点で本質的に解析学化したのがde Moivreであったが,本格的に微積分学の言葉で書き直したのがLaplaceであった.
    こうして,de Moivre-Laplaceの定理が示された.
    解析学とは極限の技法で,極限とは近似の基礎である.
    二項分布とは,二点分布に従う独立同分布列の和の分布でもあることに注意すれば,これは中心極限定理の系である.
\end{history}

\begin{history}[誤差分布と天文学]
    Simpson (1775)のLondon王立学会に当てた手紙に,天文学者は算術平均を推定量として使うが,注意深く測定した1回の方が信頼できるという意見の知識人も多いという現状を書いている.
    そこでSimpson, Lagrangeらは誤差分布として考え得る分布たち(裾が軽く,対称な分布)の算術平均の分布が調べられた.

    回帰分析だが,方程式の形をパラメトリックに仮定し,観測値を集めると,解析的には解無しになるが,実際は仮定も正しくなく,観測誤差もあるので,損失関数を最小にする係数を推定量とすることになる(決定問題).
    Gaussが最終的にこれを解決し,標本平均が位置母数の最尤推定量になるのは誤差法則が正規分布の場合に限ることを解析的に示し,このとき最小二乗法が必然的に最良になることを示した.
    算術平均を信じるなら,誤差分布は正規性を仮定することになるのだ.
\end{history}

\begin{history}[誤差とは何か]
    Youngは誤差という概念の安定性に疑問を持ち,「誤差とは,根元誤差の独立和に分解できるのではないか」「そして独立和は,元々の誤差分布に依らずに特定の分布に従うのでは?」と,裏の数理構造に目を向けた.
    これは数学的には新規性はないが,無限可分分布という概念に目を向けさせる.
    こうしてGuassとは独立に,もう一度誤差分布に正規分布を用いる妥当性が導かれる.
    つまり,3次以上の積率が無視できるほど小さいような根元誤差の話の分布は,正規分布に従うことになる.
    「微小分布の和」としての普遍性があるわけだ.
\end{history}

\begin{history}
    Chebyshev, Markovの順に十分条件が厳密に見つかっていき,
    Lyapunovが特性関数の方法で数学的に正しいものをひとまず1つ確定させた.
    Lindeberg, Levy, Feller, Kolmogorovが引き継ぎ,現代論になる.
\end{history}

\chapter{高次元確率論}

\begin{quotation}
    ここではこれ以上具体論に深入りせず,高次元確率論と確率過程論を深掘りする.
    これらはいずれも$\Om\to\Meas(\R_+,\R)$とみると,Banach空間値確率変数として統一的に見れる.
    また,平均や分散などの母数もBanach空間の点となる.実際,確率分布族をBanach空間で添字付ける見方は有効な手法である.
\end{quotation}

\section{独立確率変数列の集中不等式}

\begin{tcolorbox}[colframe=ForestGreen, colback=ForestGreen!10!white,breakable,colbacktitle=ForestGreen!40!white,coltitle=black,fonttitle=\bfseries\sffamily,
title=]
    集中不等式は,確率分布の偏差$\abs{X-\mu}$を捉える.
    極限分布がGaussであるとき,これを用いて近似することがまず考えられるが,近似誤差が大きい\ref{thm-Berry-Esseen}.
    そこで,極限定理に依らずに,「裾が軽い」ことを示す方法論が必要となる.
    その一つが集中不等式である.
    (極限定理では特性関数が活躍したが,)ここでは,積率母関数に注目することとなる.
\end{tcolorbox}

\subsection{Chebyshevの不等式}

\begin{tcolorbox}[colframe=ForestGreen, colback=ForestGreen!10!white,breakable,colbacktitle=ForestGreen!40!white,coltitle=black,fonttitle=\bfseries\sffamily,
title=]
    最も即時的で古典的な上界を与え,多くの場合線型に過ぎない.
\end{tcolorbox}

\begin{theorem}
    $X\sim(\mu,\sigma^2)$を実確率変数とする.
    \begin{enumerate}
        \item (Markov) $X\ge0$ならば,$\forall_{t>0}\;P[X\ge t]\le\frac{E[X]}{t}$.
        \item (Chebyshev) $\forall_{t>0}\;P[\abs{X-\mu}\ge t]\le\frac{\sigma^2}{t^2}$.
    \end{enumerate}
\end{theorem}

\subsection{Hoeffdingの不等式}

\begin{tcolorbox}[colframe=ForestGreen, colback=ForestGreen!10!white,breakable,colbacktitle=ForestGreen!40!white,coltitle=black,fonttitle=\bfseries\sffamily,
title=]
    Rademacher確率変数の線型和の尾部について,Gauss分布と全く同様な収束レート$e^{-t^2/2}$による評価を与えるのがHoeffdingの不等式である.
    これは理想的な形で,中心極限定理の代替となっている.
    証明はモーメント母関数により,これは一般の有界な確率変数に一般化出来るが,鋭さは落ちる.
\end{tcolorbox}

\begin{theorem}
    $X_1,\cdots,X_N$を独立なRademacher確率変数とし,$a=(a_1,\cdots,a_N)\in\R^N$とする.
    このとき,
    \[\forall_{t\in\R_+}\quad P\Square{\sum^N_{i=1}a_iX_i\ge t}\le\exp\paren{-\frac{t^2}{2\norm{a}^2_2}}.\]
\end{theorem}
\begin{corollary}
    \[P\Square{\Abs{\sum^N_{i=1}a_iX_i}\ge t}\le 2\exp\paren{-\frac{t^2}{2\norm{a}^2_2}}.\]
\end{corollary}

\begin{theorem}[Hoeffding's inequality for general bounded random variable]
    $X_1,\cdots,X_N$を独立な有界確率変数とする:$\forall_{i\in[N]}\;\exists_{m_i,M_i\in\R}\;\Im X_i\subset[m_i,X_i]$.
    このとき,
    \[\forall_{t>0}\quad P\Square{\sum^N_{i=1}(X_i-E[X_i])\ge t}\le\exp\paren{-\frac{2t^2}{\sum_{i=1}^N(M_i-m_i)^2}}\]
\end{theorem}

\begin{application}[平均の頑健推定]
    独立な観測$X_1,\cdots,X_N\sim(\mu,\sigma^2)$から平均$\mu$を,誤差$\ep$以内で推定したい.
    \begin{enumerate}
        \item $N=O(\sigma^2/\ep^2)$個の標本が存在すれば,確率$3/4$以上で$(\mu-\ep,\mu+\ep)$に入る推定量が構成できる.
        \item 任意の$\delta\in(0,1)$について,$N=O(\log(\delta^{-1})\sigma^2/\ep^2)$個の標本が存在すれば,確率$1-\delta$以上で$(\mu-\ep,\mu+\ep)$に入る推定量が構成できる.
    \end{enumerate}
\end{application}

\subsection{Chernoffの不等式}

\begin{tcolorbox}[colframe=ForestGreen, colback=ForestGreen!10!white,breakable,colbacktitle=ForestGreen!40!white,coltitle=black,fonttitle=\bfseries\sffamily,
title=]
    Bernoulli変数について,母数$p_i$が小さすぎるとき(Poissonの少数の法則的な現象のとき),Hoeffdingの不等式は見当違いの結果を与える.
\end{tcolorbox}

\begin{theorem}
    $X_i\sim B(p_i)$をBernoulli確率変数とする.和$S_N:=\sum_{i=1}^NX_i$の平均を$\mu:=E[S_N]$で表すと,
    \[\forall_{t>\mu}\quad P[S_N\ge t]\le e^{-\mu}\paren{\frac{e\mu}{t}}^t.\]
\end{theorem}
\begin{remarks}
    これはPoisson分布的な結果である.実際$X\sim\Pois(\lambda)$ならば,
    \[\forall_{t>\lambda}\quad P[X\ge t]\le e^{-\lambda}\paren{\frac{e\lambda}{t}}^t.\]
    これを,各第$i$回での成功確率$p_i$がバラバラな場合でも行っている.
\end{remarks}

\begin{remark}[Poisson尾部の観察]
    $k!$をStirlingの公式で近似すると
    \[P[X=k]\sim\frac{1}{\sqrt{2\pi k}}e^{-\lambda}\paren{\frac{e\lambda}{k}}^k\]
    を得るから,Poisson分布の尾部と言っても,一番小さい部分が殆どすべてを占めてしまう.
    大偏差原理的な現象である.
\end{remark}

\begin{theorem}[小偏差の場合]
    $X_i\sim B(p_i)$をBernoulli確率変数とする.和$S_N:=\sum_{i=1}^NX_i$の平均を$\mu:=E[S_N]$で表すと,
    \[\exists_{c>0}\;\forall_{\delta\in(0,1]}\quad P[\abs{S_N-\mu}\ge\delta\mu]\le 2e^{-c\mu\delta^2}\]
\end{theorem}
\begin{remarks}
    これはPoisson分布の
    \[\forall_{t\in(0,\lambda]}\quad P[\abs{X-\lambda}\ge t]\le 2\exp\paren{-\frac{ct^2}{\lambda}}\]
    に対応する.$\Pois(\lambda)$は平均$\lambda$近くでは$N(\lambda,\lambda)$にすごく似ているが,大偏差部分では裾は重い:$(\lambda/t)^t$.
\end{remarks}

\subsection{確率グラフ}

\begin{tcolorbox}[colframe=ForestGreen, colback=ForestGreen!10!white,breakable,colbacktitle=ForestGreen!40!white,coltitle=black,fonttitle=\bfseries\sffamily,
title=]
    ネットワークの確率モデルとなる.
    $p$が十分大きく,平均して$O(\log n)$の次数を持つとき,どの頂点の次数も$[0.9d,1.1d]$の間にあって,他と結ばれ過ぎていたり,孤立していたりすることは殆どない.
\end{tcolorbox}

\begin{definition}
    Erdos-Renyi model $G(n,p)\;(n\in\N,p\in[0,1])$とは,$n$個の頂点と,その任意の2頂点の間を確率$p$で辺が存在するグラフである.
\end{definition}

\begin{lemma}
    任意の頂点の次数の期待値は$d:=(n-1)p$となる.
\end{lemma}

\begin{theorem}[dense graphs are almost regular]
    ある$C>0$が存在して,任意の$d\ge C\log n$を満たすランダムグラフ$G\sim G(n,p)$は,十分大きな確率(0.9以上)で,任意の頂点の次数が$[0.9d,1.1d]$の間に入る.
\end{theorem}

\subsection{劣Gauss分布}

\begin{tcolorbox}[colframe=ForestGreen, colback=ForestGreen!10!white,breakable,colbacktitle=ForestGreen!40!white,coltitle=black,fonttitle=\bfseries\sffamily,
title=]
    Bernoulli確率変数$X_i$について集中不等式を示したが,どこまで一般の確率変数に適用できるのか?
\end{tcolorbox}

\begin{proposition}
    確率変数$X$について,次の4条件は同値で,$E[X]=0$ならば(5)も同値.
    また,ある定数$C>0$が存在して,パラメータ$K_i>0$はそれぞれ互いの$C$倍を超えない.
    \begin{enumerate}
        \item $X$の尾部は$\forall_{t\in\R_+}\;P[\abs{X}\ge t]\le 2\exp\paren{-\frac{t^2}{K_1^2}}$を満たす.
        \item $X$のモーメントは$\forall_{p\ge 1}\;\norm{X}_{L^p}\le K_2\sqrt{p}$を満たす.
        \item $X^2$の積率母関数は$\forall_{\abs{\lambda}\le1/K_3}\;E[e^{\lambda^2X^2}]\le\exp(K_3^2\lambda^2)$を満たす.
        \item $X^2$の積率母関数は$E[e^{X^2/K_4^2}]\le 2$.
        \item $X$の積率母関数は$\forall_{\lambda\in\R}\;E[e^{\lambda X}]\le e^{K^2_5\lambda^2}$.
    \end{enumerate}
\end{proposition}

\begin{definition}[sub-gaussian norm]
    命題の条件を満たす確率変数$X$を\textbf{劣ガウス}であるといい,劣ガウスノルムを(4)を満たす最小の定数$K_4$,すなわち
    \[\norm{X}_{\psi_2}=\inf\Brace{t>0\mid E[e^{X^2/t^2}]\le 2}\]
    で定める.
\end{definition}

\begin{example}\mbox{}
    \begin{enumerate}
        \item $X\sim N(0,\sigma^2)$ならば,$\norm{X}_{\psi_2}\le C\sigma$.
        \item $X$がRademacherならば,$\norm{X}_{\psi_2}\le\frac{1}{\sqrt{\log 2}}=:C$.
        \item $X$が有界ならば,$\norm{X}_{\psi_2}\le C\norm{X}_\infty$.
    \end{enumerate}
    一方で,Poisson分布,指数分布,Pareto分布,もちろんCauchy分布は裾が重く,劣Gaussではない.
\end{example}

\subsection{Orlicz空間}

\begin{tcolorbox}[colframe=ForestGreen, colback=ForestGreen!10!white,breakable,colbacktitle=ForestGreen!40!white,coltitle=black,fonttitle=\bfseries\sffamily,
title=]
    劣Gauss分布をさらに一般化すると,Orlicz空間にいきつく.
\end{tcolorbox}

\subsection{一般化Hoeffding不等式}

\begin{tcolorbox}[colframe=ForestGreen, colback=ForestGreen!10!white,breakable,colbacktitle=ForestGreen!40!white,coltitle=black,fonttitle=\bfseries\sffamily,
title=]
    劣Gaussノルムの言葉を用いて,Hoeffding不等式は一般の劣Gauss確率変数に一般化出来る.
\end{tcolorbox}

\subsection{劣指数分布}

\begin{lemma}
    $X$を確率変数とする.次の2条件は同値:
    \begin{enumerate}
        \item $X$は劣Gaussである.
        \item $X^2$は劣指数である.
    \end{enumerate}
    このとき,$\norm{X^2}_{\psi_1}=\norm{X}^2_{\psi_2}$.
\end{lemma}

\subsection{Bernsteinの不等式}

\begin{tcolorbox}[colframe=ForestGreen, colback=ForestGreen!10!white,breakable,colbacktitle=ForestGreen!40!white,coltitle=black,fonttitle=\bfseries\sffamily,
title=]
    劣指数分布
\end{tcolorbox}

\begin{theorem}
    $X_1,\cdots,X_N$を平均$0$の劣指数確率変数とする.このとき,
    \[\exists_{c>0}\;\forall_{t\in\R_+}\quad P\Square{\Abs{\sum^N_{i=1}X_i}\ge t}\le 2\exp\paren{-c\min\paren{\frac{t^2}{\sum^N_{i=1}\norm{X_i}^2_{\psi_1}},\frac{t}{\max_{i\in[N]}\norm{X_i}_{\psi_1}}}}.\]
\end{theorem}





\chapter{分布の扱い}

\begin{quotation}
    
\end{quotation}

\section{期待値作用素}

\begin{tcolorbox}[colframe=ForestGreen, colback=ForestGreen!10!white,breakable,colbacktitle=ForestGreen!40!white,coltitle=black,fonttitle=\bfseries\sffamily,
title=]
    一般の集合上の関数の期待値作用素の定義には,Lebesgue積分を用いる.
    経験分布論では,さらに拡張された線型作用素を用いることも考える.
\end{tcolorbox}

\subsection{積分の定義}

\begin{proposition}
    $X$を確率変数とする.$0<p<q$について,$\abs{X}^q$が可積分ならば,$\abs{X}^p$も可積分である.
\end{proposition}
\begin{proof}
    $\forall_{x\in\R}\;\abs{x}^p\le 1+\abs{x}^q$である.これと,Lebesgueの優収束定理より.
\end{proof}

\begin{proposition}[Markovの不等式]
    非負可測関数$X$と任意の$p>0,q\ge 0,\ep>0$について,
    \[\int X^q1_{\Brace{X\ge\ep}}d\mu\le\ep^{-p}\int 1_{\Brace{X\ge\ep}}X^{p+q}d\mu\le\ep^{-p}\int X^{p+q}d\mu.\]
\end{proposition}
\begin{proof}
    $\ep^p1_{\Brace{X\ge\ep}}X^q\le 1_{\Brace{X\ge\ep}}X^{p+q}\le X^{p+q}$と,積分の単調性より.
\end{proof}

\subsection{期待値の定義}

\begin{definition}[expectation / expected value / mean (value)]
    確率変数$X:\Om\to\R$が確率測度$P$について可積分のとき,$E[X]:=\int_\Om X(\om)P(d\om)$を$X$の\textbf{期待値}または\textbf{平均(値)}という.$E_P[X]$とも表す.
\end{definition}

\begin{definition}[$r$-th moment, $r$-th central moment, $r$-th absolute moment]
    確率変数$X:\Om\to\R$について,
    \begin{enumerate}
        \item $\al_r:=\mu'_r=E[X^r]$を$X$の$r$次の\textbf{積率}と呼ぶ.
        \item $\mu_r:=E[(X-E[X])^r]$を$X$の$r$次の\textbf{中心積率}と呼ぶ.
        \item $\beta_r:=E[\abs{X}^r]$を$X$の$r$次の\textbf{絶対積率}と呼ぶ.
        \item $\mu_2$を$X$の\textbf{分散}とよび,$\Var[X]$で表す.
        \item $\sqrt{\Var[X]}$を\textbf{標準偏差}と呼ぶ.
    \end{enumerate}
\end{definition}

\begin{proposition}[分散の性質]
    2乗可積分な実確率変数$X:\Om\to\R$について,
    \begin{enumerate}
        \item (分散公式) $\Var[X]=E[(X-E[X])^2]=E[X^2]-(E[X])^2$.
        \item (2次斉次性) $\forall_{a,b\in\R}\;\Var[aX+b]=a^2\Var[X]$.
    \end{enumerate}
\end{proposition}

\subsection{期待値不等式}

\begin{notation}
    積分論の記号を,期待値に関しても流用する.
    \begin{enumerate}
        \item $p\in(0,\infty)$について,$\norm{X}_p=(E[\abs{X}^p])^{1/p}$と定める.
        \item $\norm{X}_\infty:=\esssup\abs{X}$とする.
    \end{enumerate}
\end{notation}

\begin{theorem}[測度論における結果]
    $X,Y$を確率変数とする.
    \begin{enumerate}
        \item (Hölder) $\forall_{p,q\in(1,\infty)}\;\frac{1}{p}+\frac{1}{q}=\Rightarrow\norm{XY}_1\le\norm{X}_p\norm{Y}_q$.特に,$p=q=2$のときSchwarzの不等式.
        \item (Minkowski) $\forall_{p\in[1,\infty]}\;\norm{X+Y}_p\le\norm{X}_p+\norm{Y}_p$.
        \item $\forall_{p,q\in(0,\infty]}\;p<q\Rightarrow\norm{X}_p\le\norm{X}_q$.
        \item (Markov) 任意の非減少可測関数$\varphi:\R_+\to\R_+$について,$\varphi(A)>0\Rightarrow E[\abs{X}1_{\Brace{\abs{Y}\ge A}}]\le\frac{E[\abs{X}\varphi(\abs{Y})1_{\Brace{\abs{Y}\ge A}}]}{\varphi(A)}\le\frac{E[\abs{X}\varphi(\abs{Y})]}{\varphi(A)}$.
        
        特に,$P[\abs{X-E[X]}\ge A]\le\frac{\Var[X]}{A^2}$ (Chebyshev).
        \item (Jensen) 任意の開区間上の凸関数$\psi:I\to\R$について,$X,\psi(X)$が可積分かつ$P[X\in I]=1\Rightarrow\psi(E[X])\le E[\psi(X)]$.
    \end{enumerate}
\end{theorem}

\section{分布の特性値}

\begin{tcolorbox}[colframe=ForestGreen, colback=ForestGreen!10!white,breakable,colbacktitle=ForestGreen!40!white,coltitle=black,fonttitle=\bfseries\sffamily,
title=測度論による確率の特徴量の理解]
    $(\R,\B_1)$上の分布について改めて特性値を定義すると,これは「確率変数の特性値」の一般化となっている.
    確率変数の期待値は,それが誘導する分布の平均のことである.

    積分は可測関数と測度についての2変数関数とするならば,これは後者を引数とするとより一般的になるという不思議な状況を物語ってはいないか?
    これが期待値作用素の限界ということか?
\end{tcolorbox}

\subsection{確率密度関数}

\begin{definition}[probability density function]
    $\R$上の分布$\nu$がLebesgue測度$dx$に関して絶対連続であるとき\footnote{Lebesgue零集合上の確率が零であることが定義.},そのRadon-Nikodym微分$f$を\textbf{確率密度関数}とよび,$\nu$を\textbf{(絶対)連続分布}という.
\end{definition}
\begin{remark}
    確率密度関数はLebesgue零集合の差を除いて一意に定まる.
\end{remark}

\begin{lemma}
    任意の可測関数$g:\R\to\R$と絶対連続分布$\nu$について,
    \[E[g(X)]=\int_\R g(x)\nu(dx)=\int_\R g(x)f(x)dx.\]
    すなわち,左辺または右辺が存在すればもう一方も存在し,値が一致する.
\end{lemma}

\subsection{平均値と積率}

\begin{definition}[expectation]
    $(\R,\B_1)$上の確率測度$\nu$について,$\mu:=\int_\R x\nu(dx)$を,$\nu$の\textbf{期待値}または\textbf{平均(値)}という.
\end{definition}

\begin{definition}
    $(\R,\B_1)$上の確率測度$\nu$について,
    \begin{enumerate}
        \item $\al_r:=\mu'_r=\int_\R x^r\nu(dx)$を$\nu$の$r$次の\textbf{積率}と呼ぶ.
        \item $\mu_r:=\int_\R(x-\mu)-r\nu(dx)$を$\nu$の$r$次の\textbf{中心積率}と呼ぶ.
        \item $\beta_r:=\int_\R\abs{x}^r\nu(dx)$を$\nu$の$r$次の\textbf{絶対積率}と呼ぶ.
        \item $\mu_2$を$\nu$の\textbf{分散}と呼ぶ.
        \item $\sqrt{\mu_2}$を\textbf{標準偏差}と呼ぶ.
    \end{enumerate}
\end{definition}

\begin{proposition}[変数変換公式:well-definedness]
    $(\X,\A)$を可測空間とし,$X$を$\X$-値確率変数とする.このとき,
    \[\forall_{g\in\Meas(\X,\R)}\quad\int_\Om g(X(\om))P(d\om)=\int_\X g(x)P^X(dx).\]
    すなわち,左辺または右辺のいずれかの積分が存在すればもう一方も存在し,値が等しくなる.
\end{proposition}

\begin{corollary}[積率のwell-definedness]
    $g(x)=x^r$とすれば,
    \[E[X^r]=\int_\Om X(\om)^rP(d\om)=\int_\R x^r P^X(dx).\]
    すなわち,$\alpha_r(X)=\al_r(P^X)$.
\end{corollary}

\begin{definition}[skewness, kurtosis]
    位置母数でも尺度母数でもないものとして,分布の形状を表すと考えられる次の母数がある.
    \begin{enumerate}
        \item $\fraks=\gamma_1:=\frac{\mu_3}{\mu_2^{3/2}}$を\textbf{歪度}と呼ぶ.
        \item $\frakk=\gamma_2:=\frac{\mu_4}{\mu_2^2}$を\textbf{尖度}と呼ぶ.
    \end{enumerate}
\end{definition}

\subsection{共分散と相関}

\begin{definition}[covariance, correlation coefficient]\mbox{}
    \begin{enumerate}
        \item $X,Y,XY$が$P$-可積分のとき,
        \[\Cov[X,Y]=E[(X-E[X])(Y-E[Y])]\]
        を$X,Y$の共分散という.
        \item $X,Y\in L^2$で$\Var[X]\Var[Y]\ne 0$のとき,$\rho(X,Y)=\frac{\Cov[X,Y]}{\sqrt{\Var[X]}\sqrt{\Var[Y]}}$を\textbf{相関係数}という.
        \item $X=(X_i)_{i\in[r]},Y=(Y_j)_{j\in[c]}$が$r,c$次元の2乗可積分確率変数とするとき,\textbf{共分散行列}は
        \[\Cov[X,Y]=E[(X-E[X])(Y-E[Y])^\perp]=(E[(X_i-E[X_i])(Y_j-E[Y_j])])_{(i,j)\in[r]\times[c]}=(\Cov[X_i,Y_j])_{ij}\]
        で定まる$r\times c$行列をいう.
        $\Cov[X,Y]=\Cov[Y,X]^\perp$が成り立つ.
        \item $\Var[X]:=\Cov[X,X]$で定まる$r\times r$行列を,\textbf{分散共分散行列}という.
        \item $\Corr[X]=(\rho(X_i,X_j))_{i,j\in[n]}$で定まる$r\times r$行列を\textbf{相関行列}という.これは,対角要素が1に基準化された無次元量だと考えられる.
    \end{enumerate}
\end{definition}
\begin{remarks}
    $X,Y\in L^2$は十分条件である.
\end{remarks}

\begin{proposition}\label{prop-1d-covariance}
    $X,Y,Z\in L^2$について,
    \begin{enumerate}
        \item $\Cov[X,Y]=\Cov[Y,X]$.
        \item $\Cov[aX+bY,Z]=a\Cov[X,Z]+b\Cov[Y,Z]$.
        \item $\Cov[X,1]=0$.特に,$\Cov[aX+b,Y]=a\Cov[X,Y]$.
        \item (共分散公式) $\Cov[X,Y]=E[XY]-E[X]E[Y]$.
        \item $\Cov[X,X]=\Var[X]\ge 0$.等号成立条件は$X=E[X]\;\as$
        \item (Schwarzの不等式) $\abs{\Cov[X,Y]}\le\sqrt{\Var[X]}\sqrt{\Var[Y]}$.
    \end{enumerate}
\end{proposition}

\begin{proposition}[Pearsonの不等式]
    $X\in L^1$について,$\Var[X]>0$とする.
    このとき,$\fraks^2(X)+1\le\frakk(X)$.特に$\fraks(X)=0$ならば,$\frakk(X)\ge 1$.
\end{proposition}

\begin{proposition}[共分散行列の双線型性]\mbox{}
    \begin{enumerate}
        \item 可積分確率変数$X_1,\cdots,X_m,Y_1,\cdots,Y_n$の積$X_iY_j$も可積分とする.
        このとき,
        \[\forall_{a_1,\cdots,a_m,b_1,\cdots,b_n\in\R}\quad\Cov\paren{\sum^m_{i=1}a_iX_i,\sum_{j=1}^nb_jY_j}=\sum^m_{i=1}\sum^n_{j=1}a_ib_j\Cov()X_i,Y_j)\]
        \item 各$X_i\;(i\in[m])$を$r_i$次元確率変数,各$Y_j\;(j\in[n])$を$c_j$次元確率変数とする.
        すべての$X_i,Y_j,X_iY_j\in L^1$のとき,任意の$r\times r_i$定数行列$A_i$と$c\times c_j$定数行列$B_j$について
        \[\Cov\paren{\sum_{i=1}^mA_iX_i,\sum^n_{j=1}B_jY_j}=\sum^m_{i=1}\sum^n_{j=1}A_i\Cov(X_i,Y_j)B_j^\perp\]
    \end{enumerate}
\end{proposition}
\begin{remark}
    行列$M$に関して,$\abs{M}=(\Tr(MM^\perp))^{1/2}$とすると,$\abs{X_i\otimes Y_j}=\abs{X_i}\abs{Y_j}$だから,$\abs{X_i}\abs{Y_j}$が可積分であることと,$\abs{X_i\otimes Y_j}$が可積分であることと,$X_i,Y_j$の要素のペアの積がすべて可積分であることは同値になる.
\end{remark}

\begin{proposition}[和の分散]
    $X_1,\cdots,X_n\in L^2$について,
    \[\Var\paren{\sum^n_{i=1}X_i}=\sum^n_{i=1}\Var(X_i)+2\sum_{(i,j):1\le i<j\le n}\Cov(X_i,Y_j).\]
\end{proposition}

\begin{proposition}[相関係数の値域]
    $\mu_X=E[X],\mu_Y=E[Y],\sigma_X=\sqrt{\Var[X]},\sigma_Y=\sqrt{\Var[Y]}$とする.
    \[\abs{\rho(X,Y)}\le 1\]
    等号成立条件は,$Y=\pm\frac{\sigma_Y}{\sigma_X}(X-\mu_X)+\mu_Y\;\as$.
\end{proposition}
\begin{proof}
    $\sigma_X\sigma_Y\ne 0$のとき,
    \[0\le E[(\sigma_X^{-1}(X-\mu_X)\pm\sigma_Y^{-1}(Y-\mu_Y))^2]=2(1\pm\rho(X,Y))\]
    より.
\end{proof}

\begin{proposition}[ランダムな双線型形式の期待値]
    $m$次元確率変数$X$と$n$次元確率変数$Y$とについて,$\abs[X],\abs{Y},\abs{X}\abs{Y}$を可積分とする.
    このとき,$m\times n$-定数行列$A$について,
    \[E(X^TAY)=\Tr(\Cov(X,Y)^\perp A)+E(X)^\perp AE(Y)\]
\end{proposition}

\subsection{分散共分散行列}

\begin{tcolorbox}[colframe=ForestGreen, colback=ForestGreen!10!white,breakable,colbacktitle=ForestGreen!40!white,coltitle=black,fonttitle=\bfseries\sffamily,
title=]
    分散共分散行列と半正定値行列は同一視出来る.\footnote{\url{https://ja.wikipedia.org/wiki/分散共分散行列}}
\end{tcolorbox}

\begin{lemma}
    分散共分散行列$\Var[X]$は半正定値である.
\end{lemma}
\begin{proof}
    $\forall_{u\in\R^{d'}}\;u^\perp\Var[X]u=E[(u\cdot(X-E[X]))^2]$より.
\end{proof}
\begin{remark}[退化した多次元確率変数]
    $\Var[X]$が正定値でないとすると,$\exists_{u\in\R^d\setminus\{0\}}\;u^\perp\Var[X]u=E([u^\perp(X-E(X))]^2)=0$である.
    すなわち,$X$は確率1で超平面$u^\perp(X-E(X))=0$上に値を取る.
\end{remark}

\section{分布関数}

\begin{tcolorbox}[colframe=ForestGreen, colback=ForestGreen!10!white,breakable,colbacktitle=ForestGreen!40!white,coltitle=black,fonttitle=\bfseries\sffamily,
title=]
    分布関数には3つの同値な定義がある.これらが同値である理由は,Riemann-Stieltjes積分による.
    任意の単調増加関数$F$に対して,有界関数$f$の積分が定まるが,次の3条件を満たすことと,$F$が絶対連続な確率分布を定めることとが同値になる.
\end{tcolorbox}

\subsection{定義と特徴付け}

\begin{definition}[distribution function]\mbox{}
    \begin{enumerate}
        \item 次の3条件を満たす有界関数$F:\R\to[0,1]$を\textbf{分布関数}という.
        \begin{enumerate}[(i)]
            \item 有界性: $\lim_{x\to-\infty}F(x)=0,\lim_{x\to\infty}F(x)=1$.
            \item 右連続性.
            \item 広義単調増加性.
        \end{enumerate}
        \item 実確率変数$X$に対して,$F^X(x):=P[X\le x]$により定まる関数$F^X:\R\to[0,1]$は分布関数になる.これを\textbf{$X$の(累積)分布関数}という.
        \item $(\R,\B_1)$上の確率測度$\nu$に対して,$F_\nu(x):=\nu((-\infty,x])=\int^x_{-\infty}f(y)dy$により定まる関数$F^X:\R\to[0,1]$は分布関数になる.これを\textbf{$\nu$の(累積)分布関数}という.
    \end{enumerate}
\end{definition}

\begin{lemma}
    $F:\R\to\R$を有界関数とする.このとき,次は同値:
    \begin{enumerate}
        \item $F$は分布関数である.
        \item ある確率測度$\nu\in P(\R)$が存在して,この累積分布関数である:$F=F_\nu$.
    \end{enumerate}
\end{lemma}
\begin{proof}\mbox{}
    \begin{description}
        \item[(1)$\Rightarrow$(2)] 任意のBorel集合$(-\infty,x]\in\B_1$について,$\nu((-\infty,x]):=F(x)$とすると,
        $(-\infty,x]$は任意の$\R$の開・閉区間を生成するので,
        $\B_1$を生成することから,
        $\nu$は$\B_1$上の測度に延長する.
        さらに,$\nu(\R)=F(\infty)=1$だから,$\nu$は確率測度である.
        \item[(2)$\Rightarrow$(1)] $F(x)=\nu((\infty,x])$であるから,
        \begin{enumerate}[(i)]
            \item 明らか.
            \item 確率測度の連続性より.
            \item 確率測度の正性より.
        \end{enumerate}
    \end{description}
\end{proof}

\subsection{分布関数の構成}

\begin{tcolorbox}[colframe=ForestGreen, colback=ForestGreen!10!white,breakable,colbacktitle=ForestGreen!40!white,coltitle=black,fonttitle=\bfseries\sffamily,
title=]
    $D(\R;[0,1])$の中で,$F$は凸集合をなし,さらにある意味で稠密である.
\end{tcolorbox}

\begin{proposition}
    $\sum_{j\in\N}p_j=1$を満たす任意の非負実数列$(p_j)$と分布関数の列$(F_j)$について,
    \[F(x):=\sum_{j\in\N}p_jF_j(x)\]
    は分布関数である.
\end{proposition}
\begin{remark}
    どうやら一様収束はしない.
\end{remark}

\subsection{分布の微分としての性質}

\begin{tcolorbox}[colframe=ForestGreen, colback=ForestGreen!10!white,breakable,colbacktitle=ForestGreen!40!white,coltitle=black,fonttitle=\bfseries\sffamily,
title=]
    一般に,定数でない連続関数$F:[a,b]\to\R$が,殆ど至る所で微分係数をもち,かつそれが消えているとき,\textbf{特異}であるという.
    特異な連続分布関数を持つような分布のみ,確率密度関数や確率質量関数に当たるものを持たない.
\end{tcolorbox}

\begin{definition}[singular continuous distribution]
    $F$を分布関数とし,
    $f(x):=F(x)-\lim_{y\nearrow x}F(y)\ge0$とする.
    \begin{enumerate}
        \item $f(x)\ne0$であるとき,$x\in\R$を$F$の\textbf{不連続点}といい,$f(x)$を\textbf{飛躍量}という.
        \item $\forall_{\ep>0}\;F(x+\ep)>F(x-\ep)$が成り立つとき,点$x$は$F$の\textbf{増加点}であるという.不連続点ならば増加点である.
        \item 連続な分布関数$F$の増加点全体の集合がLebesgue零集合であるとき,$F$を\textbf{連続な特異}であるという.
    \end{enumerate}
\end{definition}

\begin{theorem}[Lebesgue分解]
    任意の分布関数$F$は,絶対連続分布$F_a$と離散分布$F_d$と連続特異分布$F_s$の凸結合として一意的に表せる.
\end{theorem}

\subsection{一意性定理}

\begin{proposition}[一意性定理]
    $\nu_1=\nu_2$と$F_{\nu_1}=F_{\nu_2}$は同値.
\end{proposition}

\subsection{弱収束の特徴付け}

\begin{tcolorbox}[colframe=ForestGreen, colback=ForestGreen!10!white,breakable,colbacktitle=ForestGreen!40!white,coltitle=black,fonttitle=\bfseries\sffamily,
title=]
    そもそも分布関数は,連続点上での値が定まれば,大域的にも定まる.
    この点に注意すれば,弱収束は,分布関数の各点収束で特徴付けられる.
\end{tcolorbox}

\begin{proposition}[Prokhorovの定理の一部]
    $\R$上の分布$(\nu_n),\nu$の分布関数を$F_n,F$とする.
    次の2条件は同値.
    \begin{enumerate}
        \item $\nu_n\wto\nu$.
        \item $F$の任意の連続点$x\in\R$について,$F_n(x)\to F(x)$.
    \end{enumerate}
\end{proposition}

\begin{remarks}
    $F$も分布関数である(右連続である)という仮定を取るとどうなるか.
    分布関数列の各点収束極限$F$は右連続とは限らないが,連続点上での値を変えない右連続化$F_0$は常に存在する.
    しかし最も肝心なのは,$F(\pm\infty)$の値が変わってしまう.
\end{remarks}

\subsection{分布関数の弱収束}

\begin{tcolorbox}[colframe=ForestGreen, colback=ForestGreen!10!white,breakable,colbacktitle=ForestGreen!40!white,coltitle=black,fonttitle=\bfseries\sffamily,
title=]
    \cite{清水良一}.
\end{tcolorbox}

\section{確率母関数}

\begin{tcolorbox}[colframe=ForestGreen, colback=ForestGreen!10!white,breakable,colbacktitle=ForestGreen!40!white,coltitle=black,fonttitle=\bfseries\sffamily,
title=]
    $\Z_+:=\N$上の離散分布を扱う際には,確率母関数もよく用いられ,ここから積率母関数も特性関数も求まる.
    $\varphi(u)=g(e^{iu})$なので,特性関数の微分と確率母関数の微分は本質的に等価で,便利な方を使えば良い.
\end{tcolorbox}

\subsection{特性関数と確率母関数}

\begin{definition}[characteristic function, probability generating function]\mbox{}
    \begin{enumerate}
        \item 実数上の確率変数$\varphi(u):=\sum_{x\in\X}e^{iux}p_x:\R\to\C$を,確率関数$\mu=(p_x)_{x\in\X}$の\textbf{特性関数}または\textbf{Fourier変換}という.$\abs{e^{iux}}=1$より,これは必ず収束する.\footnote{理論解析の極みのような存在である.well-definedであり,一般性を持ち,また性質が理想的である.また,確率変数が確率密度関数を持つ場合、特性関数と密度関数は互いにもう一方のフーリエ変換になっているという意味で双対である。}
        \item 特に$\X=\N\subset\R$のとき,実数上の確率変数$g(z):=\sum_{x\in\N}p_xz^x$を\textbf{確率母関数}という.特性関数は,これに$z=e^{iu}$と変数変換を合成した場合である:$\varphi(u)=g(e^{iu})$.\footnote{これは特性関数の自然数への特価である.$e$なぞ持ち出さなくても良い場合,$\varphi(u)=g(e^{iu})$という関係がある.}
    \end{enumerate}
\end{definition}

\begin{lemma}[fractional moment:分散公式・積率は特性関数の微分・平均と分散は確率母関数の微分]\label{lemma-variance-formula}\mbox{}
    \begin{enumerate}
        \item (分散公式) $\sigma^2=\alpha_2-\mu^2=\sum_{x\in\X}x^2p_x-\paren{\sum_{x\in\X}xp_x}^2$.
        \item $\beta_r<\infty$のとき,$\alpha_r=i^{-r}\varphi^{(r)}(0)$.
        \item $g$が$z=1$で項別微分可能であるとき,階乗モーメントが$g^{(r)}(1)=\sum^\infty_{x=r}x(x-1)\cdots(x-r+1)p_x$である.
        \item 特に,平均と分散について$\al_1=g'(1),\mu_2=g''(1)+g'(1)-(g'(1))^2$が成り立つ.
    \end{enumerate}
\end{lemma}
\begin{proof}\mbox{}
    \begin{enumerate}
        \item \begin{align*}
            \sigma^2&=\sum_{x\in\X}(x-\mu)^2p_x\\
            &=\sum_{x\in\X}x^2p_x-2\mu\underbrace{\sum_{x\in\X}xp_x}_{=\mu}+\mu^2\underbrace{\sum_{x\in\X}p_x}_{=1}\\
            &=\alpha_2-\mu^2.
        \end{align*}
        \item $\beta_r<\infty$ならば,$\varphi^{(r)}(u)=\sum_{x\in\X}(ix)^re^{iux}p_x$は収束し(すなわち項別微分可能で),関数$\varphi^{(r)}:\R\to\R$を定める.$u=0$として,$\varphi^{(r)}(0)=i^r\sum_{x\in\X}x^rp_x=i^r\alpha_r$.
        \item 一般に,項別微分可能ならば$g^{(r)}(z)=\sum^\infty_{x=0}x(x-1)\cdots(x-r+1)p_xz^{x-r}$であるが,いま$\X=\Z$としているから,$z=1$のとき,$g^{(r)}(1)=\sum^\infty_{x=r}x(x-1)\cdots(x-r+1)p_x$である.
        また,特に$g'(1)=\mu$,$g''(1)=\sum_{x\in\X}x(x-1)p_x$であるから,(1)より,$\sigma^2=\underbrace{g''(1)+g'(1)}_{=\sum(x(x-1)+x)p_x}-(g'(1))^2$
    \end{enumerate}
\end{proof}


\subsection{分位点関数}

\begin{definition}[quantile function]\mbox{}
    \begin{enumerate}
        \item 累積分布関数$F$が連続かつ狭義単調増加であるとき,逆関数$F^{-1}$を\textbf{分位点関数}と定める.
        \item 一般の場合,$F_L^{-1}(u):=\inf\Brace{x\in\R\mid F(x)\ge u},F_R^{-1}(u):=\sup\Brace{x\in\R\mid P(X\ge x)\ge1-u}$と定める.
    \end{enumerate}
\end{definition}

\begin{lemma}\mbox{}
    \begin{enumerate}
        \item $F^{-1}_L$は左連続である.
        \item $F^{-1}_R$は右連続である.
    \end{enumerate}
\end{lemma}

\begin{definition}
    分布関数$F(x)$において,$x\to\pm\infty$としたときの収束の速さを\textbf{裾の重さ}という.
    期待値や分散は積分で定義されるために,裾の重さに影響を受けやすい.
\end{definition}

\subsection{母関数の概念の射程}

\begin{tcolorbox}[colframe=ForestGreen, colback=ForestGreen!10!white,breakable,colbacktitle=ForestGreen!40!white,coltitle=black,fonttitle=\bfseries\sffamily,
title=]
    母関数(generating function)とは「数列の定める関数」である.\footnote{母関数の「母」は,関数を母親と見て数列の各項を子供と見立てて,そのように呼んでいます.(ちなみに,英語では generating function という味わいのない呼び方をします.)}
    「数列を各項ごとに調べるよりも一度に扱った方が物事が見えてくることが多いので,母関数を導入して,その性質を調べることは数学の常套手段です.」
    保型関数もその大成功例である.
    そこで,確率分野でもFourier展開を考えると,その係数が積率である.
    離散数学,物理学,統計学へ.
\end{tcolorbox}

\begin{remark}[generating function]
    一般に母関数とは,
    Knuthの第一章に乗っているくらいに組合せ数学でよく使われる手法で,今回の離散変数のような列$(f_n)_{n\in\N}$という対象に対して定義される,
    そのrig $R$上の形式的冪級数の集合$R[[z]]$の元$f(z)=\sum_{n=0}^\infty f_nz^n$のことをいう.
    これにより,数列が関数として生まれ変わったこととなり,こちらを解析することができる.また,元の数列も微分の言葉で再現できる.
    $R=\N,\Q,\R,\C$などが主に考えられ,後2者についてはTaylor展開の理論が,$\C$では解析接続の理論が模範としてあり,そこでは母関数とは解析関数のことをいう.
    最初にこの方法を始めたのは一般線型回帰問題を解くために\footnote{数列とその母関数の対応は線型同型の代表例に他ならない.}A. de Moivreが創始し,James Stirling, Euler, Laplaceが応用した.
\end{remark}
\begin{example}[母関数の応用]
    関手$\mathrm{Seq}\to\mathrm{Fun}$に他ならない.
    $G(z)$が列$(a_n)$の母関数で,$H(z)$が列$(b_n)$の母関数とする.
    \begin{enumerate}
        \item 加算:$\alpha G(z)+\beta H(z)$.
        \item シフト:$z^mG(z)$.
        \item 乗算:$GH$.
    \end{enumerate}
    (1),(2)の組み合わせが,漸化式を解くという行為である.
\end{example}

\begin{definition}[moment]
    一般に,関数$f:\R\to\R$の,$x=c$を中心とすr$n$次の積率を
    \[\mu_n^{(c)}=\int^\infty_{-\infty}(x-c)^nf(x)dx\]
    と定める.$f$を密度関数とする測度の重心は$\mu=\frac{\mu^{(0)}_1}{\mu^{(0)}_0}$と表せる.
    ほとんどの場合,中心$c$は重心=平均に取る.
\end{definition}

\section{特性関数}

\begin{tcolorbox}[colframe=ForestGreen, colback=ForestGreen!10!white,breakable,colbacktitle=ForestGreen!40!white,coltitle=black,fonttitle=\bfseries\sffamily,
title=積分変換]
    期待値作用素が積分によって定義されるため,「ある関数の平均を考える」こと,すなわち
    積分変換による分析手法が肝要になる.
    Fourier変換により,確率変数$X\in\Meas(\Om,\R)$を,関数$\varphi:\C\to[\Delta]$に
    $E[e^{iuX}]=E[\cos u X]+iE[\sin uX]$により対応づけて考える.

    逆転公式とは,母関数から元の確率分布を復元する算譜である.
\end{tcolorbox}

\subsection{特性関数とFourier変換}

\begin{tcolorbox}[colframe=ForestGreen, colback=ForestGreen!10!white,breakable,colbacktitle=ForestGreen!40!white,coltitle=black,fonttitle=\bfseries\sffamily,
title=]
    特性関数と分布の対応は,緩増加超関数の空間上のFourier逆変換$\F^{-1}:\cS'(\R)\iso\cS'(\R)$の$P(\R)$への制限が与える.
    Fourier解析的には,$u\in\R$に対して,その指標$e_u:\R\to\partial\Delta$の積分を対応させる合成関数であるから,「指標関数」が正しいかもしれない.
\end{tcolorbox}

\begin{definition}[characteristic function]
    $(\R,\B_1)$上の確率測度$\nu$に対して,
    \[\varphi(u):=\int e^{iux}\nu(dx)=\int\cos(ux)\nu(dx)+i\int\sin(ux)\nu(dx)\]
    により定まる関数$\varphi:\R\to\C$を\textbf{特性関数}という.
    $\abs{e^{iux}}=1$より,特性関数は常に存在する.
\end{definition}

\begin{proposition}[確率変数の演算との対応]
    $X$の特性関数を$\varphi$とする.
    \begin{enumerate}
        \item $\frac{X-\mu}{\sigma^2}$の特性関数は,$\varphi\paren{\frac{u}{\sigma^2}}e^{-i\mu u/\sigma^2}$.
        \item $P^X$が対称ならば,$\varphi$も対称な実数値関数となる.
    \end{enumerate}
\end{proposition}

\begin{lemma}[inversion formula]\label{lemma-反転公式}
    確率測度$\nu$とその特性関数$\varphi$について,区間$(a,b)$で$\nu[\{a\}]=\nu[\{b\}]=0$ならば,すなわち,$a,b$が$F$の連続点ならば,
    \[\nu[(a,b)]=F(b)-F(a)=\frac{1}{2\pi}\lim_{T\to\infty}\int^T_{-T}\frac{e^{-iua}-e^{-iub}}{iu}\varphi(u)du.\]
\end{lemma}

\begin{theorem}[一意性定理]
    特性関数の全体と確率測度の全体とに標準的な全単射が存在する.すなわち,任意の$\mu_1,\mu_2\in\P^d$について,次は同値.
    \begin{enumerate}
        \item $\mu_1=\mu_2$.
        \item $\varphi_{\mu_1}=\varphi_{\mu_2}$.
    \end{enumerate}
\end{theorem}
\begin{proof}
    $d=1$の場合について,2通りの方法で示す.
    \begin{description}
        \item[反転公式による証明] 
        $\varphi_\mu=\varphi_{\wt{\mu}}$とする.
        反転公式\ref{lemma-反転公式}
        により,任意の(端点が$F_\mu,F_{\wt{\mu}}$の連続点である)開区間の測度は$\varphi$が一意に定める:$F_\mu(b)-F_\mu(a)=F_{\wt{\mu}}(b)-F_{\wt{\mu}}(a)$.
        $b\in\R$が$F_\mu,F_{\wt{\mu}}$両方の連続点であるとき,$F_\mu(b)=F_{\wt{\mu}}(b)$.
        分布関数は右連続で,不連続点は高々可算個だから,これは$F_\mu=F_{\wt{\mu}}$を含意する.分布関数の一意性より,$\mu=\wt{\mu}$.
        \item[緩増加超関数の空間上のFourier変換による証明]
        まず,$P(\R)\subset\cS'(\R)$を示す.
        $\mu\in P(\R)$に対して,$\brac{\mu,-}:\cS(\R)\to\R$は有界線型汎関数であることを示す.
        $\abs{\brac{\mu,f}}\le\Abs{\int_\R f(x)\mu(dx)}\le\norm{f}_\infty$.
        このとき,$\mu\in\cS'(\R)$のFourier逆変換が$\varphi_\mu$である.
        $\cS'(\R)$上のFourier変換は同型対応である.
    \end{description}
\end{proof}

\begin{corollary}
    $L^2$-確率変数$X\in\L^2$がある$\al>0$について$\varphi_X(u)=\exp(-\abs{u}^\al)$ならば,$\al=2$で,$X\sim N(0,2)$である.
\end{corollary}

\subsection{特性関数の特徴付け}

\begin{proposition}
    関数$\varphi:\R\to\C$が特性関数であるならば,次の3条件が成り立つ.
    \begin{enumerate}
        \item $\varphi(0)=1$.
        \item 一様連続である.
        \item 正定値である:$\forall_{n\in\N,\xi_j\in\R,z_j\in\C}\;\sum^n_{j,k=1}\varphi(\xi_j-\xi_k)z_j\o{z}_k\ge0$.
    \end{enumerate}
\end{proposition}

\begin{theorem}[Bochner]\label{thm-Bochner}
    関数$\varphi:\R\to\C$が次の3条件を満たすならば,ある一次元分布$\mu\in\P(\R)$の特性関数である.特に,命題の3条件を満たすならば,特性関数である.
    \begin{enumerate}
        \item $\xi=0$で連続.
        \item $\varphi(0)=1$.
        \item 正定値である.
    \end{enumerate}
\end{theorem}

\subsection{特性関数と分布の収束の対応}

\begin{example}
    \[\nu_n(A):=\sum^n_{j=1}\frac{1}{n}\delta_{\frac{j}{n}}(A)\quad(A\in\B(\R))\]
    は一様分布$U(0,1)$に弱収束する.
    実際,積分の定義より,
    \[\int_\R g(x)\nu_n(dx)=\sum^n_{j=1}\frac{1}{n}g\paren{\frac{j}{n}}\xrightarrow{n\to\infty}\int^1_0g(x)dx.\]
\end{example}

\begin{theorem}
    確率測度$\nu_n$の特性関数を$\varphi_n$とする.(1)$\Rightarrow$(2)である.各点収束極限$\varphi$が原点で連続であるとき,(2)$\Rightarrow$(1)でもある.
    \begin{enumerate}
        \item ある確率測度$\nu$について,$\nu_n\Rightarrow\nu$である.
        \item $\varphi_n$が関数$\varphi$に各点収束する.
    \end{enumerate}
\end{theorem}

\begin{corollary}[Glivenko]
    確率測度$\nu_n,\nu$の特性関数を$\varphi_n,\varphi$とする.
    このとき,次の2条件は同値.
    \begin{enumerate}
        \item $\nu_n\to\nu$.
        \item $\forall_{u\in\R}\;\varphi_n(u)\to\varphi(u)$.
    \end{enumerate}
\end{corollary}
\begin{proof}\mbox{}
    \begin{description}
        \item[(1)$\Rightarrow$(2)] 定義より自明.
        \item[(2)$\Rightarrow$(1)] 定理による.
    \end{description}
\end{proof}

\subsection{特性関数の滑らかさと積率の存在}

\begin{tcolorbox}[colframe=ForestGreen, colback=ForestGreen!10!white,breakable,colbacktitle=ForestGreen!40!white,coltitle=black,fonttitle=\bfseries\sffamily,
title=]
    特性関数は,分布の滑らかさに関する情報を次のような形で保持している.
\end{tcolorbox}

\begin{proposition}[特性関数の導関数の表示]
    $r$位の絶対積率が存在するとする:$\beta_r<\infty$.
    このとき,特性関数$\varphi$は$C^r$級で,さらに$\varphi^{(r)}$は一様連続でもあり,
    \[\dd{^r}{u^r}\varphi(u)=i^r\int e^{iux}x^r\nu(dx)\]
    と表示できる.
    特に,積率$\al_r$について,$\varphi^{(r)}(0)=i^r\al_r$.
\end{proposition}

\begin{theorem}
    
\end{theorem}

\begin{proposition}
    $\nu\in\P^d$とする.ある$m\in\Z_+$に関して,
    \[\int\abs{u}^m\abs{\varphi_F(u)}du<\infty\]
    ならば,$\nu$の確率密度関数
    \[f(x)=\frac{1}{(2\pi)^d}\int e^{-iu\cdot x}\varphi_F(u)du\]
    は$C^m$-級で,
    さらに$\forall_{k\le m}\;f^{(k)}\in C_0(\R^d)$が成り立つ.
\end{proposition}

\begin{proposition}[特性関数の滑らかさが示唆する絶対積率の存在]
    分布$\nu$の特性関数$\varphi$が$u=0$において$2r$次までの微分を持つとする.
    このとき,$\abs{\beta_{2r}}<\infty$.
\end{proposition}

\subsection{特性関数のTaylor展開}

\begin{tcolorbox}[colframe=ForestGreen, colback=ForestGreen!10!white,breakable,colbacktitle=ForestGreen!40!white,coltitle=black,fonttitle=\bfseries\sffamily,
title=]
    特性関数をTaylor展開することで,高次の積率$\al_n$も求めることが出来る.
\end{tcolorbox}

前節の定理はTaylorの定理と併せれば,次のようにも捉えられる.

\begin{corollary}[特性関数のTaylor展開係数からの積率の計算]
    $\beta_r<\infty$とする.このとき,$\varphi$は$u=0$の周りで展開
    \[\varphi(u)=\sum^r_{n=0}\al_n\frac{(iu)^n}{n!}+o(u^r)\quad(u\to0)\]
    を持つ.すなわち,中心積率は特性関数を用いて次のように表せる:
    \[\al_n=\left.\frac{1}{i^n}\dd{^n\varphi(u)}{u^n}\right|_{u=0}\]
\end{corollary}

\begin{corollary}[平均と分散の特性関数による特徴付け]\mbox{}\label{cor-mean-and-variance-in-terms-of-characteristic-function}
    \begin{enumerate}
        \item 平均(1次の積率)は$\al_1=\frac{1}{i}\varphi'(0)$.
        \item 分散(2次の中心積率)は$\mu_2=-\varphi''(0)+(\varphi'(0))^2$.
    \end{enumerate}
\end{corollary}

\section{キュムラント関数}

\begin{tcolorbox}[colframe=ForestGreen, colback=ForestGreen!10!white,breakable,colbacktitle=ForestGreen!40!white,coltitle=black,fonttitle=\bfseries\sffamily,
title=]
    特性関数のTaylor展開係数には,積率$\al_r$が登場した.
    キュムラント関数のTaylor展開係数を,キュムラントと言い,これが平均・分散の一般化となっている.
\end{tcolorbox}

\subsection{第2キュムラント母関数}

\begin{tcolorbox}[colframe=ForestGreen, colback=ForestGreen!10!white,breakable,colbacktitle=ForestGreen!40!white,coltitle=black,fonttitle=\bfseries\sffamily,
title=]
    キュムラントはキュムラント母関数=積率母関数の対数の展開係数として得られるが,特性関数の対数からも,虚数単位$i$に関する修正を施して得られる.
    その関係は,特性関数と積率の関係と並行である.この関数を,第2キュムラント母関数という.
\end{tcolorbox}

\begin{proposition}[キュムラント関数のMaclaurin展開]
    $\beta_r<\infty$とする.
    このとき,関数$\psi(u):=\log\varphi(u)$は$u=0$の近傍で定まり,次の形の展開を持つ:
    \[\psi(u)=\sum^r_{j=1}\kappa_j\frac{(iu)^j}{j!}+o(u^r)\quad(\abs{u}\to0).\]
\end{proposition}

\begin{definition}[cuumulant]
    関数$\psi(u):=\log\varphi(u)$を\textbf{第2キュムラント母関数}または\textbf{キュムラント関数}といい,この展開係数$\kappa_n=\frac{1}{i^n}\left.\pp{^n}{u^n}\psi(u)\right|_{u=0}$を,分布$\nu$の\textbf{$n$次のキュムラント}と呼ぶ.
    この$n\in\N$は,多重指数$n\in\N^d$と読み替えれば,多次元の場合にも通用する.
    分布の高次の形態を表す特性値である.
\end{definition}

\begin{example}[キュムラントの例]\mbox{}
    \begin{enumerate}
        \item 1次のキュムラント$(\kappa_n)_{\abs{n}=1}$とは,\textbf{平均ベクトル}である.
        \item 2次のキュムラント$(\kappa_n)_{\abs{n}=2}$とは,\textbf{分散共分散行列}である.
    \end{enumerate}
    さらに高次のキュムラントはテンソルになる\ref{def-higher-cumulant}.
    正規分布ではここは消える.これが平均と分散までが大事であることに繋がるが,これは人類の認知負荷の限界と関係があるのか?
\end{example}

\begin{corollary}[キュムラントの積率・中心積率による表現]
    可積分性を仮定する.
    \begin{enumerate}
        \item $\kappa_1=\al_1$.
        \item $\kappa_2=\al_2-\al_1^2=\mu_2$.
        \item $\kappa_3=\al_3-3\al_1\al_2+2\al_1^3=\mu_3$.
        \item $\kappa_4=\al_4-3\al_2^2-4\al_1\al_3+12\al_1^2\al_2-6\al_1^4=\mu_4-3\mu_2^2$.
    \end{enumerate}
\end{corollary}

\begin{corollary}[積率・中心積率のキュムラントによる表現]
    可積分性を仮定する.
    \begin{enumerate}
        \item $\al_2=\kappa_2+\kappa_1^2$,$\mu_2=\kappa_2$.
        \item $\al_3=\kappa_3+3\kappa_1\kappa_2+\kappa^3_1$,$\mu_3=\kappa_3$.
        \item $\al_4=\kappa_4+3\kappa_2^2+4\kappa_1\kappa_3+6\kappa_1^3\kappa_2+\kappa^4_1$,$\mu_4=\kappa_4+3\kappa_2^2$.
    \end{enumerate}
\end{corollary}

\section{積率母関数}

\subsection{Laplace変換}

\begin{tcolorbox}[colframe=ForestGreen, colback=ForestGreen!10!white,breakable,colbacktitle=ForestGreen!40!white,coltitle=black,fonttitle=\bfseries\sffamily,
title=]
    $\R_+$上の分布に限っては,特性関数と同等にLaplace変換も扱う.
\end{tcolorbox}

\begin{notation}
    \[\P_+:=\Brace{\mu\in\P(\R)\mid\supp\mu\subset[0,\infty)}\]
    と表す.
\end{notation}

\begin{definition}
    $F\in\P_+$に対して,
    \[\L_F(u):=\int^\infty_0e^{-ux}F(dx)\]
    で定まる関数$\L_F:\R_+\to\R_+$を,$F$の\textbf{Laplace変換}という.
\end{definition}

\begin{theorem}[一意性定理]
    $F,G\in\P_+$について,次の2条件は同値.
    \begin{enumerate}
        \item $F=G$.
        \item $\L_F=\L_G$.
    \end{enumerate}
\end{theorem}

\subsection{積率母関数}

\begin{tcolorbox}[colframe=ForestGreen, colback=ForestGreen!10!white,breakable,colbacktitle=ForestGreen!40!white,coltitle=black,fonttitle=\bfseries\sffamily,
title=]
    通常はLaplace変換$L(t)=E[e^{-tX}]$ではなく,その逆$M(t):=L(-t)$が用いられ,これを統計学の観点からは積率母関数という.
    積率母関数が$u=0$の近傍で定義されるとき,これにより確率分布が一意に決定される.
    理論的な欠点は常に存在するとは限らないところで,漸近展開などで用いられるのは特性関数である.
    存在するならば特性関数に対して$\varphi(u)=\fM(e^{iu})$なる関係があり,解析接続を通じて一意性定理が述べられる.

    積率母関数の対数をキュムラント母関数という.一方で,特性関数の対数は第2キュムラント母関数という.
\end{tcolorbox}

\begin{definition}[moment generating function]
    $(\R,\B_1)$上の確率測度$\nu$に対して,$M_\nu(t):=\int_R e^{tx}\nu(dx)$により定まる関数$\R\nrightarrow\R$を\textbf{積率母関数}という.
    定義域を
    \[D_\nu:=\Brace{t\in\R^d\mid\int e^{t\cdot x}F(dx)<\infty}\]
    で表す.
\end{definition}

\begin{lemma}[マクローリン展開係数は積率である]
    $\fM_\nu$が原点の近傍で存在するならば,そこで$C^\infty$級であり,$\forall_{n\in\Z^d_+}\;\partial^n\fM_\nu(0)=\al_n$.
    特に,
    \begin{enumerate}
        \item $\al_1=\partial\fM_\nu(0)=\partial(\log\fM_\nu)(0)=\partial\fC_\nu(0)$.
        \item $\mu_2=\partial^2(\log\fM_\nu)(0)=\partial^2\fC_\nu(0)$.
    \end{enumerate}
\end{lemma}

\begin{theorem}[特性関数との関係と一意性定理]\mbox{}
    \begin{enumerate}
        \item $\nu\in\P^d$とする.$D_\nu$が$0\in\R^d$の近傍ならば,ある整数$\ep$が存在して,積率母関数$\fM_\nu$は$\cR_\ep:=((-\ep,\ep)\times i\R)^d$上の解析関数$\fM^\dagger_\nu$に解析接続され,$\varphi_\nu(-)=\fM^\dagger_\nu(i-)$が成り立つ.
        \item 正則関数$\fM^\dagger_\nu$は,$z\in\D_\ep:=\Brace{\zeta\in\C\mid\abs{\zeta}<\ep}^d$上において,絶対収束級数
        \[\fM^\dagger_\nu(z)=\sum^\infty_{k=1}\frac{1}{k!}\int(z\cdot x)^k\nu(dx)\]
        なる展開を持つ.
        \item $\nu_1,\nu_2\in\P^d$のとき,原点のある開近傍$U\subset D_{\nu_1}\cap D_{\nu_2}$上で$\fM_{\nu_1}=\fM_{\nu_2}$ならば,$\nu_1=\nu_2$が成り立つ.
    \end{enumerate}
\end{theorem}

\begin{remark}
    なお,積率母関数は,確率母関数$G(z):=E[z^X]$に対して$G[e^t]=\fM_X(t)$なる関係もある.
\end{remark}

\subsection{キュムラント母関数}

\begin{tcolorbox}[colframe=ForestGreen, colback=ForestGreen!10!white,breakable,colbacktitle=ForestGreen!40!white,coltitle=black,fonttitle=\bfseries\sffamily,
title=]
    積率母関数の対数の展開係数をキュムラントといい,この特徴量の列も分布を特徴付ける.
    キュムラントのことは,その性質を研究したThorvald N. Thieleに因み、ティエレの半不変数(semi-invariant)とも呼ぶ。\footnote{統計学の分野で尤度に関する初期の考察を行い、保険数学の分野でHafnia保険会社を設立し、数学部長を務め、デンマーク保険統計協会を設立した。}
    積率との繋がりが深い.
\end{tcolorbox}

\begin{definition}
    $\nu\in\P^d$に対して,
    $\fC_\nu(t):=\log\fM_F(t)$で定まる$D_\nu$上の関数を,$F$の\textbf{キュムラント母関数}という.
\end{definition}
\begin{remark}[affine変換に対する半不変性]
    半不変数というのは,積率は
    $Y=\al X+\beta$なるaffine変換に対して
    $E[Y^2]=\al^2 E[\al^2]+2\al\beta E[X]+\beta^2$なるように変換されるのに対して,
    \[\kappa_1^Y=\al\kappa_1^X+\beta,\quad\kappa_n^Y=\al^n\kappa_n^X\;(n=2,3,\cdots)\]
    というように殆ど形を変えないことにちなむ.
\end{remark}

\begin{example}[キュムラント母関数は極めて少ししか変わらなくても,密度関数は見かけ上全く異なる例]
    
\end{example}

\subsection{積率問題}

\begin{tcolorbox}[colframe=ForestGreen, colback=ForestGreen!10!white,breakable,colbacktitle=ForestGreen!40!white,coltitle=black,fonttitle=\bfseries\sffamily,
title=]
    積率問題の解の存在は,Hahn-Banachの定理の応用である.
    ここでは確率的な文脈であるが,物理的にも,電荷密度$\rho$に関する条件として読める.
\end{tcolorbox}

\begin{problem}[moment problem]
    実数列$\al:\N\to\R$に対して,これを積率の列に持つ分布が一意に定まるか?
\end{problem}

\begin{proposition}
    有限個への制限$\al|_{[N]}$について,これを満たす積率問題の解は存在する.
\end{proposition}

\begin{theorem}[存在の必要十分条件]
    区間$[a,b]$上の分布であって,積率$\al$を持つものを考える.
    \begin{enumerate}
        \item 積率問題は解を持つ.
        \item $\exists_{c>0}\;\forall_{N\in\N}\;\forall_{a_k\in\R}\;\Abs{\sum^N_{k=0}a_k\al_k}\le c\max_{a\le x\le b}\Abs{\sum^N_{k=0}a_kx^k}$.
    \end{enumerate}
\end{theorem}

\begin{theorem}[一意性の十分条件]
    $(\al_n)_{n\in\N}$が$\exists_{t_0>0}\;\sum^\infty_{n=1}\frac{\abs{\al_n}}{n!}t^n_0<\infty$を満たすならば,$\al$を積率とする分布はただ一つである.
\end{theorem}

\begin{example}
    3次以上のキュムラントがすべて$0$になる確率分布は,正規分布に限る.
\end{example}

\subsection{多次元確率変数の特性関数}

\begin{tcolorbox}[colframe=ForestGreen, colback=ForestGreen!10!white,breakable,colbacktitle=ForestGreen!40!white,coltitle=black,fonttitle=\bfseries\sffamily,
title=]
    複素数値関数が特性関数になるためにはBochnerの定理が十分条件を与える.
\end{tcolorbox}


\section{確率変数の独立性と期待値}

\begin{tcolorbox}[colframe=ForestGreen, colback=ForestGreen!10!white,breakable,colbacktitle=ForestGreen!40!white,coltitle=black,fonttitle=\bfseries\sffamily,
title=]
    有限個の確率変数$X_1,\cdots,X_n$の独立性とは,これが定める$(X_1,\cdots,X_n)$の結合分布が,
    各$X_i$の分布$P^{X_i}$が定める直積速度$P^{X_1}\times\cdots\times P^{X_n}$に一致することをいう.
    本質的に「直積」とみなせるものを独立性という.
\end{tcolorbox}

\subsection{独立な確率変数の積の期待値}

\begin{notation}
    $X_j:(\Om,\F,P)\to(\X_j,\B_j)$を確率変数とする.
\end{notation}

\begin{proposition}
    $n\ge 2$を整数とする.可測関数$f_j:\X_j\to\o{\R}$について,$f_j(X_j)\in\L^1(\Om,\F,P)$とする.
    このとき,$X_1,\cdots,X_n$が独立ならば,積$\prod^n_{i=1}f_i(X_i)$も可積分で,
    \[E\Square{\prod^n_{i=1}f_i(X_i)}=\prod^n_{i=1}E[f_i(X_i)]\]
    を満たす.
\end{proposition}
\begin{remark}
    複素数値可測関数にも一般化出来る.
\end{remark}

\begin{corollary}[独立性の十分条件]
    任意の有界可測関数$f_i$について
    \[E\Square{\prod^n_{i=1}f_i(X_i)}=\prod^n_{i=1}E[f_i(X_i)]\]
    が成り立つならば,$X_1,\cdots,X_n$は独立である.
\end{corollary}

\subsection{確率変数の独立性と共分散}

\begin{proposition}
    $X,Y\in\L^1(\Om,\F,P)$を独立とする.このとき,$\Cov[X,Y]=0$.
\end{proposition}

\subsection{確率変数の独立性と特性関数}

\begin{tcolorbox}[colframe=ForestGreen, colback=ForestGreen!10!white,breakable,colbacktitle=ForestGreen!40!white,coltitle=black,fonttitle=\bfseries\sffamily,
title=]
    平均作用素$E$の積に対する振る舞いによって,独立性の十分条件を与えることが出来た.
    特性関数によっても特徴付けることが出来る.
\end{tcolorbox}

\begin{notation}
    多変量確率変数$Y$の特性関数を$\varphi_Y$と表す.
    $X_j\;(j\in[n])$を$d_j$次元確率変数とする.
    $X:=(X_1,\cdots,X_n)$は$d:=\sum_{j=1}^nd_j$次元確率変数である.
    $X,X_j$の特性関数は
    \[\varphi_X(u)=E\Square{e^{i\sum^n_{j=1}u_j\cdot X_j}}\;(u\in\R^d),\qquad\varphi_{X_j}(u_j)=E\Square{e^{iu_j\cdot X_j}}\;(u_j\in\R^{d_j})\]
    となる.
\end{notation}

\begin{theorem}[Kac]\label{thm-Kac}
    確率変数列$X_1,\cdots,X_n$について,次の2条件は同値.
    \begin{enumerate}
        \item $X_1,\cdots,X_n$は独立である.
        \item $\varphi_X=\varphi_{X_1}\otimes\cdots\otimes\varphi_{X_n}:=\prod_{j=1}^n\varphi_{X_j}(u_j)$である.
    \end{enumerate}
\end{theorem}

\begin{corollary}
    $d$次元確率変数の列$X_1,\cdots,X_n$が独立ならば,
    \[\varphi_{\sum_{j=1}^nX_j}(u)=\prod^n_{j=1}\varphi_{X_j}(u)\;(u\in\R^d)\]
\end{corollary}

\subsection{独立同分布に従う確率変数列}

\begin{notation}
    $X_1,\cdots,X_n$が$\R^d$に値を取る$\iid$であって,$E[X_1]=\mu,\Var[X_1]=\Sigma$であることを,$X_1,\cdots,X_n\sim\iid(\mu,\Sigma)$で表す.
\end{notation}

\subsection{独立確率変数の和と畳み込み}

\begin{tcolorbox}[colframe=ForestGreen, colback=ForestGreen!10!white,breakable,colbacktitle=ForestGreen!40!white,coltitle=black,fonttitle=\bfseries\sffamily,
title=]
    確率変数の和は,分布の畳み込みに対応し,特性関数のテンソル積に対応する.
\end{tcolorbox}

\begin{corollary}
    $d$次元確率変数$X_1,\cdots,X_n$が独立であれば,$X:=X_1+\cdots+X_n$について
    \[\forall_{u\in\R^d}\quad\varphi_X(u)=\prod^n_{j=1}\varphi_{X_j}(u).\]
\end{corollary}

\begin{example}
    Hermite分布とSkellam分布\ref{def-Hermite-and-Skella}.
\end{example}

\begin{notation}
    $A-x:=\Brace{y\in\R^d\mid y+x\in A}$と表す.
\end{notation}

\begin{definition}[convolution]
    $\nu_1,\nu_2\in\P(\R^d)$に対して,
    \[\nu(A):=\int_{\R^d}\nu_1(A-x)\nu_2(dx)\]
    で定まる確率測度$\nu$を$\nu_1*\nu_2$で表す.
\end{definition}

\begin{lemma}\mbox{}
    \begin{enumerate}
        \item $\nu_1*\nu_2(A)=\nu_2*\nu_1(A)=\int_{\R^{2d}}1_A(x_1+x_2)\nu_1(dx_1)\nu_2(dx_2)$.
        \item 独立な確率変数$X_1,X_2$が$P^{X_1}=\nu_1,P^{X_2}=\nu$を満たすとき,$P^{X_1+X_2}=\nu_1*\nu_2$.
        \item $\varphi_{\nu_1*\nu_2}=\varphi_{\nu_1}\otimes\varphi_{\nu_2}$.
    \end{enumerate}
    (3)は$\nu_1,\nu_2$の独立性を特徴付けない点に注意.
\end{lemma}

\begin{lemma}\label{lemma-semigroup-of-distributions}
    $P(\R^d)$は$*$を積として,単位元$\delta_0$を持つ可換な半群をなす.
\end{lemma}

\begin{definition}[reproducing property]
    ある分布族について,畳み込みについて閉じていることを\textbf{分布族の再生性}という.
\end{definition}

\begin{example}\mbox{}\label{exp-reproducing-families}
    \begin{enumerate}
        \item $G(\al,\nu_1)*G(\al,\nu_2)=G(\al,\nu_1+\nu_2)$.
        \item $N(\mu_1,\Sigma_1)*N(\mu_2,\Sigma_2)=N(\mu_1+\mu_2,\Sigma_1+\Sigma_2)$.
        \item $C(\mu_1,\sigma_1)*C(\mu_2,\sigma_2)=C(\mu_1+\mu_2,\sigma_1+\sigma_2)$.
        \item $B(n_1,p)*B(n_2,p)=B(n_1+n_2,p)$.
        \item $\Pois(\lambda_1)*\Pois(\lambda_2)=\Pois(\lambda_1+\lambda_2)$.
    \end{enumerate}
\end{example}

\section{多次元分布の扱い}

\subsection{多次元確率変数の分布}

\begin{tcolorbox}[colframe=ForestGreen, colback=ForestGreen!10!white,breakable,colbacktitle=ForestGreen!40!white,coltitle=black,fonttitle=\bfseries\sffamily,
    title=]
    離散確率分布をtableにまとめた時,分布表の中の部分が結合分布となり,縁の部分(合計欄)が周辺分布となる.
\end{tcolorbox}

\begin{definition}[joint / simultaneous distribution, marginal distribution]
    $X=(X_1,\cdots,X_d):\Om\to\R^d$を確率変数とする.
    \begin{enumerate}
        \item $(\R^d,\B_d)$上の確率分布$P^X$を$X_1,\cdots,X_d$の\textbf{結合分布}または\textbf{同時分布}という.
        \item 各$X_i$の分布$P^{X_i}$を\textbf{周辺分布}という.$P^{(X_{i_1},\cdots,X_{i_k})}$も,$(X_{i_1},\cdots,X_{i_k})$の周辺分布という.
    \end{enumerate}
\end{definition}

\begin{example}[multinomial distribution]
    $k$個の背反な事象$\sum^k_{i=1}A_i=\Om$について,$P(A_i)=p_i$とする.各$A_i$の起こる回数を$X_i$とすると,確率ベクトル$(X_1,\cdots,X_k)$の分布を\textbf{$k$項分布}$M(n;p_1,\cdots,p_k)$という.
    $k=2$の場合は二項分布に等しくなる.
    また,各$X_i$の周辺分布は2項分布$B(n,p_i)$である.
    多項分布は必ず$\sum_{i=1}^kX_i=n$という線型関係を満たすので,この意味で退化した分布である.
\end{example}

\subsection{共分散}

\begin{definition}[covariance, correlation coefficient]
    2乗可積分実確率変数$X,Y$について,
    \begin{enumerate}
        \item 
        $\Cov[X,Y]:=E[(X-E[X])(Y-E[Y])]$
        を\textbf{共分散}という.
        \item $\rho(X,Y):=\frac{\Cov[X,Y]}{\sqrt{\Var[X]}\sqrt{\Var[Y]}}$
        を\textbf{相関係数}という.
    \end{enumerate}
\end{definition}

\begin{proposition}\label{prop-covariance-formula}
    2乗可積分実確率変数$X,Y,Z$について,
    \begin{enumerate}
        \item $\Cov[X,Y]=\Cov[Y,X]$.
        \item $\forall_{a,b\in\R}\;\Cov[aX+bY,Z]=a\Cov[X,Z]+b\Cov[Y,Z]$.
        \item $\Cov[X,X]=\Var[X]\ge 0$.等号が成立するならば$X=E[X]\;\as$.
        \item $\Cov[X,1]=0$.$\forall_{a,b\in\R}\;\Cov[aX+b,Y]=a\Cov[X,Y]$.
        \item $\Cov[X,Y]=E[XY]-E[X]E[Y]$.
    \end{enumerate}
\end{proposition}

\begin{example}[多項分布の共分散]
    $X=(X_1,\cdots,X_k)\sim M(n;p_1,\cdots,p_k)$とする.
    \begin{enumerate}
        \item $E[X_i]=np_i$.
        \item $\Var[X_i]=np_i(1-p_i)$.
        \item $\Cov[X_{i_1},X_{i_2}]=-np_{i_1}p_{i_2}\;(i_1\ne i_2\in[k])$.
    \end{enumerate}
\end{example}
\begin{proof}
    多項定理より,
    \[(e^{u_1}p_1+\cdots+e^{u_k}p_k)^n=\sum_{x_1,\cdots,x_k}{}^*P^X[\Brace{(x_1,\cdots,x_k)}]e^{u_1x_1+\cdots+u_kx_k}.\]
    ただし,${}^*$は線形関係$x_1+\cdots+x_k=n$を満たす$x_1,\cdots,x_k$についての和とする.
    \begin{enumerate}
        \item 両辺の$(u_1,\cdots,u_k)=(0,\cdots,0)$における$u_i$偏微分係数より,$np_i=\sum_{x_1,\cdots,x_k}{}^*x_iP^X[\Brace{(x_1,\cdots,x_k)}]=E[X_i]$.
        \item 同様に$u_i$の2階微分を考えて,$E[X_i^2]=n(n-1)p_i^2+np_i$を得る.よって,$\Var[X_i]=E[X_i^2]-(E[X_i])^2=np_i(1-p_i)$.
        \item $i_1\ne i_2$に関して,順に$u_{i_1},u_{i_2}$での偏微分を考えることにより,$E[X_{i_1}X_{i_2}]=n(n-1)p_{i_1}p_{i_2}$.
        よって,共分散公式\ref{prop-covariance-formula}より,
        \[\Cov[X_{i_1},X_{i_2}]=E[X_{i_1}X_{i_2}]-E[X_{i_1}]E[X_{i_2}]=-np_{i_1}p_{i_2}.\]
    \end{enumerate}
\end{proof}
\begin{remarks}
    証明が技巧的すぎる,多項定理に指数関数を代入する.
\end{remarks}

\subsection{共分散行列}

\begin{definition}[covariance matrix,  variance-covariance matrix]
    $X:\Om\to\R^d$を確率変数とする.
    \begin{enumerate}
        \item $X$が可積分のとき,項別積分
        \[E[X]=\begin{bmatrix}
            E[X_1]\\\vdots\\E[X_d]
        \end{bmatrix}\]
        を$X$の\textbf{平均ベクトル}という.
        \item $X,Y$が2乗可積分のとき,
        \[\Cov[X,Y]=\begin{bmatrix}
            \Cov[X_1,Y_1]&\cdots&\Cov[X_1,Y_s]\\
            \vdots&\ddots&\vdots\\
            \Cov[X_d,Y_1]&\cdots&\Cov[X_d,Y_s]
        \end{bmatrix}\]
        を$X,Y$の\textbf{共分散行列}という.
        \item $\Cov[X,X]$を$X$の\textbf{分散共分散行列}または\textbf{分散行列}と呼ぶ.
    \end{enumerate}
\end{definition}
\begin{remark}
    この共分散行列は、シンプルではあるが、非常に多岐にわたる分野でとても有用なツールである。分散共分散行列からは、データの相関を完全に失わせるような写像を作る変換行列を作ることができる。これは、違った見方をすれば、データを簡便に記述するのに最適な基底を取っていることになる。(分散共分散行列のその他の性質やその証明については、en:Rayleigh quotientを参照) これは、統計学では主成分分析 (PCA) と呼ばれており、画像処理の分野では、カルーネン・レーベ変換(KL-transform) と呼ばれている。\footnote{\url{https://ja.wikipedia.org/wiki/分散共分散行列}}
\end{remark}

\begin{notation}[期待値作用素の拡張]
    期待値作用素$E$を行列値確率変数$M=[M_{ij}]\in M_{ij}(\R)$上に対しても$E:M_{ij}(\R)\to M_{ij}(\R);E[M]=[E[M_{ij}]]$と拡張すると,平均ベクトルは$E[M]$,共分散行列は$\Cov(X,Y)=E[(X-E[X])(Y-E[Y])^T]$と表せる.
\end{notation}

\begin{proposition}\mbox{}\label{prop-multidimensional-covariacne}
    \begin{enumerate}
        \item 可積分$d$次元確率変数$X$,$m\times d$行列$A$,$a\in\R^m$に対して,$E[AX+a]=AE[X]+a$.
        \item 2乗可積分$d$次元確率変数$X$,2乗可積分$s$次元確率変数$Y$に対して,$\Cov[X,Y]=\Cov[Y,X]^T$.また,$\Cov[X,Y]=E[XY^T]-E[X]E[Y]^T$.
        \item 2乗可積分$d$次元確率変数$X,Y$,2乗可積分$s$次元確率変数$Z$に対して,$\Cov[X+Y,Z]=\Cov[X,Z]+\Cov[Y,Z]$.
        \item 2乗可積分$d$次元確率変数$X$,$m\times d$行列$A$,$a\in\R^m$,2乗可積分$s$次元確率変数$Y$,$b\in\R^n$に対して,$\Cov[AX+a,BY+b]=A\Cov[X,Y]B^T$.
    \end{enumerate}
\end{proposition}

\begin{proposition}[確率ベクトルの2次形式の平均]
    $X$を$d$次元2乗可積分確率変数,$G$を$d\times d$定数行列とする.このとき,
    \[E[X^TGX]=\Tr(G\Var[X])+E[X]^TGE[X].\]
\end{proposition}

\subsection{積率の定義}

\begin{tcolorbox}[colframe=ForestGreen, colback=ForestGreen!10!white,breakable,colbacktitle=ForestGreen!40!white,coltitle=black,fonttitle=\bfseries\sffamily,
    title=]
    積率の次元もベクトル値$n\in\N^d$で指定できる.
\end{tcolorbox}

\begin{notation}
    $x=(x_1,\cdots,x_d)\in\R^d$に対して,$\partial_i:=\pp{}{x_i}$とし,$n:=(n_1,\cdots,n_d)\in\Z^d_+$に対して,
    \begin{align*}
        \abs{n}&:=n_1+\cdots+n_d,&n!&:=n_1!\cdots n_d!,\\
        x^n&:=x_1^{n_1}\cdots x_d^{n_d},&\partial^n:=\partial_1^{n_1}\cdots\partial_d^{n_d}.
    \end{align*}
    ただし,$x^0_j=1,\partial^0=1$とよむ.
\end{notation}

\begin{definition}[moment, central moment]
    $d$次元確率変数$X=(X_1,\cdots,X_d)$に対して,可積分性の仮定の下で,
    \begin{enumerate}
        \item $\al_n:=E[X^n]$を\textbf{$n$次積率}という.
        \item $\mu_n:=E[(X-E[X])^n]$を\textbf{$n$次中心積率}という.
    \end{enumerate}
\end{definition}

\begin{definition}
    $(\R^d,\B_d)$上の確率測度$\nu$に対して,可積分性の仮定の下で,
    \begin{enumerate}
        \item $\mu:=\paren{\int_{\R^d}x_i\nu(dx)}\in\R^d$を\textbf{平均(ベクトル)}という.
        \item $\al_n:=\int_{\R^d}x^n\nu(dx)$を\textbf{$n$次の積率}という.
        \item $\nu_n:=\int_{\R^d}(x-\mu)^n\nu(dx)$を\textbf{$n$次の中心積率}という.
        \item 特に$(\mu_n)_{\abs{n}=2}$を$\nu$の\textbf{分散共分散行列}という.
    \end{enumerate}
\end{definition}

\subsection{キュムラントの定義}

\begin{definition}\label{def-higher-cumulant}
    第2キュムラント母関数によるキュムラントの定義
    \[\psi(u)=\sum^r_{j=1}\kappa_j\frac{(iu)^j}{j!}+o(u^r)\quad(\abs{u}\to0).\]
    は,$n$を多重指数$\mathbf{n}$に読み替えることで,$\kappa_\bn:=i^{-\abs{\bn}}\partial^{\bn}\psi(0)$そのまま拡張される.
\end{definition}

\begin{notation}
    $(\partial_{u_i})_0$を,$u:=(u_i)=0$における$u_i$-偏微分係数を表す.
\end{notation}

\begin{discussion}[キュムラントのテンソル表現]
    $d$次元確率分布$\nu$について,
    $\al_1,\cdots,\al_r\in[d]$に対して
    \[\lambda^{\al_1\cdots\al_r}:=(-i)^r(\partial_{u_{\al_1}})_0\cdots(\partial_{u_{\al_r}})_0\log\varphi(u)\]
    と定める.これも$\nu$の$r$次のキュムラントといい,行列$(\lambda^{\al_1\cdots\al_r})_{\al_1,\cdots,\al_r\in[d]}$は対称テンソルを定める.
\end{discussion}
\begin{remarks}
    $\br:=(r_1,\cdots,r_d)\in\Z^d_+$が$r_k:=\Abs{\Brace{j\in[r]\mid\al_j=k}}$を満たすとする.このとき,
    \[\lambda^{\al_1\cdots\al_r}=\kappa_\br\]
\end{remarks}


\section{コピュラ}

\begin{tcolorbox}[colframe=ForestGreen, colback=ForestGreen!10!white,breakable,colbacktitle=ForestGreen!40!white,coltitle=black,fonttitle=\bfseries\sffamily,
title=]
    相関係数は,2つの確率変数に対して,実数$[-1,1]$を対応させるが,もっと表現力豊かに,
    変量間の従属性を,多次元分布によって直接的に表現することを考える.\footnote{この単語は元々音楽や言語学で使われていたが、統計学の用語として用いたのは、1959 年にスクラー (Abe Sklar) がパリ大学統計学会誌 (the Statistical Institute of the University of Paris) で発表したのが最初である。}

    これはなんだか,周辺分布の張り合わせとして多次元分布を表現するホモトピー論のようである.
    逆に,コピュラなく,直積によって周辺分布を張り合わせた場合が,独立性である.
\end{tcolorbox}

\begin{notation}
    $I:=[0,1]$とする.
\end{notation}

\subsection{定義と例}

\begin{definition}[copula]
    次の条件を満たす関数$C:I^2\to I$を\textbf{接合関数}という:
    \begin{enumerate}
        \item $C(-,0)=C(0,-)=0$.
        \item $C(-,1)=C(1,-)=\id_I$.
        \item $\forall_{u_1,u_2,v_1,v_2\in I}\;u_1\le v_1,u_2\le v_2\Rightarrow C(v_1,v_2)-C(u_1,v_2)-C(v_1,u_2)+C(u_1,u_2)\ge0$.
    \end{enumerate}
\end{definition}

\begin{example}[凸関数とproduct copula]
    $C^2$級関数$C$が$\partial_{u_1}\partial_{u_2}C(u_1,u_2)\ge0$をみたすとき,コピュラである.
    特に,$C^\Pi(u_1,u_2):=u_1u_2$を\textbf{積コピュラ}という.
\end{example}

\begin{example}[二次元の分布関数が定めるコピュラ]
    \begin{enumerate}\mbox{}
        \item 区間$(0,1)$上の一様分布に従う2つの確率変数$X_1,X_2\sim U((0,1))$の分布関数$F$はコピュラである.
        \item 
    \end{enumerate}
\end{example}

\subsection{Sklarの定理}

\begin{tcolorbox}[colframe=ForestGreen, colback=ForestGreen!10!white,breakable,colbacktitle=ForestGreen!40!white,coltitle=black,fonttitle=\bfseries\sffamily,
title=]
    多次元分布は,周辺分布関数とコピュラによって定まる.
\end{tcolorbox}

\begin{theorem}[2次元の場合 (Sklar 1959)]
    $(X_1,X_2)$の結合分布関数を$F$,それぞれの周辺分布関数を$F_1,F_2$とする.
    このとき,コピュラ$C$が存在して,
    \[\forall_{x_1,x_2\in\R}\quad F(x_1,x_2)=C(F_1(x_1),F_2(x_2))\]
    と表せる.
    $F_1,F_2$が連続のとき,$C$は一意である.
\end{theorem}
\begin{remarks}
    逆関数$F_1^{-1},F_2^{-1}$が存在するときは,これを用いて$C(u_1,u_2)=F(F_1^{-1}(u_1),F_2^{-1}(u_2))$とコピュラを構成できる.
\end{remarks}

\subsection{Archimedesコピュラ}


\chapter{確率分布の例}

\begin{quotation}
    離散分布というクラスは,極限構成について閉じていないという意味で,理論的な実用性がない.
    極限について議論するには,絶対連続分布と実数というところに行き着く.
    このクラスは「絶対連続」なる名前だが,確率が積分によって表される確率分布という意味である.
    分布の特性値は,確率分布が定める積分に,適切な積分核を挟んで得られるものであることが明確になる.
\end{quotation}

\section{一次元離散分布の例}

\subsection{デルタ分布}

\begin{tcolorbox}[colframe=ForestGreen, colback=ForestGreen!10!white,breakable,colbacktitle=ForestGreen!40!white,coltitle=black,fonttitle=\bfseries\sffamily,
title=]
Shanonnのエントロピーとは自己情報量$I(p):=-\log_2p\;(p\in\X)$が定める積分作用素$H:\Meas(\X,\R)\to[0,\log\al]$であるが,これが最小になるときの確率分布である.
\end{tcolorbox}

\begin{definition}[degenerated / delta distribution]
    ある一点のみで$1$をとる確率質量関数が定める確率分布を\textbf{退化分布}と呼ぶ.
    デルタ関数$\delta_a\;(a\in X)$がこの測度$\ep_a(A):=1_{\Brace{B\in\A\mid a\in B}}\;(A\in\A)$を定める.
    これは自然な埋め込み$X\mono\M(X)$とも考えられる.\footnote{例えば弱位相について,この埋め込みの像の凸包は稠密である.}
\end{definition}

\subsection{経験分布}

\begin{tcolorbox}[colframe=ForestGreen, colback=ForestGreen!10!white,breakable,colbacktitle=ForestGreen!40!white,coltitle=black,fonttitle=\bfseries\sffamily,
title=]
    経験分布関数と順序統計量は一対一対応する.
\end{tcolorbox}

\begin{definition}[order statistic, empirical distribution function, resampling from sample, bootstrap method]
    $X_1,\cdots,X_n\sim F,\iid$とする.
    \begin{enumerate}
        \item これらの確率変数の値を小さい順に並べ替えて得る列$X_{(1)}\le X_{(2)}\le\cdots\le X_{(n)}$を\textbf{順序統計量}という.
        \item 順序統計量の第$i$成分$X_{(i)}$を\textbf{第$i$順序統計量}といい,第一順序統計量を\textbf{最小値},第$n$順序統計量を\textbf{最大値}という.
        \item 特定の値$x\in\Om$に対して,$x$以下となる観測値の割合を返す関数$F_n(x):=\frac{1}{n}\abs{\Brace{i\in[n]\mid X_i\le x}}$を\textbf{経験分布関数}という.\footnote{点$x_i\in\Om$に確率$1/n$を持つような離散確率分布の累積分布関数となっているので,記法も寄せている.}
        \item すでに得られた標本から再び標本抽出を行うことを\textbf{標本からのリサンプリング}とよぶが,これは経験分布$F_n$に従う確率変数を観測することにあたる.
        \item すでに得られた標本$(x_i)_{i\in[n]}$の経験分布$F_n$からリサンプリングを繰り返し,$F_n$を代用して仮想的な標本の取り直しを行う方法を\textbf{ブートストラップ法}という.
    \end{enumerate}
\end{definition}
\begin{remark}
    大数の強法則によって,$P\Square{\lim_{n\to\infty}\abs{F_n(x)-F(x)}=0}=1$が従うが,より強い「一様大数の法則」ともいうべき結果がある.
\end{remark}

\begin{theorem}[Glivenko-Cantelli (1933)]
    \[P\Square{\lim_{n\to\infty}\sup_{x\in\R}\abs{F_n(x)-F(x)}=0}=1.\]
\end{theorem}

\subsection{Rademacher分布}

\begin{tcolorbox}[colframe=ForestGreen, colback=ForestGreen!10!white,breakable,colbacktitle=ForestGreen!40!white,coltitle=black,fonttitle=\bfseries\sffamily,
title=]
    機械学習研究者の十八番.
    Rademacher過程は,幅1のランダムウォークと考えられる.
\end{tcolorbox}

\begin{definition}[Rademacher distribution, Rademacher / Steinhaus / symmetric Bernoulli variable]\mbox{}
    \begin{enumerate}
        \item $\Z$上の確率質量関数
        \[f(k)=\frac{1}{2}(\delta(k-1)+\delta(k+1))\]
        によって定まる離散分布を,\textbf{Rademacher分布}という.
        \item Rademacher確率分布に従う独立同分布な確率変数列$\ep_1,\cdots,\ep_n$に対して,$X:=\sum^n_{i=1}\ep_i$と定めた$X:\R^n\to\R$を\textbf{Rademacher過程}という.
        \item 複素数値Rademacher変数はSteinhaus変数とも呼ばれ,$P(a<\arg(\ep)<b)=\frac{1}{2\pi}(b-a)$で定まる.
    \end{enumerate}
\end{definition}
\begin{remark}
    $X$がRademacher分布に従う確率変数ならば,$\frac{X+1}{2}\sim B(1/2)$.
\end{remark}

\begin{proposition}[Van Zuijlen's bound]
    \[P\paren{\Abs{\frac{\sum^n_{i=1}X_i}{\sqrt{n}}}\le 1}\ge\frac{1}{2}.\]
\end{proposition}

\subsection{離散一様分布}

\begin{tcolorbox}[colframe=ForestGreen, colback=ForestGreen!10!white,breakable,colbacktitle=ForestGreen!40!white,coltitle=black,fonttitle=\bfseries\sffamily,
title=]
    整数$ 1, 2, \cdots, N$から$k$個の標本が非復元抽出され、離散一様分布と同様に、標本の抽出のされ方に整数による差はないとする。
    ここで未知の最大値$ N $を推定する問題が生じる。このような問題を一般に German tank problem(ドイツ戦車問題)と呼び、第二次世界大戦中のドイツでの戦車生産数の最大値を推定するという問題に由来する。 
\end{tcolorbox}

\begin{definition}[discrete uniform distribution]
    $0<\abs{\X}<\infty$を満たす有限集合上の,定値関数$p=\frac{1}{\abs{\X}}$を確率関数として定まる測度
    $U:\FinSet\to\M(\X)$を\textbf{離散一様分布}という.
\end{definition}
\begin{remarks}
    Shanonnのエントロピーとは自己情報量$I(p):=-\log_2p\;(p\in\X)$が定める積分作用素$H:\Meas(\X,\R)\to[0,\log\al]$であるが,これが最大になるときの確率分布である.
\end{remarks}

\subsection{二項分布}

\begin{tcolorbox}[colframe=ForestGreen, colback=ForestGreen!10!white,breakable,colbacktitle=ForestGreen!40!white,coltitle=black,fonttitle=\bfseries\sffamily,
title=]
    600人の中で1年を365日として,今日誕生日の人が$x$人である確率は,$b\paren{x;600,\frac{1}{365}}$となる.
\end{tcolorbox}

\begin{definition}[binomial distribution]
    可測空間$(n+1,P(n+1))$上の
    \textbf{パラメータ$x,p$の二項分布}$B(n,p):\N\times[0,1]\to\M(n+1)$は,成功確率が$p$で一定な試行(Bernoulli試行という)を独立に$n$回続けるという確率空間$2^n$における確率分布であり,
    確率変数を成功回数$x\in n+1$とすると,
    確率質量関数$b:n+1\to[0,1]$は$b(x;n,p)=\begin{pmatrix}n\\x\end{pmatrix}p^xq^{n-x}$と表される.
    $\Bernoulli(1,p):=B(1,p)$をBernoulli分布という.
\end{definition}

\begin{proposition}[二項分布の平均と分散]\mbox{}
    \begin{enumerate}
        \item 平均について,$\al_1(b(n,p))=np$.
        \item 分散について,$\mu_2(b(n,p))=npq$.
    \end{enumerate}
\end{proposition}
\begin{proof}\mbox{}
    \begin{enumerate}
        \item \begin{align*}
            \mu&=\sum^n_{x=0}x\begin{pmatrix}n\\x\end{pmatrix}p^xq^{n-x}\\
            &=\sum^n_{x=0}x\frac{n!}{x!(n-x)!}p^xq^{n-x}\\
            &=np\sum^n_{x=1}\frac{(n-1)!}{(x-1)!(n-x)!}p^{x-1}q^{n-x}\\
            &=np\sum^{n-1}_{y=0}\frac{(n-1)!}{y!(n-1-y)}p^yq^{n-1-y}&y:=x-1\\
            &=np\sum^{n-1}_{y=0}b(n-1,p)=np.
        \end{align*}
        \item 確率関数$(b(x;n,p))_{x\in n+1}$が定める母関数$g:\R\to\R$は
        \begin{align*}
            g(z)&=\sum^n_{x=0}b(x;n,p)z^x\\
            &=\sum^n_{x=0}\begin{pmatrix}n\\x\end{pmatrix}p^xq^{1-x}z^x
            =(pz+q)^n
        \end{align*}
        と表せる.$g'(z)=np(pz+q)^{n-1},g''(z)=n(n-1)p^2(pz+q)^{n-2}$より,分散公式\ref{lemma-variance-formula}から,
        \[\sigma^2=g''(1)+g'(1)-g'(1)^2=n(n-1)p^2+np-n^2p^2=npq.\]
    \end{enumerate}
\end{proof}
\begin{remarks}
    確率母関数が,二項展開の式に一致することから,ここに予想だにしなかった抜け道がある.
\end{remarks}

\subsection{Poisson分布}

\begin{tcolorbox}[colframe=ForestGreen, colback=ForestGreen!10!white,breakable,colbacktitle=ForestGreen!40!white,coltitle=black,fonttitle=\bfseries\sffamily,
title=極限分布のお手本]
    所与の時間間隔で,確率が線型に減衰する現象の観測回数を確率変数としたモデルである.
    放射線物質から一定期間に放射される粒子の数の経時変化はPoisson過程であり,一定期間に起こる事故の数など.
    逆に発生間隔は指数分布となる.
\end{tcolorbox}

\subsubsection{定義}

\begin{definition}[Poisson distribution (1838)]
    可測空間$(\N,P(\N))$上の\textbf{Poisson分布}$\Pois(\lambda):(0,\infty)\to\M(\N)$とは,確率質量関数$p(x;\lambda)=\frac{\lambda^x}{x!}e^{-\lambda}:\N\to[0,1]$が定める確率分布である.
    母数$\lambda$は,単位時間当たりの事象の平均発生回数などの割合と見なされる場合は,\textbf{到着率}と呼ばれる.
    平均も分散も$\lambda$に一致する.
\end{definition}



\subsubsection{極限分布としてのPoisson分布}\begin{proposition}[Poisson's limit theorem]
    減衰する確率$p(n):=\frac{\lambda}{n}$についての二項分布$B(n,p(n))$は,$\lim_{n\to\infty}b(x;n,p)=\frac{\lambda^x}{x!}e^{-\lambda}\;(\forall_{x\in\N})$を満たす.
\end{proposition}
\begin{proof}
    \begin{align*}
        b\paren{x;n,\frac{\lambda}{n}}&=\begin{pmatrix}n\\x\end{pmatrix}\paren{\frac{\lambda}{n}}^x\paren{1-\frac{\lambda}{n}}^{n-x}\\
        &=\frac{n(n-1)\cdots (n-x+1)}{x!}\paren{\frac{\lambda}{n}}^x\paren{1-\frac{\lambda}{n}}^{n-x}\\
        &=\frac{\lambda^x}{x!}\paren{1-\frac{1}{n}}\cdots\paren{1-\frac{x-1}{n}}\paren{1+\frac{-\lambda}{n}}^{n/(-\lambda)\cdot(-\lambda)}\paren{1-\frac{\lambda}{n}}^{-x}\\
        &\xrightarrow{n\to\infty}\frac{\lambda^x}{x!}e^{-\lambda}
    \end{align*}
    この収束は一様収束である.
\end{proof}

\begin{proposition}[Poisson分布の確率母関数]
    Poisson分布$\Pois(\lambda)$の定める確率母関数は$g(z)=e^{\lambda(z-1)}$である.
\end{proposition}
\begin{proof}
    確率変数列$(p(x;\lambda))_{x\in\N}=\paren{\frac{\lambda^x}{x!}e^{-\lambda}}_{x\in\N}$が定める母関数は,
    \begin{align*}
        g(z)&=\sum_{x=0}^\infty \paren{\frac{\lambda^x}{x!}e^{-\lambda}}z^x\\
        &=e^{-\lambda}\sum_{x=0}^\infty\frac{(\lambda z)^x}{x!}=e^{-\lambda}e^{\lambda z}=e^{\lambda(z-1)}.
    \end{align*}
\end{proof}

\begin{history}
    歴史的に有名な事例としては、ロシア生まれでドイツで活躍した経済学者、統計学者のボルトケヴィッチによる「プロイセン陸軍で馬に蹴られて死亡した兵士数」の例が知られている。ボルトケヴィッチは著書"Das Gesetz der kleinen Zahlen "(The Law of Small Numbers)[4]において、プロイセン陸軍の14の騎兵連隊の中で、1875年から1894年にかけての20年間で馬に蹴られて死亡する兵士の数について調査しており、1年間当たりに換算した当該事案の発生件数の分布が母数 0.61 のポアソン分布によく従うことを示している。 
\end{history}

\subsubsection{ポアソン過程}

\begin{tcolorbox}[colframe=ForestGreen, colback=ForestGreen!10!white,breakable,colbacktitle=ForestGreen!40!white,coltitle=black,fonttitle=\bfseries\sffamily,
title=]
    Poisson点過程/Poisson点場において,ある有界領域内の点の数はPoisson分布に従う確率変数となる.
    このことから名付けられたのがPoisson点過程で,Poisson自体はこれについて研究していない.
\end{tcolorbox}

\begin{definition}[Poisson process]
    $\lambda>0$を定数として,次の3条件を満たす確率過程$(P_k(t))_{k\in\N}$を\textbf{Poisson過程}という:
    時間$(0.t]$に点事象が起こる回数が$k$回である確率を$P_k(t)$とし,
    \begin{enumerate}
        \item 区間$(t,t+\Delta t]$に1回だけ事象が起こる確率は$\Delta t\searrow 0$のとき$\lambda\Delta t+o(\Delta t)$である.
        \item 区間$(t,t+\Delta t]$に2回以上事象が起こる確率は$o(\Delta t)$である.
        \item 区間$(t,t+\Delta t]$に点事象が起こる回数は$(0,t]$で起こる回数とその起こり方に「独立」である.\footnote{確率過程の独立はまだ定義していない.}
    \end{enumerate}
\end{definition}

\begin{lemma}
    Poisson過程は存在する.
\end{lemma}

\begin{proposition}
    Poisson過程$(P_k(t))_{k\in\N}$は$\N$上のPoisson分布$\Pois(\lambda t)$である.
\end{proposition}
\begin{proof}\mbox{}
    \begin{description}
        \item[関係式の導出] 区間$(t,t+\Delta t]$で点事象が1回起こる確率が$\lambda\Delta(t)+o(\Delta t)$,2回以上起こる確率が$o(\Delta t)$より,0回起こる確率が$1-\lambda\Delta t-o(\Delta t)$だから,
        \[P_k(t+\Delta t)=P_k(t)(1-\lambda\Delta t-o(\Delta t))+P_{k-1}(t)(\lambda\Delta t-o(\Delta t))+o(\Delta t)\]
        となる.ただし,$k\in\N,P_{-1}(t)=0$とした.
        $\Delta t\searrow 0$を考えることで,微分方程式
        \[P'_k(t)=\lambda(P_{k-1}(t)-P_k(t)),\quad P_{-1}(t)=0\]
        を得る.
        \item[分布の導出]
        まず$k=0$とすると,
        \[P'_0(t)=-\lambda P_0(t),\quad P_0(0)=1\]
        より,$P_0(t)=e^{-\lambda t}$.
        次に$k=1$とすると,
        \[P'_1(t)=\lambda(e^{-\lambda t})-P_1(t),\quad P_1(0)=0\]
        より,$P_1(0)=\lambda te^{-\lambda t}$.
        $k=2$とすると,
        \[P'_2(t)=\lambda(\lambda te^{-\lambda t}-P_2(t)),\quad P_2(0)=0\]
        より,$P_2(t)=\frac{(\lambda t)^2}{2}e^{-\lambda t}$.以下帰納的に,$P_k(t)=\frac{(\lambda t)^k}{k!}e^{-\lambda t}$を得る.
    \end{description}
\end{proof}
\begin{proof}[\textbf{[別証明]}]
    過程$(P_k(t))_{k\in\N}$の定める確率母関数$g(z,t)=\sum^\infty_{k=0}P_k(t)z^k$が,Poisson分布$\Pois(\lambda t)$の確率母関数$e^{\lambda t(z-1)}$に一致することを示しても良い.
\end{proof}

\subsubsection{正規近似}

\begin{tcolorbox}[colframe=ForestGreen, colback=ForestGreen!10!white,breakable,colbacktitle=ForestGreen!40!white,coltitle=black,fonttitle=\bfseries\sffamily,
title=]
    正規分布よりも裾が重いことが分かる.
\end{tcolorbox}

\begin{theorem}
    $X\sim\Pois(\lambda)$とする.このとき,$\frac{X-\lambda}{\sqrt{\lambda}}\xrightarrow{d}N(0,1)$.
\end{theorem}

\subsection{負の二項分布}

\begin{tcolorbox}[colframe=ForestGreen, colback=ForestGreen!10!white,breakable,colbacktitle=ForestGreen!40!white,coltitle=black,fonttitle=\bfseries\sffamily,
title=]
    
\end{tcolorbox}

\begin{definition}[negative binomial distribution, geometric distribution]\mbox{}
    \begin{enumerate}
        \item 成功の確率が$p$の試行を独立に繰り返すBernoulli試行\ref{exp-Bernoulli-Process}に於て,$k$回成功するまでに必要な失敗の回数$x$が測度空間$(\N,P(\N))$上に定める分布を\textbf{負の二項分布}$\NB(k,p):\Z_+\times[0,1]\to\M(\N)$という.
        その確率関数は$p(x;k,p)=\begin{pmatrix}x+k-1\\x\end{pmatrix}p^kq^x\;(x\in\Z_+)$と表せる.
        \item 特に$k=1$の場合を\textbf{幾何分布}$G(p):=\NB(1,p)$といい,確率関数は$p(x;1,p)=pq^x\;(x\in\Z_+)$と表せる.
        \item 負の二項分布$NB(k,p)$は$k\in(0,\infty)$上に延長できる.
    \end{enumerate}
\end{definition}

\begin{proposition}[負の二項分布の確率母関数・平均・分散]\mbox{}
    \begin{enumerate}
        \item 負の二項分布$\NB(b,p)$の確率母関数は$g(z)=p^k(1-qz)^{-k}=\paren{\frac{p}{1-qz}}^k$で表せる.
        \item 負の二項分布$\NB(b,p)$の平均は$\al_1=\frac{kq}{p}$,分散は$\mu_2=\frac{kq}{p^2}$である.
    \end{enumerate}
\end{proposition}
\begin{proof}\mbox{}
    \begin{enumerate}
        \item $\abs{zq}<1$すなわち$\abs{z}<\frac{1}{q}$のとき,$(1-qz)^{-k}$の$z=0$におけるTaylor展開を考えると
        \begin{align*}
            (1-qz)^{-k}&=1+k(qz)+\frac{k(k+1)}{2!}(qz)^2+\cdots=\sum^\infty_{x=0}\frac{(-k)(-k-1)\cdots(-k-x+1)}{x!}(-qz)^x\\
            &=\sum^\infty_{x=0}\frac{(x+k-1)(x+k-2)\cdots(k+1)k}{x!}(qz)^x\\
            &=\sum^\infty_{x=0}\begin{pmatrix}x+k-1\\x\end{pmatrix}(qz)^x.
        \end{align*}
        $\left.\frac{(-k)(-k-1)\cdots(-k-x+1)}{x!}\right|_{x=0}=1$としたことに注意.
        よって,$\NB(k,p)$の確率母関数は$g(z)=p^k(1-qz)^{-k}$.
        \item 確率母関数と平均・分散の関係\ref{lemma-variance-formula}(3)より,
        \begin{align*}
            g'(z)&=kp^kq(1-qz)^{-k-1},&\mu&=g'(1)=\frac{kp^kq}{p^{k+1}}=\frac{kq}{p},\\
            g''(z)&=k(k+1)p^kq^2(1-qz)^{-k-2},&\sigma^2&=g''(1)+g'(1)-(g'(1))^2\\
            &&&=\frac{k(k+1)q^2}{p^2}+\frac{kq}{p}-\frac{k^2q^2}{p^2}\\
            &&&=\frac{kq(p+q)}{p^2}=\frac{kq}{p^2}.
        \end{align*}
    \end{enumerate}
\end{proof}

\begin{proposition}[Poisson近似]
    平均$\lambda:=\frac{kq}{p}\in(0,\infty)$を一定にして,成功数を表す母数を$k\to\infty$とすると(このとき失敗率$q$は極めて小さくなる),Poisson分布$\Pois(\lambda)$に収束する:$NB(k,p)\to\Pois(\lambda)\;(k\to\infty)$.
\end{proposition}
\begin{proof}
    確率関数が$\wt{p}(x)=\frac{\lambda^x}{x!}e^{-\lambda}$に収束することを見れば良い.
    \begin{align*}
        p(x;k,p)&=\begin{pmatrix}x+k-1\\x\end{pmatrix}p^kq^x\\
        &=\frac{(x+k-1)\cdots(k+1)k}{x!}p^kq^x\\
        &=\frac{(x+k-1)\cdots(k+1)k}{k^x}x!\paren{\frac{kq}{p}}^xp^{k+x}\\
        &\xrightarrow[\lambda=\const]{k\to\infty}1\cdot\frac{\lambda^x}{x!}e^{-\lambda}.
    \end{align*}
    実際,
    \[p^{k+x}=\paren{\frac{p}{p+q}}^{k+x}=\paren{1+\frac{\lambda}{k}}^{-(k+x)}\xrightarrow{k\to\infty}e^{-\lambda}.\]
\end{proof}

\begin{remark}[幾何分布の無記憶性]
    $X\sim G(\theta)$とするとき,
    \[P(X\ge m+n|X\ge m)=P(X\ge n)\quad(m,n\in\N)\]
    が成り立つ.これは,時刻$m$までに成功していないことはその後の成功までの待ち時間の分布に影響しないことを意味しているとみなせる.
\end{remark}

\subsection{Katz族}

\begin{definition}[Katz family]
    $\N$上の離散分布の族であって,$a,b\in\R$を用いた漸化式
    \[p_0>0,\quad p_x=\paren{a+\frac{b}{x}}p_{x-1},\quad x\in\N\]
    を満たす族を\textbf{Katz族}という.
    この漸化式を満たすことを,\textbf{$(a,b,0)$-クラスの分布}であるという.
\end{definition}
\begin{example}\mbox{}
    \begin{enumerate}
        \item $\delta_0:a+b=0,p_0=1$のとき.
        \item $B(n,\theta):a=-\frac{\theta}{1-\theta},\quad b=\frac{(n+1)\theta}{1-\theta},\quad p_0=(1-\theta)^n$.
        \item $\Pois(\lambda):a=0,b=\lambda,p_0=e^{-\lambda}$.
        \item $NB(k,\theta):a=1-\theta,b=(k-1)(1-\theta),p_0=\theta^k$.
    \end{enumerate}
\end{example}

\begin{proposition}[実際の母数はかなり狭い]
    $\N$上の確率分布$p=(p_x)_{x\in\N}$がKatz族であるとする.このとき,$p$は上の例の4つの場合に限られる.
    \begin{enumerate}
        \item $a+b=0$ならば,$p=\delta_0$.
        \item $a=0,b>0$ならば,$p=\Pois(b)$.
        \item $a+b>0,a\in(0,1)$ならば,$p=NB((a+b)/a,1-a)$.
        \item $a+b>0,a\in(-\infty,0)$ならば,$\exists_{n\in\N}\;a(n+1)+b=0,p=B(n,-a/(1-a))$.
    \end{enumerate}
    特に,$p$がデルタ測度$\delta_0$でなければ,$a+b>0$かつ$a<1$であり,確率母関数は
    \[g(z)=\begin{cases}
        \paren{\frac{1-az}{1-a}}^{-\frac{a+b}{a}},&a\ne 0,\\
        \exp(b(z-1)),&a=0.
    \end{cases}\]
    と表せる.
\end{proposition}

\subsection{超幾何分布}

\begin{tcolorbox}[colframe=ForestGreen, colback=ForestGreen!10!white,breakable,colbacktitle=ForestGreen!40!white,coltitle=black,fonttitle=\bfseries\sffamily,
title=]
    二項分布を$2$からの復元抽出だと思えば,これを非復元抽出にするとより複雑な関数形が得られる.
    これは主に個体群生態学で使用される標識再捕獲法などで使われる.
\end{tcolorbox}

\begin{definition}[hypergeometric distribution]
    成功$n$個,失敗$N-n$個が入った多重集合から$r$個玉を取り出したときにそのうち成功が$x$個である確率$p(x;N,n,r)=\frac{\begin{pmatrix}n\\x\end{pmatrix}\begin{pmatrix}N-n\\r-x\end{pmatrix}}{\begin{pmatrix}N\\r\end{pmatrix}}$が,
    $X:=\Brace{x\in\N\mid\max\{0,r-N+n\}\le x\le\min\{r,n\}}$として可測空間$(X,P(X))$上に
    定める確率分布$H(N,n,r)$を\textbf{超幾何分布}という.
\end{definition}

\begin{proposition}\mbox{}
    \begin{enumerate}
        \item 平均は$\al_1=r\frac{n}{N}$.
        \item 分散は$\mu_2=r\paren{\frac{N-r}{N-1}}\paren{\frac{n}{N}}\paren{1-\frac{n}{N}}$.
    \end{enumerate}
\end{proposition}

\begin{example}[mark and recapture methodにおけるLincoln-Peterson推定量]
    個体数$N$を推定するために,$n$匹を捕まえて,標識をつけて再放流する.十分な時間経過後にsamplingし,標識がついているものが$r$匹だった場合,$\wt{N}:=\frac{nr}{x}$によって$N$が推定できる.
    これは,確率$p(x;N.n,r)$が最大になるような$N$の選び方で,最尤法の考え方である.
\end{example}

\subsection{負の超幾何分布}

\subsection{対数分布}

\begin{definition}
    $\N$上の分布・\textbf{対数分布}$\Log(\theta):=(p_x)_{x\in\N}\;(\theta\in(0,1))$とは,確率関数
    \[p_x=\begin{cases}
        0&x=0,\\
        \frac{-1}{\log(1-\theta)}\frac{\theta^x}{x}&x\in\N_{>0},
    \end{cases}\]
    で定まる分布をいう.
\end{definition}

\begin{proposition}
    $\Log(\theta)$の確率母関数は$g(z)=\frac{\log(1-\theta z)}{\log(1-\theta)}$である.
\end{proposition}

\subsection{Ord族}

\begin{definition}[Ord family]
    $\Z$上の分布であって,確率関数が差分方程式
    \[p_x-p_{x-1}=\frac{(a-x)p_x}{(a+b_0)+(b_1-1)x+b_2x(x-1)}\]
    を満たすもの全体を\textbf{Ord族}という.
\end{definition}
\begin{example}
    Katz族と,超幾何分布,負の超幾何分布はOrd族である.
\end{example}

\section{多次元離散分布の例}

\subsection{確率変数の積}

\begin{definition}[marginal distribution, joint / simultaneous distribution]
    $I:=\Brace{1,\cdots,d}$について,確率変数$X_j:\Om\to\X_j\;(j\in I)$の積$X:=(X_j)_{j\in I}:\Om\to\X:=\prod_{j=1}^d\X_j$は$d$次元確率変数である.
    \begin{enumerate}
        \item $J\subsetneq I$について,分布$(p_J(x_j))_{x_j\in\X_J}$を$X_I$の\textbf{周辺分布}と呼ぶ.
        \item これに対して$X$の分布自体のことを,区別して\textbf{結合分布}または\textbf{同時分布}という.
    \end{enumerate}
\end{definition}

\begin{definition}[covariance, correlation coefficient]
    2つの実確率変数$X:\Om\to\X,Y:\Om\to\Y$について,
    \begin{enumerate}
        \item $\Cov[X,Y]:=\E[(X-\E[X])(Y-\E[Y])]$を\textbf{共分散}と呼ぶ.
        \item $\rho(X,Y):=\frac{\Cov[X,Y]}{\sqrt{\Var[X]}\sqrt{\Var[Y]}}$を\textbf{相関係数}と呼ぶ.
    \end{enumerate}
\end{definition}

\begin{lemma}\mbox{}
    \begin{enumerate}
        \item $E[X^2]<\infty,E[Y^2]<\infty$のとき,または,$E[\abs{X}]<\infty,E[\abs{Y}]<\infty$かつ$X,Y$が独立のとき,共分散は存在する.
        \item $E[X^2]<\infty,E[Y^2]<\infty,\Var(X),\Var(Y)>0$のとき,相関係数は存在し,$\Im\rho\subset[-1,1]$を満たす.
    \end{enumerate}
\end{lemma}

\subsection{多項分布}

\begin{tcolorbox}[colframe=ForestGreen, colback=ForestGreen!10!white,breakable,colbacktitle=ForestGreen!40!white,coltitle=black,fonttitle=\bfseries\sffamily,
title=]
    二項分布は多項分布の周辺分布であった.
\end{tcolorbox}

\begin{definition}[multinomial distribution]
    $I=\Brace{E_1,\cdots,E_k}$とし,
    \[T_I:=\Brace{x_I=(x_j)_{j\in I}\in\N^k\;\middle|\;\sum^k_{j=1}x_j=n}\]
    とする.
    1回の試行で$E_1,\cdots,E_k$のいずれかが起こるとし,それぞれの生起確率を$p_1,\cdots,p_k,\sum_{j=1}^kp_j=1$とする.
    このとき,$E_1,\cdots,E_k$が起きた回数が$x_1,\cdots,x_k$回である確率は
    \[f_I(x_I;p)=\frac{n!}{x_1!\cdots x_k!}\prod_{j=1}^kp_j^{x_j}\quad(x_I=(x_j)_{j\in I}\in T_I)\]
    である.これを確率関数とする$T_I$上の離散分布を,パラメータ$(n,p)\in\N\times[0,1]^k$の\textbf{多項分布}$\Mult(n,p)$という.
\end{definition}

\begin{proposition}
    $g_I(z_I;p)=(p_1z_1+\cdots+p_kz_k)^n$が確率母関数である.
\end{proposition}
\begin{proof}
    多項定理より.
\end{proof}

\begin{proposition}
    $\Cov[X_i,X_j]=\begin{cases}
        np_i(1-p_i)&i=j\\
        -np_ip_j&i\ne j
    \end{cases}$
\end{proposition}

\subsection{2変量Poisson分布}

\begin{definition}[bivariate Poisson distribution]
    確率変数$U_1,U_2,U_3$が独立で$U_i\sim\Pois(\lambda_i)$を満たすとする.
    このとき,$X_1=U_1+U_3,X_2=U_2+U_3$で定まる確率変数$(X_1,X_2):\Om\to\N^2$が定める分布を\textbf{2変量Poisson分布}といい,$\BPois(\lambda_1,\lambda_2,\lambda_3)$で表す.周辺分布は$X_1\sim\Pois(\lambda_1+\lambda_3),X_2\sim\Pois(\lambda_2+\lambda_3)$である.
\end{definition}

\begin{proposition}[分散や混合積率は確率母関数の項別微分で求める]\mbox{}
    \begin{enumerate}
        \item 確率母関数は$g(z_1,z_2)=\exp\paren{\lambda_1(z_1-1)+\lambda_2(z_2-1)+\lambda_3(z_1z_2-1)}$.
        \item $\Cov[X_1,X_2]=\lambda_3$.
    \end{enumerate}
\end{proposition}

\subsection{負の多項分布}

\begin{tcolorbox}[colframe=ForestGreen, colback=ForestGreen!10!white,breakable,colbacktitle=ForestGreen!40!white,coltitle=black,fonttitle=\bfseries\sffamily,
title=]
    負の二項分布$NB(k,p)$の確率母関数は,$\hat{q}:=1/p,\hat{p}:=q/p=(1-p)/p$とすると,
    \[g(z)=p^k(1-qz)^{-k}=(\hat{q}-\hat{p}z)^{-k}\]
    と表わせ,この形での一般化を考える.
\end{tcolorbox}

\begin{definition}
    $k>0,P_i>0$をパラメータとする\textbf{負の多項分布}$NM(k,(P_1,\cdots,P_d))$とは,$Q:=1+\sum^d_{i=1}P_i$とするとき,確率母関数
    \[g(z_1,\cdots,z_d)=\paren{Q-\sum^d_{i=1}P_iz_i}^{-k}\]
    が定める$\N^d$上の分布を言う.
\end{definition}

\begin{proposition}[確率母関数の微分からわかること]\mbox{}
    \begin{enumerate}
        \item $\Cov[X_i,X_j]=\begin{cases}
            kP_i(1+P_i)&i=j\\
            kP_iP_j&i\ne j
        \end{cases}$
    \end{enumerate}
\end{proposition}

\section{一次元連続分布の例}

\subsection{一様分布}

\begin{definition}[uniform distribution]
    $(\X,\B_1)$上の\textbf{一様分布}$U(a,b)\;(a<b\in\R)$とは,確率密度関数$f$
    \begin{align*}
        f(x)&:=\frac{1}{b-a}1_{(a,b)}=\begin{cases}
            \frac{1}{b-a},&a<x<b,\\
            0,&\otherwise.
        \end{cases}&\nu[A]&:=\int_Af(x)dx\;(A\in\B_1)
    \end{align*}
    が定める確率分布をいう.
\end{definition}

\begin{proposition}\mbox{}
    \begin{enumerate}
        \item 特性関数は$\varphi(u)=\frac{e^{ibu}-e^{iau}}{(b-a)iu}$.ただし$\varphi(0)=1$とする.
        \item $\al_1=\frac{a+b}{2}$.
        \item $\mu_2=\frac{(b-a)^2}{12}$.
        \item $\gamma_1=0,\gamma_2=\frac{9}{5}$.
    \end{enumerate}
\end{proposition}
\begin{proof}\mbox{}
    \begin{enumerate}
        \item $E[e^{iux}]=\int^b_ae^{iux}\frac{1}{b-a}dx$なので.
    \end{enumerate}
\end{proof}

\subsection{Gamma分布・指数分布・カイ2乗分布}

\subsubsection{Gamma分布の性質}

\begin{definition}[gamma distribution]
    $(\X,\B_1)$上の\textbf{Gamma分布}$G(\al,\nu)\;(\al,\nu\in\R_{>0})$とは,
    確率密度関数
    \[f(x)=g(x;\al,\nu):=\frac{1}{\Gamma(\nu)}\al^\nu x^{\nu-1}e^{-\al x}1_{x>0}\]
    が定める分布をいう.
    ただし,Gamma関数とは,$\Gamma(\nu)=\int^\infty_0t^{\nu-1}e^{-t}dt$.
\end{definition}

\begin{proposition}\mbox{}
    \begin{enumerate}
        \item $\varphi(u)=\frac{1}{(1-\frac{iu}{\al})^\nu}$.
        \item $\al_1=\frac{\nu}{\al}$.
        \item $\mu_2=\frac{\nu}{\al^2}$.
        \item $\gamma_1=\frac{2}{\sqrt{\nu}},\gamma_2=3+\frac{6}{\nu}$.
    \end{enumerate}
\end{proposition}
\begin{proof}\mbox{}
    \begin{enumerate}
        \item \begin{align*}
            \varphi(u)&=\int^\infty_0e^{iux}\frac{1}{\Gamma(\nu)}\al^\nu x^{\nu-1}e^{-\al x}dx\\
            &=\int^\infty_0\frac{1}{\Gamma(\nu)}\al^\nu x^{\nu-1}e^{-\al\paren{1-\frac{iu}{\al}}x}dx
        \end{align*}
        でこの後pathを考えるらしい.
    \end{enumerate}
\end{proof}

\subsubsection{指数分布}

\begin{tcolorbox}[colframe=ForestGreen, colback=ForestGreen!10!white,breakable,colbacktitle=ForestGreen!40!white,coltitle=black,fonttitle=\bfseries\sffamily,
title=]
    カイ2乗分布は線型理論の騎手で,Pearsonによる.
\end{tcolorbox}

\begin{definition}[exponential distribution, chi-square distribution]\mbox{}\label{def-exponential-distribution}
    \begin{enumerate}
        \item $\Exp(\gamma):=G(\gamma,1)$を母数$\gamma$の\textbf{指数分布}という.
        \item $\chi^2(k):=G\paren{\frac{1}{2},\frac{k}{2}}$を自由度$k$の\textbf{カイ2乗分布}という.\footnote{統計推測で重要なクラスである.}
    \end{enumerate}
\end{definition}

\begin{lemma}[確率密度関数]\mbox{}
    \begin{enumerate}
        \item 指数分布の確率密度関数は,
        \[g(x;\gamma,1)=\frac{1}{\Gamma(1)}\gamma^1x^0e^{-\gamma x}1_{x>0}=\frac{\gamma}{e^{\gamma x}}1_{x>0}.\]
        \item 
    \end{enumerate}
\end{lemma}

\begin{lemma}
    指数分布$X_\lambda$の特性関数は
    \[E[e^{iuX_\lambda}]=\frac{\lambda}{\lambda-iu}.\]
\end{lemma}

\subsection{正規分布}

\begin{definition}[normal distribution]
    $(\X,\B_1)$上の\textbf{正規分布}$N(\mu,\sigma^2)\;(\mu\in\R,\sigma>0)$とは,
    確率密度関数
    \[\phi(x;\mu,\sigma^2):=\frac{1}{\sqrt{2\pi\sigma^2}}\exp\paren{-\frac{(x-\mu)^2}{2\sigma^2}}\]
    が定める分布をいう.
\end{definition}

\begin{proposition}[分布の特性値]\mbox{}
    \begin{enumerate}
        \item well-definedness:$\int_\R\phi(x;\mu,\sigma^2)dx=1$.
        \item 特性関数:$\varphi(u)=\exp\paren{i\mu u-\frac{1}{2}\sigma^2u^2}$.
        \item 平均と分散:$\al_1=\mu,\mu_2=\sigma^2$.
        \item 歪度と尖度:$\gamma_1=0,\gamma_2=3$.
        \item 中心積率:$\mu_{2r+1}=0,\quad\mu_{2r}=\frac{(2r)!}{2^rr!}\sigma^{2r}=(2r-1)!!(\sigma^2)^r$.ただし,$(2r-1)!!=(2r-1)(2r-3)\cdots 3\cdot 1$である.
        \item 積率母関数:$\fM(t)=\exp\paren{\mu t+\frac{\sigma^2t^2}{2}}$.
        \item キュムラントは$\kappa_1=\mu,\kappa_2=\sigma^2,\kappa_r=0\;(r\ge 3)$.\footnote{この3次以上のキュムラントが消えることが正規分布の特徴で,積率による中心極限定理の証明に利用される.}
    \end{enumerate}
\end{proposition}
\begin{proof}\mbox{}
    \begin{enumerate}
        \item gamma積分に変換するとわかる.
        \item 
        \item 特性関数より\ref{cor-mean-and-variance-in-terms-of-characteristic-function},
        \begin{align*}
            \al_1&=\frac{1}{i}\varphi'(0)=\frac{1}{i}(i\mu)=\mu,\\
            \mu_2&=-\varphi''(0)+(\varphi'(0))^2=\sigma^2-(i\mu)^2+(i\mu)^2=\sigma^2.
        \end{align*}
    \end{enumerate}
\end{proof}

\begin{definition}[standard normal distribution]
    $N(0,1)$を\textbf{標準正規分布}という.
\end{definition}

\begin{lemma}[scaling]
    \[X\sim N(\mu,\sigma^2)\Lrarrow \frac{X-\mu}{\sigma}\sim N(0,1)\]
\end{lemma}
\begin{proof}
    \begin{align*}
        P\paren{\frac{X-\mu}{\sigma}\le Z}&=P\paren{X\le\mu+\sigma Z\footnote{一種の漸近展開だという.}}\\
        &=\int^{\mu+\sigma Z}_{-\infty}\phi(x;\mu,\sigma^2)dx.
    \end{align*}
\end{proof}
\begin{remark}
    この操作を基準化という.ある種のaffine変換だという.
    ある意味での等質性を表す.
\end{remark}

\subsection{Beta分布}

\begin{definition}[beta distribution]
    $(\X,\B_1)$上の\textbf{(第1種)ベータ分布}$\Beta(\al,\beta)=B_E(\al,\beta)\;(\al,\beta\in\R_{>0})$とは,
    確率密度関数
    \[f(x):=\frac{1}{B(\al,\beta)}x^{\al-1}(1-x)^{\beta-1}1_{(0,1)}(x)\]
    が定める確率分布をいう.ただし,$B(\al,\beta)=\int^1_0x^{\al-1}(1-x)^{\beta-1}dx$.
\end{definition}

\begin{proposition}\mbox{}
    \begin{enumerate}
        \item $\al_1=\frac{\al}{\al+\beta}$.
        \item $\mu_2=\frac{\al\beta}{(\al+\beta)^2(\al+\beta+1)}$.
        \item $\gamma_1=\frac{2(\beta-\al)\sqrt{\al+\beta+1}}{(\al+\beta+2)\sqrt{\al\beta}}$.
        \item $\gamma_2=\frac{3(\al+\beta+1)(\al^2\beta+\al\beta^2+2\al^2-2\al\beta+2\beta^2)}{\al\beta(\al+\beta+2)(\al+\beta+3)}$.
    \end{enumerate}
\end{proposition}
\begin{proof}\mbox{}
    \begin{enumerate}
        \item \begin{align*}
            \al_1&=\int^1_0x\frac{1}{B(\al,\beta)}x^{\al-1}(1-x)^{\beta-1}dx\\
            &=\frac{B(\al+1,\beta)}{B(\al,\beta)}\underbrace{\int^1_0\frac{1}{B(\al+1,\beta)}x^\al(1-x)^{\beta-1}dx}_{=1}\\
            &=\frac{\Gamma(\al+1)\Gamma(\beta)}{\Gamma(\al+1+\beta)}\frac{\Gamma(\al+\beta)}{\Gamma(\al)\Gamma(\beta)}=\frac{\al}{\al+\beta}.
        \end{align*}
    \end{enumerate}
\end{proof}

\begin{remark}
    beta分布は種々の分布が表現できるので便利,Rで遊ぶと良い.
\end{remark}

\subsection{Cauchy分布}

\begin{definition}[Cauuchy distribution]
    $(\X,\B_1)$上の\textbf{Cauchy分布}$C(\mu,\sigma)\;(\mu\in\R,\sigma\in\R_{>0})$とは,
    確率密度関数
    \[f(x)=\frac{1}{\pi\sigma\paren{1+\paren{\frac{x-\mu}{\sigma}}^2}}\]
    が定める分布をいう.
    $\mu$は位置母数,$\sigma$は尺度母数と解釈される.
\end{definition}

\begin{proposition}\mbox{}
    \begin{enumerate}
        \item 裾が重く($1/x^2$のオーダー),平均と分散は存在しない.
        \item $\varphi(u)=\exp(i\mu u-\sigma\abs{u})$.明らかに$u=0$で微分可能でない.
    \end{enumerate}
\end{proposition}

\begin{history}
    Cauchy(1853)によると考えられていたが,Poissonが1824年にすでに注目していた.
\end{history}

\subsection{Weibull分布}

\begin{tcolorbox}[colframe=ForestGreen, colback=ForestGreen!10!white,breakable,colbacktitle=ForestGreen!40!white,coltitle=black,fonttitle=\bfseries\sffamily,
title=]
    Weibull分布は寿命分布として用いられ,また極値分布としても現れる.
\end{tcolorbox}

\begin{definition}[Weibull distribution]
    $(\X,\B_1)$上の\textbf{Weibull分布}$W(\nu,\al)\;(\nu,\al\in\R_{>0})$とは,
    確率密度関数
    \[f(x)=\frac{\nu}{\al}\paren{\frac{x}{\al}}^{\nu-1}\exp\paren{-\paren{\frac{x}{\al}}^\nu}1_{x>0}=\int^x_{-\infty}f(y)dy\]
    が定める分布である.
    $\al$が尺度母数,$\nu$が形状母数と解釈される.
\end{definition}

\begin{proposition}\mbox{}
    \begin{enumerate}
        \item 分布関数は$F(x)=\Brace{1-\exp\paren{-\paren{\frac{x}{\al}}^\nu}}1_{x>0}$.
        \item $\al_1=\al\Gamma\paren{\frac{\nu+1}{\nu}}$.
        \item $\mu_2=\al^2\Brace{\Gamma\paren{\frac{\nu+2}{\nu}}-\Gamma\paren{\frac{\nu+1}{\nu}}^2}$.
    \end{enumerate}
\end{proposition}

\begin{proposition}
    $X\sim\Exp\paren{\frac{1}{\al^\nu}}$ならば,$X^{\frac{1}{\nu}}\sim W(\nu,\al)$.
\end{proposition}

\subsection{対数正規分布}

\begin{definition}[log-normal distribution]
    $(\X,\B_1)$上の\textbf{対数正規分布}$L_N(\nu,\sigma)\;(\nu\in\R,\sigma\in\R_{>0})$とは,
    確率密度関数
    \[f(x)=\frac{1}{\sqrt{2\pi\sigma^2}}\frac{1}{x}\exp\Brace{-\frac{(\log x-\mu)^2}{2\sigma^2}}1_{x>0}\]
    が定める分布である.
\end{definition}

\begin{proposition}
    \[\log X\sim N(\mu,\sigma^2)\Lrarrow X\sim L_N(\mu,\sigma)\]
\end{proposition}

\begin{proposition}\mbox{}
    \begin{enumerate}
        \item $\al_1=\exp\paren{\mu+\frac{\sigma^2}{2}}$.
        \item $\mu_2=\paren{e^{\sigma^2}-1}\exp(2\mu+\sigma^2)$.
        \item $\gamma_1=(\om+2)\sqrt{\om-1}$.ただし,$\om:=\exp(\sigma^2)$とした.
        \item $\gamma_2=\om^4+2\om^3+3\om^2-3$.
    \end{enumerate}
\end{proposition}

\subsection{logistic分布}

\begin{definition}[logistic distribution]
    $(\X,\B_1)$上の\textbf{ロジスティック分布}$\Log(\mu,\sigma)\;(\mu\in\R,\sigma\in\R_{>0})$とは,
    分布関数
    \[F(x)=\frac{1}{1+\exp\paren{-\frac{x-\mu}{\sigma}}}\]
    で,確率密度関数
    \[f(x)=\frac{1}{\sigma}\frac{\exp\paren{\frac{x-\mu}{\sigma}}}{\Brace{1+\exp\paren{\frac{x-\mu}{\sigma}}}^2}\]
    が与える分布をいう.よって,分布は$x=\mu$に関して対称である.
\end{definition}

\begin{lemma}
    \[I(r):=\int^\infty_0x^r\frac{e^{-x}}{(1+e^{-x})^2}dx=(1-2^{-(r-1)})\Gamma(r+1)\zeta(r)\quad(r>1)\]
\end{lemma}
\begin{proof}
    \begin{align*}
        I(r)&=\int^\infty_0x^r\frac{e^{-x}}{(1+e^{-x})^2}dx=\int^\infty_0\frac{rx^{r-1}}{1+e^x}dx\\
        &=\int^\infty_0rx^{r-1}e^{-x}\sum^\infty_{j=0}(-1)^je^{-jx}dx\\
        &=\Gamma(r+1)\sum^\infty_{n=1}(-1)^{n-1}\frac{1}{n^r}\quad(r>0)
    \end{align*}
    ここで,$\zeta(r)=\sum^\infty_{n=1}\frac{1}{n^r}\;(r>1)$について
    \[(1-2^{-(r-1)})\zeta(r)=\sum^\infty_{n=1}(-1)^{n-1}n^{-r}\]
    が成り立つから,最後の変形を得る.
\end{proof}

\begin{proposition}\mbox{}
    \begin{enumerate}
        \item 中心積率は,奇数について$\mu_r=0$,2以上の偶数について
        \[\mu_r=2\sigma^r\Gamma(r+1)\sum^\infty_{n=1}(-1)^{n-1}\frac{1}{n^r}=2\sigma^r(1-2^{-(r-1)})\Gamma(r+1)\zeta(r).\]
        \item $\al_1=\mu,\m_2=\frac{\pi^2\sigma^2}{3},\mu_4=\frac{7\pi^4\sigma^4}{15}$.
        \item $\gamma_1=0,\gamma_2=4.2$.
        \item $\varphi(u)=e^{i\mu u}\frac{\pi\sigma u}{\sinh(\pi\sigma u)}$.
    \end{enumerate}
\end{proposition}
\begin{proof}\mbox{}
    \begin{enumerate}
        \item 奇数の時は対称性より.
        \item $\zeta(2)=\frac{\pi^2}{6},\zeta(4)=\frac{\pi^4}{90}$より.
        \item $-R,R,-R+2\pi i,R+2\pi i$を頂点にもつ長方形についての留数計算より.
    \end{enumerate}
\end{proof}

\subsection{Pareto分布}

\begin{tcolorbox}[colframe=ForestGreen, colback=ForestGreen!10!white,breakable,colbacktitle=ForestGreen!40!white,coltitle=black,fonttitle=\bfseries\sffamily,
title=]
    パレート分布は所得の分布に当てはまるという.
\end{tcolorbox}

\begin{definition}
    $(\X,\B_1)$上の\textbf{パレート分布}$P_A(b,a)\;(a,b\in\R_{>0})$とは,
    分布関数を
    \[F(x)=1-\paren{\frac{b}{x}}^a1_{x\ge b}\]
    確率密度関数を
    \[f(x):=ab^ax^{-(a+1)}1_{x\ge b}\]
    とする分布をいう.
\end{definition}

\begin{proposition}\mbox{}
    \begin{enumerate}
        \item $\al_1=ab(a-1)^{-1}\;(a>1)$.
        \item $\mu_2=ab^2(a-1)^{-2}(a-2)^{-1}\;(a>2)$.
        \item $\gamma_1=2\frac{a+1}{a-3}\sqrt{\frac{a-2}{a}}\;(a>3)$.
        \item $\gamma_2=\frac{3(a-2)(3a^2+a+2)}{a(a-3)(a-4)}\;(a>4)$.
    \end{enumerate}
\end{proposition}

\subsection{逆正規分布}

\begin{tcolorbox}[colframe=ForestGreen, colback=ForestGreen!10!white,breakable,colbacktitle=ForestGreen!40!white,coltitle=black,fonttitle=\bfseries\sffamily,
title=]
    ドリフト付きのBrown運動の到達時間の分布として現れる.
    キュムラント母関数$\log M$が,正規分布のキュムラント母関数の逆数になっていることから.
\end{tcolorbox}

\begin{definition}[inverse Gaussian distribution / Wald distribution]
    確率密度関数
    \[f(x;\delta,\gamma)=1_{(0,\infty)}(x)\frac{\delta e^{\gamma\delta}}{\sqrt{2\pi}}x^{-3/2}\exp\paren{-\frac{1}{2}\paren{\gamma^2x+\frac{\delta^2}{x}}}\]
    で定まる確率分布$\IG(\delta,\gamma):\R_{>0}\times\R_{\ge0}\to P(\R)$を\textbf{逆正規分布}または\textbf{Wald分布}という.
\end{definition}

\begin{lemma}\mbox{}
    \begin{enumerate}
        \item $\gamma>0$のとき,確率密度関数は
        \[f(x)=1_{(0,\infty)}(x)\paren{\frac{\delta^2}{2\pi}}^{1/2}x^{-3/2}\exp\paren{-\frac{\delta^2(x-\delta\gamma^{-1})^2}{2(\delta\gamma^{-1})^2x}}\]
        とも表せる.
        \item $\gamma=0$のとき,確率密度関数は,$c:=\delta^2$と定めて,
        \[f(x;c)=1_{(0,\infty)}(x)\sqrt{\frac{c}{2\pi}}x^{-3/2}\exp\paren{-\frac{c}{2x}}\]
        と表せる.これが定める分布を\textbf{Levy分布}といい,$\Levy(c):=\IG(c^{1/2},0)$と表す.
        \item $\IG(\delta,\gamma)$の積率母関数は,
        \begin{align*}
            M(s)&=\int^\infty_0e^{sx}f(x)dx\\
            &=\exp\paren{\delta\paren{\gamma-\sqrt{\gamma^2-2s}}}&s\le\frac{\gamma^2}{2},\gamma\ge0\\
            &=\exp\paren{\gamma\delta\paren{1-\sqrt{1-\frac{2s}{\gamma^2}}}}&s\le\frac{\gamma^2}{2},\gamma>0.
        \end{align*}
        \item $\IG(\delta,\gamma)$の特性関数は$\varphi(u)=\exp\paren{\gamma\delta\paren{1-\sqrt{1-\frac{2iu}{\gamma^2}}}}$となる.
        \item キュムラント母関数は$\log\varphi(u)=\frac{\delta}{\gamma}iu+\sum^\infty_{r=2}(2r-3)(2r-5)\cdots 1\cdot\frac{\delta}{\gamma^{2r-1}}\cdot\frac{(iu)^r}{r!}$で,キュムラントは
        \begin{align*}
            \kappa_1&=\al_1=\frac{\delta}{\gamma},&\kappa_r=(2r-3)(2r-5)\cdots 1\cdot\frac{\delta}{\gamma^{2r-1}}
        \end{align*}
        \item Levy分布$\Levy(c)$のLaplace変換は
        \[\int^\infty_0e^{-\lambda x}f(x)dx=e^{-\sqrt{c\lambda}}\quad(\lambda\in\R_+)\]
        である.
    \end{enumerate}
\end{lemma}

\subsection{逆Gamma分布}

\begin{proposition}
    $X\sim\Gamma(n/2,n/2)\;(n\in\N)$とし,確率変数$Y:=1/X$を考えると,
    この確率密度関数は
    \begin{align*}
        f_Y(y)&=f_X(1/y)y^{-2}\\
        &=\frac{(n/2)^{n/2}}{\Gamma(n/2)}(1/y)^{n/2-1}\exp\paren{-\frac{n/2}{y}}y^{-2}\\
        &=\frac{(n/2)^{n/2}}{\Gamma(n/2)}y^{-n/2-1}\exp\paren{-\frac{n/2}{y}}&y>0
    \end{align*}
    と表せる.
\end{proposition}
\begin{proof}
    $y>0$のとき$P[Y\le y]=P[X\ge 1/y]$である.
\end{proof}

\subsection{Laplace分布}

\begin{definition}
    密度関数
    \[f(x)=\frac{1}{2}e^{-\abs{x}}\]
    が定める分布を\textbf{標準Laplace分布(両側指数分布)}という.
\end{definition}

\begin{lemma}
    \begin{enumerate}
        \item 特性関数は$\varphi(u)=(1+u^2)^{-1}$.
        \item $X,Y,X',Y'$が独立に標準正規分布$N(0,1)$に従うとする.このとき,$V:=XY+X'Y'$は密度$f$を持つLaplace分布に従う.
    \end{enumerate}
\end{lemma}

\subsection{Marcenko-Pastur分布}

\begin{tcolorbox}[colframe=ForestGreen, colback=ForestGreen!10!white,breakable,colbacktitle=ForestGreen!40!white,coltitle=black,fonttitle=\bfseries\sffamily,
title=]
    $\MP(\lambda,1)$分布はWishart行列の固有値の標本スペクトル分布の漸近分布として導出された.
\end{tcolorbox}

\begin{definition}
    パラメータ$(\lambda,\sigma^2)\in\R^2_{>0}$が定める定数$a:=\sigma^2(1-\sqrt{\lambda})^2,b:=\sigma^2(1+\sqrt{\lambda})^2$について,
    確率密度関数
    \[f(x;\lambda,\sigma^2):=\frac{1}{2\pi\sigma^2}\frac{\sqrt{(b-x)(x-a)}}{\lambda x}1_{[a,b]}(x)+1_{[1,\infty)}(\lambda)\paren{1-\frac{1}{\lambda}}\delta_0(x)\]
    が定める分布$\MP(\lambda,\sigma^2):\R_{>0}^2\to P(\R)$を\textbf{Marcenko-Pastur分布}という.
\end{definition}

\begin{lemma}
    $X\sim\MP(\lambda,\sigma^2)$ならば,$\sigma^{-2}X\sim\MP(\lambda):=\MP(\lambda,1)$.
\end{lemma}

\begin{proposition}
    $X_{ij}\sim\iid N(0,1)\;(i\in[d],j\in[n])$を成分とする$d\times n$ランダム行列$X$に対し,$n^{-1}XX^\perp$の固有値を$0\le\lambda_1\le\cdots\le\lambda_d$とする.
    \begin{enumerate}
        \item 標本スペクトル分布は$\mu_{d,n}(dx):=\frac{1}{d}\sum^d_{i=1}\delta_{\lambda_i}(dx)$で与えられる.
        \item $d=d_n$が$\frac{d_n}{n}\to\lambda\in(0,\infty)$を満たすとき,$\mu_{d_n,n}\xrightarrow{d}\MP(\lambda)$を満たす.
    \end{enumerate}
\end{proposition}

\subsection{Pearson系}

\begin{definition}
    $\R$上の確率密度関数$p$であって,微分方程式
    \[\dd{}{x}\log p(x)=\frac{(x-\lambda)-a}{b_2(x-\lambda)^2+b_1(x-\lambda)+b_0}\]
    を満たす
    \[p(x)=C\exp\paren{\int\frac{(x-\lambda)-a}{b_0+b_1(x-\lambda)+b_2(x-\lambda)^2}dx}\]
    の形をした分布を\textbf{Pearson系}という.
    以降,$\lambda=0$とする.
    \begin{enumerate}
        \item I型分布とは,$p(x)=C\paren{1-\frac{x}{a_1}}^{m_1}\paren{1-\frac{x}{a_2}}^{m_2}\;(x\in(a_1,a_2))$の形のものをいう.ただし,$a_1<0<a_2,m_1,m_2>-1$.
    \end{enumerate}
\end{definition}
\begin{example}\mbox{}
    \begin{enumerate}
        \item Beta分布はI型Pearson分布である.
        \item 
    \end{enumerate}
\end{example}

\subsection{Poisson分布確率変数の和差}

\begin{definition}\label{def-Hermite-and-Skellam}
    独立な$Y\sim\Pois(\al),Z\sim\Pois(\beta)$に対して,
    \begin{enumerate}
        \item $X=Y+2Z$の分布を,パラメータ$(\al,\beta)\in\R_{>0}^2$の\textbf{Hermite分布}とよび,$\Hermite(\al,\beta)$で表す.
        \item $X=Y-Z$の分布を,パラメータ$(\al,\beta)\in\R_{>0}^2$の\textbf{Skellam分布}とよび,$\Skellam(\al,\beta)$で表す.
    \end{enumerate}
\end{definition}

\begin{lemma}\mbox{}
    \begin{enumerate}
        \item Hermite分布の特性関数は,$\varphi(u)=\exp\paren{\al(e^{iu}-1)+\beta(e^{2iu}-1)}$.
        \item Hermite分布の確率母関数は$g(z)=\exp\paren{\al(z-1)+\beta(z^2-1)}$.
        \item Skellam関数の特性関数は$\varphi(u)=\exp\paren{\al(e^{iu}-1)+\beta(e^{-iu}-1)}$.
    \end{enumerate}
\end{lemma}


\section{多次元連続分布の例}

\begin{tcolorbox}[colframe=ForestGreen, colback=ForestGreen!10!white,breakable,colbacktitle=ForestGreen!40!white,coltitle=black,fonttitle=\bfseries\sffamily,
title=]
    1次元の絶対連続分布の概念を容易に$d$次元に拡張できる.
\end{tcolorbox}

\subsection{多次元の確率密度関数}

\begin{tcolorbox}[colframe=ForestGreen, colback=ForestGreen!10!white,breakable,colbacktitle=ForestGreen!40!white,coltitle=black,fonttitle=\bfseries\sffamily,
title=]
    Radon-Nikodymの定理より,$\R^d$上の絶対連続分布と確率密度関数は一対一対応する.
\end{tcolorbox}

\begin{notation}
    $\R^d$の矩形$\prod^d_{i=1}(a_i,b_i]$の全体を$\E^d$と表す.
\end{notation}

\begin{definition}
    $\R^d$上の確率分布$\nu$が\textbf{絶対連続分布}であるとは,積分可能な関数$f:\R\to\R_+$が存在して
    \[\forall_{A\in\E^d}\quad\nu(A)=\int_Af(x)dx\]
    と表せることをいう.$f$を$\nu$の\textbf{確率密度関数}という.
\end{definition}

\subsection{確率密度関数と独立性}

\begin{tcolorbox}[colframe=ForestGreen, colback=ForestGreen!10!white,breakable,colbacktitle=ForestGreen!40!white,coltitle=black,fonttitle=\bfseries\sffamily,
title=]
    独立性と特性関数について成り立つKacの定理\ref{thm-Kac}と同様の状況が,確率密度関数についても成り立つ.
\end{tcolorbox}

\begin{proposition}[独立性の確率密度関数による特徴付け]
    $P^X$が確率密度関数$p$を持つとする.このとき,次の2条件は同値.
    \begin{enumerate}
        \item $X_1,\cdots,X_n$は独立.
        \item $p=\bigotimes^n_{i=1}p_i\;\ae$.
    \end{enumerate}
    ただし,$p_i$は$X_i$の周辺密度関数とした.
\end{proposition}

\subsection{畳み込み}

\begin{tcolorbox}[colframe=ForestGreen, colback=ForestGreen!10!white,breakable,colbacktitle=ForestGreen!40!white,coltitle=black,fonttitle=\bfseries\sffamily,
title=]
    絶対連続分布の畳み込みは,その確率密度関数の畳み込みに対応する.
\end{tcolorbox}

\subsection{変数変換と確率密度関数}

\begin{example}[独立正規分布の比はCauchy分布]
    $Y_1,Y_2\sim N(0,1)$を独立とする.$X:=\frac{Y_1}{Y_2}$はCauchy分布にしたがう.
\end{example}

\begin{example}[Box-Muller transform]
    $Y_1,Y_2\sim U(0,1)$を独立とする.
    \[\begin{cases}
        X_1=\sqrt{-2\log Y_1}\cos(2\pi Y_2)\\
        X_2=\sqrt{-2\log Y_2}\sin(2\pi Y_2)
    \end{cases}\]
    と定めると,これは独立な標準正規確率変数となる.
\end{example}

\subsection{多変量正規分布}

\subsubsection{定義と特性値}

\begin{definition}
    $\mu\in\R^d$,$\Sigma\in\GL_d(\R)$を正定値実対称行列として,$(\R^d,\B(\R^d))$上の測度
    \[\mu_{\mu,\Sigma}(dx)=\phi(x;\mu,\Sigma)=\frac{1}{(2\pi)^{d/2}(\det\Sigma)^{1/2}}\exp\paren{-\frac{1}{2}(x-m)^\perp\cdot\Sigma^{-1}(x-m)}dx\]
    を,平均値$\mu$,共分散行列$\Sigma$に関する\textbf{$d$次元正規分布}といい,$N_d(0,\Sigma)$で表す.
\end{definition}

\begin{proposition}
    $d$次元正規分布$N_d(0,\Sigma)$の特性関数は
    \[\varphi(u)=\exp\paren{i\mu\cdot u-\frac{1}{2}u^\top\Sigma u}\]
\end{proposition}

\begin{corollary}
    キュムラント母関数は$\psi(u)=\log\varphi(u)=i\mu\cdot u-\frac{1}{2}u^\top\Sigma u$である.
    特に,平均ベクトルは$\mu$,分散共分散行列が$\Sigma$で,3次以上のキュムラントはすべて消えている.
\end{corollary}

\subsubsection{成分間の独立性の特徴付け}

\begin{proposition}[多次元正規確率変数の成分間の独立性]
    $Y_1:=(X_1,\cdots,X_{k_1})^\top,\cdots,Y_l=(X_{k_{l-1}},\cdots,X_d)^\top\;(l\ge 2)$のように,$X$を$l$個の確率変数$Y_1,\cdots,Y_l$に分ける.
    この分割に対して,$\Sigma$のブロック$\Sigma_{a,b}:=\Cov[Y_a,Y_b]\;(a,b\in[l])$を考える.
    \begin{enumerate}
        \item $Y_1,\cdots,Y_l$は独立.
        \item $\forall_{a,b\in[l]}\;a\ne b\Rightarrow\Sigma_{a,b}=O$.
    \end{enumerate}
\end{proposition}

\subsubsection{半正定値行列への拡張}

\begin{tcolorbox}[colframe=ForestGreen, colback=ForestGreen!10!white,breakable,colbacktitle=ForestGreen!40!white,coltitle=black,fonttitle=\bfseries\sffamily,
title=]
    任意の半正定値行列は分散共分散行列とみなすことが出来る.だから,それに対応した正規分布が存在するはずである.
    しかし,確率密度行列の言葉では,$\det\Sigma=0$になる関係上,定義できない.実際,確率密度行列は連続にはならない.
    そこで,特性関数を用いて定義する.
\end{tcolorbox}

\begin{definition}
    $\Sigma\in M_d(\R)$を半正定値行列とする(特に,非退化の可能性がある).
    列$(\Sigma_n)$を$\Sigma_n:=\Sigma+n^{-1}I_n$で定めるとこれは正定値行列の列で,$\Sigma_n\xrightarrow{n\to\infty}\Sigma$.
    対応する正規分布$\nu_n:=N_d(\mu,\Sigma_n)$の特性関数の列は
    \[\varphi_{\nu_n}(u)=\exp\paren{iu^\top\mu-\frac{1}{2}u^\top\Sigma_nu}\xrightarrow{n\to\infty}\varphi(u)=\exp\paren{iu^\top\mu-\frac{1}{2}u^\top\Sigma u}\]
    と各点収束し,原点で連続である.
    よって,Bochnerの定理\ref{thm-Bochner}より,これはある確率分布$\nu$の特性関数である.この分布を,$d$変量正規分布$N_d(\mu,\Sigma)$と呼ぶ.
\end{definition}

\subsubsection{正規分布に従うことの特徴付け}

\begin{proposition}
    $X:\Om\to\R^d$を確率変数,$\mu\in\R^d$を平均ベクトル,$\Sigma\in M_d(\R)$を半正定値行列とする.
    このとき,次の3条件は同値.
    \begin{enumerate}
        \item $X\sim N_d(\mu,\Sigma)$.
        \item $\forall_{u\in\R^d}\;u^\top X\sim N(u^\top\mu,u^\top\Sigma u)$.
        \item $\forall_{A\in M_{nd}(\R),b\in\R^n}\;AX+b\sim N_n(A\mu+b,A\Sigma A^\top)$.
    \end{enumerate}
\end{proposition}

\begin{corollary}[ユニタリ変換は成分の独立性を変えない]
    $X\sim N_d(\mu,\sigma^2I_d),U\in M_d(\R)$を直交行列とする.このとき,$UX\sim N_d(U\mu,\sigma^2I_d)$である.
    特に,$UX$の成分は再び独立である.
\end{corollary}

\subsection{Dirichlet分布}

\begin{tcolorbox}[colframe=ForestGreen, colback=ForestGreen!10!white,breakable,colbacktitle=ForestGreen!40!white,coltitle=black,fonttitle=\bfseries\sffamily,
title=]
    ベータ分布の高次元化である.
\end{tcolorbox}

\begin{definition}
    $\Delta_p:=\Brace{(x_1,\cdots,x_p)\in\R_{>0}^p\mid\sum^p_{i=1}x_i<1}\;(p\in\N)$について,
    \[f(x_1,\cdots,x_p)=\frac{\Gamma(\nu)}{\prod^{p+1}_{i=1}\Gamma(\nu_i)}\paren{\prod^d_{i=1}x_i^{\nu_i-1}}\paren{1-\sum^p_{i=1}x_i}^{\nu_{p+1}-1}1_{\Delta_p}(x_1,\cdots,x_p)\qquad\nu_i>0,\nu:=\sum^{p+1}_{i=1}\nu_i\]
    が定める分布を\textbf{Dirichlet分布}$\Dirichlet(\nu_1,\cdots,\nu_{p+1}):\R_{>0}^{p+1}\to P(\R^p)$という.
\end{definition}

\begin{lemma}
    $(X_1,\cdots,X_p)\sim\Dirichlet(\nu_1,\cdots,\nu_{p+1})$とする.
    \[\forall_{j\in[p]}\;X_j\sim\Beta\paren{\nu_j,\sum_{i\ne j}\nu_i}\]
\end{lemma}

\section{指数型分布族}

\begin{tcolorbox}[colframe=ForestGreen, colback=ForestGreen!10!white,breakable,colbacktitle=ForestGreen!40!white,coltitle=black,fonttitle=\bfseries\sffamily,
title=]
    ここから,分布の性質の議論から,分布族の性質を考えることへ,視点を移し,「確率分布へのパラメータの入り方」を議論する.
\end{tcolorbox}

\subsection{定義と例}

\begin{tcolorbox}[colframe=ForestGreen, colback=ForestGreen!10!white,breakable,colbacktitle=ForestGreen!40!white,coltitle=black,fonttitle=\bfseries\sffamily,
title=]
    指数型分布族は,有限個の$x$の関数$T_i\in\Meas(\X,\R)$の1次式の指数関数の$\Meas(\X,\R)$-倍で表せるRadon-Nikodym微分を持つクラスである.
    このクラスに対しては,
\end{tcolorbox}

\begin{definition}
    $\P=\{P_\theta\}_{\theta\in\Theta}\subset P(\X)$を参照測度$\nu$に対する絶対連続確率分布の族とし,$p_\theta$をそのRadon-Nikodym微分とする.
    \[p_\theta(x)=g(x)\exp\paren{\sum^m_{i=1}a_i(\theta)T_i(x)-\psi(\theta)}\]
    と表せるとき,$\P$を\textbf{指数型分布族}という.
\end{definition}

\subsection{標準指数型分布族}

\begin{definition}[密度関数による指数型分布族の生成]
    参照測度$\nu$に関する密度関数$g$を考える:$g\ge0,\int g(x)\nu(dx)=1$.
    \[\Theta_0(g):=\Brace{\theta\in\R^p\;\middle|\;\int e^{\theta\cdot x}g(x)\nu(dx)<\infty}\]
    とおくと,$\{0\}\in\Theta_0(g)\ne\emptyset$である.
    $g$のキュムラント母関数を$\psi(\theta):=\log\paren{\int e^{\theta\cdot x}g(x)\nu(dx)}$と表すとき,密度関数
    \[f(x|\theta):=g(x)\exp\paren{x\cdot\theta-\psi(\theta)}\quad\theta\in\Theta_0(g)\]
    を,\textbf{$g(x)\nu(dx)$で生成される$p$次の自然指数型分布族}という.
\end{definition}

\begin{lemma}
    $\Theta_0(g)$は凸集合である.これを\textbf{自然パラメータ空間}という.
\end{lemma}

\subsection{共役事前分布}

\begin{tcolorbox}[colframe=ForestGreen, colback=ForestGreen!10!white,breakable,colbacktitle=ForestGreen!40!white,coltitle=black,fonttitle=\bfseries\sffamily,
title=]
    Howard RaiffaとRobert Schlaiferによるベイジアン決定理論で作られた概念である.
    共役というのは,事後分布が,事前分布の代数的な閉式で表せることを指している(したがって,数値積分が必要なく,解析的に計算可能).
\end{tcolorbox}

\begin{theorem}[Bayesの定理(密度版)]
    確率変数$(\theta,X)$は,参照測度$\mu(dx)\nu(d\theta)$に関する同時密度関数$f(x,\theta)\mu(dx)\nu(d\theta)$をもつとし,
    $f_X(x),\pi(\theta),f(x|\theta)$をそれぞれ,$X,\theta$の周辺密度と,$\theta$を与えた下での$x$の条件付き密度とする.このとき,
    $x$を与えた下での$\theta$の条件付き密度は,
    \[\forall_{x\in\X}\;f_X(x)>0\Rightarrow f(\theta|x)=\frac{\pi(\theta)f(x|\theta)}{\int_\Theta\pi(\theta)f(x|\theta)\nu(d\theta)}\quad\nu\text{-}\ae\]
    と表せる.このことを,$f(\theta|x)\propto\pi(\theta)f(x|\theta)$と表す.
\end{theorem}

\begin{definition}[conjugate prior]
    事後分布$g(-|x)$が事前分布$\pi(-)$と同じ分布型であるとき,$\pi$を$\theta$の\textbf{共役事前分布}といい,$f(x|\theta)$をその尤度関数という.
\end{definition}

\section{指数分散モデル}

\subsection{構成}

\begin{notation}
    一次元確率分布$\mu(dx)$が生成する1次元自然指数型分布族
    \[\exp\paren{x\theta-\psi(\theta)}\mu(dx),\quad\theta\in\Theta_0\]
    を考える.ただし,$\psi$は$\mu$の(第2)キュムラント母関数とした.
    \[\Theta_0:=\Brace{\theta\in\R\mid\int e^{\theta x}\nu(dx)<\infty}\]
    は$\Theta_0^\circ\ne\emptyset$を満たすとする.$\mu$の積率母関数$\fM$が存在するとし,集合
    \[\Lambda:=\Brace{\lambda>0\mid\fM(\theta)^\lambda\text{はある確率分布}\mu_\lambda\text{の積率母関数}}\]
    を考える.
\end{notation}

\begin{lemma}[exponential dispersion model]\mbox{}
    \begin{enumerate}
        \item 任意の$\lambda\in\Lambda$について,$\lambda\psi(\theta)$はある分布$\mu_\lambda$のキュムラント母関数であり,$\exp(x\theta-\lambda\psi(\theta))\mu_\lambda(dx)\;(\theta\in\Theta_0)$は分布を定める.
        \item (1)の分布に従う確率変数$X$について,$Y:=\lambda^{-1}X$の分布は$\wt{\mu}_\lambda:=(\lambda^{-1})_*\mu_\lambda$について次を満たす:
        \[\forall_{B\in\B^1}\quad P^Y(B)=\int_B\exp(\lambda(y\theta-\psi(\theta)))\wt{\mu}_\lambda(dy),\quad\theta\in\Theta_0,\lambda\in\Lambda.\]
    \end{enumerate}
    (2)が定める分布$(P^Y)_{\lambda\in\Lambda}$を$\mu$または$\psi$が定める\textbf{指数分散モデル}といい,$\EDM(\theta,\lambda)$で表す.
\end{lemma}

\begin{remark}
    定義から$\N\subset\Lambda$である.
\end{remark}

\begin{lemma}
    $Y\sim\EDM(\theta,\lambda)$とする.
    \begin{enumerate}
        \item $Y$のキュムラント母関数は$\kappa_Y(u)=\lambda\paren{\psi\paren{\theta+\frac{u}{\lambda}}-\psi(\theta)}$である.特に,$E[Y]=\psi'(\theta)$,$\Var[Y]=\lambda^{-1}\psi''(\theta)$.
        \item 平均値関数を$\tau(\theta):=\psi'(\theta):\Theta^\circ\to\R$で,その値域を$D:=\tau(\Theta^\circ)$で表す.このとき,$\tau$は単射である.
        \item $V(\mu):=\tau'(\tau^{-1}(\mu)):D\to\R_+$について,$\Var[Y]=\lambda^{-1}V(\mu)$と表せる.
    \end{enumerate}
    この$V(\mu)$を\textbf{単位分散関数}という.
\end{lemma}
\begin{remark}
    $\lambda=1$でない限り,$V(\mu)$そのものは分散ではない.これより,$\EDM(\theta,\lambda)$を$\EDM(\mu,\lambda)$で表す.
\end{remark}

\begin{lemma}
    定数$w_i\ge0,w:=\sum^n_{i=1}w_i>0$と,独立確率変数列$Y_i\sim\EDM(\mu,\lambda w_i)$について,
    \[\frac{1}{w}\sum^n_{i=1}w_iY_i\sim\EDM(\mu,\lambda w)\]
\end{lemma}

\subsection{Tweedie分布族}

\begin{definition}
    $\EDM(\mu,\lambda)$のうち,単位分散関数が$V(\mu)=\mu^p\;(p\in\R\setminus(0,1))$と表せる分布族を,\textbf{Tweedie分布族}といい,記号$\Tw_p(\mu,\lambda)$で表す.
\end{definition}
\begin{example}
    $p=0$ならば正規分布族,$p=1$ならばPoisson分布,$p=2$ならばGamma分布,$p=3$ならば逆正規分布となる.
\end{example}

\begin{proposition}
    $\EDM(\mu,\lambda)$について,$1\in D,V(1)=1$とし,さらにある関数$s:(0,\infty)\times\Lambda^{-1}\to\Lambda^{-1}$で以下を満たすものが存在すると仮定する:$Y\sim\EDM(\mu,\lambda)\Rightarrow[\forall_{c>0}\;cY\sim\EDM(c\mu,1/s(c,\lambda))]$.
    このとき,以下が成り立つ.
    \begin{enumerate}
        \item $\exists_{p\in\R}\;Y\sim\Tw_p(\mu,\lambda)$.
        \item $s(c,\lambda)=c^{2-p}/\lambda$.すなわち,$c\Tw_p(\mu,\lambda)=\Tw_p(c\mu,\lambda/c^{2-p})$.
        \item $p=0$ならば$D=\R$.また,$p\ne0$ならば$D=(0,\infty)$.特に,$\L(Y)$は無限分解可能である.
    \end{enumerate}
\end{proposition}

\begin{thebibliography}{99}
    \bibitem{Voevodsky}
    Vladimir Voevodsky "Notes on categorical probability"
    \bibitem{Billingsley}
    "Convergence of Probability Measures"
    \bibitem{Bogachev}
    "Weak Convergence of Measures"
    \bibitem{伊藤清}
    伊藤清『確率論』
    \bibitem{Kolmogorov}
    Kolmogorov 『確率論の基礎概念』
    \bibitem{吉田}
    吉田朋広『数理統計学』(朝倉書店,2006)
    \bibitem{竹村}
    竹村彰道『現代数理統計学』(学術図書,2020)
    \bibitem{久保川}
    久保川達也『現代数理統計学の基礎』(共立出版,2017)
    \bibitem{西山陽一}
    西山陽一『マルチンゲール理論による統計解析』(近代科学社,2011)
    \bibitem{丸山徹}
    丸山徹 - 確率測度の*弱収束\url{https://core.ac.uk/download/pdf/145720102.pdf}
    \bibitem{清水良一}
    清水良一.(1976).中心極限定理.
    \bibitem{Adams}
    Adams, W. J. (1974). \textit{The Life and Times of the  Central Limit Theorem}.
    \bibitem{Kolmogorov}
    Gnedenko, B. V., and Kolmogorov A. N. (1954). \textit{Limit Distribution for Sums of Independent Random Variables}
\end{thebibliography}

\end{document}