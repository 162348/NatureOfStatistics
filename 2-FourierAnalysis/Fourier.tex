\documentclass[uplatex,dvipdfmx]{jsreport}
\title{Fourier解析}
\author{司馬博文}
\date{\today}
\pagestyle{headings} \setcounter{secnumdepth}{4}
\usepackage{mathtools}
%\mathtoolsset{showonlyrefs=true} %labelを附した数式にのみ附番される設定.
%\usepackage{amsmath} %mathtoolsの内部で呼ばれるので要らない.
\usepackage{amsfonts} %mathfrak, mathcal, mathbbなど.
\usepackage{amsthm} %定理環境.
\usepackage{amssymb} %AMSFontsを使うためのパッケージ.
\usepackage{ascmac} %screen, itembox, shadebox環境.全てLATEX2εの標準機能の範囲で作られたもの.
\usepackage{comment} %comment環境を用いて,複数行をcomment outできるようにするpackage
\usepackage{wrapfig} %図の周りに文字をwrapさせることができる.詳細な制御ができる.
\usepackage[usenames, dvipsnames]{xcolor} %xcolorはcolorの拡張.optionの意味はdvipsnamesはLoad a set of predefined colors. forestgreenなどの色が追加されている.usenamesはobsoleteとだけ書いてあった.
\setcounter{tocdepth}{2} %目次に表示される深さ.2はsubsectionまで
\usepackage{multicol} %\begin{multicols}{2}環境で途中からmulticolumnに出来る.

\usepackage{url}
\usepackage[dvipdfmx,colorlinks,linkcolor=blue,urlcolor=blue]{hyperref} %生成されるPDFファイルにおいて、\tableofcontentsによって書き出された目次をクリックすると該当する見出しへジャンプしたり、さらには、\label{ラベル名}を番号で参照する\ref{ラベル名}やthebibliography環境において\bibitem{ラベル名}を文献番号で参照する\cite{ラベル名}においても番号をクリックすると該当箇所にジャンプする.囲み枠はダサいので,colorlinksで囲み廃止し,リンク自体に色を付けることにした.
\usepackage{pxjahyper} %pxrubrica同様,八登崇之さん.hyperrefは日本語pLaTeXに最適化されていないから,hyperrefとセットで,(u)pLaTeX+hyperref+dvipdfmxの組み合わせで日本語を含む「しおり」をもつPDF文書を作成する場合に必要となる機能を提供する
\definecolor{花緑青}{cmyk}{0.52,0.03,0,0.27}
\definecolor{サーモンピンク}{cmyk}{0,0.65,0.65,0.05}
\definecolor{暗中模索}{rgb}{0.2,0.2,0.2}

\usepackage{tikz}
\usetikzlibrary{positioning,automata} %automaton描画のため
\usepackage{tikz-cd}
\usepackage[all]{xy}
\def\objectstyle{\displaystyle} %デフォルトではxymatrix中の数式が文中数式モードになるので,それを直す.\labelstyleも同様にxy packageの中で定義されており,文中数式モードになっている.

\usepackage[version=4]{mhchem} %化学式をTikZで簡単に書くためのパッケージ.
\usepackage{chemfig} %化学構造式をTikZで描くためのパッケージ.
\usepackage{siunitx} %IS単位を書くためのパッケージ

\usepackage{ulem} %取り消し線を引くためのパッケージ
\usepackage{pxrubrica} %日本語にルビをふる.八登崇之(やとうたかゆき)氏による.

\usepackage{graphicx} %rotatebox, scalebox, reflectbox, resizeboxなどのコマンドや,図表の読み込み\includegraphicsを司る.graphics というパッケージもありますが,graphicx はこれを高機能にしたものと考えて結構です(ただし graphicx は内部で graphics を読み込みます)

\usepackage[breakable]{tcolorbox} %加藤晃史さんがフル活用していたtcolorboxを,途中改ページ可能で.
\tcbuselibrary{theorems} %https://qiita.com/t_kemmochi/items/483b8fcdb5db8d1f5d5e
\usepackage{enumerate} %enumerate環境を凝らせる.
\usepackage[top=15truemm,bottom=15truemm,left=10truemm,right=10truemm]{geometry} %足助さんからもらったオプション

%%%%%%%%%%%%%%% 環境マクロ %%%%%%%%%%%%%%%

\usepackage{listings} %ソースコードを表示できる環境.多分もっといい方法ある.
\usepackage{jvlisting} %日本語のコメントアウトをする場合jlistingが必要
\lstset{ %ここからソースコードの表示に関する設定.lstlisting環境では,[caption=hoge,label=fuga]などのoptionを付けられる.
%[escapechar=!]とすると,LaTeXコマンドを使える.
  basicstyle={\ttfamily},
  identifierstyle={\small},
  commentstyle={\smallitshape},
  keywordstyle={\small\bfseries},
  ndkeywordstyle={\small},
  stringstyle={\small\ttfamily},
  frame={tb},
  breaklines=true,
  columns=[l]{fullflexible},
  numbers=left,
  xrightmargin=0zw,
  xleftmargin=3zw,
  numberstyle={\scriptsize},
  stepnumber=1,
  numbersep=1zw,
  lineskip=-0.5ex
}
%\makeatletter %caption番号を「[chapter番号].[section番号].[subsection番号]-[そのsubsection内においてn番目]」に変更
%    \AtBeginDocument{
%    \renewcommand*{\thelstlisting}{\arabic{chapter}.\arabic{section}.\arabic{lstlisting}}
%    \@addtoreset{lstlisting}{section}
%    }
%\makeatother
\renewcommand{\lstlistingname}{算譜} %caption名を"program"に変更

\newtcolorbox{tbox}[3][]{%
colframe=#2,colback=#2!10,coltitle=#2!20!black,title={#3},#1}

%%%%%%%%%%%%%%% フォント %%%%%%%%%%%%%%%

\usepackage{textcomp, mathcomp} %Text Companionとは,T1 encodingに入らなかった文字群.これを使うためのパッケージ.\textsectionでブルバキに!
\usepackage[T1]{fontenc} %8bitエンコーディングにする.comp系拡張数学文字の動作が安定する.

%%%%%%%%%%%%%%% 数学記号のマクロ %%%%%%%%%%%%%%%

\newcommand{\abs}[1]{\lvert#1\rvert} %mathtoolsはこうやって使うのか!
\newcommand{\Abs}[1]{\left|#1\right|}
\newcommand{\norm}[1]{\|#1\|}
\newcommand{\Norm}[1]{\left\|#1\right\|}
%\newcommand{\brace}[1]{\{#1\}}
\newcommand{\Brace}[1]{\left\{#1\right\}}
\newcommand{\paren}[1]{\left(#1\right)}
\newcommand{\bracket}[1]{\langle#1\rangle}
\newcommand{\brac}[1]{\langle#1\rangle}
\newcommand{\Bracket}[1]{\left\langle#1\right\rangle}
\newcommand{\Brac}[1]{\left\langle#1\right\rangle}
\newcommand{\Square}[1]{\left[#1\right]}
\renewcommand{\o}[1]{\overline{#1}}
\renewcommand{\u}[1]{\underline{#1}}
\renewcommand{\iff}{\;\mathrm{iff}\;} %nLabリスペクト
\newcommand{\pp}[2]{\frac{\partial #1}{\partial #2}}
\newcommand{\ppp}[3]{\frac{\partial #1}{\partial #2\partial #3}}
\newcommand{\dd}[2]{\frac{d #1}{d #2}}
\newcommand{\floor}[1]{\lfloor#1\rfloor}
\newcommand{\Floor}[1]{\left\lfloor#1\right\rfloor}
\newcommand{\ceil}[1]{\lceil#1\rceil}

\newcommand{\iso}{\xrightarrow{\,\smash{\raisebox{-0.45ex}{\ensuremath{\scriptstyle\sim}}}\,}}
\newcommand{\wt}[1]{\widetilde{#1}}
\newcommand{\wh}[1]{\widehat{#1}}

\newcommand{\Lrarrow}{\;\;\Leftrightarrow\;\;}

%ノルム位相についての閉包 https://newbedev.com/how-to-make-double-overline-with-less-vertical-displacement
\makeatletter
\newcommand{\dbloverline}[1]{\overline{\dbl@overline{#1}}}
\newcommand{\dbl@overline}[1]{\mathpalette\dbl@@overline{#1}}
\newcommand{\dbl@@overline}[2]{%
  \begingroup
  \sbox\z@{$\m@th#1\overline{#2}$}%
  \ht\z@=\dimexpr\ht\z@-2\dbl@adjust{#1}\relax
  \box\z@
  \ifx#1\scriptstyle\kern-\scriptspace\else
  \ifx#1\scriptscriptstyle\kern-\scriptspace\fi\fi
  \endgroup
}
\newcommand{\dbl@adjust}[1]{%
  \fontdimen8
  \ifx#1\displaystyle\textfont\else
  \ifx#1\textstyle\textfont\else
  \ifx#1\scriptstyle\scriptfont\else
  \scriptscriptfont\fi\fi\fi 3
}
\makeatother
\newcommand{\oo}[1]{\dbloverline{#1}}

\DeclareMathOperator{\grad}{\mathrm{grad}}
\DeclareMathOperator{\rot}{\mathrm{rot}}
\DeclareMathOperator{\divergence}{\mathrm{div}}
\newcommand{\False}{\mathrm{False}}
\newcommand{\True}{\mathrm{True}}
\DeclareMathOperator{\tr}{\mathrm{tr}}
\newcommand{\M}{\mathcal{M}}
\newcommand{\cF}{\mathcal{F}}
\newcommand{\cD}{\mathcal{D}}
\newcommand{\fX}{\mathfrak{X}}
\newcommand{\fY}{\mathfrak{Y}}
\newcommand{\fZ}{\mathfrak{Z}}
\renewcommand{\H}{\mathcal{H}}
\newcommand{\fH}{\mathfrak{H}}
\newcommand{\bH}{\mathbb{H}}
\newcommand{\id}{\mathrm{id}}
\newcommand{\A}{\mathcal{A}}
% \renewcommand\coprod{\rotatebox[origin=c]{180}{$\prod$}} すでにどこかにある.
\newcommand{\pr}{\mathrm{pr}}
\newcommand{\U}{\mathfrak{U}}
\newcommand{\Map}{\mathrm{Map}}
\newcommand{\dom}{\mathrm{Dom}\;}
\newcommand{\cod}{\mathrm{Cod}\;}
\newcommand{\supp}{\mathrm{supp}\;}
\newcommand{\otherwise}{\mathrm{otherwise}}
\newcommand{\st}{\;\mathrm{s.t.}\;}
\newcommand{\lmd}{\lambda}
\newcommand{\Lmd}{\Lambda}
%%% 線型代数学
\newcommand{\Ker}{\mathrm{Ker}\;}
\newcommand{\Coker}{\mathrm{Coker}\;}
\newcommand{\Coim}{\mathrm{Coim}\;}
\newcommand{\rank}{\mathrm{rank}}
\newcommand{\lcm}{\mathrm{lcm}}
\newcommand{\sgn}{\mathrm{sgn}}
\newcommand{\GL}{\mathrm{GL}}
\newcommand{\SL}{\mathrm{SL}}
\newcommand{\alt}{\mathrm{alt}}
%%% 複素解析学
\renewcommand{\Re}{\mathrm{Re}\;}
\renewcommand{\Im}{\mathrm{Im}\;}
\newcommand{\Gal}{\mathrm{Gal}}
\newcommand{\PGL}{\mathrm{PGL}}
\newcommand{\PSL}{\mathrm{PSL}}
\newcommand{\Log}{\mathrm{Log}\,}
\newcommand{\Res}{\mathrm{Res}\,}
\newcommand{\on}{\mathrm{on}\;}
\newcommand{\hatC}{\hat{\C}}
\newcommand{\hatR}{\hat{\R}}
\newcommand{\PV}{\mathrm{P.V.}}
\newcommand{\diam}{\mathrm{diam}}
\newcommand{\Area}{\mathrm{Area}}
\newcommand{\Lap}{\Laplace}
\newcommand{\f}{\mathbf{f}}
\newcommand{\cR}{\mathcal{R}}
\newcommand{\const}{\mathrm{const.}}
\newcommand{\Om}{\Omega}
\newcommand{\Cinf}{C^\infty}
\newcommand{\ep}{\epsilon}
\newcommand{\dist}{\mathrm{dist}}
\newcommand{\opart}{\o{\partial}}
%%% 解析力学
\newcommand{\x}{\mathbf{x}}
%%% 集合と位相
\renewcommand{\O}{\mathcal{O}}
\renewcommand{\S}{\mathcal{S}}
\renewcommand{\U}{\mathcal{U}}
\newcommand{\V}{\mathcal{V}}
\renewcommand{\P}{\mathcal{P}}
\newcommand{\R}{\mathbb{R}}
\newcommand{\N}{\mathbb{N}}
\newcommand{\C}{\mathbb{C}}
\newcommand{\Z}{\mathbb{Z}}
\newcommand{\Q}{\mathbb{Q}}
\newcommand{\TV}{\mathrm{TV}}
\newcommand{\ORD}{\mathrm{ORD}}
\newcommand{\Tr}{\mathrm{Tr}\;}
\newcommand{\Card}{\mathrm{Card}\;}
\newcommand{\Top}{\mathrm{Top}}
\newcommand{\Disc}{\mathrm{Disc}}
\newcommand{\Codisc}{\mathrm{Codisc}}
\newcommand{\CoDisc}{\mathrm{CoDisc}}
\newcommand{\Ult}{\mathrm{Ult}}
\newcommand{\ord}{\mathrm{ord}}
\newcommand{\maj}{\mathrm{maj}}
%%% 形式言語理論
\newcommand{\REGEX}{\mathrm{REGEX}}
\newcommand{\RE}{\mathbf{RE}}

%%% Fourier解析
\newcommand*{\Laplace}{\mathop{}\!\mathbin\bigtriangleup}
\newcommand*{\DAlambert}{\mathop{}\!\mathbin\Box}
%%% Graph Theory
\newcommand{\SimpGph}{\mathrm{SimpGph}}
\newcommand{\Gph}{\mathrm{Gph}}
\newcommand{\mult}{\mathrm{mult}}
\newcommand{\inv}{\mathrm{inv}}
%%% 多様体
\newcommand{\Der}{\mathrm{Der}}
\newcommand{\osub}{\overset{\mathrm{open}}{\subset}}
\newcommand{\osup}{\overset{\mathrm{open}}{\supset}}
\newcommand{\al}{\alpha}
\newcommand{\K}{\mathbb{K}}
\newcommand{\Sp}{\mathrm{Sp}}
\newcommand{\g}{\mathfrak{g}}
\newcommand{\h}{\mathfrak{h}}
\newcommand{\Exp}{\mathrm{Exp}\;}
\newcommand{\Imm}{\mathrm{Imm}}
\newcommand{\Imb}{\mathrm{Imb}}
\newcommand{\codim}{\mathrm{codim}\;}
\newcommand{\Gr}{\mathrm{Gr}}
%%% 代数
\newcommand{\Ad}{\mathrm{Ad}}
\newcommand{\finsupp}{\mathrm{fin\;supp}}
\newcommand{\SO}{\mathrm{SO}}
\newcommand{\SU}{\mathrm{SU}}
\newcommand{\acts}{\curvearrowright}
\newcommand{\mono}{\hookrightarrow}
\newcommand{\epi}{\twoheadrightarrow}
\newcommand{\Stab}{\mathrm{Stab}}
\newcommand{\nor}{\mathrm{nor}}
\newcommand{\T}{\mathbb{T}}
\newcommand{\Aff}{\mathrm{Aff}}
\newcommand{\rsub}{\triangleleft}
\newcommand{\rsup}{\triangleright}
\newcommand{\subgrp}{\overset{\mathrm{subgrp}}{\subset}}
\newcommand{\Ext}{\mathrm{Ext}}
\newcommand{\sbs}{\subset}\newcommand{\sps}{\supset}
\newcommand{\In}{\mathrm{In}}
\newcommand{\Tor}{\mathrm{Tor}}
\newcommand{\p}{\mathfrak{p}}
\newcommand{\q}{\mathfrak{q}}
\newcommand{\m}{\mathfrak{m}}
\newcommand{\cS}{\mathcal{S}}
\newcommand{\Frac}{\mathrm{Frac}\,}
\newcommand{\Spec}{\mathrm{Spec}\,}
\newcommand{\bA}{\mathbb{A}}
\newcommand{\Sym}{\mathrm{Sym}}
\newcommand{\Ann}{\mathrm{Ann}}
%%% 代数的位相幾何学
\newcommand{\Ho}{\mathrm{Ho}}
\newcommand{\CW}{\mathrm{CW}}
\newcommand{\lc}{\mathrm{lc}}
\newcommand{\cg}{\mathrm{cg}}
\newcommand{\Fib}{\mathrm{Fib}}
\newcommand{\Cyl}{\mathrm{Cyl}}
\newcommand{\Ch}{\mathrm{Ch}}
%%% 数値解析
\newcommand{\round}{\mathrm{round}}
\newcommand{\cond}{\mathrm{cond}}
\newcommand{\diag}{\mathrm{diag}}
%%% 確率論
\newcommand{\calF}{\mathcal{F}}
\newcommand{\X}{\mathcal{X}}
\newcommand{\Meas}{\mathrm{Meas}}
\newcommand{\as}{\;\mathrm{a.s.}} %almost surely
\newcommand{\io}{\;\mathrm{i.o.}} %infinitely often
\newcommand{\fe}{\;\mathrm{f.e.}} %with a finite number of exceptions
\newcommand{\F}{\mathcal{F}}
\newcommand{\bF}{\mathbb{F}}
\newcommand{\W}{\mathcal{W}}
\newcommand{\Pois}{\mathrm{Pois}}
\newcommand{\iid}{\mathrm{i.i.d.}}
\newcommand{\wconv}{\rightsquigarrow}
\newcommand{\Var}{\mathrm{Var}}
\newcommand{\xrightarrown}{\xrightarrow{n\to\infty}}
\newcommand{\au}{\mathrm{au}}
\newcommand{\cT}{\mathcal{T}}
%%% 情報理論
\newcommand{\bit}{\mathrm{bit}}
%%% 積分論
\newcommand{\calA}{\mathcal{A}}
\newcommand{\calB}{\mathcal{B}}
\newcommand{\D}{\mathcal{D}}
\newcommand{\Y}{\mathcal{Y}}
\newcommand{\calC}{\mathcal{C}}
\renewcommand{\ae}{\mathrm{a.e.}\;}
\newcommand{\cZ}{\mathcal{Z}}
\newcommand{\fF}{\mathfrak{F}}
\newcommand{\fI}{\mathfrak{I}}
\newcommand{\E}{\mathcal{E}}
\newcommand{\sMap}{\sigma\textrm{-}\mathrm{Map}}
\DeclareMathOperator*{\argmax}{arg\,max}
\DeclareMathOperator*{\argmin}{arg\,min}
\newcommand{\cC}{\mathcal{C}}
\newcommand{\comp}{\complement}
\newcommand{\J}{\mathcal{J}}
\newcommand{\sumN}[1]{\sum_{#1\in\N}}
\newcommand{\cupN}[1]{\cup_{#1\in\N}}
\newcommand{\capN}[1]{\cap_{#1\in\N}}
\newcommand{\Sum}[1]{\sum_{#1=1}^\infty}
\newcommand{\sumn}{\sum_{n=1}^\infty}
\newcommand{\summ}{\sum_{m=1}^\infty}
\newcommand{\sumk}{\sum_{k=1}^\infty}
\newcommand{\sumi}{\sum_{i=1}^\infty}
\newcommand{\sumj}{\sum_{j=1}^\infty}
\newcommand{\cupn}{\cup_{n=1}^\infty}
\newcommand{\capn}{\cap_{n=1}^\infty}
\newcommand{\cupk}{\cup_{k=1}^\infty}
\newcommand{\cupi}{\cup_{i=1}^\infty}
\newcommand{\cupj}{\cup_{j=1}^\infty}
\newcommand{\limn}{\lim_{n\to\infty}}
\renewcommand{\l}{\mathcal{l}}
\renewcommand{\L}{\mathcal{L}}
\newcommand{\Cl}{\mathrm{Cl}}
\newcommand{\cN}{\mathcal{N}}
\newcommand{\Ae}{\textrm{-a.e.}\;}
\newcommand{\csub}{\overset{\textrm{closed}}{\subset}}
\newcommand{\csup}{\overset{\textrm{closed}}{\supset}}
\newcommand{\wB}{\wt{B}}
\newcommand{\cG}{\mathcal{G}}
\newcommand{\Lip}{\mathrm{Lip}}
\newcommand{\Dom}{\mathrm{Dom}}
%%% 数理ファイナンス
\newcommand{\pre}{\mathrm{pre}}
\newcommand{\om}{\omega}

%%% 統計的因果推論
\newcommand{\Do}{\mathrm{Do}}
%%% 数理統計
\newcommand{\bP}{\mathbb{P}}
\newcommand{\compsub}{\overset{\textrm{cpt}}{\subset}}
\newcommand{\lip}{\textrm{lip}}
\newcommand{\BL}{\mathrm{BL}}
\newcommand{\G}{\mathbb{G}}
\newcommand{\NB}{\mathrm{NB}}
\newcommand{\oR}{\o{\R}}
\newcommand{\liminfn}{\liminf_{n\to\infty}}
\newcommand{\limsupn}{\limsup_{n\to\infty}}
%\newcommand{\limn}{\lim_{n\to\infty}}
\newcommand{\esssup}{\mathrm{ess.sup}}
\newcommand{\asto}{\xrightarrow{\as}}
\newcommand{\Cov}{\mathrm{Cov}}
\newcommand{\cQ}{\mathcal{Q}}
\newcommand{\VC}{\mathrm{VC}}
\newcommand{\mb}{\mathrm{mb}}
\newcommand{\Avar}{\mathrm{Avar}}
\newcommand{\bB}{\mathbb{B}}
\newcommand{\bW}{\mathbb{W}}
\newcommand{\sd}{\mathrm{sd}}
\newcommand{\w}[1]{\widehat{#1}}
\newcommand{\bZ}{\mathbb{Z}}
\newcommand{\Bernoulli}{\mathrm{Bernoulli}}
\newcommand{\Mult}{\mathrm{Mult}}
\newcommand{\BPois}{\mathrm{BPois}}
\newcommand{\fraks}{\mathfrak{s}}
\newcommand{\frakk}{\mathfrak{k}}
\newcommand{\IF}{\mathrm{IF}}
\newcommand{\bX}{\mathbf{X}}
\newcommand{\bx}{\mathbf{x}}
\newcommand{\indep}{\raisebox{0.05em}{\rotatebox[origin=c]{90}{$\models$}}}
\newcommand{\IG}{\mathrm{IG}}
\newcommand{\Levy}{\mathrm{Levy}}
\newcommand{\MP}{\mathrm{MP}}
\newcommand{\Hermite}{\mathrm{Hermite}}
\newcommand{\Skellam}{\mathrm{Skellam}}
\newcommand{\Dirichlet}{\mathrm{Dirichlet}}
\newcommand{\Beta}{\mathrm{Beta}}
\newcommand{\bE}{\mathbb{E}}
\newcommand{\bG}{\mathbb{G}}
\newcommand{\MISE}{\mathrm{MISE}}
\newcommand{\logit}{\mathtt{logit}}
\newcommand{\expit}{\mathtt{expit}}
\newcommand{\cK}{\mathcal{K}}
\newcommand{\dl}{\dot{l}}
\newcommand{\dotp}{\dot{p}}
\newcommand{\wl}{\wt{l}}
%%% 函数解析
\renewcommand{\c}{\mathbf{c}}
\newcommand{\loc}{\mathrm{loc}}
\newcommand{\Lh}{\mathrm{L.h.}}
\newcommand{\Epi}{\mathrm{Epi}\;}
\newcommand{\slim}{\mathrm{slim}}
\newcommand{\Ban}{\mathrm{Ban}}
\newcommand{\Hilb}{\mathrm{Hilb}}
\newcommand{\Ex}{\mathrm{Ex}}
\newcommand{\Co}{\mathrm{Co}}
\newcommand{\sa}{\mathrm{sa}}
\newcommand{\nnorm}[1]{{\left\vert\kern-0.25ex\left\vert\kern-0.25ex\left\vert #1 \right\vert\kern-0.25ex\right\vert\kern-0.25ex\right\vert}}
\newcommand{\dvol}{\mathrm{dvol}}
\newcommand{\Sconv}{\mathrm{Sconv}}
\newcommand{\I}{\mathcal{I}}
\newcommand{\nonunital}{\mathrm{nu}}
\newcommand{\cpt}{\mathrm{cpt}}
\newcommand{\lcpt}{\mathrm{lcpt}}
\newcommand{\com}{\mathrm{com}}
\newcommand{\Haus}{\mathrm{Haus}}
\newcommand{\proper}{\mathrm{proper}}
\newcommand{\infinity}{\mathrm{inf}}
\newcommand{\TVS}{\mathrm{TVS}}
\newcommand{\ess}{\mathrm{ess}}
\newcommand{\ext}{\mathrm{ext}}
\newcommand{\Index}{\mathrm{Index}}
\newcommand{\SSR}{\mathrm{SSR}}
\newcommand{\vs}{\mathrm{vs.}}
\newcommand{\fM}{\mathfrak{M}}
\newcommand{\EDM}{\mathrm{EDM}}
\newcommand{\Tw}{\mathrm{Tw}}
\newcommand{\fC}{\mathfrak{C}}
\newcommand{\bn}{\mathbf{n}}
\newcommand{\br}{\mathbf{r}}
\newcommand{\Lam}{\Lambda}
\newcommand{\lam}{\lambda}
\newcommand{\one}{\mathbf{1}}
\newcommand{\dae}{\text{-a.e.}}
\newcommand{\td}{\text{-}}
\newcommand{\RM}{\mathrm{RM}}
%%% 最適化
\newcommand{\Minimize}{\text{Minimize}}
\newcommand{\subjectto}{\text{subject to}}
\newcommand{\Ri}{\mathrm{Ri}}
%\newcommand{\Cl}{\mathrm{Cl}}
\newcommand{\Cone}{\mathrm{Cone}}
\newcommand{\Int}{\mathrm{Int}}
%%% 圏
\newcommand{\varlim}{\varprojlim}
\newcommand{\Hom}{\mathrm{Hom}}
\newcommand{\Iso}{\mathrm{Iso}}
\newcommand{\Mor}{\mathrm{Mor}}
\newcommand{\Isom}{\mathrm{Isom}}
\newcommand{\Aut}{\mathrm{Aut}}
\newcommand{\End}{\mathrm{End}}
\newcommand{\op}{\mathrm{op}}
\newcommand{\ev}{\mathrm{ev}}
\newcommand{\Ob}{\mathrm{Ob}}
\newcommand{\Ar}{\mathrm{Ar}}
\newcommand{\Arr}{\mathrm{Arr}}
\newcommand{\Set}{\mathrm{Set}}
\newcommand{\Grp}{\mathrm{Grp}}
\newcommand{\Cat}{\mathrm{Cat}}
\newcommand{\Mon}{\mathrm{Mon}}
\newcommand{\CMon}{\mathrm{CMon}} %Comutative Monoid 可換単系とモノイドの射
\newcommand{\Ring}{\mathrm{Ring}}
\newcommand{\CRing}{\mathrm{CRing}}
\newcommand{\Ab}{\mathrm{Ab}}
\newcommand{\Pos}{\mathrm{Pos}}
\newcommand{\Vect}{\mathrm{Vect}}
\newcommand{\FinVect}{\mathrm{FinVect}}
\newcommand{\FinSet}{\mathrm{FinSet}}
\newcommand{\OmegaAlg}{\Omega$-$\mathrm{Alg}}
\newcommand{\OmegaEAlg}{(\Omega,E)$-$\mathrm{Alg}}
\newcommand{\Alg}{\mathrm{Alg}} %代数の圏
\newcommand{\CAlg}{\mathrm{CAlg}} %可換代数の圏
\newcommand{\CPO}{\mathrm{CPO}} %Complete Partial Order & continuous mappings
\newcommand{\Fun}{\mathrm{Fun}}
\newcommand{\Func}{\mathrm{Func}}
\newcommand{\Met}{\mathrm{Met}} %Metric space & Contraction maps
\newcommand{\Pfn}{\mathrm{Pfn}} %Sets & Partial function
\newcommand{\Rel}{\mathrm{Rel}} %Sets & relation
\newcommand{\Bool}{\mathrm{Bool}}
\newcommand{\CABool}{\mathrm{CABool}}
\newcommand{\CompBoolAlg}{\mathrm{CompBoolAlg}}
\newcommand{\BoolAlg}{\mathrm{BoolAlg}}
\newcommand{\BoolRng}{\mathrm{BoolRng}}
\newcommand{\HeytAlg}{\mathrm{HeytAlg}}
\newcommand{\CompHeytAlg}{\mathrm{CompHeytAlg}}
\newcommand{\Lat}{\mathrm{Lat}}
\newcommand{\CompLat}{\mathrm{CompLat}}
\newcommand{\SemiLat}{\mathrm{SemiLat}}
\newcommand{\Stone}{\mathrm{Stone}}
\newcommand{\Sob}{\mathrm{Sob}} %Sober space & continuous map
\newcommand{\Op}{\mathrm{Op}} %Category of open subsets
\newcommand{\Sh}{\mathrm{Sh}} %Category of sheave
\newcommand{\PSh}{\mathrm{PSh}} %Category of presheave, PSh(C)=[C^op,set]のこと
\newcommand{\Conv}{\mathrm{Conv}} %Convergence spaceの圏
\newcommand{\Unif}{\mathrm{Unif}} %一様空間と一様連続写像の圏
\newcommand{\Frm}{\mathrm{Frm}} %フレームとフレームの射
\newcommand{\Locale}{\mathrm{Locale}} %その反対圏
\newcommand{\Diff}{\mathrm{Diff}} %滑らかな多様体の圏
\newcommand{\Mfd}{\mathrm{Mfd}}
\newcommand{\LieAlg}{\mathrm{LieAlg}}
\newcommand{\Quiv}{\mathrm{Quiv}} %Quiverの圏
\newcommand{\B}{\mathcal{B}}
\newcommand{\Span}{\mathrm{Span}}
\newcommand{\Corr}{\mathrm{Corr}}
\newcommand{\Decat}{\mathrm{Decat}}
\newcommand{\Rep}{\mathrm{Rep}}
\newcommand{\Grpd}{\mathrm{Grpd}}
\newcommand{\sSet}{\mathrm{sSet}}
\newcommand{\Mod}{\mathrm{Mod}}
\newcommand{\SmoothMnf}{\mathrm{SmoothMnf}}
\newcommand{\coker}{\mathrm{coker}}

\newcommand{\Ord}{\mathrm{Ord}}
\newcommand{\eq}{\mathrm{eq}}
\newcommand{\coeq}{\mathrm{coeq}}
\newcommand{\act}{\mathrm{act}}

%%%%%%%%%%%%%%% 定理環境(足助先生ありがとうございます) %%%%%%%%%%%%%%%

\everymath{\displaystyle}
\renewcommand{\proofname}{\bf [証明]}
\renewcommand{\thefootnote}{\dag\arabic{footnote}} %足助さんからもらった.どうなるんだ?
\renewcommand{\qedsymbol}{$\blacksquare$}

\renewcommand{\labelenumi}{(\arabic{enumi})} %(1),(2),...がデフォルトであって欲しい
\renewcommand{\labelenumii}{(\alph{enumii})}
\renewcommand{\labelenumiii}{(\roman{enumiii})}

\newtheoremstyle{StatementsWithStar}% ?name?
{3pt}% ?Space above? 1
{3pt}% ?Space below? 1
{}% ?Body font?
{}% ?Indent amount? 2
{\bfseries}% ?Theorem head font?
{\textbf{.}}% ?Punctuation after theorem head?
{.5em}% ?Space after theorem head? 3
{\textbf{\textup{#1~\thetheorem{}}}{}\,$^{\ast}$\thmnote{(#3)}}% ?Theorem head spec (can be left empty, meaning ‘normal’)?
%
\newtheoremstyle{StatementsWithStar2}% ?name?
{3pt}% ?Space above? 1
{3pt}% ?Space below? 1
{}% ?Body font?
{}% ?Indent amount? 2
{\bfseries}% ?Theorem head font?
{\textbf{.}}% ?Punctuation after theorem head?
{.5em}% ?Space after theorem head? 3
{\textbf{\textup{#1~\thetheorem{}}}{}\,$^{\ast\ast}$\thmnote{(#3)}}% ?Theorem head spec (can be left empty, meaning ‘normal’)?
%
\newtheoremstyle{StatementsWithStar3}% ?name?
{3pt}% ?Space above? 1
{3pt}% ?Space below? 1
{}% ?Body font?
{}% ?Indent amount? 2
{\bfseries}% ?Theorem head font?
{\textbf{.}}% ?Punctuation after theorem head?
{.5em}% ?Space after theorem head? 3
{\textbf{\textup{#1~\thetheorem{}}}{}\,$^{\ast\ast\ast}$\thmnote{(#3)}}% ?Theorem head spec (can be left empty, meaning ‘normal’)?
%
\newtheoremstyle{StatementsWithCCirc}% ?name?
{6pt}% ?Space above? 1
{6pt}% ?Space below? 1
{}% ?Body font?
{}% ?Indent amount? 2
{\bfseries}% ?Theorem head font?
{\textbf{.}}% ?Punctuation after theorem head?
{.5em}% ?Space after theorem head? 3
{\textbf{\textup{#1~\thetheorem{}}}{}\,$^{\circledcirc}$\thmnote{(#3)}}% ?Theorem head spec (can be left empty, meaning ‘normal’)?
%
\theoremstyle{definition}
 \newtheorem{theorem}{定理}[section]
 \newtheorem{axiom}[theorem]{公理}
 \newtheorem{corollary}[theorem]{系}
 \newtheorem{proposition}[theorem]{命題}
 \newtheorem*{proposition*}{命題}
 \newtheorem{lemma}[theorem]{補題}
 \newtheorem*{lemma*}{補題}
 \newtheorem*{theorem*}{定理}
 \newtheorem{definition}[theorem]{定義}
 \newtheorem{example}[theorem]{例}
 \newtheorem{notation}[theorem]{記法}
 \newtheorem*{notation*}{記法}
 \newtheorem{assumption}[theorem]{仮定}
 \newtheorem{question}[theorem]{問}
 \newtheorem{counterexample}[theorem]{反例}
 \newtheorem{reidai}[theorem]{例題}
 \newtheorem{ruidai}[theorem]{類題}
 \newtheorem{problem}[theorem]{問題}
 \newtheorem{algorithm}[theorem]{算譜}
 \newtheorem*{solution*}{\bf{[解]}}
 \newtheorem{discussion}[theorem]{議論}
 \newtheorem{remark}[theorem]{注}
 \newtheorem{remarks}[theorem]{要諦}
 \newtheorem{image}[theorem]{描像}
 \newtheorem{observation}[theorem]{観察}
 \newtheorem{universality}[theorem]{普遍性} %非自明な例外がない.
 \newtheorem{universal tendency}[theorem]{普遍傾向} %例外が有意に少ない.
 \newtheorem{hypothesis}[theorem]{仮説} %実験で説明されていない理論.
 \newtheorem{theory}[theorem]{理論} %実験事実とその(さしあたり)整合的な説明.
 \newtheorem{fact}[theorem]{実験事実}
 \newtheorem{model}[theorem]{模型}
 \newtheorem{explanation}[theorem]{説明} %理論による実験事実の説明
 \newtheorem{anomaly}[theorem]{理論の限界}
 \newtheorem{application}[theorem]{応用例}
 \newtheorem{method}[theorem]{手法} %実験手法など,技術的問題.
 \newtheorem{history}[theorem]{歴史}
 \newtheorem{usage}[theorem]{用語法}
 \newtheorem{research}[theorem]{研究}
 \newtheorem{shishin}[theorem]{指針}
 \newtheorem{yodan}[theorem]{余談}
 \newtheorem{construction}[theorem]{構成}
% \newtheorem*{remarknonum}{注}
 \newtheorem*{definition*}{定義}
 \newtheorem*{remark*}{注}
 \newtheorem*{question*}{問}
 \newtheorem*{problem*}{問題}
 \newtheorem*{axiom*}{公理}
 \newtheorem*{example*}{例}
 \newtheorem*{corollary*}{系}
 \newtheorem*{shishin*}{指針}
 \newtheorem*{yodan*}{余談}
 \newtheorem*{kadai*}{課題}
%
\theoremstyle{StatementsWithStar}
 \newtheorem{definition_*}[theorem]{定義}
 \newtheorem{question_*}[theorem]{問}
 \newtheorem{example_*}[theorem]{例}
 \newtheorem{theorem_*}[theorem]{定理}
 \newtheorem{remark_*}[theorem]{注}
%
\theoremstyle{StatementsWithStar2}
 \newtheorem{definition_**}[theorem]{定義}
 \newtheorem{theorem_**}[theorem]{定理}
 \newtheorem{question_**}[theorem]{問}
 \newtheorem{remark_**}[theorem]{注}
%
\theoremstyle{StatementsWithStar3}
 \newtheorem{remark_***}[theorem]{注}
 \newtheorem{question_***}[theorem]{問}
%
\theoremstyle{StatementsWithCCirc}
 \newtheorem{definition_O}[theorem]{定義}
 \newtheorem{question_O}[theorem]{問}
 \newtheorem{example_O}[theorem]{例}
 \newtheorem{remark_O}[theorem]{注}
%
\theoremstyle{definition}
%
\raggedbottom
\allowdisplaybreaks
%\usepackage{mathtools}
%\mathtoolsset{showonlyrefs=true} %labelを附した数式にのみ附番される設定.
%\usepackage{amsmath} %mathtoolsの内部で呼ばれるので要らない.
\usepackage{amsfonts} %mathfrak, mathcal, mathbbなど.
\usepackage{amsthm} %定理環境.
\usepackage{amssymb} %AMSFontsを使うためのパッケージ.
\usepackage{ascmac} %screen, itembox, shadebox環境.全てLATEX2εの標準機能の範囲で作られたもの.
\usepackage{comment} %comment環境を用いて,複数行をcomment outできるようにするpackage
\usepackage{wrapfig} %図の周りに文字をwrapさせることができる.詳細な制御ができる.
\usepackage[usenames, dvipsnames]{xcolor} %xcolorはcolorの拡張.optionの意味はdvipsnamesはLoad a set of predefined colors. forestgreenなどの色が追加されている.usenamesはobsoleteとだけ書いてあった.
\setcounter{tocdepth}{2} %目次に表示される深さ.2はsubsectionまで
\usepackage{multicol} %\begin{multicols}{2}環境で途中からmulticolumnに出来る.

\usepackage{url}
\usepackage[dvipdfmx,colorlinks,linkcolor=blue,urlcolor=blue]{hyperref} %生成されるPDFファイルにおいて、\tableofcontentsによって書き出された目次をクリックすると該当する見出しへジャンプしたり、さらには、\label{ラベル名}を番号で参照する\ref{ラベル名}やthebibliography環境において\bibitem{ラベル名}を文献番号で参照する\cite{ラベル名}においても番号をクリックすると該当箇所にジャンプする.囲み枠はダサいので,colorlinksで囲み廃止し,リンク自体に色を付けることにした.
\usepackage{pxjahyper} %pxrubrica同様,八登崇之さん.hyperrefは日本語pLaTeXに最適化されていないから,hyperrefとセットで,(u)pLaTeX+hyperref+dvipdfmxの組み合わせで日本語を含む「しおり」をもつPDF文書を作成する場合に必要となる機能を提供する
\definecolor{花緑青}{cmyk}{0.52,0.03,0,0.27}
\definecolor{サーモンピンク}{cmyk}{0,0.65,0.65,0.05}
\definecolor{暗中模索}{rgb}{0.2,0.2,0.2}

\usepackage{tikz}
\usetikzlibrary{positioning,automata} %automaton描画のため
\usepackage{tikz-cd}
\usepackage[all]{xy}
\def\objectstyle{\displaystyle} %デフォルトではxymatrix中の数式が文中数式モードになるので,それを直す.\labelstyleも同様にxy packageの中で定義されており,文中数式モードになっている.

\usepackage[version=4]{mhchem} %化学式をTikZで簡単に書くためのパッケージ.
\usepackage{chemfig} %化学構造式をTikZで描くためのパッケージ.
\usepackage{siunitx} %IS単位を書くためのパッケージ

\usepackage{ulem} %取り消し線を引くためのパッケージ
\usepackage{pxrubrica} %日本語にルビをふる.八登崇之(やとうたかゆき)氏による.

\usepackage{graphicx} %rotatebox, scalebox, reflectbox, resizeboxなどのコマンドや,図表の読み込み\includegraphicsを司る.graphics というパッケージもありますが,graphicx はこれを高機能にしたものと考えて結構です(ただし graphicx は内部で graphics を読み込みます)

\usepackage[breakable]{tcolorbox} %加藤晃史さんがフル活用していたtcolorboxを,途中改ページ可能で.
\tcbuselibrary{theorems} %https://qiita.com/t_kemmochi/items/483b8fcdb5db8d1f5d5e
\usepackage{enumerate} %enumerate環境を凝らせる.
\usepackage[top=15truemm,bottom=15truemm,left=10truemm,right=10truemm]{geometry} %足助さんからもらったオプション

%%%%%%%%%%%%%%% 環境マクロ %%%%%%%%%%%%%%%

\usepackage{listings} %ソースコードを表示できる環境.多分もっといい方法ある.
\usepackage{jvlisting} %日本語のコメントアウトをする場合jlistingが必要
\lstset{ %ここからソースコードの表示に関する設定.lstlisting環境では,[caption=hoge,label=fuga]などのoptionを付けられる.
%[escapechar=!]とすると,LaTeXコマンドを使える.
  basicstyle={\ttfamily},
  identifierstyle={\small},
  commentstyle={\smallitshape},
  keywordstyle={\small\bfseries},
  ndkeywordstyle={\small},
  stringstyle={\small\ttfamily},
  frame={tb},
  breaklines=true,
  columns=[l]{fullflexible},
  numbers=left,
  xrightmargin=0zw,
  xleftmargin=3zw,
  numberstyle={\scriptsize},
  stepnumber=1,
  numbersep=1zw,
  lineskip=-0.5ex
}
%\makeatletter %caption番号を「[chapter番号].[section番号].[subsection番号]-[そのsubsection内においてn番目]」に変更
%    \AtBeginDocument{
%    \renewcommand*{\thelstlisting}{\arabic{chapter}.\arabic{section}.\arabic{lstlisting}}
%    \@addtoreset{lstlisting}{section}
%    }
%\makeatother
\renewcommand{\lstlistingname}{算譜} %caption名を"program"に変更

\newtcolorbox{tbox}[3][]{%
colframe=#2,colback=#2!10,coltitle=#2!20!black,title={#3},#1}

%%%%%%%%%%%%%%% フォント %%%%%%%%%%%%%%%

\usepackage{textcomp, mathcomp} %Text Companionとは,T1 encodingに入らなかった文字群.これを使うためのパッケージ.\textsectionでブルバキに!
\usepackage[T1]{fontenc} %8bitエンコーディングにする.comp系拡張数学文字の動作が安定する.

%%%%%%%%%%%%%%% 数学記号のマクロ %%%%%%%%%%%%%%%

\newcommand{\abs}[1]{\lvert#1\rvert} %mathtoolsはこうやって使うのか!
\newcommand{\Abs}[1]{\left|#1\right|}
\newcommand{\norm}[1]{\|#1\|}
\newcommand{\Norm}[1]{\left\|#1\right\|}
%\newcommand{\brace}[1]{\{#1\}}
\newcommand{\Brace}[1]{\left\{#1\right\}}
\newcommand{\paren}[1]{\left(#1\right)}
\newcommand{\bracket}[1]{\langle#1\rangle}
\newcommand{\brac}[1]{\langle#1\rangle}
\newcommand{\Bracket}[1]{\left\langle#1\right\rangle}
\newcommand{\Brac}[1]{\left\langle#1\right\rangle}
\newcommand{\Square}[1]{\left[#1\right]}
\renewcommand{\o}[1]{\overline{#1}}
\renewcommand{\u}[1]{\underline{#1}}
\renewcommand{\iff}{\;\mathrm{iff}\;} %nLabリスペクト
\newcommand{\pp}[2]{\frac{\partial #1}{\partial #2}}
\newcommand{\ppp}[3]{\frac{\partial #1}{\partial #2\partial #3}}
\newcommand{\dd}[2]{\frac{d #1}{d #2}}
\newcommand{\floor}[1]{\lfloor#1\rfloor}
\newcommand{\Floor}[1]{\left\lfloor#1\right\rfloor}
\newcommand{\ceil}[1]{\lceil#1\rceil}

\newcommand{\iso}{\xrightarrow{\,\smash{\raisebox{-0.45ex}{\ensuremath{\scriptstyle\sim}}}\,}}
\newcommand{\wt}[1]{\widetilde{#1}}
\newcommand{\wh}[1]{\widehat{#1}}

\newcommand{\Lrarrow}{\;\;\Leftrightarrow\;\;}

%ノルム位相についての閉包 https://newbedev.com/how-to-make-double-overline-with-less-vertical-displacement
\makeatletter
\newcommand{\dbloverline}[1]{\overline{\dbl@overline{#1}}}
\newcommand{\dbl@overline}[1]{\mathpalette\dbl@@overline{#1}}
\newcommand{\dbl@@overline}[2]{%
  \begingroup
  \sbox\z@{$\m@th#1\overline{#2}$}%
  \ht\z@=\dimexpr\ht\z@-2\dbl@adjust{#1}\relax
  \box\z@
  \ifx#1\scriptstyle\kern-\scriptspace\else
  \ifx#1\scriptscriptstyle\kern-\scriptspace\fi\fi
  \endgroup
}
\newcommand{\dbl@adjust}[1]{%
  \fontdimen8
  \ifx#1\displaystyle\textfont\else
  \ifx#1\textstyle\textfont\else
  \ifx#1\scriptstyle\scriptfont\else
  \scriptscriptfont\fi\fi\fi 3
}
\makeatother
\newcommand{\oo}[1]{\dbloverline{#1}}

\DeclareMathOperator{\grad}{\mathrm{grad}}
\DeclareMathOperator{\rot}{\mathrm{rot}}
\DeclareMathOperator{\divergence}{\mathrm{div}}
\newcommand{\False}{\mathrm{False}}
\newcommand{\True}{\mathrm{True}}
\DeclareMathOperator{\tr}{\mathrm{tr}}
\newcommand{\M}{\mathcal{M}}
\newcommand{\cF}{\mathcal{F}}
\newcommand{\cD}{\mathcal{D}}
\newcommand{\fX}{\mathfrak{X}}
\newcommand{\fY}{\mathfrak{Y}}
\newcommand{\fZ}{\mathfrak{Z}}
\renewcommand{\H}{\mathcal{H}}
\newcommand{\fH}{\mathfrak{H}}
\newcommand{\bH}{\mathbb{H}}
\newcommand{\id}{\mathrm{id}}
\newcommand{\A}{\mathcal{A}}
% \renewcommand\coprod{\rotatebox[origin=c]{180}{$\prod$}} すでにどこかにある.
\newcommand{\pr}{\mathrm{pr}}
\newcommand{\U}{\mathfrak{U}}
\newcommand{\Map}{\mathrm{Map}}
\newcommand{\dom}{\mathrm{Dom}\;}
\newcommand{\cod}{\mathrm{Cod}\;}
\newcommand{\supp}{\mathrm{supp}\;}
\newcommand{\otherwise}{\mathrm{otherwise}}
\newcommand{\st}{\;\mathrm{s.t.}\;}
\newcommand{\lmd}{\lambda}
\newcommand{\Lmd}{\Lambda}
%%% 線型代数学
\newcommand{\Ker}{\mathrm{Ker}\;}
\newcommand{\Coker}{\mathrm{Coker}\;}
\newcommand{\Coim}{\mathrm{Coim}\;}
\newcommand{\rank}{\mathrm{rank}}
\newcommand{\lcm}{\mathrm{lcm}}
\newcommand{\sgn}{\mathrm{sgn}}
\newcommand{\GL}{\mathrm{GL}}
\newcommand{\SL}{\mathrm{SL}}
\newcommand{\alt}{\mathrm{alt}}
%%% 複素解析学
\renewcommand{\Re}{\mathrm{Re}\;}
\renewcommand{\Im}{\mathrm{Im}\;}
\newcommand{\Gal}{\mathrm{Gal}}
\newcommand{\PGL}{\mathrm{PGL}}
\newcommand{\PSL}{\mathrm{PSL}}
\newcommand{\Log}{\mathrm{Log}\,}
\newcommand{\Res}{\mathrm{Res}\,}
\newcommand{\on}{\mathrm{on}\;}
\newcommand{\hatC}{\hat{\C}}
\newcommand{\hatR}{\hat{\R}}
\newcommand{\PV}{\mathrm{P.V.}}
\newcommand{\diam}{\mathrm{diam}}
\newcommand{\Area}{\mathrm{Area}}
\newcommand{\Lap}{\Laplace}
\newcommand{\f}{\mathbf{f}}
\newcommand{\cR}{\mathcal{R}}
\newcommand{\const}{\mathrm{const.}}
\newcommand{\Om}{\Omega}
\newcommand{\Cinf}{C^\infty}
\newcommand{\ep}{\epsilon}
\newcommand{\dist}{\mathrm{dist}}
\newcommand{\opart}{\o{\partial}}
%%% 解析力学
\newcommand{\x}{\mathbf{x}}
%%% 集合と位相
\renewcommand{\O}{\mathcal{O}}
\renewcommand{\S}{\mathcal{S}}
\renewcommand{\U}{\mathcal{U}}
\newcommand{\V}{\mathcal{V}}
\renewcommand{\P}{\mathcal{P}}
\newcommand{\R}{\mathbb{R}}
\newcommand{\N}{\mathbb{N}}
\newcommand{\C}{\mathbb{C}}
\newcommand{\Z}{\mathbb{Z}}
\newcommand{\Q}{\mathbb{Q}}
\newcommand{\TV}{\mathrm{TV}}
\newcommand{\ORD}{\mathrm{ORD}}
\newcommand{\Tr}{\mathrm{Tr}\;}
\newcommand{\Card}{\mathrm{Card}\;}
\newcommand{\Top}{\mathrm{Top}}
\newcommand{\Disc}{\mathrm{Disc}}
\newcommand{\Codisc}{\mathrm{Codisc}}
\newcommand{\CoDisc}{\mathrm{CoDisc}}
\newcommand{\Ult}{\mathrm{Ult}}
\newcommand{\ord}{\mathrm{ord}}
\newcommand{\maj}{\mathrm{maj}}
%%% 形式言語理論
\newcommand{\REGEX}{\mathrm{REGEX}}
\newcommand{\RE}{\mathbf{RE}}

%%% Fourier解析
\newcommand*{\Laplace}{\mathop{}\!\mathbin\bigtriangleup}
\newcommand*{\DAlambert}{\mathop{}\!\mathbin\Box}
%%% Graph Theory
\newcommand{\SimpGph}{\mathrm{SimpGph}}
\newcommand{\Gph}{\mathrm{Gph}}
\newcommand{\mult}{\mathrm{mult}}
\newcommand{\inv}{\mathrm{inv}}
%%% 多様体
\newcommand{\Der}{\mathrm{Der}}
\newcommand{\osub}{\overset{\mathrm{open}}{\subset}}
\newcommand{\osup}{\overset{\mathrm{open}}{\supset}}
\newcommand{\al}{\alpha}
\newcommand{\K}{\mathbb{K}}
\newcommand{\Sp}{\mathrm{Sp}}
\newcommand{\g}{\mathfrak{g}}
\newcommand{\h}{\mathfrak{h}}
\newcommand{\Exp}{\mathrm{Exp}\;}
\newcommand{\Imm}{\mathrm{Imm}}
\newcommand{\Imb}{\mathrm{Imb}}
\newcommand{\codim}{\mathrm{codim}\;}
\newcommand{\Gr}{\mathrm{Gr}}
%%% 代数
\newcommand{\Ad}{\mathrm{Ad}}
\newcommand{\finsupp}{\mathrm{fin\;supp}}
\newcommand{\SO}{\mathrm{SO}}
\newcommand{\SU}{\mathrm{SU}}
\newcommand{\acts}{\curvearrowright}
\newcommand{\mono}{\hookrightarrow}
\newcommand{\epi}{\twoheadrightarrow}
\newcommand{\Stab}{\mathrm{Stab}}
\newcommand{\nor}{\mathrm{nor}}
\newcommand{\T}{\mathbb{T}}
\newcommand{\Aff}{\mathrm{Aff}}
\newcommand{\rsub}{\triangleleft}
\newcommand{\rsup}{\triangleright}
\newcommand{\subgrp}{\overset{\mathrm{subgrp}}{\subset}}
\newcommand{\Ext}{\mathrm{Ext}}
\newcommand{\sbs}{\subset}\newcommand{\sps}{\supset}
\newcommand{\In}{\mathrm{In}}
\newcommand{\Tor}{\mathrm{Tor}}
\newcommand{\p}{\mathfrak{p}}
\newcommand{\q}{\mathfrak{q}}
\newcommand{\m}{\mathfrak{m}}
\newcommand{\cS}{\mathcal{S}}
\newcommand{\Frac}{\mathrm{Frac}\,}
\newcommand{\Spec}{\mathrm{Spec}\,}
\newcommand{\bA}{\mathbb{A}}
\newcommand{\Sym}{\mathrm{Sym}}
\newcommand{\Ann}{\mathrm{Ann}}
%%% 代数的位相幾何学
\newcommand{\Ho}{\mathrm{Ho}}
\newcommand{\CW}{\mathrm{CW}}
\newcommand{\lc}{\mathrm{lc}}
\newcommand{\cg}{\mathrm{cg}}
\newcommand{\Fib}{\mathrm{Fib}}
\newcommand{\Cyl}{\mathrm{Cyl}}
\newcommand{\Ch}{\mathrm{Ch}}
%%% 数値解析
\newcommand{\round}{\mathrm{round}}
\newcommand{\cond}{\mathrm{cond}}
\newcommand{\diag}{\mathrm{diag}}
%%% 確率論
\newcommand{\calF}{\mathcal{F}}
\newcommand{\X}{\mathcal{X}}
\newcommand{\Meas}{\mathrm{Meas}}
\newcommand{\as}{\;\mathrm{a.s.}} %almost surely
\newcommand{\io}{\;\mathrm{i.o.}} %infinitely often
\newcommand{\fe}{\;\mathrm{f.e.}} %with a finite number of exceptions
\newcommand{\F}{\mathcal{F}}
\newcommand{\bF}{\mathbb{F}}
\newcommand{\W}{\mathcal{W}}
\newcommand{\Pois}{\mathrm{Pois}}
\newcommand{\iid}{\mathrm{i.i.d.}}
\newcommand{\wconv}{\rightsquigarrow}
\newcommand{\Var}{\mathrm{Var}}
\newcommand{\xrightarrown}{\xrightarrow{n\to\infty}}
\newcommand{\au}{\mathrm{au}}
\newcommand{\cT}{\mathcal{T}}
%%% 情報理論
\newcommand{\bit}{\mathrm{bit}}
%%% 積分論
\newcommand{\calA}{\mathcal{A}}
\newcommand{\calB}{\mathcal{B}}
\newcommand{\D}{\mathcal{D}}
\newcommand{\Y}{\mathcal{Y}}
\newcommand{\calC}{\mathcal{C}}
\renewcommand{\ae}{\mathrm{a.e.}\;}
\newcommand{\cZ}{\mathcal{Z}}
\newcommand{\fF}{\mathfrak{F}}
\newcommand{\fI}{\mathfrak{I}}
\newcommand{\E}{\mathcal{E}}
\newcommand{\sMap}{\sigma\textrm{-}\mathrm{Map}}
\DeclareMathOperator*{\argmax}{arg\,max}
\DeclareMathOperator*{\argmin}{arg\,min}
\newcommand{\cC}{\mathcal{C}}
\newcommand{\comp}{\complement}
\newcommand{\J}{\mathcal{J}}
\newcommand{\sumN}[1]{\sum_{#1\in\N}}
\newcommand{\cupN}[1]{\cup_{#1\in\N}}
\newcommand{\capN}[1]{\cap_{#1\in\N}}
\newcommand{\Sum}[1]{\sum_{#1=1}^\infty}
\newcommand{\sumn}{\sum_{n=1}^\infty}
\newcommand{\summ}{\sum_{m=1}^\infty}
\newcommand{\sumk}{\sum_{k=1}^\infty}
\newcommand{\sumi}{\sum_{i=1}^\infty}
\newcommand{\sumj}{\sum_{j=1}^\infty}
\newcommand{\cupn}{\cup_{n=1}^\infty}
\newcommand{\capn}{\cap_{n=1}^\infty}
\newcommand{\cupk}{\cup_{k=1}^\infty}
\newcommand{\cupi}{\cup_{i=1}^\infty}
\newcommand{\cupj}{\cup_{j=1}^\infty}
\newcommand{\limn}{\lim_{n\to\infty}}
\renewcommand{\l}{\mathcal{l}}
\renewcommand{\L}{\mathcal{L}}
\newcommand{\Cl}{\mathrm{Cl}}
\newcommand{\cN}{\mathcal{N}}
\newcommand{\Ae}{\textrm{-a.e.}\;}
\newcommand{\csub}{\overset{\textrm{closed}}{\subset}}
\newcommand{\csup}{\overset{\textrm{closed}}{\supset}}
\newcommand{\wB}{\wt{B}}
\newcommand{\cG}{\mathcal{G}}
\newcommand{\Lip}{\mathrm{Lip}}
\newcommand{\Dom}{\mathrm{Dom}}
%%% 数理ファイナンス
\newcommand{\pre}{\mathrm{pre}}
\newcommand{\om}{\omega}

%%% 統計的因果推論
\newcommand{\Do}{\mathrm{Do}}
%%% 数理統計
\newcommand{\bP}{\mathbb{P}}
\newcommand{\compsub}{\overset{\textrm{cpt}}{\subset}}
\newcommand{\lip}{\textrm{lip}}
\newcommand{\BL}{\mathrm{BL}}
\newcommand{\G}{\mathbb{G}}
\newcommand{\NB}{\mathrm{NB}}
\newcommand{\oR}{\o{\R}}
\newcommand{\liminfn}{\liminf_{n\to\infty}}
\newcommand{\limsupn}{\limsup_{n\to\infty}}
%\newcommand{\limn}{\lim_{n\to\infty}}
\newcommand{\esssup}{\mathrm{ess.sup}}
\newcommand{\asto}{\xrightarrow{\as}}
\newcommand{\Cov}{\mathrm{Cov}}
\newcommand{\cQ}{\mathcal{Q}}
\newcommand{\VC}{\mathrm{VC}}
\newcommand{\mb}{\mathrm{mb}}
\newcommand{\Avar}{\mathrm{Avar}}
\newcommand{\bB}{\mathbb{B}}
\newcommand{\bW}{\mathbb{W}}
\newcommand{\sd}{\mathrm{sd}}
\newcommand{\w}[1]{\widehat{#1}}
\newcommand{\bZ}{\mathbb{Z}}
\newcommand{\Bernoulli}{\mathrm{Bernoulli}}
\newcommand{\Mult}{\mathrm{Mult}}
\newcommand{\BPois}{\mathrm{BPois}}
\newcommand{\fraks}{\mathfrak{s}}
\newcommand{\frakk}{\mathfrak{k}}
\newcommand{\IF}{\mathrm{IF}}
\newcommand{\bX}{\mathbf{X}}
\newcommand{\bx}{\mathbf{x}}
\newcommand{\indep}{\raisebox{0.05em}{\rotatebox[origin=c]{90}{$\models$}}}
\newcommand{\IG}{\mathrm{IG}}
\newcommand{\Levy}{\mathrm{Levy}}
\newcommand{\MP}{\mathrm{MP}}
\newcommand{\Hermite}{\mathrm{Hermite}}
\newcommand{\Skellam}{\mathrm{Skellam}}
\newcommand{\Dirichlet}{\mathrm{Dirichlet}}
\newcommand{\Beta}{\mathrm{Beta}}
\newcommand{\bE}{\mathbb{E}}
\newcommand{\bG}{\mathbb{G}}
\newcommand{\MISE}{\mathrm{MISE}}
\newcommand{\logit}{\mathtt{logit}}
\newcommand{\expit}{\mathtt{expit}}
\newcommand{\cK}{\mathcal{K}}
\newcommand{\dl}{\dot{l}}
\newcommand{\dotp}{\dot{p}}
\newcommand{\wl}{\wt{l}}
%%% 函数解析
\renewcommand{\c}{\mathbf{c}}
\newcommand{\loc}{\mathrm{loc}}
\newcommand{\Lh}{\mathrm{L.h.}}
\newcommand{\Epi}{\mathrm{Epi}\;}
\newcommand{\slim}{\mathrm{slim}}
\newcommand{\Ban}{\mathrm{Ban}}
\newcommand{\Hilb}{\mathrm{Hilb}}
\newcommand{\Ex}{\mathrm{Ex}}
\newcommand{\Co}{\mathrm{Co}}
\newcommand{\sa}{\mathrm{sa}}
\newcommand{\nnorm}[1]{{\left\vert\kern-0.25ex\left\vert\kern-0.25ex\left\vert #1 \right\vert\kern-0.25ex\right\vert\kern-0.25ex\right\vert}}
\newcommand{\dvol}{\mathrm{dvol}}
\newcommand{\Sconv}{\mathrm{Sconv}}
\newcommand{\I}{\mathcal{I}}
\newcommand{\nonunital}{\mathrm{nu}}
\newcommand{\cpt}{\mathrm{cpt}}
\newcommand{\lcpt}{\mathrm{lcpt}}
\newcommand{\com}{\mathrm{com}}
\newcommand{\Haus}{\mathrm{Haus}}
\newcommand{\proper}{\mathrm{proper}}
\newcommand{\infinity}{\mathrm{inf}}
\newcommand{\TVS}{\mathrm{TVS}}
\newcommand{\ess}{\mathrm{ess}}
\newcommand{\ext}{\mathrm{ext}}
\newcommand{\Index}{\mathrm{Index}}
\newcommand{\SSR}{\mathrm{SSR}}
\newcommand{\vs}{\mathrm{vs.}}
\newcommand{\fM}{\mathfrak{M}}
\newcommand{\EDM}{\mathrm{EDM}}
\newcommand{\Tw}{\mathrm{Tw}}
\newcommand{\fC}{\mathfrak{C}}
\newcommand{\bn}{\mathbf{n}}
\newcommand{\br}{\mathbf{r}}
\newcommand{\Lam}{\Lambda}
\newcommand{\lam}{\lambda}
\newcommand{\one}{\mathbf{1}}
\newcommand{\dae}{\text{-a.e.}}
\newcommand{\td}{\text{-}}
\newcommand{\RM}{\mathrm{RM}}
%%% 最適化
\newcommand{\Minimize}{\text{Minimize}}
\newcommand{\subjectto}{\text{subject to}}
\newcommand{\Ri}{\mathrm{Ri}}
%\newcommand{\Cl}{\mathrm{Cl}}
\newcommand{\Cone}{\mathrm{Cone}}
\newcommand{\Int}{\mathrm{Int}}
%%% 圏
\newcommand{\varlim}{\varprojlim}
\newcommand{\Hom}{\mathrm{Hom}}
\newcommand{\Iso}{\mathrm{Iso}}
\newcommand{\Mor}{\mathrm{Mor}}
\newcommand{\Isom}{\mathrm{Isom}}
\newcommand{\Aut}{\mathrm{Aut}}
\newcommand{\End}{\mathrm{End}}
\newcommand{\op}{\mathrm{op}}
\newcommand{\ev}{\mathrm{ev}}
\newcommand{\Ob}{\mathrm{Ob}}
\newcommand{\Ar}{\mathrm{Ar}}
\newcommand{\Arr}{\mathrm{Arr}}
\newcommand{\Set}{\mathrm{Set}}
\newcommand{\Grp}{\mathrm{Grp}}
\newcommand{\Cat}{\mathrm{Cat}}
\newcommand{\Mon}{\mathrm{Mon}}
\newcommand{\CMon}{\mathrm{CMon}} %Comutative Monoid 可換単系とモノイドの射
\newcommand{\Ring}{\mathrm{Ring}}
\newcommand{\CRing}{\mathrm{CRing}}
\newcommand{\Ab}{\mathrm{Ab}}
\newcommand{\Pos}{\mathrm{Pos}}
\newcommand{\Vect}{\mathrm{Vect}}
\newcommand{\FinVect}{\mathrm{FinVect}}
\newcommand{\FinSet}{\mathrm{FinSet}}
\newcommand{\OmegaAlg}{\Omega$-$\mathrm{Alg}}
\newcommand{\OmegaEAlg}{(\Omega,E)$-$\mathrm{Alg}}
\newcommand{\Alg}{\mathrm{Alg}} %代数の圏
\newcommand{\CAlg}{\mathrm{CAlg}} %可換代数の圏
\newcommand{\CPO}{\mathrm{CPO}} %Complete Partial Order & continuous mappings
\newcommand{\Fun}{\mathrm{Fun}}
\newcommand{\Func}{\mathrm{Func}}
\newcommand{\Met}{\mathrm{Met}} %Metric space & Contraction maps
\newcommand{\Pfn}{\mathrm{Pfn}} %Sets & Partial function
\newcommand{\Rel}{\mathrm{Rel}} %Sets & relation
\newcommand{\Bool}{\mathrm{Bool}}
\newcommand{\CABool}{\mathrm{CABool}}
\newcommand{\CompBoolAlg}{\mathrm{CompBoolAlg}}
\newcommand{\BoolAlg}{\mathrm{BoolAlg}}
\newcommand{\BoolRng}{\mathrm{BoolRng}}
\newcommand{\HeytAlg}{\mathrm{HeytAlg}}
\newcommand{\CompHeytAlg}{\mathrm{CompHeytAlg}}
\newcommand{\Lat}{\mathrm{Lat}}
\newcommand{\CompLat}{\mathrm{CompLat}}
\newcommand{\SemiLat}{\mathrm{SemiLat}}
\newcommand{\Stone}{\mathrm{Stone}}
\newcommand{\Sob}{\mathrm{Sob}} %Sober space & continuous map
\newcommand{\Op}{\mathrm{Op}} %Category of open subsets
\newcommand{\Sh}{\mathrm{Sh}} %Category of sheave
\newcommand{\PSh}{\mathrm{PSh}} %Category of presheave, PSh(C)=[C^op,set]のこと
\newcommand{\Conv}{\mathrm{Conv}} %Convergence spaceの圏
\newcommand{\Unif}{\mathrm{Unif}} %一様空間と一様連続写像の圏
\newcommand{\Frm}{\mathrm{Frm}} %フレームとフレームの射
\newcommand{\Locale}{\mathrm{Locale}} %その反対圏
\newcommand{\Diff}{\mathrm{Diff}} %滑らかな多様体の圏
\newcommand{\Mfd}{\mathrm{Mfd}}
\newcommand{\LieAlg}{\mathrm{LieAlg}}
\newcommand{\Quiv}{\mathrm{Quiv}} %Quiverの圏
\newcommand{\B}{\mathcal{B}}
\newcommand{\Span}{\mathrm{Span}}
\newcommand{\Corr}{\mathrm{Corr}}
\newcommand{\Decat}{\mathrm{Decat}}
\newcommand{\Rep}{\mathrm{Rep}}
\newcommand{\Grpd}{\mathrm{Grpd}}
\newcommand{\sSet}{\mathrm{sSet}}
\newcommand{\Mod}{\mathrm{Mod}}
\newcommand{\SmoothMnf}{\mathrm{SmoothMnf}}
\newcommand{\coker}{\mathrm{coker}}

\newcommand{\Ord}{\mathrm{Ord}}
\newcommand{\eq}{\mathrm{eq}}
\newcommand{\coeq}{\mathrm{coeq}}
\newcommand{\act}{\mathrm{act}}

%%%%%%%%%%%%%%% 定理環境(足助先生ありがとうございます) %%%%%%%%%%%%%%%

\everymath{\displaystyle}
\renewcommand{\proofname}{\bf [証明]}
\renewcommand{\thefootnote}{\dag\arabic{footnote}} %足助さんからもらった.どうなるんだ?
\renewcommand{\qedsymbol}{$\blacksquare$}

\renewcommand{\labelenumi}{(\arabic{enumi})} %(1),(2),...がデフォルトであって欲しい
\renewcommand{\labelenumii}{(\alph{enumii})}
\renewcommand{\labelenumiii}{(\roman{enumiii})}

\newtheoremstyle{StatementsWithStar}% ?name?
{3pt}% ?Space above? 1
{3pt}% ?Space below? 1
{}% ?Body font?
{}% ?Indent amount? 2
{\bfseries}% ?Theorem head font?
{\textbf{.}}% ?Punctuation after theorem head?
{.5em}% ?Space after theorem head? 3
{\textbf{\textup{#1~\thetheorem{}}}{}\,$^{\ast}$\thmnote{(#3)}}% ?Theorem head spec (can be left empty, meaning ‘normal’)?
%
\newtheoremstyle{StatementsWithStar2}% ?name?
{3pt}% ?Space above? 1
{3pt}% ?Space below? 1
{}% ?Body font?
{}% ?Indent amount? 2
{\bfseries}% ?Theorem head font?
{\textbf{.}}% ?Punctuation after theorem head?
{.5em}% ?Space after theorem head? 3
{\textbf{\textup{#1~\thetheorem{}}}{}\,$^{\ast\ast}$\thmnote{(#3)}}% ?Theorem head spec (can be left empty, meaning ‘normal’)?
%
\newtheoremstyle{StatementsWithStar3}% ?name?
{3pt}% ?Space above? 1
{3pt}% ?Space below? 1
{}% ?Body font?
{}% ?Indent amount? 2
{\bfseries}% ?Theorem head font?
{\textbf{.}}% ?Punctuation after theorem head?
{.5em}% ?Space after theorem head? 3
{\textbf{\textup{#1~\thetheorem{}}}{}\,$^{\ast\ast\ast}$\thmnote{(#3)}}% ?Theorem head spec (can be left empty, meaning ‘normal’)?
%
\newtheoremstyle{StatementsWithCCirc}% ?name?
{6pt}% ?Space above? 1
{6pt}% ?Space below? 1
{}% ?Body font?
{}% ?Indent amount? 2
{\bfseries}% ?Theorem head font?
{\textbf{.}}% ?Punctuation after theorem head?
{.5em}% ?Space after theorem head? 3
{\textbf{\textup{#1~\thetheorem{}}}{}\,$^{\circledcirc}$\thmnote{(#3)}}% ?Theorem head spec (can be left empty, meaning ‘normal’)?
%
\theoremstyle{definition}
 \newtheorem{theorem}{定理}[section]
 \newtheorem{axiom}[theorem]{公理}
 \newtheorem{corollary}[theorem]{系}
 \newtheorem{proposition}[theorem]{命題}
 \newtheorem*{proposition*}{命題}
 \newtheorem{lemma}[theorem]{補題}
 \newtheorem*{lemma*}{補題}
 \newtheorem*{theorem*}{定理}
 \newtheorem{definition}[theorem]{定義}
 \newtheorem{example}[theorem]{例}
 \newtheorem{notation}[theorem]{記法}
 \newtheorem*{notation*}{記法}
 \newtheorem{assumption}[theorem]{仮定}
 \newtheorem{question}[theorem]{問}
 \newtheorem{counterexample}[theorem]{反例}
 \newtheorem{reidai}[theorem]{例題}
 \newtheorem{ruidai}[theorem]{類題}
 \newtheorem{problem}[theorem]{問題}
 \newtheorem{algorithm}[theorem]{算譜}
 \newtheorem*{solution*}{\bf{[解]}}
 \newtheorem{discussion}[theorem]{議論}
 \newtheorem{remark}[theorem]{注}
 \newtheorem{remarks}[theorem]{要諦}
 \newtheorem{image}[theorem]{描像}
 \newtheorem{observation}[theorem]{観察}
 \newtheorem{universality}[theorem]{普遍性} %非自明な例外がない.
 \newtheorem{universal tendency}[theorem]{普遍傾向} %例外が有意に少ない.
 \newtheorem{hypothesis}[theorem]{仮説} %実験で説明されていない理論.
 \newtheorem{theory}[theorem]{理論} %実験事実とその(さしあたり)整合的な説明.
 \newtheorem{fact}[theorem]{実験事実}
 \newtheorem{model}[theorem]{模型}
 \newtheorem{explanation}[theorem]{説明} %理論による実験事実の説明
 \newtheorem{anomaly}[theorem]{理論の限界}
 \newtheorem{application}[theorem]{応用例}
 \newtheorem{method}[theorem]{手法} %実験手法など,技術的問題.
 \newtheorem{history}[theorem]{歴史}
 \newtheorem{usage}[theorem]{用語法}
 \newtheorem{research}[theorem]{研究}
 \newtheorem{shishin}[theorem]{指針}
 \newtheorem{yodan}[theorem]{余談}
 \newtheorem{construction}[theorem]{構成}
% \newtheorem*{remarknonum}{注}
 \newtheorem*{definition*}{定義}
 \newtheorem*{remark*}{注}
 \newtheorem*{question*}{問}
 \newtheorem*{problem*}{問題}
 \newtheorem*{axiom*}{公理}
 \newtheorem*{example*}{例}
 \newtheorem*{corollary*}{系}
 \newtheorem*{shishin*}{指針}
 \newtheorem*{yodan*}{余談}
 \newtheorem*{kadai*}{課題}
%
\theoremstyle{StatementsWithStar}
 \newtheorem{definition_*}[theorem]{定義}
 \newtheorem{question_*}[theorem]{問}
 \newtheorem{example_*}[theorem]{例}
 \newtheorem{theorem_*}[theorem]{定理}
 \newtheorem{remark_*}[theorem]{注}
%
\theoremstyle{StatementsWithStar2}
 \newtheorem{definition_**}[theorem]{定義}
 \newtheorem{theorem_**}[theorem]{定理}
 \newtheorem{question_**}[theorem]{問}
 \newtheorem{remark_**}[theorem]{注}
%
\theoremstyle{StatementsWithStar3}
 \newtheorem{remark_***}[theorem]{注}
 \newtheorem{question_***}[theorem]{問}
%
\theoremstyle{StatementsWithCCirc}
 \newtheorem{definition_O}[theorem]{定義}
 \newtheorem{question_O}[theorem]{問}
 \newtheorem{example_O}[theorem]{例}
 \newtheorem{remark_O}[theorem]{注}
%
\theoremstyle{definition}
%
\raggedbottom
\allowdisplaybreaks
\usepackage[math]{anttor}
\begin{document}
\tableofcontents

\chapter{級数論}



\chapter{Fourier級数導入}

\section{Hilbert空間の正規直交系}

\begin{notation}\mbox{}
    \begin{enumerate}
        \item $L_{2\pi}:=\Brace{f:\R\to\C\mid fは\text{Lebesgue}可測で,周期2\pi を持つ}$.
        \item $L^p_{2\pi}:=\Brace{f\in L_{2\pi}\mid f|_{(-\pi,\pi]}\in L^p((-\pi,\pi])}$.
        \item $C_{2\pi}:=\Brace{f\in C^0(\R,\C)\mid fは周期2\pi を持つ}$.
        \item $C^k_{2\pi}:=\Brace{f\in C^k(\R,\C)\mid k\in\N\cup\{\infty\}, fは周期2\pi を持つ}$.
    \end{enumerate}
\end{notation}

\section{Fourier係数の性質}

\begin{proposition}[表示]
    $f(t)=\sum^\infty_{n=-\infty}c_ne^{int}$が$t\in(-\pi,\pi]$上一様収束する時,
    \[c_n=\frac{1}{2\pi}\int^\pi_{-\pi}f(t)e^{-int}dt\]
    が必要.
\end{proposition}
\begin{proof}
    $e^{int}$は正規直交系だから,
    \begin{align*}
        (f,e^{int})&=\paren{\lim_{N\to\infty}S_N[f](t),e^{int}}\\
        &=\lim_{N\to\infty}(S_N[f](t),e^{int})&内積の連続性\ref{lemma-norm-is-continuous}\\
        &=\lim_{N\to\infty}\sum^N_{\nu=-N}c_\nu(e^{i\nu t},e^{int})&内積の線形性\\
        &=c_n.
    \end{align*}
\end{proof}

\begin{lemma}\mbox{}
    \begin{enumerate}
        \item $f(t)=f(-t)$のとき,$c_n[f]=c_{-n}[f]$.したがって,足し合わせると$\cos$が出てくる.
        \item $f(t)=-f(-t)$のとき,$c_n[f]=-c_{-n}[f]$.したがって,足し合わせると$\sim$が出てくる.
        \item $f(t)\in\R$ならば,$\o{c_n[f]}=c_{-n}[f]$.
    \end{enumerate}
\end{lemma}
\begin{proof}\mbox{}
    \begin{enumerate}
        \item \begin{align*}
            c_n[f]&=\int^\pi_{-\pi}f(t)e^{int}dt\\
            &=\int_{\pi}^{-\pi}f(-x)e^{-inx}(-dx)\\
            &=\int_{-\pi}^{\pi}f(x)e^{i(-n)x}dx=c_{-n}[f].
        \end{align*}
        \item 同様.
        \item \[\o{c_n[f]}=\int^\pi_{-\pi}f(t)e^{-int}dt=c_{-n}[f].\]
        あとは積分と複素共役の可換性についてであるが,実部と虚部に分けて考えると,積分の線形性より従う.
    \end{enumerate}
\end{proof}

\begin{example}
    \[f(x)=\frac{1}{\frac{5}{4}+\cos x}\]
    のFourier展開を考える.
\end{example}

\section{Riemann-Lebesgueの定理とFourier係数の減衰}

\begin{theorem}[Riemann-Lebesgue]
    任意の$f\in L^1(\R)$に関して,
    \[\F[f](\lambda)=\hat{f}(\lambda):=\int^\infty_{-\infty}f(t)e^{-it\lambda}dt\xrightarrow{\abs{\lambda}\to\infty}0.\footnote{極限の取り方が対称なので,指数の符号はあまり問題でない.}\]
\end{theorem}
\begin{proof}\mbox{}
    \begin{enumerate}
        \item $f=\chi_{[a,b]}\;(-\infty<a<b<\infty)$のとき,
        \begin{align*}
            \abs{\hat{f}(\lambda)}&=\Abs{\int^b_ae^{-it\lambda}dt}\\
            &=\Abs{\frac{e^{-ib\lambda}-e^{-ia\lambda}}{i\lambda}}\le\frac{2}{\abs{\lambda}}\xrightarrow{\abs{\lambda}\to\infty}0.
        \end{align*}
        \item $f$が単関数のとき,同様.
        \item $S:=\Brace{s=\sum^n_{i=1}b_i\chi_{[a_i,b_i]}\in L^1(\R)\mid b_i\in\C,-\infty<a<b<\infty,n\in\N}$は$L^1(\R)$で$\norm{-}_1$について稠密だから,一般の$f\in L^1(\R)$についても同様に示せる.
        
        実際,任意の$f\in L^1(\R)$と$\ep>0$に対して,$g\in S$が存在して,$\norm{f-g}_1=\int\abs{f(x)-g(x)}dx<\ep$を満たす.
        いま,(2)より$\exists_{N\in\N}\;\abs{\lambda}>N\Rightarrow\Abs{\int g(t)e^{it\lambda}dt}<\ep$であるから,
        \[\Abs{f(t)e^{it\lambda}dt}\le\int\abs{f(t)-g(t)}dt+\Abs{\int g(t)e^{it\lambda}dt}<2\ep\quad(\abs{\lambda}>N).\]
    \end{enumerate}
\end{proof}
\begin{remarks}
    稠密だと限りなく$S$に近い点が取れるから,このように評価してしまえば収束性については全く同じ結論を得るのか.
\end{remarks}

\begin{corollary}
    任意の可積分な周期関数$f\in L^1_{2\pi}$に対して,Fourier係数は収束する:$c_n[f]\to0\;(\abs{n}\to\infty)$.
\end{corollary}
\begin{proof}
    定理と同様の手順で示せる.
\end{proof}

\begin{corollary}[減衰の精緻化]
    $f$が$C^K$級のとき,
    \begin{enumerate}
        \item $\forall_{j=1,\cdots,K}\;\forall_{n\in\Z}\;(in)^jc_n[f]=c_n[f^{(j)}]$.
        \item $\forall_{n\in\Z\setminus\{0\}}\;\abs{c_n[f]}\le\frac{1}{\abs{n}^K}\norm{f^{(K)}}_\infty$.
        \item $n^K\cdot c_n[f]\xrightarrow{\abs{n}\to\infty}0$.
    \end{enumerate}
\end{corollary}
\begin{proof}\mbox{}
    \begin{enumerate}
        \item 部分積分により,
        \begin{align*}
            c_n[f]&=\frac{1}{2\pi}\Square{f(t)\frac{e^{-int}}{-in}}^\pi_{-\pi}-\frac{1}{2\pi}\int^\pi_{-\pi}f'(t)\frac{e^{-int}}{-in}dt\\
            &=\frac{1}{2\pi}\frac{f(\pi)}{in}2\sin(n\pi)+\frac{1}{in}\frac{1}{2\pi}\int^\pi_{-\pi}f'(t)e^{-int}dt=\frac{c_n[f']}{in}.
        \end{align*}
        これを繰り返すとわかる.
        \item \begin{align*}
            \abs{c_n[f]}&\le\frac{1}{\abs{in}^K}\frac{1}{2\pi}\int^\pi_{-\pi}\abs{f^{(k)}(t)}\abs{e^{-int}}dt\\
            &\le\frac{1}{\abs{n}^K}\frac{1}{2\pi}\int^\pi_{-\pi}\norm{f^{(k)}}_\infty dt=\frac{1}{\abs{n}^K}\norm{f^{(k)}}_\infty.
        \end{align*}
        \item (3)と系から明らか.
    \end{enumerate}
\end{proof}
\begin{remarks}
    $k$回連続微分可能な関数のFourier係数は,多項式$n^{-k}$よりも速く収束する.
    これにより,$f\in C^2_{2\pi}$に対しては,Fourier級数$\sum^\infty_{-\infty}c_n[f]e^{inx}$が絶対収束することがわかった.
    しかし,その収束先が$f$に一致するかは議論が必要である.
\end{remarks}

\section{Dirichlet核}

\begin{definition}[Dirichlet kernel]
    Dirichlet核を
    \[D_K(t):=\frac{1}{2\pi}\sum^K_{n=-K}e^{int}\]
    と定めると,Fourier和は
    \begin{align*}
        S_K[f](x)&=\sum^K_{n=-K}c_n[f]e^{inx}\\
        &=\int^\pi_{-\pi}f(t)\frac{1}{2\pi}\sum^K_{n=-K}e^{in(x-t)}dt\\
        &=\int^{\pi+x}_{-\pi+x}f(x-t)D_K(t)dt
    \end{align*}
    と表せる.
\end{definition}

\chapter{関数解析的なFourier級数論}

\begin{quotation}
    Hilbert空間の同型$H\simeq_\Hilb l^2(A)$をなす場合に限れば,Fourier級数論は極めて明晰に展開される.
    これは,トーラス群$\bT$のコンパクト性による.
    理論の動機がわかるように,三角級数論から精緻化していく.
    三角級数は,連続関数に対する多項式が一様近似可能であるように,任意の周期関数を一様近似可能である.
    この手法は任意の正規直交系に対して一般化でき,これをFourier級数というのである.

    一方で,$L^1(\bT)$などのFourier級数論は,内積が使えないBanach空間となるので,Banach空間論を用いたより精緻な議論をする.
\end{quotation}

\section{三角級数論}

\begin{notation}
    $\T\subset\C$を,絶対値が$1$の複素数がなす単位円とする.
\end{notation}
\begin{discussion}
    今後,$T$上の関数を考えるが,$T$上の関数$F$は,$\R$上の周期$2\pi$を持つ関数$f$の全体と,$f(t)=F(e^{it})$によって対応を持つ.
    以降,$L(T)$と$L_{2\pi}$とを同一視し,$f\in L_{2\pi}$に対して$f(e^{it})$を$f(t)$と略記する.
\end{discussion}

\begin{notation}
    Lebesgue可測で周期$2\pi$を持つ$\R$上の複素関数全体の集合であって,
    \[\norm{f}_p:=\paren{\frac{1}{2\pi}\int^\pi_{-\pi}\abs{f(t)}^pdt}^{1/p}\]
    が有限になるものを$L^p(T)\;(1\le p<\infty)$で表す.
    いま,$L^p(T)$は,$\mu$をLebesgue測度として,$L^p(T,\mu)$のノルムを$2\pi$で割ったものとなっている.
    これは$\norm{1}_p=1$となるための便宜である.
    一方で,$L^\infty(T)=L^\infty(T,\mu)$となる.
    $C(T)$は一様ノルムを備え,$C_b(T)$と一致する.
\end{notation}

\subsection{三角級数の定義}

\begin{tcolorbox}[colframe=ForestGreen, colback=ForestGreen!10!white,breakable,colbacktitle=ForestGreen!40!white,coltitle=black,fonttitle=\bfseries\sffamily,
title=]
    $2\pi$周期関数といえば三角多項式である.これによって任意の周期関数を近似する.
\end{tcolorbox}

\begin{definition}[trigonometric polynomials]
    次の形を持つ有限和を,\textbf{三角多項式}という.
    \[f(t)=a_0+\sum^N_{n=1}a_n\cos nt+b_n\sin nt=\sum^N_{n=-N}c_ne^{int}\quad(t\in\R,a_n,b_n,c_n\in\C)\]
\end{definition}

\begin{notation}
    $u_n(t):=e^{int}\;(n\in\Z)$と表す.
    \[(f,g):=\frac{1}{2\pi}\int^\pi_{-\pi}f(t)\o{g(t)}dt\]
    とすると,これは$L^2(T)$上に内積を定め,ノルムと整合する.
\end{notation}

\begin{lemma}[trigonometric system]
    $(u_n)_{n\in\Z}$は$L^2(T)$の正規直交系である:
    \[(u_n,u_m)=\frac{1}{2\pi}\int^\pi_{-\pi}e^{i(n-m)t}dt=\begin{cases}
        1,&n=m,\\
        0,&n\ne m.
    \end{cases}\]
    これを三角多項式系という.
\end{lemma}

\subsection{三角多項式系の稠密性}

\begin{tcolorbox}[colframe=ForestGreen, colback=ForestGreen!10!white,breakable,colbacktitle=ForestGreen!40!white,coltitle=black,fonttitle=\bfseries\sffamily,
title=]
    実は$\{u_n\}\subset L^2(T)$は稠密な部分空間を生成する.これが示せれば,$\{u_n\}$が基底であること(正規直交系としての極大性)を得る.
    これは,有界閉区間$I\subset\R$内について,$C(I)$内で多項式系は稠密であること(Stone-Weierstrass)を,$I$上の周期関数に限れば三角多項式系ですでに稠密であるという消息に精緻化したものと見れる.
\end{tcolorbox}

\begin{theorem}
    任意の$f\in C(T)$に対して,三角多項式$P$が存在して一様に近似できる:
    $\forall_{f\in C(T)}\;\forall_{\ep>0}\;\exists_{P}\;\forall_{t\in\R}\abs{f(t)-P(t)}<\ep$.
    特に,三角多項式系は$L^2(T)$で稠密である.
\end{theorem}
\begin{proof}
    $T$はコンパクトであるから$C_c(T)=C(T)$で,$C(T)$は$L^2(T)$上稠密である.
\end{proof}

\subsection{Fourier級数の言葉による一般化}

\begin{theorem}[Fejer]
    $f\in C(T)$の定めるFourier級数の部分和の算術平均は,$f$に一様収束する.
\end{theorem}

\section{Fourier級数}

\begin{tcolorbox}[colframe=ForestGreen, colback=ForestGreen!10!white,breakable,colbacktitle=ForestGreen!40!white,coltitle=black,fonttitle=\bfseries\sffamily,
title=]
    この三角級数についての研究結果の精緻化は,Fourier級数の言葉を用いてなされる.
\end{tcolorbox}

\begin{definition}[Fourier coefficient, Fourier series, partial sum]
    $f\in L^1(\T)$について,
    \begin{enumerate}
        \item $\wh{f}(n):=\frac{1}{2\pi}\int^\pi_{-\pi}f(t)e^{-int}dt\;(n\in\Z)$を\textbf{Fourier係数}という.
        \item $f$の定める\textbf{Fourier級数}とは,$\sum^\infty_{-\infty}\wh{f}(n)e^{int}$をいう.
        \item Fourier級数の部分和を$s_N(t):=\sum^N_{-N}\wh{f}(n)e^{int}$で表す.
    \end{enumerate}
\end{definition}
\begin{remarks}[Fourier係数とはなにか]
    三角多項式$f(t)=\sum^N_{n=-N}c_ne^{int}$を見ると,上述の部分和の定義は自然である.
    また,$c_n=\frac{1}{2\pi}\int^\pi_{-\pi}f(t)e^{-int}dt$と表せる.これは,$e^{int}$の係数だけ定数に落ちるので抽出可能で,他の項は$2\pi$周期を持つ(あるいは複素平面上の周回積分な)ので積分すると$0$である.
    これは実は正規直交基底$(u_n(t)=e^{int})_{n\in\Z}$に対して,$c_n=(f|u_n)$となっている.
    一般に,$(u_\al)_{\al\in A}$を正規直交基底とするHilbert空間$H$の元$x\in H$に対して,$\wh{x}(\al):=(x,u_\al)\;(\al\in A)$によって定まる族$(\wh{x}(\al))_{\al\in A}$を$x$の\textbf{Fourier係数}という.
\end{remarks}

\section{Fourier級数とは正規直交基底論に他ならない}

\begin{tcolorbox}[colframe=ForestGreen, colback=ForestGreen!10!white,breakable,colbacktitle=ForestGreen!40!white,coltitle=black,fonttitle=\bfseries\sffamily,
title=めっちゃ幾何学的だしHilbert空間的な行為!]
    直交系$\{u_\al\}_{\al\in A}$の部分列$(u_\al)_{\al\in F}$が定める($x\in H$の)Fourier部分和は,$M_F$上におけるその点$x$の最良の近似になる.
    ということは,その極限たるFourier級数は,$x$自身に収束することが期待される.

    このように,Fourier変換の理論とは,直交系に対して,それぞれの成分を考えて足し合わせるという極めてEuclid空間的な発想の,無限化によって複雑に思えるようになっただけである.
\end{tcolorbox}

\begin{lemma}[有限な正規直交系についての観察]\label{lemma-finite-Fourier}
    $\{u_\al\}_{\al\in A}$をHilbert空間$H$の直交系とし,$F\subset A$を有限部分集合とする.
    $\{u_\al\}_{\al\in F}$の生成する部分空間を$M_F$で表す.
    \begin{enumerate}
        \item $\varphi:A\to\C$は$F$内に台を持つとする.このとき,$y=\sum_{\al\in F}\varphi(\al)u_\al\in M_F$について,
        \begin{enumerate}[(a)]
            \item $\forall_{\al\in A}\;\wh{y}(\al)=\varphi(\al)$を満たし,
            \item $\norm{y}^2=\sum_{\al\in F}\abs{\varphi(a)}^2$を満たす.
        \end{enumerate}
        \item $x\in H$とし,$F$の定める部分和を$s_F(x)=\sum_{\al\in F}\wh{x}(\al)u_\al$とする.このとき,
        \begin{enumerate}[(a)]
            \item $\forall_{s\in M_F}\;s\ne s_F(x)\Rightarrow\norm{x-s_F(x)}<\norm{x-s}$で,
            \item $\sum_{\al\in F}\abs{\wh{x}(\al)}^2\le\norm{x}^2$.
        \end{enumerate}
    \end{enumerate}
\end{lemma}
\begin{proof}\mbox{}
    \begin{enumerate}
        \item $(u_\al)$が正規直交基底であることから従う.
        \begin{enumerate}[(a)]
            \item $\wh{y}(\al)=(y,u_\al)=\varphi(\al)$.
            \item $y$は有限和で表せているから,ノルムの定義$\norm{y}^2=(y|y)=\paren{\sum_{\al\in F}\wh{y}(\al)u_\al,\sum_{\al\in F}\wh{y}(\al)u_\al}=\sum_{\al\in F}\abs{\wh{y}(\al)}^2$より.
        \end{enumerate}
        \item 
        \begin{enumerate}[(a)]
            \item $x\in H$を任意に取る.その$F$に関するFourier部分和を$s_F:=s_F(x)\in M_F$とすると,このFourier係数は
            \[\wh{s_F}(\al)=(s_F,u_\al)=\wh{x}(\al)\]
            に他ならないわけだから,$x$と$s_F$とでは,$(u_\al)_{\al\in F}$に関しては係数は一致する.
            すなわち,$x-s_F$なる項は,$M_F$と直交する:$\forall_{\al\in F}\;x-s_F\perp u_\al$.

            任意の$s\in M_F$について,$s_F-s\in M_F$だから,これも$(x-s_F)\perp(s_F-s)$であるから,Pythagorasの恒等式より,
            \[\norm{x-s}^2=\norm{(x-s_F)+(s_F-s)}^2=\norm{x-s_F}^2+\norm{s_F-s}^2.\]
            \item 特に$s=0$の場合を考えると,$\norm{x-s_F}^2+\norm{s_F}^2=\norm{x}^2$.よって,$\norm{s_F}^2=\sum_{\al\in F}\abs{\wh{x}(\al)}^2\le\norm{x}^2$.
        \end{enumerate}
    \end{enumerate}
\end{proof}
\begin{remarks}[Fourier級数の動作原理]\mbox{}
    \begin{enumerate}
        \item つまり,有限の範囲では,正規直交基底$(u_\al)$の有限部分に対して,そのFourier係数とは,その基底に対する展開係数に他ならず,ノルムについてもEuclid空間で見慣れたPythagorasの定理が成り立つ.
        さらに,$P$を正規直交系$(u_\al)$の有限線型結合全体のなす部分空間とすると,Fourier係数を定める変換$\F:P\iso C_c(A)=:l^2_c(A)$は等長同型であることも述べている.
        \item は,$x\in H$のFourier級数の(任意の)部分和$s_F$は,直交関係$(x-s_F)\perp M_F$を持つ.すなわち,部分和を考えている範囲$F\subset A$が定める部分空間$M_F$内の,$x$に対する最良の近似である.
        これがFourier級数の原理である.
    \end{enumerate}
\end{remarks}

\section{無限化の準備}

\begin{tcolorbox}[colframe=ForestGreen, colback=ForestGreen!10!white,breakable,colbacktitle=ForestGreen!40!white,coltitle=black,fonttitle=\bfseries\sffamily,
title=]
    この有限範囲の観察を無限化することを考えるが,その手順は添字集合$A$上の数え上げ測度に関するLebesgue積分に従う.
    $\sum_{\al\in A}\varphi(\al)$を,有限な部分和の上限とする定義の仕方は奇妙に思えるかもしれないが,これはLebesgue積分の定義に他ならない.
\end{tcolorbox}

\begin{definition}[無限和の定義]\label{def-infinite-sum}
    添字集合$A$上の数え上げ測度を$\mu$と表し,その上の$p$乗可積分関数全体の空間を$l^p(A):=L^p(\mu)$で表す.
    $l^2(A)$は,$(\varphi,\psi):=\sum_{\al\in A}\varphi(\al)\o{\psi(\al)}$を内積として,Hilbert空間を定める.
    また,$l^2(A)=L^2(\mu)$であり,$A$上のコンパクト集合とは有限集合に他ならないから,有限な台を持つ関数の全体は$l^2(A)$上で稠密である.
    特に,$\varphi\in l^2(A)$ならば,集合$\Brace{\al\in A\mid \varphi(\al)\ne0}$は高々可算である.

    故に,$\varphi(\al)\in[0,\infty]$について,$\sum_{\al\in A}\varphi(\al):=\sup_{F\subset A,\abs{F}<\infty}\sum_{\al\in F}\varphi(\al)$とする.
\end{definition}
\begin{proof}
    $A_n:=\Brace{\al\in A\mid\abs{\varphi(a)}>\frac{1}{n}}$とおく.このとき,
    \[\sum_{\al\in A_n}\abs{n\varphi(\al)}^2\le n^2\sum_{\al\in A}\abs{\varphi(\al)}^2\in\R\]
    より,各$A_n$は有限集合だから,$\Brace{\al\in A\mid \varphi(\al)\ne0}=\cup_{n\in\N}A_n$は高々可算である.
\end{proof}

\begin{lemma}[連続延長が等長同型であり続けるための条件]\label{lemma-continuous-extension-of-isometry}
    $X,Y$を距離空間とし,$f:X\to Y$を連続写像とする.このとき,さらに次の3条件が満たされれば,$f$は等長同型である.
    \begin{enumerate}
        \item $X$は完備である.
        \item $X$の稠密部分集合$X_0$が存在して,$f|_{X_0}$は等長である.
        \item $f(X_0)$は$Y$上稠密である.
    \end{enumerate}
\end{lemma}
\begin{remark}
    $f$が全写であるという結論だけでも十分強い.
\end{remark}

\section{Hilbert空間論}

\begin{tcolorbox}[colframe=ForestGreen, colback=ForestGreen!10!white,breakable,colbacktitle=ForestGreen!40!white,coltitle=black,fonttitle=\bfseries\sffamily,
title=]
    可分Hilbert空間$H$の正規直交基底$(u_\al)_{\al\in A}$が定めるFourier係数の対応$H\to l^2(A)$は等長同型である.
    そして,Fourier変換は$H$の内積と$l^2(A)$の内積とを結びつける:Parsevalの恒等式.
    これは正規直交基底であることの特徴付けにもなる.
\end{tcolorbox}

\begin{theorem}[Bessel inequality, Riesz-Fischer theorem]\label{thm-Bessel-and-Riesz-Fischer}
    $(u_\al)_{\al\in A}$を$H$の正規直交系とし,$P$を$(u_\al)$が有限生成する空間(有限線型和全体のなす空間)とする.
    \begin{enumerate}
        \item $\forall_{x\in H}\;\sum_{\al\in A}\abs{\wh{x}(\al)}^2\le\norm{x}^2$.
        \item Fourier係数の対応
        \[\xymatrix@R-2pc{
            \F:H\ar[r]&l^2(A)\\
            \rotatebox[origin=c]{90}{$\in$}&\rotatebox[origin=c]{90}{$\in$}\\
            x\ar@{|->}[r]&\F[x]:=\wh{x}
        }\]
        は連続な線型写像で,$\o{P}$への制限は等長同型$\o{P}\simeq l^2(A)$を定める.
    \end{enumerate}
\end{theorem}
\begin{proof}\mbox{}
    \begin{enumerate}
        \item 有限の場合の$\sum_{\al\in F}\abs{\wh{x}(\al)}^2\le\norm{x}^2$という消息\ref{lemma-finie-Fourier}(2)(b)と,無限和の定義\ref{def-infinite-sum}より,
        $\sum_{\al\in A}\abs{\wh{x}(\al)}^2=\sup_{F\subset A,\abs{F}<\infty}\sum_{\al\in F}\abs{\wh{x}(\al)}^2\le\norm{x}^2$と従う.
        \item 
        \begin{enumerate}[(a)]
            \item まず,$\F$のwell-defined性は(1)のBesselの不等式により,$\wh{x}:A\to\R$はたしかに(数え上げ測度について)二乗可積分である.
            \item 線形性は,Fourier係数は内積を通じて定義しているので明らか:$\F[ax+by]=\wh{ax+by}=a\wh{x}+b\wh{y}=a\F[x]+b\F[y]$.
            \item 連続性は(1)のBesselの不等式により,$\norm{\F[y]-\F[x]}_2=\norm{\wh{y}-\wh{x}}_2\le\norm{y-x}$から従う.
            \item 有限の場合の観察\ref{lemma-finite-Fourier}(1)より,$\F$は任意の$P$の元を,$C_c(A)$上に全写に写し,さらに等長同型である.
            \item $\o{P}$は完備空間$H$の閉部分集合であるから完備,$l^2_c(A)=C_c(A)$は$l^2(A)$内で稠密なので,
            よって補題\ref{lemma-continuous-extension-of-isometry}より,連続な延長$\o{\F}:\o{P}\to l^2(A)$は等長同型である.
        \end{enumerate}
    \end{enumerate}
\end{proof}
\begin{remarks}
    Besselの不等式は,有限の場合にてFourier部分和は$M_F$への射影であると観測した\ref{def-infinite-sum}から,(射影がノルム減少的なので)長さは当然短くなる.
    無限和はその上限と定義したので,不等号は保存される,というのみの話である.

\end{remarks}

\begin{remark}
    証明の(2)(e)部分は$L^2(T)$が稠密である(Hilbert空間である)ことによるため,$L^p$空間が完備であるという事実をRiesz-Fisherの定理とも呼ばれている.
\end{remark}

\begin{theorem}[Parseval's identity:正規直交基底の特徴付け]\label{thm-Parseval}
    $(u_\al)_{\al\in A}$を$H$の正規直交系する.このとき,次の4条件は同値:
    \begin{enumerate}
        \item $(u_\al)$は完全である/基底である:$H$内の極大な正規直交系である.
        \item $(u_\al)$の有限線型結合全体の集合$P$は$H$上稠密である.
        \item $\forall_{x\in H}\;\sum_{\al\in A}\abs{\wh{x}(\al)}^2=\norm{x}^2$.
        \item (Parseval's identity) $\forall_{x,y\in H}\;\sum_{\al\in A}\wh{x}(\al)\o{\wh{y}(\al)}=(x,y)$.\footnote{右辺は$\wh{x},\wh{y}\in l^2(A)$の内積である.}
    \end{enumerate}
\end{theorem}
\begin{proof}\mbox{}
    \begin{description}
        \item[(1)$\Rightarrow$(2)] 
        $P$は稠密でないと仮定すると,$P^\perp\ne0$は零でない元を含む(直交分解定理の系).
        よって,このうちノルムが$1$であるものを$\{u_\al\}_{\al\in A}$に追加した集合は再び正規直交系で,極大性に矛盾する.
        \item[(2)$\Rightarrow$(3)] 
        Riesz-Fischerの定理\ref{thm-Bessel-and-Riesz-Fischer}(2)より,$H$と$l^2(A)$とは等長同型である:$\norm{\wh{x}}^2_2=\norm{x}^2$.
        \item[(3)$\Rightarrow$(4)] $H$も$l^2(A)$もHilbert空間であるから,
        それぞれの空間で極化恒等式
        \[4(x|y)=\norm{x+y}^2-\norm{x-y}^2+i\norm{x+iy}^2-i\norm{x-iy}^2\]
        を用いて,ノルムの等式$\norm{\wh{x}}^2_2=\norm{x}^2$は内積の等式$(\wh{x}|\wh{y})=(x|y)$に書き直せる.
        \item[(4)$\Rightarrow$(1)]
        (1)が成り立たないとすると,$\{u_\al\}_{\al\in A}$のすべてと直交する元$u\ne0\in H$が取れる.
        この元について,ノルムは$(u,u)=\norm{u}^2>0$であるが,対応するFourier係数はすべて$0$なので,Parsevalの恒等式に反する.
    \end{description}
\end{proof}

\begin{corollary}[Hilbert空間の元のFourier係数と有限次元線型空間の「線型独立性」とが対応する]
    $(u_\al)_{\al\in A}$を$H$の正規直交基底とする.このとき,$x,y\in H$のFourier係数が写像$A\to\C$として一致する$\wh{x}=\wh{y}$ならば,$x=y$である.
\end{corollary}
\begin{proof}
    $x-y$を考えると,このFourier係数はすべて$0$である.よって,Parsevalの恒等式の特別な場合(定理の(3))より,$0=\norm{x-y}^2$.よって,$x=y$.
\end{proof}

Hilbert空間の同型とは,内積を保存する線型作用素をいうから,次が結論づけられたことになる.

\begin{corollary}
    $(u_\al)_{\al\in A}$は$H$の正規直交基底であるとし,$\wh{x}(\al):=(x,u_\al)$をFourier係数とする.
    このとき,写像$\wh{-}:H\to l^2(A)$はHilbert空間の同型である.
\end{corollary}

\begin{tbox}{red}{}
    $H=L^2(T)$の場合について考えてみると,Fourier展開は$L^2(T)\sim_\Hilb l^2(\Z)$を与える.
    Riesz-Fischerの定理より,二乗和が絶対収束する数列$(c_n)\in l^2(\Z)$に対して,必ず$c_n=\frac{1}{2\pi}\int^\pi_{-\pi}f(t)e^{-int}dt\;(n\in\Z)$を満たす周期関数$f\in L^2(T)$が一意的に存在する.
    またParsevalの定理より,$\norm{f-s_N}^2_2=\sum_{\abs{n}\ge N}\abs{\wh{f}(n)}^2$であるから,$N\to\infty$のときこのノルムは$0$に収束する.よって,$f$のFourier級数は$f$に$L^2$-収束する.
    一方で,概収束を示すには,より精緻な議論を要する.
\end{tbox}

\section{Hilbert空間の分類}

\begin{tcolorbox}[colframe=ForestGreen, colback=ForestGreen!10!white,breakable,colbacktitle=ForestGreen!40!white,coltitle=black,fonttitle=\bfseries\sffamily,
title=]
    Hilbert空間の分類は$l^2(A),l^2(B)$の空間に移して考えることで即時に決着がつき,これがFourier変換論の最初の帰結になる.
    まず任意のHilbert空間は基底を持つことを示せれば,Hilbと$l^2(-)$の間に対応がついたことになり,あとは$l^2(A)\sim_\Hilb l^2(B)\Leftrightarrow A\sim_\Set B$を示せば良い.
\end{tcolorbox}

\section{連続関数のFourier級数}

\begin{tcolorbox}[colframe=ForestGreen, colback=ForestGreen!10!white,breakable,colbacktitle=ForestGreen!40!white,coltitle=black,fonttitle=\bfseries\sffamily,
title=連続関数のFourier級数が各点収束するための条件を探す]
以降,Banach空間論の技術を援用して,より一般の内積が使えない場合のFourier級数論を展開する.
\end{tcolorbox}

\subsection{殆どの連続関数はFourier展開に失敗する}

\begin{tcolorbox}[colframe=ForestGreen, colback=ForestGreen!10!white,breakable,colbacktitle=ForestGreen!40!white,coltitle=black,fonttitle=\bfseries\sffamily,
title=]
    Fourier変換は,Dirichlet核との畳み込みの極限と見れるため,この作用素的な考察によって,ほとんどの$\R$上の点(稠密)上でFourier級数は収束しない関数が$C(T)$のほとんど(稠密)であることがわかる.
    逆を言えば,任意の$x\in\R$について,その上で収束するFourier級数を持つ連続関数の全体は,$C(T)$のBaireの第一類部分集合をなす.
\end{tcolorbox}

\begin{problem}
    Fourier変換は二乗可積分関数上に同型$\F:L^2(T)\iso l^2(\Z)$を定め,特にParsevalの定理より,任意の$f\in L^2(T)$について,Fourier級数は$L^2$-収束するのであった.
    よって特にCauchy列であるから,Egoroffの定理より,Fourier級数列の部分列が存在して,$f$に概収束する.
    では,$L^2(T)$の元とはいわず一般に,$f\in C(T)$のFourier級数はいつ各点収束するのか?:どんな$f\in C(T)$と$x\in\R$について,$\lim_{n\to\infty}s_n(f;x)=f(x)$か?
\end{problem}

\begin{definition}[Dirichlet kernel]
    Dirichlet核を$D_n(t):=\sum^n_{k=-n}e^{ikt}$と定めると,Fourier部分和は
    \[s_n(f;x)=\sum_{k=-n}^ne^{inx}\cdot\frac{1}{2\pi}\int^\pi_{-\pi}f(t)e^{-ikt}=\frac{1}{2\pi}\int^\pi_{-\pi}f(t)D(x-t)dt=(f*D_n)(x)\]
    と表せる.すなわち,ただの有限和ではなく,積分と見れる.
\end{definition}

\begin{discussion}[一様有界性の原理より,すべての連続関数のFourier級数が概収束する訳ではない]
    \[s^*(f;x):=\sup_{n\in\N}\abs{s_n(f;x)}\]
    とおく.$\Lambda_nf:=s_n(f;0)$として,$\Lambda_n:C(T)\to\R$を定めると,Fourier部分和とはDirichlet核を積分核とする積分でもあったから,これはBanach空間$C(T)$上の有界線型汎関数である.
    こうして得た作用素の族$(\Lambda_n)_{n\in\N}$のそれぞれについて,作用素ノルムは三角不等式により
    \[\norm{\Lambda_n}\le\frac{1}{2\pi}\int^\pi_{-\pi}\abs{D_n(t)}dt=\norm{D_n}_1\]
    と評価できる.
    実は,$\norm{\Lambda_n}\to\infty\;(n\to\infty)$である.
    よって,Banach-Steinhausの定理より,ある稠密な$G_\delta$-集合$E_0\subset C(T)$が存在して,
    その上では$\{\Lambda_n\}$は有界でない.すなわち,
    $\forall_{f\in E_0}\;s^*(f;0)=\infty$.
\end{discussion}
\begin{proof}\mbox{}
    \begin{description}
        \item[Dirichlet核の列は発散する] \begin{align*}
            e^{it/2}D_n(t)&=e^{it/2}\sum_{k=-n}^ne^{ikt},&e^{-it/2}D_n(t)&=e^{-it/2}\sum_{k=-n}^ne^{ikt},
        \end{align*}
        の片々を引いて,
        \begin{align*}
            2i\sin xD_n(t)&=e^{ikt}e^{it/2}-e^{-ikt}e^{-it/2}=2i\sin\paren{n+\frac{1}{2}}t\\
            D_n(t)&=\frac{\sin\paren{n+\frac{1}{2}}t}{\sin(t/2)}.
        \end{align*}
        これより,
        \begin{align*}
            \norm{D_n}_1&>\frac{2}{\pi}\int^\pi_0\Abs{\sin\paren{n+\frac{1}{2}}t}\frac{dt}{t}=\frac{2}{\pi}\int^{(n+1/2)\pi}_0\abs{\sin t}\frac{dt}{t}\\
            &>\frac{2}{\pi}\sum^n_{k=1}\frac{1}{k\pi}\int^{k\pi}_{(k-1)\pi}\abs{\sin t}dt=\frac{4}{\pi^2}\sum^n_{k=1}\frac{1}{k}\to\infty
        \end{align*}
        と評価できる.
        \item[$\Gamma_n$の列のノルムはDirichlet核のノルムに一致する]
        任意の$n\in\N$を取る.列$\{f_i\}\subset L^1(T)$であって,$\Lambda_n(f_j)\xrightarrow{j\to\infty}\norm{D_n}_1$であるものが取れれば,$\norm{\Lambda_n}=\norm{D_n}_1$がわかる.
        \[g(t):=\begin{cases}
            1,&D_n(t)\ge0,\\
            -1,&D_n(t)<0.
        \end{cases}\]
        と定めると,これは可積分だから,ある$\{f_j\}\subset C(T)$が存在して,$-1\le f_j\le 1$を満たし,$g$に各点で収束する.
        よって,Lebesgueの優収束定理より,
        \[\lim_{j\to\infty}\Lambda_n(f_j)=\lim_{j\to\infty}\frac{1}{2\pi}\int^\pi_{-\pi}f_(-t)D_n(t)dt=\frac{1}{2\pi}\int^\pi_{-\pi}g(-t)D_n(t)dt=\norm{D_n}_1.\]
    \end{description}
\end{proof}

\begin{proposition}
    任意の$x\in\R$に対して稠密な$G_\delta$集合$E_x\subset C(T)$が存在して,$\forall_{f\in E_x}\;s^*(f;x)=\infty$が成り立つ.
\end{proposition}

\begin{theorem}
    稠密な$G_\delta$-集合$E\subset C(T)$が存在して,次を満たす:
    \begin{quotation}
        任意の$f\in E$に対して,集合$Q_f:=\Brace{x\in\R\mid s^*(f;x)=\infty}$は$\R$の稠密な$G_\delta$-集合である.
    \end{quotation}
\end{theorem}

実は,$E$と$Q_f$は非可算無限集合である.

\begin{proposition}
    完備距離空間$X$は孤立点を持たないとする.このとき,$G_\delta$-である可算な稠密部分集合は存在しない.
\end{proposition}

\begin{tbox}{red}{}
    以上より,結論としては,$C(T)$のうちほとんどの関数(稠密)は,$\R$上のうちほとんどの点(稠密)でFourier級数は発散する.
\end{tbox}

\subsection{条件を緩めることによって解決可能}

\begin{definition}
    $0<\al\le 1$について,$\al$-次のLipschitz条件を満たす関数$f:[a,b]\to\C$を
    \[M_f:=\sup_{s\ne t}\frac{\abs{f(s)-f(t)}}{\abs{s-t}^\al}<\infty.\]
    と定める.その全体を$\Lip\al$と表す.
\end{definition}

\begin{proposition}
    $\Lip\al$はノルム$\norm{f}=\abs{f(a)}+M_f$あるいはノルム$\norm{f}=M_f+\sup_{x\in[a,b]}\abs{f(x)}$についてBanach空間となる.
\end{proposition}

\begin{proposition}
    ある$0<\al\le 1$について,$f\in C(T)\cap\Lip\al$は,そのFourier級数は$f$に各点収束する.
\end{proposition}

\section{可積分関数のFourier係数}

\begin{tcolorbox}[colframe=ForestGreen, colback=ForestGreen!10!white,breakable,colbacktitle=ForestGreen!40!white,coltitle=black,fonttitle=\bfseries\sffamily,
title=]
    では,$L^2$までは行かなくとも,もっと緩い可積分性によって$C(T)$の条件を緩めるとどうか?
\end{tcolorbox}

\begin{theorem}[Riemann-Lebesgueの補題]
    任意の$f\in L^1(T)$に関して,対応するFourier係数$\wh{f}:\Z\to\C$は,$\wh{f}\in C_0(\Z)$である.
    \[\F[f](n)~\wh{f}(n)=\frac{1}{2\pi}\int^\pi_{-\pi}f(t)e^{-int}dt\xrightarrow{\abs{n}\to\infty}0.\]
\end{theorem}

これは,$L^1(\R)\sim C_0(\Z)=:c_0$なるRiesz-Fischerの再来のような関係を期待させる.\footnote{$Z$に離散位相を入れると,$\Z$は局所コンパクトハウスドルフで,まさに$c_0=C_0(\Z)$である.}
任意の,無限遠点で消失する係数列$(a_n)$に対して,これをFourier係数として持つ可積分関数$f\in L^1(T)$は存在するか?
これは直感的には極めて存在しそうであるが,開写像定理より否と証明される.

\begin{theorem}[Riemann-Lebesgueの補題の部分逆は成り立つ]
    Fourier係数の対応
    \[\xymatrix@R-2pc{
        \F:L^1(T)\ar[r]&c_0\\
        \rotatebox[origin=c]{90}{$\in$}&\rotatebox[origin=c]{90}{$\in$}\\
        f\ar@{|->}[r]&\wh{f}
    }\]
    は単射な有界線型作用素であるが,全射ではない.
\end{theorem}

\section{Poisson積分}

\begin{tcolorbox}[colframe=ForestGreen, colback=ForestGreen!10!white,breakable,colbacktitle=ForestGreen!40!white,coltitle=black,fonttitle=\bfseries\sffamily,
title=]
    Poisson積分とは,$T$上のRadon測度$\mu_z$を,$[-\pi,\pi]$上のLebesgue測度に変換したときに生じる積分核$P_r(\theta-t)$に関する畳み込みである.
    
\end{tcolorbox}

\begin{notation}
    $K$をコンパクトハウスドルフ空間,$H$をそのコンパクト部分集合とする($K=[\Delta]$の境界の一般化).
    ノルム$\norm{f}_E:=\sup_{x\in E}\abs{f(x)}\;(E\subset K)$に関して,
    $A\subset C(K)$は$1\in A$を満たし,$\forall_{f\in A}\;\norm{f}_K=\norm{f}_H$を満たす関数のなす線型部分空間とする.
    これは最大値の原理を満たす連続関数全体の空間となる:$\forall_{f\in A,x\in K}\;\abs{f(x)}\le\norm{f}_H$.
    これはすなわち,任意の点$x\in K$について,評価作用素$\ev_x:M\to\bF$はノルム$1$の有界線型汎関数である.
    なお,$M$とは,
    制限作用素$|_H:C(K)\to C(H)$の$A$への制限は全単射な等長同型$A\iso |_H(A)=:M\subset C(H)$を引き起こすときの像とした.
    特に,任意の$M$の元は$A$の元であるような$K$全体への一意な延長を持つ.
    
    よって,$\ev_x:M\to\bF$は,延長$\Lambda:C(H)\to\bF$を持ち,これは再びノルム$1$である:$\Lambda 1=1,\norm{\Lambda}=1$.
    実はこのことから,$\Lambda\in (C(H))^*$は正な線型汎関数であることが従う.
    するとRieszの表現定理より,$H$上のあるRadon測度(正則なBorel測度)$\mu_x$が存在して,
    \[\forall_{f\in C(H)}\;\Lambda f=\int_Hfd\mu_x\]
    と表現される.特に,$f\in A$については$f(x)=\int_Hfd\mu_x$.
\end{notation}

\begin{proposition}
    任意の$x\in K$に対応した境界$H$上の測度$\mu_x$が存在して,任意の$f\in A$に対して次が成り立つ:
    \[f(x)=\int_Hfd\mu_x.\]
\end{proposition}

$\mu_x$の一意性は一般には言えないが,次のような状況では成り立つ.

\begin{example}[Poisson積分の設定]
    $K:=[\Delta],H:=T=\partial\Delta$とする.このとき,任意の多項式$f$は最大値の原理を満たす:$\norm{f}_U=\norm{f}_T$.
    そこで,$A\subset C([\Delta])$を,多項式の全体を含み,$\forall_{f\in A}\;\norm{f}_U=\norm{f}_T$を満たす関数の線型部分空間とする.
    このとき,任意の$z\in\Delta$に対して,$T$上のBorel測度$\mu_z$が存在して
    \[f(z)=\int_Tfd\mu_z\quad(f\in A)\]
    を満たす.
\end{example}

\begin{definition}
    \[P_r(\theta-t)=\sum^\infty_{n=-\infty}r^{\abs{n}}e^{in(\theta-t)}\quad t,\theta\in\R\]
    を\textbf{Poisson核}という.これは,次のようにも表現できる:
    \[P_r(\theta-t)=\frac{1-r^2}{1-2r\cos(\theta-t)+r^2}.\]
    特に,$\forall_{0\le r<1}\;P_r(\theta-t)\ge0$である.
    なお,$r\to1$の極限について,$(P_r)$はBanach代数$L^1(T)$の近似的単位元をなす.
\end{definition}
\begin{proof}
    $z=re^{i\theta}$とすると,$(ze^{-it})^{n}=r^ne^{in(\theta-t)}$で,この実部と$r^{\abs{-n}}e^{-in(\theta-t)}$の実部は一致する.
    よって,$P_r(\theta-t)$の実部は,次の式の実部に一致する:
    \[1+2\sum^\infty_{n=1}(ze^{-it})^n=\frac{e^{it}+z}{e^{it}-z}=\frac{1-r^2+2ir\sin(\theta-t)}{\abs{1-ze^{-it}}^2}.\]
\end{proof}

\begin{theorem}[Poisson積分]
    $A\subset C([\Delta])$を線型部分空間とする.$A$が多項式の全体を含み,$\forall_{f\in A}\;\sup_{z\in\Delta}\abs{f(z)}=\sup_{z\in T}\abs{f(z)}$を満たすならば,
    任意の$f\in A$と$z\in\Delta$について,次のPoisson積分表現が出来る:
    \[f(z)=\frac{1}{2\pi}\int^\pi_{-\pi}\frac{1-r^2}{1-2r\cos(\theta-t)+r^2}f(e^{it})dt\quad(z=re^{i\theta})\]
\end{theorem}
\begin{remarks}
    逆に,任意の$T$上の可積分関数$f:T\to\C$を与えた時,これの単位円板上への延長$\o{f}:[\Delta]\to\C$であって,
    $\sup_{z\in U}\abs{\o{f}(z)}=\sup_{z\in T}\abs{f(z)}$を満たすもの(これは調和性に同値)が一意に存在する.
    この自然な対応を掴むのが関数解析.
\end{remarks}

\section{半平面上のPoisson核}

\begin{definition}\mbox{}
    \begin{enumerate}
        \item $H(t):=e^{-\abs{t}}$と定める.
        \item そのFourier変換を$h_\lambda(x):=(H(\lambda t),e^{-itx})=\int^\infty_{-\infty}H(\lambda t)e^{itx}dm(t)\;(\lambda>0)$と定める.これを\textbf{上半平面におけるPoisson核}という.
    \end{enumerate}
\end{definition}

\begin{lemma}[上半平面のPoisson核の性質]\mbox{}
    \begin{enumerate}
        \item $h_\lambda(x)=\sqrt{\frac{2}{\pi}}\frac{\lambda}{\lambda^2+x^2}$.
        \item $\int^\infty_{-\infty}h_\lambda(x)dm(x)=1$.
        \item $0<H(t)\le 1$.$H(\lambda t)\xrightarrow{\lambda\to0}1$.
    \end{enumerate}
\end{lemma}

\begin{theorem}
    $f\in L^p(\R)\;(p\in[1,\infty])$に対して,$(f*h_\lambda)(x)$は上半平面上で定義された$x+i\lambda\in\bH$の調和関数である.
\end{theorem}

\chapter{Fourier変換と超関数論}

\begin{quotation}
    前節で,Fourier変換を通じれば,関数をその成分$\al e^{i\lambda x}$に分解できることが解った.
    物理的にはそれぞれの成分は調和振動子に当たる.
    今後は群$\R$は可換であってもコンパクトではないので,
    これを一般的に展開するには,創作が必要になり,超関数論が要請される.
\end{quotation}

\section{Formal Property}

\begin{tcolorbox}[colframe=ForestGreen, colback=ForestGreen!10!white,breakable,colbacktitle=ForestGreen!40!white,coltitle=black,fonttitle=\bfseries\sffamily,
title=]
    周期関数$f\in L^2(T)$や$f\in L^1(T)$に関するFourier変換を,まずFourier係数列への対応$\F:L^1(T)\to C_0(\Z)=c_0$として考えた.では$\Z$上から$\R$上へと拡張しようとすると,ほとんどの$f\in C(T)$では失敗するが,$f\in L^p(T)\;(1<p\le\infty)$では殆ど至る所の$x\in\R$で成功する.
    一般の関数$f\in L^1(\R)$については,Fourier変換を実数上の関数として調べる.
\end{tcolorbox}

\subsection{特性関数}

\begin{tcolorbox}[colframe=ForestGreen, colback=ForestGreen!10!white,breakable,colbacktitle=ForestGreen!40!white,coltitle=black,fonttitle=\bfseries\sffamily,
title=]
    指標とは,加法から乗法への群準同型をいう.
    指標との畳み込み$\wh{f}(t)=(f*e_t)(0)$を\textbf{Fourier変換}という.
\end{tcolorbox}

\begin{notation}[normalized Lebesgue measure]\mbox{}
    \begin{enumerate}
        \item $dx$を$\R$上のLebesgue測度,$m$をそれを$\sqrt{2\pi}$で割ったものとする:
        $\int_\R f(x)dm(x)=\frac{1}{\sqrt{2\pi}}\int_\R f(x)dx$.
        
        一般に,$dm_n(x)=(2\pi)^{-n/2}dx$.
        \item 内積のスケール変化に合わせて,\textbf{$p$-ノルム}の定義も変わる
        $\norm{f}_p:=\paren{\int_\R\abs{f(x)}^pdm(x)}^{1/p}\quad(1\le p<\infty)$.
        \item \textbf{畳み込み}もリスケールする:
        $(f*g)(x):=\int_\R f(x-y)g(y)dm(y)\quad(x\in\R)$.
        \item $f\in L^1$に対して,指標$e_t(x)=e^{itx}$との畳み込みでの$u=0$の値
        (各積分核$e^{-ixt}\;(t\in\R)$に関する積分変換)を\textbf{Fourier変換}$\wh{-}:L^1\to L^1$といい,次のように表す:
        \[\hat{f}(t):=(f*e_t)(0)=\int_\R f(x)e_{-t}(x)dm(x)=\int_\R f(x)e_t(0-x)dm(x)=\int_\R f(x)e^{-ixt}dm(x)\quad(t\in\R).\]
        \item $L^p(\R),C_0(\R)$を$L^p,C_0$と略記する.
        \item 多重指数$\al\in\N^n$に対して,\textbf{偏微分作用素}$D_\al$を$D_\al:=(i)^{-n}D^\al=\paren{\frac{1}{i}\pp{}{x_1}}^{\al_1}\cdots\paren{\frac{1}{i}\pp{}{x_n}}^{\al_n}$で表す.このとき,$D_\al e_t=t^\al e_t$が成り立つ.これがFourier変換の偏微分方程式論への応用可能性を支える.
        \item \textbf{遷移作用素}$\tau_x\;(x\in\R^n)$を$(\tau_xf)(y):=f(y-x)$で定める.
    \end{enumerate}
\end{notation}

\begin{definition}[character, characteristic function]\mbox{}
    \begin{enumerate}
        \item 加法群$\R$から$\C$の乗法部分群$S^1$への群準同型,すなわち可換群$\R$の連続な1次元ユニタリ表現を\textbf{指標}という:$\abs{\varphi(t)}=1,\varphi(s+t)=\varphi(s)\varphi(t)$.
        \item 特に,$t\in\R^n$の指標$e_t$とは,指数関数$e_t(x):=e^{it\cdot x}\;(x\in\R^n)$をいう.2つの引数について全く対称であることに注意.
        \item 指標$t\in\R^n$の,$\R^n$上の確率測度に関する積分(との合成)$\varphi(u)=\int_\R e^{iut}dt$を特性関数という.
    \end{enumerate}
\end{definition}

\begin{tcolorbox}[colframe=ForestGreen, colback=ForestGreen!10!white,breakable,colbacktitle=ForestGreen!40!white,coltitle=black,fonttitle=\bfseries\sffamily,
    title=]
    基底$x:=(x^1,\cdots,x^n)$に対して,これらの組み合わせの冪を簡潔に表記する記法を導入する.
\end{tcolorbox}

\begin{definition}[multi-index]
    集合$\N^d$に各点和と$l_1$-ノルム$\abs{\al}:=\al_1+\cdots+\al_d$,階乗$\al!=\al_1!\cdots\al_d!$を備えた区間
    の元を\textbf{$d$次の多重指数}という.
    \begin{enumerate}
        \item 二項係数について$\begin{pmatrix}\al\\\beta\end{pmatrix}=\begin{pmatrix}\al_1\\\beta_1\end{pmatrix}\cdots\begin{pmatrix}\al_d\\\beta_d\end{pmatrix}$とする.
        \item $f\in C^d(\R^d)$に対する偏微分について,$\partial^\al f=\frac{\partial^{\al_1+\cdots+\al_d}}{\partial x_1^{\al_1}\cdots\partial x_d^{\al_d}}f$とする.
        \item $x\in\R^d$に対して,$x^\al=x_1^{\al_1}\cdots x_d^{\al_d}$とする.
    \end{enumerate}
\end{definition}

\subsection{Fourier変換の関手性}

\begin{tcolorbox}[colframe=ForestGreen, colback=ForestGreen!10!white,breakable,colbacktitle=ForestGreen!40!white,coltitle=black,fonttitle=\bfseries\sffamily,
title=]
    Fourier変換は,群準同型$(L_1,\delta_0,*)\to(L_1,1,\cdot)$を与える.また,
    指標による積を平行移動に,平行移動を指標による積に変換する.
\end{tcolorbox}

\begin{theorem}[Fourier変換の関手性]
    $f\in L^1,\al,\lambda\in\R$とする.
    \begin{enumerate}
        \item $\wh{\tau_xf}=e_{-x}\wh{f}$.すなわち,$\wh{f(x-\al)}=\wh{f}(t)e^{-i\al t}$.
        \item $\wh{e_xf}=\tau_x\wh{f}$.すなわち,$\wh{f(x)e^{i\al x}}=\wh{f}(t-\al)$.
        \item $\wh{(f*g)}=\wh{f}\wh{g}$.
        \item $h(x)=f\paren{\frac{x}{\lambda}}\;(\lambda>0)$のFourier変換は,$\wh{h}(t)=\lambda^n\wh{f}(\lambda t)$.ただし$f\in L^1(\R^n)$とした.
    \end{enumerate}
    また,次が成り立つ:
    \begin{enumerate}\setcounter{enumi}{4}
        \item $g(x)=\o{f(-x)}\Rightarrow\wh{g}(t)=\o{\wh{f}(t)}$.
        \item $g(x)=-ixf(x)\land g\in L^1\Rightarrow\wh{f}$は微分可能で$\wh{f}'(t)=\wh{g}(t)$.
    \end{enumerate}
\end{theorem}
\begin{proof}\mbox{}
    \begin{enumerate}
        \item \begin{align*}
            \wh{\tau_xf}(t)&=\int(\tau_xf)\cdot e_{-t}(y)dm(y)\\
            &=\int f(y)(\tau_{-x}e_{-t})(y)dm(y)\\
            &=\int f(y)e_{-t}(y+x)dm(y)\\
            &=e_{-t}(x)\wh{f}(t)=e_{-x}(t)\wh{f}(t).
        \end{align*}
        \item \begin{align*}
            \wh{e_xf}(t)&=\int(e_xf)(y)e_{-t}(y)dm(y)\\
            &=\int f(y)e_{-(t-x)}(y)dm(y)=(f*e_{t-x})(0)=(\tau_x\wh{f})(t).
        \end{align*}
        \item Fubiniの定理より,
        \begin{align*}
            \wh{f*g}(t)&=\int_{\R\times\R}\paren{f(y-x)g(x)dx}e_{-t}(y)dm(y)\\
            &=\int_\R g(x)e_{-t}(x)dx\int_\R f(y-x)e_{-t}(y-x)dm(y)=\wh{g}(t)\wh{f}(t).
        \end{align*}
        Fibiniの定理より,$\mu\dae x$について$\int_\R f(y-x)e_{-t}(y-x)dm(y)$は可積分で,その値は一定であることに注意.
        \item 簡単な変数変換により分かる.1次元の場合は次のようになり,$n$次元の場合は$\lambda I$の行列式$\lambda^n$がかかる:
        \begin{align*}
            \wh{h}(t)&=\int_\R h(x)e_{-t}(x)dm(x)=\int_\R f\paren{\frac{x}{\lambda}}e_{-t}\paren{\frac{x}{\lambda}}dm(x)\\
            &=\lambda\int_\R f(y)e_{-t}(y)dx=\lambda\wh{f}(t).
        \end{align*}
    \end{enumerate}
\end{proof}
\begin{remarks}
    以上の結果は,ひとえに$m$の平行移動不変性と,任意の$t\in\R$について積分核$x\mapsto e^{itx}$は$\R$の指標であることによる.
    なお,$\R$の連続な指標はすべて指数関数によって表現される.
\end{remarks}

\begin{remark}[(6)の逆]
    微分を$ti$による積に変換することから,Fourier変換は常微分方程式論でも応用される.
    $f,f'\in L^1$のとき,$\wh{f'}=it\wh{f}(t)$である.
\end{remark}

\section{The Inversion Theorem}

\begin{tcolorbox}[colframe=ForestGreen, colback=ForestGreen!10!white,breakable,colbacktitle=ForestGreen!40!white,coltitle=black,fonttitle=\bfseries\sffamily,
title=]
    元の世界に戻す方法があって初めて,Formal Propertyが応用上の意味を持つ.
    まず$\F:L^1\to C_0$がノルム減少的な線型写像であることが分かる.このうち,$L_1\subset\Im\F$の上では可逆である.
    特に$\S$上では同型を定める.
\end{tcolorbox}

\subsection{ノルム減少性}

\begin{discussion}[Fourier級数に関する逆転公式]
    Fourier係数列が
    \[c_n=(f,e^{inx})=\frac{1}{2\pi}\int^\pi_{-\pi}f(x)e^{-inx}dx\quad(n\in\Z)\]
    であった場合,元の関数は$f(x)=\sum^\infty_{n=-\infty}c_ne^{inx}\in L^2(T)$となるのであった.
    しかし,$f$の級数の収束は各点においては定まらない.
    しかし,追加で$c_n\in L^1(\mu)=l^\infty$を仮定すれば,級数は一様収束するから,Fourier級数は$f$に殆ど至る所収束する.
\end{discussion}

\begin{notation}
    任意の関数$f\in\Map(\R,\C)$と点$y\in\R$について,$f$の$y$-平行移動を
    \[f_y(x)=f(x-y)\]
    で表す.
\end{notation}

\begin{lemma}
    $f\in L^p\;(1\le p<\infty)$に対する
    平行移動写像
    \[\xymatrix@R-2pc{
        \R\ar[r]&L^p(\R)\\
        \rotatebox[origin=c]{90}{$\in$}&\rotatebox[origin=c]{90}{$\in$}\\
        y\ar@{|->}[r]&f_y=(f\cdot-y)
    }\]
    は一様連続である.
\end{lemma}

\begin{theorem}
    $\F:L^1\to C_0$がノルム減少的な写像(short map)であることを定立する.
    \begin{enumerate}
        \item $f\in L^1$ならば$\wh{f}\in C_0$である.
        \item $\norm{\wh{f}}_\infty\le\norm{f}_1$.
    \end{enumerate}
\end{theorem}

\subsection{Poisson核}

\begin{tcolorbox}[colframe=ForestGreen, colback=ForestGreen!10!white,breakable,colbacktitle=ForestGreen!40!white,coltitle=black,fonttitle=\bfseries\sffamily,
title=]
    正関数$H$と,そのFourier変換$h_\lambda$であって積分が明らかなものの組を把握しておく.
    これはなんでも良いが,Poisson核とする.
\end{tcolorbox}

\begin{proposition}
    $f\in L^1$ならば,
    \[(f*h_\lambda)(x)=\int^\infty_{-\infty}H(\lambda t)\wh{f}(t)e^{ixt}dm(t).\]
\end{proposition}

\begin{proposition}
    $g\in L^\infty$は$x\in\R$において連続であるとする.このとき,
    \[\lim_{\lambda\to 0}(g*h_\lambda)(x)=g(x).\]
\end{proposition}

\begin{proposition}
    $f\in L^p\;(1\le p<\infty)$について,$\lim_{\lambda\to0}\norm{f*h_\lambda-f}_p=0$.
\end{proposition}

\subsection{Gauss核}

\begin{lemma}
    $\phi_n(x):=\exp\paren{-\frac{1}{2}\abs{x}^2}\;(x\in\R^n)$とすると,
    \begin{enumerate}
        \item $\phi_n\in\S_n$.
        \item $\wh{\phi}_n=\phi_n$.
        \item $\phi_n(0)=\int_{\R_n}\wh{\phi_n}dm_n$.
    \end{enumerate}
\end{lemma}

\subsection{逆転公式}

\begin{theorem}[The Inversion Theorem]\mbox{}
    \begin{enumerate}
        \item $g\in\S_n$のとき,$g(x)=\int_{\R^n}\wh{g}e_xdm_n\;(x\in\R^n)$.
        \item $f,\wh{f}\in L^1(\R^n)$のとき,$f_0(x):=\int_{\R^n}\wh{f}e_xdm_n\;(x\in\R^n)$と定めると$f_0\in C_0(\R^n)$で,$f=f_0\;\ae x$.
    \end{enumerate}
    $f,\wh{f}\in L^1$で,
    \[g(x)=\int^\infty_{-\infty}\wh{f}(t)e^{ixt}dm(t)\quad(x\in\R)\]
    を満たすとする.このとき,$g\in C_0$かつ$f=g\;\ae$
\end{theorem}

\begin{theorem}[The Uniqueness Theorem]
    $f\in L^1$かつ$\wh{f}=0$ならば,$f(x)=0\;\ae$
\end{theorem}

\section{The Plancherel Theorem}

\begin{tcolorbox}[colframe=ForestGreen, colback=ForestGreen!10!white,breakable,colbacktitle=ForestGreen!40!white,coltitle=black,fonttitle=\bfseries\sffamily,
title=$L^1$に入る前に,$L^2$上での完成された議論を見る]
    $\S_n$は$L_1$上でも$L_2$上でも稠密であるため,$L^1\cap L^2$上のPlancherel作用素は$L^2$上に延長し,\textbf{等長なままである}:$\norm{f_2}=\norm{\wh{f}_2}$.
    Lebesgue測度は有界でないから$L_2$は$L_1$の部分集合ではないが,$L_2$に「制限」するとFourier-Plancherel変換はすごく振る舞いが良い.
\end{tcolorbox}

\begin{theorem}
    次の条件を満たす線型等長同型$\Psi:L^2(\R^n)\mono L^2(\R^n)$が一意的に存在する:$\forall_{f\in\S_n}\;\Psi f=\wh{f}$.
\end{theorem}

\section{The Banach Algebra $L^1$}

\section{分布}

\begin{tcolorbox}[colframe=ForestGreen, colback=ForestGreen!10!white,breakable,colbacktitle=ForestGreen!40!white,coltitle=black,fonttitle=\bfseries\sffamily,
title=]
    解析学にぴったりな議論空間を,代数のことばで得ることを考える.
\end{tcolorbox}

\subsection{試験関数の空間の定義}

\begin{notation}
    $\D=\D(\R)=C_c^\infty(\R)$を隆起関数という.
\end{notation}

\begin{definition}[test function space]
    非空な開集合$\Om\subset\R^n$に対して,台がコンパクト集合$K\subset\Om$に含まれる関数のなすFrechet空間$\D_K$の合併
    $\D(\Om):=\cup_{K\compsub\Om}\D_K$を\textbf{試験関数の空間}という.
    すなわち,$\D(\Om):=C^\infty_c(\Om)=\Brace{\phi\in C^\infty(\Om)\mid\supp\phi\compsub\Om}$.
\end{definition}

\begin{lemma}
    ノルムの列$\norm{\phi}_N:=\max\Brace{\abs{D^\al\phi(x)}\in\R_+\mid x\in\Om,\abs{\al}\le N}$を考える.
    \begin{enumerate}
        \item このノルムの列は,任意の部分空間$\D_K\subset\D(\Om)$に元々のFrechet空間としての局所凸位相$\tau_K$を定める.
        \item このノルムの列は,$\D(\Om)$に局所凸で距離化可能な位相を定めるが,完備でない.
    \end{enumerate}
\end{lemma}

\subsection{完備な位相の導入}

\begin{tcolorbox}[colframe=ForestGreen, colback=ForestGreen!10!white,breakable,colbacktitle=ForestGreen!40!white,coltitle=black,fonttitle=\bfseries\sffamily,
title=]
    完備だが,距離化可能でない位相$\tau$を代わりに導入し,以後は$\D(\R^n)$にはこの位相を入れ,こちらを研究の道具とする.
\end{tcolorbox}

\begin{definition}\mbox{}
    \begin{enumerate}
        \item 絶対凸集合(均衡凸集合)$W\subset\D(\Om)$であって,$\forall_{K\compsub\Om}\;\D_K\cap W\in\tau_K$を満たすもの全体の集合を$\beta$と表す.
        \item $\beta$の元の平行移動$\phi+W\;(\varphi\in\D(\Om),W\in\beta)$で表される集合の任意合併で得られる集合全体の集合を$\tau$で表す.
    \end{enumerate}
\end{definition}

\begin{theorem}\mbox{}\label{thm-topology-on-the-space-of-test-functions}
    \begin{enumerate}
        \item $\tau$は$\D(\Om)$に位相を定め,$\beta$は$\tau$の局所基を定める.
        \item $\tau$は局所凸である.
    \end{enumerate}
\end{theorem}

\section{急減少関数空間}

\begin{tcolorbox}[colframe=ForestGreen, colback=ForestGreen!10!white,breakable,colbacktitle=ForestGreen!40!white,coltitle=black,fonttitle=\bfseries\sffamily,
title=]
    可積分関数の$\F$による像は可積分とは限らない.
    そこで,$\F$が保存する性質を探したい.
    そのうち一つは急減少性である.
    $\F$は複素Frechet空間$S(\R^n)\subset C^\infty(\R^n)$上に線型自己同型を定め,標準的なFourier変換$\F:L^2(\R^n)\to L^2(\R^n)$の自然な延長となる.
    この性質により,突然この空間が調和解析の中心へと躍り出る.
    緩増加関数の空間$D'(X)$は,線型偏微分方程式を解く自然な場となる.
\end{tcolorbox}

\subsection{定義}

\begin{tcolorbox}[colframe=ForestGreen, colback=ForestGreen!10!white,breakable,colbacktitle=ForestGreen!40!white,coltitle=black,fonttitle=\bfseries\sffamily,
title=]
    一般に$\R^n$上の急減少関数とは,$\R$の座標写像の任意の冪との積写像が有界写像であることをいう.
    が,主に議論の対象となる,無限階微分可能な急減少関数とは,任意階の(偏)導関数が急減少であることをいう.
\end{tcolorbox}

\begin{definition}[rapidly decreasing function, the Schwartz space]
    $f\in C^\infty(\R^n)$が\textbf{急減少偏導関数を持つ滑らかな関数}であるとは,次の同値な2条件を満たすことをいう:
    \begin{enumerate}
        \item $\forall_{N\in\N}\;\sup_{\abs{\al}\le N}\sup_{x\in\R^n}(1+\abs{x}^2)N\abs{(D_\al f)(x)}<\infty$.
        \item $\forall_{n\in\N,\al\in\N^n}\;\forall_{P\in\R[x]}\;P\cdot D_\al f\in l^\infty(\R^n)$.
    \end{enumerate}
    この関数がなす位相線形空間を\textbf{Schwartz空間}といい,$\S_n=\S(\R^n)$で表す.
\end{definition}
\begin{remarks}
    無限回微分可能な関数であるが,$\abs{x}\to\infty$を考えたときに,任意階の導関数が,$x$の任意の負冪よりも速くゼロに収束するものをいう.
    これは確率分布との関連で捉えると見通しが良い.どうして収束の速さがFourier変換と関係があるのだろうか.
\end{remarks}

\begin{lemma}[局所凸位相線形空間である]\mbox{}
    \begin{enumerate}
        \item 線型空間$X$上の分離的な半ノルム列$\P$が誘導する始位相は第2可算な局所凸位相で,$E\subset X$が有界であることと$p:E\to\R$が有界であることとは同値になる.
        \item ノルムの列$\sup_{\abs{\al}\le N}\sup_{x\in\R^n}(1+\abs{x}^2)N\abs{(D_\al f)(x)}\;(N\in\N)$は,$\S_n$上に局所凸な位相を定める.
    \end{enumerate}
\end{lemma}

\begin{example}\mbox{}
    \begin{enumerate}
        \item 隆起関数はSchwartz関数である:$C^\infty_c(\R^n)\mono\cS(\R^n)$.
        \item Gauss関数も急減少する:$\forall_{i\in\N^n}\;\forall_{a>0}\;x^ie^{-a\abs{x}^2}\in S(\R^n)$.
    \end{enumerate}
\end{example}

\subsection{性質}

\begin{tcolorbox}[colframe=ForestGreen, colback=ForestGreen!10!white,breakable,colbacktitle=ForestGreen!40!white,coltitle=black,fonttitle=\bfseries\sffamily,
title=]
    $(\S_n,\delta,*)$は「部分群」をなし,
    Fourier変換は$\S_n$の自己同型で,$*$の構造と両立する.
\end{tcolorbox}

\begin{theorem}\mbox{}
    \begin{enumerate}
        \item $\S_n$はFrechet空間である.
        \item 次が定める線型写像$\S_n\to\S_n$は連続である:$f\mapsto Pf,f\mapsto gf,f\mapsto D_\al f$.
        \item $f\in\S_n,P\in \R[x]$について,$\wh{P(D)f}=P\wh{f}$かつ$\wh{Pf}=P(-D)\wh{f}$.
        \item (逆転定理) $\F|_{\S_n}$は周期$4$の線型同型(全単射な連続線型写像)$\S_n\to\S_n$である.
    \end{enumerate}
\end{theorem}

\begin{theorem}
    $f,g\in\S_n$とする.
    \begin{enumerate}
        \item $f*g\in\S_n$.
        \item $\wh{fg}=\wh{f}*\wh{g}$.
    \end{enumerate}
\end{theorem}

\section{緩増加分布空間}

\subsection{定義と例}

\begin{theorem}\mbox{}
    \begin{enumerate}
        \item $\D(\R^n)\subset\S_n$は稠密である.
        \item 包含$i:\D(\R^n)\to\S_n$は連続である.
    \end{enumerate}
\end{theorem}

\begin{definition}[tempered distribution]
    $L\in\S_n^*$に対して,定理より$u_L:=L\circ i\in\D'(\R^n)$となる.
    また,$\D(\R^n)$の稠密性より,$L\mapsto u_L$の対応は単射である.こうして対応$\S'_n\ni L\mapsto u_L\in\D'(\R^n)$が定まる.
    この像に入る$\D'(\R^n)$の元を\textbf{緩増加分布}という.
    すなわち,$u\in\D'(\R^n)$であって,$\S_n$上への連続な延長を持つものをいう.
\end{definition}

\begin{example}[任意の可積分関数は緩増加である]\mbox{}
    \begin{enumerate}
        \item コンパクトな台を持つ超関数は緩増加である.
        \item $\R^n$上のBorel測度$\mu$であって,$\exists_{k\in\N}\;\int_{\R^n}(1+\abs{x}^2)^{-k}d\mu(x)<\infty$を満たすものは緩増加である.
        \item $g\in\L(\R^n)$が$\exists_{N>0,p\in[1,\infty)}\;\int_{\R^n}\abs{(1+\abs{x}^2)^{-N}g(x)}^pdm_n(x)=C<\infty$を満たすならば緩増加である.
        \item 任意の$g\in L^p(\R^n)\;(p\in[1,\infty])$は緩増加である.
    \end{enumerate}
    $e^x\cos(e^x)$は緩増加であるが,$e^x$はそうではない.
\end{example}

\subsection{閉性}

\begin{theorem}
    $\al\in\N^m,P\in\R[x],g\in\S_n$とする.$u\in\S'_n$ならば,$D^\al u,Pu,gu\in\S'_n$である.
\end{theorem}

\subsection{Fourier変換}

\begin{tcolorbox}[colframe=ForestGreen, colback=ForestGreen!10!white,breakable,colbacktitle=ForestGreen!40!white,coltitle=black,fonttitle=\bfseries\sffamily,
title=]
    $\S_n$にはFourier変換が定まっているので,超関数の方法で$\S'_n$上にも定める.
\end{tcolorbox}

\begin{definition}
    $u\in\S'_n$のFourier変換を,
    $\wh{u}(\phi):=u(\wh{\phi})\;(\phi\in\S_n)$で定める.
\end{definition}

\begin{theorem}\mbox{}
    \begin{enumerate}
        \item $\F:\S'_n\mono\S'_n$は周期4の線型同型(全単射な連続線型写像であり,逆も連続線型)である.
        \item $u\in\S'_n,P\in\R[x]$のとき,$\wh{P(D)u}=P\wh{u}$かつ$\wh{Pu}=P(-D)\wh{u}$.
    \end{enumerate}
\end{theorem}

\begin{example}[多項式のFourier変換]
    
\end{example}

\begin{corollary}
    超関数$u$について,次の2条件は同値.
    \begin{enumerate}
        \item $u$はある多項式$P\in\S'_n$のFourier変換である.
        \item $\supp u$は$\{0\}$または$\emptyset$である.
    \end{enumerate}
\end{corollary}

\subsection{畳み込み}

\begin{tcolorbox}[colframe=ForestGreen, colback=ForestGreen!10!white,breakable,colbacktitle=ForestGreen!40!white,coltitle=black,fonttitle=\bfseries\sffamily,
title=]
    超関数の方法で畳み込みを定義する.
\end{tcolorbox}

\begin{lemma}[Schwartz空間での収束の特徴付け]
    $w:=(1,0,\cdots,0)\in\R^n,\phi\in\S_n$とし,
    \[\phi_\ep(x):=\frac{\phi(x+\ep w)-\phi(x)}{\ep}\quad(x\in\R^n,\ep>0)\]
    とすると,$\phi_\ep\xrightarrow{\ep\to0}\pp{\phi}{x_1}$.
\end{lemma}

\begin{definition}
    $u\in\S'_n,\phi\in\S_n$について,$(u*\phi)(x):=u(\tau_x\widecheck{\phi})\;(x\in\R^n)$と定める.
\end{definition}

\begin{theorem}
    $\phi\in\S_n,u\in\S'_n$とする.
    \begin{enumerate}
        \item $u*\phi\in C^\infty(\R^n)$かつ$\forall_{m\in\N,\al\in\N^m}\;D^\al(u*\phi)=(D^\al u)*\phi=u*(D^\al\phi)$.
        \item $u*\phi$は多項式のオーダーで増加し,特に緩増加である.
        \item $\wh{u*\phi}=\wh{\phi}\wh{u}$.
        \item $\forall_{\phi\in\S_n}\;(u*\phi)*\psi=u*(\phi*\psi)$.
        \item $\wh{u}*\wh{\phi}=\wh{\phi u}$.
    \end{enumerate}
\end{theorem}

\section{Paley-Wienerの定理}

\begin{tcolorbox}[colframe=ForestGreen, colback=ForestGreen!10!white,breakable,colbacktitle=ForestGreen!40!white,coltitle=black,fonttitle=\bfseries\sffamily,
title=コンパクト台を持つ緩増加関数のFourier変換は整関数を定める]
    関数の無限大での減衰挙動と,そのFourier変換の解析性との間に対応がつく.このクラスの結果をPaley-Wienerの定理という.
    古典的な結果は,$\R$への制限が$L^2$に属するような指数型整関数$\C\to\C$の特徴付けを与える定理である.
\end{tcolorbox}

\subsection{古典的結果}

\begin{notation}
    $rB=\Delta(0,r)$とする.
\end{notation}

\begin{lemma}
    整関数$f:\C^n\to\C$が$\R^n$にて消えるならば,$f=0$である.
\end{lemma}

\begin{theorem}
    試験関数$\phi\in\D(\R^n)$とそのFourier変換$f:\C^n\to\C$について,次の2条件は同値.
    \begin{enumerate}
        \item ある試験関数$\phi\in\D(\R^n)$が存在して$\exists_{r>0}\;\supp\phi\subset rB$を満たし,
        \[f(z)=\int_{\R^n}\phi(t)e^{-iz\cdot t}dm_n(t)\quad(z\in\C^n)\]
        と表せる.
        \item $f$は整関数で,$\forall_{N\in\N}\;\exists_{\gamma_N<\infty}\;\forall_{z\in\C^n}\abs{f(z)}\le\gamma_N(1+\abs{z})^{-N}e^{r\abs{\Im z}}$.
    \end{enumerate}
\end{theorem}

\subsection{緩増加超関数への一般化}

\begin{discussion}
    $u$を$\R^n$上のコンパクト台を持つ超関数とすると,緩増加である:$u\in\S'_n$.
    よって,$\wh{u}\in\S'_n$が$\wh{u}(\phi):=u(\wh{\phi})\;(\phi\in\S_n)$と定まるのであった.

    このとき,$\wh{u}(x)=u(e_{-x})\;(x\in\R^n)$となる.
    実際,$\wh{f}(x)=\int fe_xdm_n$に注意すると,
    \[\wh{u}(\phi)=\int\wh{u}(x)\phi(x)dx=\int u(x)\wh{\phi}(x)dx=u(\wh{\phi}).\]
    すなわち,超関数$\wh{u}:\R^n\to\R$は関数としてしっかり定まっている.
    これは解析接続によって$\C^n$上に延長する.
\end{discussion}

\begin{theorem}
    $u\in\D'(\R^n)$と$f:\C^n\to\C$について,次の2条件は同値.
    \begin{enumerate}
        \item $u\in\D'(\R^n)$であって$\exists_{r>0}\;\supp u\subset rB$を満たすものが存在し,$u$は次数$N$を持ち,$\forall_{z\in\C^n}\;f(z)=u(e_{-z})$を満たす.
        \item $f$は整関数で,$f|_{\R^n}$は$u$のFourier変換であり,また$\exists_{\gamma<\infty}\;\forall_{z\in\C^n}\;\abs{f(z)}\le\gamma(1+\abs{z})^Ne^{r\abs{\Im z}}$を満たす.
    \end{enumerate}
\end{theorem}

\begin{definition}
    この定理によって与えられる整関数$\wh{u}(z)=u(e_{-z})$を,$u$の\textbf{Fourier-Laplace変換}ともいう.
\end{definition}

\subsection{Laplace変換}

\begin{tcolorbox}[colframe=ForestGreen, colback=ForestGreen!10!white,breakable,colbacktitle=ForestGreen!40!white,coltitle=black,fonttitle=\bfseries\sffamily,
title=]
    普段は$\R_+$へ制限した退化した姿を見るが,$f$が有界なとき$\L f$は右半平面に正則な関数を定め,
    そして$f$が緩増加関数であるとき$\L f$は整関数を定める.
    そのTaylor級数展開は,関数をモーメントの線型和として表示する(積率母関数).
    その一部を取り出したものがMellin変換である.
\end{tcolorbox}

\begin{definition}
    関数$f:(0,\infty)\to\R$について,
    \begin{enumerate}
        \item $\L f(x):=\int^\infty_0f(t)e^{-tp}dt$を\textbf{Laplace変換}という.
        \item $\M f(x):=\int^\infty_0f(t)t^{x-1}dt$を\textbf{Mellin変換}という.
    \end{enumerate}
\end{definition}

\section{Sobolevの補題}

\begin{thebibliography}{9}
    \bibitem{Korevaar}
    J. Korevaar "Fourier Analysis and Related Topics"
    \bibitem{高橋陽一郎}
    高橋陽一郎『実関数とFourier解析』
    \bibitem{Rudin}
    Rudin "Real and Complex Analysis" 3rd
    \bibitem{級数}
\end{thebibliography}

\end{document}