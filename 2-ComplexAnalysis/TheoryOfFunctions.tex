\documentclass[uplatex, dvipdfmx]{jsreport}
\title{関数論再建}
\author{司馬博文}
\date{\today}
\pagestyle{headings} \setcounter{secnumdepth}{4}
\usepackage{mathtools}
%\mathtoolsset{showonlyrefs=true} %labelを附した数式にのみ附番される設定.
%\usepackage{amsmath} %mathtoolsの内部で呼ばれるので要らない.
\usepackage{amsfonts} %mathfrak, mathcal, mathbbなど.
\usepackage{amsthm} %定理環境.
\usepackage{amssymb} %AMSFontsを使うためのパッケージ.
\usepackage{ascmac} %screen, itembox, shadebox環境.全てLATEX2εの標準機能の範囲で作られたもの.
\usepackage{comment} %comment環境を用いて,複数行をcomment outできるようにするpackage
\usepackage{wrapfig} %図の周りに文字をwrapさせることができる.詳細な制御ができる.
\usepackage[usenames, dvipsnames]{xcolor} %xcolorはcolorの拡張.optionの意味はdvipsnamesはLoad a set of predefined colors. forestgreenなどの色が追加されている.usenamesはobsoleteとだけ書いてあった.
\setcounter{tocdepth}{2} %目次に表示される深さ.2はsubsectionまで
\usepackage{multicol} %\begin{multicols}{2}環境で途中からmulticolumnに出来る.

\usepackage{url}
\usepackage[dvipdfmx,colorlinks,linkcolor=blue,urlcolor=blue]{hyperref} %生成されるPDFファイルにおいて、\tableofcontentsによって書き出された目次をクリックすると該当する見出しへジャンプしたり、さらには、\label{ラベル名}を番号で参照する\ref{ラベル名}やthebibliography環境において\bibitem{ラベル名}を文献番号で参照する\cite{ラベル名}においても番号をクリックすると該当箇所にジャンプする.囲み枠はダサいので,colorlinksで囲み廃止し,リンク自体に色を付けることにした.
\usepackage{pxjahyper} %pxrubrica同様,八登崇之さん.hyperrefは日本語pLaTeXに最適化されていないから,hyperrefとセットで,(u)pLaTeX+hyperref+dvipdfmxの組み合わせで日本語を含む「しおり」をもつPDF文書を作成する場合に必要となる機能を提供する
\definecolor{花緑青}{cmyk}{0.52,0.03,0,0.27}
\definecolor{サーモンピンク}{cmyk}{0,0.65,0.65,0.05}
\definecolor{暗中模索}{rgb}{0.2,0.2,0.2}

\usepackage{tikz}
\usetikzlibrary{positioning,automata} %automaton描画のため
\usepackage{tikz-cd}
\usepackage[all]{xy}
\def\objectstyle{\displaystyle} %デフォルトではxymatrix中の数式が文中数式モードになるので,それを直す.\labelstyleも同様にxy packageの中で定義されており,文中数式モードになっている.

\usepackage[version=4]{mhchem} %化学式をTikZで簡単に書くためのパッケージ.
\usepackage{chemfig} %化学構造式をTikZで描くためのパッケージ.
\usepackage{siunitx} %IS単位を書くためのパッケージ

\usepackage{ulem} %取り消し線を引くためのパッケージ
\usepackage{pxrubrica} %日本語にルビをふる.八登崇之(やとうたかゆき)氏による.

\usepackage{graphicx} %rotatebox, scalebox, reflectbox, resizeboxなどのコマンドや,図表の読み込み\includegraphicsを司る.graphics というパッケージもありますが,graphicx はこれを高機能にしたものと考えて結構です(ただし graphicx は内部で graphics を読み込みます)

\usepackage[breakable]{tcolorbox} %加藤晃史さんがフル活用していたtcolorboxを,途中改ページ可能で.
\tcbuselibrary{theorems} %https://qiita.com/t_kemmochi/items/483b8fcdb5db8d1f5d5e
\usepackage{enumerate} %enumerate環境を凝らせる.
\usepackage[top=15truemm,bottom=15truemm,left=10truemm,right=10truemm]{geometry} %足助さんからもらったオプション

%%%%%%%%%%%%%%% 環境マクロ %%%%%%%%%%%%%%%

\usepackage{listings} %ソースコードを表示できる環境.多分もっといい方法ある.
\usepackage{jvlisting} %日本語のコメントアウトをする場合jlistingが必要
\lstset{ %ここからソースコードの表示に関する設定.lstlisting環境では,[caption=hoge,label=fuga]などのoptionを付けられる.
%[escapechar=!]とすると,LaTeXコマンドを使える.
  basicstyle={\ttfamily},
  identifierstyle={\small},
  commentstyle={\smallitshape},
  keywordstyle={\small\bfseries},
  ndkeywordstyle={\small},
  stringstyle={\small\ttfamily},
  frame={tb},
  breaklines=true,
  columns=[l]{fullflexible},
  numbers=left,
  xrightmargin=0zw,
  xleftmargin=3zw,
  numberstyle={\scriptsize},
  stepnumber=1,
  numbersep=1zw,
  lineskip=-0.5ex
}
%\makeatletter %caption番号を「[chapter番号].[section番号].[subsection番号]-[そのsubsection内においてn番目]」に変更
%    \AtBeginDocument{
%    \renewcommand*{\thelstlisting}{\arabic{chapter}.\arabic{section}.\arabic{lstlisting}}
%    \@addtoreset{lstlisting}{section}
%    }
%\makeatother
\renewcommand{\lstlistingname}{算譜} %caption名を"program"に変更

\newtcolorbox{tbox}[3][]{%
colframe=#2,colback=#2!10,coltitle=#2!20!black,title={#3},#1}

%%%%%%%%%%%%%%% フォント %%%%%%%%%%%%%%%

\usepackage{textcomp, mathcomp} %Text Companionとは,T1 encodingに入らなかった文字群.これを使うためのパッケージ.\textsectionでブルバキに!
\usepackage[T1]{fontenc} %8bitエンコーディングにする.comp系拡張数学文字の動作が安定する.

%%%%%%%%%%%%%%% 数学記号のマクロ %%%%%%%%%%%%%%%

\newcommand{\abs}[1]{\lvert#1\rvert} %mathtoolsはこうやって使うのか!
\newcommand{\Abs}[1]{\left|#1\right|}
\newcommand{\norm}[1]{\|#1\|}
\newcommand{\Norm}[1]{\left\|#1\right\|}
%\newcommand{\brace}[1]{\{#1\}}
\newcommand{\Brace}[1]{\left\{#1\right\}}
\newcommand{\paren}[1]{\left(#1\right)}
\newcommand{\bracket}[1]{\langle#1\rangle}
\newcommand{\brac}[1]{\langle#1\rangle}
\newcommand{\Bracket}[1]{\left\langle#1\right\rangle}
\newcommand{\Brac}[1]{\left\langle#1\right\rangle}
\newcommand{\Square}[1]{\left[#1\right]}
\renewcommand{\o}[1]{\overline{#1}}
\renewcommand{\u}[1]{\underline{#1}}
\renewcommand{\iff}{\;\mathrm{iff}\;} %nLabリスペクト
\newcommand{\pp}[2]{\frac{\partial #1}{\partial #2}}
\newcommand{\ppp}[3]{\frac{\partial #1}{\partial #2\partial #3}}
\newcommand{\dd}[2]{\frac{d #1}{d #2}}
\newcommand{\floor}[1]{\lfloor#1\rfloor}
\newcommand{\Floor}[1]{\left\lfloor#1\right\rfloor}
\newcommand{\ceil}[1]{\lceil#1\rceil}

\newcommand{\iso}{\xrightarrow{\,\smash{\raisebox{-0.45ex}{\ensuremath{\scriptstyle\sim}}}\,}}
\newcommand{\wt}[1]{\widetilde{#1}}
\newcommand{\wh}[1]{\widehat{#1}}

\newcommand{\Lrarrow}{\;\;\Leftrightarrow\;\;}

%ノルム位相についての閉包 https://newbedev.com/how-to-make-double-overline-with-less-vertical-displacement
\makeatletter
\newcommand{\dbloverline}[1]{\overline{\dbl@overline{#1}}}
\newcommand{\dbl@overline}[1]{\mathpalette\dbl@@overline{#1}}
\newcommand{\dbl@@overline}[2]{%
  \begingroup
  \sbox\z@{$\m@th#1\overline{#2}$}%
  \ht\z@=\dimexpr\ht\z@-2\dbl@adjust{#1}\relax
  \box\z@
  \ifx#1\scriptstyle\kern-\scriptspace\else
  \ifx#1\scriptscriptstyle\kern-\scriptspace\fi\fi
  \endgroup
}
\newcommand{\dbl@adjust}[1]{%
  \fontdimen8
  \ifx#1\displaystyle\textfont\else
  \ifx#1\textstyle\textfont\else
  \ifx#1\scriptstyle\scriptfont\else
  \scriptscriptfont\fi\fi\fi 3
}
\makeatother
\newcommand{\oo}[1]{\dbloverline{#1}}

\DeclareMathOperator{\grad}{\mathrm{grad}}
\DeclareMathOperator{\rot}{\mathrm{rot}}
\DeclareMathOperator{\divergence}{\mathrm{div}}
\newcommand{\False}{\mathrm{False}}
\newcommand{\True}{\mathrm{True}}
\DeclareMathOperator{\tr}{\mathrm{tr}}
\newcommand{\M}{\mathcal{M}}
\newcommand{\cF}{\mathcal{F}}
\newcommand{\cD}{\mathcal{D}}
\newcommand{\fX}{\mathfrak{X}}
\newcommand{\fY}{\mathfrak{Y}}
\newcommand{\fZ}{\mathfrak{Z}}
\renewcommand{\H}{\mathcal{H}}
\newcommand{\fH}{\mathfrak{H}}
\newcommand{\bH}{\mathbb{H}}
\newcommand{\id}{\mathrm{id}}
\newcommand{\A}{\mathcal{A}}
% \renewcommand\coprod{\rotatebox[origin=c]{180}{$\prod$}} すでにどこかにある.
\newcommand{\pr}{\mathrm{pr}}
\newcommand{\U}{\mathfrak{U}}
\newcommand{\Map}{\mathrm{Map}}
\newcommand{\dom}{\mathrm{Dom}\;}
\newcommand{\cod}{\mathrm{Cod}\;}
\newcommand{\supp}{\mathrm{supp}\;}
\newcommand{\otherwise}{\mathrm{otherwise}}
\newcommand{\st}{\;\mathrm{s.t.}\;}
\newcommand{\lmd}{\lambda}
\newcommand{\Lmd}{\Lambda}
%%% 線型代数学
\newcommand{\Ker}{\mathrm{Ker}\;}
\newcommand{\Coker}{\mathrm{Coker}\;}
\newcommand{\Coim}{\mathrm{Coim}\;}
\newcommand{\rank}{\mathrm{rank}}
\newcommand{\lcm}{\mathrm{lcm}}
\newcommand{\sgn}{\mathrm{sgn}}
\newcommand{\GL}{\mathrm{GL}}
\newcommand{\SL}{\mathrm{SL}}
\newcommand{\alt}{\mathrm{alt}}
%%% 複素解析学
\renewcommand{\Re}{\mathrm{Re}\;}
\renewcommand{\Im}{\mathrm{Im}\;}
\newcommand{\Gal}{\mathrm{Gal}}
\newcommand{\PGL}{\mathrm{PGL}}
\newcommand{\PSL}{\mathrm{PSL}}
\newcommand{\Log}{\mathrm{Log}\,}
\newcommand{\Res}{\mathrm{Res}\,}
\newcommand{\on}{\mathrm{on}\;}
\newcommand{\hatC}{\hat{\C}}
\newcommand{\hatR}{\hat{\R}}
\newcommand{\PV}{\mathrm{P.V.}}
\newcommand{\diam}{\mathrm{diam}}
\newcommand{\Area}{\mathrm{Area}}
\newcommand{\Lap}{\Laplace}
\newcommand{\f}{\mathbf{f}}
\newcommand{\cR}{\mathcal{R}}
\newcommand{\const}{\mathrm{const.}}
\newcommand{\Om}{\Omega}
\newcommand{\Cinf}{C^\infty}
\newcommand{\ep}{\epsilon}
\newcommand{\dist}{\mathrm{dist}}
\newcommand{\opart}{\o{\partial}}
%%% 解析力学
\newcommand{\x}{\mathbf{x}}
%%% 集合と位相
\renewcommand{\O}{\mathcal{O}}
\renewcommand{\S}{\mathcal{S}}
\renewcommand{\U}{\mathcal{U}}
\newcommand{\V}{\mathcal{V}}
\renewcommand{\P}{\mathcal{P}}
\newcommand{\R}{\mathbb{R}}
\newcommand{\N}{\mathbb{N}}
\newcommand{\C}{\mathbb{C}}
\newcommand{\Z}{\mathbb{Z}}
\newcommand{\Q}{\mathbb{Q}}
\newcommand{\TV}{\mathrm{TV}}
\newcommand{\ORD}{\mathrm{ORD}}
\newcommand{\Tr}{\mathrm{Tr}\;}
\newcommand{\Card}{\mathrm{Card}\;}
\newcommand{\Top}{\mathrm{Top}}
\newcommand{\Disc}{\mathrm{Disc}}
\newcommand{\Codisc}{\mathrm{Codisc}}
\newcommand{\CoDisc}{\mathrm{CoDisc}}
\newcommand{\Ult}{\mathrm{Ult}}
\newcommand{\ord}{\mathrm{ord}}
\newcommand{\maj}{\mathrm{maj}}
%%% 形式言語理論
\newcommand{\REGEX}{\mathrm{REGEX}}
\newcommand{\RE}{\mathbf{RE}}

%%% Fourier解析
\newcommand*{\Laplace}{\mathop{}\!\mathbin\bigtriangleup}
\newcommand*{\DAlambert}{\mathop{}\!\mathbin\Box}
%%% Graph Theory
\newcommand{\SimpGph}{\mathrm{SimpGph}}
\newcommand{\Gph}{\mathrm{Gph}}
\newcommand{\mult}{\mathrm{mult}}
\newcommand{\inv}{\mathrm{inv}}
%%% 多様体
\newcommand{\Der}{\mathrm{Der}}
\newcommand{\osub}{\overset{\mathrm{open}}{\subset}}
\newcommand{\osup}{\overset{\mathrm{open}}{\supset}}
\newcommand{\al}{\alpha}
\newcommand{\K}{\mathbb{K}}
\newcommand{\Sp}{\mathrm{Sp}}
\newcommand{\g}{\mathfrak{g}}
\newcommand{\h}{\mathfrak{h}}
\newcommand{\Exp}{\mathrm{Exp}\;}
\newcommand{\Imm}{\mathrm{Imm}}
\newcommand{\Imb}{\mathrm{Imb}}
\newcommand{\codim}{\mathrm{codim}\;}
\newcommand{\Gr}{\mathrm{Gr}}
%%% 代数
\newcommand{\Ad}{\mathrm{Ad}}
\newcommand{\finsupp}{\mathrm{fin\;supp}}
\newcommand{\SO}{\mathrm{SO}}
\newcommand{\SU}{\mathrm{SU}}
\newcommand{\acts}{\curvearrowright}
\newcommand{\mono}{\hookrightarrow}
\newcommand{\epi}{\twoheadrightarrow}
\newcommand{\Stab}{\mathrm{Stab}}
\newcommand{\nor}{\mathrm{nor}}
\newcommand{\T}{\mathbb{T}}
\newcommand{\Aff}{\mathrm{Aff}}
\newcommand{\rsub}{\triangleleft}
\newcommand{\rsup}{\triangleright}
\newcommand{\subgrp}{\overset{\mathrm{subgrp}}{\subset}}
\newcommand{\Ext}{\mathrm{Ext}}
\newcommand{\sbs}{\subset}\newcommand{\sps}{\supset}
\newcommand{\In}{\mathrm{In}}
\newcommand{\Tor}{\mathrm{Tor}}
\newcommand{\p}{\mathfrak{p}}
\newcommand{\q}{\mathfrak{q}}
\newcommand{\m}{\mathfrak{m}}
\newcommand{\cS}{\mathcal{S}}
\newcommand{\Frac}{\mathrm{Frac}\,}
\newcommand{\Spec}{\mathrm{Spec}\,}
\newcommand{\bA}{\mathbb{A}}
\newcommand{\Sym}{\mathrm{Sym}}
\newcommand{\Ann}{\mathrm{Ann}}
%%% 代数的位相幾何学
\newcommand{\Ho}{\mathrm{Ho}}
\newcommand{\CW}{\mathrm{CW}}
\newcommand{\lc}{\mathrm{lc}}
\newcommand{\cg}{\mathrm{cg}}
\newcommand{\Fib}{\mathrm{Fib}}
\newcommand{\Cyl}{\mathrm{Cyl}}
\newcommand{\Ch}{\mathrm{Ch}}
%%% 数値解析
\newcommand{\round}{\mathrm{round}}
\newcommand{\cond}{\mathrm{cond}}
\newcommand{\diag}{\mathrm{diag}}
%%% 確率論
\newcommand{\calF}{\mathcal{F}}
\newcommand{\X}{\mathcal{X}}
\newcommand{\Meas}{\mathrm{Meas}}
\newcommand{\as}{\;\mathrm{a.s.}} %almost surely
\newcommand{\io}{\;\mathrm{i.o.}} %infinitely often
\newcommand{\fe}{\;\mathrm{f.e.}} %with a finite number of exceptions
\newcommand{\F}{\mathcal{F}}
\newcommand{\bF}{\mathbb{F}}
\newcommand{\W}{\mathcal{W}}
\newcommand{\Pois}{\mathrm{Pois}}
\newcommand{\iid}{\mathrm{i.i.d.}}
\newcommand{\wconv}{\rightsquigarrow}
\newcommand{\Var}{\mathrm{Var}}
\newcommand{\xrightarrown}{\xrightarrow{n\to\infty}}
\newcommand{\au}{\mathrm{au}}
\newcommand{\cT}{\mathcal{T}}
%%% 情報理論
\newcommand{\bit}{\mathrm{bit}}
%%% 積分論
\newcommand{\calA}{\mathcal{A}}
\newcommand{\calB}{\mathcal{B}}
\newcommand{\D}{\mathcal{D}}
\newcommand{\Y}{\mathcal{Y}}
\newcommand{\calC}{\mathcal{C}}
\renewcommand{\ae}{\mathrm{a.e.}\;}
\newcommand{\cZ}{\mathcal{Z}}
\newcommand{\fF}{\mathfrak{F}}
\newcommand{\fI}{\mathfrak{I}}
\newcommand{\E}{\mathcal{E}}
\newcommand{\sMap}{\sigma\textrm{-}\mathrm{Map}}
\DeclareMathOperator*{\argmax}{arg\,max}
\DeclareMathOperator*{\argmin}{arg\,min}
\newcommand{\cC}{\mathcal{C}}
\newcommand{\comp}{\complement}
\newcommand{\J}{\mathcal{J}}
\newcommand{\sumN}[1]{\sum_{#1\in\N}}
\newcommand{\cupN}[1]{\cup_{#1\in\N}}
\newcommand{\capN}[1]{\cap_{#1\in\N}}
\newcommand{\Sum}[1]{\sum_{#1=1}^\infty}
\newcommand{\sumn}{\sum_{n=1}^\infty}
\newcommand{\summ}{\sum_{m=1}^\infty}
\newcommand{\sumk}{\sum_{k=1}^\infty}
\newcommand{\sumi}{\sum_{i=1}^\infty}
\newcommand{\sumj}{\sum_{j=1}^\infty}
\newcommand{\cupn}{\cup_{n=1}^\infty}
\newcommand{\capn}{\cap_{n=1}^\infty}
\newcommand{\cupk}{\cup_{k=1}^\infty}
\newcommand{\cupi}{\cup_{i=1}^\infty}
\newcommand{\cupj}{\cup_{j=1}^\infty}
\newcommand{\limn}{\lim_{n\to\infty}}
\renewcommand{\l}{\mathcal{l}}
\renewcommand{\L}{\mathcal{L}}
\newcommand{\Cl}{\mathrm{Cl}}
\newcommand{\cN}{\mathcal{N}}
\newcommand{\Ae}{\textrm{-a.e.}\;}
\newcommand{\csub}{\overset{\textrm{closed}}{\subset}}
\newcommand{\csup}{\overset{\textrm{closed}}{\supset}}
\newcommand{\wB}{\wt{B}}
\newcommand{\cG}{\mathcal{G}}
\newcommand{\Lip}{\mathrm{Lip}}
\newcommand{\Dom}{\mathrm{Dom}}
%%% 数理ファイナンス
\newcommand{\pre}{\mathrm{pre}}
\newcommand{\om}{\omega}

%%% 統計的因果推論
\newcommand{\Do}{\mathrm{Do}}
%%% 数理統計
\newcommand{\bP}{\mathbb{P}}
\newcommand{\compsub}{\overset{\textrm{cpt}}{\subset}}
\newcommand{\lip}{\textrm{lip}}
\newcommand{\BL}{\mathrm{BL}}
\newcommand{\G}{\mathbb{G}}
\newcommand{\NB}{\mathrm{NB}}
\newcommand{\oR}{\o{\R}}
\newcommand{\liminfn}{\liminf_{n\to\infty}}
\newcommand{\limsupn}{\limsup_{n\to\infty}}
%\newcommand{\limn}{\lim_{n\to\infty}}
\newcommand{\esssup}{\mathrm{ess.sup}}
\newcommand{\asto}{\xrightarrow{\as}}
\newcommand{\Cov}{\mathrm{Cov}}
\newcommand{\cQ}{\mathcal{Q}}
\newcommand{\VC}{\mathrm{VC}}
\newcommand{\mb}{\mathrm{mb}}
\newcommand{\Avar}{\mathrm{Avar}}
\newcommand{\bB}{\mathbb{B}}
\newcommand{\bW}{\mathbb{W}}
\newcommand{\sd}{\mathrm{sd}}
\newcommand{\w}[1]{\widehat{#1}}
\newcommand{\bZ}{\mathbb{Z}}
\newcommand{\Bernoulli}{\mathrm{Bernoulli}}
\newcommand{\Mult}{\mathrm{Mult}}
\newcommand{\BPois}{\mathrm{BPois}}
\newcommand{\fraks}{\mathfrak{s}}
\newcommand{\frakk}{\mathfrak{k}}
\newcommand{\IF}{\mathrm{IF}}
\newcommand{\bX}{\mathbf{X}}
\newcommand{\bx}{\mathbf{x}}
\newcommand{\indep}{\raisebox{0.05em}{\rotatebox[origin=c]{90}{$\models$}}}
\newcommand{\IG}{\mathrm{IG}}
\newcommand{\Levy}{\mathrm{Levy}}
\newcommand{\MP}{\mathrm{MP}}
\newcommand{\Hermite}{\mathrm{Hermite}}
\newcommand{\Skellam}{\mathrm{Skellam}}
\newcommand{\Dirichlet}{\mathrm{Dirichlet}}
\newcommand{\Beta}{\mathrm{Beta}}
\newcommand{\bE}{\mathbb{E}}
\newcommand{\bG}{\mathbb{G}}
\newcommand{\MISE}{\mathrm{MISE}}
\newcommand{\logit}{\mathtt{logit}}
\newcommand{\expit}{\mathtt{expit}}
\newcommand{\cK}{\mathcal{K}}
\newcommand{\dl}{\dot{l}}
\newcommand{\dotp}{\dot{p}}
\newcommand{\wl}{\wt{l}}
%%% 函数解析
\renewcommand{\c}{\mathbf{c}}
\newcommand{\loc}{\mathrm{loc}}
\newcommand{\Lh}{\mathrm{L.h.}}
\newcommand{\Epi}{\mathrm{Epi}\;}
\newcommand{\slim}{\mathrm{slim}}
\newcommand{\Ban}{\mathrm{Ban}}
\newcommand{\Hilb}{\mathrm{Hilb}}
\newcommand{\Ex}{\mathrm{Ex}}
\newcommand{\Co}{\mathrm{Co}}
\newcommand{\sa}{\mathrm{sa}}
\newcommand{\nnorm}[1]{{\left\vert\kern-0.25ex\left\vert\kern-0.25ex\left\vert #1 \right\vert\kern-0.25ex\right\vert\kern-0.25ex\right\vert}}
\newcommand{\dvol}{\mathrm{dvol}}
\newcommand{\Sconv}{\mathrm{Sconv}}
\newcommand{\I}{\mathcal{I}}
\newcommand{\nonunital}{\mathrm{nu}}
\newcommand{\cpt}{\mathrm{cpt}}
\newcommand{\lcpt}{\mathrm{lcpt}}
\newcommand{\com}{\mathrm{com}}
\newcommand{\Haus}{\mathrm{Haus}}
\newcommand{\proper}{\mathrm{proper}}
\newcommand{\infinity}{\mathrm{inf}}
\newcommand{\TVS}{\mathrm{TVS}}
\newcommand{\ess}{\mathrm{ess}}
\newcommand{\ext}{\mathrm{ext}}
\newcommand{\Index}{\mathrm{Index}}
\newcommand{\SSR}{\mathrm{SSR}}
\newcommand{\vs}{\mathrm{vs.}}
\newcommand{\fM}{\mathfrak{M}}
\newcommand{\EDM}{\mathrm{EDM}}
\newcommand{\Tw}{\mathrm{Tw}}
\newcommand{\fC}{\mathfrak{C}}
\newcommand{\bn}{\mathbf{n}}
\newcommand{\br}{\mathbf{r}}
\newcommand{\Lam}{\Lambda}
\newcommand{\lam}{\lambda}
\newcommand{\one}{\mathbf{1}}
\newcommand{\dae}{\text{-a.e.}}
\newcommand{\td}{\text{-}}
\newcommand{\RM}{\mathrm{RM}}
%%% 最適化
\newcommand{\Minimize}{\text{Minimize}}
\newcommand{\subjectto}{\text{subject to}}
\newcommand{\Ri}{\mathrm{Ri}}
%\newcommand{\Cl}{\mathrm{Cl}}
\newcommand{\Cone}{\mathrm{Cone}}
\newcommand{\Int}{\mathrm{Int}}
%%% 圏
\newcommand{\varlim}{\varprojlim}
\newcommand{\Hom}{\mathrm{Hom}}
\newcommand{\Iso}{\mathrm{Iso}}
\newcommand{\Mor}{\mathrm{Mor}}
\newcommand{\Isom}{\mathrm{Isom}}
\newcommand{\Aut}{\mathrm{Aut}}
\newcommand{\End}{\mathrm{End}}
\newcommand{\op}{\mathrm{op}}
\newcommand{\ev}{\mathrm{ev}}
\newcommand{\Ob}{\mathrm{Ob}}
\newcommand{\Ar}{\mathrm{Ar}}
\newcommand{\Arr}{\mathrm{Arr}}
\newcommand{\Set}{\mathrm{Set}}
\newcommand{\Grp}{\mathrm{Grp}}
\newcommand{\Cat}{\mathrm{Cat}}
\newcommand{\Mon}{\mathrm{Mon}}
\newcommand{\CMon}{\mathrm{CMon}} %Comutative Monoid 可換単系とモノイドの射
\newcommand{\Ring}{\mathrm{Ring}}
\newcommand{\CRing}{\mathrm{CRing}}
\newcommand{\Ab}{\mathrm{Ab}}
\newcommand{\Pos}{\mathrm{Pos}}
\newcommand{\Vect}{\mathrm{Vect}}
\newcommand{\FinVect}{\mathrm{FinVect}}
\newcommand{\FinSet}{\mathrm{FinSet}}
\newcommand{\OmegaAlg}{\Omega$-$\mathrm{Alg}}
\newcommand{\OmegaEAlg}{(\Omega,E)$-$\mathrm{Alg}}
\newcommand{\Alg}{\mathrm{Alg}} %代数の圏
\newcommand{\CAlg}{\mathrm{CAlg}} %可換代数の圏
\newcommand{\CPO}{\mathrm{CPO}} %Complete Partial Order & continuous mappings
\newcommand{\Fun}{\mathrm{Fun}}
\newcommand{\Func}{\mathrm{Func}}
\newcommand{\Met}{\mathrm{Met}} %Metric space & Contraction maps
\newcommand{\Pfn}{\mathrm{Pfn}} %Sets & Partial function
\newcommand{\Rel}{\mathrm{Rel}} %Sets & relation
\newcommand{\Bool}{\mathrm{Bool}}
\newcommand{\CABool}{\mathrm{CABool}}
\newcommand{\CompBoolAlg}{\mathrm{CompBoolAlg}}
\newcommand{\BoolAlg}{\mathrm{BoolAlg}}
\newcommand{\BoolRng}{\mathrm{BoolRng}}
\newcommand{\HeytAlg}{\mathrm{HeytAlg}}
\newcommand{\CompHeytAlg}{\mathrm{CompHeytAlg}}
\newcommand{\Lat}{\mathrm{Lat}}
\newcommand{\CompLat}{\mathrm{CompLat}}
\newcommand{\SemiLat}{\mathrm{SemiLat}}
\newcommand{\Stone}{\mathrm{Stone}}
\newcommand{\Sob}{\mathrm{Sob}} %Sober space & continuous map
\newcommand{\Op}{\mathrm{Op}} %Category of open subsets
\newcommand{\Sh}{\mathrm{Sh}} %Category of sheave
\newcommand{\PSh}{\mathrm{PSh}} %Category of presheave, PSh(C)=[C^op,set]のこと
\newcommand{\Conv}{\mathrm{Conv}} %Convergence spaceの圏
\newcommand{\Unif}{\mathrm{Unif}} %一様空間と一様連続写像の圏
\newcommand{\Frm}{\mathrm{Frm}} %フレームとフレームの射
\newcommand{\Locale}{\mathrm{Locale}} %その反対圏
\newcommand{\Diff}{\mathrm{Diff}} %滑らかな多様体の圏
\newcommand{\Mfd}{\mathrm{Mfd}}
\newcommand{\LieAlg}{\mathrm{LieAlg}}
\newcommand{\Quiv}{\mathrm{Quiv}} %Quiverの圏
\newcommand{\B}{\mathcal{B}}
\newcommand{\Span}{\mathrm{Span}}
\newcommand{\Corr}{\mathrm{Corr}}
\newcommand{\Decat}{\mathrm{Decat}}
\newcommand{\Rep}{\mathrm{Rep}}
\newcommand{\Grpd}{\mathrm{Grpd}}
\newcommand{\sSet}{\mathrm{sSet}}
\newcommand{\Mod}{\mathrm{Mod}}
\newcommand{\SmoothMnf}{\mathrm{SmoothMnf}}
\newcommand{\coker}{\mathrm{coker}}

\newcommand{\Ord}{\mathrm{Ord}}
\newcommand{\eq}{\mathrm{eq}}
\newcommand{\coeq}{\mathrm{coeq}}
\newcommand{\act}{\mathrm{act}}

%%%%%%%%%%%%%%% 定理環境(足助先生ありがとうございます) %%%%%%%%%%%%%%%

\everymath{\displaystyle}
\renewcommand{\proofname}{\bf [証明]}
\renewcommand{\thefootnote}{\dag\arabic{footnote}} %足助さんからもらった.どうなるんだ?
\renewcommand{\qedsymbol}{$\blacksquare$}

\renewcommand{\labelenumi}{(\arabic{enumi})} %(1),(2),...がデフォルトであって欲しい
\renewcommand{\labelenumii}{(\alph{enumii})}
\renewcommand{\labelenumiii}{(\roman{enumiii})}

\newtheoremstyle{StatementsWithStar}% ?name?
{3pt}% ?Space above? 1
{3pt}% ?Space below? 1
{}% ?Body font?
{}% ?Indent amount? 2
{\bfseries}% ?Theorem head font?
{\textbf{.}}% ?Punctuation after theorem head?
{.5em}% ?Space after theorem head? 3
{\textbf{\textup{#1~\thetheorem{}}}{}\,$^{\ast}$\thmnote{(#3)}}% ?Theorem head spec (can be left empty, meaning ‘normal’)?
%
\newtheoremstyle{StatementsWithStar2}% ?name?
{3pt}% ?Space above? 1
{3pt}% ?Space below? 1
{}% ?Body font?
{}% ?Indent amount? 2
{\bfseries}% ?Theorem head font?
{\textbf{.}}% ?Punctuation after theorem head?
{.5em}% ?Space after theorem head? 3
{\textbf{\textup{#1~\thetheorem{}}}{}\,$^{\ast\ast}$\thmnote{(#3)}}% ?Theorem head spec (can be left empty, meaning ‘normal’)?
%
\newtheoremstyle{StatementsWithStar3}% ?name?
{3pt}% ?Space above? 1
{3pt}% ?Space below? 1
{}% ?Body font?
{}% ?Indent amount? 2
{\bfseries}% ?Theorem head font?
{\textbf{.}}% ?Punctuation after theorem head?
{.5em}% ?Space after theorem head? 3
{\textbf{\textup{#1~\thetheorem{}}}{}\,$^{\ast\ast\ast}$\thmnote{(#3)}}% ?Theorem head spec (can be left empty, meaning ‘normal’)?
%
\newtheoremstyle{StatementsWithCCirc}% ?name?
{6pt}% ?Space above? 1
{6pt}% ?Space below? 1
{}% ?Body font?
{}% ?Indent amount? 2
{\bfseries}% ?Theorem head font?
{\textbf{.}}% ?Punctuation after theorem head?
{.5em}% ?Space after theorem head? 3
{\textbf{\textup{#1~\thetheorem{}}}{}\,$^{\circledcirc}$\thmnote{(#3)}}% ?Theorem head spec (can be left empty, meaning ‘normal’)?
%
\theoremstyle{definition}
 \newtheorem{theorem}{定理}[section]
 \newtheorem{axiom}[theorem]{公理}
 \newtheorem{corollary}[theorem]{系}
 \newtheorem{proposition}[theorem]{命題}
 \newtheorem*{proposition*}{命題}
 \newtheorem{lemma}[theorem]{補題}
 \newtheorem*{lemma*}{補題}
 \newtheorem*{theorem*}{定理}
 \newtheorem{definition}[theorem]{定義}
 \newtheorem{example}[theorem]{例}
 \newtheorem{notation}[theorem]{記法}
 \newtheorem*{notation*}{記法}
 \newtheorem{assumption}[theorem]{仮定}
 \newtheorem{question}[theorem]{問}
 \newtheorem{counterexample}[theorem]{反例}
 \newtheorem{reidai}[theorem]{例題}
 \newtheorem{ruidai}[theorem]{類題}
 \newtheorem{problem}[theorem]{問題}
 \newtheorem{algorithm}[theorem]{算譜}
 \newtheorem*{solution*}{\bf{[解]}}
 \newtheorem{discussion}[theorem]{議論}
 \newtheorem{remark}[theorem]{注}
 \newtheorem{remarks}[theorem]{要諦}
 \newtheorem{image}[theorem]{描像}
 \newtheorem{observation}[theorem]{観察}
 \newtheorem{universality}[theorem]{普遍性} %非自明な例外がない.
 \newtheorem{universal tendency}[theorem]{普遍傾向} %例外が有意に少ない.
 \newtheorem{hypothesis}[theorem]{仮説} %実験で説明されていない理論.
 \newtheorem{theory}[theorem]{理論} %実験事実とその(さしあたり)整合的な説明.
 \newtheorem{fact}[theorem]{実験事実}
 \newtheorem{model}[theorem]{模型}
 \newtheorem{explanation}[theorem]{説明} %理論による実験事実の説明
 \newtheorem{anomaly}[theorem]{理論の限界}
 \newtheorem{application}[theorem]{応用例}
 \newtheorem{method}[theorem]{手法} %実験手法など,技術的問題.
 \newtheorem{history}[theorem]{歴史}
 \newtheorem{usage}[theorem]{用語法}
 \newtheorem{research}[theorem]{研究}
 \newtheorem{shishin}[theorem]{指針}
 \newtheorem{yodan}[theorem]{余談}
 \newtheorem{construction}[theorem]{構成}
% \newtheorem*{remarknonum}{注}
 \newtheorem*{definition*}{定義}
 \newtheorem*{remark*}{注}
 \newtheorem*{question*}{問}
 \newtheorem*{problem*}{問題}
 \newtheorem*{axiom*}{公理}
 \newtheorem*{example*}{例}
 \newtheorem*{corollary*}{系}
 \newtheorem*{shishin*}{指針}
 \newtheorem*{yodan*}{余談}
 \newtheorem*{kadai*}{課題}
%
\theoremstyle{StatementsWithStar}
 \newtheorem{definition_*}[theorem]{定義}
 \newtheorem{question_*}[theorem]{問}
 \newtheorem{example_*}[theorem]{例}
 \newtheorem{theorem_*}[theorem]{定理}
 \newtheorem{remark_*}[theorem]{注}
%
\theoremstyle{StatementsWithStar2}
 \newtheorem{definition_**}[theorem]{定義}
 \newtheorem{theorem_**}[theorem]{定理}
 \newtheorem{question_**}[theorem]{問}
 \newtheorem{remark_**}[theorem]{注}
%
\theoremstyle{StatementsWithStar3}
 \newtheorem{remark_***}[theorem]{注}
 \newtheorem{question_***}[theorem]{問}
%
\theoremstyle{StatementsWithCCirc}
 \newtheorem{definition_O}[theorem]{定義}
 \newtheorem{question_O}[theorem]{問}
 \newtheorem{example_O}[theorem]{例}
 \newtheorem{remark_O}[theorem]{注}
%
\theoremstyle{definition}
%
\raggedbottom
\allowdisplaybreaks
%\usepackage{mathtools}
%\mathtoolsset{showonlyrefs=true} %labelを附した数式にのみ附番される設定.
%\usepackage{amsmath} %mathtoolsの内部で呼ばれるので要らない.
\usepackage{amsfonts} %mathfrak, mathcal, mathbbなど.
\usepackage{amsthm} %定理環境.
\usepackage{amssymb} %AMSFontsを使うためのパッケージ.
\usepackage{ascmac} %screen, itembox, shadebox環境.全てLATEX2εの標準機能の範囲で作られたもの.
\usepackage{comment} %comment環境を用いて,複数行をcomment outできるようにするpackage
\usepackage{wrapfig} %図の周りに文字をwrapさせることができる.詳細な制御ができる.
\usepackage[usenames, dvipsnames]{xcolor} %xcolorはcolorの拡張.optionの意味はdvipsnamesはLoad a set of predefined colors. forestgreenなどの色が追加されている.usenamesはobsoleteとだけ書いてあった.
\setcounter{tocdepth}{2} %目次に表示される深さ.2はsubsectionまで
\usepackage{multicol} %\begin{multicols}{2}環境で途中からmulticolumnに出来る.

\usepackage{url}
\usepackage[dvipdfmx,colorlinks,linkcolor=blue,urlcolor=blue]{hyperref} %生成されるPDFファイルにおいて、\tableofcontentsによって書き出された目次をクリックすると該当する見出しへジャンプしたり、さらには、\label{ラベル名}を番号で参照する\ref{ラベル名}やthebibliography環境において\bibitem{ラベル名}を文献番号で参照する\cite{ラベル名}においても番号をクリックすると該当箇所にジャンプする.囲み枠はダサいので,colorlinksで囲み廃止し,リンク自体に色を付けることにした.
\usepackage{pxjahyper} %pxrubrica同様,八登崇之さん.hyperrefは日本語pLaTeXに最適化されていないから,hyperrefとセットで,(u)pLaTeX+hyperref+dvipdfmxの組み合わせで日本語を含む「しおり」をもつPDF文書を作成する場合に必要となる機能を提供する
\definecolor{花緑青}{cmyk}{0.52,0.03,0,0.27}
\definecolor{サーモンピンク}{cmyk}{0,0.65,0.65,0.05}
\definecolor{暗中模索}{rgb}{0.2,0.2,0.2}

\usepackage{tikz}
\usetikzlibrary{positioning,automata} %automaton描画のため
\usepackage{tikz-cd}
\usepackage[all]{xy}
\def\objectstyle{\displaystyle} %デフォルトではxymatrix中の数式が文中数式モードになるので,それを直す.\labelstyleも同様にxy packageの中で定義されており,文中数式モードになっている.

\usepackage[version=4]{mhchem} %化学式をTikZで簡単に書くためのパッケージ.
\usepackage{chemfig} %化学構造式をTikZで描くためのパッケージ.
\usepackage{siunitx} %IS単位を書くためのパッケージ

\usepackage{ulem} %取り消し線を引くためのパッケージ
\usepackage{pxrubrica} %日本語にルビをふる.八登崇之(やとうたかゆき)氏による.

\usepackage{graphicx} %rotatebox, scalebox, reflectbox, resizeboxなどのコマンドや,図表の読み込み\includegraphicsを司る.graphics というパッケージもありますが,graphicx はこれを高機能にしたものと考えて結構です(ただし graphicx は内部で graphics を読み込みます)

\usepackage[breakable]{tcolorbox} %加藤晃史さんがフル活用していたtcolorboxを,途中改ページ可能で.
\tcbuselibrary{theorems} %https://qiita.com/t_kemmochi/items/483b8fcdb5db8d1f5d5e
\usepackage{enumerate} %enumerate環境を凝らせる.
\usepackage[top=15truemm,bottom=15truemm,left=10truemm,right=10truemm]{geometry} %足助さんからもらったオプション

%%%%%%%%%%%%%%% 環境マクロ %%%%%%%%%%%%%%%

\usepackage{listings} %ソースコードを表示できる環境.多分もっといい方法ある.
\usepackage{jvlisting} %日本語のコメントアウトをする場合jlistingが必要
\lstset{ %ここからソースコードの表示に関する設定.lstlisting環境では,[caption=hoge,label=fuga]などのoptionを付けられる.
%[escapechar=!]とすると,LaTeXコマンドを使える.
  basicstyle={\ttfamily},
  identifierstyle={\small},
  commentstyle={\smallitshape},
  keywordstyle={\small\bfseries},
  ndkeywordstyle={\small},
  stringstyle={\small\ttfamily},
  frame={tb},
  breaklines=true,
  columns=[l]{fullflexible},
  numbers=left,
  xrightmargin=0zw,
  xleftmargin=3zw,
  numberstyle={\scriptsize},
  stepnumber=1,
  numbersep=1zw,
  lineskip=-0.5ex
}
%\makeatletter %caption番号を「[chapter番号].[section番号].[subsection番号]-[そのsubsection内においてn番目]」に変更
%    \AtBeginDocument{
%    \renewcommand*{\thelstlisting}{\arabic{chapter}.\arabic{section}.\arabic{lstlisting}}
%    \@addtoreset{lstlisting}{section}
%    }
%\makeatother
\renewcommand{\lstlistingname}{算譜} %caption名を"program"に変更

\newtcolorbox{tbox}[3][]{%
colframe=#2,colback=#2!10,coltitle=#2!20!black,title={#3},#1}

%%%%%%%%%%%%%%% フォント %%%%%%%%%%%%%%%

\usepackage{textcomp, mathcomp} %Text Companionとは,T1 encodingに入らなかった文字群.これを使うためのパッケージ.\textsectionでブルバキに!
\usepackage[T1]{fontenc} %8bitエンコーディングにする.comp系拡張数学文字の動作が安定する.

%%%%%%%%%%%%%%% 数学記号のマクロ %%%%%%%%%%%%%%%

\newcommand{\abs}[1]{\lvert#1\rvert} %mathtoolsはこうやって使うのか!
\newcommand{\Abs}[1]{\left|#1\right|}
\newcommand{\norm}[1]{\|#1\|}
\newcommand{\Norm}[1]{\left\|#1\right\|}
%\newcommand{\brace}[1]{\{#1\}}
\newcommand{\Brace}[1]{\left\{#1\right\}}
\newcommand{\paren}[1]{\left(#1\right)}
\newcommand{\bracket}[1]{\langle#1\rangle}
\newcommand{\brac}[1]{\langle#1\rangle}
\newcommand{\Bracket}[1]{\left\langle#1\right\rangle}
\newcommand{\Brac}[1]{\left\langle#1\right\rangle}
\newcommand{\Square}[1]{\left[#1\right]}
\renewcommand{\o}[1]{\overline{#1}}
\renewcommand{\u}[1]{\underline{#1}}
\renewcommand{\iff}{\;\mathrm{iff}\;} %nLabリスペクト
\newcommand{\pp}[2]{\frac{\partial #1}{\partial #2}}
\newcommand{\ppp}[3]{\frac{\partial #1}{\partial #2\partial #3}}
\newcommand{\dd}[2]{\frac{d #1}{d #2}}
\newcommand{\floor}[1]{\lfloor#1\rfloor}
\newcommand{\Floor}[1]{\left\lfloor#1\right\rfloor}
\newcommand{\ceil}[1]{\lceil#1\rceil}

\newcommand{\iso}{\xrightarrow{\,\smash{\raisebox{-0.45ex}{\ensuremath{\scriptstyle\sim}}}\,}}
\newcommand{\wt}[1]{\widetilde{#1}}
\newcommand{\wh}[1]{\widehat{#1}}

\newcommand{\Lrarrow}{\;\;\Leftrightarrow\;\;}

%ノルム位相についての閉包 https://newbedev.com/how-to-make-double-overline-with-less-vertical-displacement
\makeatletter
\newcommand{\dbloverline}[1]{\overline{\dbl@overline{#1}}}
\newcommand{\dbl@overline}[1]{\mathpalette\dbl@@overline{#1}}
\newcommand{\dbl@@overline}[2]{%
  \begingroup
  \sbox\z@{$\m@th#1\overline{#2}$}%
  \ht\z@=\dimexpr\ht\z@-2\dbl@adjust{#1}\relax
  \box\z@
  \ifx#1\scriptstyle\kern-\scriptspace\else
  \ifx#1\scriptscriptstyle\kern-\scriptspace\fi\fi
  \endgroup
}
\newcommand{\dbl@adjust}[1]{%
  \fontdimen8
  \ifx#1\displaystyle\textfont\else
  \ifx#1\textstyle\textfont\else
  \ifx#1\scriptstyle\scriptfont\else
  \scriptscriptfont\fi\fi\fi 3
}
\makeatother
\newcommand{\oo}[1]{\dbloverline{#1}}

\DeclareMathOperator{\grad}{\mathrm{grad}}
\DeclareMathOperator{\rot}{\mathrm{rot}}
\DeclareMathOperator{\divergence}{\mathrm{div}}
\newcommand{\False}{\mathrm{False}}
\newcommand{\True}{\mathrm{True}}
\DeclareMathOperator{\tr}{\mathrm{tr}}
\newcommand{\M}{\mathcal{M}}
\newcommand{\cF}{\mathcal{F}}
\newcommand{\cD}{\mathcal{D}}
\newcommand{\fX}{\mathfrak{X}}
\newcommand{\fY}{\mathfrak{Y}}
\newcommand{\fZ}{\mathfrak{Z}}
\renewcommand{\H}{\mathcal{H}}
\newcommand{\fH}{\mathfrak{H}}
\newcommand{\bH}{\mathbb{H}}
\newcommand{\id}{\mathrm{id}}
\newcommand{\A}{\mathcal{A}}
% \renewcommand\coprod{\rotatebox[origin=c]{180}{$\prod$}} すでにどこかにある.
\newcommand{\pr}{\mathrm{pr}}
\newcommand{\U}{\mathfrak{U}}
\newcommand{\Map}{\mathrm{Map}}
\newcommand{\dom}{\mathrm{Dom}\;}
\newcommand{\cod}{\mathrm{Cod}\;}
\newcommand{\supp}{\mathrm{supp}\;}
\newcommand{\otherwise}{\mathrm{otherwise}}
\newcommand{\st}{\;\mathrm{s.t.}\;}
\newcommand{\lmd}{\lambda}
\newcommand{\Lmd}{\Lambda}
%%% 線型代数学
\newcommand{\Ker}{\mathrm{Ker}\;}
\newcommand{\Coker}{\mathrm{Coker}\;}
\newcommand{\Coim}{\mathrm{Coim}\;}
\newcommand{\rank}{\mathrm{rank}}
\newcommand{\lcm}{\mathrm{lcm}}
\newcommand{\sgn}{\mathrm{sgn}}
\newcommand{\GL}{\mathrm{GL}}
\newcommand{\SL}{\mathrm{SL}}
\newcommand{\alt}{\mathrm{alt}}
%%% 複素解析学
\renewcommand{\Re}{\mathrm{Re}\;}
\renewcommand{\Im}{\mathrm{Im}\;}
\newcommand{\Gal}{\mathrm{Gal}}
\newcommand{\PGL}{\mathrm{PGL}}
\newcommand{\PSL}{\mathrm{PSL}}
\newcommand{\Log}{\mathrm{Log}\,}
\newcommand{\Res}{\mathrm{Res}\,}
\newcommand{\on}{\mathrm{on}\;}
\newcommand{\hatC}{\hat{\C}}
\newcommand{\hatR}{\hat{\R}}
\newcommand{\PV}{\mathrm{P.V.}}
\newcommand{\diam}{\mathrm{diam}}
\newcommand{\Area}{\mathrm{Area}}
\newcommand{\Lap}{\Laplace}
\newcommand{\f}{\mathbf{f}}
\newcommand{\cR}{\mathcal{R}}
\newcommand{\const}{\mathrm{const.}}
\newcommand{\Om}{\Omega}
\newcommand{\Cinf}{C^\infty}
\newcommand{\ep}{\epsilon}
\newcommand{\dist}{\mathrm{dist}}
\newcommand{\opart}{\o{\partial}}
%%% 解析力学
\newcommand{\x}{\mathbf{x}}
%%% 集合と位相
\renewcommand{\O}{\mathcal{O}}
\renewcommand{\S}{\mathcal{S}}
\renewcommand{\U}{\mathcal{U}}
\newcommand{\V}{\mathcal{V}}
\renewcommand{\P}{\mathcal{P}}
\newcommand{\R}{\mathbb{R}}
\newcommand{\N}{\mathbb{N}}
\newcommand{\C}{\mathbb{C}}
\newcommand{\Z}{\mathbb{Z}}
\newcommand{\Q}{\mathbb{Q}}
\newcommand{\TV}{\mathrm{TV}}
\newcommand{\ORD}{\mathrm{ORD}}
\newcommand{\Tr}{\mathrm{Tr}\;}
\newcommand{\Card}{\mathrm{Card}\;}
\newcommand{\Top}{\mathrm{Top}}
\newcommand{\Disc}{\mathrm{Disc}}
\newcommand{\Codisc}{\mathrm{Codisc}}
\newcommand{\CoDisc}{\mathrm{CoDisc}}
\newcommand{\Ult}{\mathrm{Ult}}
\newcommand{\ord}{\mathrm{ord}}
\newcommand{\maj}{\mathrm{maj}}
%%% 形式言語理論
\newcommand{\REGEX}{\mathrm{REGEX}}
\newcommand{\RE}{\mathbf{RE}}

%%% Fourier解析
\newcommand*{\Laplace}{\mathop{}\!\mathbin\bigtriangleup}
\newcommand*{\DAlambert}{\mathop{}\!\mathbin\Box}
%%% Graph Theory
\newcommand{\SimpGph}{\mathrm{SimpGph}}
\newcommand{\Gph}{\mathrm{Gph}}
\newcommand{\mult}{\mathrm{mult}}
\newcommand{\inv}{\mathrm{inv}}
%%% 多様体
\newcommand{\Der}{\mathrm{Der}}
\newcommand{\osub}{\overset{\mathrm{open}}{\subset}}
\newcommand{\osup}{\overset{\mathrm{open}}{\supset}}
\newcommand{\al}{\alpha}
\newcommand{\K}{\mathbb{K}}
\newcommand{\Sp}{\mathrm{Sp}}
\newcommand{\g}{\mathfrak{g}}
\newcommand{\h}{\mathfrak{h}}
\newcommand{\Exp}{\mathrm{Exp}\;}
\newcommand{\Imm}{\mathrm{Imm}}
\newcommand{\Imb}{\mathrm{Imb}}
\newcommand{\codim}{\mathrm{codim}\;}
\newcommand{\Gr}{\mathrm{Gr}}
%%% 代数
\newcommand{\Ad}{\mathrm{Ad}}
\newcommand{\finsupp}{\mathrm{fin\;supp}}
\newcommand{\SO}{\mathrm{SO}}
\newcommand{\SU}{\mathrm{SU}}
\newcommand{\acts}{\curvearrowright}
\newcommand{\mono}{\hookrightarrow}
\newcommand{\epi}{\twoheadrightarrow}
\newcommand{\Stab}{\mathrm{Stab}}
\newcommand{\nor}{\mathrm{nor}}
\newcommand{\T}{\mathbb{T}}
\newcommand{\Aff}{\mathrm{Aff}}
\newcommand{\rsub}{\triangleleft}
\newcommand{\rsup}{\triangleright}
\newcommand{\subgrp}{\overset{\mathrm{subgrp}}{\subset}}
\newcommand{\Ext}{\mathrm{Ext}}
\newcommand{\sbs}{\subset}\newcommand{\sps}{\supset}
\newcommand{\In}{\mathrm{In}}
\newcommand{\Tor}{\mathrm{Tor}}
\newcommand{\p}{\mathfrak{p}}
\newcommand{\q}{\mathfrak{q}}
\newcommand{\m}{\mathfrak{m}}
\newcommand{\cS}{\mathcal{S}}
\newcommand{\Frac}{\mathrm{Frac}\,}
\newcommand{\Spec}{\mathrm{Spec}\,}
\newcommand{\bA}{\mathbb{A}}
\newcommand{\Sym}{\mathrm{Sym}}
\newcommand{\Ann}{\mathrm{Ann}}
%%% 代数的位相幾何学
\newcommand{\Ho}{\mathrm{Ho}}
\newcommand{\CW}{\mathrm{CW}}
\newcommand{\lc}{\mathrm{lc}}
\newcommand{\cg}{\mathrm{cg}}
\newcommand{\Fib}{\mathrm{Fib}}
\newcommand{\Cyl}{\mathrm{Cyl}}
\newcommand{\Ch}{\mathrm{Ch}}
%%% 数値解析
\newcommand{\round}{\mathrm{round}}
\newcommand{\cond}{\mathrm{cond}}
\newcommand{\diag}{\mathrm{diag}}
%%% 確率論
\newcommand{\calF}{\mathcal{F}}
\newcommand{\X}{\mathcal{X}}
\newcommand{\Meas}{\mathrm{Meas}}
\newcommand{\as}{\;\mathrm{a.s.}} %almost surely
\newcommand{\io}{\;\mathrm{i.o.}} %infinitely often
\newcommand{\fe}{\;\mathrm{f.e.}} %with a finite number of exceptions
\newcommand{\F}{\mathcal{F}}
\newcommand{\bF}{\mathbb{F}}
\newcommand{\W}{\mathcal{W}}
\newcommand{\Pois}{\mathrm{Pois}}
\newcommand{\iid}{\mathrm{i.i.d.}}
\newcommand{\wconv}{\rightsquigarrow}
\newcommand{\Var}{\mathrm{Var}}
\newcommand{\xrightarrown}{\xrightarrow{n\to\infty}}
\newcommand{\au}{\mathrm{au}}
\newcommand{\cT}{\mathcal{T}}
%%% 情報理論
\newcommand{\bit}{\mathrm{bit}}
%%% 積分論
\newcommand{\calA}{\mathcal{A}}
\newcommand{\calB}{\mathcal{B}}
\newcommand{\D}{\mathcal{D}}
\newcommand{\Y}{\mathcal{Y}}
\newcommand{\calC}{\mathcal{C}}
\renewcommand{\ae}{\mathrm{a.e.}\;}
\newcommand{\cZ}{\mathcal{Z}}
\newcommand{\fF}{\mathfrak{F}}
\newcommand{\fI}{\mathfrak{I}}
\newcommand{\E}{\mathcal{E}}
\newcommand{\sMap}{\sigma\textrm{-}\mathrm{Map}}
\DeclareMathOperator*{\argmax}{arg\,max}
\DeclareMathOperator*{\argmin}{arg\,min}
\newcommand{\cC}{\mathcal{C}}
\newcommand{\comp}{\complement}
\newcommand{\J}{\mathcal{J}}
\newcommand{\sumN}[1]{\sum_{#1\in\N}}
\newcommand{\cupN}[1]{\cup_{#1\in\N}}
\newcommand{\capN}[1]{\cap_{#1\in\N}}
\newcommand{\Sum}[1]{\sum_{#1=1}^\infty}
\newcommand{\sumn}{\sum_{n=1}^\infty}
\newcommand{\summ}{\sum_{m=1}^\infty}
\newcommand{\sumk}{\sum_{k=1}^\infty}
\newcommand{\sumi}{\sum_{i=1}^\infty}
\newcommand{\sumj}{\sum_{j=1}^\infty}
\newcommand{\cupn}{\cup_{n=1}^\infty}
\newcommand{\capn}{\cap_{n=1}^\infty}
\newcommand{\cupk}{\cup_{k=1}^\infty}
\newcommand{\cupi}{\cup_{i=1}^\infty}
\newcommand{\cupj}{\cup_{j=1}^\infty}
\newcommand{\limn}{\lim_{n\to\infty}}
\renewcommand{\l}{\mathcal{l}}
\renewcommand{\L}{\mathcal{L}}
\newcommand{\Cl}{\mathrm{Cl}}
\newcommand{\cN}{\mathcal{N}}
\newcommand{\Ae}{\textrm{-a.e.}\;}
\newcommand{\csub}{\overset{\textrm{closed}}{\subset}}
\newcommand{\csup}{\overset{\textrm{closed}}{\supset}}
\newcommand{\wB}{\wt{B}}
\newcommand{\cG}{\mathcal{G}}
\newcommand{\Lip}{\mathrm{Lip}}
\newcommand{\Dom}{\mathrm{Dom}}
%%% 数理ファイナンス
\newcommand{\pre}{\mathrm{pre}}
\newcommand{\om}{\omega}

%%% 統計的因果推論
\newcommand{\Do}{\mathrm{Do}}
%%% 数理統計
\newcommand{\bP}{\mathbb{P}}
\newcommand{\compsub}{\overset{\textrm{cpt}}{\subset}}
\newcommand{\lip}{\textrm{lip}}
\newcommand{\BL}{\mathrm{BL}}
\newcommand{\G}{\mathbb{G}}
\newcommand{\NB}{\mathrm{NB}}
\newcommand{\oR}{\o{\R}}
\newcommand{\liminfn}{\liminf_{n\to\infty}}
\newcommand{\limsupn}{\limsup_{n\to\infty}}
%\newcommand{\limn}{\lim_{n\to\infty}}
\newcommand{\esssup}{\mathrm{ess.sup}}
\newcommand{\asto}{\xrightarrow{\as}}
\newcommand{\Cov}{\mathrm{Cov}}
\newcommand{\cQ}{\mathcal{Q}}
\newcommand{\VC}{\mathrm{VC}}
\newcommand{\mb}{\mathrm{mb}}
\newcommand{\Avar}{\mathrm{Avar}}
\newcommand{\bB}{\mathbb{B}}
\newcommand{\bW}{\mathbb{W}}
\newcommand{\sd}{\mathrm{sd}}
\newcommand{\w}[1]{\widehat{#1}}
\newcommand{\bZ}{\mathbb{Z}}
\newcommand{\Bernoulli}{\mathrm{Bernoulli}}
\newcommand{\Mult}{\mathrm{Mult}}
\newcommand{\BPois}{\mathrm{BPois}}
\newcommand{\fraks}{\mathfrak{s}}
\newcommand{\frakk}{\mathfrak{k}}
\newcommand{\IF}{\mathrm{IF}}
\newcommand{\bX}{\mathbf{X}}
\newcommand{\bx}{\mathbf{x}}
\newcommand{\indep}{\raisebox{0.05em}{\rotatebox[origin=c]{90}{$\models$}}}
\newcommand{\IG}{\mathrm{IG}}
\newcommand{\Levy}{\mathrm{Levy}}
\newcommand{\MP}{\mathrm{MP}}
\newcommand{\Hermite}{\mathrm{Hermite}}
\newcommand{\Skellam}{\mathrm{Skellam}}
\newcommand{\Dirichlet}{\mathrm{Dirichlet}}
\newcommand{\Beta}{\mathrm{Beta}}
\newcommand{\bE}{\mathbb{E}}
\newcommand{\bG}{\mathbb{G}}
\newcommand{\MISE}{\mathrm{MISE}}
\newcommand{\logit}{\mathtt{logit}}
\newcommand{\expit}{\mathtt{expit}}
\newcommand{\cK}{\mathcal{K}}
\newcommand{\dl}{\dot{l}}
\newcommand{\dotp}{\dot{p}}
\newcommand{\wl}{\wt{l}}
%%% 函数解析
\renewcommand{\c}{\mathbf{c}}
\newcommand{\loc}{\mathrm{loc}}
\newcommand{\Lh}{\mathrm{L.h.}}
\newcommand{\Epi}{\mathrm{Epi}\;}
\newcommand{\slim}{\mathrm{slim}}
\newcommand{\Ban}{\mathrm{Ban}}
\newcommand{\Hilb}{\mathrm{Hilb}}
\newcommand{\Ex}{\mathrm{Ex}}
\newcommand{\Co}{\mathrm{Co}}
\newcommand{\sa}{\mathrm{sa}}
\newcommand{\nnorm}[1]{{\left\vert\kern-0.25ex\left\vert\kern-0.25ex\left\vert #1 \right\vert\kern-0.25ex\right\vert\kern-0.25ex\right\vert}}
\newcommand{\dvol}{\mathrm{dvol}}
\newcommand{\Sconv}{\mathrm{Sconv}}
\newcommand{\I}{\mathcal{I}}
\newcommand{\nonunital}{\mathrm{nu}}
\newcommand{\cpt}{\mathrm{cpt}}
\newcommand{\lcpt}{\mathrm{lcpt}}
\newcommand{\com}{\mathrm{com}}
\newcommand{\Haus}{\mathrm{Haus}}
\newcommand{\proper}{\mathrm{proper}}
\newcommand{\infinity}{\mathrm{inf}}
\newcommand{\TVS}{\mathrm{TVS}}
\newcommand{\ess}{\mathrm{ess}}
\newcommand{\ext}{\mathrm{ext}}
\newcommand{\Index}{\mathrm{Index}}
\newcommand{\SSR}{\mathrm{SSR}}
\newcommand{\vs}{\mathrm{vs.}}
\newcommand{\fM}{\mathfrak{M}}
\newcommand{\EDM}{\mathrm{EDM}}
\newcommand{\Tw}{\mathrm{Tw}}
\newcommand{\fC}{\mathfrak{C}}
\newcommand{\bn}{\mathbf{n}}
\newcommand{\br}{\mathbf{r}}
\newcommand{\Lam}{\Lambda}
\newcommand{\lam}{\lambda}
\newcommand{\one}{\mathbf{1}}
\newcommand{\dae}{\text{-a.e.}}
\newcommand{\td}{\text{-}}
\newcommand{\RM}{\mathrm{RM}}
%%% 最適化
\newcommand{\Minimize}{\text{Minimize}}
\newcommand{\subjectto}{\text{subject to}}
\newcommand{\Ri}{\mathrm{Ri}}
%\newcommand{\Cl}{\mathrm{Cl}}
\newcommand{\Cone}{\mathrm{Cone}}
\newcommand{\Int}{\mathrm{Int}}
%%% 圏
\newcommand{\varlim}{\varprojlim}
\newcommand{\Hom}{\mathrm{Hom}}
\newcommand{\Iso}{\mathrm{Iso}}
\newcommand{\Mor}{\mathrm{Mor}}
\newcommand{\Isom}{\mathrm{Isom}}
\newcommand{\Aut}{\mathrm{Aut}}
\newcommand{\End}{\mathrm{End}}
\newcommand{\op}{\mathrm{op}}
\newcommand{\ev}{\mathrm{ev}}
\newcommand{\Ob}{\mathrm{Ob}}
\newcommand{\Ar}{\mathrm{Ar}}
\newcommand{\Arr}{\mathrm{Arr}}
\newcommand{\Set}{\mathrm{Set}}
\newcommand{\Grp}{\mathrm{Grp}}
\newcommand{\Cat}{\mathrm{Cat}}
\newcommand{\Mon}{\mathrm{Mon}}
\newcommand{\CMon}{\mathrm{CMon}} %Comutative Monoid 可換単系とモノイドの射
\newcommand{\Ring}{\mathrm{Ring}}
\newcommand{\CRing}{\mathrm{CRing}}
\newcommand{\Ab}{\mathrm{Ab}}
\newcommand{\Pos}{\mathrm{Pos}}
\newcommand{\Vect}{\mathrm{Vect}}
\newcommand{\FinVect}{\mathrm{FinVect}}
\newcommand{\FinSet}{\mathrm{FinSet}}
\newcommand{\OmegaAlg}{\Omega$-$\mathrm{Alg}}
\newcommand{\OmegaEAlg}{(\Omega,E)$-$\mathrm{Alg}}
\newcommand{\Alg}{\mathrm{Alg}} %代数の圏
\newcommand{\CAlg}{\mathrm{CAlg}} %可換代数の圏
\newcommand{\CPO}{\mathrm{CPO}} %Complete Partial Order & continuous mappings
\newcommand{\Fun}{\mathrm{Fun}}
\newcommand{\Func}{\mathrm{Func}}
\newcommand{\Met}{\mathrm{Met}} %Metric space & Contraction maps
\newcommand{\Pfn}{\mathrm{Pfn}} %Sets & Partial function
\newcommand{\Rel}{\mathrm{Rel}} %Sets & relation
\newcommand{\Bool}{\mathrm{Bool}}
\newcommand{\CABool}{\mathrm{CABool}}
\newcommand{\CompBoolAlg}{\mathrm{CompBoolAlg}}
\newcommand{\BoolAlg}{\mathrm{BoolAlg}}
\newcommand{\BoolRng}{\mathrm{BoolRng}}
\newcommand{\HeytAlg}{\mathrm{HeytAlg}}
\newcommand{\CompHeytAlg}{\mathrm{CompHeytAlg}}
\newcommand{\Lat}{\mathrm{Lat}}
\newcommand{\CompLat}{\mathrm{CompLat}}
\newcommand{\SemiLat}{\mathrm{SemiLat}}
\newcommand{\Stone}{\mathrm{Stone}}
\newcommand{\Sob}{\mathrm{Sob}} %Sober space & continuous map
\newcommand{\Op}{\mathrm{Op}} %Category of open subsets
\newcommand{\Sh}{\mathrm{Sh}} %Category of sheave
\newcommand{\PSh}{\mathrm{PSh}} %Category of presheave, PSh(C)=[C^op,set]のこと
\newcommand{\Conv}{\mathrm{Conv}} %Convergence spaceの圏
\newcommand{\Unif}{\mathrm{Unif}} %一様空間と一様連続写像の圏
\newcommand{\Frm}{\mathrm{Frm}} %フレームとフレームの射
\newcommand{\Locale}{\mathrm{Locale}} %その反対圏
\newcommand{\Diff}{\mathrm{Diff}} %滑らかな多様体の圏
\newcommand{\Mfd}{\mathrm{Mfd}}
\newcommand{\LieAlg}{\mathrm{LieAlg}}
\newcommand{\Quiv}{\mathrm{Quiv}} %Quiverの圏
\newcommand{\B}{\mathcal{B}}
\newcommand{\Span}{\mathrm{Span}}
\newcommand{\Corr}{\mathrm{Corr}}
\newcommand{\Decat}{\mathrm{Decat}}
\newcommand{\Rep}{\mathrm{Rep}}
\newcommand{\Grpd}{\mathrm{Grpd}}
\newcommand{\sSet}{\mathrm{sSet}}
\newcommand{\Mod}{\mathrm{Mod}}
\newcommand{\SmoothMnf}{\mathrm{SmoothMnf}}
\newcommand{\coker}{\mathrm{coker}}

\newcommand{\Ord}{\mathrm{Ord}}
\newcommand{\eq}{\mathrm{eq}}
\newcommand{\coeq}{\mathrm{coeq}}
\newcommand{\act}{\mathrm{act}}

%%%%%%%%%%%%%%% 定理環境(足助先生ありがとうございます) %%%%%%%%%%%%%%%

\everymath{\displaystyle}
\renewcommand{\proofname}{\bf [証明]}
\renewcommand{\thefootnote}{\dag\arabic{footnote}} %足助さんからもらった.どうなるんだ?
\renewcommand{\qedsymbol}{$\blacksquare$}

\renewcommand{\labelenumi}{(\arabic{enumi})} %(1),(2),...がデフォルトであって欲しい
\renewcommand{\labelenumii}{(\alph{enumii})}
\renewcommand{\labelenumiii}{(\roman{enumiii})}

\newtheoremstyle{StatementsWithStar}% ?name?
{3pt}% ?Space above? 1
{3pt}% ?Space below? 1
{}% ?Body font?
{}% ?Indent amount? 2
{\bfseries}% ?Theorem head font?
{\textbf{.}}% ?Punctuation after theorem head?
{.5em}% ?Space after theorem head? 3
{\textbf{\textup{#1~\thetheorem{}}}{}\,$^{\ast}$\thmnote{(#3)}}% ?Theorem head spec (can be left empty, meaning ‘normal’)?
%
\newtheoremstyle{StatementsWithStar2}% ?name?
{3pt}% ?Space above? 1
{3pt}% ?Space below? 1
{}% ?Body font?
{}% ?Indent amount? 2
{\bfseries}% ?Theorem head font?
{\textbf{.}}% ?Punctuation after theorem head?
{.5em}% ?Space after theorem head? 3
{\textbf{\textup{#1~\thetheorem{}}}{}\,$^{\ast\ast}$\thmnote{(#3)}}% ?Theorem head spec (can be left empty, meaning ‘normal’)?
%
\newtheoremstyle{StatementsWithStar3}% ?name?
{3pt}% ?Space above? 1
{3pt}% ?Space below? 1
{}% ?Body font?
{}% ?Indent amount? 2
{\bfseries}% ?Theorem head font?
{\textbf{.}}% ?Punctuation after theorem head?
{.5em}% ?Space after theorem head? 3
{\textbf{\textup{#1~\thetheorem{}}}{}\,$^{\ast\ast\ast}$\thmnote{(#3)}}% ?Theorem head spec (can be left empty, meaning ‘normal’)?
%
\newtheoremstyle{StatementsWithCCirc}% ?name?
{6pt}% ?Space above? 1
{6pt}% ?Space below? 1
{}% ?Body font?
{}% ?Indent amount? 2
{\bfseries}% ?Theorem head font?
{\textbf{.}}% ?Punctuation after theorem head?
{.5em}% ?Space after theorem head? 3
{\textbf{\textup{#1~\thetheorem{}}}{}\,$^{\circledcirc}$\thmnote{(#3)}}% ?Theorem head spec (can be left empty, meaning ‘normal’)?
%
\theoremstyle{definition}
 \newtheorem{theorem}{定理}[section]
 \newtheorem{axiom}[theorem]{公理}
 \newtheorem{corollary}[theorem]{系}
 \newtheorem{proposition}[theorem]{命題}
 \newtheorem*{proposition*}{命題}
 \newtheorem{lemma}[theorem]{補題}
 \newtheorem*{lemma*}{補題}
 \newtheorem*{theorem*}{定理}
 \newtheorem{definition}[theorem]{定義}
 \newtheorem{example}[theorem]{例}
 \newtheorem{notation}[theorem]{記法}
 \newtheorem*{notation*}{記法}
 \newtheorem{assumption}[theorem]{仮定}
 \newtheorem{question}[theorem]{問}
 \newtheorem{counterexample}[theorem]{反例}
 \newtheorem{reidai}[theorem]{例題}
 \newtheorem{ruidai}[theorem]{類題}
 \newtheorem{problem}[theorem]{問題}
 \newtheorem{algorithm}[theorem]{算譜}
 \newtheorem*{solution*}{\bf{[解]}}
 \newtheorem{discussion}[theorem]{議論}
 \newtheorem{remark}[theorem]{注}
 \newtheorem{remarks}[theorem]{要諦}
 \newtheorem{image}[theorem]{描像}
 \newtheorem{observation}[theorem]{観察}
 \newtheorem{universality}[theorem]{普遍性} %非自明な例外がない.
 \newtheorem{universal tendency}[theorem]{普遍傾向} %例外が有意に少ない.
 \newtheorem{hypothesis}[theorem]{仮説} %実験で説明されていない理論.
 \newtheorem{theory}[theorem]{理論} %実験事実とその(さしあたり)整合的な説明.
 \newtheorem{fact}[theorem]{実験事実}
 \newtheorem{model}[theorem]{模型}
 \newtheorem{explanation}[theorem]{説明} %理論による実験事実の説明
 \newtheorem{anomaly}[theorem]{理論の限界}
 \newtheorem{application}[theorem]{応用例}
 \newtheorem{method}[theorem]{手法} %実験手法など,技術的問題.
 \newtheorem{history}[theorem]{歴史}
 \newtheorem{usage}[theorem]{用語法}
 \newtheorem{research}[theorem]{研究}
 \newtheorem{shishin}[theorem]{指針}
 \newtheorem{yodan}[theorem]{余談}
 \newtheorem{construction}[theorem]{構成}
% \newtheorem*{remarknonum}{注}
 \newtheorem*{definition*}{定義}
 \newtheorem*{remark*}{注}
 \newtheorem*{question*}{問}
 \newtheorem*{problem*}{問題}
 \newtheorem*{axiom*}{公理}
 \newtheorem*{example*}{例}
 \newtheorem*{corollary*}{系}
 \newtheorem*{shishin*}{指針}
 \newtheorem*{yodan*}{余談}
 \newtheorem*{kadai*}{課題}
%
\theoremstyle{StatementsWithStar}
 \newtheorem{definition_*}[theorem]{定義}
 \newtheorem{question_*}[theorem]{問}
 \newtheorem{example_*}[theorem]{例}
 \newtheorem{theorem_*}[theorem]{定理}
 \newtheorem{remark_*}[theorem]{注}
%
\theoremstyle{StatementsWithStar2}
 \newtheorem{definition_**}[theorem]{定義}
 \newtheorem{theorem_**}[theorem]{定理}
 \newtheorem{question_**}[theorem]{問}
 \newtheorem{remark_**}[theorem]{注}
%
\theoremstyle{StatementsWithStar3}
 \newtheorem{remark_***}[theorem]{注}
 \newtheorem{question_***}[theorem]{問}
%
\theoremstyle{StatementsWithCCirc}
 \newtheorem{definition_O}[theorem]{定義}
 \newtheorem{question_O}[theorem]{問}
 \newtheorem{example_O}[theorem]{例}
 \newtheorem{remark_O}[theorem]{注}
%
\theoremstyle{definition}
%
\raggedbottom
\allowdisplaybreaks
\usepackage[math]{anttor}
\begin{document}
\tableofcontents

\chapter{正則関数}

\section{正則関数の定義と幾何学的性質}

\begin{tcolorbox}[colframe=ForestGreen, colback=ForestGreen!10!white,breakable,colbacktitle=ForestGreen!40!white,coltitle=black,fonttitle=\bfseries\sffamily,
title=]
    のちに等角写像として特徴付けるが,それ以前にそのことを匂わせる幾何学的性質を豊かに持つ.
\end{tcolorbox}

\begin{definition}[normal / analytic function]
    関数$f:\C\osup D\to\C$が各点において微分係数を持つとき,\textbf{正則}または\textbf{解析的}という.
\end{definition}

\begin{lemma}[正則関数が定める実関数]
    $f:D\to\C$を正則関数とする.$f=u+iv$と表せるとき,$u,v:D\to\R$を\textbf{実部}と\textbf{虚部}という.これらは,
    \begin{enumerate}
        \item Cauchy-Riemann方程式を満たす:$u_x=v_y,u_y=-v_x$.
        \item 調和関数である:$\Laplace u=0,\Laplace v=0$.
    \end{enumerate}
    一般に,上の2条件を満たす実関数の組$(u,v)$を\textbf{共役調和関数}という.次が成り立つ:
    \begin{enumerate}\setcounter{enumi}{2}
        \item $u,v$を調和関数とする.$u+iv$は正則関数である.
    \end{enumerate}
\end{lemma}

\begin{corollary}[正則関数の微分係数の各種表現とJacobianの特徴付け]
    $f$が$z_0\in D$で正則とする.
    \begin{enumerate}
        \item $f'(z_0)=\pp{f}{z}(z_0)=2\pp{u}{z}(z_0)$.
        \item $F(x,y):=f(z)$とすると,$F:\R^2\to\C$も微分可能であり,$\det J_F(x_0,y_0)=\abs{f'(z_0)}^2$.
    \end{enumerate}
\end{corollary}
\begin{remarks}[正則関数の局所可逆性について]
    ここから,$f'(z)\ne0$ならば$z$の近傍で$f$位相同型を定めることが,陰関数定理から従う.
    が,より複素解析的な証明の方が簡潔で本質的である.
    またJacobianが$\abs{f'(z)}^2$であることについては,線積分について無限小線分の長さが$\abs{f'(z)}$倍されることと等角写像であることとの単純な帰結ともみれる!
    Jacobi行列の非対角成分がないのである.
\end{remarks}

\begin{tbox}{red}{正則関数の大域的可逆性について}
    陰関数定理の議論の時に夢想したことがあるであろうことに,各点でJacobianが$0$でないならば,その局所的な可逆写像を繋ぎ合わせて大域的な逆射を構成できないか?ということである.
    正則関数において,この探求の行手を阻むものはただ一つで,像が重なってしまうことである.
    そこで,定義域と値域を,重なったフィルムを引き剥がすように拡張すれば,一価な単射が定まるかもしれない.
\end{tbox}

\section{正則性の特徴付け}

\subsection{Cauchy-Riemann作用素による特徴付け}

\begin{definition}[Wirtinger derivative / Cauchy-Riemann operator]
    次によって定まる作用素$\O(D)\to\O(D)$
    \begin{align*}
        \partial_zf&=\frac{\partial}{\partial z}=\frac{1}{2}\left(\frac{\partial}{\partial x}\textcolor{red}{-}i\frac{\partial}{\partial y}\right)\\
        \overline{\partial}=\partial_{\overline{z}}f&=\frac{\partial}{\partial\overline{z}}=\frac{1}{2}\left(\frac{\partial}{\partial x}\textcolor{red}{+}i\frac{\partial}{\partial y}\right)
    \end{align*}
    を\textbf{コーシー・リーマン作用素}という.
\end{definition}

\begin{definition}[totally differentiable]
    関数$f:D\to\C$が\textbf{全微分可能}とは,実線型関数$L:D\to \C;x+yi\mapsto\al x+\beta y\;(\al,\beta\in\C)$が存在して,$f(w+z)=f(w)+L(z)+o(\abs{z})$が成り立つことをいう.
\end{definition}

\begin{lemma}[Wirtinger微分のwell-defined性]
    $f:D\to\C$が全微分可能であるとする.
    \begin{enumerate}
        \item $L(z)=f_x(w)x+f_y(w)y$が成り立つ.
        \item $f_z:=(f_x-if_y)/2,f_{\o{z}}:=(f_x+if_y)$と置くと,$L(z)=f_z(w)z+f_{\o{z}}(w)\o{z}$が成り立つ.
    \end{enumerate}
\end{lemma}

\begin{proposition}[微分作用素による特徴付け]
    $f:D\to\C$について,次は同値.
    \begin{enumerate}
        \item $f$は$a\in D$で正則.
        \item $f$は$a\in D$で全微分可能かつ$f_{\o{z}}(a)=0$.
    \end{enumerate}
\end{proposition}

\subsection{等角写像としての特徴付け}

\begin{definition}[conformal mapping]
    $C^1$-級写像$f:D\to\C$が$p\in D$で\textbf{等角}であるとは,
    $f$が引き起こす$C^1$-道の対応$\gamma\mapsto f\circ\gamma$が,
    $p$での接空間の内積を保つことをいう.
\end{definition}

\begin{proposition}[等角性による特徴付け]
    $f:D\to\C$が$C^1$-級であるとする.次は同値.
    \begin{enumerate}
        \item $p\in D$で等角である.
        \item $\partial_zf(p)\ne0$かつ$\partial_{\o{z}}f(p)=0$.
    \end{enumerate}
\end{proposition}

\section{整級数}

\begin{tcolorbox}[colframe=ForestGreen, colback=ForestGreen!10!white,breakable,colbacktitle=ForestGreen!40!white,coltitle=black,fonttitle=\bfseries\sffamily,
title=]
    正則関数を構成する強力な手法となる.特に,理論展開の初期段階で指数関数を定義するのに用いられる.
    この手法の\textbf{普遍性},すなわち任意の正則関数が整級数展開によって得られることはのちの理論で得られる.
\end{tcolorbox}

\begin{definition}
    級数のうち,$\{a_n\}\subset\C$を用いて
    $\sum_{n=0}^\infty a_nz^n$と表せるものを\textbf{整級数}という.
\end{definition}

\subsection{Abelの定理}

\begin{theorem}[Abel]
    任意の整級数に対して,\textbf{収束半径}$R\in[0,\infty]$が定まる:
    \begin{enumerate}
        \item $\Delta(0,R)$上にて,整級数は絶対収束する.
        \item $\forall_{r\in[0,R)}$について$[\Delta(0,r)]$上にて,整級数は一様収束する.
        \item $\Delta(\infty,R)$上にて,級数は発散する.
        \item $\Delta(0,R)$上にて整級数は微分可能で,その導関数は項別微分によって得られ,同じ収束半径を持つ.特に,整関数は$\Delta(0,R)$上で正則関数を定める.
        \item 収束半径$R$は$1/R:=\limsup_{n\to\infty}\sqrt[n]{\abs{a_n}}$で与えられる.
    \end{enumerate}
\end{theorem}
\begin{proof}
    $R$が求まればこれは一意であることは明らか.いま,$1/R:=\limsup_{n\to\infty}\sqrt[n]{\abs{a_n}}$として,(1)から(4)の性質を示す.
    \begin{enumerate}
        \item 任意に$z\in\Delta(0,R)$を取ると,$\abs{z}<\rho<R$を満たす$\rho$が取れる:$1/\rho>1/R$.
        このとき,limsupの定義から,$\exists_{n_0\in\N}\;\forall_{n\ge n_0}\;\abs{a_n}^{1/n}<1/\rho\Leftrightarrow\abs{a_n}<1/\rho^n$.
        特に$\abs{a_nz^n}<\paren{\frac{\abs{z}}{\rho}}^n$であるから,三角不等式より,整級数はこの$z$において絶対収束する.
        \item また特に,任意の$\rho<\rho'<R$に対して,$\abs{a_nz^n}\le(\rho/\rho')^n$によってえ収束する正項優級数が構成できるから,Weierstrassの$M$-判定法より,一様収束である.
        \item $\abs{z}>R$を満たす$z\in\C$に対しては,$R<\rho<\abs{z}$を満たす$\rho$が取れる:$1/\rho<1/R$.このとき$\abs{a_n}>1/\rho^n$を満たす$n$が無限に存在するから,$\abs{a_nz^n}>\paren{\frac{\abs{z}}{\rho}}^n\;\io$
        特に整級数は発散する.
        \item 項別微分が定める級数$\sum^{\infty}_{n=1}na_nz^{n-1}$は同じ収束半径を持つことは,$\lim_{n\to\infty}\sqrt[n]{n}=1$による.
        あとは,$\Delta(0,R)$上で定める一様収束極限を$f_1$とし,$f'(z)=f_1(z)$を示せば良い.
    \end{enumerate}
\end{proof}

\subsection{収束円周上での挙動}

\begin{tcolorbox}[colframe=ForestGreen, colback=ForestGreen!10!white,breakable,colbacktitle=ForestGreen!40!white,coltitle=black,fonttitle=\bfseries\sffamily,
title=]
    $S$は$(-\infty,1)$に関して対称な角領域で,1を頂点とし,角が$\pi$より小さいものとなる.
    特に,$S$内の$1$を通る任意の曲線は,単位円周$\partial\Delta$に接しない.
    $\exists_{M\in\R}\;\abs{1-z}\le M(1-\abs{z})$とは,点$1$よりも円周$\partial\Delta$の方に一定以上の比率で近づかないことをいう.
    これをStolzの角ともいう.
\end{tcolorbox}

\begin{theorem}[Abel 2]
    収束列$(a_n)\in c(\N;\C)$が定める級数$f(z):=\sum_{n\in\N}a_nz^n$について,
    $\frac{\abs{1-z}}{1-\abs{z}}$が有界になるような経路$z\in\Im\gamma\subset\Delta$で
    近づければ,$\lim_{S\ni z\to1}f(z)=f(1)$が成り立つ.
\end{theorem}

\section{関数列の一様収束}

\begin{tcolorbox}[colframe=ForestGreen, colback=ForestGreen!10!white,breakable,colbacktitle=ForestGreen!40!white,coltitle=black,fonttitle=\bfseries\sffamily,
title=]
    復習する.
\end{tcolorbox}

\begin{definition}
    $E\subset\C$上の
    関数列$(f_n)$が\textbf{一様収束}するとは,$\forall_{\ep>0}\;\exists_{n_0\in\N}\;\forall_{n\ge n_0}\;\forall_{x\in E}\;\abs{f_n(x)-f(x)}<\ep$を満たすことをいう.
\end{definition}

\subsection{一様収束の性質}

\begin{theorem}[一様収束は連続性を保つ]
    $(f_n)$を$E\subset\C$上の連続関数列とし,極限$f$に一様収束するとする.
    このとき,$f$は連続である.
\end{theorem}
\begin{proof}
    任意の$x_0\in E$と$\ep>0$をとる.
    \begin{enumerate}
        \item $f$は$(f_n)$の一様収束極限だから,$\exists_{n\in\N}\;\forall_{x\in E}\;\abs{f_n(x)-f(x)}<\ep/3$.
        \item $f_n$は連続だから,$\exists_{\delta>0}\;\forall_{x\in E}\;\abs{x-x_0}<\delta\Rightarrow\abs{f_n(x_0)-f_n(x)}<\ep/3$.
    \end{enumerate}
    以上より,任意の$\abs{x-x_0}<\delta$を満たす$x\in E$に対して,
    \[\abs{f(x)-f(x_0)}\le\abs{f(x)-f_n(x)}+\abs{f_n(x)-f_n(x_0)}+\abs{f_n(x_0)-f(x_0)}<\ep.\]
\end{proof}

\begin{theorem}
    $E\subset S$を距離空間$S$の部分集合とし,$x\in S$をその集積点とする.
    $(f_n)$が$f$に一様収束するとき,$(\lim_{t\to x}f_n(t))_{n\in\N}$は収束し,
    \[\lim_{n\to\infty}\lim_{t\to x}f_n(t)=\lim_{t\to x}\lim_{n\to\infty}f_n(t)\]
\end{theorem}

\subsection{コンパクト一様収束}

\begin{tcolorbox}[colframe=ForestGreen, colback=ForestGreen!10!white,breakable,colbacktitle=ForestGreen!40!white,coltitle=black,fonttitle=\bfseries\sffamily,
title=]
    一方で,連続関数の列が連続関数に収束するとき,そのモードが一様収束であるとは限らない.
\end{tcolorbox}

\begin{theorem}
    $(f_n)$をコンパクト集合$K$上の連続関数の列とする.このとき,
    \begin{enumerate}
        \item $(f_n)$はある連続関数$f$に各点収束する.
        \item $(f_n)$は広義単調減少列である.
    \end{enumerate}
    ならば,$(f_n)$は$f$に一様収束する.
\end{theorem}

\subsection{一様収束と導関数}

\begin{theorem}
    $[a,b]$上の可微分関数の列$(f_n)$は,ある$x_0\in[a,b]$において各点収束するとする.
    導関数が定める列$(f'_n)$が一様収束するならば,元の列$(f_n)$も一様収束し,極限と微分が可換になる:$\forall_{x\in[a,b]}\;f'(x)=\lim_{n\to\infty}f'(x)$.
\end{theorem}

\subsection{一様収束の判定法}

\begin{proposition}[一様収束の判定法]
    $(f_n)$を$E$上の関数の列で,各点収束極限
    $f$を持つとする.
    \begin{enumerate}
        \item $(f_n)$は一様収束する.
        \item (Cauchy criterion) $\forall_{\ep>0}\;\exists_{n_0\in\N}\;\forall_{m,n\ge n_0}\;\forall_{x\in E}\;\abs{f_n(x)-f_m(x)}<\ep$.
        \item $\norm{f_n-f}_\infty\to0$.
    \end{enumerate}
\end{proposition}

\begin{proposition}[Weierstrass $M$-test]
    関数列$(f_n)$は収束する優級数$\{M_n\}\subset\C$を持つとする:$\forall_{n\in\N}\;\norm{f_n}_\infty\le\abs{M_n},\sum_{n\in\N}M_n\in\C$.このとき,級数列$\paren{i=1}^nf_i$は一様収束する.
\end{proposition}

\section{指数関数}

\begin{tcolorbox}[colframe=ForestGreen, colback=ForestGreen!10!white,breakable,colbacktitle=ForestGreen!40!white,coltitle=black,fonttitle=\bfseries\sffamily,
title=]
    多項式,有理関数の次に,
    絶対に外せない正則関数が指数関数である.
    これを早速整級数を用いて定義する.
\end{tcolorbox}

\subsection{定義と性質}

\begin{definition}
    \[e^z:=1+\frac{z}{1!}+\frac{z^2}{2!}+\cdots+\frac{z^n}{n!}+\cdots\]
    によって定まる整関数$\exp:\C\to\C$を\textbf{指数関数}という.
\end{definition}
\begin{proof}
    この整級数が$\C$上で収束することを示すには,$\sqrt[n]{n!}\to\infty$を示せば良い.
\end{proof}

\begin{lemma}
    $e$は
    \begin{enumerate}
        \item 微分方程式$f'(z)=f(z),f(0)=1$を満たすただ一つの解である.
        \item 指数法則を満たす.
    \end{enumerate}
\end{lemma}

\subsection{周期性}

\begin{tcolorbox}[colframe=ForestGreen, colback=ForestGreen!10!white,breakable,colbacktitle=ForestGreen!40!white,coltitle=black,fonttitle=\bfseries\sffamily,
title=]
    写像$f(y)=e^{iy}$は,実数の加法群から単位円周$S^1:=\partial\Delta$の乗法群へのLie群としての準同型を定めており,この核は$2\pi\Z$となる.
    $\exp$が定める同型$\o{\exp}:\Z/2\pi\Z\iso S^1$がトーラスとの同型を導く.
    よって,$\exp$の逆関数というものは本来考えられず,あるとするならば無限個の値を持つ多価関数で,それぞれ$2\pi i$の整数倍の差を持つ.
\end{tcolorbox}

\begin{theorem}[指数関数の周期性]\label{thm-period-of-exponential}
    指数関数expは周期である.すなわち,
    ある正実数$\pi\in\R_{>0}$が存在して,
    \[ e^w=1\Leftrightarrow w=2\pi in\;\;(n\in\Z) \]
    を満たす.
\end{theorem}
\begin{proof}\mbox{}
    $w=x+yi$と置くと,$e^{x+yi}=e^xe^{yi}=e^x(\cos y+i\sin y)=1$であるが,$|\cos y+i\sin y|=\cos^2y+\sin^2y=1$
    より,$|e^x|=1$.$\exp:\R\to\R_{>0}$は全単射なので,$x=0$と分かる.

    続いて,$y$の一番小さい解は$y=2\pi$であること,そしてその他の元は全てこれの整数倍で尽くされることを示す.
\end{proof}

\subsection{対数関数}

\begin{tcolorbox}[colframe=ForestGreen, colback=ForestGreen!10!white,breakable,colbacktitle=ForestGreen!40!white,coltitle=black,fonttitle=\bfseries\sffamily,
title=]
    $\log:\C^\times\to?$の像は代数的に考えれば$2\pi\Z$のようなものになるべきである.
    あるいは,実用上は局所切断(分枝)を取りたいものである.
\end{tcolorbox}

\begin{definition}
    $f:D\to\C$が\textbf{対数関数}であるとは,$\exp:\C\to\C^*$の$D$上での切断であることをいう:$\forall_{z\in D}\;\exp(\Log(z))=z$.
    このとき$0\notin D$が必要.
\end{definition}
\begin{remarks}
    こうして,対数関数をまず,指数関数の局所切断の全体と捉えておく.
    これは真の対数関数の制限ともみなせる.
    Ahlforsでは真の対数関数をまずは(定義域をRiemann面に拡張することなく)集合値関数と捉えており,制限として得られる一価関数を\textbf{分枝}と呼んでいる.
\end{remarks}

\begin{theorem}
    $\Log:R:=\C\setminus(-\infty,0]\to\C$を$\Log(z):=\log\abs{z}+i\arg z$と定めると,これは$\exp$の$R$上での切断である:$\forall_{z\in R}\;\exp(\Log(z))=z$.
    また,$R$は$\Log$の定義域として極大である.
\end{theorem}

\section{一次変換}

\begin{tcolorbox}[colframe=ForestGreen, colback=ForestGreen!10!white,breakable,colbacktitle=ForestGreen!40!white,coltitle=black,fonttitle=\bfseries\sffamily,
title=]
    多項式と有理関数とは解析関数の重要な例であるが,これらのより自然な見方を探求してみる.
    すると正則関数の例の一つに収まるには極めて豊かな性質が見えてくる.
\end{tcolorbox}

\subsection{一次変換群}

\begin{tcolorbox}[colframe=ForestGreen, colback=ForestGreen!10!white,breakable,colbacktitle=ForestGreen!40!white,coltitle=black,fonttitle=\bfseries\sffamily,
title=]
    1次変換群は$\Aut(\hatC)=\GL_2(\C)/\C I=\PGL_2(\C)$の構造を持つ複素3次元のLie群で,これは射影直線$P^1(\C)$またはRiemann球面の位相変換群である.
    任意のRiemann面の基本群はMobius群の離散部分群となり,特に重要な離散部分群としてmodular群を持つ.
\end{tcolorbox}

\begin{definition}
    $\begin{pmatrix}
        a&b\\c&d
    \end{pmatrix}$を用いて$S(z):=\frac{az+b}{cz+d}$で定まる変換を\textbf{1次変換}という.
\end{definition}
\begin{discussion}
    一次変換の扱いにくさの本質は非斉次性である.
    そこで,$z=z_1/z_2$と見ると斉次化出来て,
    \[w:=\frac{w_1}{w_2}=\frac{az_1+bz_2}{cz_1+dz_2}\]
    という,「比の間の対応」とみなせる.
    このことから,任意の行列のスカラー倍は同じ変換を定めることが分かる.
    そこで,商群$\GL_2(\C)/\C^\times\simeq\SL_2(\C)/\{\pm1\}$を$\PSL(\C)$と表し,これが一次変換群に同型である.
    なお,行列の$\al$倍は行列式の$\al^2$倍に等しく,行列式を1に規格化する操作は,2つのスカラー倍$\pm\al$によって達成されるので,さらに$\pm1$で割る必要がある.

    点$z$に対して一見自由度を増やして2変数を用いて$z=z_1/z_2$と見てから割り,「$\C$の非零元の間の比全体の集合」を考えることは,
    斉次座標によって多様体の構造を持つ射影直線$P^1(\C)$上の点を考えることに等しい.そこでは,$\infty=[1:0]$と対応するから,無限が出現しないという美点もある.
    構成からこれはRiemann球面$\hatC$に同相である.
\end{discussion}

\begin{proposition}[標準分解]
    任意の$S\in\Aut(\hatC)$は,
    \begin{enumerate}
        \item 平行移動$\begin{pmatrix}
            1&\al\\0&1
        \end{pmatrix}$
        \item 回転・拡大$\begin{pmatrix}\al&0\\0&1\end{pmatrix}$
        \item 反転$\begin{pmatrix}0&1\\1&0\end{pmatrix}$
    \end{enumerate}
    の積に分解できる.
\end{proposition}

\subsection{非調和比}

\begin{theorem}[一次変換は鋭推移的である]
    $\hat{\C}$の相異なる3点$z_2,z_3,z_4$に対して,この順に$1,0,\infty$に移す一次変換$S_{z_2,z_3,z_4}$がただ一つ存在する.
    すなわち,対応$S:[\hatC]^3\to\Aut(\hatC)$は単射である.
\end{theorem}
\begin{proof}\mbox{}
    \begin{description}
        \item[構成] \[S(z)=\begin{cases}
            \paren{\frac{z_2-z_4}{z_2-z_3}}\paren{\frac{z-z_3}{z-z_4}}&z_2,z_3,z_4\in\C のとき,\\\vspace{1mm}
            \frac{z-z_3}{z-z_4}&z_2=\infty のとき,\\\vspace{1mm}
            \frac{z_2-z_4}{z-z_4}&z_3=\infty のとき,\\\vspace{1mm}
            \frac{z-z_3}{z_2-z_3}&z_4=\infty のとき,
        \end{cases}\]
        と置けば良い.
        \item[一意性] $1,0,\infty$を動かさない1次変換は恒等写像だけであることを示せば良い.$S=\frac{az+b}{cz+d}$と置く.
        \begin{align*}
            \frac{a+b}{c+d}&=1.&\frac{b}{d}&=0,&\frac{a}{c}&=\infty,
        \end{align*}
        より,$b=c=0,a=d$.$ad-bc=1$より,$a=d=1$.よって,$S=\id_{\hatC}$.
    \end{description}
\end{proof}

\begin{definition}[cross ratio / anharmonic ratio]
    相異なる3点$z_2,z_3,z_4\in\hatC$に対して,$(z_1,z_2,z_3,z_4):=S_{z_2,z_3,z_4}(z_1)$を\textbf{非調和比}という.
\end{definition}

\begin{theorem}
    相異なる4点$z_1,z_2,z_3,z_4\in\hatC$と任意の一次変換$T\in\Aut(\hatC)$に対して,$(Tz_1,Tz_2,Tz_3,Tz_4)=(z_1,z_2,z_3,z_4)$
\end{theorem}
\begin{proof}
    対応$S:[\hatC]^3\to\Aut(\hatC)$の単射性より,$S_{z_2,z_3,z_4}\circ T^{-1}=S_{Tz_2,Tz_3,Tz_4}$.
    定理の主張は,$S_{z_2,z_3,z_4}(z_1)=S_{z_2,z_3,z_4}T^{-1}(Tz_1)=S_{Tz_2,Tz_3,Tz_4}(Tz_1)$より従う.
\end{proof}

\subsection{円円対応}

\begin{notation}
    $\infty$を通る直線を$\C$上の円と呼ぶ.
\end{notation}

\begin{theorem}
    相異なる4点$z_1,z_2,z_3,z_4\in\hatC$に対して,次の2条件は同値.
    \begin{enumerate}
        \item $(z_1,z_2,z_3,z_4)\in\R$.
        \item 4点$z_1,z_2,z_3,z_4$は$\C$上で同一の直線または円の上にある.
    \end{enumerate}
\end{theorem}

\begin{corollary}
    一次変換は円を円に写す.
\end{corollary}

\subsection{対称性:一次変換による幾何}

\begin{tcolorbox}[colframe=ForestGreen, colback=ForestGreen!10!white,breakable,colbacktitle=ForestGreen!40!white,coltitle=black,fonttitle=\bfseries\sffamily,
title=]
    $\C$上の直線を用いた幾何,特に対称性の概念を,円に対して一般化することに応用できる.
    この先の消息に,Schwartzの鏡映の原理などがある.
\end{tcolorbox}

\begin{definition}
    $z,z^*\in\C$が円$C\subset\C$に関して\textbf{対称}であるとは,$\forall_{z_1,z_2,z_3\in[C]^3}\;(z^*,z_1,z_2,z_3)=\o{(z,z_1,z_2,z_3)}$を満たすことをいう.
\end{definition}
\begin{remarks}
    円に関する対称点は,単位円に対する反転変換の一般化となっている.これを\textbf{鏡映変換}という.
\end{remarks}

\begin{theorem}[対称の原理]
    一次変換$S\in\Aut(\hatC)$は円$C_1$を円$C_2$に写すとする.このとき,円$C_1$に対称な2点は円$C_2$に関して対称に写される.
\end{theorem}
\begin{proof}
    一度実軸への対称変換を経由して考えれば良い.
\end{proof}

\subsection{Steinerの円}

\begin{tcolorbox}[colframe=ForestGreen, colback=ForestGreen!10!white,breakable,colbacktitle=ForestGreen!40!white,coltitle=black,fonttitle=\bfseries\sffamily,
title=]
    $z$の平面と$w$の平面との座標直線の対応をみるのは等角写像に対する標準的な分析である.
    一次変換についてはSteinerの円が知られている.
\end{tcolorbox}

\section{等角写像の例}

\begin{tcolorbox}[colframe=ForestGreen, colback=ForestGreen!10!white,breakable,colbacktitle=ForestGreen!40!white,coltitle=black,fonttitle=\bfseries\sffamily,
title=]
    一次変換は,正則関数の代数的な考察と幾何的な考察(例えば円の族を向きを含めて円の族に写す)とが交差する良い例であった.
    同様のことを,初等関数で考える.
    これは実関数のグラフを見ることに相等する.

    また,この方法によって正則関数が特徴付けられるということが,Riemannの写像定理である.
    その手法はRiemann面を導入することで,非単射な正則関数についても実行可能になる.
\end{tcolorbox}

\subsection{初等関数}

\begin{definition}[ramification point, branch point, fundamental region]\mbox{}
    \begin{enumerate}
        \item $f:O\to\C$について,点$z_0\in O$の周りを一周しても元の値に戻らないとき,$z_0$を\textbf{分岐点},$f(z_0)$を\textbf{分岐値}という.
        \item $\C$からたかだか有限個の曲線を除いた領域へ単射に写される定義域の部分集合を\textbf{基本領域}という.
    \end{enumerate}
\end{definition}

\begin{example}[多項式]
    $w=z^n\;(n\in\N)$は
    角領域を$\C\setminus(0,\infty)$に写す.
    \begin{enumerate}
        \item $n$個の角領域$(k-1)\frac{2\pi}{n}<\arg z<k\frac{2\pi}{n}$の各々と,$\C\setminus(0,\infty)$が微分同相になる.
        \item なお,このとき$0,\infty$を通る任意の単純閉曲線に沿った切り口で$\C$を貼り合わせればよく,その選択に関してRiemann面はwell-definedである.一方で点$w=0$は$n$枚のシートとつながっている.これを\textbf{$n$位の分岐点}という.
    \end{enumerate}
\end{example}

\begin{example}[指数関数]
    $w=e^z$は帯領域を$\C\setminus[0,\infty)$に写す.
    \begin{enumerate}
        \item 帯領域$(k-1)2\pi<y<k2\pi$の各々を$\C\setminus[0,\infty)$に写す.
    \end{enumerate}
\end{example}

\chapter{複素積分}

\begin{quotation}
    多くの正則関数の性質は,積分の言葉を用いれば証明できる.
    その理由は,任意の正則関数がCauchyの積分表示を持つため,積分論を通じて正則関数を調べることが出来る.

    積分の中でも特に,微分係数の定義に用いられる形
    \[\lim_{\zeta\to z}\frac{f(\zeta)-f(z)}{\zeta-z}\]
    に注目する.
\end{quotation}

\section{積分の定義}

\begin{tcolorbox}[colframe=ForestGreen, colback=ForestGreen!10!white,breakable,colbacktitle=ForestGreen!40!white,coltitle=black,fonttitle=\bfseries\sffamily,
title=]
    Cauchyの積分定理は,微分形式の積分を$df=\partial_zfdz+\partial_{\o{z}}fd\o{z}$に関して自然に定義したとき,Greenの定理の直接的な帰結である(微分形式は最低でも$C^1$級であった).
    しかしこのとき,曲線の$C^1$-級という,ベクトル解析的な消息が入ってしまう.
    より複素解析的な議論は,複素積分をRiemann和の議論まで戻って議論するところから始まる.
\end{tcolorbox}

\subsection{積分の定義}

\begin{notation}[curve, path]
    \textbf{曲線}を連続関数$[a,b]\to X$の意味で使い,そのうち特に\textbf{道}を区分的に$C^1$級な曲線として使う.
\end{notation}

\begin{discussion}[複素積分の定義: contour integral]
    $f$を連続関数とする.
    \begin{enumerate}
        \item 道$\gamma$に沿った$f:\Im\gamma\to\C$の\textbf{複素線積分}を,
        \[\int_\gamma f(z)dz=\int^b_af(z(t))z'(t)dt.\]
        で定める.これは$\C$-線形写像となり,三角不等式(ある種のノルム減少性)を満たす.
        \item 一般の曲線$\gamma$に沿った複素線積分を,
        \[\int_\gamma fdz:=\lim_{\abs{\Delta}\to0}\sum^n_{j=1}f(\gamma(\xi_j))(\gamma(t_j)-\gamma(t_{j-1}))\]
        で定め,これが実数値となるとき\textbf{可積分}であるという.
        \item 同様にして,スカラー場の線積分の対応物として,\textbf{弧長積分}が得られる:
        \[\int_\gamma f\abs{dz}:=\lim_{\abs{\Delta}\to0}\sum^n_{j=1}f(\gamma(\xi_j))\abs{\gamma(t_j)-\gamma(t_{j-1})}.\]
    \end{enumerate}
\end{discussion}

\begin{proposition}\mbox{}
    \begin{enumerate}
        \item $C^1$-級曲線は長さ確定であり,$\Length(\gamma):=\int_\gamma 1\cdot\abs{dz}=\int_\gamma\abs{\gamma'(t)}dt$.
        \item 形式的三角不等式:$\gamma$を道,$f$を連続とすると,
        \[\Abs{\int_\gamma fdz}\le\int_\gamma\norm{f}_\gamma\abs{dz}\le\norm{f}_\gamma\Length(\gamma)\]
        \item 関数$f:D\to\C$が連続で,曲線$\gamma:[a,b]\to D$が長さ確定ならば,$f$は$\gamma$上可積分である.
        \item 特に$\gamma$が道でもあるとき,その積分値は1-形式の積分と一致する:
        \[\int_\gamma fdz=\int^b_af(\gamma(t))\gamma'(t)dt.\]
    \end{enumerate}
\end{proposition}

\subsection{ベクトル解析からの流入}

\begin{discussion}[ベクトル解析の結果の翻訳]
    Euclid空間の可縮な領域$D\subset\R^n$について,任意の$C^\infty$-級の$p$-形式$\om\in\Om^p(U)\;(p\ge 1)$は,閉ならば完全形式である.
    これはポテンシャルが存在するためである.この方法を用いれば,可縮は仮定が緩いにしても,星型領域上では同様のことが正則関数についても示せる.
\end{discussion}

\begin{theorem}
    $D$を星型領域,$a\in D^\circ$をその内点とする.
    $f:D\to\C$が連続で,$D\setminus\{a\}$上正則とすると,
    \begin{enumerate}
        \item $f$は$D$上で正則な原始関数を持つ.
        \item $D$内の任意の道$\gamma$について,$\int_\gamma fdz=0$.
    \end{enumerate}
\end{theorem}

\section{回転数}

\begin{tcolorbox}[colframe=ForestGreen, colback=ForestGreen!10!white,breakable,colbacktitle=ForestGreen!40!white,coltitle=black,fonttitle=\bfseries\sffamily,
title=]
    まず,対数関数とその複素積分を用いて,曲線がある点の周りを何周するかを解析的なことばで捉えられる.
\end{tcolorbox}

\subsection{定義と性質}

\begin{definition}[winding number]
    閉道$\gamma:[a,b]\to\C$について,
    \[\xymatrix@R-2pc{
        n(\gamma,-):\C\setminus\Im\gamma\ar[r]&\Z\\
        \rotatebox[origin=c]{90}{$\in$}&\rotatebox[origin=c]{90}{$\in$}\\
        a\ar@{|->}[r]&\frac{1}{2\pi i}\int_\gamma\frac{d\zeta}{\zeta-a}
    }\]
    を\textbf{回転数}という.
\end{definition}
\begin{proof}
    曲線$\gamma$を表すパラメータを$z:[\al,\beta]\to\Im\gamma$とすると,
    \[h(t):=\int^t_\al\frac{z'(t)}{z(t)-a}dt\]
    とおき,$e^{h(\beta)}=1$を示せば,指数関数の周期性から$n(\gamma,-)\subset 2\pi i\Z$が示せる.

    微積分学の基本定理より,
    \[h'(t)=\frac{z'(t)}{z(t)-a}\quad\fe\]
    これより,関数$H(t):=e^{-h(t)}(z(t)-a)$は$[\al,\beta]$上定値関数である.実際,この導関数は
    \[H'(t)=-h'(t)e^{-h(t)}(z(t)-a)+e^{-h(t)}z'(t)=0\quad\fe\]
    特に$t=\beta$の場合を考えて$e^{h(\beta)}=\frac{z(\beta)-a}{z(\al)-a}=1$.
\end{proof}
\begin{remarks}[対数関数のRiemann面を登っていく描像]
    直感的には,
    \[\int_\gamma\frac{dz}{z-a}=\int_\gamma d\log(z-a)=\int_\gamma d\log\abs{z-a}+i\int_\gamma d\arg(z-a)\]
    が成り立つが,$\arg$には$\gamma$上一価な分枝が定義出来ない.
    第二項が$2\pi i$という値の原因になっている.
\end{remarks}

\begin{proposition}[回転数の性質]
    $\gamma$を閉道,$\C\setminus\Im\gamma=\sqcup_{\lambda\in\Lambda}V_\lambda$を連結成分への分解とする.
    \begin{enumerate}
        \item $n(\gamma,z)$の各連結成分への制限は定値.
        \item 非有界な連結成分の上では$n(\gamma,z)=0$を満たす.
    \end{enumerate}
\end{proposition}

\section{Cauchyの定理}

\begin{tcolorbox}[colframe=ForestGreen, colback=ForestGreen!10!white,breakable,colbacktitle=ForestGreen!40!white,coltitle=black,fonttitle=\bfseries\sffamily,
title=]
    Cauchyの定理を編み上げる過程は,位相的な洗練である.
    はじめは長方形領域から行い,最終的にはホモトピーの言葉で定理を完成させる.
    なお,積分領域$T$や$[\Delta(a,r)]$に対して,これを含む連結な開近傍上で正則関数が定義されていることを考えることに注意.
\end{tcolorbox}

\subsection{Cauchyの定理}

\begin{lemma}[閉三角形領域に対するCauchyの定理]
    $D\subset\C$を領域,$p\in D$とし,連続関数$f:D\to\C$は$D\setminus\{p\}$上正則とする.
    このとき,任意の閉三角形$T\subset D$について,
    \[\int_{\partial T}f(z)dz=0.\]
\end{lemma}
\begin{remarks}
    有限個の特異点(ただしその上でも連続)を除いて任意の閉三角形$T$上で正則な関数は,$\partial T$上での積分が零である.
    実は有限個の特異点$(\zeta_j)$上で連続性がわからなくとも,$\lim_{z\to\zeta_j}(z-\zeta_j)f(z)=0$がなりたてば良い.
\end{remarks}
\begin{proof}
    一般の三角形$T\subset D$について,$\eta(T):=\int_{\partial T}f(z)\;dz$と置く.
    \begin{description}
        \item[$p\notin T$の場合] $T$を4つの三角形$(T_1^j)_{j\in[4]}$に分割すると$\eta(T)=\sum^4_{j=1}\eta(T_1^j)$.この時,$T_1:=\max_{j\in[4]}T_1^j$と置くと,
        \[|\eta(T)|\le 4|\eta(T_1)|.\]
        同様にして分割を繰り返すと,三角形の包含列
        \[ T=:T_0\supset T_1\supset T_2\supset\cdots \]
        を得,
        \[ |\eta(T)|\le 4^{j}|\eta(T_j)|\;\;\;(j\in\N) \]
        が成り立つ.各三角形について$z_j\in T_j$を取る.すると,
        \[ k>j\Rightarrow z_k\in T_j \]
        が成り立つ.$diam(T_j):=\max\{|z-w|\mid z,w\in T_j\}$と置くと,$diam(T_j)=2^{-j}diam(T)$だから,$(z_j)$はCauchy列である.よって,$a:=\bigcap_{j=0}^\infty T_j$とおけばこれが極限点$a=\lim_{j\to\infty}z_j$に他ならない.

        $p\notin T$としたから$f$は$a$で微分可能で,
        \[ f(z)=f(a)+f'(a)(z-a)+(z-a)\varphi(z)\;\;\;(\lim_{z\to a}\varphi(z)=0) \]
        と置けるから,
        \begin{align*}
            \eta(T_j)&:=\int_{\partial T_j}(f(a)+f'(a)(z-a)+(z-a)\varphi(z))\;dz\\
            &=\int_{\partial T_j}(z-a)\varphi(z)\;dz\;\;\;(補題)\\
            |\eta(T_j)|&\le\int_{\partial T_j}|z-a||\varphi(z)|\;|dz|\\
            &\le diam(T_j)|\varphi|_{T_j}L(\partial T_j)\\
            &= 2^{-j}diam(T)|\varphi|_{T_j}2^{-j}L(\partial T)\\
            &= 4^{-j}diam(T)L(\partial T)|\varphi|_{T_j}.
        \end{align*}
        すると,
        \begin{align*}
            |\eta(T)|&\le 4^j|\eta(T_j)|\\
            &\le diam(T)L(\partial T)|\varphi|_{T_j}\xrightarrow{j\to\infty}0.
        \end{align*}
        \item[$p\in T$の場合] $T$を同様に$4^j$個の三角形に分割する.どの段階でも,$p$を含む$T^k$は高々$6$個である.
        上での議論より,$p$を含まない全ての$T^k_j$について,$\eta(T^k)=0$であることはわかっているから,$|\eta(T^{k_0})|:=\max_{k}|\eta(T^k)|$と置くと,
        \begin{align*}
            |\eta(T)|&\le\sum^{4^j}_{k=1}|\eta(T^k_j)|\le 6|\eta(T^{k_0}_j)|\\
            |\eta(T^k)|&=\int_{\partial T^k}|f(z)|\;|dz|\le|f|_TL(\partial T^k)=|f|_T2^{-j}L(\partial T)\\
            |\eta(T)|&\le 6|f|_T2^{-j}L(\partial T)\xrightarrow{j\to\infty}0.
        \end{align*}
    \end{description}
\end{proof}

\subsection{Cauchyの積分表示}

\begin{theorem}[閉円板に対するCauchyの積分表示]
    $f:D\to\C$を正則とする.$[\Delta(a,r)]\subset D$ならば,
    \[\forall_{z\in\Delta(a,r)}\quad f(z)=\frac{1}{2\pi i}\int_{\partial\Delta(a,r)}\frac{f(\zeta)}{\zeta-z}d\zeta\]
\end{theorem}

\begin{theorem}[一般の曲線に対するCauchyの積分表示]
    $D$を星型領域,$f:D\to\C$を正則関数,$\gamma:[\al,\beta]\to D$を閉道とする.このとき,次が成り立つ:
    \[\forall_{z\in D\setminus\Im\gamma}\quad\frac{1}{2\pi i}\int_\gamma\frac{f(\zeta)}{\zeta-z}d\zeta=n(\gamma,z)f(z).\]
\end{theorem}

\subsection{高階導関数に対するCauchyの評価}

\begin{tcolorbox}[colframe=ForestGreen, colback=ForestGreen!10!white,breakable,colbacktitle=ForestGreen!40!white,coltitle=black,fonttitle=\bfseries\sffamily,
title=]
    微分も積分を用いてかけることを通じて,正則関数が$C^\infty$-級であることを示す.
\end{tcolorbox}

\begin{lemma}
    $\gamma:[a,b]\to D$を道,$\varphi:\Im\gamma\to\C$を連続関数とする.$\C\setminus\Im\gamma$上の関数
    \[F_n(z):=\int_\gamma\frac{\varphi(\zeta)}{(\zeta-z)^n}d\zeta\]
    は$\C\setminus\Im\gamma$上正則であり,その導関数は$F'_n(z)=nF_{n+1}(z)$を満たす.
\end{lemma}

\begin{theorem}
    正則関数$f:D\to\C$について,
    \begin{enumerate}
        \item $f$は$D$上無限回微分可能である.
        \item 任意の円板$[\Delta(a,r)]\subset D$に対して,$n$階導関数は
        \[f^{(n)}(z)=\frac{n!}{2\pi i}\int_{\partial\Delta(a,r)}\frac{f(\zeta)}{(\zeta-z)^{n+1}}d\zeta\]
        と表せる.
        \item $\abs{f^{(n)}(a)}\le\frac{n!}{r^n}\norm{f}_{\partial\Delta(a,r)}$.
    \end{enumerate}
\end{theorem}

\section{局所整級数展開}

\begin{tcolorbox}[colframe=ForestGreen, colback=ForestGreen!10!white,breakable,colbacktitle=ForestGreen!40!white,coltitle=black,fonttitle=\bfseries\sffamily,
title=]
    正則関数の最も重要な性質に急ごう.
    任意の正則関数は,収束する整級数(関数要素)の「貼り合わせ」によって定まる.
    この観点から得られる大域的解析関数をWeierstrassの大域的解析関数というのであった.
\end{tcolorbox}

\subsection{Taylorの定理}

\begin{theorem}[Taylorの定理]
    領域上の正則関数$f:D\to\C$と任意の点$z_0\in D$について,
    \begin{enumerate}
        \item 任意の$n\in\N$について,正則関数$f_n\in\O(D)$が存在し,次が成り立つ
        \[f(z)=\sum^{n-1}_{k=0}\frac{f^{(k)}(z_0)}{k!}(z-z_0)^k+f_n(z)(z-z_0)^n\quad\on D.\]
        \item このとき剰余項$f_n$は,任意の閉円板$[\Delta(z_0,r)]\subset D$に対して,その内部$\Delta(z_0,r)$上で
        \[f_n(z)=\frac{1}{2\pi i}\int_{\partial\Delta(z_0,r)}\frac{f(\zeta)}{(\zeta-z_0)^n(\zeta-z)}d\zeta\quad\on\Delta(z_0,r).\]
        と表示出来る.
    \end{enumerate}
\end{theorem}

\subsection{整級数展開}

\begin{corollary}
    領域上の正則関数$f:D\to\C$と任意の点$z_0\in D$について,任意の開円板$\Delta(z_0,r)\subset D$上において
    \[f(z)=\sum^\infty_{n=0}\frac{f^(n)(z)}{n!}(z-z_0)^n\quad\on\Delta(z_0,r)\]
    が成り立つ.ただし,右辺の収束は広義一様収束とした.
\end{corollary}

\section{特異点}

\begin{tcolorbox}[colframe=ForestGreen, colback=ForestGreen!10!white,breakable,colbacktitle=ForestGreen!40!white,coltitle=black,fonttitle=\bfseries\sffamily,
title=]
    Cauchyの積分表示による簡単な帰結を確認する.
\end{tcolorbox}

\begin{definition}
    連続関数$f:D\to\C$について,値が定義されないことも含めて,関数が正則にならない点$z_0\in\C$を\textbf{特異点}という.
    \begin{enumerate}
        \item 他の特異点を含まない開近傍が取れるような特異点を\textbf{孤立特異点}という.
        \item 孤立特異点のうち$\lim_{z\to a}f(z)\in\hatC$が定義出来ないとき,これを\textbf{真性特異点}という.
    \end{enumerate}
\end{definition}

\subsection{定数関数の特徴付け}

\begin{corollary}[Liouville]
    $\C$上で有界な整関数は,定数関数である.
\end{corollary}

\begin{corollary}[代数学の基本定理]
    定数でない多項式は零点を持つ.
\end{corollary}

\subsection{正則関数の特徴付け}

\begin{corollary}[Morera]
    領域$D$上の連続関数$f:D\to\C$について,
    任意の三角形$T\subset D$に対して
    \[\int_{\partial T}f(\zeta)d\zeta=0\]
    が成り立つならば,$D$上正則である.
\end{corollary}

\subsection{特異点の除去}

\begin{tcolorbox}[colframe=ForestGreen, colback=ForestGreen!10!white,breakable,colbacktitle=ForestGreen!40!white,coltitle=black,fonttitle=\bfseries\sffamily,
title=]
    特異点が除去できる十分条件は2つほどあるが,本質は特異点周りの漸近挙動である.
\end{tcolorbox}

\begin{corollary}[連続性による除去]
    連続関数$f:D\to\C$が高々有限個の点を除いて正則ならば,$D$上で正則である.
\end{corollary}

\begin{corollary}[除去可能特異点の特徴付け]
    1点$p\in D$を除いた領域上の関数$f:D\setminus\{p\}\to\C$は正則であるとする.このとき,次の2条件は同値.
    \begin{enumerate}
        \item $f$は$D$上正則に延長でき,その延長は一意的である.
        \item $\lim_{z\to p}(z-p)f(z)=0$.
    \end{enumerate}
\end{corollary}

\subsection{零点と極}

\begin{tcolorbox}[colframe=ForestGreen, colback=ForestGreen!10!white,breakable,colbacktitle=ForestGreen!40!white,coltitle=black,fonttitle=\bfseries\sffamily,
title=]
    零点は特異点ではなく,極は正則性を失うという意味で特異点であるが,この2つは表裏一体である.
    いずれも孤立する.
\end{tcolorbox}

\begin{definition}
    $f:D\to\C$を零でない正則関数とする.
    \begin{enumerate}
        \item $f(z_0)=0$を満たす点$z_0\in D$を\textbf{零点}という.
        \item $\min\Brace{n\in\N\mid f^{(n)}(z_0)=0}$を,零点$z_0$の\textbf{位数}という.
    \end{enumerate}
\end{definition}

\begin{proposition}[位数は有限である]
    $f:D\to\C$を正則関数,$z_0\in D$を零点とする.$\forall_{j\in\N}\;f^{(j)}(z_0)=0$が成り立つならば,$f$は零関数である.
\end{proposition}

\subsection{一致の定理}

\begin{theorem}
    $f:D\to\C$を零でない正則関数とする.このとき,$f$の零点はいずれも孤立点である.
\end{theorem}

\begin{corollary}[一致の定理]
    $f,g:D\to\C$を正則関数である.ある$D$内に集積点を持つ部分集合上で値が一致するならば,$D$上で一致する.
\end{corollary}

\chapter{留数解析}

\section{留数定理}

\chapter{無限積展開}

\chapter{調和関数}

\chapter{Dirichlet問題}

\end{document}